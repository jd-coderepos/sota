\documentclass[10pt]{article}


\usepackage{hyperref}
\usepackage{times}
\usepackage{graphicx}
\usepackage{wrapfig}
\usepackage{color}
\usepackage{url}
\usepackage{cite}
\usepackage{amsmath}
\usepackage{amsthm,amssymb}
\usepackage{proof}
\usepackage[matrix,arrow,curve]{xy}

\allowdisplaybreaks

\theoremstyle{plain}
\newtheorem{theorem}{Theorem}
\newtheorem{lemma}[theorem]{Lemma}
\newtheorem{corollary}[theorem]{Corollary}

\theoremstyle{definition}
\newtheorem{definition}[theorem]{Definition}
\newtheorem{remark}[theorem]{Remark}
\newtheorem{notation}[theorem]{Notation}
\newtheorem{example}[theorem]{Example}


\newcommand{\entail}{\vdash}
\newcommand{\define}[1]{{\em #1}}

\newcommand{\setsize}{\sharp^{\rm set}}
\newcommand{\vectoset}{{\rm VtoS}}

 \newcommand{\bor}{\ |\ }
\newcommand{\borsmall}{\,|\,}
\newcommand{\tensor}{\otimes}
\newcommand{\loli}{\mathbin{\multimap}}

\newcommand{\FinSet}{\mbox{\bf FinSet}}
\newcommand{\FinVec}{\mbox{\bf FinVec}}
\newcommand{\truthcat}{\mbox{\bf ThTble}}

\newcommand{\pair}[1]{{\langle\,{#1}\,\rangle}}
\newcommand{\punit}{{\star}}
\newcommand{\tunit}{{\bf 1}}
\newcommand{\bottype}{{\bf 0}}
\newcommand{\ttrue}{{\tt t}\!{\tt t}}
\newcommand{\ffalse}{{\tt f}\!{\tt f}}
\newcommand{\ifterm}[3]{{{\tt if}\,{#1}\,{\tt then}\,{#2}\,{\tt else}\,{#3}}}
\newcommand{\letunit}[2]{{\tt let}\,\star={#1}\,{\tt in}\,{#2}}
\newcommand{\bit}{{\tt Bool}}

\newcommand{\opeq}{\mathrel{\simeq_{\rm op}}}
\newcommand{\axeq}{\mathrel{\simeq_{\rm ax}}}

\newcommand{\Tb}[1]{{\it Tb}(#1)}

\newcommand{\defas}{\mathrel{{:}{=}}}
\newcommand{\denot}[1]{{[\![{#1}]\!]}}
\newcommand{\fsdenot}[1]{{[\![{#1}]\!]^{\rm set}}}
\newcommand{\fvdenot}[1]{{[\![{#1}]\!]^{\rm vec}}}





\begin{document}

\title{Finite Vector Spaces as Model of Simply-Typed
  Lambda-Calculi\footnote{Accepted at ICTAC 2014. The final
    publication will be available at \url{http://link.springer.com}}}


\author{
  \scriptsize
  \begin{tabular}{c}
    {\large Beno\^it Valiron\footnote{Partially supported by the ANR project ANR-2010-BLAN-021301 LOGOI.}}\1.5ex]
    University of Pennsylvania,\\
    Philadelphia, US
  \end{tabular}
  \
\begin{array}{@{}l@{~~~~}l@{~~~~}l@{}}
  M,N,P & {:}{:}{=} & 
  x \bor \lambda x.M \bor MN \bor
  \pi_l(M) \bor \pi_r(M) \bor \pair{M,N}\bor\punit\bor
  \\
  &&
  \ttrue \bor \ffalse \bor \ifterm{M}{N}{P}\bor\letunit{M}{N}
  \\
  A,B & {:}{:}{=} &
  \bit \bor A \to B \bor A \times B \bor \tunit.
\end{array}

  \infer{\Delta,x:A\entail x:A}{}
  \quad
  \infer{\Delta\entail\punit:\tunit}{}
  \quad
  \infer{\Delta\entail\ttrue,\ffalse:\bit}{}
  \quad
  \infer{\Delta\entail\lambda x.M:A\to B}{\Delta,x:A\entail M:B}
  \quad
  \infer{\Delta\entail\pi_i(M):A_i}{\Delta\entail M:A_l\times A_r}
  
  \infer{\Delta\entail MN:B}{\begin{array}{l}\Delta\entail M:A\to B \\
      \Delta\entail
    N:A\end{array}}
  ~~~
  \infer{\Delta\entail \pair{M,N}:A{\times}B}{\begin{array}{l}\Delta\entail M:A\\
    \Delta\entail N : B\end{array}}
  ~~~
  \infer{\Delta\entail \ifterm{M}{N_1}{N_2}:A}{\begin{array}{l}\Delta\entail M:\bit\\
    \Delta\entail N_1,N_2 : A\end{array}}
  ~~~
  \infer{\Delta\entail \letunit{M}{N}:A}{\begin{array}{l}\Delta\entail M:\tunit\\
    \Delta\entail N : A\end{array}}
  
  \begin{array}{rcl}
    (\lambda x.M)N &\to& M[N/x]
    \\
    \letunit{\star}{M}&\to& M
  \end{array}
  \quad
  \begin{array}{rcl}
    \pi_l\pair{M,N}  &\to& M
    \\
    \pi_r\pair{M,N}  &\to& N
  \end{array}
  \quad
  \begin{array}{rcl}
    \ifterm{\ttrue}{M}{N} &\to& M
    \\
    \ifterm{\ffalse}{M}{N} &\to& N
  \end{array}
  
  \infer{MN\to M'N}{M\to M'}
  \qquad
  \infer{\pi_l(M)\to \pi_l(M')}{M\to M'}
  \qquad
  \infer{\pi_r(M)\to \pi_r(M')}{M\to M'}
  
  \infer{\ifterm{M}{N_1}{N_2}\to \ifterm{M'}{N_1}{N_2}}{M\to M'}
  ~~
  \infer{\letunit{M}{N}\to\letunit{M'}{N}}{M\to M'}
  
\begin{array}{@{}l@{~}l@{~}l@{}}
  C[-] & {:}{:}{=} & 
  x \bor [-]\bor \lambda x.C[-] \bor C[-]N \bor MC[-] \bor\pi_l(C[-]) \bor
  \pi_r(C[-]) \bor
  \pair{C[-],N}\bor
  \\
  &&\pair{M,C[-]}\borsmall
  \punit\borsmall
  \ttrue \borsmall \ffalse \bor \ifterm{C[-]}{N}{P}\borsmall
  \ifterm{\!M\!}{C[-]}{P}\borsmall\\
  &&
  \ifterm{M}{N\!}{C[-]}\bor
  \letunit{C[-]}{M}\bor\letunit{M}{C[-]}.
\end{array}

  \begin{array}{rcl}
    \fsdenot{\Delta,x:A\entail x:A} &:& (d,a) \longmapsto a
    \1.2ex]
    \fsdenot{\Delta\entail \ffalse:\bit} &:& d \longmapsto \ffalse
    \1.2ex]
    \fsdenot{\Delta\entail MN:B} &:& d \longmapsto
    \fsdenot{M}(d)(\fsdenot{N}(d))
    \1.2ex]
    \fsdenot{\Delta\entail\pi_r(M):B} &=&
    \fsdenot{M} ; \pi_r
  \end{array}
  
  \fsdenot{\Delta\entail\lambda x.M:A\to B} ~~=~~ d \longmapsto (a
  \mapsto \fsdenot{M}(d,a))
  \quad
  \fsdenot{\Delta\entail\letunit{M}{N}:A} ~~=~~ \fsdenot{N}
  
  \fsdenot{\Delta\entail\ifterm{M}{N}{P}:A} ~~=~~ d \longmapsto 
  \left\{
    \begin{array}{ll}
      \fsdenot{N}(d)
      &
      \textrm{if~~ ,}
      \\
      \fsdenot{P}(d)
      &
      \textrm{if~~ .}
    \end{array}
  \right.
  
\label{eq:adj}
\xymatrix{
  \FinSet\ar@/^1ex/[r]^{F} &\FinVec \ar@/^1ex/[l]^{G}.
}

\begin{array}{l@{}lll}
m_{X,Y} :{}&\langle X\times Y\rangle &\to& \langle X\rangle\otimes \langle
Y\rangle
\\
&(x,y) &\mapsto& x\tensor y,
\end{array}
\quad
\begin{array}{l@{}lll}
m_{1} : {}&\langle 1\rangle &\to& I
\\
&\star&\mapsto& 1.
\end{array}

  \begin{array}{rcl}
    \fvdenot{\Delta,x:A\entail x:A} &:& d\tensor b_a \longmapsto a
    \1.2ex]
    \fvdenot{\Delta\entail \ffalse:\bit} &:& d \longmapsto \ffalse
    \1.2ex]
    \fvdenot{\Delta\entail MN:B} &:& d \longmapsto
    \fvdenot{M}(d)(\fvdenot{N}(d))
    \1.2ex]
    \fvdenot{\Delta\entail\pi_r(M):B} &=&
    \fvdenot{M} ; \pi_r
  \end{array}
  
  \begin{array}{rcl}
    \fvdenot{\Delta\entail\lambda x.M:A\to B} &=& d \longmapsto (b_a
    \mapsto \fvdenot{M}(d\tensor b_a))
    \1.5ex]
    \fvdenot{\Delta\entail\ifterm{M}{N}{P}:A} &=& d \longmapsto 
    \alpha\cdot\fvdenot{N}(d) + 
    \beta\cdot\fsdenot{P}(d)
    \\end{minipage}}
\end{table*}



The adjunction in Eq.~\eqref{eq:adj}
generates a linear comonad on . If  is a
finite vector space, we define the finite vector space  as the
vector space freely generated from the set :
it consists of the space
.
If  is a linear map, the map  is defined as

The comultiplication and the counit of the comonad are respectively 

and

where
 and .
Every element  is a commutative comonoid when equipped with the
natural transformations
 and

where
 and .
This makes the category  into a linear category.

In particular, the coKleisli category  coming from the
comonad is cartesian closed: the product of  and  is
, the usual product of vector spaces, and the terminal object is 
the vector space .
This coKleisli category is the usual
one: the objects are the objects of , and the morphisms
 are the morphisms .  The
identity  is the counit and the composition of 
and  is



There is a canonical full embedding  of categories sending
 on .
It sends an object  to the set of vectors of  (i.e. it acts
as the forgetful functor on objects) and sends the linear map  to the map
.

This functor preserves the cartesian closed structure: the terminal
object 
 of
 is sent to the set containing only , that is, the singleton-set
. The product space  is sent to the set of vectors
, which is exactly the set-product of 
and . Finally, the function space  is in exact correspondence with
the set of set-functions .


\begin{remark}\label{rem:ext-prod}
  The construction proposed as side example by Hyland and Schalk~\cite{hyland03glueing} 
  considers finite vector spaces with a field of characteristic~. There,
  the modality is built using the exterior product algebra, and it turns out to
  be identical to the functor we use in the present paper. Note though,
  that their construction does not work with
  fields of other characteristics.
\end{remark}

\begin{remark}
  Quantitative models of linear logic such as finiteness
  spaces~\cite{finiteness} are also based on vector spaces;
  however, in these cases the procedure to build a comonad
  does not play well with the finite dimension the vector spaces
  considered in this paper: the definition of the comultiplication
  assumes that the space  is infinitely dimensional.
\end{remark}


\subsection{Finite vector spaces as a model}
\label{sec:lc-finvec-model}



Since  is a cartesian closed category, one can model
terms of  as linear maps. Types are interpreted as
follows. The unit type is . The boolean type is
. The product is the
usual product space: , whereas the arrow type is
.
A typing judgment  is represented
by a morphism of  of type

inductively defined as in Table~\ref{tab:denot-lc-alg}. The
variable  stands for a base element  of , and  is a base element of
. The functions  and  are the left and
right projections of the product.

  Note that because of the equivalence between  and
  , the map in Eq.~\eqref{eq:map} is a morphism of
  , as desired.


\refstepcounter{theorem}\label{ex:numunit-alg}\medskip
\noindent
{\bf Example~\ref{ex:numunit-alg}.}~
In {\FinSet}, there was only one Church numeral based on type
  . In , there are more elements in the
  corresponding space
   and we get more
  distinct Church numerals.

  Assume that the finite field under consideration is the 2-elements
  field . Then
  
  The space  is freely generated from the vectors of : it
  therefore consists of just the four vectors .
  The space of morphisms  is the space
  . It is generated by two functions:
   sending  to  and  to , and 
  sending  to . The space 
  therefore also contains 4 vectors: , ,  and
  . Finally, the vector space 
  is freely generated from the 4 base elements , ,
   and , therefore containing 
  vectors. Morphisms
   can be
  represented by  matrices with coefficients in
  . 
\begin{wrapfigure}{r}{0.24\textwidth}
\begin{minipage}{.2\textwidth}
  f_\starf_0b_{0}b_{f_0}b_{f_\star}b_{f_0+f_\star}
\end{minipage}
\end{wrapfigure}
The basis elements  are ordered as above, as
  are the basis elements , as shown on the right.
  The Church numeral  sends all of its arguments to the
  identity function, that is, . The Church numeral 
  is the identity. So their respective matrices are
  
  and
  .
  The next two Church numerals are
   and
  , which is also .
  So  with the field of characteristic  distinguishes
  null, even and odds numerals over the type .

  Note that this characterization is similar to the  
  Example~\ref{ex:numbool}, except that there, the type over which the
  Church numerals were built was . Over ,
  Example~\ref{ex:numunit} stated that all Church numerals collapse.

\begin{example}
  \label{ex:numunit-alg2}
  The fact that  with the field of characteristic 2 can
  be put in parallel with {\FinSet} when considering Church numerals
  is an artifact of the fact that the field has only two elements. If
  instead one chooses another field
   of
  characteristic , with  prime,
  then this is in general not true anymore. In this case,
  , and 
  has dimension  with basis elements 
   sending  and  when .
  It therefore consists of  vectors. Let us represent a function
   with  where
  . A morphism
   can be
  represented with a  matrix. The basis elements  of  are ordered
  lexicographically:
  
  as are the basis elements .
\begin{table*}[t]
  \caption{The  Church numerals over type  in  with .}
  \label{tab:sevenmat}
  \scalebox{.84}{\begin{minipage}{5.67in}
  
  
  
  \end{minipage}}
\end{table*}

  The Church numeral  is again the constant function
  returning the identity, that is, .  The numeral
   sends  onto the function sending
   onto . The numeral 
  sends  onto the function sending 
  onto . The numeral  sends
   onto the function sending  onto
  . And so on.
  
  In particular, each combination  can
  be considered as a function . The sequence 
  eventually loops. The order
  of the loop is , the least common multiple of all
  integers , and for all  we have : there are  distinct
  Church numerals in the model
   with a field of characteristic  prime.

  For  we recover the  distinct Church numerals. But for
  , we deduce that there are  distinct Church numerals (the
   corresponding matrices are reproduced in
  Table~\ref{tab:sevenmat}). As there is almost a factorial function, the number of
  distinct Church numerals grows fast as  grows: With , there
  are  distinct numerals, and with  there are  distinct
  numerals.
\end{example}



\begin{example}
  \label{ex:numbool-alg}
  Let us briefly reprise Example~\ref{ex:numbool} in
  the context of . Even with a field of characteristic
  , the vector space  is
  relatively
  large:  has dimension 2 and consists of  vectors, 
  then has dimension  and consists of  vectors. The dimension
  of the homset  is , and it contains 
  vectors. Using the representation of the two previous examples, a
  Church numeral is then a matrix of size .
  
  Let us represent a function  as a tuple  lexicographically ordered
  
  representing the map sending  to
  .
  These form the basis elements of the range of the matrix. The
  domain of the matrix consists of all the  combinations of 0/1
  values that these can take. Ordered lexicographically, they form the
  basis of the domain of the matrix.
  
  As before, the Church numeral  is constant while 
   is the identity. 
  The numeral  sends each of the -tuples  to the -tuple
  
  and so forth.
So for example, the negation sending  to  is the -tuple
   and is sent by  to the tuple
   which is indeed the identity.
  
  If one performs the calculation, one finds out that in
  , over the type , there are exactly  distinct
  Church numerals. The numerals ,  and  are
  uniquely determined, and then the semantics distinguishes the
  equivalence classes , for
  . The  non-constant Church numerals are
  represented in Table~\ref{tab:14num}: First column contains numbers 
  to , second columns numbers  to . The matrices are represented
  as rectangles made of  squares. Black squares mean 
  and white squares mean .
\end{example}



\subsection{Properties of the \texorpdfstring{}{FinVec} Model}


\begin{table*}[b]
  \caption{The  non-constant Church numerals over  in  with .}
  \label{tab:14num}
  \def\b{\rule{0.026cm}{0.03cm}}
  \def\a{\phantom{\rule{0.026cm}{0.03cm}}}
  \def\mynl{\-1ex]}
  \scalebox{.89}{\begin{minipage}{5.32in}
 \end{minipage}}
\end{table*}

As shown in the next results, this semantics is both sound and
adequate with respect to the operational equivalence.
Usually adequacy uses non-terminating terms. Because the
language is strongly normalizing, we adapt the notion.
However, because there are usually more maps between  and
 than between  and  (as shown
in Examples~\ref{ex:numunit-alg}, \ref{ex:numunit-alg2}
and~\ref{ex:numbool-alg}), the model fails to be fully abstract.

\begin{lemma}
  \label{lem:axdenot-st-alg}
  If  then .\qed
\end{lemma}


\begin{theorem}
  \label{th:sound-st-alg}
  If  and  then 
  .
\end{theorem}

\begin{proof}
  The proof is similar to the proof of Theorem~\ref{th:sound} and
  proceeds by contrapositive, using Lemmas~\ref{lem:uniq-value},
  \ref{lem:sn}, \ref{lem:opax} and~\ref{lem:axdenot-st-alg}.
\end{proof}

\begin{theorem}[Adequacy]
  Given two closed terms  and  of type ,  if and
  only if .
\end{theorem}

\begin{proof}
  The left-to-right direction is Theorem~\ref{th:sound-st-alg}. For
  the right-to-left direction, since the terms  and  are closed
  of type , one can choose the context  to be , and
  we have  if and only if . From
  Lemma~\ref{lem:sn}, there exists such a boolean : we deduce from
  Lemma~\ref{lem:opax} that . We conclude with
  Lemma~\ref{lem:axdenot-st-alg}.
\end{proof}

\begin{remark}
\label{rem:notfullyabst}
The model  is not fully abstract. 
Indeed, consider the two valid typing judgments

and
.
The denotations of both of these judgments are linear maps 
. According to the rules of
Table~\ref{tab:denot-lc-alg}, the denotation of the first term is 
the constant function sending all non-zero
vectors  to .

For the second term,
suppose that  is 
equal to
.
Let .
Then since , 
the denotation of the second term is the function sending 
to , equal to
 from what we just discussed.
We conclude that if , then : the denotation of
 sends  to .

Nonetheless, they are clearly operationally equivalent in  since their denotation in {\FinSet} is the same.
The language is not expressive enough to distinguish between these two
functions. Note that there exists operational settings where these
would actually be different, for example if we were to allow divergence.
\end{remark}

\begin{remark}
Given a term , another question one could ask is whether the set of
terms  in  generates a free family of vectors in the
vector space . It turns
out not: The field structure brought into the model introduces
interferences, and algebraic sums coming from operationally distinct
terms may collapse to a representable element. For example, supposing
for simplicity that the characteristic of the field is , consider
the terms , ,
 and  defined as ,
all of types . They are clearly operationally distinct,
and their denotations live in . They can be
written as a  matrices along the bases
 for the domain and
 for the range. The respective images of the 
terms are

,
,
,

and clearly,


So if the model we are interested in is , the language is
missing some structure to correctly handle the algebraicity.
\end{remark}


\section{An algebraic lambda-calculus}
\label{sec:alglc}

\begin{table}[t]
    \caption{Rewrite system for the algebraic fragment of
      .}
    \label{tab:rw-alg}
\scalebox{.83}{\begin{minipage}{5.7in}
\end{minipage}}
\end{table}


To solve the problem, we extend the language  by adding an
algebraic structure to mimic the notion of linear distribution
existing in . The extended language 
is a call-by-name variation of~\cite{ad08,adv11} and reads as follows:

The scalar  ranges over the field.
The values are now

.
The typing rules are the same for the regular constructs. The new
constructs are typed as follows: for all , , 
and provided that ,
then  and 
The rewrite rules are extended as follows.

\smallskip
\noindent
1)~~ A set of algebraic rewrite rules shown in
Table~\ref{tab:rw-alg}. We shall explicitly talk about {\em algebraic
  rewrite rules} when referring to these extended rules. The top row
consists of the associativity and commutativity (AC) rules. We shall
use the term {\em modulo AC} when referring to a rule or property that
is true when not regarding AC rules. For example, modulo AC the term  is
in normal form and 
reduces to .
The reduction
rules from  will be called {\em non-algebraic}.

\smallskip
\noindent
2)~~
 The relation between the algebraic structure and the other
  constructs: one says that a construct  is
  \define{distributive} when for all , , 
   and 
  .
The following constructs are distributive:
,
,
,
,
and the pairing construct factors: 
, 
 and
.

\smallskip
\noindent
3)~~
 Two congruence rules. If , then 
  and .

\begin{remark}\label{rem:red-strat}
  Note that if  reduces to
  , it does {\em not} reduce to
  . If it did, one would get an
  inconsistent calculus~\cite{ad08}. For example, the term  would reduce both to
   and to
  .
  We'll come back to this distinction in Section~\ref{sec:call-value-reduction}.
\end{remark}

The algebraic extension preserves the safety properties, the
characterization of values and the strong normalization.
Associativity and commutativity induce a subtlety.

\begin{lemma}
  \label{lem:sn-AC}
  The algebraic fragment of  is strongly
  normalizing modulo AC.
\end{lemma}

\begin{proof}
  The proof can be done as in~\cite{ad08}, using the same measure on
  terms that decreases with algebraic rewrites. The measure, written
  , is defined by , ,
  , .
\end{proof}

\begin{lemma}[Safety properties mod AC]
  \label{lem:safety-prop-alg-modAC}
  A well-typed term  is a value or, if not,
  reduces to some  via a sequence of steps among
  which one is {\em not} algebraic.\qed
\end{lemma}


\begin{lemma}
  \label{lem:uniq-value-alg}
  Any value of type  has AC-normal form ,  or
  , with .
  \qed
\end{lemma}

\begin{lemma}
  \label{lem:sn-alg}
  Modulo AC,  is strongly normalizing.
\end{lemma}

\begin{proof}
  The proof is done by defining an intermediate language  where scalars are omitted. Modulo AC, this language is
  essentially the language 
  of~\cite{groote94}, and is therefore SN. Any term of  can be re-written as a term of . With
  Lemma~\ref{lem:safety-prop-alg-modAC}, by eliminating some algebraic
  steps a sequence of reductions in  can be
  rewritten as a sequence of reductions in . We
  conclude with Lemma~\ref{lem:sn-AC}, saying there is always a finite
  number of these eliminated algebraic rewrites.
\end{proof}



\subsection{Operational equivalence}
\label{sec:lc-cat-alg}

As for , we define an operational equivalence on terms of the 
language
. A {\em context } for this language has the same
grammar as for , augmented with algebraic
structure: 
.

For , instead of using closed contexts of type
, we shall use contexts of type : thanks to
Lemma~\ref{lem:uniq-value-alg}, there are distinct normal forms for
values of type , making this type a good (and slightly simpler)
candidate.

We therefore say that  and  are
operationally equivalent, written , if for all closed
contexts  of type  where the hole binds , for
all  normal forms of type ,  if and only if .


\subsection{Axiomatic equivalence}

The axiomatic equivalence on  consists of the one
of , augmented with the added reduction rules. 


\begin{lemma}
  \label{lem:opax-alg}
  If  and  then .\qed
\end{lemma}




\subsection{Finite vector spaces as a model}
\label{sec:model-alglc}

The category  is a denotational model of the language
. Types are interpreted as for the language
 in Section~\ref{sec:lc-finvec-model}. Typing judgments are
also interpreted in the same way, with the following additional
rules. First,
. Then . Finally, we have
.

\begin{remark}
With the extended term constructs, the language 
does not share the drawbacks of  emphasized in
Remark~\ref{rem:notfullyabst}. In particular, the two valid typing
judgments  and  are now operationally
distinct. For example, if one chooses the context , the
term  reduces to  whereas the term
 reduces to .
\end{remark}

\begin{lemma}
  \label{lem:axdenot-alg}
  If  in  then .\qed
\end{lemma}


\begin{theorem}
  \label{th:sound-alg}
  Let  be two valid typing judgments in
  . If  then we also have
  .
\end{theorem}

\begin{proof}
  The proof is similar to the proof of Theorem~\ref{th:sound}: Assume
  . Then there exists a context  that distinguishes
  them. The call-by-name reduction preserves the type from
  Lemma~\ref{lem:safety-prop-alg-modAC}, and  and  can be rewritten
  as the terms
   and , and these are axiomatically
  equivalent to distinct normal forms, from Lemmas~\ref{lem:sn-alg}
  and~\ref{lem:opax-alg}. We conclude from Lemmas~\ref{lem:opax-alg}
  and~\ref{lem:axdenot-alg} that the
  denotations of  and  are distinct.
\end{proof}

\subsection{Two auxiliary constructs}
\label{sec:aux-constructs}

Full completeness requires some machinery.  It is obtained by
showing that for every type , for every vector  in
, there are two terms  and  such that  and  sends
 to  and all other 's to .


We first define a family of terms 
inductively on :  and 
.
One can show that .
Then assume that  is the
order of the field. Let   be the term 
. The denotation of  is such that  if
 and  otherwise.


The mutually recursive definitions of  and  read as
follows.

\smallskip
\noindent
{\em At type .}~~
The term  is simply .
The term  is 
.

\smallskip
\noindent
{\em At type .}~~
As for the type , the term 
is simply . The term
 is reusing the
definition of : it is the term

.


\smallskip
\noindent
{\em At type }.~~
If
, then , with  and . 
By induction, one can construct  and : the term
 is . Similarly, one can construct the terms
 and : the term  is 
.

\smallskip
\noindent
{\em At type .}~~
Consider
. The domain of 
is finite-dimensional: let  be its basis, and let 
be the value . Then, using the terms  and
, one can define  as the term
.
Similarly, one can construct  and , and from
the construction in the previous paragraph we can also generate
. The term
 is then defined as


\subsection{Full completeness}

We are now ready to state completeness, whose proof is simply by
observing that any  can be realized by the term .

\begin{theorem}[Full completeness]
  \label{th:comp-alg}
  For any type , any vector  of  in  is
  representable in the language .\qed
\end{theorem}


\begin{theorem}
  \label{th:eq-alg}
  For all  and ,  if and only if
  .\qed
\end{theorem}

A corollary of the full completeness is that the semantics  is also adequate
and fully abstract with respect to .


\section{Discussion}
\label{sec:discussion}

\subsection{Simulating the vectorial structure.}
\label{sec:fact-alg}

As we already saw, there is a full embedding of category . This embedding can be
understood as ``mostly'' saying that the vectorial structure ``does not
count'' in , as one can simulate it with finite sets.
Because of Theorems~\ref{th:eq} and~\ref{th:eq-alg}, on the syntactic
side algebraic terms can also be simulated by the regular .


\begin{table}[b]
  \caption{Relation between  and }
  \label{tab:l-l-alg}
  \scalebox{.93}{\begin{minipage}{5.1in}
  
  \end{minipage}}
\end{table}



In this section, for simplicity, we assume that the field
is . In general, it can be any finite size provided
that the regular lambda-calculus  is augmented with -bits,
i.e. base types with  elements (where  is the characteristic of
the field).

\begin{definition}
  The \define{vec-to-set} encoding of a type , written
  , is defined inductively as follows:
  ,
  , , and .
\end{definition}

\begin{theorem}
  \label{th:l-l-alg}
  There are two typing judgments  and , inverse of each other, in  such that any typing judgment  can be
  factored into
  
  where  is a regular lambda-term of .
\end{theorem}
\begin{proof}
The two terms  and  are
defined inductively on .
For the definition of  we are
reusing the term  of Section~\ref{sec:aux-constructs}.
The definition is in Table~\ref{tab:l-l-alg}
\end{proof}

\subsection{Categorical structures of the syntactic categories.}
\label{sec:categ-struct-synt}

Out of the language  one can define a syntactic 
category: objects are
types and morphisms  are valid typing judgments  modulo
operational equivalence. 
Because of Theorem~\ref{th:eq}, this category is
cartesian closed, and one can easily see that the product of 
and  is , that the terminal
object is , that
projections are defined with  and , and that the
lambda-abstraction plays the role of the internal morphism.



The language  almost defines a cartesian closed category: by
Theorem~\ref{th:eq-alg}, it is clear that pairing and lambda-abstraction form a
product and an internal hom. However, it is missing a terminal object
(the type  doesn't make one as  and
 are operationally distinct). There is no type
corresponding to the vector space . It is not
difficult, though, to extend
the language to support it: it is enough to only add a type . Its only
inhabitant will then be the term : it make a terminal
object for the syntactic
category.

Finally, Theorem~\ref{th:l-l-alg} is essentially giving us a functor
 corresponding to the full embedding
. This makes
a full correspondence between the two models  and , and
 and , showing that computationally the algebraic
structure is virtually irrelevant.




\subsection{(Co)Eilenberg-Moore category and call-by-value}
\label{sec:call-value-reduction}


From a linear category with modality  there are two canonical cartesian
closed categories: the coKleisli category, but also the (co)Eilenberg-Moore
category: here, objects are still those of , but morphisms
are now .

According to~\cite{valiron13typed}, such a model would correspond to
the call-by-value (or, as coined by~\cite{diazcaro-phd} {\em call-by-base})
strategy for the algebraic structure discussed in
Remark~\ref{rem:red-strat}.


\subsection{Generalizing to modules}
\label{sec:generalizing-modules}

To conclude this discussion, let us consider a generalization of finite vector
spaces to finite modules over finite semi-rings.

Indeed, the model of linear logic this paper uses would work in the context of a
finite semi-ring instead of a finite field, as long as addition and
multiplication have distinct units. For example, by using the
semiring  where  one recover sets and relations.
However, we heavily rely on the fact
that we have a finite field  in the construction of
Section~\ref{sec:aux-constructs}, yielding the completeness result in
Theorem~\ref{th:comp-alg}.

This particular construction works because one can construct any
function between any two finite vector spaces as polynomial, for the
same reason as any function  can be realized as a polynomial.


\begin{thebibliography}{10}

\bibitem{abramski00}
S.~Abramsky, R.~Jagadeesan, and P.~Malacaria.
\newblock Full abstraction for {PCF}.
\newblock {\em Inf. and Comp.}, 163:409--470, 2000.

\bibitem{adv11}
P.~Arrighi, A.~D\'{i}az-Caro, and B.~Valiron.
\newblock A type system for the vectorial aspects of the linear-algebraic
  lambda-calculus.
\newblock In {\em Proc. of DCM}, 2011.

\bibitem{ad08}
P.~Arrighi and G.~Dowek.
\newblock Linear-algebraic -calculus.
\newblock In {\em Proc. of RTA}, pages 17--31, 2008.

\bibitem{benton}
N.~Benton.
\newblock A mixed linear and non-linear logic: Proofs, terms and models.
\newblock Technical report, Cambridge U., 1994.

\bibitem{bierman}
G.~Bierman.
\newblock {\em On Intuitionistic Linear Logic}.
\newblock PhD thesis, Cambridge U., 1993.

\bibitem{bucciarelli12}
A.~Bucciarelli, T.~Ehrhard, and G.~Manzonetto.
\newblock A relational semantics for parallelism and non-determinism in a
  functional setting.
\newblock {\em A. of Pure and App. Logic}, 163:918--934, 2012.

\bibitem{groote94}
P.~de~Groote.
\newblock Strong normalization in a non-deterministic typed lambda-calculus.
\newblock In {\em Logical Foundations of Computer Science}, volume 813, pages
  142--152, 1994.

\bibitem{diazcaro-phd}
A.~D\'{i}az-Caro.
\newblock {\em Du Typage Vectoriel}.
\newblock PhD thesis, U. de Grenoble, 2011.

\bibitem{finiteness}
T.~Ehrhard.
\newblock Finiteness spaces.
\newblock {\em Math. Str. Comp. Sc.}, 15:615--646, 2005.

\bibitem{pcs}
T.~Ehrhard, M.~Pagani, and C.~Tasson.
\newblock The computational meaning of probabilistic coherence spaces.
\newblock In {\em Proc. of LICS}, 2011.

\bibitem{ehrhard03differential}
T.~Ehrhard and L.~Regnier.
\newblock The differential lambda-calculus.
\newblock {\em Th. Comp. Sc.}, 309:1--41, 2003.

\bibitem{ll}
J.-Y. Girard.
\newblock Linear logic.
\newblock {\em Th. Comp. Sc.}, 50:1--101, 1987.

\bibitem{sysF}
J.-Y. Girard, Y.~Lafont, and P.~Taylor.
\newblock {\em Proof and Types}.
\newblock CUP, 1990.

\bibitem{hillebrand-phd}
G.~G. Hillebrand.
\newblock {\em Finite Model Theory in the Simply Typed Lambda Calculus}.
\newblock PhD thesis, Brown University, 1991.

\bibitem{hyland03glueing}
M.~Hyland and A.~Schalk.
\newblock Glueing and orthogonality for models of linear logic.
\newblock {\em Th. Comp. Sc.}, 294:183--231, 2003.

\bibitem{james11quantum}
R.~P. James, G.~Ortiz, and A.~Sabry.
\newblock Quantum computing over finite fields.
\newblock Draft, 2011.

\bibitem{lambek-scott}
J.~Lambek and P.~J. Scott.
\newblock {\em Introduction to Higher-Order Categorical Logic}.
\newblock CUP, 1994.

\bibitem{lang}
S.~Lang.
\newblock {\em Algebra}.
\newblock Springer, 2005.

\bibitem{lidl1997finite}
R.~Lidl.
\newblock {\em Finite fields}, volume~20.
\newblock CUP, 1997.

\bibitem{maclane}
S.~Mac~Lane.
\newblock {\em Categories for the Working Mathematician}.
\newblock Springer, 1998.

\bibitem{milner77}
R.~Milner.
\newblock Fully abstract models of typed lambda-calculi.
\newblock {\em Th. Comp. Sc.}, 4:1--22, 1977.

\bibitem{plotkin77}
G.~Plotkin.
\newblock {LCF} considered as a programming language.
\newblock {\em Th. Comp. Sc.}, 5:223--255, 1977.

\bibitem{pratt92mail}
V.~R. Pratt.
\newblock Re: Linear logic semantics (barwise).
\newblock On the {TYPES} mailing list, February 1992\footnote{\scriptsize \url{http://www.seas.upenn.edu/sweirich/types/archive/1992/msg00047.html}}.

\bibitem{pratt94chu}
V.~R. Pratt.
\newblock Chu spaces: Complementarity and uncertainty in rational mechanics.
\newblock Technical report, Stanford U., 1994.

\bibitem{fin-model}
S.~Salvati.
\newblock Recognizability in the simply typed lambda-calculus.
\newblock In {\em Logic, Language, Information and Computation}, pages 48--60.
  2009.

\bibitem{schumacher10modal}
B.~Schumacher and M.~D. Westmoreland.
\newblock Modal quantum theory.
\newblock In {\em Proc. of QPL}, 2010.

\bibitem{dana93pcf}
D.~S. Scott.
\newblock A type-theoretic alternative to {CUCH}, {ISWIM}, {OWHY}.
\newblock {\em Th. Comp. Sc.}, 121:411--440, 1993.

\bibitem{fin-model-fin-posets}
P.~Selinger.
\newblock Order-incompleteness and finite lambda reduction models.
\newblock {\em Th. Comp. Sc.}, 309:43--63, 2003.

\bibitem{soloviev83}
S.~Soloviev.
\newblock Category of finite sets and cartesian closed categories.
\newblock {\em J. of Soviet Math.}, 22(3), 1983.

\bibitem{valiron13typed}
B.~Valiron.
\newblock A typed, algebraic, computational lambda-calculus.
\newblock {\em Math. Str. Comp. Sc.}, 23:504--554, 2013.

\bibitem{vaux09}
L.~Vaux.
\newblock The algebraic lambda-calculus.
\newblock {\em Math. Str. Comp. Sc.}, 19:1029--1059, 2009.

\bibitem{winskel-book}
G.~Winskel.
\newblock {\em The Formal Semantics of Programming Languages}.
\newblock MIT Press, 1993.

\end{thebibliography}


\end{document}
