\documentclass[11pt]{article}
\usepackage{url}
\usepackage{epsf}
\usepackage{graphicx}
\usepackage{amsfonts}
\usepackage{amssymb}
\usepackage{amsmath}
\usepackage{latexsym}
\usepackage{setspace, float}
\usepackage{color}
\usepackage[normalem]{ulem}
\usepackage[table]{xcolor}

\usepackage{makeidx}
\usepackage{multirow}
\usepackage{verbatim}
\usepackage{xy}\xyoption{all}

\urlstyle{rm}

\newtheorem{lemma}{Lemma}
\newtheorem{theorem}{Theorem}

\newcommand{\squeezelist}{\setlength{\itemsep}{0pt}}

\newcommand{\qed}{\rule{0.5em}{1.5ex}}
\newcommand{\fqed}{{\hfill~\qed}}
\newenvironment{proof}{{\noindent \bf Proof.}}
                      {{\hfill \fqed} \vspace{1em}}


\topmargin 0in                  \headheight 0pt                 \headsep 0in                    \textheight 9in                 
\textwidth 6.5in
\oddsidemargin 0in              
\newtheorem{property}[theorem]{Property}
\newtheorem{corollary}[theorem]{Corollary}
\newtheorem{fact}[theorem]{Fact}

\newtheorem{op}[theorem]{Problem}


\newcommand{\e}{\varepsilon}
\newcommand\R{\mathbb{R}}
\newcommand\Ln{\mathbb{L}}
\newcommand\C{\mathcal{C}}

\newcommand{\mst}{}
\newcommand{\udg}{}

\floatstyle{ruled}
\newfloat{algorithm}{htbp}{loa}
\floatname{algorithm}{Algorithm}


\title{Connectivity of Graphs Induced by Directional Antennas}
\author{
Mirela Damian\thanks{Villanova University, Villanova, USA, \texttt{mirela.damian@villanova.edu}.
\newline Partially supported by NSF grant CCF-0728909.}
\and
Robin Flatland\thanks{Siena College, Loudonville, USA, \texttt{flatland@siena.edu}.}}
\date{}

\begin{document}
\maketitle

\begin{abstract}
This paper addresses the problem of finding an orientation and a minimum radius
for directional antennas of a fixed angle placed at the points of a planar set
, that induce a strongly connected communication graph.
We consider problem instances in which antenna angles are fixed at 
and , and establish upper and lower bounds for the minimum radius
necessary to guarantee strong connectivity. In the case of  angles, we
establish a lower bound of  and an upper bound of .
In the case of  angles, we establish a lower bound of
 and an upper bound of .
Underlying our results is the assumption that the unit disk graph for
 is connected.
\end{abstract}


\section{Introduction}
Let  be a set of points in the plane representing wireless nodes.
Assume that each node is equipped with one directional antenna,
geometrically represented as a wedge with angular aperture  and radius 
(see Figure~\ref{fig:antenna}a). An antenna orientation is given by the
counterclockwise angle  measured from the positive -axis to the bisector
of the wedge.
The \emph{communication graph}  formed by the antennas placed at points
in  is a directed graph with vertex set  and edges 
directed from  to , if and only if the antenna wedge at  covers .
Let  denote the unit disk graph for  (i.e., the graph in which any two
points in  within unit distance are connected by an edge).
In this paper we address the following problem.

\begin{center}
\begin{minipage}{0.75\linewidth}
Let  be a planar point set such that  is connected.
For a fixed angle , find an orientation 
of the antennas at the points in  and a minimum radius  for which
the communication graph  is \emph{strongly connected}.
\end{minipage}
\end{center}

\medskip
\noindent
We consider instances of this problem for  (Section~\ref{sec:180}) and
 (Section~\ref{sec:90}), and establish lower and upper bounds for the
minimum radius required to guarantee strong connectivity.
For the case , we establish a lower bound of  and an upper bound
of .
For the case  angles, we establish a lower bound of
 and an upper bound of . Underlying these
results is the assumption that  is connected. We note that the recent related work by 
Ben-Moshe et. al~\cite{bcckms-dawn-10}
also considers -antennas but with orientations
restricted to the four standard quadrant directions, and it gives an algorithm for
constructing a bidirectional communication graph using a radius value 
 that is at most twice the optimal value.

Throughout the paper, we use the notation  to refer to a minimum spanning tree of 
of \emph{maximum degree five}, which can be constructed using the algorithm
by Wu et al.~\cite{WDJLH06}. We say that a point  \emph{sees}
 if and only if the antenna wedge at  covers .

\begin{figure}[htpb]
\centering
\includegraphics[width=0.4\linewidth]{antenna}
\caption{(a) Directional antenna represented as a wedge of angle  and radius  (b)  is a
directed edge in the communication graph.}
\label{fig:antenna}
\end{figure}


\section{ Antennas}
\label{sec:180}

\begin{theorem}
For directional antennas with ,  is
sometimes necessary to build a strongly connected communication graph.
\label{thm:180main}
\end{theorem}
\begin{proof}
Figure~\ref{fig:lowerbound180} shows a point set for which
 is
necessary.
The solid line segments show the UDG; all angles are .
Note that for any , any antenna placed at the
point labeled  covers exactly two of its neighbors in the UDG
and no other points. Split the UDG into two pieces,  and ,
by removing the edge
connecting  to its uncovered neighbor. Let  be the
piece containing . Observe that
for any point  in ,
the distance from  to any point in  is at least .
Since messages must flow from  to ,
 is necessary.
\end{proof}
\begin{figure}[hp]
\centering
\includegraphics[width=0.65\linewidth]{lowerbound180}
\caption{ is necessary for this point set when .}
\label{fig:lowerbound180}
\end{figure}


\begin{theorem}
For directional antennas with , 
suffices to build a strongly connected communication graph for a planar
point set .
\end{theorem}
\begin{proof}
We begin by constructing .  Let a point in  with a highest
 coordinate be the root.
We first partition the nodes of  into groups, then show inductively how to
orient the antennas in each group to form the communication graph.
To identify the groups, pick a node of height one and place it in a group
along with its children.  Conceptually imagine removing this group from the
tree, then repeat the process until no nodes are left (with the
possible exception of the root).  See Figure~\ref{fig:groups} for an
example of node grouping.
\begin{figure}[htpb]
\centering
\includegraphics[width=0.35\linewidth]{180groups}
\caption{Theorem~\ref{thm:180main}: Tree partitioned into groups.}
\label{fig:groups}
\end{figure}

We prove the theorem inductively on the number of groups  in .
The base case corresponds to a tree with one group only (). Let  be
the parent node, and  an arbitrary child of .
Orient the antennas at  and  so that they are both aligned
with the segment  and cover opposite sides of the plane.
See Figure~\ref{fig:placement180basecase}.
This placement establishes direct bidirectional
communication between  and  since the two cones overlap along
the segment .
For the other children (if any), orient their antennas in any direction
that includes . This enables direct communication from each child
 to .
Communication from  to  is indirect if  lies outside the
antenna wedge at , in which case the communication path is
. Note that the distance from  to any other child
of  is at most , therefore the radius  claimed by the
theorem guarantees connectivity from  to .

\begin{figure}[hptb]
\centering
\includegraphics[width=0.3\linewidth]{placement180basecase}
\caption{Theorem~\ref{thm:180main}: Orientation of antennas in the base case.}
\label{fig:placement180basecase}
\end{figure}

The inductive hypothesis claims that, in the case of a tree
 composed of  node groups, for some , there is an
orientation of antennas at the nodes of  that satisfies the theorem.
In addition, the inductive hypothesis requires that the root of 
can reach any hop within a unit distance.
Note that this is true of the base case, with help from 
if the hop is not covered by 's wedge.

To prove the inductive step, consider a tree  with 
groups. Let  be the root of , and call the group containing
 the \emph{root group}.
We discuss four cases, depending on the number of nodes in the root
group. The antenna placement for each of these cases is depicted in
Figure~\ref{fig:placement180}.
Observe that in the case of a triplet (Figure~\ref{fig:placement180}b),
's antenna is oriented so that is covers both children.
It can be verified, in the cases with one and two children depicted
in Figure~\ref{fig:placement180}(a, b),
 guarantees strong connectivity, and  can reach any
hop within a unit distance.

The cases with the root group composed of four and five nodes
follow a similar pattern, once divided into pairs and triplets of nodes,
as depicted in Figure~\ref{fig:placement180} (c, d).
The dotted lines indicate children that have been paired.
Pairs consisting of two children
(see  in Figure~\ref{fig:placement180}c) must be carefully
selected to achieve ; the
requirement is that the two paired children form a smallest
angle at . Since the cases under discussion involve
three or four children of , an upper bound on a smallest such
angle is . It follows that the distance between the paired
children is at most .
Then  guarantees the following:
(i) each pair and triplet of nodes is strongly connected, (ii)
each node in a pair can send messages to  (because a node in a
pair can reach its counterpart node with
a setting , and at least one node in a pair can
reach  with a setting ), and
(iii)  can reach any hop within unit distance (which includes its children).
It follows that the communication graph for the root group is strongly connected.
It remains to show that  is strongly connected.

(We pause here to note that a group cannot have five children, since each node
has degree at most five in , and the parent accounts for one of these
degree units. The one exception is the root of the entire tree. But since the
root is the point with a highest  coordinate, all its children must lie in
a halfplane. The minimum angle separating two adjacent points in a planar
minimum spanning tree is , so the root can also have most four
children.)

\begin{figure}[htpb]
\centering
\includegraphics[width=0.85\linewidth]{placement180}
\caption{Theorem~\ref{thm:180main}: Orientation of antennas in node groups.}
\label{fig:placement180}
\end{figure}

By the inductive hypothesis, each subtree 
attached to a node in the root group forms a strongly connected
communication graph. In addition, the root of  can reach any
hop within unit distance, and therefore it can send messages up to
the point of attachment. So to complete the inductive step, we
must show that each node in the root group can send messages down
to the root of an attached subtree. This is trivially true for , since  can communicate with any
point within unit distance, as established above.

Now consider the case when a child  of  is the attachment point
for a subtree with root .  If  plays the role of  in a
pair or a triplet (as in Figures~\ref{fig:placement180}a,b) and
the antenna wedge at  does not cover , then 
suffices to establish the communication path .
If  plays the role of  in a triplet (as in Figure~\ref{fig:placement180}b)
and  does not see , then there are two cases to
consider. First, if  sees , then the communication path is
. Otherwise,  must see .
The segment  cannot cross  since minimum spanning tree
edges do not cross.
This implies that  is confined to the shaded region from Figure~\ref{fig:bound180},
therefore the distance from  to  is at most . It follows that
 establishes the communication path .

\begin{figure}[htpb]
\centering
\includegraphics[width=0.3\linewidth]{bound180}
\caption{The communication path from  to  is .}
\label{fig:bound180}
\end{figure}

Finally, suppose that  is attached to a child of  that was
paired with another child of , such as  in
Figure~\ref{fig:placement180}c. If  cannot see ,
then  must be able to see , so  can use the communication path
. Since the distance between  and 
is at most , the last hop from  to  is no
longer than , matching the upper bound on 
stated in the theorem.

It is possible that the root group contains a single node,
the root  of . In this case, we deal
with  separately by orienting its antenna in the negative  direction.
The root is a highest point and therefore it can see all its
children, establishing direct communication with them.
By the inductive hypothesis, the children of  can also send
messages to , so this case is settled as well.
\end{proof}

\section{Antennas}
\label{sec:90}

\begin{theorem}
For directional antennas with ,  is
sometimes necessary to build a communication graph.
\end{theorem}
\begin{proof}
Consider a set of points positioned along the -axis
at unit intervals.
An antenna placed at a point  can only cover points to one side,
say its left side (so the antenna at  is oriented to the left).
To enable messages to flow from points left of  to points
right of , the antenna at some point left of  must be oriented
to the right. The nearest such point is 's left neighbor, which
is at distance two from 's right neighbor.
Therefore,  is necessary.
\end{proof}


\begin{theorem}
For any four points in general position, their antennas
can be oriented such that (i) a radius equal to the maximum pairwise
distance between the four points guarantees strong connectivity
of the four points, and (ii) the four antennas cover the entire plane.
\label{thm:fourpoints}
\end{theorem}
\begin{proof}
Consider first the case when the four points lie in convex
position. Let  be the points in clockwise order around
the hull. Since  is convex, the segments  and  must
intersect, as illustrated in Figure~\ref{fig:fourpoints}a.
Assume without loss of generality that  is the longer of the
two segments, and therefore the projection of at least one of 
and  onto the line supported by  lies on the segment .
The orientation of antennas depends on the counterclockwise
angle  from  to . We will assume ;
the case when  is handled
symmetrically by reflection about the vertical.
It is not difficult to see that the orientation of antennas
from Figure~\ref{fig:fourpoints}a covers the entire plane, since
 and  intersect and  is less than . This
settles claim (ii) of the lemma. We now turn to claim (i) of the lemma.

\begin{figure}[htpb]
\centering
\includegraphics[width=\linewidth]{fourpoints}
\caption{ Antennas for points  in convex position (a) and corresponding communication graph (b).
Antennas for points  in non-convex position (c) and corresponding communication graph (d).}
\label{fig:fourpoints}
\end{figure}

Let each antenna wedge have radius equal to the maximum
pairwise distance between .
First note that, for each node pair  and , each
node is contained in the antenna wedge of the counterpart node,
enabling direct communication between the nodes within a pair.
Communication between the pairs is settled as follows. Assume
that it is the projection of  that lies on the segment
, as shown in Figure~\ref{fig:fourpoints}a.
Then  is contained in 's wedge, and 's wedge
contains , thus enabling full communication between the pairs, as
illustrated in Figure~\ref{fig:fourpoints}b.

Consider now the case when the four points do not lie
in convex position. Then three of the points, say , comprise a
triangle that contains the fourth point, . See Figure~\ref{fig:fourpoints}c.
Assume without loss of generality that  is a longest edge of
. Then the the projection  of  onto 
lies interior to the segment . Let  contain 
(the situation when  contains  is vertically
symmetric). The antenna orientation is depicted in
Figure~\ref{fig:fourpoints}c: all antenna wedges have one boundary line
parallel to ; the antennas at points  and  face each other,
and similarly at points  and . This guarantees coverage of the entire
plane. In terms of connectivity, note that the nodes within each pair
 and  can communicate directly in both directions.
Since , both  and  can see , and  can
see  and  (recall that  is the longer side of ,
therefore  is acute, which implies that  sees ).
This establishes full communication between the pairs.
\end{proof}

\begin{theorem}
For directional antennas with ,  suffices
to build a strongly connected communication graph for a planar point set .
\label{thm:90main}
\end{theorem}
\begin{proof}
The case when  consists of two points  and  is trivial: orient the antennas
at  and  to point to each other.
If  consists of three points  and , then  has at least two angles
strictly smaller than . Orient the antennas at the apexes of these two angles
to cover the entire triangle, then orient the third antenna toward either of the other two
(see Figure~\ref{fig:3points}a).
\begin{figure}[hptb]
\centering
\includegraphics[width=0.4\linewidth]{3points}
\caption{A set  of  points. (a) Antenna orientation  (b) Communication graph.}
\label{fig:3points}
\end{figure}
Then  suffices to form a strongly connected communication graph, since
.

We now turn to the general case .
We create groups of nodes in  recursively as follows.
Starting at the bottom of , identify a smallest subtree
 of four or more nodes, whose removal does not disconnect
. It can be verified that such a subtree is topologically
equivalent to one of the subtrees shown in Figure~\ref{fig:nodegroups} (note
that the dashed connections are possible, but not required in the subtree).
Remove  from  and recurse. The process stops when 
is left with three or fewer nodes.
The result is a collection  of node groups, each with four or more vertices,
with the possible exception of the root subtree (the one containing
the root of ). In each group we select four representative nodes, one
of which must be the root of the group subtree, and the other three could be
arbitrary.
For definitiveness we select the nodes shaded in Figure~\ref{fig:nodegroups} as
representative nodes.

\begin{figure}[hptb]
\centering
\includegraphics[width=0.55\linewidth]{nodegroups}
\caption{Groups of four nodes that enable the use of Theorem~\ref{thm:fourpoints}. Dashed connections may or may not exist.}
\label{fig:nodegroups}
\end{figure}


We prove the theorem inductively on the number of groups  in .
The base case with  corresponds to a group of nodes arranged in a subtree
topologically equivalent to one of the trees depicted in Figure~\ref{fig:nodegroups}.
The representative node set in each group is , with  the
root of the group subtree. Note that the maximum pairwise distance between
nodes in  is  for the subtree depicted in
Figure~\ref{fig:nodegroups}a, and  for the subtrees
depicted in Figure~\ref{fig:nodegroups}(b, c, d).
We use Theorem~\ref{thm:fourpoints} on  to determine an orientation of the
antennas at nodes in  that strongly connects , for .
Then  enables any node in  to reach any hop
within unit distance, because the antennas at nodes in 
collectively cover the entire plane.

The inductive hypothesis is that there is an orientation of antennas at the nodes
of  consisting of  or fewer groups, that satisfies the theorem.
In addition, the inductive hypothesis requires that the root of 
can reach any hop within a unit distance. This additional requirement was already
established for the base case.

To prove the inductive step, consider a tree  with  groups. Assume first
that the root group contains at least four nodes, so they are arranged in a subtree
 topologically equivalent to
one of the trees from Figure~\ref{fig:nodegroups}.
As in the base case, we orient the antennas at the representative nodes of
 as in Theorem~\ref{thm:fourpoints}, to establish coverage of the plane and
strong connectivity between these nodes, for .
For each non-representative node , orient the antenna at  to point towards
a closest representative node .
A simple analysis of the tree topologies from Figure~\ref{fig:nodegroups}
indicate that, in order to establish a connection from  to , a
radius of  for the antenna at  suffices for the cases depicted in
Figure~\ref{fig:nodegroups}(a,d), and a radius of  suffices for the cases depicted
in Figure~\ref{fig:nodegroups}(b, c). Summing up these values with ,
we obtain that  establishes full connectivity between the nodes of 
(since one of the nodes in  can reach  in this case as well).
We now show that  guarantee strong connectivity of .

\begin{figure}[hptb]
\centering
\includegraphics[width=\linewidth]{90induction}
\caption{(a) The inductive step for Theorem~\ref{thm:90main}. The communication paths indicated by circular arcs are guaranteed by the inductive hypothesis. The directed dashed edges represent communication paths established in the inductive step. (b) The root subtree with three nodes (boxed) viewed as part of a child subtree.}
\label{fig:90induction}
\end{figure}

The inductive hypothesis, along with the fact that each child in  is
within unit distance from its parent, implies that each subtree attached to a node
 can send messages up to 
(see the circular arcs in Figure~\ref{fig:90induction}a).
We have established that a setting of  enables strong connectivity
between the nodes of . It follows that  enables each node
 to reach each child  of , because with this setting at
least one of the nodes in  can reach  (since their antennas cover the entire plane),
and  can reach any node in .
In addition,  enables the root of  to reach any hop
within unit distance (by a similar argument).
This settles the inductive step for this case.

If the node group at the root of  contains fewer than four nodes,
this group can be viewed as attached to the root  of any adjacent node group
in . This idea is illustrated in Figure~\ref{fig:90induction}b, where
a root subtree of three nodes is attached in turn to each ``full'' subtree (with four or more nodes).
Regardless of the topological structure of the root subtree , the
maximum distance between any two nodes in  does not exceed .
Orient the antennas at each node in  towards . Then
each node in  can reach  with . A simple analysis
of the configurations from Figure~\ref{fig:90induction}b shows that
a representative node of the subtree rooted at  can reach any node in 
with an increase of  in its transmission radius. Then the same analysis
as before shows that  settles the inductive step. It is likely that
a more complex analysis of this special case (with the root group composed of
three or fewer nodes) can maintain the previously established bound of .
Such an analysis is left for future work.
\end{proof}


\begin{thebibliography}{1}

\bibitem{WDJLH06}
Weili Wu, Hongwei Du, Xiaohua Jia, Yingshu Li, and Schott C.-H. Huang.
\newblock Minimum connected dominating sets and maximal independent sets in
  unit disk graphs.
\newblock {\em Theoretical Computer Science}, 352:1--7, 2006.

\bibitem{bcckms-dawn-10}
Boaz Ben-Moshe, Paz Carmi, Lilach Chaitman, Matthew Katz, Gila Morgenstern, and Yael Stein.
\newblock Direction Assignment in Wireless Networks.
\newblock {\em Proc. 22nd Canad. Conf. Comput. Geom.}, pages 39--42,  2010.


\end{thebibliography}

\end{document}
