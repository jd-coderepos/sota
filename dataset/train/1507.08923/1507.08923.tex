\documentclass[final,5p,times,twocolumn]{elsarticle_mod}
\usepackage{url}
\usepackage{amsmath}
\usepackage{amssymb}
\usepackage{amsthm}
\usepackage{graphics}
\usepackage{graphicx}
\usepackage{times}
\usepackage{algorithm,algorithmic}


\renewcommand{\figurename}{Fig.}

\newcommand{\IFF}{\leftrightarrow}
\newcommand{\IMP}{\rightarrow}
\newcommand{\BetaD}[2]{\mathrm{Beta}(#1,#2)}
\newcommand{\BetaF}[2]{\mathrm{B}(#1,#2)}
\newcommand{\EBeta}[2]{\mathrm{B}(#1,#2)}
\newcommand{\ExpD}[1]{\mathrm{Exp}(#1)}
\newcommand{\GammaD}[2]{\mathrm{\Gamma}(#1,#2)}
\newcommand{\cv}[1]{\mathbf{cv}[#1]}
\newcommand{\con}[1]{\mathbf{cv}[#1]}
\newcommand{\cov}[2]{\mathbf{cov}[#1,#2]}
\newcommand{\E}[1]{\mathbf{E}\left[#1\right]}
\newcommand{\F}[1]{\mathbf{E^2}\left[#1\right]}
\newcommand{\ELRI}[1]{\mathbf{ELRI}[#1]}
\newcommand{\ii}{\mathbf{i}}
\newcommand{\std}[1]{\mathbf{std}[#1]}
\newcommand{\var}[1]{\mathbf{var}\left[#1\right]}



\newcommand{\UPB}[2]{\frac{1}{#1} \ln \left( \frac{n (\ln n)^{k-1}}{(#2 -1)!} \right)}

\newtheorem{theorem}{Theorem}
\newtheorem{Corollary}[theorem]{Corollary}
\newtheorem{Definition}[theorem]{Definition}
\newtheorem{Lemma}[theorem]{Lemma}
\newtheorem{Remark}[theorem]{Remark}
\newtheorem{Claim}[theorem]{Claim}

\newcommand{\NN}{\mathbb{N}}
\newcommand{\RR}{\mathbb{R}}
\newcommand{\ZZ}{\mathbb{Z}}
\newcommand{\card}[1]{\mathrm{card}{\left(#1\right)}}

\newcommand{\NODES}{\mathcal{N}}
\newcommand{\Res}[2]{\hbox{Res}\left[#1|z=#2\right]}
\newcommand{\Chord}{\mathrm{Chord}}
\newcommand{\fChord}[1]{\mathrm{fChord_{#1}}}
\newcommand{\mChord}[1]{\mathrm{r_{#1}\!-\!Chord}}
\newcommand{\uChord}[1]{\mathrm{uChord_{#1}}}
\newcommand{\SmallC}[2]{\mathrm{S_{#1}^{c}(#2)}}
\newcommand{\SmallF}[2]{\mathrm{S_{#1}^{f}(#2)}}
\newcommand{\SmallU}[2]{\mathrm{S_{#1}^{u}(#2)}}
\newcommand{\LargeC}[2]{\mathrm{L_{#1}^{c}(#2)}}
\newcommand{\LargeF}[2]{\mathrm{L_{#1}^{f}(#2)}}
\newcommand{\LargeU}[2]{\mathrm{L_{#1}^{u}(#2)}}
\newcommand{\ESmallC}[2]{\mathbf{E}\left[\mathrm{S_{#1}^{c}\left(#2\right)}\right]}
\newcommand{\ESmallF}[2]{\mathbf{E}\left[\mathrm{S_{#1}^{f}\left(#2\right)}\right]}
\newcommand{\ESmallU}[2]{\mathbf{E}\left[\mathrm{S_{#1}^{u}\left(#2\right)}\right]}
\newcommand{\ELargeC}[2]{\mathbf{E}\left[\mathrm{L_{#1}^{c}\left(#2\right)}\right]}
\newcommand{\ELargeF}[2]{\mathbf{E}\left[\mathrm{L_{#1}^{f}\left(#2\right)}\right]}
\newcommand{\ELargeU}[2]{\mathbf{E}\left[\mathrm{L_{#1}^{u}\left(#2\right)}\right]}

\urldef{\mailsa}\path|rafal.kapelko@pwr.edu.pl| 
\newcommand{\keywords}[1]{\par\addvspace\baselineskip
\noindent\keywordname\enspace\ignorespaces#1}


\begin{document}

\begin{frontmatter}

\title{On the Displacement for Covering a Unit Interval with Randomly Placed Sensors}

\author[pwr]{Rafa\l{} Kapelko\corref{cor1}\fnref{pwrfootnote}}
\ead{rafal.kapelko@pwr.edu.pl}
\author[scs]{Evangelos Kranakis\fnref{scsfootnote}}\ead{kranakis@scs.carleton.ca }
\fntext[scsfootnote]{Research supported by grant nr S500129/K1102.}
\fntext[scsfootnote]{Research supported in part by NSERC Discovery grant.}
\cortext[cor1]{Corresponding author at: Department of Computer Science,
Faculty of Fundamental Problems of Technology, Wroc{\l}aw University of Technology, 
 Wybrze\.{z}e Wyspia\'{n}skiego 27, 50-370 Wroc\l{}aw, Poland. Tel.: +48 71 320 33 62; fax: +48 71 320 07 51.}
\address[pwr]{ Department of Computer Science, Faculty of Fundamental Problems of Technology, Wroc{\l}aw University of Technology, Poland}
\address[scs]{School of Computer Science, Carleton University, Ottawa, ON, Canada}
\begin{abstract}

Consider  mobile sensors placed independently at random with the uniform distribution on a barrier represented as the unit line segment . 
The sensors have identical sensing radius, say . When a sensor is displaced on the line a distance equal to  it consumes energy (in movement) which is proportional to some (fixed) 
power  of the distance  traveled. The energy consumption of a system of  sensors thus displaced is defined as the sum of the energy consumptions for the displacement 
of the individual sensors. 

We focus on the problem of energy efficient displacement of the sensors so that in their final placement the sensor system ensures coverage of the barrier and the energy consumed 
for the displacement of the sensors to these final positions is minimized in expectation. In particular, we analyze the problem of displacing the sensors from their initial positions 
so as to attain coverage of the unit interval and derive trade-offs for this displacement as a function of the sensor range. We obtain several tight bounds in this setting 
thus generalizing several of the results of ~\cite{spa_2013} to any power .  


\end{abstract} 

\begin{keyword}
  barrier, displacement, distance, random, sensors,
\end{keyword}

\end{frontmatter}

\section{Introduction}
One of the most important problems in sensor networks is minimizing battery consumption when accomplishing various tasks such as monitoring an environment, tracking events along a barrier and communicating. In this study, the environment being considered consists of a line segment barrier (which for simplicity is set to the unit interval ), while the accompanying monitoring problem investigated is ensuring coverage of the barrier in the sense that every point in the line segment is within the range of a sensor. 

We consider the case where the sensors are equipped with omnidirectional sensing antennas of identical range ; thus a sensor placed at location  in the unit interval 
can sense any point at distance at most  either to the left or right of . 
The initial placement of the sensors does not guarantee barrier coverage since the sensors have been placed initially independently at random with the uniform distribution on a barrier. 
To attain coverage of the line segment it is required to displace the sensors from their original locations to new positions on the line while at the same time taking 
into account their sensing range . Further, for some fixed constant  if a sensor is displaced a distance  the energy consumed by this sensor is considered 
to be proportional to . 
More generally, for a set of  sensors, if the th sensor is displaced a distance , for , 
then the energy consumed by the whole system of  sensors is .  

In this paper we study the minimum total (or sum) energy consumption (in expectation) in the movement of the sensors so as to attain coverage of the unit segment when 
the energy consumed per sensor is proportional to some (fixed) power of the distance traveled. The present study generalizes some known results (see~\cite{spa_2013}) on the sensor displacement for  to arbitrary . Motivation for the extended model being proposed is that the energy consumption induced by individual sensor displacement may not be linear in this displacement, but rather be dependent on some power of the distance traversed. Further, the parameter  in the exponent may well represent various conditions of the surface of the barrier, e.g., friction, lubrication, etc, which may affect the overall energy consumption of the sensor system. 

\subsection{Related work}

There is extensive literature about area and barrier (also known as perimeter) coverage by a set of sensors (e.g., see \cite{abbasi2009movement,barriercoverageNodeDegree,SSL07,kumar2005,saipulla2009,swat2012}). The coverage problem for planar domains with pre-existing anchor (or destination) points was introduced in \cite{tcs2009}. The deterministic version of the sensor displacement problem on a linear domain (or interval) was introduced in \cite{adhocnow2009}. Several optimization variants of the displacement problem were considered.The complexity of finding an algorithm that optimizes the displacement depends 1) on the types of the sensors, 2) the type of the domain, and 3) whether one is minimizing the sum or maximum of the sensor movements. For the unit interval the problem of minimizing the sum is NP-complete if the sensors may have different ranges but is in polynomial time when
all the sensor ranges are identical \cite{adhocnow2010}. The problem of 
minimizing the maximum is NP-complete if the region consists of two intervals \cite{adhocnow2009}
but is polynomial time for a single interval even when the sensors may have different
ranges \cite{swat2012}.
Related work on deterministic
algorithms for minimizing the total and maximum movement
of sensors for barrier coverage of a planar region may
be found in \cite{tcs2009}. 


More importantly, our work is closely related to the work of \cite{spa_2013} where the authors consider the expected minimum total displacement for establishing full coverage of a unit interval
for  sensors placed uniformly at random. Our analysis and problem statement generalizes some of the work of \cite{spa_2013} from  to all exponents . A comprehensive study of sensor displacement to arbitrary probability distributions using techniques from queueing theory can be found in the forthcoming \cite{kranakis2014scheduling}.


\subsection{Outline and results of the paper}

Our work generalizes some of the work
of \cite{spa_2013} to the more general setting
when the cost of movement is proportional to a fixed power
of the distance displacement.

The overall organization of the paper is as follows. 
In Section~\ref{sec:Introduction}
we provide several basic combinatorial facts
that will be used in the sequel. In
Section~\ref{tight:sec} we prove combinatorially how
to obtain tight bounds when the range of the sensors 
is . 
We show that the expected sum of displacement to the power  is 

when  is an even positive number ,
and in  when  is an odd natural number.
In
Section~\ref{Minimum displacement:sec}
we prove the occurrence of threshold
whereby the expected minimum sum of
displacements to the power  ( is positive natural number) remains in 
provided that  where  and 
In Section~\ref{upper bounds:sec} we
study the more general version of the sensors movement to the power  where  and
. If  we first present the Algorithm \ref{alg_power} that uses expected 

total movement to power  
where 
Finally,
Section~\ref{conclusion}
provides the conclusions. 
\section{Basic facts}
\label{sec:Introduction}
In this section we recall some known facts about special functions and special numbers which will be useful in the analysis
in the next sections. The Euler Beta function (see \cite{NIST})

is defined for all complex numbers  
such as  and . 
Moreover, for positive integer numbers  we have

Let us define a function  on the interval 
We say that a random variable  concentrated on the interval
 has the  distribution with parameters
 if it has the probability density function
 Hence,

We will  use the following notations
for the rising and falling factorial respectively \cite{concrete_1994}


Let    be the Stirling numbers of the first and second kind respectively, which are
defined for all integer numbers such that  
The following two equations for Stirling numbers of the first and second kind are well known  (see \cite[identity 6.10]{concrete_1994} and \cite[identity 6.13]{concrete_1994}):


Assume that  is a constant independent of  Then the following Stirling numbers

are polynomials in the variable  and of degree  (see \cite{concrete_1994}).\\
Let    be the Eulerian numbers, which are
defined for all integer numbers such that  
The following three identities for Euler numbers are well known (see identities   and  in \cite{concrete_1994}):



Let  be non-negative integers. Notice that (see \cite[identity 5.41]{concrete_1994})

Observe that

Applying this formula for  to  we easily derive

We will also use Euler's Finite Difference Theorem (see \cite[identity 10.1]{Gould}).
Assume that  is a natural number. Let  and   Then

\section{Tight bounds for total displacement to the power  when }
\label{tight:sec}
In this section we extend Theorem 1 from \cite{spa_2013} for the displacement
to the power  where  is a positive natural number.  Assume that  sensors with range  are thrown uniformly and independently at random in the unit interval
and move from their current location to the anchor location  for 
Notice that the only way to attain the coverage is for the sensors to occupy the positions  for 
We prove that the expected sum of displacement to the power  is 

when  is an even positive number,
and in  when  is an odd natural number.
We begin with the following lemma which will be helpful in the proof of Theorem \ref{thm:mainexactodd}.
It is worth pointing out that the proof of Lemma \ref{lem:sum} is technically complicated.
Our proof of Lemma \ref{lem:sum} proceeds along
the following steps.
Firstly, we  reduce the inner sum  

to the sum 

which is known (see equation (\ref{eq:identity})).
Then we have the following sum

where  is the polynomial of variable  of degree less than or equal 
Finally, the asymptotic follows from Euler's Finite Difference Theorem (see equation ( \ref{eq:triangle})).
\begin{Lemma}
\label{lem:sum}
Assume that  is an even positive number. Then

\end{Lemma}
\begin{proof}
As a first step, we evaluate the inner sum.
Let 
Applying equation (\ref{eq:stirling}) for  , 
and equation (\ref{eq:identity}) for  ,
as well equation (\ref{eq:stirling2}) for    we deduce that

Hence 

Now we prove that  is the polynomial of variable  of degree less than or equal 
Observe that

Since  is the polynomial of variable  of degree 
 is the polynomial of variable  of degree 
and  is the polynomial of variable  of degree  (see (\ref{eq:stirling_since})),
we obtain that,  is the polynomial of variable  of degree less than or equal  

Now we give the coefficient of the term  in the polynomials 
Applying identity (\ref{eq:euler2}) for   and identity (\ref{eq:euler3}) for
  we observe that the coefficient of the term  in the polynomials  equals

Therefore, from equation (\ref{eq:euler1}) we have

Using identity (\ref{eq:altalt}) we deduce that


We now apply equation ( \ref{eq:triangle})) in order to  get


Putting everything together, we finally obtain

This completes the proof of Lemma \ref{lem:sum}.
\end{proof}

\subsection{Tight bound for total displacement to the power  when  and  is an even positive number}

\begin{theorem} 
\label{thm:mainexactodd} 
Let  be an even positive number. Assume that  mobile sensors are thrown uniformly and independently at random in the unit interval. The expected sum
of displacements to the power  of all sensors to move from their current location to the anchor location 
for  respectively is  
\end{theorem}
\begin{proof}
Let  be the th order statistic, i.e., the position of the th sensor in interval 
We know that the random variable  has the  distribution. For example see \cite{nagaraja_1992}.
Assume that  is an even positive number.
Let  be the expected distance to the power  between  and the  sensor anchor location,  on the
unit interval, hence given by:

Now we define

for  and  Observe that

From the definition of Beta function and identity (\ref{Emult}) we get

Hence applying Lemma \ref{lem:sum} we conclude that

This finishes the proof of Theorem \ref{thm:mainexactodd}. 
\end{proof}
\subsection{Tight bound for total displacement to the power  when  and  is an odd natural number}
\begin{theorem} 
\label{thm:mainexact_oddtwo} 
Let  be an odd natural number. Assume that  mobile sensors are thrown uniformly and independently at random in the unit interval. The expected sum
of displacements to the power  of all sensors to move from their current location to anchor location 
for  respectively is 
\end{theorem}
\begin{proof} Let  be an odd natural number.
Firstly, observe that the result for  follows from [\cite{spa_2013}, Theorem 1]. Therefore, we may assume that 
Let  be the expected distance to the power  between  and the  target anchor location,  on the
unit interval, hence given by:


First we prove the upper bound.
We use discrete H\"older inequality with parameters   
and get

Next we use H\"older inequality for integrals with parameters    and get

so 

Putting together Theorem \ref{thm:mainexactodd} for  and equations (\ref{eq:holders}), (\ref{eq:holder10}) we deduce that


Next we prove the lower bound.
We use discrete H\"older inequality with parameters   
and get

Next we use H\"older inequality for integrals with parameters   and get

so 

Putting together Theorem \ref{thm:mainexactodd} for  and equations (\ref{eq:holder3}), (\ref{eq:holdera}) we obtain

This finishes the proof of the lower bound and completes the proof of Theorem \ref{thm:mainexact_oddtwo}. 
\end{proof}

\section{A Threshold on the minimum displacement}
\label{Minimum displacement:sec}
In this section
we prove the occurrence of threshold
whereby the expected minimum sum of
displacements to power  where  is positive natural number, remains in 
provided that  where  and 
\begin{Definition}
Given  we denote by  the expected minimum sum of displacement to the power  (where  is positive natural number) of  sensors
with range 
\end{Definition}
\begin{theorem}
\label{cov1:thm}
Assume that  is a natural number. Let  be the range of the sensors.
If    where  and  then

\end{theorem}
\begin{proof} Let  be a natural number.
Assume that  where  and 
Throughout the proof we use the fact that


First we prove the upper bound
. This is easy because we can
displace the sensors to the anchor locations ,
for  at a total displacement cost of
 This suffices if 
since in this case the contiguous coverage is assured.

Next we prove the lower bound
.
We would like to know how much we can reduce
the sum of displacements if we change the radius
from 
to  where  and 
Let  be the sequence such that 
  and  for 
Let  be the position of the ith sensor in the interval 
It is sufficient to show that

Let us recall that  for  Using the inequality
(for )

we get

By Theorem \ref{thm:mainexactodd}, Theorem \ref{thm:mainexact_oddtwo}
we know 

Assume that 
for  Let  be the smallest positive , such that 

Clearly, if the th sensor occupies position  for  then the distance between consecutive sensors is equal to 
Observe that  for  and
 Hence,

Therefore, we conclude that for all sequences  such that 
  and  for 

Putting together (\ref{eq:mink}), (\ref{eq:exest}) and (\ref{eq:asyn}) we get

This is sufficient to complete the proof of Theorem~\ref{cov1:thm}.
\end{proof}

\section{Upper bounds for total displacement when }
\label{upper bounds:sec}
Now we study a more general version of the sensor movement to power  where 
Suppose that  sensors with radius  are thrown randomly and independently with the uniform distribution in the unit interval.
The question is how to estimate the  total expected movement  to the power  for ? 
If  we present Algorithm \ref{alg_power} that uses expected 

total movement to power  
where  The correctness of the algorithm is derived from Theorem \ref{thm:alg}.

We begin with a theorem which indicates how to apply the results of 
Theorem \ref{thm:mainexactodd}  and Theorem \ref{thm:mainexact_oddtwo}
to displacements to the fractional power .
\begin{theorem}
 \label{thm:fractional}
Let  . Assume that  mobile sensors are thrown uniformly and independently at random in the unit interval. The expected sum
of displacements to the power  of all sensors to move from their current location to anchor location 
for  respectively is 
\end{theorem}
\begin{proof}
By Theorem \ref{thm:mainexactodd} and Theorem \ref{thm:mainexact_oddtwo} we may assume that  and 
Let  be the expected distance to the power  between  and the  anchor location,  on the
unit interval, hence given by:

Then we use discrete H\"older inequality with parameters   
and get

Next we use H\"older inequality for integrals with parameters    and get

so 

Putting together Theorem \ref{thm:mainexactodd}, Theorem \ref{thm:mainexact_oddtwo}
and equations (\ref{eq:holder}), (\ref{eq:holder15}) we deduce that

This finishes the proof of Theorem \ref{thm:fractional}. 
\end{proof}
Now we give a lemma which indicates how to scale the results of Theorem \ref{thm:fractional}
to intervals of arbitrary length.
\begin{Lemma}
\label{lem:scale}
Let  Assume that  mobile sensors are thrown uniformly and independently at random in the interval of length  The sensors are to be moved to
equidistant positions (within the interval) at distance  from each other. Then the total expected movement to the power  of the sensors
is 
\end{Lemma}
\begin{proof}
Assume that  sensors are in the interval  Then multiply their coordinates by  From Theorem \ref{thm:fractional} 
the total movement to the power  in the unit interval is in  Now by multiplying their coordinates by 
we get the desired result. 
\end{proof}
Our upper bound on the total sensor movement to power  is based on the Algorithm \ref{alg_power}.
\begin{algorithm}
\caption{Displacement to the power  when  \,\,\,\,  
 is the real solution of the equation  such that 
}
\label{alg_power}
\begin{algorithmic}[1]
 \REQUIRE  mobile sensors with identical sensing radius   placed uniformly and independently at random on the interval 
 \ENSURE  The final positions of sensors to attain coverage of the interval 
\STATE Divide the interval into subintervals of length ;
   \IF{there is a subinterval with fewer than  sensors} 
   \STATE{moves all  sensors to positions that are equidistant;} 
   \ELSE 
   \STATE{ in each subinterval choose  sensors at random and move the chosen sensors to equidistant position
  so as to cover the subinterval;}
   \ENDIF
\end{algorithmic}
\end{algorithm}
\begin{theorem}
\label{thm:alg}
Let   and  where 
is the solution of the equation  such that 
Assume that  sensors of radius  are thrown randomly and independently
with uniform distribution on a unit interval. 
Then the total expected movement to power  of sensors required to cover the interval is in

\end{theorem} 
\begin{proof} Assume that 
Let  and   
 is the solution of the equation  such that 
First of all, observe that  for 
We will prove that the total expected movement to power  of Algorithm \ref{alg_power}
is in 

There are two cases to consider.

Case 1: There exists a subinterval with fewer than 

sensors. In this case the total expected movement to power  is 
 by Theorem \ref{thm:fractional}.

Case 2:  All subintervals contain at least  sensors. From the inequality 

we deduce that, 

Hence it is possible to choose  sensors at random in each subinterval
with more than 

sensors. 
Let us consider the sequence

Applying inequality  we see that

Observe that


Putting together Equation (\ref{eq:an1}) and Equation (\ref{eq:an2}) we get

Therefore,  
chosen sensors are enough to attain the coverage.
By the independence of the sensors positions, the  
chosen sensors in any given subinterval are distributed randomly and independently with uniform distribution over the subinterval of length

By Lemma \ref{lem:scale} the total expected movement to power  inside each subinterval is 

Since, there are  subintervals, the total expected movement to power  over all subintervals must be in


It remains to consider the probability with which each of these cases occurs. The proof of the theorem will be a consequence of the following Claim.
\begin{Claim}
\label{claim:first}
Let 
 The probability that fewer than 
  
sensors fall in any subinterval is
 
\end{Claim}
\begin{proof} (Claim~\ref{claim:first})
First of all, from the inequality  we get
 
Hence,

The number of sensors falling in a subinterval is a Bernoulli process with probability of success  By Chernoff bounds, the probability
that a given subinterval has fewer than 
 
sensors is less than 
 
Specifically we use the Chernoff bound 



As there are  subintervals, the event that one has fewer than 

sensors occurs with probability less than  This and Equation \eqref{eq:chernof} completes the proof of Claim \ref{claim:first}. 
\end{proof}
Using Claim  \ref{claim:first} we can upper bound the total expected movement to power  as follows:

which proves Theorem \ref{thm:alg}. 
\end{proof}


\section{Conclusion}
\label{conclusion}
In this paper
we studied the expected minimum total (or sum) energy consumption in the movement of sensors with identical range when the energy consumed per sensor is proportional to some (fixed) power of the distance traveled. We obtained bounds on the expected minimum energy consumed 
depending on the range of the sensors. \bibliographystyle{plain}
\bibliography{refs}

\end{document}
