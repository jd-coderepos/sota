\documentclass{sig-alternate}


\usepackage{epsfig,subcaption,endnotes,svg, textcomp, float, algorithm, algpseudocode, url, balance}
\usepackage{paralist}
\usepackage{multirow}
\usepackage{dcolumn}
\usepackage{nameref}
\usepackage{color}
\usepackage{microtype}
\usepackage{pifont}


\newcommand{\avantguard}{\textsc{Avant-Guard}}

\pdfpagewidth=8.5in
\pdfpageheight=11in


\numberofauthors{4} 



\author{
	Moreno Ambrosin, Mauro Conti\thanks{\scriptsize Mauro Conti is supported by a Marie Curie Fellowship funded by the European Commission under the agreement No. PCIG11-GA-2012-321980. This work is also partially supported by the TENACE PRIN Project 20103P34XC funded by the Italian MIUR, and by the Project ``Tackling Mobile Malware with Innovative Machine Learning Techniques'' funded by the University of Padua.}, Fabio De Gaspari,\\
	\affaddr{University of Padua, Italy}\\
	{\{surname\}@math.unipd.it}\\
	{fabio.degaspari@studenti.unipd.it}
	\and
	Radha Poovendran\\
	\affaddr{University of Washington, USA}\\
	{rp3@uw.edu}	
}


\newfont{\mycrnotice}{ptmr8t at 7pt}
\newfont{\myconfname}{ptmri8t at 7pt}
\let\crnotice\mycrnotice \let\confname\myconfname 

\permission{Permission to make digital or hard copies of all or part of this work for personal or classroom use is granted without fee provided that copies are not made or distributed for profit or commercial advantage and that copies bear this notice and the full citation on the first page. Copyrights for components of this work owned by others than ACM must be honored. Abstracting with credit is permitted. To copy otherwise, or republish, to post on servers or to redistribute to lists, requires prior specific permission and/or a fee. Request permissions from permissions@acm.org.}
\conferenceinfo{ASIA CCS'15,}{April 14--17, 2015, Singapore.}
\copyrightetc{Copyright \copyright~2015 ACM \the\acmcopyr}
\crdata{978-1-4503-3245-3/15/04\ ...\\left\{ {IP, port } \right\}\left\{ {IP, port } \right\}64513ABRAISN_ARBISN_A+1ISN_RRBISN_RBRBISN_ARBRISN_A+1ISN_BISN_B+1RA\left\{ {IP, port} \right\}\left\{ {IP_{src}, port_{src}, port_R, \delta_{seq}} \right\}IP_{src}port_{src}port_R\delta_{seq}ABRBR\left\{ {IP_{B}, port_{B}} \right\}R2^{16} - 1024 = 64512BB6451264512P_pEEEEEIP_ET\times2^{count_{IP_E}}count_{IP_E}IP_ETP_p2^{16}=65535<IP_{dst}, port_{dst}>2^{16+32+16}P_pP_pP_pP_pP_p0.010.052^{20}2^{22}2^{22}P_p=0.052^{22}P_p=0.05$, LineSwitch requires 769.487~sec to be saturated against only 74.718~sec required when running \avantguard. 
When using lower (and more realistic) migration probability values, the time difference increases even more, as shown in Figure~\ref{fig:time_saturation}.
 
For completeness, we compared the average overhead introduced by \avantguard~with the overhead introduced by LineSwitch, evaluating both a scenario without attack, and a scenario under SYN flooding based control plane saturation attack.
All overhead data is expressed with respect to the standard OpenFlow protocol, under normal network conditions (e.g., no attacks performed).
We sampled the time required by the legitimate client (see Figure~\ref{fig:sim_setup}) to download a web page of size 1~KB from the web server. 
In the regular traffic scenario, \avantguard~introduces an average time overhead of 41.83\%, while LineSwitch incurs only a 7.67\% overhead. Moreover, under control plane saturation attack with an attack rate of 6.5~Mbps, \avantguard~introduces an overhead of 36.92\%, while LineSwitch introduces only a 5.45\% overhead. 
Finally, under control plane saturation attack, both \avantguard~and LineSwitch guarantee a 100\% page retrieval success rate.




\section{Conclusion}\label{sec:conclusion}
In this paper we analyzed the effects of the control plane saturation attack based on SYN flooding, one of the most widespread types of denial of service attack, 
when applied to Software Defined Networks (SDN) architecture, and in particular to its reference implementation, OpenFlow.
We showed that the extensive communication needed by the control plane and the data plane in SDN amplifies the effect of typical denial of service attacks, resulting in an overload of the control plane and in the possible impairment of large parts of the network. 
Furthermore we considered \avantguard~\cite{AvantGuard}, which is, to the best of our knowledge, the only currently proposed solution against control plane saturation attack. We showed that in its original design, subtle points were not taken into consideration, opening critical system vulnerabilities.
To this aim we proposed LineSwitch, a solution based on probability and blacklisting which offers both resiliency against SYN flooding-based control plane saturation attacks and protection from buffer saturation vulnerabilities.
Our preliminary evaluation demonstrates that LineSwitch imposes a negligible overhead, which can be dynamically adjusted to fit the network needs, 
while successfully defending the OpenFlow switch and controller from attacks that can potentially disrupt the functionality of the network.

\balance
\bibliographystyle{abbrv}
\bibliography{bibliography}

\end{document}
