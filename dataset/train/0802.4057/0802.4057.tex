\documentclass[times, 10pt]{article} 
\usepackage{latex8}
\usepackage{times}

\usepackage{dsfont}

\usepackage{xypic}
\usepackage{subfigure}
\usepackage{amsthm}\theoremstyle{proof}
\usepackage{mathrsfs}
\usepackage{amsmath,amssymb}
\usepackage{xspace}
\usepackage{stmaryrd}
\usepackage{url}

\usepackage{xy}
\xyoption{frame}
\xyoption{arrow}
\xyoption{curve}
\xyoption{arc}
\xyoption{line}
\xyoption{matrix}
\xyoption{tile}
\xyoption{ps}

\makeatletter
\newdimen\inferLineSkip         \inferLineSkip=2pt
\newdimen\inferLabelSkip        \inferLabelSkip=5pt
\def\inferTabSkip{\quad}
\newdimen\@LeftOffset   \newdimen\@RightOffset  \newdimen\@SavedLeftOffset      \newdimen\UpperWidth
\newdimen\LowerWidth
\newdimen\LowerHeight
\newdimen\UpperLeftOffset
\newdimen\UpperRightOffset
\newdimen\UpperCenter
\newdimen\LowerCenter
\newdimen\UpperAdjust
\newdimen\RuleAdjust
\newdimen\LowerAdjust
\newdimen\RuleWidth
\newdimen\HLabelAdjust
\newdimen\VLabelAdjust
\newdimen\WidthAdjust
\newbox\@UpperPart
\newbox\@LowerPart
\newbox\@LabelPart
\newbox\ResultBox
\newif\if@inferRule     \newif\if@DoubleRule    \newif\if@ReturnLeftOffset      \newif\if@MathSaved     \def\DeduceSym{\vtop{\baselineskip4\p@ \lineskiplimit\z@
    \vbox{\hbox{.}\hbox{.}\hbox{.}}\hbox{.}}}
\def\@SaveMath{\@MathSavedfalse \ifmmode \ifinner
       \relax \relax\fi }
\def\@IFnextchar#1#2#3{\let\reserved@e=#1\def\reserved@a{#2}\def\reserved@b{#3}\futurelet
    \reserved@c\@IFnch}
\def\@IFnch{\ifx \reserved@c \@sptoken \let\reserved@d\@xifnch
      \else \ifx \reserved@c \reserved@e\let\reserved@d\reserved@a\else
          \let\reserved@d\reserved@b\fi
      \fi \reserved@d}
\def\@ifEmpty#1#2#3{\def\@tempa{\@empty}\def\@tempb{#1}\relax
       \ifx \@tempa \@tempb #2\else #3\fi }
\def\infer{\@SaveMath \@IFnextchar *{\@inferSteps}{\relax
       \@IFnextchar ={\@inferDoubleRule}{\@inferOneStep}}}
\def\@inferOneStep{\@inferRuletrue \@DoubleRulefalse
       \@IFnextchar [{\@infer}{\@infer[\@empty]}}
\def\@inferDoubleRule={\@inferRuletrue \@DoubleRuletrue
       \@IFnextchar [{\@infer}{\@infer[\@empty]}}
\def\@inferSteps*{\@IFnextchar [{\@@inferSteps}{\@@inferSteps[\@empty]}}
\def\@@inferSteps[#1]{\@deduce{#1}[\DeduceSym]}
\def\deduce{\@SaveMath \@IFnextchar [{\@deduce{\@empty}}
       {\@inferRulefalse \@infer[\@empty]}}
\def\@deduce#1[#2]#3#4{\@inferRulefalse
       \@infer[\@empty]{#3}{\@SaveMath \@infer[{#1}]{#2}{#4}}}
\def\@infer[#1]#2#3{\relax
\if@ReturnLeftOffset \else \@SavedLeftOffset=\@LeftOffset \fi
       \setbox\@LabelPart=\hbox{}\relax
       \setbox\@LowerPart=\hbox{}\relax
\global\@LeftOffset=0pt
       \setbox\@UpperPart=\vbox{\tabskip=0pt \halign{\relax
               \global\@RightOffset=0pt \@ReturnLeftOffsettrue &&
               \inferTabSkip
               \global\@RightOffset=0pt \@ReturnLeftOffsetfalse \cr
               #3\cr}}\relax
\UpperLeftOffset=\@LeftOffset
       \UpperRightOffset=\@RightOffset
\LowerWidth=\wd\@LowerPart
       \LowerHeight=\ht\@LowerPart
       \LowerCenter=0.5\LowerWidth
\UpperWidth=\wd\@UpperPart \advance\UpperWidth by -\UpperLeftOffset
       \advance\UpperWidth by -\UpperRightOffset
       \UpperCenter=\UpperLeftOffset
       \advance\UpperCenter by 0.5\UpperWidth
\ifdim \UpperWidth > \LowerWidth
\UpperAdjust=0pt
       \RuleAdjust=\UpperLeftOffset
       \LowerAdjust=\UpperCenter \advance\LowerAdjust by -\LowerCenter
       \RuleWidth=\UpperWidth
       \global\@LeftOffset=\LowerAdjust
\else   \ifdim \UpperCenter > \LowerCenter
\UpperAdjust=0pt
       \RuleAdjust=\UpperCenter \advance\RuleAdjust by -\LowerCenter
       \LowerAdjust=\RuleAdjust
       \RuleWidth=\LowerWidth
       \global\@LeftOffset=\LowerAdjust
\else   \UpperAdjust=\LowerCenter \advance\UpperAdjust by -\UpperCenter
       \RuleAdjust=0pt
       \LowerAdjust=0pt
       \RuleWidth=\LowerWidth
       \global\@LeftOffset=0pt
\fi\fi
\if@inferRule
\setbox\ResultBox=\vbox{
               \moveright \UpperAdjust \box\@UpperPart
               \nointerlineskip \kern\inferLineSkip
               \if@DoubleRule
               \moveright \RuleAdjust \vbox{\hrule width\RuleWidth
                       \kern 1pt\hrule width\RuleWidth}\relax
               \else
               \moveright \RuleAdjust \vbox{\hrule width\RuleWidth}\relax
               \fi
               \nointerlineskip \kern\inferLineSkip
               \moveright \LowerAdjust \box\@LowerPart }\relax
\@ifEmpty{#1}{}{\relax
\HLabelAdjust=\wd\ResultBox     \advance\HLabelAdjust by -\RuleAdjust
       \advance\HLabelAdjust by -\RuleWidth
       \WidthAdjust=\HLabelAdjust
       \advance\WidthAdjust by -\inferLabelSkip
       \advance\WidthAdjust by -\wd\@LabelPart
       \ifdim \WidthAdjust < 0pt \WidthAdjust=0pt \fi
\VLabelAdjust=\dp\@LabelPart
       \advance\VLabelAdjust by -\ht\@LabelPart
       \VLabelAdjust=0.5\VLabelAdjust  \advance\VLabelAdjust by \LowerHeight
       \advance\VLabelAdjust by \inferLineSkip
\setbox\ResultBox=\hbox{\box\ResultBox
               \kern -\HLabelAdjust \kern\inferLabelSkip
               \raise\VLabelAdjust \box\@LabelPart \kern\WidthAdjust}\relax
}\relax \else \setbox\ResultBox=\vbox{
               \moveright \UpperAdjust \box\@UpperPart
               \nointerlineskip \kern\inferLineSkip
               \moveright \LowerAdjust \hbox{\unhbox\@LowerPart
                       \@ifEmpty{#1}{}{\relax
                       \kern\inferLabelSkip \unhbox\@LabelPart}}}\relax
       \fi
\global\@RightOffset=\wd\ResultBox
       \global\advance\@RightOffset by -\@LeftOffset
       \global\advance\@RightOffset by -\LowerWidth
       \if@ReturnLeftOffset \else \global\@LeftOffset=\@SavedLeftOffset \fi
\box\ResultBox
       \@RestoreMath
}
\makeatother

\def\makenewenum#1#2{\newcounter{cnt#1}
\newenvironment{#1}{\begin{list}{\makebox[0pt][r]{#2}}{\setlength{\itemsep}{0pt}\setlength{\parsep}{.2em}\setlength{\leftmargin}{2.5em}\setlength{\labelwidth}{.2em}\usecounter{cnt#1}}}{\end{list}}}
\makenewenum{enu}{\arabic{cntenu}.}
\makenewenum{enum}{\em(\arabic{cntenum})}
\makenewenum{itenum}{\em(\arabic{cntenum})}
\makenewenum{en}{(\roman{cnten})}
\makenewenum{renum}{\em(\roman{cntrenum})}
\makenewenum{enhyphen}{(\roman{cntenhyphen}')}

\newcommand{\lvseq}[1]{\begin{array}{l}
    #1
  \end{array}
}

\newcommand{\varimp}{\supset} 

\newcommand{\MSQR}{\textsf{MSQR}} 
\newcommand{\MSpQR}{\textsf{MSpQR}} 
\newcommand{\QNJ}{\textsf{QNJ}} 
\newcommand{\uni}{\bigcirc}
\newcommand{\measB}{\blacksquare}
\newcommand{\measD}{\blacklozenge}
\newcommand{\uniB}{\square}
\newcommand{\uniD}{\lozenge}
\newcommand{\Un}{\textsf{U}}
\newcommand{\Me}{\textsf{M}}
\newcommand{\me}[1]{\stackrel{#1}{\to}}
\newcommand{\un}[1]{\stackrel{#1}{\to}}

\newtheorem{theorem}{Theorem}
\newtheorem{proposition}{Proposition}
\newtheorem{lemma}{Lemma}
\newtheorem{definition}{Definition}
\newtheorem{fact}{Fact}
\newtheorem{remark}{Remark}
\newtheorem{example}{Example}
\newtheorem{corollary}{Corollary}

\mathcode`\:="003A
\newcommand{\RAA}{\mathit{RAA}}
\newcommand{\func}{\mathit{fun}}
\newcommand{\Unr}{\Un \mathit{refl}}
\newcommand{\Uns}{\Un \mathit{symm}}
\newcommand{\Unt}{\Un \mathit{trans}}
\newcommand{\inv}{\mathit{inv}}
\newcommand{\measBp}{\blacksquare_\mathsf{p}}
\newcommand{\measDp}{\blacklozenge_\mathsf{p}}
\newcommand{\Met}{\Me t}
\renewcommand{\Un}{\mathsf{U}}
\renewcommand{\Me}{\mathsf{M}}
\renewcommand{\me}[1]{\stackrel{\mathsf{#1}}{\to}}
\newcommand{\PMe}{\mathsf{P}}
\newcommand{\pt}{\mathsf{p}t}
\newcommand{\ps}{\mathsf{1}s}
\newcommand{\MeI}{\Me I}
\newcommand{\MeE}{\Me E}
\newcommand{\class}{\mathit{class}}
\newcommand{\Meser}{\Me\mathit{ser}}
\newcommand{\Mesrefl}{\Me\mathit{srefl}}
\newcommand{\Metrans}{\Me\mathit{trans}}
\newcommand{\PMetrans}{\PMe\mathit{trans}}
\newcommand{\Mesubone}{\Me\mathit{sub1}}
\newcommand{\Mesubtwo}{\Me\mathit{sub2}}
\newcommand{\PMesubone}{\PMe\mathit{sub1}}
\newcommand{\PMesubtwo}{\PMe\mathit{sub2}}

\newcommand{\LF}{\mathit{LF}}
\newcommand{\RF}{\mathit{RF}}
\newcommand{\RFclosed}{\overline{\mathit{RF}}}

\newcommand{\I}{\mathscr{I}}

\makeatletter
\DeclareSymbolFont{symbolsC}{U}{pxsyc}{m}{n}
\SetSymbolFont{symbolsC}{bold}{U}{pxsyc}{bx}{n}
\DeclareFontSubstitution{U}{pxsyc}{m}{n}
\def\re@DeclareMathSymbol#1#2#3#4{\let#1=\undefined
    \DeclareMathSymbol{#1}{#2}{#3}{#4}}
    \re@DeclareMathSymbol{\Diamonddot}{\mathord}{symbolsC}{144}
\makeatother


\title{A Qualitative Modal Representation of Quantum Register Transformations \\
(Extended Version)}
\author{Andrea Masini \qquad Luca Vigan\`o \qquad Margherita Zorzi  \\
Department of Computer Science \\
University of Verona, Italy \\
\{andrea.masini  luca.vigano  margherita.zorzi\}@univr.it}

\begin{document}

\maketitle
\thispagestyle{empty}

\begin{abstract}
  We introduce two modal natural deduction systems that are suitable to
  represent and reason about transformations of quantum registers in an abstract, qualitative, way.
  Quantum registers represent quantum systems, and can be viewed as the 
  structure of quantum data for quantum operations. 
  Our systems provide a modal framework for reasoning about operations on quantum registers (unitary transformations and measurements)
  in terms of possible worlds (as abstractions of quantum registers) and accessibility relations between these worlds. We give a Kripke--style
  semantics that formally describes quantum register transformations, and prove the 
  soundness and completeness of our systems with respect to this semantics.
\end{abstract}


\Section{Introduction}
Quantum computing 
defines an alternative computational paradigm, based on a quantum 
model~\cite{BasDa05short} rather than a classical one. The basic units of the quantum model are the 
\textit{quantum bits}, or \textit{qubits} for short (mathematically, normalized vectors of the Hilbert Space 
).  Qubits represent informational units and can assume both classical values 0 and 1, and all their superpositional values.

A \emph{quantum register} is a generalization of the qubit: a generic quantum register is the representation of a quantum state of  qubits (mathematically, it is a normalized vector of the
Hilbert space ). In this paper, we are not interested in the structure of  quantum registers, but rather in the way quantum registers are transformed. Hence, we will abstract away from the internals of quantum registers and represent them in a generic way in order to 
describe how operations transform a register into another one.

It is possible to modify a quantum register in two ways: by applying a unitary transformation 
or by measuring.  Unitary transformations (corresponding to the so-called unitary operators of 
the Hilbert space) model the internal evolution of a quantum system, whereas measurements 
correspond to the results of the interaction between a quantum system and an observer.  
The outcome of an observation can be either the reduction to a quantum
state or the reduction to a classical (non quantum) state. 
In particular, in this paper, we say that a quantum register  is \emph{classical} iff  is idempotent with respect to measurement, i.e.~each measurement of  has  as outcome. 
We call a measurement \emph{total} when the outcome of the measurement 
is a classical register.

We propose to model measurement and unitary transformations by means of suitable modal 
operators. More specifically, the main contribution of this paper is the formalization of 
a modal natural deduction system~\cite{Prawitz65,TroelstraSchwichtenberg96} 
in order to represent (in an abstract, qualitative, way) the fundamental operations on quantum registers: unitary transformations and total measurements. We call this system . We also formalize a variant of this system, called , to represent the case of generic (not necessarily total) measurements. 

It is important to observe that our logical systems are not a quantum
logic. Since 1936~\cite{vN36}, various logics have been investigated
as a means to formalize reasoning about propositions taking into account
the principles of quantum theory,
e.g.~\cite{DallaCh77,DallaChiara86}. In general, it is
possible to view quantum logic as a logical axiomatization of quantum
theory, which provides an adequate foundation for a theory of
reversible quantum processes,
e.g.~\cite{AbDu06,BaltSmets04,BalSmets06,Mittel79}.

Our work moves from quite a different point of view: we do not aim to propose a general logical formalization of quantum theory, rather we describe how it is possible to use modal logic to
reason in a simple way about quantum register transformations.  
Informally, in our proposal, a modal world represents (an abstraction of) a quantum register.
The discrete temporal evolution of a quantum register is controlled and determined by a 
sequence of unitary transformations and measurements that can change the description of a
quantum state into other descriptions. So, the evolution of a quantum register can be viewed as 
a graph, where the nodes are the (abstract) quantum registers and the arrows represent quantum transformations. The arrows give us the so-called accessibility relations of Kripke models and two
nodes linked by an arrow represent two related quantum states: the target node is obtained from
the source node by means of the operation specified in the decoration of the arrow.

Modal logic, as a logic of possible worlds, is thus a natural way to
represent this description of a quantum system: the worlds model the
quantum registers and the relations of accessibility between worlds
model the dinamical behavior of the system, as a consequence of the
application of measurements and unitary transformations. To emphasize
this semantic view of modal logic, we give our deduction system in the
style of \emph{labelled
  deduction}~\cite{Gabbay96,Simpson93,Vigano00a}, a framework for
giving uniform presentations of different non-classical logics.  The
intuition behind labelled deduction is that the labelling (sometimes
also called prefixing, annotating or subscripting) allows one to
explicitly encode in the syntax additional information, of a semantic
or proof-theoretical nature, that is otherwise implicit in the logic
one wants to capture. Most notably, in the case of modal logic, this
additional information comes from the underlying Kripke semantics: the
labelled formula  intuitively means that  holds at the world
denoted by the label  within the underlying Kripke structure
(i.e.~model), and labels also allow one to specify at the syntactic
level how the different worlds are related in the Kripke structures
(e.g.~the formula  specifies that the world denoted by  is
accessible from that denoted by ).

We proceed as follows. In Section~\ref{sec:syntax}, we define the labelled modal natural deduction system , which contains two modal operators suitable to represent and reason about unitary transformations and total measurements of quantum registers. In Section~\ref{sec:semantics}, we 
give a  possible worlds semantics that formally describes these quantum register transformations, 
and prove the soundness and completeness of  with respect to this semantics. In Section~\ref{sec:partial}, we formalize , a variant of  that provides a modal system 
representing all the possible (thus not necessarily total) measurements.  We
conclude in Section~\ref{sec:conclusions} with a brief summary and a discussion of future work.
Full proofs of the technical results are given in the appendix.



\section{The deduction system }
\label{sec:syntax}

Our labelled modal natural deduction system , which formally represents unitary transformations and total measurements of quantum registers,
comprises of rules that derive formulas of two kinds: modal formulas and relational formulas. We 
thus define a modal language and a relational language. 

The alphabet of the \emph{relational language} consists of:
\begin{itemize}
\item the binary symbols  and ,
\item a denumerable set  of \textit{labels}.
\end{itemize}
Metavariables , possibly annotated with subscripts and superscripts, range over the set of 
labels.
For brevity, we will sometimes speak of a ``world"  meaning that the label  stands for a world
, where  is an interpretation function mapping labels into 
worlds as formalized in Definition~\ref{def:semantics} below.

The set of \textit{relational formulas} (\textit{r--formulas}) is given
by expressions of the form  and .

The alphabet of the \textit{modal language}  consists of:
\begin{itemize}
\item a denumerable set   of \textit{propositional symbols},
\item the standard \textit{propositional connectives}  and ,
\item the unary \textit{modal operators}  and .
\end{itemize}
The set of \emph{modal formulas} (\emph{m--formulas}) is the least
set that contains  and the propositional symbols, and is closed
under the propositional connectives and the modal operators.  
Metavariables , , , possibly indexed, range over modal formulas.
Other connectives can be defined in the usual manner, e.g.~, 
, 
,
, , etc.

Let us give, in a rather informal way, the intuitive meaning of the modal operators of our language:
\begin{itemize}
 \item  means:  is true after the application of any
   unitary transformation.
\item  means:  is true in each quantum register obtained by a total
  measurement.
\end{itemize}

A \emph{labelled formula} (\emph{l--formula}) 
is an expression , where  is a label and  is an m--formula. 
A \emph{formula} is either an r--formula or an l--formula. The metavariable , possibly 
indexed, ranges over formulas. We write  to denote that the label  occurs in the 
formula , so that  denotes the substitution of the label  for all occurences of 
 in .

\begin{figure*}[t]
  
  In ,  is fresh: it is different from  and does not occur in any assumption on which  depends other than . \\
  In ,  is fresh: it is different from  and does not occur in  nor in any assumption on which  depends other than .
  \caption{The rules of }
  \label{fig:rules}
\end{figure*}

Figure~\ref{fig:rules} shows the rules of , where the notion of \emph{discharged/open assumption} is standard~\cite{Prawitz65,TroelstraSchwichtenberg96}, e.g.~the formula  is 
discharged in the rule :
\begin{description}
\item[Propositional rules:] The rules ,  and  are just the labelled version of the standard (\cite{Prawitz65,TroelstraSchwichtenberg96}) natural deduction rules for implication introduction and elimination and for 
  \emph{reductio ad absurdum}, where we do not enforce Prawitz's side
  condition that .\footnote{See~\cite{Vigano00a} for a
    detailed discussion on the rule , which in particular
    explains how, in order to maintain the duality of modal operators
    like  and , the rule must allow one to derive
     from a contradiction  at a possibly different world
    , and thereby discharge the assumption .} The
  ``mixed'' rule  allows us to derive a generic formula
   whenever we have obtained a contradiction  at a world
  .
\item[Modal rules:] We give the rules for a generic modal operator , with a corresponding generic accessibility relation , since all the modal operators share the structure of these basic introduction/elimination rules; this holds because, for instance, we express  as the metalevel 
implication  for an arbitrary  accessible from . In particular:
\begin{itemize}
\item if   is  then  is ,
\item if  is  then  is .
\end{itemize}
\item[Other rules:] \
\begin{itemize} 
\item In order to axiomatize , 
we add rules , , and , formalizing that  is an equivalence relation.
\item 
In order to axiomatize , 
we add rules formalizing the following properties:
\begin{itemize}
\item If  then there is specific unitary transformation
  (depending on  and ) that generates  from : rule .
\item The total measurement process is serial: rule  says that if from the assumption  we can derive  for a \emph{fresh}  (i.e.~ is different from  and does not occur in  nor in any assumption on which  depends other than ), then we can discharge the assumption (since there always is some  such that ) and conclude .
\item The total measurement process is shift-reflexive: rule .
\item Invariance with respect to classical worlds: rules  and  say that, if 
 and , then  must be equal to  and so we can substitute the one for the other in any formula .
\end{itemize}
\end{itemize}
\end{description}

\begin{definition}[Derivations and proofs]
  A \emph{derivation} of a formula  from a set of formulas
   in  is a tree formed using the rules in ,
  ending with  and depending only on a finite subset of
  ; we then write .  A derivation of
   in  depending on the empty set, , is
  a \emph{proof} of  in  and we then say that 
  is a theorem of .
\end{definition}

For instance, the following labelled formula schemata are all provable
in  (where, in parentheses, we give the intuitive meaning of
each formula in terms of quantum register transformations):
\begin{enumerate}
\item \\ (the identity transformation is unitary).
\item  \\ (each unitary transformation is invertible).
\item \\ (unitary transformations are composable).
\item  \\
(it is always possible to perform a total measurement of a quantum register).
\item\label{example-one}  \\ 
(it is always possible to perform a total measurement with a complete reduction of a quantum register to a classical one).
\item\label{example-two}
  \\ 
 (total measurements are composable).
\end{enumerate}

As concrete examples, Figure~\ref{fig:examples} contains the proofs of the formulas~\ref{example-one} and~\ref{example-two}, where, for simplicity, here and in the following (cf.~Figure~\ref{fig:example-MSpQR}), we employ the rules for 
equivalence () and for negation ( and ), 
which are derived from the propositional rules as is standard. For instance, 

We can similarly derive rules about r--formulas. For instance, we can derive a rule for the transitivity 
of  as shown at the top of the proof of the formula~\ref{example-two} in Figure~\ref{fig:examples}:

abbreviates


\begin{figure*}[t]
  2em]
     \infer[\varimp I^1]{x: \measB A \varimp \measB\measB A}
      {
      \infer[\measB I^2]{x: \measB\measB A}
       {
       \infer[\measB I^3]{y: \measB A}
        {
        \infer[\measB E]{z: A}
         {
         [x: \measB A]^1
         &
         \infer[\Mesubone]{x \Me z}
          {
          [y \Me z]^3
          &
          \infer[\Mesrefl]{z \Me z}{[y \Me z]^3}
          &
          [x \Me y]^2
          }
         }
        }
       }
      }
     \end{array}
     
\begin{array}{llll}
\xymatrix{
v \ar[r]^M & w &
}
&
\xymatrix{
v \ar[r]^M & w\ar@(r,u)[]_M & &
}
&
\xymatrix{
v\ar@(r,u)[]_M \ar@{-//}[r]_M \ar@{.>}[dr]|-{?}_U & \\
& & 
}
&
\xymatrix{
v \ar[r]^M & w\ar@(r,u)[]_M \ar@{-//}[r]_M \ar@{.>}[dr]|-{?}_U & \\
& & 
} \\
\text{(ii)} & \text{(iii)} & \text{(iv)} & \text{(iii) and (iv)}
\end{array}

\begin{array}{lll}
\Gamma \vDash \alpha & \mathrm{iff} & 
\forall \mathscr{M}, \I.\ \mathscr{M} \vDash \I(\Gamma) \Longrightarrow \mathscr{M} 
\vDash \I(\alpha) \\
& \mathrm{iff} & 
\forall \mathscr{M}, \I.\ \mathscr{M}, \I \vDash \Gamma \Longrightarrow \mathscr{M}, \I\vDash \alpha
\end{array}

    \renewcommand{\arraystretch}{4}
  \begin{array}{l}
    \varimp I, \ \varimp E,\ \mathit{RAA},\ \bot E,\ \bigstar I^*,\ \bigstar E,\ \Unr,\ \Uns,\ \Unt,\\
    \infer[\PMe\Un I]{x \Un y}{x \PMe y}
    \qquad
    \infer[\PMetrans]{x \PMe z}{x \PMe y & y \PMe z}
    \qquad
    \infer[\class^*]{\alpha}{\infer*{\alpha}{[x \PMe y]\, [y \PMe y]}}
    \qquad
    \infer[\PMesubone]{\alpha(y/x)}{\alpha(x) & x \PMe x & x \PMe y}
    \qquad
    \infer[\PMesubtwo]{\alpha(x/y)}{\alpha(y) & x \PMe x & x \PMe y}
    \end{array}
  
     \begin{array}{c}
    \infer[\class^1]{x: \neg \boxdot \neg (A \varimp \boxdot A)}
     {
     \infer[\neg I^2]{x: \neg \boxdot \neg (A \varimp \boxdot A)}
      {
      \infer[\neg E]{y:\bot}
       {
       \infer[\boxdot E]{y: \neg (A \varimp \boxdot A)}
        {
        [x: \boxdot \neg (A \varimp \boxdot A)]^2 & [x \PMe y]^1
        }
       &
       \infer[\varimp I^3]{y: A \varimp \boxdot A}
        {
        \infer[\boxdot I^4]{y: \boxdot A}
         {
         \infer[\PMesubone]{z:A}
          {
          [y:A]^3 & [y \PMe y]^1& [y \PMe z]^4 
          }
         }
        }
       }
      }
     } 
    \end{array}
   
  \vcenter{
    \infer[\RAA]{x:A}
     {
     \infer*{y: \bot}{[x: \neg A]}
     }
    }

  \vcenter{
    \infer[\bot E]{\alpha}{x: \bot}
    }

  \vcenter{
    \infer[\bigstar I]{x: \bigstar A}
     {
     \infer*{y:A}{[x Ry]}
     }
    }

\infer[\bigstar E]{y:A}
 {
 x: \bigstar A
 & 
 x R y
 }

\infer[\Meser]{\alpha}{\infer*{\alpha}{[x \Me y]}}

\infer[\Mesubone]{\alpha(y/x)}
 {
\alpha(x)
 & 
x \Me x
 & 
x\Me y
 }

\overline{(\LF,\RF)} \equiv \{x R y \mid (\LF,\RF)  \vdash x R y\} 

\mathscr{S}^c = \langle \mathscr{M}^c, \I^c \rangle = 
\langle W^c, U^c, M^c, V^c, \I^c \rangle

  (\LF_i \cup \{a: \neg \bigstar A, c: \neg A\},\RF_i \cup \{a R c\}) \vdash a_j:\bot
  
  (\LF_i \cup \{a: \neg \bigstar A\},\RF_i \cup \{a R c\}) \vdash c: A\,,
  
  (\LF_i \cup \{a: \neg \bigstar A\},\RF_i) \vdash a: \bigstar A\,. 
  
  (\LF_i \cup \{a: \neg \bigstar A\},\RF_i) \vdash a: \neg \bigstar A\,, 
  
  (\LF_i \cup \{a: \neg \bigstar A\},\RF_i) \vdash a:\bot\,,
  
  (\LF^*,\RF^*) = \overline{(\bigcup_{i \geq 0} \LF_i, \bigcup_{i \geq 0} \RF_i)}
  
  (\bigcup_{i \geq 0} \LF_i, \bigcup_{i \geq 0} \RF_i)
  
  \overline{(\bigcup_{i \geq 0} \LF_i, \bigcup_{i \geq 0} \RF_i)}\,.
  
\mathscr{S}^c = \langle \mathscr{M}^c, \I^c \rangle = 
\langle W^c, U^c, M^c, V^c, \I^c \rangle

  (a_i,a_j) \in R^c  \text{ iff } \{A \mid \Box A \in a_i\} \subseteq a_j\,, 

is not applicable in our setting, since  does \emph{not} 
imply .  We would therefore be unable to prove completeness for r--formulas, 
since there would be cases, e.g.~when , where  but 
 and thus .  Hence, we instead define 
 iff ; note that therefore 
implies .  As a further comparison with the standard 
definition, note that in the canonical model the label  can be identified with the set of
formulas . Moreover, we immediately have:
\begin{fact}\label{fact-R}
   iff .
\end{fact}

The deductive closure of  for r--formulas ensures not only completeness for
r--formula, as shown in Theorem~\ref{theorem:completeness-MSQR} below, but also that
the conditions on  are satisfied, so that  is really a structure
for . More concretely:
\begin{itemize}
\item  is an equivalence relation by construction and rules , , and . 
For instance, for transitivity, consider an arbitrary context  from which we build 
. Assume  and .  Then 
 and . Since  is
deductively closed, by~\emph{1} in Lemma~\ref{lemma-cs} and rule , we have 
. Thus,  and  is indeed transitive.

\item  holds by construction and rule .

\item  holds by construction and rule .
For the sake of contradiction, consider an arbitrary  and a variable  
that do not satisfy the property. Define . 
Then it cannot be the case that , for otherwise  would be derivable by an application of the rule . Thus, 
. But then  must be in the chain of contexts built in 
Lemma~\ref{lemma-cs}. So, by the maximality of , we have that 
, contradicting our assumption. Hence, for some , the r--formula 
 is in , which is what we had to show.

\item  holds by construction and rule .

\item   holds by construction and
rules  and  since  is a classical world.
Consider an arbitrary context  from which we build  and assume
 and . Then  and 
. Thus, for each , we also have 
; otherwise, since  is deductively closed, we would have 
 and also  by \emph{1} in 
Lemma~\ref{lemma-cs} and rule , and thus a contradiction. Similarly, if 
 then  by rule . Hence, for each 
m--formula , we have that  iff , which
means that  and  are equal with respect to m--formulas. 

Under the same assumptions, we can similarly show that  and  are equal with 
respect to r--formulas, i.e.~that whenever  contains an r--formula that includes  then it also contains the same r--formula with  substituted for , and vice versa. 
To this end, we must consider 8 different cases corresponding to 8 different r--formulas. 
\begin{itemize}
\item If  for some , then from the assumption that 
 we have , 
by \emph{1} in Lemma~\ref{lemma-cs} and rule . Therefore, 
 by rule .  
\item We can reason similarly for  and also apply rules  
and  to conclude that then also . 
\item If  for some , then from the assumption that 
 we have , 
by \emph{1} in Lemma~\ref{lemma-cs} and rule , and thus 
, by rule . Therefore, 
 by rule .  
\item We can reason similarly for  and also apply rules ,
, and  to conclude that then also . 
\item If  for some , then from the assumption that 
 we have ,
by \emph{1} in Lemma~\ref{lemma-cs} and the derived rule .
\item We can reason similarly for  and also apply rule 
to conclude that then also . 
\item If  for some , then from the assumptions that 
 and  we have 
, by \emph{1} in Lemma~\ref{lemma-cs} and rule .  
\item We can reason similarly for  and apply rule 
to conclude that then also . 
\end{itemize}
Hence,  and  are equal also with respect to r--formulas, and thus
 whenever  and , which is what we had to show.
\end{itemize}

By Lemma~\ref{lemma-cs} and Fact~\ref{fact-R}, it follows that:
\begin{lemma}\label{lemma-truth}
   iff .
\end{lemma}
\begin{proof}
  We proceed by induction on the grade of , and we treat only the step case where 
   is ; the other cases follow analogously. For the left-to-right direction, 
  assume .  Then, by Lemma~\ref{lemma-cs}, 
   implies , for all .
  Fact~\ref{fact-R} and the induction hypothesis yield that 
   for all 
   such that , 
  i.e.~ by Definition~\ref{def:truth}.
  For the converse, assume .  Then, by
  Lemma~\ref{lemma-cs},  and , for 
  some . Fact~\ref{fact-R} and the induction hypothesis yield
   and 
  , 
  i.e.~ by Definition~\ref{def:truth}.
\end{proof}

We can now finally show:
\begin{theorem}[Completeness of ]\label{theorem:completeness-MSQR}
 implies .
\end{theorem}
\begin{proof}
  If , then , and thus 
   by Fact~\ref{fact-R}.  
  
  If , then  is consistent; otherwise
  there exists a  such that , and then 
  . Therefore, by Lemma~\ref{lemma-lindenbaum}, 
   is included in a maximally consistent
  context .  Then, by
  Lemma~\ref{lemma-truth}, , 
  i.e.~, and thus 
  .
\end{proof}

We can reason similarly to show the soundness and completeness of  with respect to the corresponding semantics: 
Theorem~\ref{theorem:soundness-completeness-MSpQR} follows from Theorems~\ref{theorem:soundness-MSpQR}
and~\ref{theorem:completeness-MSpQR} below.

\begin{theorem}[Soundness of ]\label{theorem:soundness-MSpQR}
 implies .
\end{theorem}

\begin{proof}
We let  be an arbitrary model and prove that if  then 
 implies   for any .
The proof proceeds by induction on the structure of the derivation of  from .  The base case, where , is trivial.  There is one step case for each 
rule of , where the soundness of the rules , , , , 
, ,  follows exactly like in the proof of Theorem~\ref{theorem:soundness-MSQR}.
 
The soundness of the rules  and  follows exactly like in the proof of 
Theorem~\ref{theorem:soundness-MSQR}, with the only difference that when  is 
 then  is .

The rule  is sound by property \emph{(i)} in the definition of the semantics for .

The rule  is sound by property \emph{(ii)} in the definition of the semantics for .

The soundness of the rule   follows like for the soundness
of the rule  in the proof of 
Theorem~\ref{theorem:soundness-MSQR}, this time exploiting property  \emph{(iii)} in the 
definition of the semantics for .

The soundness of the rules  and  follows like for the soundness
of the rules  and  in the proof of 
Theorem~\ref{theorem:soundness-MSQR}, this time exploiting property  \emph{(iv)} in the 
definition of the semantics for .
\end{proof}

To prove completeness (Theorem~\ref{theorem:completeness-MSQR}), we proceed like for the case
of , mutatis mutandis in the construction of the canonical model. In particular, given a 
maximal consistent context , we define the canonical structure
 by setting 
\begin{itemize}
\item  iff .
\end{itemize}
To show that the conditions on  are satisfied, so that  is really a structure
for , we reuse the results proved for  and in addition show the following:
\begin{itemize}
\item  holds by construction and rule .

\item  holds by construction 
and rule .

\item  holds by construction and rule 
. For the sake of contradiction, consider an arbitrary  and a variable  
that do not satisfy the property. Define . 
Then it cannot be the case that , for otherwise  would be derivable by an application of the rule . Thus, 
. But then  must be in the chain of contexts built in 
Lemma~\ref{lemma-cs}. So, by the maximality of , we have that 
, contradicting our assumption. Hence, for some , the r--formulas 
 and  are both in , which is what we had to show.

\item   holds by construction and
rules  and  since  is a classical world.
Consider an arbitrary context  from which we build  and assume
 and . Then  and 
. Thus, for each , we also have 
; otherwise, since  is deductively closed, we would have 
 and also  by \emph{1} in 
Lemma~\ref{lemma-cs} and rule , and thus a contradiction. Similarly, if 
 then  by rule . Hence, for each 
m--formula , we have that  iff , which
means that  and  are equal with respect to m--formulas. 

Under the same assumptions, we can similarly show that  and  are equal with 
respect to r--formulas, i.e.~that whenever  contains an r--formula that includes  then it also contains the same r--formula with  substituted for , and vice versa. 
To this end, we must consider 8 different cases corresponding to 8 different r--formulas. 
\begin{itemize}
\item If  for some , then from the assumption that 
 we have , 
by \emph{1} in Lemma~\ref{lemma-cs} and rule . Therefore, 
 by rule .  
\item We can reason similarly for  and also apply rules  
and  to conclude that then also . 
\item If  for some , then from the assumption that 
 we have , 
by \emph{1} in Lemma~\ref{lemma-cs} and rule , and thus 
, by rule . Therefore, 
 by rule .  
\item We can reason similarly for  and also apply rules ,
, and  to conclude that then also . 
\item If  for some , then from the assumption that 
 we have ,
by \emph{1} in Lemma~\ref{lemma-cs} and the rule .
\item We can reason similarly for  and also apply rule 
to conclude that then also . 
\item If  for some , then from the assumptions that 
 and  we have 
, by \emph{1} in Lemma~\ref{lemma-cs} and rule .  
\item We can reason similarly for  and apply rule 
to conclude that then also . 
\end{itemize}
Hence,  and  are equal also with respect to r--formulas, and thus
 whenever  and , which is what we had to show.
\end{itemize}

Proceeding like for , we then have:
\begin{theorem}[Completeness of ]\label{theorem:completeness-MSpQR}
 implies . \hfill 
\end{theorem}

\end{document}
