It is difficult to gauge the appropriateness of a
concurrency model without comparing it against its contemporaries.
With that in mind, this section presents a comparison of
{\qsname} with four well-established languages.

\subsection{A Variety of Languages}
The comparison languages should:
be modern, well-known, and represent a variety of different
underlying design choices.
For the purposes of the evaluation,
this means that we should select from different programming paradigms,
different approaches to shared memory,
different concurrency safety guarantees,
and different threading implementations.
We have chosen a selection of languages:
C++/TBB (Threading Building Blocks)~\cite{5672517},
Erlang~\cite{Armstrong:1996:CPE:229883},
Go~\cite{golang}, and
Haskell~\cite{PeytonJones:1996:CH:237721.237794}. 
This selection attempts to combine a reasonable number of the facets
outlined above to give a complete picture.
To make the diversity clear, we present this in
Table~\ref{tab:language_characteristics}.
\begin{table*}[htb]
\caption{Language characteristics}
\label{tab:language_characteristics}
\footnotesize
\centering
\begin{tabular}{cccccc}
  \hline
  Language & Races    & Threads & Paradigm   & Memory     & Approach \\ \hline
  C++/TBB  & possible & OS      & Imperative & Shared     & Skeletons/traditional \\
  Go       & possible & light   & Imperative & Shared     & Goroutines/channels \\
  Haskell  & none     & light   & Functional & STM        & STM/Repa \\
  Erlang   & none     & light   & Functional & Non-shared & Actors \\
  {\qsname}& none     & light   & O-O        & Non-shared & Active Objects \\
  \hline
\end{tabular}
\end{table*}

The Memory column refers to how memory is shared between threads.
Erlang has no sharing between different processes:
when data is sent between processes
it is copied in its entirety.
In {\qsname} the programmer is only able ti access shared memory
through a handler.
In Haskell it is perfectly possible to construct data races if
one uses mutable references.
However, to include one more race-free model to this comparison,
we restrict ourselves to using Haskell's STM implementation,
the \texttt{par} construct which executes pure computations in parallel,
as well as the Repa library which is specialized to work on parallel arrays. 



\subsection{Parallel Benchmarks}

The parallel benchmarks are meant to measure
how well a language can handle taking a particular
program and scaling it given more computational
resources (cores).
Note that it is common in the Erlang and {\qsname} implementations of the
Cowichan problems that a
significant amount of time is spent
sharing results among the threads.
Therefore, to more clearly see the effect of different optimizations,
and to separate computational effects from communication effects,
we distinguish the time spent computing versus the time spent
communicating the results.

\subsubsection{Execution Time}
We can see the graph of performance given 32 cores in
\fig~\ref{fig:parallel_benchmarks}.
\begin{figure*}[htb]
  \centering
  \includegraphics[width=\linewidth]{parallel.pdf}
  \caption{Execution times of parallel tasks on different languages, executed on 32 cores}
  \label{fig:parallel_benchmarks}
\end{figure*}
As with {\qsname}, to give a clearer picture of
the performance characteristics of Erlang,
we also distinguish the computation
time from the communication time.
We can see that {\qsname} and Erlang both spend a majority
of their time in communication,
with the exception of the \bname{chain} problem, 
which has much less communication between the workers.
It is useful to consider both the total and the computation
time: in non-benchmark style problems it is more likely
that the workloads fall somewhere in the middle.
For example, the \bname{chain} problem, which is 
the composition of the other smaller benchmarks,
does not suffer from nearly the same communication
burden as they do.

Erlang has unfavorable performance results compared to
the other languages.
Due to Erlang's data representation
(forces to use linked lists to represent matrices)
and its execution model
(cannot use the HiPE-optimized builds~\cite{erlanghipe} with the multithreaded runtime),
it generally falls far behind the other approaches,
even Haskell and {\qsname}.

\begin{sloppypar}
Besides Erlang,
the other languages are more closely grouped.
The geometric means for total time are,
in increasing order:
C++/TBB (0.32s), Go (0.57s), Haskell (0.89s), {\qsname} (1.35s), 
and Erlang (18.07s).
For computation-only time, the order is:
{\qsname} (0.29s), C++/TBB (0.32s), Go (0.57s), Haskell (0.89s) and
Erlang (4.32s).
\end{sloppypar}

Note that this puts {\qsname} first because many of the cache effects
are removed due to the predistribution of the data before
the timing starts.
This is only included as a sanity test,
to show that the lower-bound for the {\qsname} implementation
is competitive with the other approaches.

\subsubsection{Scalability} The other aspect that we investigated was the speedup of
the benchmarks across 32 cores.
In \fig~\ref{fig:parallel_speedup}
\begin{figure*}[htb]
  \centering
  \includegraphics[width=\linewidth]{parallel_speedup.pdf}
  \caption{Speedup over single-core performance, up to 32 cores}
  \label{fig:parallel_speedup}
\end{figure*}
we can see the performance of
the various languages on the different problems.
We can see that on \bname{chain}, 
most languages manage to achieve a speedup of at least 5x.
Go is the exception to this, and performance decreases past 8 cores.
Erlang also sees a performance degradation,
though only past 16 cores.
This was also an effect that was seen in 
our language comparison
study~\cite{nanz-et-al:2013:benchmarking-multicore-languages}
from which the implementation was taken;
the implementation was also reviewed by a key Go developer in the study.

Also of note is the performance of Haskell on the \bname{randmat} benchmark.
This is one of the few benchmarks where Repa could not be effectively
used due to the nature of the problem,
so the basic \texttt{par}-based concurrency primitives are used.
The basic strategy has chunks of the output array constructed
in parallel, then concatenated together.
The concatenation is sequential, however, putting a limit on the
maximum speedup;
using the ThreadScope~\cite{Wheeler:2010:VMM:1673012.1673015} performance 
reporting tool, we could see that
the stop-the-world garbage collector was intervening too often.
The last unexpected result was the inability for the Erlang
version of the \bname{winnow} program to speedup past about 2-3x.
This was examined in detail but no cause could be found.
Precise timing data can be found in Table~\ref{tab:parallel}.


\newcommand{\perfcolwidth}{0.75cm}
\newcommand{\pcw}{p{\perfcolwidth}}
\newcommand{\perfcolspec}{p{1.1cm} p{1.6cm} p{0.3cm} \pcw  \pcw \pcw  \pcw \pcw  \pcw}
\begin{table*}[htb]
\caption{Parallel benchmark times (in seconds)}
\label{tab:parallel}
{

\renewcommand{\tabcolsep}{2pt}


\footnotesize
\parbox{0.4999\linewidth}{
\centering
\begin{tabular}{\perfcolspec}\hline
         &           &   & \multicolumn{6}{c}{Threads}                   \\
  Task   & Lang      & V & 1     & 2     & 4     & 8     & 16    & 32    \\ 
  \hline
 randmat & c++       & T & 0.44  & 0.23  & 0.13  & 0.08  & 0.06  & 0.08  \\ 
 randmat & erlang    & T & 30.93 & 18.01 & 10.20 & 5.77  & 4.05  & 4.14  \\ 
 randmat & erlang    & C & 20.69 & 11.26 & 5.63  & 2.99  & 1.73  & 1.50  \\ 
 randmat & go        & T & 0.78  & 0.43  & 0.24  & 0.14  & 0.09  & 0.08  \\ 
 randmat & haskell   & T & 0.68  & 0.43  & 0.36  & 0.44  & 0.62  & 1.03  \\ 
 randmat & {\qsname} & T & 0.72  & 0.43  & 0.29  & 0.22  & 0.21  & 0.23  \\ 
 randmat & {\qsname} & C & 0.59  & 0.30  & 0.15  & 0.08  & 0.05  & 0.05  \\ 
 thresh  & c++       & T & 1.00  & 0.66  & 0.34  & 0.18  & 0.12  & 0.11  \\ 
 thresh  & erlang    & T & 31.82 & 22.35 & 17.77 & 14.48 & 12.88 & 11.96 \\ 
 thresh  & erlang    & C & 19.30 & 10.74 & 5.97  & 2.77  & 1.47  & 0.89  \\ 
 thresh  & go        & T & 0.95  & 0.60  & 0.37  & 0.22  & 0.17  & 0.17  \\ 
 thresh  & haskell   & T & 1.56  & 0.96  & 0.69  & 0.55  & 0.51  & 0.50  \\ 
 thresh  & {\qsname} & T & 3.71  & 2.72  & 2.28  & 2.10  & 2.11  & 2.15  \\ 
 thresh  & {\qsname} & C & 1.87  & 1.08  & 0.54  & 0.31  & 0.16  & 0.09  \\ 
 winnow  & c++       & T & 2.04  & 1.03  & 0.53  & 0.29  & 0.18  & 0.15  \\ 
 winnow  & erlang    & T & 31.03 & 26.02 & 25.04 & 24.75 & 24.38 & 23.95 \\ 
 winnow  & erlang    & C & 4.06  & 2.58  & 1.84  & 1.46  & 1.29  & 1.24  \\ 
 winnow  & go        & T & 2.47  & 1.29  & 0.71  & 0.46  & 0.32  & 0.28  \\ 
 winnow  & haskell   & T & 5.43  & 2.77  & 1.42  & 0.80  & 0.48  & 0.52  \\ 
 winnow  & {\qsname} & T & 5.16  & 3.74  & 3.04  & 2.69  & 2.58  & 2.57  \\ 
 winnow  & {\qsname} & C & 2.83  & 1.40  & 0.72  & 0.36  & 0.19  & 0.10  \\ 
   \hline
\end{tabular}
}
\hspace{0.1ex}
\parbox{0.4999\linewidth}{
\centering
\begin{tabular}{\perfcolspec}\hline
          &          &   & \multicolumn{6}{c}{Threads}                    \\
  Task   & Lang      & V & 1      & 2     & 4     & 8     & 16    & 32    \\ 
  \hline
 outer   & c++       & T & 1.59   & 0.83  & 0.42  & 0.23  & 0.15  & 0.14  \\ 
 outer   & erlang    & T & 61.57  & 38.21 & 21.19 & 17.57 & 11.67 & 8.05  \\ 
 outer   & erlang    & C & 40.66  & 22.54 & 10.45 & 6.05  & 3.12  & 2.52  \\ 
 outer   & go        & T & 2.47   & 1.44  & 0.84  & 0.57  & 0.60  & 0.67  \\ 
 outer   & haskell   & T & 5.49   & 2.76  & 1.40  & 0.74  & 0.41  & 0.36  \\ 
 outer   & {\qsname} & T & 2.58   & 1.62  & 1.15  & 0.93  & 0.90  & 0.89  \\ 
 outer   & {\qsname} & C & 1.87   & 0.93  & 0.46  & 0.24  & 0.12  & 0.06  \\ 
 product & c++       & T & 0.44   & 0.23  & 0.13  & 0.09  & 0.08  & 0.12  \\ 
 product & erlang    & T & 15.89  & 13.94 & 12.66 & 12.08 & 11.82 & 11.33 \\ 
 product & erlang    & C & 3.35   & 1.95  & 0.90  & 0.45  & 0.24  & 0.15  \\ 
 product & go        & T & 0.76   & 0.46  & 0.29  & 0.19  & 0.15  & 0.13  \\ 
 product & haskell   & T & 0.45   & 0.25  & 0.16  & 0.11  & 0.11  & 0.15  \\ 
 product & {\qsname} & T & 1.49   & 1.33  & 1.27  & 1.24  & 1.28  & 1.34  \\ 
 product & {\qsname} & C & 0.32   & 0.16  & 0.08  & 0.04  & 0.02  & 0.01  \\ 
 chain   & c++       & T & 5.57   & 2.76  & 1.42  & 0.76  & 0.43  & 0.32  \\ 
 chain   & erlang    & T & 120.59 & 69.00 & 32.06 & 18.48 & 13.23 & 16.01 \\ 
 chain   & erlang    & C & 119.68 & 68.13 & 30.93 & 17.75 & 12.63 & 15.15 \\ 
 chain   & go        & T & 7.39   & 4.09  & 2.39  & 1.79  & 1.93  & 2.60  \\ 
 chain   & haskell   & T & 13.78  & 7.71  & 4.62  & 3.30  & 2.74  & 2.94  \\ 
 chain   & {\qsname} & T & 5.60   & 2.88  & 1.56  & 0.97  & 0.68  & 0.67  \\ 
 chain   & {\qsname} & C & 5.54   & 2.75  & 1.40  & 0.74  & 0.40  & 0.25  \\ 
   \hline
\end{tabular}
}\\
\begin{center}
 V - Timing Variant, T - Total time, C - Compute time.
\end{center}
}
\end{table*}

\subsection{Concurrent Benchmarks}
The concurrent programming tasks are compared in
\fig~\ref{fig:concurrent_benchmarks} and
\begin{figure*}[htb]
  \centering
  \includegraphics[width=\textwidth]{concurrent.pdf}
  \caption{Execution times of concurrent tasks on different languages}
  \label{fig:concurrent_benchmarks}
\end{figure*}
exact times in Table~\ref{tab:concurrent}.
\begin{table}[htb]
\caption{Concurrent benchmark times (in seconds)}
\label{tab:concurrent}
\centering
{
\footnotesize
\begin{tabular}{lrrrrr}
  \hline
      Task  & c++   & erlang & go    & haskell & {\qsname} \\ 
  \hline
 chameneos  & 0.32  & 8.67   & 2.40  & 61.97   & 4.71 \\ 
 condition  & 15.92 & 2.15   & 5.95  & 26.05   & 1.48 \\ 
 mutex      & 0.14  & 6.13   & 0.17  & 0.86    & 0.47 \\ 
 prodcons   & 0.40  & 8.78   & 0.66  & 2.99    & 1.33 \\ 
 threadring & 34.13 & 3.30   & 13.98 & 57.44   & 5.82 \\ 
   \hline
\end{tabular}
}
\end{table}

Haskell tends to perform the worst,
which is likely due to the use of STM,
which incurs an extra level of bookkeeping on every
operation.
Erlang performs better, but in general lags behind the other approaches.
C++/TBB tends to be the fastest, except in the
\bname{condition} and \bname{threadring} benchmarks,
which are both essentially single-threaded;
they are designed to test context switching overhead in various forms.
Go does quite well uniformly, never the fastest,
but never the slowest.
Lastly, {\qsname} performs mostly in line with C++/TBB and Go,
however it is the fastest in the condition benchmark.
In increasing order of geometric means:
C++/TBB (1.57s),
Go (1.82s),
{\qsname} (1.91s),
Erlang (5.01s), and
Haskell (12.20s).


\subsection{Summary}
This evaluation presents a wide variety of approaches
to concurrency and situates {\qsname} among them.
In particular,
we can see that {\qsname} is generally quite comparable
on coordination or concurrency tasks,
falling in the middle of the pack after Go and C++/TBB,
but faring better than Erlang and Haskell.
Note, however, that neither Go nor C++/TBB offers
any of the guarantees of {\qsname},
and {\qsname} offers more guarantees than Erlang.

For parallel problems, {\qsname} ranks 4th with all communication
burdens accounted for, but 1st when only computation is considered.
Again, we present both of these results only to say that
we expect real world usage to be between these extremes;
see the \bname{chain} benchmark for an indication of performance
on tasks which are not dominated by communication.
In any case, it still remains that {\qsname} outpaces Erlang in all cases,
and can provide reasonable performance comparable with the other approaches,
while still providing more guarantees in the execution model.

\begin{sloppypar}
For all problems, concurrent and parallel,
the geometric means are:
C++/TBB (0.71s),
Go (1.02s),
{\qsname} (1.61s),
Haskell (3.30s), and
Erlang (9.51s).
This places {\qsname} as the best performing of the
safe languages.
\end{sloppypar}