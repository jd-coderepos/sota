\documentclass[11pt]{article}

\usepackage{psfrag}
\usepackage{amsmath}
\usepackage{amssymb}
\usepackage{amscd}
\usepackage{picinpar}
\usepackage{times}
\usepackage{pb-diagram}
\usepackage{graphicx}
\usepackage{wrapfig}
\usepackage{xspace}
\usepackage{algorithmic, algorithm}   
\usepackage{multirow}

\newcommand\mymark{\indent \ \ }
\def\gl{\lambda}
\def\C{\mathbb{C}}
\def\R{\mathbb{R}}
\def\H{\mathbf{H}}
\def\T{\mathcal{T}}
\def\A{\mathbf{A}}
\def\B{\mathbf{B}}
\def\del{\mathrm{Del}}
\def\D{\mathcal{D}}
\let\eps\varepsilon
\def\reals{\mathbb{R}}
\def\pow{\mbox{Pow}}
\def\polylog{\operatorname{polylog}}
\def\LL{\mathbf{L}}
\def\etal{\emph{et~al.}\xspace}
\def\ie{\emph{i.e.}}
\def\eg{\emph{e.g.}}
\def\vitae{vit\ae{}}


\newtheorem{theorem}{Theorem}[section]
\newtheorem{lemma}[theorem]{Lemma}
\newtheorem{proposition}[theorem]{Proposition}
\newtheorem{corollary}[theorem]{Corollary}
\newtheorem{definition}[theorem]{Definition}

\newenvironment{proof}[1][Proof]{\begin{trivlist}
\item[\hskip \labelsep {\bfseries #1}]}{\end{trivlist}}
\newenvironment{example}[1][Example]{\begin{trivlist}
\item[\hskip \labelsep {\bfseries #1}]}{\end{trivlist}}
\newenvironment{remark}[1][Remark]{\begin{trivlist}
\item[\hskip \labelsep {\bfseries #1}]}{\end{trivlist}}
\newcommand{\qed}{\nobreak \ifvmode \relax \else
      \ifdim\lastskip<1.5em \hskip-\lastskip
      \hskip1.5em plus0em minus0.5em \fi \nobreak
      \vrule height0.75em width0.5em depth0.25em\fi}



\setlength{\tabcolsep}{1pt}

\textwidth 6.5in \textheight 9.0in \oddsidemargin 0.0in
\evensidemargin 0.0in \topmargin -0.5in
\addtolength{\columnsep}{2mm}

\begin{document}

\title{Discrete Conformal Deformation: Algorithm and Experiments}

\author{ 
Jian Sun \thanks{Mathematical Sciences Center, Tsinghua University, Beijing, 100084, China. \textit{Email: jsun@math.tsinghua.edu.cn.}}
\and Tianqi Wu \thanks{Mathematical Sciences Center, Tsinghua University, Beijing, 100084, China. \textit{Email: mike890505gmail.com.}}
\and Xianfeng Gu \thanks{Department of Computer Science, Stony Brook University, New York 11794, USA. \textit{Email:  gucs.stonybrook.edu}}
\and Feng Luo \thanks{Department of Mathematics, Rutgers University, New Brunswick, NJ 08854, USA.\textit{Email: fluomath.rutgers.edu}}
}




\date{}


\maketitle

\begin{abstract}
In this paper, we introduce a definition of discrete conformality for
triangulated surfaces with flat cone metrics and describe an algorithm for solving the problem
of prescribing curvature, that is to deform the metric discrete conformally 
so that the curvature of the resulting metric coincides with the 
prescribed curvature. We explicitly construct a discrete conformal map 
between the input triangulated surface and the deformed triangulated surface.  
Our algorithm can handle the surface with any topology with or without boundary, 
and can find a deformed metric for any prescribed curvature  satisfying the 
Gauss-Bonnet formula. In addition, we present the numerical examples to show 
the convergence of 
our discrete conformality and to demonstrate the efficiency and the robustness
of our algorithm. 
\end{abstract}

\newpage
\section{Introduction}
In this paper, we introduce a definition of discrete conformality for
triangle meshes and describe an algorithm for solving the problem
of prescribing curvature, that is to deform the metric discrete conformally 
so that the curvature of the resulting metric coincides with the 
prescribed curvature. In addition, we explicitly construct a 
discrete conformal map between the original triangle mesh 
and the deformed triangle mesh. The problem of prescribing curvature
has many applications in various engineering fields including
computer vision, image processing, and computer graphics. 
For instance, by setting the curvature to be zero, one can discrete conformally
flatten a triangle mesh into the plane and thus obtain a discrete conformal 
parametrization of the mesh. 

Our discrete conformal deformation consists of two basic operations:
{\it vertex scaling} and 
{\it cocircular diagonal switch} (see Figure~\ref{fig:basic-operations}). 
Assume a closed surface  
is equipped with a triangulation  where ,  and  are the vertex set, 
the edge set and the triangle set, respectively. An edge length assignment
 assigns any edge  with the length
, which determines a metric on  provided that
the triangle inequalities are satisfies for all triangles in . 
The operation of vertex scaling is a special way of changing the edge lengths. Specifically, 
the vertex scaling of the edge length assignment  by a 
function  is another edge length assignment, 
denoted , so that for any edge  with the endpoints  

We call the function  the discrete conformal factor. A discrete conformal factor 
is legitimate if the edge length assignment  satisfies the triangle inequalities
for all the triangles in . By a simple dimension counting,  the vertex scalings of  
will not in general cover all possible edge length assignments on .
For an edge  in , 
denote  and  the two triangles in  incident to , 
and , respectively , are two other edges of  respectively  listed
counterclockwise, as shown in Figure~\ref{fig:basic-operations}. Define the length cross ratio
of the edge  under the edge length assignment  as . 
Then it is easy to verify that an edge length assignment  is a vertex scaling of  if 
and only if the length cross ratio is preserved, i.e., , for any edge  
in . 


The vertex scaling operatoion was introduced by	Roek and Williams in physics~\cite{Rocek} 
and independently by Luo in mathematics~\cite{luo}. Luo established a (convex) variational principle 
associated to the vertex scaling operation.  This variational principle has many nice properties.
The one most relevant to the applications in engineering fields
is that there is an efficient algorithm to solve the problem of prescribing curvature, 
and thus the problem of discrete conformal parametrization as a special case. 
The main observation is that given a triangulation  and an edge length assignment 
over its edges , 
the conformal factor  so that the metric determined by  achieves 
the prescribed curvature is the unique minimizer of a convex energy, 
whose gradient and Hessian can be explicitly estimated. 
Thus, the minimizer can be efficiently computed by Newton's method. 
The convex energy discovered by Luo~\cite{luo} takes the form of path integral of 
a differential one-form.  An explicit formula of this convex energy based on Lobachevsky function
was found later by Springborn et al.~\cite{ssp}. 
However, there are the cases where the discrete conformal factor  solving the prescribing 
curvature problem does not exist. In fact, in those cases, the minimizer  of the above convex 
energy is not legitimate. 


\begin{figure}[t]
\begin{center}
\begin{tabular}{cc}
\includegraphics[width=0.5\textwidth]{./pics/vertex_scaling.pdf} &
\includegraphics[width=0.5\textwidth]{./pics/diagonal_switch.pdf}\\
Vertex scaling & Cocircular diagonal switch
\end{tabular}
\end{center}
\vspace{-0.1in}
\caption{Two basic operations in discrete conformal deformation.
\label{fig:basic-operations}}
\end{figure}

To tackle the issue of existence, we introduce the second operation: diagonal switch. 
Let  be an edge in  adjacent to two distinct triangles  and  in , 
the diagonal switch of the edge  replaces  by the other diagonal 
of the quadrilateral . This also replaces the triangles  by two new triangles
, as shown in Figure~\ref{fig:basic-operations}, and produces a new triangulation  
on .  With the diagonal switch operation, 
we can extend the domain of legitimate discrete conformal factors. To see this, we start with a Euclidean
triangulation  and an initial edge length assignment  over the edges in , and 
then we vertex scale  by continuously changing the function  along the gradient of the 
above convex energy. At some point, some triangle in  may become degenerate under the new 
edge length assignment , that is the triangle inequality becomes equality. 
It was shown by Luo~\cite{luo} that in any degenerated triangle one of its inner angle must equal . 
By diagonally switching the edge opposite to that angle, the degenerated triangle is removed. 
In this way, one may make the conformal 
factor  legitimate. However, the diagonal switch operation brings up many complicated issues. 
For instance, with diagonal switch, a priori, the energy depends on not only the discrete conformal 
factor , but also the triangulations on , which are combinatorial structures. 
{\it Can the energy with combinatorial variables still be convex?}
In addition, the new edges are emerging with the diagonal switch operation. 
{\it What is the assignment of the lengths for these edges which are not in the 
initial triangulation?}
Furthermore, {\it if multiple triangles simultaneously become degenerate, 
do different sequences of diagonal switch operations lead to the same solution?}

Our key observation to make the operation of diagonal switch work nicely is to switch an edge 
well before its incident triangle become degenerate. Specifically, an edge  shared by the triangles 
is switched when it fails to be Delaunay, that is the sum of the angles opposite to  in  and 
becomes bigger than . We call it cocircular diagonal switch 
as the edge  is switched at the moment that the quadrilateral  
become cocircular. See Figure~\ref{fig:basic-operations}.
We will answer the above three questions later. 
Roughly speaking,  two PL metrics on  are discrete conformal if one can be deformed to the other by a sequence of vertex 
scalings and cocircular diagonal switches. The rigorous definition is given in Definition~\ref{dc}. Based on this discrete
conformality, there always exists a PL metric which is discrete conformal to the initial PL metric and 
achieves any prescribed curvature. Furthermore, such metric can be computed using an efficient algorithm through
minimizing a convex energy. The algorithm can deal with the surface with any topology with or without boundary. 

In this paper, we describe our theory of discrete conformality with a focus on explaining
the algorithm for solving the problem of prescribing curvature, and present the
numerical examples, in particular to show the convergence of our discrete conformality. 
For the rigorous mathematical treatment of our theory, the interested readers are 
referred to~\cite{glsw1, glsw2}. 

\vspace{0.1in}
\noindent{\bf Related work.~}
There has been a lot of research into discrete conformality and we will not attempt 
a comprehensive review here. Instead, we focus on methods closely related to 
ours. Note all previous work deals with the concept of discrete conformality with 
fixed triangulations.  

Bobenko, Pinkall and Springborn~\cite{bps} introduced a geometric interpretation 
to the vertex scaling operation in both Euclidean and hyperbolic geometry using the 
volume of generalized hyperbolic tetrahedron. Glickenstein~\cite{Glickenstein1, Glickenstein2} 
extended the vertex scaling operation  to 3-dimensional piecewise flat manifolds.

One closely related work is circle patterns where a system of circles associated with
vertices. Two triangulated surfaces are considered conformally equivalent if the 
intersection angles of the circles are equal in both triangulated surfaces. 
The idea of approaching discrete
conformality through circle patterns goes back to Thurston~\cite{Stephenson03}. 
Rodin and Sullivan~\cite{RS} proved the Thurston's conjecture that Riemann mapping 
can be approximated by tangential circle packings (i.e., circle patterns with  
intersection angles) of hexagonal triangulations, and He and Schramm~\cite{he98} 
later showed the convergence is . Colin de Verdi\'{e}re~\cite{verdiere} 
discovered a variational 
principle for circle patterns with intersection angles in , 
and Chow and Luo~\cite{chow} introduced discrete Ricci flow based on circle packing and 
established a convergence theorem.. 
An issue with circle patterns is that not all metrics can be realized by circle patterns with 
intersection angles in . To tackle this issue, Bowers and Stephenson~\cite{Bowers} 
introduced inversive circle patterns where circles are not necessarily intersect.
Guo~\cite{Guo} established a variational principle for inversive circle patterns 
and showed that inversive distance circle patterns are locally rigid, i.e., locally 
determined by the curvature. Luo gave a proof for global rigidity in~\cite{Luo2011}.
However, the question of existence to the problem of prescribing curvature 
remains open. It is interesting to see if diagonal switch can help solving the existence problem. 
Many practical algorithms based on circle patterns have been proposed for conformally 
flattening triangulated surfaces, including~\cite{Kharevych, Jin}.

Conformality is closely related to harmonicity. Pinkall and Polthier~\cite{pinkall} proposed
an approach for flattening a triangulated surface by computing a pair of 
discrete harmonic functions conjugate to each other. 
Gu and Yau~\cite{Gu:2003} proposed a method to conformally flatten a surface
into the plane using holomorphic one-form. Assume  is a 
holomorphic one-form of the surface, and then the metric 
 is conformal and flat when . 
Noticing that any holomorphic one-form can be decomposed as 
 where  is a real harmonic one-form and 
 is its conjugate, Gu and Yau developed discrete algorithms
to approximate holomorphic one-forms from a triangulated surface by 
computing discrete harmonic one-forms and their conjugates. 


Another class of methods achieve conformality by minimizing conformal distortion. 
In these methods, piecewise linear maps are used to approximate actual conformal maps. 
Noticing that  for a conformal map , 
Levy et al.~\cite{Levy} proposed a method to find a piecewise linear map  from a 
triangulated surface into the plane by minimizing . 
Lipman~\cite{Lipman12} proposed a method to find a piecewise linear map  whose
conformal distortion  is bounded. 
Lui~\cite{Lui} et al. noticed that the magnitude of the Beltrami coefficient 
 is constant for the extremal map  
(the map with minimal conformal distortion) and proposed an iterative procedure 
to find a piecewise linear map  whose Beltrami coefficient has constant magnitude. 



\section{PL metrics and triangulations}
The purpose of this section is to explain the relation between
PL metrics and triangulations and to familiarize the readers with a more
general triangulated surface than the one usually encountered
in many engineering fields with the structure of simplicial complex (i.e., 
any higher dimensional simplex is uniquely determined by its vertices). 
For simplicity, we assume the surface  is closed without boundary. 
We will discuss how to deal with the 
surfaces with boundary in Section~\ref{surfaceswithboundary}. 

We start with a triangulated surface  with which we are familiar, i.e., embedded in  
where each triangle is the convex hull of its three vertices. For example, the boundary of a 
tetrahedron in  is such a triangle mesh with four Euclidean 
triangles, as shown in the leftmost picture in Figure~\ref{fig:triangulations}. 
Denote  the set of vertices. 
Note that other than the vertices, any point  has a flat neighborhood. 
This is obvious if  is in the interior of a triangle which is Euclidean. 
For  in the interior of an edge, one can flatten the 
two triangles incident to the edge into the Euclidean plane, and thus  also has
an (intrinsically) flat neighborhood. For a vertex , it has a neighborhood like a 
cone, as shown in Figure~\ref{fig:triangulations}. Thus the metric on  is flat with possible
cone singularities at a discrete set of vertices. We call such a metric a {\it polyhedral metric} 
or simply  PL metric. In general, a surface with a PL metric is obtained by isometrically 
gluing pairs of edges of a finite collection of Euclidean triangles. See Figure~\ref{fig:triangulations}
for examples. The converse also holds, that is any surface with a PL metric can be 
partitioned into Euclidean triangles. In fact, this can be done by keeping connecting pairs 
of cone singular vertices with intrinsically straight edges on  until no edge can be added
without intersecting the previously added edges in their interiors. Each partition is in fact
a triangulation with Euclidean triangles (in short {\it Euclidean triangulation}) on the surface .
The curvature of a PL metric is  everywhere except at the cone singular vertices where
the curvature is defined as  less the cone angle. Given a Euclidean triangulation
 on the surface ,  one can evaluate the cone angle at a vertex  
by summing the inner angles at  in the triangulation , and even deform the PL metric  
by changing the edge length. For an edge , let  be the length of  measured in the metric 
. The edge length assignment  with  uniquely determines
the PL metric .

\begin{figure}[t]
\begin{center}
\begin{tabular}{c}
\includegraphics[width=0.8\textwidth]{./pics/triangulations.pdf}
\end{tabular}
\end{center}
\vspace{-0.1in}
\caption{Triangulations of the boundary of a tetrahedron: The second row shows the gluing pattern of 
the triangles for different triangulations, where the edges marked with the same symbol are glued together. 
The triangulation  () is obtained by switching an edge in  (). 
\label{fig:triangulations}}
\end{figure}


Given a PL metric  on , there may be more than one Euclidean triangulations. 
Figure~\ref{fig:triangulations} shows three different triangulations of the boundary of a tetrahedron, 
where the triangulation  respectively  is obtained by diagonally switching the edge  in
 respectively the edge  in . It is generally true that any two (Euclidean) 
triangulations on  with the same set of vertices  are related by diagonal switches~\cite{Hatcher}. 
Among those Euclidean triangulations, there always exists a Delaunay triangulation where
every edge is Delaunay, that is the sum of the angles opposite to the edge is no bigger than ~\cite{Bobenko07}. 
There may exist more than one Delaunay triangulations.  If it happens that the sum of the angles 
opposite to an edge is exactly , then by (cocircular) diagonally switching that edge, 
we obtain another Delaunay triangulation. In fact, any two Delaunay triangulations are related by 
a sequence of cocircular diagonal switches. 


In the paper, we fix the topology of the closed surface  and a finite non-empty set 
and call the pair  a {\it marked surface}. A PL metric on the pair  is a PL metric
on  with the cone singularities in , the curvature of a PL metric on  is the function 
 sending a vertex  to  less than the cone angle at , and a 
triangulation of the pair  is a triangulation on  with vertex set . The curvature 
of a PL metric satisfies the Gauss-Bonnet formula:  where 
is the Euler characteristic number of . If  is a triangulation on , then 
. 

Let  be the space of PL metrics on  
\footnote{Strictly speaking, we should consider the set of equivalence classes of PL metrics where two PL metric
 on  are equivalent if there is an isometry  that is homotopic
to the identity map on . However this difference is subtle and can be ignored, especially for the purpose
of understanding the algorithm. }. Given a triangulation  of  with set of edges , let 
be the set of edge length assignments so that the triangle inequalities are satisfied for all triangles in . 
 is a convex polytope in . Since any edge length assignment  
determines a PL metric  on  with , there is an injective map 

sending  to a PL metric  on . The image  is the space
of all PL metrics  on  for which  is a Euclidean triangulation in . 
From the previous discussion, for any PL metric  on 
, there exists a Euclidean triangulation  on  whose edge length assignment is
given by the metric , i.e, there exists an edge length assignment  with .  
Thus we have  where the union is over all triangulations on . Notice that
 where  is the Euler characteristic number of , which is independent of . 
This means that   is a manifold of dimension 
with coordinate charts , as illustrated in 
Figure~\ref{fig:pl_metrics}. Note that in general .

\begin{figure}[t]
\begin{center}
\begin{tabular}{c}
\includegraphics[width=0.8\textwidth]{./pics/pl_metrics.pdf}
\end{tabular}
\vspace{-0.1in}
\end{center}
\caption{Coverings of the space of PL metrics .
\label{fig:pl_metrics}}
\end{figure}

Now we consider a subset of :

As we discussed before, for any PL metric , there is a Delaunay triangulation  whose edge length assignment 
is given by the metric , i.e., . 
Thus the set  also covers . 
In fact, this set forms a cell decomposition of ~\cite{rivin, glsw1}, as illustrated in Figure~\ref{fig:pl_metrics}.
Thus one may say that Delaunay triangulation is canonical as it is uniquely determined by the PL metric except 
for those metrics on the cell boundary which have multiple Delaunay triangulations.


Finally, we remark that in a Euclidean triangulation  on , it is possible to have multiple intrinsically 
straight edges between two vertices (e.g., the edges marked with ``'' and ``'' between 
 and  in the triangulation  in Figure~\ref{fig:triangulations}), 
and even to have an intrinsically straight loop edge (e.g., the edge marked with ``'' in the 
triangulation  in Figure~\ref{fig:triangulations}). Note, even if we start with a mesh 
with the structure of simplicial complex, we may end up with a mesh
with the more general structure as above as we allow diagonal switches. 

\section{Discrete Conformality}
Now we are ready to present our definition of discrete conformality. 


\begin{definition} \label{dc}(Discrete conformality for surfaces without boundary)
 Two PL metrics  on  are discrete conformal  if there exist sequences
 of PL
metrics  on  and triangulations
 of   satisfying

(a) each  is Delaunay in ,

(b) if , then  for a conformal factor 
where  and  are the edge length assignments over the edges of  with  and 
 for any edge  in . 

(c) if , then \footnote{Strictly speaking, 
 in the sense of equivalence class, that is   is isometric 
to  by an isometry homotopic to the identity in .}, and  are related by cocircular diagonal switches.
\end{definition}
This definition means that  are discrete conformal if and only if there exists a path 
connecting two PL metrics in the space of 
so that within a cell , the metrics deform along the path by vertex scalings, 
and on the cell boundary, the Delaunay triangulation is changed to another
via cocircular diagonal switches.  

The condition (a) in the definition is critical. Note that the operation of vertex scaling 
depends on the choice of triangulations. The vertex scaling of the same PL metric but under different
Euclidean triangulations may generate different PL metrics. So the previous definition of discrete conformality 
by vertex scaling~\cite{Rocek, luo, bps} heavily depends on the triangulations, which is
not inherent to PL metrics. On the other hand, by restricting to Delaunay triangulations which are canonical 
to PL metrics, our definition of discrete conformality is inherent to PL metrics. With this definition, 
we are able to prove the following uniformization theorem in~\cite{glsw1}. 

\begin{theorem} \label{thm:main} Suppose  is a closed connected marked surface
and   is any PL metric on .  Then for any  with , there
exists a PL metric , unique up to scaling,  on  so
that  is discrete conformal to  and the discrete curvature
of  is . 
\end{theorem}
In the above theorem, the conditions on the curvature  are necessary for
 to be a curvature of a PL metric on . The theorem states that those conditions are 
also sufficient for  to be achieved by a metric that is discrete conformal to the given
metric . This solves the existence and the uniqueness of the discrete conformal deformation 
mentioned in the introduction.

\subsection{Surfaces with boundary}
\label{surfaceswithboundary}
To deal with a surface with boundary, our strategy is to double the surface to 
remove the boundary, that is to make another
copy of the original surface and glue them along the boundary, and apply 
the discrete conformal deformation described above to the doubled surface, and finally cut out
a copy of the deformed surface from the deformed doubled surface. 

Let  be the boundary of the marked surface . Given a PL metric  on ,  consists
of a set of closed polygonal loops. A Euclidean triangulation on  is a partition of  into 
Euclidean triangles with the vertices . Note those edges of the polygonal loops of  have to be 
in the triangulation. 
For a vertex  on the boundary, its curvature is defined as  less than the cone angle at . 
With this definition, the Gauss-Bonnet theorem still holds: . 

The doubled surface of  is defined by taking the disjoint union of two 
copies of  and identifying the points on the boundary by an homeomorphism 
 which preserves the vertices on the boundary. Denote  
the doubled surface of .  A PL metric  on  induces 
a PL metric  on  by forcing the gluing map  to 
be isometric in . We call  the doubled metric of . 
Conversely, a PL metric on the doubled surface  is said 
to respect the doubling structure if it is the doubled metric of a PL metric on . 
Let the map  be the 
mirror map sending a point to the other copy. The map  is a self-isometric map if 
the PL metric on  respects the doubling structure. 
For convenience, the set of fixed points 
of the map  is called the boundary of .


\begin{definition}
\label{dcwithboundary}(Discrete conformality for surfaces with boundary)
 Two PL metrics  on the surface  with boundary are discrete conformal  if 
 their doubled metrics on the doubled surface of  are discrete conformal according to the definition~\ref{dc}. 
\end{definition}

\begin{theorem}
\label{theorem:dc_boundary} 
Suppose  is a connected marked surface with 
boundary and  is any PL metric on . Then for any  
with  and  for a vertex  on the boundary, 
there exists a PL metric , unique up to scaling, on the surface  
so that  is discrete conformal to  and the discrete curvature
of  is the prescribed curvature . 
\end{theorem}

The proof of the above theorem is deferred to the appendix. The basic idea is as follows. 
We obtain the doubled surface , and prescribe the curvature  for  as 
follows: for a vertex  on the boundary, set  
and for a vertex in the interior, set . 
It is easy to verify that the curvature  satisfies the hypotheses imposed in Theorem~\ref{thm:main}
to a target curvature on . Thus there exists a PL metric  discrete
conformal to  and the discrete curvature of  is the curvature of .
It remains to show that  respects the doubling structure and the restriction of  onto
 is the PL metric  with the property stated in the theorem. The key is to show that the conformal 
factor  remains respecting the conformal structure, i.e., , and the Delaunay triangulation 
of  under metric  has certain symmetric property. Specifically, 
any triangle  crossing an edge  on the boundary has to
have two vertices  so that , and moreover, if the third vertex  of the 
triangle  is neither  nor , the neighboring triangle  with  has also to cross
the edge  and two triangles  and  form a cocircular quadrilateral, as shown in Figure~\ref{fig:boundary_triangles}. 
In addition, the boundary edge  remains straight after the discrete conformal deformation, 
and subdivide the crossed cocircular quadrilaterals into two identical pieces. This makes it easy 
to algorithmically cut out a copy of the deformed surface from the deformed doubled surface. 

\begin{figure}[t]
\begin{center}
\begin{tabular}{c}
\includegraphics[width=0.6\textwidth]{./pics/boundary_triangles.pdf}
\end{tabular}
\end{center}
\vspace{-0.1in}
\caption{The symmetric property of the triangles crossing an edge 
on the boundary.
\label{fig:boundary_triangles}}
\end{figure}



\section{Convex Energy}
\label{sec:convexenergy}
In this section, we describe a convex energy for solving the problem of prescribing curvature.  
This will answer the first question raised in the introduction positively, i.e.,  there is
still a convex energy even with the seeming appearance of combinatorial variables for changing 
triangulations. Roughly speaking, as our discrete conformality only involves Delaunay
triangulations, which are canonical and determined by PL metrics, the combinatorial variables
of triangulations are not independent. 

\begin{figure}[t]
\begin{center}
\begin{tabular}{c}
\includegraphics[width=0.75\textwidth]{./pics/conformal_class.pdf}
\end{tabular}
\end{center}
\vspace{-0.1in}
\caption{1-1 correspondence between the conformal factors and the PL metrics on  discrete 
conformal to a PL metric . 
\label{fig:conformal_class}}
\end{figure}


Given a PL metric  on , we let  denote the space of PL metrics that
are discrete conformal to . The following lemma about  is important. 
\begin{lemma}
\label{lem:conformalmetricspace}
There is a  diffeomorphism from  to  where a point  is 
understood as a discrete conformal factor on . 
\end{lemma}
This lemma means there is a one-to-one correspondence between the PL metrics
discrete conformal to  and all discrete conformal factors on .
The energy is defined over , the space of all discrete conformal factors.  
The rigorous mathematical proof of this lemma uses the Teichm\"uller theory by 
establishing a one-to-one correspondence between PL metrics on  and the 
hyperbolic metrics on  with cusps and 
decorations at ~\cite{glsw1}. In this paper, we will not explain this 
connection to hyperbolic metrics. Instead, we will give an intuitive explanation of
the lemma aiming at the understanding of the algorithmic aspects, which although does not 
mathematically prove it. 

The space  has a cell 
decomposition induced by that of , where a cell is the intersection 
 for some triangulation  on . 
Note that the number of cells in  is finite~\cite{glsw1}. See Figure~\ref{fig:conformal_class}.
Let  be a Delaunay triangulation in the initial PL metric , and  be the
edge length assignment with  for any edge  of . Given a conformal factor 
, let  also denote a path in  from  and , 
that is  with  and .
We have . As we move along the path , we continuously 
deform the PL metric  discrete conformally through vertex scaling  by .
This will trace out a path  in the cell . 
At some point, this path may hit the boundary of 
the cell. Assume that happened at  and for example the quadrilateral  with the diagonal 
becomes cocircular in the metric .
We diagonal switch the edge  to the edge  and obtain another Delaunay 
triangulation  in , as shown in Figure~\ref{fig:basic-operations}.
Note that  is also Delaunay in . Due to the well-known Ptolemy identity 
for a  cocircular quadrilateral, we have 

where  and  are the endpoints of .
If let  be the edge length assignment over the 
edges of  so that  for  and 

then we have . Note that  for the new edge  
depends only on , in particular is independent of the conformal factor .  
This answers the second question raised in the introduction on how to assign the initial edge lengths
for the new edges after diagonal switches. 
Repeat the above procedure as we continuously move along the path . At the end, 
we reach a metric  in the cell   
for some triangulation . Mathematically, we can show that the final metric  is independent 
of the choice of path, namely if we choose another path connecting  and  and repeat the above
procedure, we reach the same metric . Thus  depends only on the initial PL metric  and the 
conformal factor . We write .  Conversely, for any PL metric , one can find a conformal
factor  so that . To see this, from the definition of discrete conformality, 
there is a path in  connecting  and . From the above procedure, it is easy to trace out 
a path  starting at  so that  is the path 
in  in  connecting  and . This shows that there is 
a one-to-one correspondence between  and . This in fact answers the third question 
raised in the introduction positively, i.e., different orders of switching diagonals leading to 
the same final PL metric.


We follow Luo~\cite{luo} and define the energy as a path integral of a differential 1-form on . 
Given a PL metric  on , let  be the curvature map
so that  is the curvature of the PL metric  on  for any conformal factor . 
Label the vertices  using . 
Let  and  denote the curvature  and the conformal factor  evaluated at the vertex , respectively.
Given a Euclidean triangulation  of , associate each edge  into two oriented 
half edges, one from  to  and the other from  to . 
Let  be the set of oriented edges in  starting 
at the vertex  and pointing to the vertex . Note that  may not be empty. 
Let  be the set of oriented edges in  starting from the vertex , i.e.,  
. 
For an edge  shared by the triangles  and , let  and  be the angles opposite 
to  in  and  respectively. We have the following lemma on the curvature .  
\begin{lemma}
\begin{itemize}
\item[(i)]  is a  function on  for any vertex . 
\item[(ii)] Let  be a Delaunay triangulation in the metric  and then 

\item[(iii)] The matrix  is semi-positive definite and its null space
 only consists of constant vectors. 
\end{itemize}
\label{lem:pKpw}
\end{lemma}
\begin{proof}
For a  so that  is in the interior of a cell of , the above lemma was proved 
by Luo~\cite{luo}. In fact, in our setting, due to that the triangulation  is Delaunay, we have  
 and thus  for any edge , which means the 
matrix  is diagonally dominant. 
So it remains to show that  is  on the cell boundaries.

Assume  is another Delaunay triangulation in the metric . Since  and  are related by a sequence of 
cocircular diagonal switches, we may assume  is obtained from  by one cocircular diagonal switch. 
Assume the diagonal  of the quadrilateral  is switched to the other diagonal , as shown in 
the right picture of Figure~\ref{fig:basic-operations}. 
For any vertex ,  obviously remains the same before and after the diagonal switch. From the equation~\eqref{eqn:pKpw}, 
the evaluation of  only involves the quantity  
associated to any edge . Observe that only for the sides and the diagonals of the quadrilateral , 
this quantity may differ before and after the diagonal switch. For the diagonal , this quantity is  in  
due to that  , and remains  in  as  is not an edge in , and 
similarly for another diagonal . For any side, say  (see Figure~\ref{fig:basic-operations}), 
as the angle opposite to  in the triangle  equals the angle opposite to  in the triangle , 
this quantity associated to  remains the same before and 
after the diagonal switch. This shows that  is  for any vertex . 
\end{proof}

Define a differential 1-form on the space of conformal factors as 
 for any 
From Equation~\eqref{eqn:pKpw},  for 
any , implying that  is closed and thus exact as the domain   is simply connected. 
This means the path integral of  only depends on the endpoints of the path. Given a prescribed curvature 
, define the energy  over the space of discrete conformal factors as 

Note that the gradient of the energy  and 
the Hessian of the energy . From Lemma~\ref{lem:pKpw}, the Hessian 
 is semi-positive definite and thus the energy  is convex and strictly convex restricted to the subspace 
. If the prescribed curvature  satisfies the conditions
stated in Theorem~\ref{thm:main}, there exists a discrete conformal factor  so that . 
This means , implying that  is the unique minimum of the energy  on the subspace .
Thus, one can employ the Newton's method to find  and hence the PL metric  which realizes the 
prescribed curvature. 

\section{Discrete Conformal Map}
\label{sec:disconf-map}
In this section, we construct a map  on the same marked 
surface  but with two PL metrics  and  discrete conformal to each other, which
we call the discrete conformal map from  to . 
In~\cite{glsw2}, given a PL metric  on ,  we equip  with another hyperbolic metric 
with cusps (but no decorations) at , denoted . We show that  and  are discrete conformal 
to each other if and only if  and  are isometric to each other by an isometry homotopy to the identity. 
The discrete conformal map  from  to  is defined as that isometry from  to . 
In this paper, instead of establishing the connection to the hyperbolic metric, we give a more constructive 
description of the discrete conformal map for the purpose of better understanding the algorithm of 
explicitly constructing the map. To make it concrete, assume the triangulations
 and  are Delaunay under  and , respectively. Think of the 
surface  as the disjoint union of the Euclidean triangles in   with pairs of 
edges identified by isometries, and similarly for the surface . 
Note that the map  restricted to  is the identity map on  and the task is to extend 
the map to the interiors of the edges in  and the interiors of the triangles in . 

\begin{figure}[t]
\begin{center}
\begin{tabular}{c}
\includegraphics[width=0.5\textwidth]{./pics/mapping_piece.pdf}
\end{tabular}
\vspace{-0.1in}
\end{center}
\caption{The mapping triangle of a polygonal facet of . 
\label{fig:mapping_piece}}
\end{figure}

Let  be the conformal factor so that . 
First, we consider the special case where there is a triangulation  which 
is Delaunay in both  and , i.e.,   are in the same cell .
In this case, the discrete conformal map is the so-called piecewise circumcircle
preserving projective map introduced by Bobenko et al.~\cite{bps}. 
Let the triangles  and  be the same triangle in  with 
the vertices  and the edge lengths measured in  and  respectively. The 
the map
 is defined in terms of the barycentric coordinates as

where  is the normalizing factor. 
It is shown in~\cite{bps} the map  is a projective map from 
onto  which also maps the circumcircle of  to the circumcircle
of . For two triangles  and  sharing the 
edge , the maps  and   coincide on the 
common edge . Thus, we can glue the maps on individual triangles together
to form a globally continuous map, which by definition is the discrete conformal map
. Note that the straight line remains 
straight within a triangle under the map  as any projective map preserves straight lines. 

Next, we consider the general case where  and  may not be in the same cell in .
Consider a path  with  to . 
Let  be the intersections of   with the boundaries of the cells
in  listed in the increasing order of their path parameter. See Figure~\ref{fig:conformal_class}
for an illustration. For convenience, let
 and . For any ,   and  are in the 
same cell  for some triangulation . 
Let  be the discrete conformal map from  to
 defined in the above special case. Then the discrete conformal map from  to  
by definition is the compositions of the above maps .

We now state some properties of the discrete conformal map. For their proofs, the interested readers
are referred to the paper~\cite{glsw2}. The most important property is that the map  is 
independent of the choice of the path . Namely, if we choose another path , 
we may end up with a different set of maps  but their composition 
gives the same map . Therefore, the map  is indeed a well-defined map from  to . 
The second property is that a straight line on  
remains straight within a triangle in  under the map  and similarly for a straight line 
on  under the inverse of . Another important property is
that  remains a piecewise circumcircle preserving projective map but on the smaller pieces. 
Specifically, for two triangles  and , let  
and . If , then  and 
 is the restriction onto  of the circumcircle preserving projective
map from a triangle  to the triangle . The triangle  is constructed as 
follows.  The preimage of the edges of  inside  are straight segments. 
See Figure~\ref{fig:mapping_piece} for an illustration. 
Extend them linearly to intersect the circumcircle of the triangle . One can show that there are always 
exactly three intersection points. If we labeled the intersections according to the labels of the endpoints of 
the edges in , this constructs the triangle , which we call the {\it mapping triangle} of . 
Let  and  the lengths of the edge  in  and  respectively for 
any .  Calculate  and similarly 
for . Then by replacing  by  in \eqref{eqn:cppm}, we construct the circumcircle preserving
projective map from  to . 



For the surface  with boundary, one can verify that the straight line which cuts the region 
 into two identical subregion (see Figure~\ref{fig:boundary_triangles}) is 
the image of the segment  under the discrete conformal map . 
Therefore, if let  denote the discrete conformal map on the doubled surface, then the restriction 
of  onto a copy of  is a map from  to , which we define as the discrete 
conformal map  from  to . 

\begin{figure}[!t]
\begin{center}
\begin{tabular}{c}
\includegraphics[width=0.75\textwidth]{./pics/algorithm_overview.pdf}
\end{tabular}
\end{center}
\vspace{0.1in}
\caption{The main objects and the procedures of the algorithm for solving 
the problem of prescribing curvature. 
\label{fig:algorithm_overview}}
\end{figure}



\section{Algorithm}
\label{sec:algorithm}
We have presented the main ideas of the algorithm for solving the problem 
of prescribing curvature. In this section, we give more details of the algorithm at 
the implementation level. 


The main objects and the procedures used in the algorithm are shown in Figure~\ref{fig:algorithm_overview}. 
In the problem of prescribing curvature, assume we are given a closed surface  with 
a Euclidean triangulation  which is Delaunay, and a desired curvature
. Note that the initial PL metric  on 
is determined by the edge lengths of the Euclidean triangles in . The goal of the algorithm is: 
(1) to find a triangulation  on  and an edge length assignment  
over the edges in  so that the PL metric  on  determined by 
is discrete conformal to  and the curvature of  equals , 
and (2) to construct the discrete conformal map  from  to . This is the 
core of the algorithm, which is performed by the procedure ``Deform'' using the Newton's method
described in Section~\ref{sec:convexenergy}. In many applications, 
the given Euclidean triangulation  on  may not be Delaunay. The procedure 
``Delaunay'' is employed to convert  to a Delaunay triangulation  under the same PL metric
by diagonal switches, as described in~\cite{Fisher:2006}. 
Moreover, if the initial surface has boundaries, the procedure ``Double'' is to double the surface 
to remove the boundary, and the procedure ``Cut'' is to cut out a copy of the deformed surface 
from the deformed doubled surface, as described in Section~\ref{surfaceswithboundary}. 
Both the procedures ``Double'' and ``Cut'' are straightforward to implement. 


The procedure ``Deform'' deforms the metric and also changes the triangulations as shown 
in Algorithm~\ref{alg:deform}, whose implementation is more involved than the other procedures. 
Note that the discrete conformal map  is a piecewise circumcircle preserving projective map 
on the pieces of the common refinement of the triangulations  and , denoted . 
Topologically, the refinement 
is a also polyhedral surface whose vertices consists of the
vertices  and the intersections of the edges in 
with the edges in . To see the geometry of ,
consider the discrete conformal map .
According to the theory of discrete conformal mapping described in~\cite{glsw2},
an edge in  is pulled back to  and geometrically
becomes a polygonal line which is straight inside a triangle of . 
Similarly, an edge in  is pushed forward
to  and is straight within a triangle of . Therefore,
each edge of  is geometrically straight on both  and .
See Figure~\ref{fig:refinement} for an illustration. For instance, the triangle  in
 is subdivided into three polygonal facets in  and similarly
for the triangle  in .
All of the involved polyhedral surfaces are represented by halfedge data structures.  
A mechanism is built for these polyhedral surfaces to communicate with each other as follows:
each edge in  or  has the access to its first sub-edge in ,
and each edge in  has the access to the edge in  and/or 
to which it belongs. In the example shown in Figure~\ref{fig:refinement}, for instance,
each halfedge of the edge  in  has a pointer pointing to its first
sub-halfedge in  and similarly for the halfedges of  in .
At the same time, each sub-halfedge of  () in  is equipped
with a pointer pointing back to the corresponding halfedge of  in 
( in ).

\begin{figure}[!t]
\centering
\includegraphics[width = 0.6\textwidth]{./pics/refinement.pdf}
\vspace{0.1in}
\caption{Refinement: The dotted lines in   are the images
of the edges in  under the discrete conformal map from  to  ,
which are straight within a triangle in , and similarly for the dotted lines in .
}
\label{fig:refinement}
\end{figure}



The final PL metric is determined by the edge length assignment  over the edges in .
For a vertex  of  which is the intersection of an edge  in  and 
an edge  in  in the interior, we store both its positions on  and . In this way, 
we can visualize  for any edge  in  on the input surface 
 and  for any edge  in  on the deformed surface . 
For a polygon  in , we store the edge lengths of its mapping triangle
for the purpose of constructing the discrete conformal map .  
In each iteration in the Newton's method, the conformal factor  is updated to , 
which may change the triangulation , the refinement  and  as well. 
The sub-procedure ``MoveTo'' shown in Algorithm~\ref{alg:moveto} presents more details on how to 
update the conformal factor and the combinatorial structures of  and . 


Let  for an 
edge  in  and . Consider an edge 
as shown in Figure~\ref{fig:basic-operations}.  That  is Delaunay is by cosine law
equivalent to 

which is a linear constraint in the variables . Thus if we change the variables from  to , the 
cell  becomes a convex polytope. We choose a path from  to 
so that it is a line segment in the variables . This makes it easy to detect which edge to switch first
as it amounts to compute the intersections of the line segment with the hyperplanes defined by the 
linear constraints.  

\begin{algorithm}[!h]
\floatname{algorithm}{Algorithm}
\caption{Deform(, ,  and )}
\label{alg:deform}
\begin{algorithmic}[1]
\STATE Initialize 
\STATE Set ;
\STATE Evaluate  and set 
\WHILE {} 
\STATE Evaluate  and set  
\STATE MoveTo(, , , , )
\STATE 
\ENDWHILE
\STATE Output , , .
\end{algorithmic}
\end{algorithm}

\begin{algorithm*}[!h]
\floatname{algorithm}{Algorithm}
\caption{MoveTo(, , , , )}
\label{alg:moveto}
\begin{algorithmic}[1]
\STATE Assume  with  be a path satisfying . 
\STATE Let the edge  in  be the first edge that fails the Delaunay condition along the path . 
\IF  { exists}
\STATE Assume  fails to be Delaunay at 
\STATE Switch the edge , and update 
\STATE Update : (i) for the newly generated polygons, compute the edge lengths of their mapping triangles, 
and (ii) for the vertices of  which are not the vertices of , 
compute their new positions on the edges of  under the edge length assignment . 
\STATE MoveTo(, , , , )
\ELSE
\STATE For the vertices of  which are not the vertices of , compute their new positions 
on the edges in  with the edge length assignment .
\ENDIF
\STATE Output , , , and .
\end{algorithmic}
\end{algorithm*}


\begin{figure}[!t]
\begin{center}
\begin{tabular}{c}
\includegraphics[width=0.9\textwidth]{./pics/hemisphere_500_1.pdf}
\end{tabular}
\end{center}
\vspace{0.1in}
\caption{(a): The polyhedral surface  of a spherical cap.  
(b): The doubled polyhedral surface. The Delaunay 
triangulation  consists of the triangles with blue edges. 
The black edges are non Delaunay edges in the triangulation . 
(c): Half of the doubled polyhedral surface after discrete conformal deformation. 
The Delaunay triangulation  consists of the triangles with red edges. 
The blue edges are the push-forward of the switched edges in 
under the discrete conformal map . 
Note that the blue edges may not be straight. 
(d): The triangles with red edges are the pull-back of the edges in  
under the map . Note that the red edges may not be straight. 
\label{fig:algorithm_illustration}}
\end{figure}


Finally, the purpose of constructing the refinements  and  
is for visualization. When the input Euclidean triangulation  on  is embedded 
in , we can pull back the triangulations  and  onto  for the purpose 
of visualization.  The common refinement of the triangulations  and , denoted ,
is also computed in the procedure ``Delaunay''. The procedure ``Subdivide'' is to
compute the common refinement of the triangulations of  and . In this way, we can
pull back the edges in  back to  under the identity map over  and the edges in
 back to  under the discrete conformal map  from  to . 



Figure~\ref{fig:algorithm_illustration} shows the results of the different procedures 
when the algorithm runs over the polyhedral surface of a spherical cap.  
In this example, we can embed the doubled polyhedral surface into 
and visualize both the triangulations  and  (Figure~\ref{fig:algorithm_overview}(b)). 
Moreover, we can visualize the 
pull-back of the triangulation  under the map  (Figure~\ref{fig:algorithm_overview}(d)). 
In addition, we set the target curvature  everywhere except at four marked points on the boundary 
where it is set to be . In this way, we can embed the deformed polyhedral surface 
into a rectangle as shown in Figure~\ref{fig:algorithm_overview}(c). We use the procedure described 
in~\cite{gu:2008} to layout a flat surface into the plane. 
Note in all the examples shown in the paper, we fix the  in Algorithm~\ref{alg:deform} to be . 

\begin{figure}[!ht]
\begin{center}
\begin{tabular}{c}
\includegraphics[width=0.9\textwidth]{./pics/star_eight.pdf}
\end{tabular}
\end{center}
\vspace{0.1in}
\caption{(a, d): The original polyhedral surfaces. The marked vertices
have non zero prescribed curvature. The green edges show a tree (a cut graph)
passing the marked vertices.
(b, e): The planar embedding of the deformed polyhedral surfaces after cutting
them open along the tree (the cut graph). 
The red edges are the edges in the triangulation  and the blue edges are the images of
the switched edges in the triangulation  under the discrete conformal map . 
(c, f): The red edges are the preimages of the edges in the triangulation  under the map , 
the blue edges are the edges in  got switched during the conformal deformation.
The gray edges in (b, c, e, f) are the non Delaunay edges in the 
triangulation .
\label{fig:simple_examples}}
\end{figure}

\section{Experimental Results}
In this section, we will show numerical examples, demonstrate numerically 
the convergence of our discrete conformality, and compare to the state of the art. 
For convenience, we follow the notation in Figure~\ref{fig:algorithm_overview}, and 
denote  the (doubled) input triangulation,  and  the Delaunay triangulations
under the initial metric  and the deformed metric  respectively. 

\subsection{Simple Examples}
\label{sec:simple-examples}
In this subsection, we show a few examples with small number of triangles 
for a clear illustration of the geometric deformation of the metric and the 
combinatorial changes of the triangulation. In the first couple of examples, 
to visualize the result metric, we prescribe the curvature to be zero except at 
a few vertices in order to satisfy the Gauss-Bonnet theorem. A vertex with non 
zero curvature is called {\it singular}. The first example is a polyhedral surface of 
topological sphere which we call Star shown in the left column of 
Figure~\ref{fig:simple_examples}. The total curvature 
of Star is . We choose three singular vertices as marked in 
Figure~\ref{fig:simple_examples}(a) where the curvature is set to be . 
To embedding the deformed
Star into the plane, we cut Star along a tree of the edges in  passing through three singular 
vertices. The tree is shown in green in Figure~\ref{fig:simple_examples}(a). 
The planar embedding of the deformed Star is shown in Figure~\ref{fig:simple_examples}(b),
where the red edges are the edges of the triangulation  and the blue edges are the images 
of the switched edges in the triangulation  under the discrete conformal map . 
The red edges in Figure~\ref{fig:simple_examples}(c) are the preimage of the edges in the 
triangulation  under the map .  The second example is a polyhedral surface of 
genus two which we call Eight shown in the right column of Figure~\ref{fig:simple_examples}. 
The total curvature of Eight is . We choose one singular vertex  as marked in 
Figure~\ref{fig:simple_examples}(b)  whose curvature is set to be  .
To embedding the deformed Eight into the plane, we cut Eight
along a cut graph consisting of the edges in  passing through the singular vertex.
The cut graph is shown in green in Figure~\ref{fig:simple_examples}(d). 
The planar embedding of the deformed Eight is shown in Figure~\ref{fig:simple_examples}(e). 
The red edges and the blue edges in 
Figure~\ref{fig:simple_examples}(e, b) have the same meaning as those in Star. The gray edges
are the non Delaunay edges in the triangulation . 

The main purpose of the next couple of examples is to show the triangulation of  when we 
prescribe a curvature close to the boundary of the domain of all possible curvatures. In both examples, 
the curvature is prescribed to be  at every vertex except at one vertex (labeled by  
for later reference) whose curvature is 
set to satisfy the Gauss-Bonnet theorem and usually a negative value. In the example of Star, the 
prescribed curvature at the vertex  is , and in the example of Eight, it is . 
In Figure~\ref{fig:extreme-curvatrue}, the red edges are the preimage of the edges in the triangulation 
pulled-back by the discrete conformal map  into the input surface. All of the triangle in  has 
 as its vertex. In fact, in these two examples, at least two of three vertices of any triangle 
in  are . 

\begin{figure}[!t]
\begin{center}
\begin{tabular}{cc}
\includegraphics[width=0.3\textwidth]{./pics/star_extreme.png} & 
\includegraphics[width=0.5\textwidth]{./pics/eight_extreme.png}\\
Star & Eight
\end{tabular}
\end{center}
\vspace{-0.1in}
\caption{
The color scheme of the edges are the same as that in Figure~\ref{fig:simple_examples}(c, f). 
\label{fig:extreme-curvatrue}}
\end{figure}



\subsection{Convergence}
In this subsection, we will present numerical evidences showing the convergence of our discrete conformality.
In addition, we will demonstrate the efficiency and the robustness of our algorithm, in particular against 
the quality of the input triangulations, and compare its performance to the state of the art. 
We check how much the conformality is preserved when the triangulated surfaces
are flattened into the plane, and use two types of criteria to measure the conformality. 


\subsubsection{Criteria} 
For the examples where the (approximated) ground truth of conformal flattening is known, we can compare
the results with the ground truth. 
Let  be the flattening map of the (approximated) ground truth, and 
 be the flattening map constructed by our algorithm or other methods described below. 
We use the following two norms to measure the approximation error:

where  is the area weight, which is estimated as a third of the total area of 
the triangles in  incident to the vertex . 

In general, the ground truth of conformal flattening is not known. Given an orientation 
preserving map  between two Riemann surfaces, the Beltrami coefficient is
, where  is a complex number representing the local coordinates. 
The map  sends an infinitesimal circle to an 
infinitesimal ellipse with the ratio of major semiaxis to minor semiaxis equal
. Note  as the map  preserves orientation. 
 is called the conformal distortion of the map  and  if and only if  
is conformal. So we check the conformality of the map 
by measuring how far  is away from . Specifically, we estimate  and 
.

Let  be the input triangulated surface.
In the methods we described below for comparison, 
the constructed flattening map  is piecewise linear, namely on a triangle , 
 is the linear extension of the map on the vertices of the triangle. 
let  represent the linear map . The conformal distortion 
of this linear map  can be computed as  . 
For a piecewise linear flattening map , we have

In our method, from the discussion in Section~\ref{sec:disconf-map}, the constructed flattening map  
is piecewise circumcircle preserving projective. Specifically, for a polygonal face  in
the common refinement , let  and  be the triangle in  and  containing . 
The map  restricted to , denoted , is the restriction to  of the circumcircle preserving 
projective map from the mapping triangle  to the triangle .
Let  be the linear map from  to . In~\cite{glsw2}, we have shown
that . Therefore, for our flattening map , we have the 
following upper bounds on  and , which are easy to estimate. 

where  denotes the set of the polygonal faces in  and  denotes
the area of  as a subset of the triangle . 

\subsubsection{Conformal Flattening Methods}
We briefly describe three methods including ours of conformally flattening triangulated surfaces into the plane.

\vspace{0.1in}
\noindent{\bf Method of Discrete conformal Deformation (DC).} 
This flattening method is based on our discrete conformal deformation. To flatten a triangulated surface
into the plane, we basically prescribe the curvature to be  and solve the problem of prescribing 
curvature using the algorithm described in Section~\ref{sec:algorithm}. Due to the obstruction of 
topology, the target curvature can not be  everywhere. We call those whose curvature are not zero 
the singular vertices. For a surface of topological disk, we choose three singular vertices on the boundary
and set the curvature there to be . In this way, we flatten a triangulated surface of topological
disk onto an equilateral triangle. This flattening map is guaranteed to be one to one. For a surface of 
topological sphere, as we did in Section~\ref{sec:simple-examples} for Star, we choose three singular 
vertices whose curvatures are set to be , and cut the surface along a tree of the edges in 
passing through the singular vertices for flattening the triangulated surface. 
For a surface of genus , we choose  singular vertices whose curvatures are set to be . 
and cut the surface along a cut graph consisting of the edges in  and passing the singular vertices 
for flattening the triangulated surface.  

\vspace{0.1in}
\noindent{\bf Method of Holomorphic Form (HF).} 
Gu and Yau~\cite{Gu:2003} proposed a method to conformally flatten a surface of 
genus  into the plane using holomorphic one-form. Assume  is a 
holomorphic one-form of the surface and it is well-known that the metric 
 is conformal and flat when . Noticing that 
any holomorphic one-form can be decomposed as 
where  is a real harmonic one-form and  is its conjugate, Gu and Yau 
developed discrete algorithms for approximating from a triangulated surface
a basis  of the space of real harmonic one-forms
and their conjugates . Then 
 
contains a basis of the space of holomorphic one-forms and any linear combination
 is a holomorphic one form. Integrate the real part 
and the imaginary part of  along the edges of the triangulated surface to
obtain the -coordinates and the -coordinates  respectively for the vertices. 
Note the -coordinate functions computed by integration are multi-valued 
at a subset of vertices. The edges with both endpoints multi-valued form a cut-graph 
of the surface. Cut the surface along this cut-graph and obtain a fundamental domain 
of the surface. The -coordinate functions conformally map this fundamental domain 
into a planar region. For the surfaces with boundary, one can double the surface to remove
the boundary by gluing two copies along the boundary, and apply the above algorithm and 
take half of the computed planar embedding. For a surface with topological disk, in order
to obtain an embedding onto unit disk, the following procedure is used: (1) remove a triangle
from the given triangulated surface to make an annulus; (2) apply the above algorithm to 
obtain an planar embedding of rectangular shape; (3) take exponential to map the rectangle domain
into an annulus with unit outer radius, and put back the removed triangle to obtain the final embedding
onto unit disk.  
In our experiments, we use the implementation made available to us by the authors. 


\vspace{0.1in}
\noindent{\bf Method of Bounded Distortion (BD).} 
Lipman~\cite{Lipman12} considered the problem of finding a piecewise linear map  mapping
a triangulated surface into the plane so that the conformal distortion  is less than 
some prescribed number . This amounts to solve a non-convex optimization problem, 
which was reduced to a conic optimization problem by restricting the domain of optimization
to a convex subset. In~\cite{Lipman12}, Lipman also proposed a binary search strategy to find
a map with the ``optimal'' conformal distortion. Note the reduced conic optimization may miss
a map with the conformal distortion less than the prescribed number, and thus gives a wrong 
feedback to the binary search, which therefore may not reach a map with true
optimal conformal distortion. In our experiments, the following iterative procedure is used to 
find a map with small conformal distortion, which is more efficient comparing to 
the binary search strategy. In each iteration, assume a piecewise linear map  is given,
and one construct a convex set of maps whose conformal distortion is less than 
 and then use the conic optimization to find the next map  
in this convex set.  The iteration is started with the Tutte embedding where the position 
of an interior vertex is the average of the positions of its neighboring vertices, and 
iterate at most 10 times. For the vertices on the boundary, we fix the positions of 
three of them and impose the linear constraints on the others so that the resulting 
range is a triangle. The optimization package MOSEK~\cite{mosek} is used for conic optimization. 
In our experiments, we use the implementation made available to us by the author. 


\subsubsection{Examples}
Now we run the above conformal flattening methods over several examples to show their performance, in particular
to compare their convergence property.  

\vspace{0.1in}
\noindent{\bf Spherical Cap.} A spherical cap is a portion of a sphere cut off by a plane, which can be
conformally flattened onto unit disk by composing the stereographic projection with a scaling. So in this 
example, we have the ground truth for conformal flattening, denoted as . We run the aforementioned 
methods over four continuously refined triangulated surfaces with approximately , ,  and  vertices, 
which are obtained by applying Cocone~\cite{Amenta00asimple}, a surface reconstruction algorithm, to the randomly drawn samples 
on the spherical cap and some additional samples on its boundary. Figure~\ref{fig:hemisphere_input} shows the triangulated surfaces
with  and  vertices.  

\begin{figure}[!t]
\begin{center}
\begin{tabular}{cc}
\includegraphics[width=0.42\textwidth]{./pics/hemisphere_1000.png} & 
\includegraphics[width=0.42\textwidth]{./pics/hemisphere_4000.png}
\end{tabular}
\end{center}
\vspace{-0.1in}
\caption{The triangulated surfaces of Spherical Cap with  (Left) and  (Right) vertices. 
\label{fig:hemisphere_input}}
\end{figure}


\begin{figure}[!h]
\begin{center}
\begin{tabular}{ccc}
\includegraphics[width=0.33\textwidth]{./pics/convergence_1000.jpg} & 
\includegraphics[width=0.33\textwidth]{./pics/riemann_holo_1000.jpg}& 
\includegraphics[width=0.33\textwidth]{./pics/bc_yaron_1000.jpg} \\
\includegraphics[width=0.33\textwidth]{./pics/convergence_map_4000_1.jpg} & 
\includegraphics[width=0.33\textwidth]{./pics/riemann_holo_map_4000_1.jpg} &
\includegraphics[width=0.33\textwidth]{./pics/bc_yaron_map_4000_1.jpg} \\
{\bf DC} & {\bf HF} & {\bf BD}
\end{tabular}
\end{center}
\vspace{-0.1in}
\caption{Results for Spherical Cap: The first row shows the planar embedding of the vertices
for the triangulated surface with  vertices. The blue "*" is the ground truth and the 
red "o" is the results computed by different methods; 
The second row plots the conformal distortion  of the planar embedding computed by 
different methods from the triangulated surface
with  vertices. Note that for the purpose of comparison, the range of color map
is fixed as , although the maximal conformal distortion of the planar embedding 
by HF and BD is larger than , as shown in Table~\ref{tbl:hemisphere}.
\label{fig:hemisphere}}
\end{figure}

\begin{table}[!h]
\begin{center}
\begin{tabular}{| c | c | c | c | c |}
\hline
Method  & 1000 &  4000 & 16000 & 64000   \\
\hline
{\bf DC} & (0.0102, 0.0137) & (0.0034, 0.0048) & (0.0014, 0.0020) & (0.0010, 0.0014) \\
\hline
{\bf HF} & (0.0023, 0.0078 ) & (0.0010, 0.0036) & (0.0011, 0.0030) & (0.0009, 0.0016)\\
\hline
{\bf BD}& (0.0131, 0.0300 ) & (0.0054, 0.0150) & (0.0040, 0.0092) & (0.0023, 0.0051)\\
\hline
\multicolumn{5}{|c|}{ } \\
\hline
{\bf DC} & (0.0505, 0.1592) & (0.0303, 0.1084) & (0.0160, 0.0570) & (0.0082, 0.0377) \\
\hline
{\bf HF} & (0.0638, 12.9485 ) & (0.0270, 1.4460) & (0.0169, 1.3563) & (0.0085, 0.8276)\\
\hline
{\bf BD}& (0.0829, 0.6582 ) & (0.0547, 0.9318) & (0.0313, 1.0055) & (0.0192, 1.1988)\\
\hline
\multicolumn{5}{|c|}{ } \\
\hline
{\bf DC} & 0.152 & 0.636 & 4.54 & 26.9 \\
\hline
{\bf HF} &  1.03 & 3.14 & 12.7 & 53.7 \\
\hline
{\bf BD}& 28.4 & 182  & 744 & 3395\\
\hline
\multicolumn{5}{|c|}{ timing (sec)} \\
\hline
\end{tabular}
\end{center}
\vspace{-0.1in}
\caption{Spherical Cap: approximation errors and running time.
\label{tbl:hemisphere}
}
\end{table}


For the methods of DC and BD, we choose three vertices  on the boundary so that they are mapped to 
the vertices of an equilateral triangle. To compare with the ground truth, we use the Schwarz-Christoffel mapping
which can explicitly evaluate the  conformal transformation mapping unit disk onto a triangle. 
In fact, we use the Schwarz-Christoffel Toolbox~\cite{driscoll} to compute the inverse map sending  to 
, respectively. In the method of HF, we have already embedded the spherical
cap onto unit disk. We align the computed embedding to the ground truth by a M\"{o}bius transformation. 


The first row of Figure~\ref{fig:hemisphere} shows the embedding of the vertices computed by different methods. 
In Table~\ref{tbl:hemisphere}, we show the approximation errors: , , , , and 
the timing in seconds used by different conformal flattening methods to compute the embedding. 
Note since the convergence is often stated for a compact region away from the boundary, we estimate the errors
 and  over the vertices which are mapped into the disk  of radius , and the errors
 and  over the polygons or triangles whose vertices are mapped into the disk . 
In the second row of Figure~\ref{fig:hemisphere}, we only plot the conformal distortion of those polygons 
or triangles used to evaluate  and . 

From Table~\ref{tbl:hemisphere}, three methods all converge about linearly in terms of the , 
 and  errors. In terms of the absolute value of these approximation errors, 
BD performs worse than DC and HF.  Only DC has a convergent  error, which is 
approximately linear. In terms of running time, DC and HF have a similar performance, 
while  is much slower.  
 



\begin{figure}[!t]
\begin{center}
\begin{tabular}{cc}
\includegraphics[width=0.35\textwidth]{./pics/hexagon_region2_0200.png} & 
\includegraphics[width=0.35\textwidth]{./pics/hexagon_region2_0100.png} \\
\includegraphics[width=0.3\textwidth]{./pics/hexagon_0200_cp.jpg} &
\includegraphics[width=0.3\textwidth]{./pics/hexagon_0100_cp.jpg}
\end{tabular}
\end{center}
\vspace{-0.1in}
\caption{Hexagonal Meshes and their circle packing. The first row: the input hexagonal triangulations with 1000 vertices (Left)
and 4000 vertices (Right). The second row: the circle packing in unit disk to the hexagonal triangulation above. 
\label{fig:hexagon_input}}
\end{figure}


\vspace{0.1in}
\noindent{\bf Hexagonal Mesh.}
The Riemann mapping from a planar region to unit disk
can be approximated using the Thurston's circle packing. 
Consider a hexagonal triangulation inside a planar region. 
One can explicitly construct a circle packing of unit disk, that is a collection of closed disks inside  unit disk 
having the following properties: (1) the interiors of the disks are disjoint; (2) the nerve of this collection 
of disks is isomorphic as graph to the -skeleton of the hexagonal triangulation; (3) the boundary of the disk
corresponding to each boundary vertex of the hexagonal triangulation tangentially touches unit circle.  
This construction induces a map, denoted , from the hexagonal triangulation into unit disk by 
mapping the vertices of the hexagonal triangulation to the centers of the corresponding disks and extending 
linearly to the triangles. Figure~\ref{fig:hexagon_input} shows two hexagonal triangulations inside a fixed planar 
region and their corresponding circle packings of unit disk.
Rodin and Sullivan~\cite{RS} showed the above induced map converges to the Riemann mapping
from the planar region to unit disk as the size of the triangles in the hexagonal triangulation goes to .

\begin{figure}[!t]
\begin{center}
\begin{tabular}{ccc}
\includegraphics[width=0.33\textwidth]{./pics/convergence_0100.jpg} & 
\includegraphics[width=0.33\textwidth]{./pics/riemann_holo_0100.jpg}& 
\includegraphics[width=0.33\textwidth]{./pics/bc_yaron_0100.jpg} \\
\includegraphics[width=0.33\textwidth]{./pics/convergence_map_0100_1.jpg} & 
\includegraphics[width=0.33\textwidth]{./pics/riemann_holo_map_0100_1.jpg} &
\includegraphics[width=0.33\textwidth]{./pics/bc_yaron_map_0100_1.jpg} \\
{\bf DC} & {\bf HF} & {\bf BD}
\end{tabular}
\end{center}
\vspace{-0.1in}
\caption{
Results for Hexagonal Mesh: The first row shows the planar embedding of the vertices
for the triangulated surface with  vertices. The blue "*" is the ground truth and the 
red "o" is the results computed by different methods; 
The second row plots the conformal distortion  of the planar embedding computed by 
different methods from the triangulated surface with  vertices. 
Note that for the purpose of comparison, the range of color map
is fixed as , although the maximal conformal distortion of the planar embedding 
by HF and BD is larger than .
\label{fig:hexagon}}
\end{figure}

\begin{table}[!h]
\begin{center}
\begin{tabular}{| c | c | c | c | c |}
\hline
Method  & 1000 &  4000 & 16000 & 64000   \\
\hline
{\bf DC} & (0.0257, 0.0486) & (0.0133, 0.0266) & (0.0067, 0.0141) & (0.0034, 0.0074) \\
\hline
{\bf HF} & (0.0275, 0.0538 ) & (0.0142, 0.0305) & (0.0070, 0.0154) & (0.0035, 0.0079)\\
\hline
{\bf BD}& (0.0273, 0.0524) & (0.0137, 0.0280) & (0.0069, 0.0153) & (0.0035, 0.0081)\\
\hline
\multicolumn{5}{|c|}{ } \\
\hline
{\bf DC} & (0.0650, 0.2000) & (0.0326, 0.1088) & (0.0163, 0.0490) & (0.0081, 0.0259) \\
\hline
{\bf HF} & (0.0883, 0.2238 ) & (0.0418, 0.1405) & (0.0203, 0.1386) & (0.0100, 0.1379)\\
\hline
{\bf BD}& (0.1333, 0.2028 ) & (0.0781, 0.1278) & (0.0445, 0.0690) & (0.0277, 0.0344)\\
\hline
\multicolumn{5}{|c|}{ } \\
\hline
{\bf DC} & 0.096 & 0.584 & 3.11 & 24.5 \\
\hline
{\bf HF} &  0.987 & 3.14 & 12.0 & 47.3 \\
\hline
{\bf BD}& 19.0 & 77.8  & 322 & 1821\\
\hline
\multicolumn{5}{|c|}{ timing (sec)} \\
\hline
\end{tabular}
\end{center}
\vspace{-0.1in}
\caption{Hexagonal Mesh: approximation errors and running time.
\label{tbl:hexagon}
}
\end{table}


We run the aforementioned methods over four continuously refined hexagonal triangulations of a planar region
of the side lengths , ,  and . The number of vertices in those triangulations are approximately
, ,  and . The first row in Figure~\ref{fig:hexagon_input} shows the input hexagonal triangulations. 
Note a circle packing in unit disk of a triangulation is not unique. We normalize the circle 
packing by choosing a vertex, denoted  for later reference, from the hexagonal triangulation and mapping it to the origin.
The remaining freedom is a rotation, which however doe not affect the consistency of the error estimations.
To make the normalization consistent across different hexagonal triangulations, the four chosen vertices  
(one from each triangulation) have the same coordinate. See the second row in Figure~\ref{fig:hexagon_input} for the
resulting circle packings of the hexagonal triangulations. 

For the methods of DC and BD, we again choose three vertices  
on the boundary so that they are mapped to the vertices of an equilateral triangle, and then
we map the equilateral triangle onto unit disk by the inverse of the Schwarz-Christoffel mapping
and finally, we employ an automorphism of unit disk to obtain the map  with  and 
. For the method of HF, we also apply an automorphism 
of unit disk to the computed embedding to obtain the same alignment.

Similarly, since the convergence is often stated for a compact region away from the boundary,
we estimate the errors  and  over the vertices which are more than  away 
from the boundary of the planar region, and the errors  and  over the polygons 
or triangles with their vertices satisfying the same requirement. The first row of Figure~\ref{fig:hexagon} 
shows the embedding of the vertices computed by different methods, and the second row of Figure~\ref{fig:hexagon} 
plots the conformal distortion of the approximated Riemann mapping by different methods.
In Table~\ref{tbl:hexagon}, we show the approximation errors: , , , , and 
the timing in seconds used by different conformal flattening methods for computing the embedding. 
From Table~\ref{tbl:hexagon}, a similar pattern as Spherical Cap is observed: 
all of the methods show linear convergence in the errors , , , and 
DC remains converging linearly in the  error. In this example, BD becomes convergent linearly
in the  error. This may be due to the fact that the triangles are all well-shaped in the 
hexagonal triangulations. 


\vspace{0.1in}
\noindent{\bf Planar Region.}
The main purpose of this example is to show how the quality of the input triangulation affects 
the conformality. We generate a triangulation with  vertices of a planar region as shown on 
the left in Figure~\ref{fig:plane_triangulation_input}, and then subdivide the triangulation three times 
by adding the midpoints of the edges and splitting each triangle into four smaller ones, and finally
obtain three more continuously refined triangulations of the planar region with approximately 
,  and  vertices. The right picture in Figure~\ref{fig:plane_triangulation_input}
shows the one with about  vertices. There are a few triangles, in particular near the boundary, having 
the largest angle close to .
\begin{figure}[!t]
\begin{center}
\begin{tabular}{cc}
\includegraphics[width=0.4\textwidth]{./pics/plane_triangulation_1500.png} & 
\includegraphics[width=0.4\textwidth]{./pics/plane_triangulation_5000.png} \\
\end{tabular}
\end{center}
\vspace{-0.1in}
\caption{The input triangulations of Planar Region with   vertices (Left) 
and  vertices (Right). 
\label{fig:plane_triangulation_input}}
\end{figure}

\begin{figure}[t]
\begin{center}
\begin{tabular}{ccc}
\includegraphics[width=0.33\textwidth]{./pics/convergence_map_1500_1.jpg} & 
\includegraphics[width=0.33\textwidth]{./pics/riemann_holo_map_1500_1.jpg} &
\includegraphics[width=0.33\textwidth]{./pics/bc_yaron_map_1500_1.jpg} \\
{\bf DC} & {\bf HF} & {\bf BD}
\end{tabular}
\end{center}
\vspace{-0.1in}
\caption{
Results for Planar Region: the plots the conformal distortion  of the planar embedding computed by 
different methods from the triangulation with  vertices. 
For the purpose of comparison, the range of color map is fixed as , 
although the maximal conformal distortion of the planar embeddings computed 
by HF and BD is larger than . 
\label{fig:plane_triangulation}}
\end{figure}

\begin{table}[!h]
\begin{center}
\begin{tabular}{| c | c | c | c | c |}
\hline
Method  & 1500 &  5000 & 20000 & 80000   \\
\hline
{\bf DC} & (0.0553, 0.2692) & (0.0286, 0.1292) & (0.0144, 0.0738) & (0.0072, 0.0401) \\
\hline
{\bf HF} & (0.0881, 0.9840 ) & (0.0473, 0.5649) & (0.0210, 0.5558) & (0.0093, 0.5570)\\
\hline
{\bf BD}& (0.1333, 1.307) & (0.0781, 1.285) & (0.0445, 1.482) & (0.0277, 1.324)\\
\hline
\multicolumn{5}{|c|}{ } \\
\hline
{\bf DC} & 0.096 & 0.584 & 3.11 & 24.5 \\
\hline
{\bf HF} &  0.987 & 3.14 & 12.0 & 47.3 \\
\hline
{\bf BD}& 74.2 & 227  & 940 & 4204\\
\hline
\multicolumn{5}{|c|}{ timing (sec)} \\
\hline
\end{tabular}
\end{center}
\vspace{-0.1in}
\caption{Planar Region: Approximation errors and running time.
\label{tbl:plane_triangulation}
}
\end{table}


We run the aforementioned methods over these triangulations.  
The methods of DC and BD map them onto an equilateral triangle while
the method HF map them onto unit disk. In this example, we do not have 
the ground truth and thus only estimate  and  errors as 
shown in Table~\ref{tbl:plane_triangulation}. Note the errors are estimated over the 
polygons or triangles with their vertices more than  of the diameter
of the planar region away from the boundary.
Figure~\ref{fig:plane_triangulation} shows the conformal distortion 
by different methods from the triangulation with  vertices.
As we can see, the method of DC converges linearly
in both the  and  errors, and the methods of HF and BD only converge 
in the  error. In the embedding computed by HF, there are some triangles close to 
the boundary whose orientations get reversed. 

\begin{figure}[t]
\begin{center}
\begin{tabular}{cc}
\includegraphics[width=0.3\textwidth]{./pics/left_hand_800.png} & 
\includegraphics[width=0.3\textwidth]{./pics/left_hand_10000.png} \\
\end{tabular}
\end{center}
\vspace{-0.1in}
\caption{The input triangulation of Left Hand with  vertices (Left) and  vertices (Right). 
\label{fig:left_hand_input}}
\end{figure}


\begin{figure}[t]
\begin{center}
\begin{tabular}{ccc}
\includegraphics[width=0.36\textwidth]{./pics/convergence_map_10000_1.jpg} & 
\includegraphics[width=0.28\textwidth]{./pics/riemann_holo_map_10000_1.jpg} &
\includegraphics[width=0.36\textwidth]{./pics/bc_yaron_map_10000_1.jpg} \\
{\bf DC} & {\bf HF} & {\bf BD}
\end{tabular}
\end{center}
\vspace{-0.1in}
\caption{Results for Left Hand: the plots the conformal distortion  of the 
planar embedding computed by different methods from the triangulation with  
vertices. Note that for the purpose of comparison, the range of color map is fixed as , 
although the maximal conformal distortion of the planar embeddings computed 
by HF and BD is larger than .
\label{fig:left_hand}}
\end{figure}


\begin{table}[!h]
\begin{center}
\begin{tabular}{| c | c | c | c | c |}
\hline
Method  & 800 &  2500 & 10000 & 40000   \\
\hline
{\bf DC} & (0.7436, 2.1873) & (0.3404, 1.1705) & (0.1457, 0.4359) & (0.0686, 0.2651) \\ \hline
{\bf HF} & () & (0.4247, 4.0458) & (0.1545, 1.3129) & (0.0716, 2.0045)\\
\hline
{\bf BD}& (1.1520, 6.5863) & (0.7330, 4.9571) & (0.3930, 3.9686) & (0.2873, 5.2264)\\
\hline
\multicolumn{5}{|c|}{ } \\
\hline
{\bf DC} & 0.236 & 1.27 & 5.84 & 42.3 \\
\hline
{\bf HF} &  0.966 & 2.24& 8.52 & 37.4 \\
\hline
{\bf BD}& 38.0  &  116 & 475 & 1941\\
\hline
\multicolumn{5}{|c|}{ timing (sec)} \\
\hline
\end{tabular}
\end{center}
\vspace{-0.1in}
\caption{Left Hand: approximation errors and running time.
\label{tbl:left_hand}
}
\end{table}

\vspace{0.3in}
\noindent{\bf Left Hand.}
The model Left Hand is obtained using 3D scanning. The original model has 200k vertices, which 
is simplified using Meshlab~\cite{meshlab} to the triangulated surfaces with , 
,  and  vertices. See Figure~\ref{fig:left_hand_input} for two of them. 
Again, the methods of DC and BD map these triangulated surfaces onto an equilateral triangle 
and the method HF map them onto unit disk. We estimate the  and  errors as 
shown in Table~\ref{tbl:left_hand}. Note the errors are estimated over the 
polygons or triangles with their vertices more than  of the diameter
of the planar region away from the boundary. Figure~\ref{fig:left_hand} plots the conformal
distortions of the planar embedding computed by different methods. For a better visualization, 
in this example, we show the plots over the planar embedding. 

Again the method of DC converges linearly in both the  and  errors, and the methods 
of HF and BD only converge in the  error. In the planar embedding of Left Hand with  
vertices computed by the method HF, there are some triangles even away from the boundary whose orientations 
get reversed. This is the reason that the corresponding  and  errors are 
in this case.






\vspace{0.1in}
\noindent{\bf Eight.}
Finally, we check the convergence for different methods over a model called Eight, which is a 
surface with genus . We use Loop subdivision to subdivide a triangulated Eight with about  
vertices to obtain four more refined triangulated Eight with about , ,  and  vertices. 
Figure~\ref{fig:eight_input} shows two of them. 

As the implementation of the method BD for surfaces of non disk topology is not available, we only 
show the performance of the methods of DC and HF. 
We estimate the  and  errors as shown in Table~\ref{tbl:eight}. Note the errors are
estimated over the polygons or triangles with their vertices more than one twentieth of the diameter
of Eight away from the singular vertices. 
Again, the method of DC converges linearly in both  and  errors. 
For the method of HF, the  error decreases but the convergence rate is not clear, 
and the  error does not even decrease. 
In this case, the planar embedding is in fact just an immersion and not necessary globally one-to-one. 
To visualize the conformal distortion, we plot it on the input triangulated surface as shown in Figure~\ref{fig:eight}.



\subsection{More examples and Statistics}
In this subsection, we present a few more examples and collect a few statistics
showing the performance of our algorithm. 

We run our DC method on four more examples: Maxplanck (a disk), 
Brain (a sphere), Protein (a torus), and Genus3 (a 3-hole torus). 
The results are shown in Figure~\ref{fig:more_examples}.  
The  and   errors are estimated over the polygonal faces
whose vertices are more than one sixtieth of the diameter of the 
model away from either the boundary or the singular vertices.
Note there is no singular points for Protein since its Euler characteristic
number is . For the model Genus3, we only choose one singular vertex.





In Table~\ref{tbl:statistics}, we collect the following statistics when the algorithm
runs over various examples: (1) the number of triangles in the input triangulated 
surface, labeled by {\it \#Fin}, representing the input complexity; (2) the number of 
polygonal faces in the common refinement of , labeled by {\it \#Fout}, representing
the output complexity. For surfaces with boundary, we count half of the faces
in ; (3) the number of diagonal switches needed to transform the input triangulation
 to the Delaunay triangulation , labeled by {\it \#DelSW}; (4) the running time in second
of the above diagonal switches, labeled by {\it tDelSW}; (5) the number of co-circular diagonal switches performed
during the discrete conformal deformation, labeled by {\it \#CocSW}; (6) the running time in second
of the above co-circular diagonal switches, labeled by {\it tCocSW}; (7) the number of the Newton iterations, labeled by
{\it \#Newton}; (8) the running time in second of the Newton iterations excluding tCocSW, labeled by {\it tNewton}.
From Table~\ref{tbl:statistics}, we observe that the number of faces in  is often only a few hundred more
than that in , and the Newtons method converges very fast and takes only 5-10 iterations, 
and the operation of diagonal switch costs very little. Note that the procedure of cocircular diagonal 
switch takes more time as it requires to find the first edge failing the Delaunay condition along the 
deforming path. 

\begin{figure}[t]
\begin{center}
\begin{tabular}{cc}
\includegraphics[width=0.4\textwidth]{./pics/eight.png} & 
\includegraphics[width=0.4\textwidth]{./pics/eight_12000.png} \\
\end{tabular}
\end{center}
\vspace{-0.1in}
\caption{
The input triangulations of Eight with  vertices (Left) and  vertices (Right). 
\label{fig:eight_input}}
\end{figure}

\begin{figure}[t]
\begin{center}
\begin{tabular}{ccc}
\includegraphics[width=0.4\textwidth]{./pics/convergence_map_3000.jpg} & 
\includegraphics[width=0.4\textwidth]{./pics/holomorphic_map_3000.jpg} \\
{\bf DC} & {\bf HF} 
\end{tabular}
\end{center}
\vspace{-0.1in}
\caption{Results for Eight: the conformal distortion  of the 
planar embedding computed by DC and HF from the triangulation with  vertices
plotted on the input surface and shown in two different views. 
Note that for the purpose of comparison, the range of color map is fixed as , 
although the maximal conformal distortion of the planar embeddings computed 
by HF is larger than .
\label{fig:eight}}
\end{figure}

\begin{table}[!h]
\begin{center}
\begin{tabular}{| c | c | c | c | c |c|}
\hline
Method  & 750 &  3000 & 12000 & 50000 & 200000 \\
\hline
{\bf DC} & (0.1422, 1.5259) & (0.1112, 0.4187) & ( 0.0290, 0.2091) & (0.0184, 0.1071) & (0.0085, 0.0361)\\
\hline
{\bf HF} & (0.1133, 0.3985) & (0.1236, 3.0128) & (0.0828, 6.2417) & (0.0325, 4.7646) & (0.0343, 14.158)\\
\hline
\multicolumn{6}{|c|}{ } \\
\hline
{\bf DC} & 0.040 & 0.196 & 1.12 &5.88  & 59.1 \\
\hline
{\bf HF} & 1.58 & 4.91 & 19.2 & 80.6 & 339\\
\hline
\multicolumn{6}{|c|}{ timing (sec)} \\
\hline
\end{tabular}
\end{center}
\vspace{-0.1in}
\caption{Eight: Approximation errors and timing.
\label{tbl:eight}
}
\end{table}



\begin{table}[htbp]
\begin{center}
\begin{tabular}{| c | c | c | c | c | c|c|c|c|}
\hline
Model       & ~~\#Fin (k)~~ &  ~~\#Fout (k)~~ & ~~\#DelSW~~ & ~~tDelSW~~ &~~\#CocSW~~ & ~~tCocSW~~ & ~~\#Newton~~ & ~~tNewton~~   \\
\hline
{Planar Region} & 10.22	& 10.25   	  & 3223    & 0.072  & 14      & 0.272  & 5        & 0.34\\
\hline
{Planar Region} & 40.86	& 40.90   	  & 14113   & 0.38   & 17      & 1.30   & 5        & 2.01\\
\hline
{Planar Region} & 163.47	& 163.50   & 59288   & 1.50   & 18      & 5.41   & 5        & 13.53\\
\hline
{Left Hand} & 79.98	& 80.54   	  & 14192   & 0.52   & 272     & 30.6   & 7        & 4.97\\
\hline
{Eight} & 196.61	& 196.65   	  & 68428   & 0.34   & 44     & 13.84   & 5        & 43.1\\
\hline
{Maxplanck} & 47.08 		& 47.27   	  & 5969    & 0.248  & 93      & 4.53   & 6        & 2.22\\
\hline
{Brain} 	 	& 15.00	& 15.30   	  & 1048   & 0.016   & 300     & 4.6   & 6        & 1.05\\
\hline
{Protein} 	 	& 46.24	&  46.37  	  & 3064   & 0.072   & 132     & 8.07   & 9        & 5.34\\
\hline
{Genus3} 	 	& 26.62	&  26.74  	  & 12454   & 0.052   & 111     & 4.33   & 10        & 4.56\\
\hline
\end{tabular}
\end{center}
\caption{Statistics.
\label{tbl:statistics}
}
\end{table}

\begin{figure}[!t]
\begin{center}
\begin{tabular}{cc}
\includegraphics[width=0.2\textwidth]{./pics/maxplanck.png} & 
\includegraphics[width=0.38\textwidth]{./pics/maxplanck.pdf} \\
\includegraphics[width=0.28\textwidth]{./pics/brain_fix.png} & 
\includegraphics[width=0.38\textwidth]{./pics/brain_fix.pdf} \\
\includegraphics[width=0.4\textwidth]{./pics/protein_50k.png} & 
\includegraphics[width=0.38\textwidth]{./pics/protein_50k.pdf} \\
\includegraphics[width=0.35\textwidth]{./pics/genuss3_30k.png} & 
\includegraphics[width=0.39\textwidth]{./pics/genuss3_30k.pdf} \\
\end{tabular}
\end{center}
\caption{Left: the input triangulated surfaces; Right: the planar embedding 
computed by our algorithm. The color maps plot  and the pairs of numbers
are the  errors. 
\label{fig:more_examples}}
\end{figure}


\subsection{Remark}
From the above experiments, we observe that our conformality numerically converges
to the classical one as the triangle size goes to , in a linear rate. In particular, 
this convergence behavior is independent of the quality of the triangles in the
input triangulated surfaces. It is also very efficient comparing to the state of the art. 
One disadvantage of our method is that we may subdivide the input triangles into the
polygonal pieces to accurately represent the discrete conformal map. This may increase
the complexity of the output by our algorithm. However, from the statistics we show,
the increase of the complexity is very little for most of the prescribed curvatures. 


\section{Conclusion}
We have introduced a new discrete conformality for triangulated surfaces possibly
with boundary, and showed a discrete uniformization theorem with this conformality, 
and described an algorithm for solving the problem of prescribing curvature, and 
explicitly constructed a discrete conformal map between the input triangulated surface
and the deformed triangulated surface.  
In addition, we have presented the numerical examples to show the convergence of 
our discrete conformality and to demonstrate the efficiency and the robustness
of our algorithm.

We point out a few possible few directions for future research. In~\cite{gglsw}, we 
have presented a similar discrete conformality for hyperbolic triangulated surfaces and 
a discrete uniformization theorem associated to it. We
plan to develop an algorithm based on this discrete conformality to solve the problem of 
prescribing curvature for hyperbolic triangulated surfaces. Hyperbolic triangulation is 
more natural for surfaces with genus bigger than  as their fundamental domains can be 
flattened into hyperbolic plane without choosing any singular vertices.  
It remains open whether our discrete conformality and uniformization theorem can be extended
to spherical triangulated surfaces, which is definitely worth investigating in the future.  
Another interesting future research is to see if diagonal switches can be used in 
inversive distance circle patterns. 


\section{Acknowledgment} 
We would like to thank Yaron Lipman for the implementation of the method BD, 
and Xianfeng Gu and S-T. Yau for the implementation of the method HF. 
The work is supported in part by the NSF of USA and the NSF of China.

\bibliographystyle{abbrv}
\bibliography{discrete_conformality}

\vspace{8mm}
\noindent\textbf{Appendix: Proof of Theorem~\ref{theorem:dc_boundary}}

\begin{proof}
Denote  the doubled surface of  and  the doubled metric
of . Prescribe the curvature  for  by setting 
 for a vertex  on the boundary, and  for a vertex  in the interior.  
It is easy to verify that the curvature  satisfies the hypotheses in Theorem~\ref{thm:main} imposed 
on a prescribed curvature on . Thus there exists a PL metric  discrete
conformal to  and the discrete curvature of  is the curvature of .
We will show that  respects the doubling structure and the restriction of  onto
 is the PL metric  with the property stated in the theorem. 

We first show for a PL metric  on  respecting the doubling structure, 
there is a Delaunay triangulation  in  which has the following symmetric property: 
(1) Any triangle  in  not crossing the boundary has an identical mirror triangle  in ; 
(2) Let  be set of triangles in  crossing a segment . Then any triangle  
must have a pair of vertices  with , and moreover, if the third vertex  of  is not
the endpoints of the segment , the neighboring triangle  has the property that . 
Note that the quadrilateral  must be cocircular as the segment  is the common bisector 
of the edge  and . See Figure~\ref{fig:boundary_triangles} for an illustration. 

Since one can reach a Delaunay triangulation starting from any triangulation by diagonally
switching the edges which fail to be Delaunay finite many times, we can prove this by induction
on the number of diagonal switches. We start with the triangulation  on  
so that the restrictions of  onto both copies of  are identical triangulations. Note a segment 
 must be an edge in . Thus  is empty and the symmetric property trivially holds.
Assume by diagonally switching a set of edges which fails to be Delaunay, we reach a triangulation  
satisfying the symmetric property. Assume there is an edge  which fails to be Delaunay. If  
is not a side of any triangle in  for any segment , then its mirror  also fails 
to be Delaunay. Note that if the edge  itself is a segment on , then . Switch both  
and  and reach a triangulation  which satisfies the symmetric property. If  is a side of a 
triangle in  for some segment , there are two cases: (i)  crosses ; 
and (ii)  does not cross . In the case (i), the endpoints  of  must satisfy
 and any edge in the triangles incident to  which crosses  must also fail to be Delaunay.  
For example, as shown in Figure~\ref{fig:switch_boundary_edge}, the edges  and  must also fail 
to be Delaunay. Switch these edges and reach a triangulation  which satisfies the symmetric property. 
In the second case, switch both  and . If the endpoints of  contain no endpoints of the segment 
, as shown in Figure~\ref{fig:switch_boundary_edge}, switch the diagonal  as it must also fail to be Delaunay. 
The resulting triangulation  satisfies the symmetric property. This proves that there is a 
Delaunay triangulation  in  satisfying the above symmetric property. 

\begin{figure}[!t]
\begin{center}
\begin{tabular}{c}
\includegraphics[width=0.6\textwidth]{./pics/switch_boundary_edge.pdf}\\
\end{tabular}
\end{center}
\vspace{-0.1in}
\caption{Diagonal switches for the edges of the triangles crossing a segment on the boundary.
From left to right: the edges  always fail to be Delaunay at the same time. 
After their switches, the edges  fails to be Delaunay. From right to left: the edges  and  always fail to be Delaunay at the same time.  
\label{fig:switch_boundary_edge}}
\end{figure}

Let  with  be the conformal factor 
so that . We claim  respects the doubling structure, i.e.,  
for any vertex . Otherwise, let us define a new conformal factor  so that  
for any vertex , and then , which from Lemma~\ref{lem:conformalmetricspace} 
implies the metric  is different from  . However, it is easy to verify that the 
curvature of the metric  is also equal to .
This contradicts to the uniqueness of . 

Now let  for  be a path from  to , and we have  respects the doubling 
structure for any . As discussed in Section~\ref{sec:convexenergy},  for 
is a path in . Let  is a partition of  so that for any , 
 with  is a path inside the cell  for some triangulation .
If  respects the doubling structure and  satisfies the symmetric property in the metric , 
then  remains so in any PL metric  for any . Indeed, the symmetric 
property (1) obviously holds as  respects the doubling structure. To show the symmetric property (2), it suffices 
to show is that the quadrilateral  remains cocircular.  This can be done by verifying that the sum of 
the cosines of the angles opposite to the diagonal remains  along the path . 
Furthermore, consider the region , 
as shown in Figure~\ref{fig:boundary_triangles}. One can cut it into two geometrically identical subregions using a 
straight line connecting the endpoints of the segment  and passing through the midpoints of the edges
in  of the form  with . This shows that  respects the doubling structure for 
any , in particular, so is . 
Now by construction,  respects the doubling structure.  From the previous discussion, 
lies in the cell  where  satisfies the symmetric property. 
Then using induction, we show that  respects the double structure. 
The restriction of  onto  is the PL metric  on . Finally, it is easy to verify 
that the curvature on  in  equals . This proves the theorem.  
\end{proof}





\end{document}
