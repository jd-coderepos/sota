\documentclass[11pt]{article}
\usepackage{hyperref}
\usepackage{amsmath,amsthm,amssymb}

\usepackage[small,bf]{caption}

\usepackage{graphicx}

\hypersetup{colorlinks=true, linkcolor=black, citecolor=black,pdftitle={Tight Bounds on Information Dissemination in Sparse Mobile Networks},pdfauthor={Alberto Pettarin, Andrea Pietracaprina, Geppino Pucci, and Eli Upfal}}

\newcommand{\newterm}[1]{\emph{#1}}

\newcommand{\bigO}[1]{O\left(#1\right)}
\newcommand{\bigOt}[1]{\tilde{O}\left(#1\right)}
\newcommand{\bigTh}[1]{\Theta\left(#1\right)}
\newcommand{\bigTht}[1]{\tilde{\Theta}\left(#1\right)}
\newcommand{\bigOm}[1]{\Omega\left(#1\right)}
\newcommand{\bigOmt}[1]{\tilde{\Omega}\left(#1\right)}
\newcommand{\card}[1]{\left|#1\right|}
\newcommand{\abs}[1]{\left|#1\right|}
\newcommand{\dist}[1]{||#1||}
\newcommand{\prob}[1]{\mathrm{Pr}\left({#1}\right)}
\newcommand{\expe}[1]{\mathrm{E}\left[{#1}\right]}
\newcommand{\vari}[1]{\mathrm{Var}\left({#1}\right)}

\newcommand{\seq}[4]{#1_{#2}, #1_{#3}, \ldots, #1_{#4}}
\newcommand{\set}[4]{\left\{\, \seq{#1}{#2}{#3}{#4}\,\right\}}
\newcommand{\lis}[1]{\left\langle #1 \right\rangle}

\newcommand{\sqrtn}{\sqrt{n}}

\newcommand{\isla}[3]{I_{#3}(#1; #2)}
\newcommand{\is}{\isla{\point{x}}{\beta}{t}}
\newcommand{\isdy}{\isla{\point{x_0}}{4\beta\sqrt{\log n}}{t_0}}

\newcommand{\bs}{Broadcasting scenario}
\newcommand{\gs}{Gossiping scenario}
\newcommand{\bt}{T_{\mathrm{B}}}
\newcommand{\gt}{T_{\mathrm{G}}}
\newcommand{\ct}{T_{\mathrm{C}}}
\newcommand{\et}{T_{\mathrm{E}}}

\newcommand{\B}{\mathcal{B}}
\newcommand{\I}{\mathcal{I}}
\newcommand{\E}{\mathcal{E}}
\newcommand{\R}{\mathbb{R}}
\newcommand{\Z}{\mathbb{Z}}
\newcommand{\N}{\mathbb{N}}
\newcommand{\grid}{\mathcal{G}_n}
\newcommand{\point}[1]{#1} \newcommand{\pos}[2]{x_{#1}(#2)} \newcommand{\mes}[2]{M_{#1}(#2)} \newcommand{\meet}[2]{\textrm{Meet}_{#1}(#2)}
\newcommand{\tf}[1]{\small{#1}}

\newtheorem{defi}{Definition}
\newtheorem{theo}{Theorem}
\newtheorem{lemm}{Lemma}
\newtheorem{coro}{Corollary}
\newtheorem{lemmapp}{Lemma}[section]
\newenvironment{proofof}[1]{\begin{trivlist} 
                         \item[] {\it Proof of #1:}}{\hfill 
                       \end{trivlist}} 

\newcommand{\alb}[1]{\textbf{ALB: #1}}

\begin{document}
\title{Tight Bounds on Information Dissemination\\ in Sparse Mobile Networks\thanks{Support for the first three authors was provided, in part, by MIUR of Italy
	under project AlgoDEEP, and by the University of
	Padova under the Strategic Project STPD08JA32 and Project
	CPDA099949/09.
	This work was done while the first author was visiting the Department of Computer Science of Brown University,
	partially supported by ``Fondazione Ing.~Aldo Gini'', Padova, Italy.}}

\author{
Alberto Pettarin\and Andrea Pietracaprina\and Geppino Pucci\\
	\tf{Department of Information Engineering,} \tf{University of Padova}\\
	\tf{\texttt{\{pettarin,capri,geppo\}dei.unipd.it}}
\and Eli Upfal\\
	\tf{Department of Computer Science,} \tf{Brown University}\\
	\tf{\texttt{elics.brown.edu}}
}

\date{}
\maketitle{}

\begin{abstract}
Motivated by the growing interest in mobile systems,
we study the dynamics of information dissemination
between agents moving independently on a plane.
Formally, we consider  mobile agents
performing independent random walks on an -node grid.
At time , each agent is located
at a random node of the grid and one agent has a rumor.
The spread of the rumor is governed by
a dynamic communication graph process ,
where two agents are connected by an edge in  iff
their distance at time  is within their transmission radius .
Modeling the physical reality
that the speed of radio transmission
is much faster than the motion of the agents,
we assume that the rumor can travel
throughout a connected component of 
before the graph is altered by the motion.
We study the \emph{broadcast time}  of the system,
which is the time it takes for all agents to know the rumor.
We focus on the sparse case
(below the percolation point )
where, with high probability,
no connected component in  has more than
a logarithmic number of agents
and the broadcast time is dominated by the time
it takes for many independent random walks
to meet each other.
Quite surprisingly,
we show that for a system below the percolation point
the broadcast time does not depend on the relation between
the mobility speed and the transmission radius.
In fact, we prove that

for any ,
even when the transmission range is significantly larger
than the mobility range in one step,
giving a tight characterization up to logarithmic factors.
Our result complements a recent result of
Peres et al.~(SODA 2011)
who showed that above the percolation point
the broadcast time is polylogarithmic in .
\end{abstract}





\section{Introduction}

The emergence of mobile computing devices has added a new intriguing
component, \emph{mobility}, to the study of distributed systems.
In  fully mobile systems, such as wireless mobile ad-hoc networks (MANETs),
information is generated, transmitted and consumed within the mobile
nodes, and communication is carried out without the support of 
static structures such as cell towers.  These systems have been
implemented in vehicular networks and sensor networks attached to
soldiers on a battlefield or animals in a nature
reserve~\cite{OlariuW09, Gerla05, JuangOWMPR02, Stojmenovic02}.
Characterizing the power and limitations of mobile networks requires
new models and analytical tools that address the unique properties
of these systems~\cite{GrossglauserT02,ClementiPS09}, which include:
\begin{itemize}
\item
\emph{Small transmission radius}: the transmission range of individual agents
is restricted by limitations on energy consumption and interference
from other agents;
\item
\emph{Planarity}: agents reside, move and transmit on (subsets of) a
plane.  Low diameter graphs that are often used to model static
communication networks are not useful here;
\item
\emph{Dynamic communication graphs}: communication channels between
agents are changing dynamically as mobile agents move in and out of
the transmission radius of other agents;
\item
\emph{Relative speeds}: transmission speed is significantly faster
than the physical movement of the agents.  A message can execute several hops
before the network is altered by motion.
\end{itemize}

In this work we study the dynamics of information dissemination
between agents moving independently on a plane.  We consider a system
of  mobile agents performing independent random walks on an
-node grid, starting at time  in a uniform distribution over the
grid nodes.  We focus on the fundamental communication primitive of
broadcasting a rumor originating at one arbitrary agent to all other 
agents in the system.  We characterize the \emph{broadcast time} 
of the system, which is the time it takes for all agents to receive
the rumor.


We model the spreading of information in a mobile system by a dynamic
communication graph process , where the nodes of
 are the mobile agents, and two agents are connected by an
edge iff their distance at time  is within their transmission
radius .  We are interested in \newterm{sparse systems} in which
the transmission radius is below the percolation point  \cite{Penrose03,PeresSSS11} (i.e., the minimum radius
which guarantees that  features a giant connected
component), and where, with high probability, no connected component
of  has more than a logarithmic number of agents. The
broadcast time in sparse systems is dominated by the time it takes for
many independent random walks to meet one another.  Incorporating the
fact that radio transmission is much faster than the motion of the
agents, we assume that the rumor can travel throughout a connected
component of  within one step, before the graph is altered by the
motion.


Our main result is quite surprising: we show that below the
percolation point the broadcast time does not depend on the relation
between the mobility speed and the transmission radius. We prove that
 for any  below , giving a tight
characterization up to logarithmic factors\footnote{The tilde notation hides polylogarithmic factors,
  e.g.  for some constant .}.
Our bound holds both when the transmission radius is significantly
larger than the mobility range (i.e., the distance an agent can travel
in one step), and when, in contrast to previous
work~\cite{ClementiMPS09, ClementiPS09}, the transmission radius as
well as the the mobility range are very small.  Our work complements a
recent result by Peres et al.~\cite{PeresSSS11} who proved an upper
bound polylogarithmic in  for the broadcast time in a system of 
mobile agents which follow independent Brownian motions in , with
transmission radius above the percolation point.


Our analysis techniques are applicable to a number of interesting
related problems such as covering the grid with many random walks
and bounding the extinction time in random predator-prey systems.

\subsection{Related Work}

Information dissemination has been extensively studied in the
literature under a variety of scenarios and 
objectives. Due to space limitations, we restrict our attention to
the results more directly related to our work.

A prolific line of research has addressed broadcasting and gossiping
in static graphs, where the nodes of the graph represent active
entities which exchange messages along incident edges according to
specific protocols (e.g., \emph{push}, \emph{pull}, \emph{push-pull}).
The most recent results in this area relate the performance of the
protocols to expansion properties of the underlying topology, with
particular attention to the case of social networks, where
broadcasting is often referred to as \emph{rumor spreading}
\cite{ChierichettiLP10}.  (For a relatively recent, comprehensive
survey on this subject, see~\cite{HromkovicKPRU05}.)

Unfortunately, mobile networks do not feature
properties similar to those of social networks,
mostly because of the physical limitations
of both the movement and the radio transmission processes.
Indeed, as noted in~\cite{Kleinberg07},
the short range of communication attainable
by low-power antennas enforces the same local dynamics
that are typical of disease epidemics~\cite{Durrett99}
which requires physical proximity to propagate.
Indeed, the analysis of opportunistic networks,
where nodes relay messages as they come
close one to another, apply models 
from the study of human mobility~\cite{ChaintreauHCDGS07, Chaintreau08}.
Similarly, in the theory community there has been growing
interest in modeling and analyzing information dissemination in dynamic scenarios,
where a number of agents move either in a continuous space or along
the nodes of some underlying graph and exchange information when their
positions satisfy a specified proximity constraint.

In~\cite{ClementiMPS09,ClementiPS09} the authors study the time it takes
to broadcast information from one of  mobile agents to all others.
The agents move on a square grid of  nodes and in each time step,
an agent can (a) exchange information with all agents at distance at
most  from it, and (b) move to any random node at distance at most
 from its current position. The results in these papers only
apply to a very dense scenario where the number of agents is linear in
the number of grid nodes (i.e., ).  They show that the
broadcast time is  w.h.p., when 
and  \cite{ClementiMPS09}, and it is
 w.h.p., when  \cite{ClementiPS09}.  These results
crucially rely on , which implies that
the range of agents' communications or movements at each step defines
a connected graph.

In more realistic scenarios, like the one adopted in this paper, the
number of agents is decoupled from the number of locations (i.e., the
graph nodes) and a smoother dynamics is enforced by limiting agents to
move only between neighboring nodes. A reasonable model consists of
a set of multiple, simple random walks on a graph, one for each agent,
with communication between two agents occurring when they meet at the
same node.  One variant of this setting is the so-called \emph{Frog
  Model},
where initially one of  agents is active (i.e., is performing a
random walk), while the remaining agents do not move.  Whenever an
active agent hits an inactive one, the latter is activated and starts
its own random walk.  This model was mostly studied in the infinite
grid focusing on the asymptotic (in time) shape of the set of vertices
containing all active agents~\cite{AlvesMP02, KestenS03}.

A model similar to our scenario is often employed to represent the
spreading of computer viruses in networks and the spreading time is
also referred to as \emph{infection time}.  Kesten and
Sidoravicius~\cite{KestenS05} characterized the rate at which an
infection spreads among particles
performing continuous-time random walks with the same jump
rate.  In \cite{DimitriouNS06}, the authors provide a general bound on
the average infection time when  agents (one of them initially
affected by the virus) move in an -node graph.  For general graphs,
this bound is , where  denotes the maximum
average meeting time of two random walks on the graph, and the maximum
is taken over all pairs of starting locations of the random walks.
Also, in the paper tighter bounds are provided for the complete graph
and for expanders. Observe that the  bound
specializes to  for the -node grid by
applying the known bound on  of \cite{AldousF98}.  A tight bound
of  on the infection time on the grid is
claimed in \cite{WangKK08}, based on a rather informal argument where
some unwarranted independence assumptions are made.  Our results show
that this latter bound is incorrect.

Recent work by Peres et al.~\cite{PeresSSS11} studies a process in
which agents follow independent Brownian motions in .  They
investigate several properties of the system, such as detection,
coverage and percolation times, and characterize them as functions of
the spatial density of the agents, which is assumed to be greater than
the percolation point.  Leveraging on these results, they show that
the broadcast time of a message is polylogarithmic in the number of
agents, under the assumption that a message spreads within a connected
component of the communication graph instantaneously, before the graph
is altered by agents' motion.




\subsection{Organization of the Paper}
The rest of the paper is organized as follows.  In
Section~\ref{sec:prelim}, we define the quantities of interest and
establish some technical facts which are used in the analysis.
Section~\ref{sec:gs} contains our main results: first, we prove the
upper bound on the broadcast time in the most restricted case, that
is, when the information exchange occurs through physical contact of
the agents (i.e., ), and then we provide a matching lower bound,
which holds for every value of the transmission radius  below the
percolation point.  Finally, in Section~\ref{sec:conclusions} we
briefly discuss the connection between our result and other
interesting related problems and devise some future research directions.


\section{Preliminaries}
\label{sec:prelim}
In this paper, we study the dynamics of information exchange among a
set  of  agents performing independent random walks on an
-node 2-dimensional square grid ,
which is commonly adopted as a discrete model
for the domain where agents wander.
We assume that ,
since sparse scenarios are the most interesting
from the point of view of applications;
however, our analysis can be easily extended to denser scenarios.
We suppose that the agents
are initially placed uniformly and independently at random on the grid
nodes. Time is discrete and agent moves are synchronized.  At each
step an agent residing on a node  with  neighbors (), moves to any such neighbor with probability  and
stays on  with probability . With these probabilities it
is easy to see that at any time step the agents are placed
uniformly and independently at random on the grid nodes.
The following two lemmas contain important properties
of random walks on , which will be employed for deriving our 
results\footnote{Throughout the paper, the distance between two
grid nodes  and , denoted by , is defined to be the
Manhattan distance.}. 
\begin{lemm}
\label{lemm:SRW}
Consider a random walk on , starting at time  at node .
There exists a positive constant  such that
for any node ,

\end{lemm}
\begin{proof}
The Lemma is proved in \cite[Theorem~2.2]{AlvesMP02} for the infinite
grid .  By the ``Reflection Principle''~\cite[Page 72]{Feller68},
for each walk in  that started in
, crossed a boundary and then crossed the boundary back to ,
there is a walk with the same probability that does not cross the
boundary and visits all the nodes in  that were visited by the
first walk.  Thus, restricting the walks to  can only change the
bound by a constant factor.
\end{proof}

\begin{lemm}
\label{lemm:props}
Consider the first  steps of a random walk in  which was at
node  at time .
\begin{enumerate}
\item\label{poin:dev} 
The probability that at any given step  the random
walk is at distance at least  from  is
at most .
\item\label{poin:range} 
There is a constant  such that, with probability greater than
, by time  the walk has visited at least  distinct nodes in .
\end{enumerate}
\end{lemm}
\begin{proof}
We observe that the distance from  in each coordinate defines a
martingale with bounded difference .  Then, the first property
follows from the Azuma-Hoeffding Inequality~\cite[Theorem 2.6]{MitzenmacherU05}.
As for the second property, let  be
the set of nodes reached by the walk in  steps.  By
Lemma~\ref{lemm:SRW},  (even
when  is near a boundary), while

(see~\cite{Torney86}). The result follows by applying Chebyshev's
inequality.
\end{proof}

Let  be a set of messages, which will be referred to as
\newterm{rumors} henceforth, such that for each  there is (at
least) one agent \newterm{informed} of  at time .  W.l.o.g.,
we can assume that the number of distinct rumors is at most equal to
the number of agent.  We denote by  the set of rumors that
agent  is informed of at time , for any ;
possibly, .  We assume that each agent is
equipped with a \newterm{transmission radius} , representing
the maximum distance at which the agent can send information in a
single time step.

The spread of rumors can be represented by a dynamic communication
graph process \sloppy , where , the
\newterm{visibility graph at time }, is a graph with vertex set 
and such that there is an edge between two vertices iff the
corresponding agents are within distance  at time .
Following a
common assumption justified by the physical reality that the speed of
radio transmission is much faster than the motion of the
agents~\cite{PeresSSS11}, we suppose that rumors can travel throughout
a connected component of  before the graph is altered by the
motion.  We assume that within the same connected component agents
exchange all rumors they are informed of.  Formally, let  be a
connected component of : for all , .  Note that the sets 
can only grow over time, that is, agents do not ``forget'' rumors.
The following quantities will be studied in this paper.
\begin{defi}[Broadcast Time, Gossip Time]
The \newterm{broadcast time}  of a rumor  is the
first time at which every agent is informed of , that is, for all
 and , .  The
\newterm{gossip time}  of the system is the first time at
which every agent is informed of every rumor, that is, for any  and , .
\end{defi}

Note that both  and  depend on the transmission radius
, but we will omit this dependence to simplify the notation.
We will also write  instead of  when the message 
is clearly identified by the context.


\section{Broadcasting Below the Percolation Point}
\label{sec:gs}

In this section we give bounds to the broadcast time  of a
rumor when the transmission radius is below the percolation point , that is, when all the connected components of
 comprise at most a logarithmic number of agents.  In this
regime, we show that quite surprisingly  does not depend on the
relation between the mobility speed and the transmission radius, the
reason being that the broadcast time is dominated by the time it takes
for many independent random walks to intersect one another.  In
Subsection~\ref{sec:gsub} we prove an upper bound on the broadcast
time  in the extreme case , that is, when agents can
exchange information only when they meet on a grid node. The same
upper bound clearly holds for any other . Then, in
Subsection~\ref{sec:gslb} we show that the upper bound is tight,
within logarithmic factors, for all values of the transmission radius
below the percolation point.  We also argue that the bounds on 
easily extend to gossip time .

\subsection{Upper Bound on }
\label{sec:gsub}

The main technical ingredient of the analysis carried out in this
subsection is the following lower bound on the probability that two
random walks  on the grid meet within a given time
interval and not too far from their starting positions, which is a
result of independent interest.  Observe that considering the
difference random walk  and computing the
probability that it hits the origin in the prescribed number of steps
does not provide any information about the place where the meeting
occurs, hence it is not immediate to derive our result through that
approach.

\begin{lemm}
\label{lemm:MeetingProbability}
Consider two independent simple random walks on the grid
, and ,
where  and  denote the locations of the walks at time .
Let  and define  to be the set of nodes at
distance at most  from both  and .
For , there exists a constant  such that

\end{lemm}
\begin{proof}
The case  is immediate. Consider now the case .  Let  denote the probability that a walk that started at node  at
time 0 is at node  at time , and let  be
the expected number of times that two walks which started at nodes 
and  at time 0 meet at nodes of  during the time
interval , then

Let  be the first meeting time of the walks  and
 at a node of . Then

Thus, 

It is easy to verify that .
Applying Theorem 1.2.1 in \cite{Lawler91} we have:

By bounding  and  from above with 
in the formula, easy calculations show that .  Similarly, using the fact that there are no more than
 nodes at distance exactly  from , we have:


We conclude that there is a constant  such that
.\qedhere
\end{proof}

The reminder of this section is devoted to proving the following upper
bound on the broadcast time of a single rumor  in the case .
We assume that   for some ,
and  for any other .
\begin{theo}
\label{theo:UBSpreadingTime2}
Let . For any , with probability at least ,

\end{theo}

We observe that since the diameter of  is ,
we can use Lemma~\ref{lemm:MeetingProbability} to show that
with probability at least , at time 
an agent  has met all other agents walking   in .
Thus, the theorem trivially holds for .

From now on we concentrate on the case .
	\iffalse
	Albeit our reference scenario is defined with respect to a fixed number 
	of agents (the \newterm{exact model}), technically, it is easier to
	derive an upper bound on  using a slightly modified model in
	which, initially, each grid node  holds  agents, where 
	is binomial random variable with distribution , and the
	's are mutually independent (the \newterm{binomial model}). We
	refer to  as the \newterm{density} of the binomial
	model, and let  be the random variable
	denoting the number of agents in a given instance of the model.  For
	sufficiently large density, namely , a standard
	argument based on Chernoff bound, shows that ,
	with high probability.  Then, by
	\cite[Corollary~5.9]{MitzenmacherU05}, a high-probability result for
	the binomial model with  implies a similar
	high-probability result in the exact model with  agents.
	
	The argument proceeds as follows.
	\fi
We tessellate  into
\newterm{cells} of side ,
where  is defined in Lemma~\ref{lemm:MeetingProbability}.
We say that a cell  is
\newterm{reached} at time  if  is the first time when a node
of the cell hosts an agent informed of the rumor and we call this first
visitor the \newterm{explorer} of .
We first show that, after a suitably
chosen number  of steps past ,
there is a large number of informed agents 
within distance  from .
Furthermore, we show that while the rumor spreads to cells
adjacent to , at any time  a large number of
informed agents are at locations close to .
These facts will imply that
the exploration process proceeds smoothly and that all agents are
informed of the rumor shortly after all cells are reached.

The above argument is made rigorous in the following sequence of
lemmas.
\begin{lemm}
\label{lemm:FirstPhase}
Consider an arbitrary  cell  of the tessellation.
Let  and
,
where  and .
By time , at least  agents are informed and are
at distance at most  from ,
with probability , for sufficiently large .
\end{lemm}
\begin{proof}
Since at any given time the agents are at random and independent
locations, by the Chernoff bound we have that the following
\emph{density condition} holds with probability at least ,
for sufficiently large :
for any cell  and any time instant ,
the number of agents residing in cell  at time 
is at least .
In the rest of the proof, we assume that the density condition holds.

First, we prove that, by time , there are at least  informed agents in the system. We assume that at every time
step  there is always an uninformed agent in the
same cell where the explorer resides (otherwise the sought property
follows immediately by the density condition). For , let  be the time at which the explorer
of  informs the -th agent.  For notational convenience, we let
.  To upper bound , for , we consider a
sequence of  consecutive, non-overlapping time
intervals of length  beginning from time .  By the
previous assumption, at the beginning of each interval the cell where
the explorer resides contains an uninformed agent .  Hence, by
Lemma~\ref{lemm:MeetingProbability}, the probability that the explorer
fails to meet an uninformed agent during all of these intervals is

where the last inequality holds for sufficiently large 
by our choice of .
By iterating the argument
for every , we conclude that with probability at least ,
there are at least  informed agents at time
.  Let  denote the set of informed agents identified
through the above argument, and observe that each agent of  was in
the cell containing the explorer at some time step .

To conclude the proof of the lemma, we note that, by
Lemma~\ref{lemm:props}, the probability that
the explorer, during the interval ,
reaches a grid node at distance greater than 
from its position at time  is bounded by .
Consider an arbitrary agent . As observed above,
there must have been a time instant  when
 and the explorer were in the same cell, hence at distance at most
 from .  From time  until
time  the random walk of agent  proceeds independently for
the random walk of the explorer.  By applying again
Lemma~\ref{lemm:props}, we can conclude that the probability that one
of the agents of  is at distance greater than 
from  at time  is at most .
By adding up the upper bounds to the probabilities that the event stated
in the lemma does not hold, we get
,
which is less than  for sufficiently large .
\end{proof}

\begin{lemm}
\label{lemm:SecondPhase}
Consider an arbitrary  cell  of the tessellation.
Let  and  be defined as in
Lemma~\ref{lemm:FirstPhase}, and let
,
,
and .
Then, the following two properties hold with probability
at least  for  sufficiently large:
\begin{enumerate}
\item\label{point:re-exploration}
For  and for each of its adjacent cells,
there exists a time , with ,
at which there is an informed agent in the cell;
\item\label{point:re-population}
At any time , with , there are
at least  informed agents at distance
at most  from .
\end{enumerate}
\end{lemm}
\begin{proof}
We condition on the event stated in Lemma~\ref{lemm:FirstPhase},
which occurs with probability .
Hence, assume that by time  there
are at least  informed agents at distance
at most  from .
Consider the center node  of  (resp.,  adjacent to ),
so that at  there are at least 
informed agents at distance at most  from .
By Lemma~\ref{lemm:SRW} the probability that  is not touched by an
informed agent between  and  is at most
,
which is less than , for sufficiently large ,
by our choice of .
Thus, Point~\ref{point:re-exploration} follows.

As for Point~\ref{point:re-population}, consider an informed agent  which,
at time , is at a node  at distance
at most  from .
Fix a time .
By Lemma~\ref{lemm:props} the probability that at time  agent  is
at distance greater than  from  is at most .
Hence, at time  the average number of informed agents at distance
at most  from  is at least .
Since agents move independently,
Point~\ref{point:re-population} follows by applying the
Chernoff bound to bound the probability that at time  there are
less than  informed agents at distance
at most  from , and by applying the union bound
over all time steps of the interval .
\end{proof}

We are now ready to prove the main theorem of this subsection:
\begin{proofof}{Theorem~\ref{theo:UBSpreadingTime2}}
As observed at the beginning of the subsection, we can limit ourselves
to the case .  Consider the tessellation of
 into  cells defined before, and
focus on a cell  reached for the first time at .
By Lemma~\ref{lemm:SecondPhase}, we know that with probability at
least , in each time step 
there are at least  informed agents at distance at
most  from 
and there exists a time  such that an informed agent
is again inside .
By applying again the lemma, we can conclude that, with probability at
least , at any time step  there are at least  informed agents at
distance at most  from .  Note that the
two time intervals  and  overlap and the latter one ends at least
 time steps later.  Thus, by applying the lemma 
times, we ensure that, with probability at least 
, 
from time  until the end of the broadcast, there
are always at least  informed agents at distance at
most  from .

Lemma~\ref{lemm:SecondPhase} shows that each of the neighboring cells
of  is reached within time  with
probability . Therefore, all cells are reached within time
 with probability at least .
Hence, by applying a union bound over all cells,
we can conclude that with probability at least 

there are at least  informed agents at
distance at most  from each cell of the tessellation,
from time  until the end of the broadcast.

Consider now an agent  which, at time , is uninformed and resides in
a certain cell .  By an argument similar to the one used to prove
Lemma~\ref{lemm:FirstPhase}, we can prove that  meets at least one
of the informed agents around  within  time
steps with probability at least . A union bound over all
uninformed agents completes the proof.
\end{proofof}

Observe that the broadcast time is a non-increasing function of the
transmission radius.  Therefore, the upper bound developed for the
case  holds for any , as stated in the following corollary.
\begin{coro}
\label{coro:monotone}
For any  and , 
with probability at least .
\end{coro}

As another immediate corollary of the above theorem, we can prove that
the gossiping of multiple distinct rumors completes within the same
time bound, with high probability.
\begin{coro}
For any  and , 
with probability at least .
\end{coro}

\subsection{Lower Bound on }
\label{sec:gslb}

In this subsection we prove that the result of
Corollary~\ref{coro:monotone} is indeed tight, up to logarithmic
factors, for any value  of the transmission radius below the
percolation point.  Note that this result is also a lower bound on
 if there are multiple rumors in the system.  First observe that
with probability , there exists an agent placed at
distance at least  from the source of .
W.l.o.g., we assume that the -coordinates of the positions occupied
by such an agent and the source agent differ by at least 
and that the latter is at the left of the former.  (The other cases
can be dealt with through an identical argument.)  In the proof, we
cannot solely rely on a distance-based argument since we need to take
into account the presence of ``many'' agents which may act as relay to
deliver the rumor.

We define the \emph{informed area}  at time  as the set of
grid nodes visited by any informed agent up to time , and let
 to be the rightmost grid node in .  The
\newterm{frontier} of  is the border separating the informed
area from the remaining places of the grid.  We will show that there is
a sufficiently large value  such that, at time , there is at
least one uniformed agent right of .  We need the
following definition:
\begin{defi}[Island]
\label{defi:Island}
Let  be the set of agents.  For any , let 
be the graph with vertex set  and such that there is an edge
between two vertices iff the corresponding agents are within distance
 at time .
The \newterm{island} of parameter  of
an agent  at time  is the connected component of  containing .
\end{defi}
Next, we prove an upper bound on the size of the islands.
\begin{lemm}
\label{lemm:NoBigNeighborhood}
Let .
Then, the probability that there exists
an island of parameter 
in any time instant 
with more than  agents is at most .
\end{lemm}
\begin{proof}
Since at any given time the agents are uniformly distributed in ,
the probability that a given agent is within distance 
of another given agent at time  is bounded by .
Fix a time  and let  denote the event that
there exists an island with at least  elements at time . 
Then, recalling that  is the number of unrooted 
trees over  labeled nodes, we have that 

Using definition of  and the bound  and , we have 

for a sufficiently large .
Applying the union bound on  time steps concludes the proof.
\end{proof}

Next we show that, with high probability,
the frontier of the informed area cannot advance too fast
if the transmission radius satisfies .
\begin{lemm}
\label{lemm:SlowFrontier}
Suppose .
Let  and let 
and  be two time steps.
Then, with probability ,

\end{lemm}
\begin{proof}
By Lemma~\ref{lemm:props}, with probability  an agent cannot
cover a distance of more than  in 
time steps.  Thus, with probability , up to time  the
rumor cannot propagate (directly or through intermediate agents)
between islands. By applying Lemma~\ref{lemm:NoBigNeighborhood} we
conclude that at time  the rightmost position touched by agents
of any island  is at most  right of the
rightmost position occupied by agents of  at , which is not on
the right of . Thus, the lemma follows.
\end{proof}

Finally, we can prove the main theorem of the subsection:
\begin{theo}
\label{theo:LBSpreadingTime2}
For any , suppose .
Then, with probability \sloppy ,

\end{theo}
\begin{proof}
As mentioned before, with probability 
there exists an agent  placed at
distance at least  from the source of the rumor and we may
assume that their -coordinates differ by at least  and
that the uninformed agent is to the right of the source agent.
Let  and
.
By Lemma~\ref{lemm:SlowFrontier}, with probability 
the frontier cannot move right in  steps more than
.
By Lemma~\ref{lemm:props}, with probability ,
agent  cannot move left more than ,
so that agent  cannot be informed by time .
Hence, the broadcast time is at least

with probability .
\end{proof}



\section{Further Results and Future Research}
\label{sec:conclusions}

In this work we took a step toward a better understanding of the
dynamics of information spreading in mobile networks.  We proved a
tight bound (up to logarithmic factors) on the broadcast of a rumor in
a mobile network where agents perform independent random walks on a
grid and the transmission radius defines a system below the
percolation point.  Our result complements the work of Peres et
al.~\cite{PeresSSS11}, who studied the behavior of a similar system
above the percolation point.  A similar bound holds for the gossip
problem in this model, where at time  each agent has a distinct
rumor and all agents need to receive all rumors.

Our analysis techniques are applicable to some interesting related
problems.  For example, similar bounds on the broadcast time  and
the gossip time  can be obtained for the Frog Model
\cite{AlvesMP02}, where only informed agents move and uninformed
agents remain at their initial positions.  In particular, we can show
that the broadcast time in the Frog Model is upper bounded by .  The argument is similar to the proof of
Theorem~\ref{theo:UBSpreadingTime2}, where
Lemma~\ref{lemm:MeetingProbability} is replaced with
Lemma~\ref{lemm:SRW} and the analysis of the initial phase of the
information dissemination process is carried out by using
Point~\ref{poin:range} of Lemma~\ref{lemm:props}. Also, a closer look to
Theorem~\ref{theo:LBSpreadingTime2} reveals that the same argument
employed in our dynamic model to bound  (hence, ) from below 
applies to the Frog Model. Thus, we have tight bounds, up to
logarithmic factors, in this latter model as well.

Another measure of interest in systems of mobile agents is the
\newterm{coverage time} , that is, the first time at which every
grid node has been visited at least once by an informed agent
\cite{PeresSSS11}. While in the Frog Model the broadcast time is
obviously upper bounded by the coverage time, this relation is not so
obvious in our dynamic model, since the coverage of the grid nodes
does not imply that all agents have been informed of the rumor.
Nevertheless, one can verify that, in our model, .  Indeed, by Point~\ref{point:re-population} of
Lemma~\ref{lemm:SecondPhase} and by Lemma~\ref{lemm:SRW}, after
 steps from the first time at which an informed agent
reached a given cell, all the nodes of that cell have been visited by
some informed agent.  Hence, by the cell-by-cell spreading process
devised in the proof of Theorem~\ref{theo:UBSpreadingTime2}, we can
conclude that the coverage time is bounded by .
(In fact, the same tight relation between  and  can be
proved in the Frog Model.)

\sloppy Another by-product of our techniques is a high-probability upper bound
 on the
\newterm{cover time} of  independent random walks on the -grid
(i.e., the time until each grid node has been touched by at least one
such walk), improving on previous results~\cite{AlonAKKLT08,
  ElsasserS09} providing the same bound only for the expected
value. Finally, in a closely related scenario, namely a random
\newterm{predator-prey system} where  predators
are to catch moving preys on an -node grid by performing independent
random walks~\cite{CooperFR09}, we can prove a
high-probability upper bound  
on the extinction time of the preys.

We are working now on extending our modeling and analysis techniques
to handle more complex planar domains that include both communication
and mobility barriers.

\noindent {\bf Acknowledgments:}
Many thanks to Jeff Steif for referring us to some crucial
references and to Andrea Clementi and Riccardo Silvestri for
pointing out a claim not fully justified in the proof of Lemma 4,
which appeared in the first version of the draft.

\phantomsection
\addcontentsline{toc}{chapter}{References}
\bibliographystyle{acm}
\bibliography{references}
\end{document}