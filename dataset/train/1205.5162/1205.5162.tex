\documentclass[11pt,a4paper]{article}


\usepackage{fullpage}
\usepackage{amsmath}
\usepackage{amssymb}
\usepackage{amsthm}
\usepackage{color}
\usepackage{cite}
\usepackage{graphicx}
\usepackage{hyperref}
\usepackage{verbatim}
\usepackage{subfigure}
\RequirePackage{lineno}

\newtheorem{theorem}{Theorem}
\newtheorem{lemma}[theorem]{Lemma}
\newtheorem{claim}{Claim}
\newtheorem{definition}{Definition}
\newtheorem{proposition}{Proposition}
\newtheorem{corollary}[theorem]{Corollary}
\newtheorem{conjecture}{Conjecture}
\newtheorem{remark}{Remark}
\newtheorem{open}{Open Problem}

\usepackage[paper=a4paper,left=25.4mm,right=25.4mm,top=25.4mm,bottom=30mm,bindingoffset=0mm]{geometry}

\newcommand{\Hlinecell}{{\mathcal H}_{line-cell}}
\newcommand{\Hvertexcell}{{\mathcal H}_{vertex-cell}}
\newcommand{\Hcellzone}{{\mathcal H}_{cell-zone}}
\newcommand{\Hyper}{{\mathcal H}}

\definecolor{SEB}{rgb}{0,0,1}
\def\marrow{{\marginpar[\hfill]{}}}

\newcommand{\mati}[1]{\textcolor{red}{\textsc{Mati:} #1}}
\newcommand{\ferran}[1]{\textcolor{red}{\textsc{Ferran:} #1}}
\newcommand{\jean}[1]{\textcolor{blue}{\textsc{Jean:} #1}}
\newcommand{\seb}[1]{\textcolor{blue}{[\marrow Seb: #1]}}

\newcommand{\sholong}[2]{#2}

\newcommand{\proofguardwithcells}{The first iterations of the algorithm select cells that each cover four new lines. 
We iteratively select a cell covering four lines as long as the average number of segments of uncovered lines bounding a cell is strictly greater than three. The total number of segments is , and each contribute to two cells. Every selected cell discards four lines, and exactly  segments of those. If the cell is bounded by more than four lines, we only discard exactly four of them, arbitrarily. The total number of cells after the th iteration is . The number of iterations is the largest value  that satisfies:

Hence we can select roughly  cells, covering together  lines.

In the second phase of the algorithm, we iteratively select cells covering three new lines. Following the same reasoning, and taking into account the  previously selected cells, we know that the number  of iterations satisfies:

Hence we can select roughly  more cells, covering together  lines.

Overall, we now have  cells covering  lines. It remains to cover the remaining  lines with  cells, each covering two lines, as in Theorem~\ref{warmup:guardwithcells}. The total number of cells is therefore
}

\title{Coloring and Guarding Arrangements\thanks{Research supported by the the ESF EUROCORES programme EuroGIGA, CRP ComPoSe: Fonds National de la Recherche Scientique (F.R.S.-FNRS) - EUROGIGA NR 13604, for Belgium, and MICINN Project EUI-EURC-2011-4306, for Spain.}}
\author{Prosenjit Bose\thanks{Carleton University, Ottawa, Canada. \tt{jit@scs.carleton.ca}}
\and Jean Cardinal\thanks{Universit\'e Libre de Bruxelles (ULB), Brussels, Belgium. \tt{\{jcardin, secollet, mkormanc,slanger, perouz.taslakian\}@ulb.ac.be}} 
\and S\'ebastien Collette\footnotemark[3]~~\thanks{Charg\'e de Recherches du F.R.S.-FNRS.}  
\and Ferran Hurtado\thanks{Universitat Polit\`{e}cnica de Catalunya (UPC), Barcelona, Spain.
 {\tt ferran.hurtado@upc.edu}. Partially supported by projects MTM2009-07242 and Gen. Cat. DGR 2009SGR1040.} 
  \and Matias Korman\footnotemark[3]
  \and Stefan Langerman\footnotemark[3]~~\thanks{Ma\^itre de Recherches du F.R.S.-FNRS.} 
  \and Perouz Taslakian\footnotemark[3]}


\begin{document}
\maketitle

\begin{abstract}
Given a simple arrangement of lines in the plane, what is the minimum number  of colors required to color the lines so that no cell of the arrangement is monochromatic? In this paper we give bounds on the number c both for the above question, as well as some of its variations. We redefine these problems as geometric hypergraph coloring problems. If we define  as the hypergraph where vertices are lines and edges represent cells of the arrangement, the answer to the above question is equal to the chromatic number of this hypergraph. We prove that this chromatic number is between . and .

Similarly, we give bounds on the minimum size of a subset  of the intersections of the lines in  such that every cell is bounded by at least one of the vertices in . This may be seen as a problem on guarding cells with vertices when the lines act as obstacles. The problem can also be defined as the minimum vertex cover problem in the hypergraph , the vertices of which are the line intersections, and the hyperedges are vertices of a cell. Analogously, we consider the problem of touching the lines with a minimum subset of the cells of the arrangement, which we identify as the minimum vertex cover problem in the  hypergraph.



\end{abstract}
\sholong{\newpage}{}

\section{Introduction}
While dual transformations may allow converting a combinatorial geometry problem about a \emph{configuration of points} into a problem about an \emph{arrangement of lines}, or reversely, the truth is that most mathematical questions appear to be much cleaner and natural in only one of the settings. In many cases, the dual version is considered solely when, besides making sense, it is additionally useful. Both kinds of geometric objects have inspired many problems and attracted much attention. For finite point sets the \emph{Erd\H{o}s-Szekeres problem} on finding large subsets in convex position, or the \emph{repeated distances problem} on how many times can a single distance appear between pairs of points, are examples of famous questions that have been pursued for decades and are still open. Many research problems of this kind are described in Chapter 8 in \cite{BMP}. Concerning arrangements of lines, possibly the most prevalent problems consist of studying the number of cells of each size, say triangles, that appear in every arrangement, but many other issues have been considered (see \cite{Fe,Gru1,Gru2}). There are also problems that combine both kinds of objects, like counting incidences between points and lines, or studying the arrangements of lines spanned by point sets, which includes the celebrated \emph{Sylvester-Gallai problem} on ordinary lines (those that only contain two of the points) \cite{BMP}.

In the first class of problems, substantial attention has been focusing on \emph{colored pointsets}, i.e., configurations of points that belong to several classes, the \emph{colors}, including chromatic variants of the repeated distances problem and the Erd\H{o}s-Szekeres problem, and colored versions of \emph{Tverberg's Theorem} and \emph{Helly's Theorem}. In particular there is a vast body of research on problems involving a set of \emph{red} points and a set of \emph{blue} points. Refer to \cite{KK} for a survey on red-blue problems, or to \cite{BMP} for a more generic account.

Somehow surprisingly, there is not a comparable set of questions that have been posed for colored arrangements of lines. There is a series of papers on the problem of taking bicolored sets of lines, calling \emph{monochromatic vertex} an intersection point contained only in lines of one of the colors, and discussing their existence and number \cite{Gru3,Gru4,Mo}. Another series of papers study the colorings of the so called arrangement graphs, in which vertices are the intersection points and edges are the segments between any two that are consecutive on one of the lines \cite{BEW,FHNS}.

However, many other natural questions can be asked. For example, is it true that every bicolored arrangement of lines has a monochromatic cell? We prove in this paper that the generic answer is no, but that it is yes when the colors are slightly unbalanced. This leads immediately to another question that we discuss in our work: How many colors are always sufficient, and occasionally necessary, to color any set of  lines in such a way that the induced arrangement contains no monochromatic cell?

The last question brings manifestly the flavor of \emph{Art Gallery Problems} \cite{OR1,OR2,URR}. We also consider for line arrangements several issues on this topic that apparently have not been studied before: How many vertices of an arrangement suffice to guard all its cells? How many lines are enough to guard (touch) all the cells?

While coloring and guarding arrangements of lines may appear at first glance as unrelated problems, there is a clean unifying framework provided by considering appropriate geometric hypergraphs. For example, minimally coloring an arrangement while avoiding monochromatic cells can be reformulated as follows: Let  be the geometric hypergraph where vertices are lines and edges represent cells of the arrangement. What is the chromatic number of the hypergraph? Here a proper coloring is one where no hyperedge is monochromatic.

In this work we consider several questions as the ones described in the preceding paragraphs, which arise as quite fundamental in terms of coloring and guarding arrangements of lines, and translate consistently into problems on geometric hypergraphs,  like size of a maximal independent set, size of a vertex cover, or some coloring parameter.

The terminology for hypergraphs on arrangements is introduced in Section \ref{sec:definitions}, where we also provide a table summarizing our results. Coloring problems are then discussed in Section \ref{sec:coloring} and guarding problems in Section \ref{sec:guarding}. We conclude with some observations and open problems.

\section{Definitions and Summary of Results}\label{sec:definitions}
Let  be an arrangement of a set of lines  in . We say that an arrangement of lines  is {\em simple} if every two lines intersect, and no three lines have a common intersection point. From now on, we only consider simple arrangements of lines\footnote{For non-simple arrangements, the answer to most of the problems we study are either trivial or not well defined.}. 

Any arrangement  decomposes the plane into different cells, where a {\em cell} is a maximal connected component of . We define  as the geometric hypergraph corresponding to the arrangement, where  is the set containing all cells defined by . Similarly,  is the hypergraph defined by the vertices of the arrangements and its cells, where  is the set of intersection of lines in . Finally,  is the hypergraph defined by the cells of the arrangement and its zones. The zone of a line  in  is the set of cells bounded by . The set  is defined as the set of subsets of  induced by the zones of . Note that this hypergraph is the dual hypergraph of .

An \emph{independent set} of a hypergraph  is a set  such that . This definition is the natural extension from the graph variant, and requires that no hyperedge is completely contained in . Analogously, a \emph{vertex cover} of  is a set  such that . The \emph{chromatic number}  of  is the minimum number of colors that can be assigned to the vertices  so that ; that is, no hyperedge is monochromatic.

In the forthcoming sections we give upper and lower bounds on the worst-case values for these quantities on the three hypergraphs defined from a line arrangement. Our results are summarized in Table \ref{tabres}. Note that the maximum independent set and minimum vertex cover are complementary problems. As a result, any lower bound on one gives an upper bound on the other and vice versa. This property, along with the facts that , , and , are used to complement many entries of the table. 



The definitions of an independent set and a proper coloring of the  hypergraph of an arrangement are illustrated in Figures~\ref{fig:independent} and \ref{fig:coloring}, respectively. Similarly, the definition of a vertex cover of the 
and  hypergraphs are illustrated in Figures~\ref{fig:guardingcells}, and \ref{fig:guardinglines}, respectively.

\begin{table}
\begin{center}
\begin{tabular}{|c|c|c|c|}
\hline
Hypergraph & Max. Ind. Set & Vertex Cover & Chromatic number\\
\hline
 &  (Th.~\ref{th:is})&  (Cor. \ref{cor:CL})&  (Th.~\ref{LB:chroma})\\
		    &  (Th.~\ref{th:ubis}) &  (Cor. \ref{cor:CL})&   (Th.~\ref{UB:chrom})\\
\hline
 &  (Cor. \ref{cor:VCIS}) &  (Th.~\ref{LB:guards})& 2 (Th.~\ref{tight:chromvertex})\\
 &  (Cor. \ref{cor:VCIS}) & (Th.~\ref{LB:guards})& \\
\hline
 &  (Cor. \ref{cor:CZIS})&  (Th.~\ref{guardwithcells})& 2 (Th.~\ref{tight:chromcells})\\
 &  (Cor. \ref{cor:CZIS})&  (Th.~\ref{guardwithcells})&\\
\hline
\end{tabular}
\end{center}
\caption{Summarizing table with the worst-case bounds for the different problems studied in this paper. 
}\label{tabres}
\end{table}

\begin{figure}
\begin{center}
\subfigure[\label{fig:independent}The thick lines form an independent set in the  hypergraph: no cell is bounded by
those lines only.]{\includegraphics[width=.4\textwidth]{independent.pdf}}\hspace{1cm}
\subfigure[\label{fig:coloring}A proper 3-coloring of the  hypergraph: no cell is monochromatic.]{\includegraphics[width=.4\textwidth]{coloring.pdf}}\hspace{1cm}
\subfigure[\label{fig:guardingcells}The marked intersections form a vertex cover of the  hypergraph: every cell has at least
one such intersection on its boundary. That is, these vertices \emph{guard} the cells.]{\includegraphics[width=.4\textwidth]{guardingcells.pdf}}\hspace{1cm}
\subfigure[\label{fig:guardinglines}The two marked cells form a vertex cover of the  hypergraph: every line has a segment that lies on the boundary
of one of those cells. That is, these two cells \emph{guard} the lines.]{\includegraphics[width=.4\textwidth]{guardinglines.pdf}}
\end{center}
\caption{\label{fig:examples}Illustrations of the definitions.}
\end{figure}

\section{Coloring Lines, and Related Results}\label{sec:coloring}

We first consider the chromatic number of the line-cell hypergraph of an arrangement, that is, the number of colors required for coloring the lines so that no cell has a monochromatic boundary. At the end of the section we include some similar results.

\subsection{Two-colorability}

We say that a set of lines  is {\em -colorable} if we can color  with -colors such that no cell is monochromatic (in other words, the corresponding  hyper graph has chromatic number ). Any coloring  that satisfies such a property is said to be {\em proper}. We first tackle the (simple) question of whether the two-colorable  hypergraphs have bounded size:

\begin{theorem}\label{prop:2colors}
There are arbitrarily large two-colorable sets of lines.
\end{theorem}



\begin{proof} An infinite family of such examples are provided by a set of  lines in convex position (for any \sholong{)}{, see Figure~\ref{fig:bichrom})}.Observe that in such arrangement, each cell is either bounded by  two consecutive lines,  the first and the last line or  all lines of the arrangement. It is easy to check that, if we color the lines alternatively red and blue by order of slope, no cell will be monochromatic.
\end{proof}

\begin{comment}
It is easy to find arbitrarily large arrangements of lines that are not two colorable. Hence,  a natural question --so far elusive to us-- arising from the previous result, is to determine if a given arrangement is two colorable.
\begin{open}
Is it NP-hard to decide if a set of lines can be two-colored avoiding monochromatic cells?
\end{open}
\end{comment}

The coloring used in Theorem \ref{prop:2colors} uses essentially the same number of lines of each color. We prove next that the result cannot hold when the numbers are unbalanced to a small extent.

\begin{theorem}
Each color class of a proper 2-coloring  of a set  of  lines has size at most .
\end{theorem}
\begin{proof}
Let  be the set of lines that are assigned color , and  the set of lines whose color is . Let  denote the arrangement of the lines in . As  does not completely define a cell of , each cell of  must be traversed by a line in .

We proceed iteratively: we start with , and add the lines in  one at a time. When adding a line , some cells of  will be split into two parts by a line segment induced by . A connected component of segments inside a cell of  is a set of segments whose intersection graph is connected.

To each cell  of , we assign a number, representing the number of connected components defined by the segments inside . Let  denote the number of connected components inside the cell  after adding the  line of , and  is the sum of  over all cells .

When the first line  is added,  for each cell  crossed by , and remains zero for every other cell. Therefore, . In general, when the  line  is added,  increases by . Indeed, in each cell , the blue line can only intersect each component once, otherwise the corresponding segments would create a cycle, meaning a new face bounded only by blue lines, and thus monochromatic. This implies that  intersects all previous  lines in  disjoint components. Inside a cell , if a line  intersects  components, then . Thus, .

What we also know, is that at the end of the process, each cell of  should contain at least one component, otherwise the cell is monochromatic. Thus  should be bigger or equal to the number of cells in .

We get:


\noindent which concludes the proof.
\end{proof}

\subsection{Independent lines in }\label{sec_indep}

Recall that an independent set of lines in an arrangement is defined as a subset of lines  so that no cell of the arrangement is only adjacent to lines in .

\begin{theorem}\label{th:is}
For any set  of  lines, the corresponding  hypergraph has an independent set of size .
\end{theorem}
\begin{proof}
We prove that any (inclusionwise) maximal independent set has size . Consider such a maximal independent set  of size . 
By maximality, each line  can be associated with at least one cell of  whose boundary consists only of one segment of , and segments of  lines in . We choose one such cell for each line , and call this cell . Then, we consider the set  of quadrants defined by the intersections of the lines in , each intersection defining 4 quadrants. If  has size at least 3 for some , then we charge  to one of the quadrant of  formed by the intersection of two lines of  on the boundary of . If  has size two, then we charge  to . Note that, by definition of , no quadrant can be charged more than once. Similarly, no cell of size two can be charged more than once. Hence the number of lines in  cannot exceed the sum of the number of quadrants and the number of cells of size two:

\end{proof}

\begin{theorem}\label{th:ubis}
Given a set  of  lines, an independent set of the corresponding  hypergraph has size at most .
\end{theorem}
\begin{proof}
Let  be an independent set of lines in . This means that, in the corresponding arrangement , each cell is touched by at least a line . Each line  crosses  cells of . There are  cells in total, and thus




\noindent and therefore we conclude that .
\end{proof}

\subsection{Chromatic number of }
In this section, we study the problem of coloring the  hypergraph. That is, we want to color the set  so that no cell is monochromatic. We start by giving an upper bound on the required number of colors.

\begin{theorem}\label{UB:chrom}
Any arrangement of  lines can be colored with at most  colors so that no edge of the associated  hypergraph is monochromatic.
\end{theorem}
\begin{proof}
Our coloring scheme is as follows: select the largest independent set , color all the lines of  with the same color, remove  from . We now iterate on the remaining lines, where in each step of the algorithm a different color is assigned to the lines we remove. The algorithm stops whenever the number of non-colored lines is at most  (the exact value of  will be determined afterwards). Whenever  or fewer lines remain, we complete the coloring by adding a new color to each of the remaining lines. 

In order to show that this method provides a proper coloring, first observe that any independent set of  is also an independent set of . That is, any set of lines with the same color assigned form an independent set of . In particular, there cannot exist a cell  in which all lines adjacent to  have the same color assigned.

Let  be the maximum number of colors needed for an arrangement of  lines. We will prove that  using induction. Recall that, by Theorem \ref{th:is}, the size of a maximal independent set is at least . Let  be the smallest integer such that . Our coloring strategy gives the following recursion for any . 


\end{proof}

We now construct a slightly sublogarithmic lower bound for the chromatic number of :

\begin{theorem}
\label{LB:chroma}
There exists an arrangement of  lines whose corresponding hypergraph   has chromatic number .
\end{theorem}
The proof of this claim is constructive. In the following we construct a set of (roughly)  lines, in which any -coloring will contain a monochromatic cell (for any ).  Since we are interested in the asymptotic behavior, it suffices to prove for the case in which  is a power of two (that is,  for some ). In order to proceed with the proof, we first introduce some definitions and helpful results.


For any  we consider the order in which we traverse the lines of  in the vertical line  from top to bottom. Although the permutation obtained will depend on  and , there will be exactly  different permutations in any set  of  lines. Let  be the set of different permutations that we can obtain. Each of these permutations is called a {\em snapshot} of .


With this definition we can give an intuitive idea of our construction. Consider any coloring with  colors of a set of  lines. By the pigeonhole principle there will be two lines with the same assigned color. Moreover, since the two lines must cross, these two lines must be consecutive in the ordering given by some snapshot. Our approach is to cross these two lines with a second pair of lines with the same color assigned, hence obtaining a monochromatic quadrilateral. The main difficulty of the proof is that the line set must satisfy this property for any -coloring of . In particular, we do not know at which snapshot will the two lines of the same color meet.

We say that a set of snapshots  is a {\em witness} set of  if, for any two lines , there exists a snapshot  in which the two lines appear consecutively in . It is easy to see that the whole set  is a witness sets of quadratic size for any set of lines. Since the size of the witness set has a direct impact on our bound, we first show how to construct a witness set of smaller size:

\begin{lemma}\label{LB:basicgad}
For any  there exists a set  of  lines and a witness set  such that .
\end{lemma}
\begin{proof}
We construct the arrangement by induction on  (recall that we assumed  for some ). For  our base gadget  simply consists of a single line. Note that the witness property is trivially true, since there don't exist two distinct lines in , hence we define .

As we are only interested in the ordering in which lines are crossed, we can do any transformation to a set  of lines, provided that  transformation preserves the permutations in the set . If we update the coordinates of the snapshots in the witness set accordingly, the witness property will still hold. In particular, we can transform a set  of lines so that they become almost parallel and have any desired slope. We call this operation the {\em flattening} of .

With this operation in mind we can do the induction step as follows: for any  generate a copy of gadget  and flatten the lines so that they all have small positive slope and all crossings between lines occur below the horizontal line . Let  be the transformed set of lines and  be the reflexion of  with respect to line . Gadget  is defined as the union of  and  (see Figure \ref{fig_basegadget}).

Observe that  satisfies the following properties:
\begin{figure}
   \begin{center}
     \includegraphics[width=0.7\textwidth]{fig_basegadget}
     \caption{Induction step in the gadget  construction (left). The additional snapshots are also shown  (dashed vertical lines). The generated arrangement for  and its witness set is shown in the right. For clarity lines of  have been depicted as pseudolines.}
     \label{fig_basegadget}
   \end{center}
\end{figure}

\begin{itemize}
\item[]  Gadget  has size exactly . Moreover, any two lines cross exactly once.
\item[]  The witness set  of  also acts as witness set of .
\item[]  The lines of  and  intersect in a grid-like fashion, forming cells of size  and . \end{itemize}

Observe that property  certifies that the construction is a valid set of lines, while properties  and  help us obtain a witness set  of small size; the crossing between lines of different gadgets  can be guarded with  lines (see Figure \ref{fig_basegadget}). Moreover, the crossings between lines of the same gadget can be guarded by the witness set  (which by induction satisfies ). By construction we have that .

To finish the proof we must show that  is indeed a witness set of : Let  be any two lines of . If these lines belong to the same sub-gadget, we can apply induction and obtain that they must be consecutive in one of the first snapshots. Otherwise, the two lines belong to different sub-gadgets of , hence they will be consecutive at the latter snapshots.
\end{proof}

With the preceding result we can now prove Theorem \ref{LB:chroma}:

\begin{proof}
Let  be the set of lines constructed in Lemma \ref{LB:basicgad} and let  be the witness set of  (recall that we have  and  for some ). Also, let  be the snapshots of , sorted from left to right. Consider now any coloring of  with  colors. By the pigeonhole principle, there must exist two lines  with the same color assigned. Since  is a witness set, these two lines must be consecutive at some snapshot . Whenever this happens, we say that  and  form a {\em monochromatic consecutive pair} at snapshot .

In the following, we generalize the above construction to a set  of size  for any . The key property is that in any -coloring of the set , either there is a monochromatic cell or there exist two lines that form a monochromatic consecutive pair at some snapshot  (for ). In particular, notice that the second condition cannot occur for set , hence  there must exist a monochromatic cell in any -coloring of .

We construct the set  by induction on . For  the claim is true by Lemma \ref{LB:basicgad}, hence we can focus on the inductive step. For any , we construct  with  different copies of set  flattened so that they satisfy the following properties:

\iffalse
\begin{figure}
   \begin{center}
     \includegraphics[width=0.7\textwidth]{fig_induc}
     \caption{Construction of the set  by combining  different copies of . In the Figure, each thick line represents a copy of  (and the dotted vertical lines the snapshots of , for ). Observe that all the crossings between lines of different copies occur in the vertical strip  (depicted as light gray in the figure). Since no crossing between lines of the same copy occur in the strip, the vertical ordering of the lines of a single copy at any position in the strip is exactly .
     In the right we depict a larger image of the crossing between two copies of . The main property is that any two consecutive lines of a copy of  form a quadrilateral with any other pair of consecutive lines of another copy of .}
     \label{fig_induc}
   \end{center}
\end{figure}
\fi

\begin{itemize}
\item  For any , The snapshot of each copy  at coordinate  is .
\item  No two lines of the same copy  of  cross in the vertical strip . In particular,  the snapshot taken at any coordinate of the strip is .
\item  Lines of two different copies of  cross in the  vertical strip  in a grid-like fashion. In particular any two lines that are consecutive in  form a quadrilateral with other two consecutive lines of another copy of .
\end{itemize}
This construction can be done by flattening all the copies of  so that each copy essentially becomes a thick line, and placing the different copies in convex position\sholong{.}{(see Figure \ref{fig_induc}). We define the set  as the set  (that is, we remove  from ).}
 Observe that, since  is composed of  different copies of , we indeed have .\sholong{}{Moreover, the size of  decreases by one in each iteration, hence .}

In order to complete the proof we must show that, in any coloring  of , we either have a monochromatic cell or a  monochromatic consecutive pair in  (for some ). Apply induction to the different copies of : if at least one of the copies has a monochromatic cell or has its monochromatic consecutive pairs at snapshot  (for some ) we are done, since the same property will hold for . The other case occurs when all copies of  have their monochromatic consecutive pair at snapshot . Let  be the monochromatic consecutive pair of the -th copy of  and let  be its color. By the pigeonhole principle, there must be two distinct indices  such that . By property  of our construction, the lines  form a quadrilateral in the arrangement of lines of . The quadrilateral will be monochromatic, since by definition the four lines have the same color assigned.
\end{proof}

\subsection{Other coloring results}
For the sake of completeness, we end this section by stating two easy results on coloring vertices or cells instead of lines.

\begin{theorem}\label{tight:chromvertex}
The chromatic number of  is 2.
\end{theorem}
\noindent Remark that cells of size two only have one vertex, hence cannot be polychromatic. Therefore,  we only consider cells of size at least 3.
\begin{proof}
It is known that the graph obtained from an Euclidean arrangement of lines by taking only the
bounded edges of the arrangement has chromatic number 3 \cite{FHNS}: Sweep the arrangement with a line from left to right. In this ordering, every vertex in the arrangement is adjacent to exactly two predecessors and hence we can assign the colors greedily, such that each vertex has a color different from at least one of its predecessors.
Finally, we can identify two of the three colors, and no cell which is at least a triangle can be monochromatic.
\end{proof}

The following well-known result considers the coloring of cells so that no line is only adjacent to cells of a single color class:

\begin{theorem}[Folk.]\label{tight:chromcells}
The chromatic number of  is 2.
\end{theorem}
\begin{proof}
This claim is equivalent to the fact that the dual graph of the arrangement (where vertices are 
faces, and there is an edge between two faces if they are adjacent) is bipartite. This result
has appeared in recreational texts and concited some research as well \cite{L1894,Gru5}.
\end{proof}



\iffalse
\begin{theorem}\label{LB:chrom}
There exists a set  of  lines which will contain a monochromatic cell in any -coloring.
\end{theorem}
Since we are interested in the asymptotic behavior, it suffices to show the proof of the case in which  is a power of two (that is,  for some ). We note, however, that the proof can also be extended for other values of . In order to show this result we first introduce some definitions and helpful results.

With this result we can now prove Theorem \ref{LB:chrom}:

\begin{proof}
Let  be the set of lines constructed in Lemma \ref{LB:basicgad} and let  be the witness set of . In the following, we generalize the above construction to a set  of size   and a witness set  of size  for any . In particular, set  has an empty  witness set, hence there must exist a monochromatic cell in any -coloring.

We construct set  by induction on . For  the claim is true by Lemma \ref{LB:basicgad}, hence we focus on the induction step. For any , we construct  with  different copies of set  flattened so that they satisfy the following properties:

\begin{figure}
   \begin{center}
     \includegraphics[scale=0.7]{fig_induc}
     \caption{Construction of the set  by combining  different copies of . In the Figure, each thick line represents a copy of  (and the dotted vertical lines the witness set). All the crossings between lines of different copies occur in the vertical strip  (depicted as light gray in the figure). Since there is no crossing between lines of the same copy, the vertical ordering of the lines of a single copy at any position in the strip is exactly the same as in the line .
     In the right we depict a larger image of the crossing between two copies of . The main property is that any two consecutive lines of a copy of  form a quadrilateral with any other pair of consecutive lines of another copy of .}
     \label{fig_induc}
   \end{center}
\end{figure}
\begin{itemize}
\item  For each copy , the snapshot at coordinate  is exactly the -th element of its witness set (for any ).
\item  No two lines of the same copy  of  cross in the vertical strip . In particular,  the vertical ordering of the lines of each copy inside the vertical strip is exactly the first permutation of set .
\item  Lines of two different copies of  cross in the  vertical strip  in a grid-like fashion. In particular any two lines that are consecutive in the first permutation of set  form a quadrilateral with other two consecutive lines of another copy of .
\end{itemize}
This construction can be done by flattening all the copies of  so that each copy essentially becomes a thick line, and placing the different copies in convex position (see Figure \ref{fig_induc}). We define the witness set  of  as the set containing the snapshots at the vertical lines  for . Observe that, since  is composed of  different copies of , we indeed have . Moreover, the witness set decreases by one in each iteration. Thus, in order to complete the proof we must show that it indeed is a witness set.

Consider now any coloring of  and apply induction to the different the copies of . If at least one of the copies has its monochromatic consecutive pair at snapshot  for some  we are done. Thus, we consider the case in which all copies have their monochromatic pair at the snapshot of coordinate . Let  be the monochromatic consecutive pair of the -th copy of  and let  be its color. By the pigeonhole principle, there must be two indices  such that . By property  of our construction, the lines  will form a quadrilateral in the arrangement of lines of . Since, by definition, the four lines have the same color assigned the quadrilateral will be monochromatic.
\end{proof}
\begin{corollary}
\label{UB:chrom}
There exists an arrangement of  lines whose corresponding hypergraph   has chromatic number .
\end{corollary}
\fi

\begin{comment}
\begin{theorem}\label{LB:MIS}
Given a set  of  lines, a maximal independent set of the corresponding  hypergraph has size at least .
\end{theorem}

\begin{proof}
An independent set of size  is a -tuple of lines such that no subset of size  corresponds to an -cell of the arrangement. The proof is by contradiction: we will assume that every -tuple, with , contains a subset which bounds some cell of the arrangement, and is thus not an independent set.

In , there are:
\begin{itemize}
\item at most  unbounded cells defined by two lines (2-cells).
\item less than  -cells defined by exactly  lines, .
\end{itemize}

Let  denote the number of -cells. Now we will count the number of -tuples which correspond to each of these cells. Each -cell defines  non-independent -tuples. Thus, as each -tuple defines at least one cell (otherwise one tuple is independent), we get:



Then we use the fact that  and that .


When  we get a contradiction. In other words, as soon as , we know that there are some -tuples which form an independent set.
\end{proof}

\begin{remark}
The arguments used in our proof of Thm.~\ref{LB:MIS} only require that the hypergraph has no cell of size 1, a linear number of 2-cells and a quadratic number of cells. In other words, it also applies to arrangements of -intersecting curves.
\end{remark}

\begin{corollary}\label{UB:chrom}
The chromatic number of  is at most .
\end{corollary}

\begin{proof}
We simply iteratively remove independent sets : We know there exists an independent set  of size , we color it with one color, and then inductively color .

This can be written as a recurrence: , and let  be the number of iterations required to reach 1 when starting from . We get the differential equation , yielding . Thus, .
\end{proof}

\begin{corollary}\label{UB:VC}
Given a set  of  lines, a minimal vertex cover of the corresponding  hypergraph has size at most .
\end{corollary}
\end{comment}


\section{Guarding Arrangements}\label{sec:guarding}
\sholong{}{We now consider the vertex cover problem of the above hypergraphs. That is, we would like to select the minimum number of vertices so that any hyperedge is adjacent to the selected subset. Geometrically speaking, we would like to select the minimum number of vertices (or cells or lines), so that each cell (or line or cell, respectively) contains at least one of the selected items. Recall that this problem is the complementary of the independent set problem. Hence, for each case we will study the easiest of the two problems.   

}
\subsection{Guarding cells with vertices}

We first consider the following problem: given an arrangement of lines , how many vertices do we need to pick in order to guard the whole arrangement when lines act as obstacles blocking visibility? This can be rephrased as finding the smallest subset of vertices  so that each cell contains a vertex in , and thus we are looking for bounds on the size of a vertex cover for .

\begin{theorem}\label{LB:guards}
For any set  of  lines, a vertex cover of the corresponding  hypergraph has size at most . Furthermore,  vertices might be necessary.
\end{theorem}
\begin{proof}
First notice that any arrangement can have at most  cells of size exactly . Hence, these cells can be easily guarded with  guards. Thus, we focus our attention to cells of size  or more. We will guard these cells via a 3-coloring of the vertices of the arrangement. We sweep the arrangement in a fixed direction, and color the vertices in order. 

Observe that the graph of the arrangement is 4-regular, and when a vertex  is encountered, it has exactly two neighbors (say ) that have already been colored. If  and  have distinct colors, then we assign the third color to . If they have the same color, then we consider the colors assigned to the vertices of the cell having the segments  and  on its boundary. If only two colors are present in the cell, we assign the third one to . Otherwise, we assign arbitrarily one of the two possible colors to . With this construction, it is easy to check that all cells of size at least 3 have vertices of three distinct colors on their boundary. In particular, the vertices of any color can guard all cells of size  or more. Since we used three colors and the total number of vertices is , there will be a color class with at most  vertices.

For the lower bound we will use the construction of Furedi~\emph{et al.}~\cite{furedi}. This construction creates a family of arrangements that has  triangles in which any vertex of the arrangement is incident to at most two of these triangles. In particular, any vertex cover of the triangles will need at least  vertices.
\end{proof}

Recall that the hypergraph  has  vertices. Combining this fact with the preceding bounds on the size of a vertex cover allow us to get similar bounds for the independent set problem:  
\begin{corollary}\label{cor:VCIS}
For any set  of  lines, a maximum independent set of the corresponding  hypergraph has size at least . Furthermore, there exists sets of lines whose largest independent set has size at most .
\end{corollary}

\subsection{Guarding lines with cells}

Here we consider the problem of touching all lines of  with a smallest subset of cells, i.e., we look for bounds on the size of a vertex cover for .



We begin with a simple proof that a minimal vertex cover of  hypergraph has size at most , that we will improve below.
\begin{theorem}\label{warmup:guardwithcells}
Given a set  of  lines, a minimal vertex cover of the corresponding  hypergraph has size at most .
\end{theorem}
\begin{proof}
We describe a greedy algorithm to find a vertex cover of size ; we start with an empty set . We find a pair  of lines that we still have to cover. Since every two lines cross, there must exist a cell  adjacent to both  and . We add that cell  to the set , and proceed with the unguarded cells. In the last step, if a single line  remains to be covered we add  to  any cell touching . Since each cell (except the last one) of  guards at least two lines, at most  cells will be added into .
\end{proof}

We next provide a lower bound, and improve as well on the upper bound, for large values of . 
\begin{theorem}\label{guardwithcells}
Given any set  of  lines, a minimal vertex cover of the corresponding  hypergraph has size at most . Moreover, there exists a set  of  lines, such that every vertex cover of the corresponding  hypergraph has size at least .
\end{theorem}
\begin{proof}
The lower bound is proved by the fact that there exist arrangements where the largest cell has size  (see \cite{LLMSU07}). This implies that each cell touches at most  lines, and therefore  cells are required to touch them all.

The proof of the upper bound claim is a refined version of the method in the preceding theorem: we
first select cells of size four or more and add them to , until any remaining cell that we add to our set is not guaranteed to cover more than three new lines. We then continue adding cells that cover at least three lines in the same fashion. Finally, we complete our construction with cells that cover two lines as in Theorem~\ref{warmup:guardwithcells}. \sholong{Details can be found in the Appendix}{

\proofguardwithcells}.
\end{proof}

\begin{corollary}\label{cor:CZIS}
For any set  of  lines, a maximum independent set of the corresponding  hypergraph has size at least  and at most .
\end{corollary}
This Corollary follows directly from the preceding theorem, the fact that the complement of a vertex cover is an independent set, and that  
any arrangement of  lines in general position has  cells (hence, the  hypergraph will have that many vertices).


\subsection{Guarding cells with lines}
For the sake of completeness, we also give bounds on the number of lines needed to guard (touch) all cells.

\begin{corollary}\label{cor:CL}
For any set  of  lines, its minimal vertex cover of the corresponding  hypergraph has size at least  and at most .
\end{corollary}
Proof of the lower bound is a direct consequence of the complementariness of the vertex cover/independent set and Theorem \ref{th:ubis} (upper bound on the maximum independent set of ). Analogously, the upper bound is a consequence of Theorem \ref{th:is}.
\iffalse
\begin{theorem}\label{UB:guardwithcells}
Given a set  of  lines, a minimal vertex cover of the corresponding  hypergraph has size at most , which is  for large .
\end{theorem}

\begin{proof}
Every simple arrangement has at most  triangles, and at most  cells of size 2. This means that the number of cells of size at least 4 is at least:



\noindent that we denote by .

We will greedily pick cells of size at least 4, and assign 4 of their bounding lines to the cell. While iterating, we make sure that we do not assign a line to more than one cell. This can be done as follows: pick any cell  of size at least 4, and assign 4 of its bounding lines to it. Mark every cell of the arrangement adjacent to one of the 4 assigned lines. Then iterate, making sure that we do not pick a cell which is marked. If a cell is not marked, it was not assigned to any previous line, and is thus not adjacent to any previous line.

Each line is adjacent to at most  cells, therefore when we mark all cells touching the 4 assigned lines, we mark at most  cells.
Thus, we iterate exactly  times. We end up with a set of at most  cells touching  assigned lines.

We will now pick (unmarked) cells of size at least 3. We still have at least  cells of size 3 or more (notice that, as  is a lower bound, we might not have taken every cell of size at least 4 in the first round). Pick any unmarked cell  of size at least 3, and assign 3 of its bounding lines to it. Again, mark every touched cell, and iterate exactly  times. We end up with  cells touching  lines.

So, at this step, we have a total of  cells, touching  different lines. The remaining  lines can be covered as proposed in Lemma~\ref{warmup:guardwithcells}, using at most   cells. In conclusion, we used at most  cells to touch all lines.
\end{proof}
\fi


\section{Concluding Remarks}\label{sec:conclusion}
Clearly, the main open problems arising from our work consist of closing gaps (when they exist) between lower and upper bounds; this is especially interesting in our opinion for the
problem of coloring lines without producing any monochromatic cell. \sholong{}{We observe that most of our observations hold for pseudo lines as well. Hence, another natural extension would be studying how do the bounds change when we consider families of curves that any two intersect at most  times (for some constant ).}

However, it is worth noticing that there are several computational issues that are interesting
as well. For example, it is unclear to us which is the complexity of deciding whether a given
arrangement of lines admits a two-coloring in which no cell is monochromatic.

\sholong{\newpage}{}
{
\begin{thebibliography}{1}

\bibitem{BEW} P. Bose, H. Everett, and S. Wismath.
\newblock {Properties of Arrangements}.
\newblock {\em International Journal of Computational Geometry}, 13(6), pp. 447-462, 2003.

\bibitem{BMP} P.~Brass, W.~Moser and J.~Pach.
\newblock{\em Research Problems in Discrete Geometry}.
\newblock{Vieweg Verlag, 2004.}

\bibitem{Fe} S. Felsner.
\newblock{\em Geometric Graphs and Arrangements}.
\newblock{Springer, Berlin, 2005.}

\bibitem{FHNS} S. Felsner, F. Hurtado, M. Noy and I. Streinu.
\newblock {Hamiltonicity and colorings of arrangement graphs}.
\newblock {\em Discrete Applied Mathematics}, 154(17):2470-2483, 2006.

\bibitem{furedi}
Z.~F\"{u}redi and I.~Pal\'{a}sti.
\newblock {Arrangements of lines with a large number of triangles}.
\newblock {\em American Mathematical Society}, 92(4), 1984.

\bibitem{Gru1} B. Gr\"{u}nbaum.
\newblock{\em Arrangements and Spreads}.
\newblock{Regional Conf. Ser. Math., \emph{American Mathematical Society}}, 1972 (reprinted 1980).

\bibitem{Gru2} B. Gr\"{u}nbaum.
\newblock {How many triangles?}
\newblock {\em Geombinatorics}, 8:154--159, 1998.

\bibitem{Gru3} B. Gr\"{u}nbaum.
\newblock {Arrangements of colored lines}.
\newblock {Abstract 720-50-5, \emph{Notices Amer. Math. Soc.}}, 22(1975), A-200.

\bibitem{Gru4} B. Gr\"{u}nbaum.
\newblock {Monochromatic intersection points in families of colored lines}.
\newblock {\em Geombinatorics}, 9:3--9, 1999.

\bibitem{Gru5} B. Gr\"{u}nbaum.
\newblock {Two-coloring the faces of arrangements}.
\newblock {\em Periodica Mathematica Hungarica}, 11(3):181--185, 1980.

\bibitem{KK} A. Kaneko and M. Kano.
\newblock {Discrete geometry on red and blue points in the plane - a survey -}.
\newblock {In \emph{Discrete and Computational Geometry, The Goodman-Pollack Festschrift}, Springer, Berlin, B. Aronov, S. Basu, J. Pach, M. Sharir,  eds., pp. 551-570, 2003.}

\bibitem{LLMSU07}
J. Lea\~{n}os, M. Lomel\'{\i}, C. Merino, G. Salazar, and J. Urrutia.
\newblock {Simple euclidean arrangements with no ( 5)-gons}.
\newblock {\em Discrete and Computational Geometry}, 38(3):595--603, 2007.

\bibitem{L1894} E. Lucas.
\newblock {\em  R\'{e}cr\'{e}ations math\'{e}matiques IV}. Paris, 1894.


\bibitem{Mo} T. S. Motzkin.
\newblock {Nonmixed connecting lines}.
\newblock {Abstract 67T 605, \emph{Notices Amer. Math. Soc.}}, 14(1967), p. 837.

\bibitem{OR1} J. O'Rourke.
\newblock {\em Art Gallery, Theorems and Algorithms}.
\newblock {Oxford University Press, 1987.}


\bibitem{OR2} J. O'Rourke.
\newblock {Visibility.}
\newblock {Chapter in \emph{Handbook of Discrete and Computational Geometry}, CRC Press LLC, Boca Raton, FL, 2nd edition, J.~E. Goodman and J. O'Rourke eds., pp. 643-665, 2004 2nd ed.ition).}

\bibitem{URR}  J. Urrutia.
\newblock{Art Gallery and Illumination Problems.}
\newblock{Chapter in \emph{Handbook on Computational Geometry}, Elsevier Science Publishers, J.R. Sack and J. Urrutia, eds., pp. 973-1026, 2000.}

\end{thebibliography}
}
\sholong{
\appendix
\section{Proofs omitted from the Document}

\newtheorem*{guardwithcells}{Theorem \ref{guardwithcells}}
\begin{guardwithcells}
Given any set  of  lines, a minimal vertex cover of the corresponding  hypergraph has size at most . Moreover, there exists a set  of  lines, such that every vertex cover of the corresponding  hypergraph has size at least .
\end{guardwithcells}
\begin{proof}
Proof of the lower bound was given in the main document, hence we focus in the upper bound. \proofguardwithcells
\end{proof}
}{}
\iffalse
\newpage
\appendix
\section{Possible Lower Bound on the independent set of }
\mati{This is the result that unfortunately is incorrect. At least the method we used to prove does not work. Can we find a better reasoning?}

\begin{theorem}\label{Athe:ubis}
There exists a set  of  lines whose corresponding hypergraph  has a  maximum independent set of size .
\end{theorem}

\mati{This part uses the witness set definition and Lemma \ref{LB:basicgad}}
\iffalse
Before giving the proof of the theorem, we need to introduce some definitions and helpful results. For any  we consider the order in which we traverse the lines of  in the vertical line  from top to bottom. Although the permutation obtained will depend on  and , there will be exactly  different permutations in any set  of  lines. Let  be the set of different permutations that we can obtain, sorted from left to right. Each of these permutations is called a {\em snapshot} of .

With these definitions we can give an intuitive idea of our construction. Consider an arrangement  of  lines and let  be any two distinct lines of an independent set . These two lines must be consecutive in the ordering given by some snapshot (for example, the ordering right after the lines cross). Our approach is to cross these two lines with another pair of lines of . The four lines will form a quadrilateral, hence  cannot be an independent set. The main difficulty of the proof is that the line set must satisfy this property for any subset . In particular, we do not know at which snapshot will the lines of  cross.

We say that a set of snapshots  is a witness set of  if, for any two distinct lines , there exists a snapshot   that appear consecutively in . We say that the lines  and  form a {\em bad cell} at snapshot . Clearly the set  is a witness sets of quadratic size for any set of  lines. Since the size of the witness set has a direct impact on our bound, we first show how to construct a witness set of small size:

\begin{lemma}\label{ALB:basicgad}
For any  there exists an set  of  lines and a witness set  such that .
\end{lemma}
\begin{proof}
We construct the arrangement by induction on . For  our base gadget  simply consists of a single line. Note that the witness property is trivially true, since there don't exist two distinct lines in , hence we define .

As we are only interested in the ordering in which lines are crossed, we can do any transformation to a set  of lines, provided that  transformation preserves the permutations in the set . If we update the coordinates of the snapshots in the witness set accordingly, the witness property will still hold. In particular, we can transform a set  of lines so that they become almost parallel and have any desired slope. We call this operation the {\em flattening} of .

With this operation in mind we can do the induction step as follows: for any  generate a copy of gadget  and flatten the lines so that all lines have very small (but positive) slope and all crossings between lines occur below the horizontal line . Let  be the transformed set of lines and  be the reflexion of  with respect to line . Gadget  is defined as the union of  and  (see Figure \ref{Afig_basegadget}).

Observe that  satisfies the following properties:
\begin{figure}
   \begin{center}
     \includegraphics[scale=0.7]{fig_basegadget}
     \caption{Induction step in the gadget construction (left). The additional snapshots are also shown  (dashed vertical lines). The generated arrangement for  and its witness set is shown in the right. For clarity lines of  have been depicted as pseudolines.}   \label{Afig_basegadget}
   \end{center}

\end{figure}

\begin{itemize}
\item  Gadget  has size exactly . Moreover, any two lines cross exactly once.
\item  The witness set  of  also acts as witness set of .
\item  The lines of  and  intersect in a grid-like fashion, forming cells of size  and . \end{itemize}

Observe that property  certifies that the construction is a valid set of lines, while properties  and  help us obtain a witness set  of small size; the crossing between lines of different gadgets  can be guarded with  lines (see Figure \ref{Afig_basegadget}), whereas the crossings between lines of the same gadget can be guarded by the witness set . By construction we have that .

To finish the proof we must show that  is indeed a witness set of : Let  be any two lines of . If these lines belong to the same sub-gadget, we can apply induction and obtain that they must be consecutive in one of the first snapshots. Otherwise, the two lines belong to different sub-gadgets of , hence they will be consecutive at the latter snapshots.
\end{proof}
\fi

We first give an intuitive idea of our construction: consider an arrangement  of  lines and let  be any two distinct lines of an independent set . These two lines must be consecutive in the ordering given by some snapshot (for example, the ordering right after the lines cross). Our approach is to cross these two lines with another pair of lines of . The four lines will form a quadrilateral, hence  cannot be an independent set. The main difficulty of the proof is that the line set must satisfy this property for any subset . In particular, we do not know at which snapshot will the lines of  cross.

More formally, the construction of our problem instance is as follows; let  be the largest power of two such that . In the following we construct a line arrangement of size  in which there cannot be an independent set of large size. The instance is composed of several copies of , and the construction process is very similar to the one of Lemma \ref{LB:basicgad}. The main difference is that we will interlock the  different gadgets in a way that the size of the witness set does not increase. Let  and   be the witness set of  constructed in Lemma \ref{LB:basicgad}. Recall that we have ; for simplicity in the explanation we add snapshots to  so that . Let  be the permutations of , sorted in increasing values of the -coordinates.

For any , we construct the set  as a set of  lines as follows: for , we set . For larger values of , generate a copy of gadget  where the lines have been flattened so as to satisfy the following properties:
\begin{itemize}
\item The snapshot of  at line  is the -th witness of .
\item All lines of  have slope bigger or equal to one.
\item All lines cross the segment connecting the origin and point 
\item No two lines of  cross in the vertical strip 
\end{itemize}
\begin{figure}
   \begin{center}
     \includegraphics[scale=0.7]{fig_secondgadget}
     \caption{Construction of gadget . The dashed region corresponds to the strip , where no crossing between the lines of  occurs. Observe that, by construction, all lines of  must cross the segment connecting the origin and point  and the ray going upwards from point .}
     \label{Afig_secondgadget}
   \end{center}
\end{figure}

Let  be the transformed set of lines and  be the reflexion of  with respect to line  (see Figure \ref{Afig_secondgadget}). Gadget  is defined as the union of  and . Observe that the number of lines is doubled in each step of the construction. In particular,  the line arrangement  contains  copies of gadget . Let  be the copies of  in  (sorted from top to bottom). Recall that each copy of gadget  has  lines, hence the total number of lines in  is  (and thus ).

Unlike the previous construction, we will not add snapshots to . As a result, the witness property is lost for the whole arrangement . However, note that the set  will act as a witness set of any copy of  inside .

Let  be two copies of  in  and let  be any snapshot of . We say that  and  form a {\em large quadrilateral} if any two pairs of lines  that are consecutive in  (for ) form a quadrilateral in the arrangement of . Observe that, since each time we are duplicating the arrangement, the set  satisfies the following property:

\begin{lemma}\label{Alem_charac}
Let  be any copy of  in  and let . There exist  other copies of  that form a large quadrilateral with  at .
\end{lemma}

We now proceed to show Theorem \ref{Athe:ubis}:
\begin{proof}
Consider now any independent set  of  and let  be lines of  that are in the -th copy of , and let . By definition we have .

We say that two lines  that belong to the same copy of  form a {\em bad cell} at snapshot  if the the two lines are consecutive in . We say that two events are {\em independent} if their respective snapshots are different.

\begin{lemma}\label{Alem_minbad}
There are at least  independent bad cells in .
\end{lemma}
\begin{proof}
Let  be any line of  for some . This line must cross with all other lines of . Since   is a witness set of any copy of  (and in particular of  ), for any other line  there must be some snapshot in  in which  and  form are adjacent lines. If we repeat this process for all lines of  we obtain that there must be at least  (not necessarily independent) bad cells. Since line can only be adjacent to at most two other lines in the same snapshot, for any bad cell there can be at most another bad cell that shares the same snapshot. That is, at least  bad cells will be independent. By repeating this process for all  the Lemma is shown.

\mati{We first thought that the number of independent bad events was at least . This is because we did not think that a line could be adjacent to two other lines of  at the same time. In any case, this is not important since the additional  factor does not affect asymptotic values.}
\end{proof}

\begin{lemma}\label{Alem_maxbad}
For each bad event at the -th snapshot there must be at least  good cells.
\end{lemma}
\begin{proof}
Let  be the lines that form a bad cell at the -th snapshot. By Lemma \ref{Alem_charac} these two lines cross  different copies of gadget . In particular,  and  for a quadrilateral with any two consecutive lines in the copies of gadget . None of these gadgets can contain a bad cell (otherwise the lines of  would cover a quadrilateral, which contradicts with the definition of independent set). Since each copy of  contains  lines, the Lemma is shown.
\end{proof}

\begin{corollary}\label{Acor_maxbad}
There can be at most  independent bad cells in .
\end{corollary}
\begin{proof}
Recall that at one snapshot there are  lines, hence the number of good and bad cells in each snapshot is bounded . From Lemma \ref{Alem_minbad}, there can be at most  bad cells in the -th snapshot. By summing for all values of  we obtain the claim.

\mati{This Corollary is wrong! A bad cell forbids many other pairs of lines to be "good". But we did not think that two bad cells could share the same prohibitions. In fact, if we only count good and bad cells, I have a way of constructing a larger independent set.}
\end{proof}

Let  be the number of independent bad cells. By combining Lemmas \ref{Alem_minbad} and Corollary \ref{Acor_maxbad} we have  . Since  we conclude that the size of any independent set of  is at most . Recall that  is an arrangement of size , hence we obtain .

\end{proof}

This implies a new lower bound of  for the chromatic number.
\fi
\end{document}
