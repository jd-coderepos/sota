

\documentclass[11pt,a4paper]{article}
\pdfoutput=1 
\usepackage{a4wide,url}
\usepackage{graphicx,amssymb,amsmath,amsthm,xcolor}

\graphicspath{{figures/}{}}


\newtheorem{definition}{Definition}
\newtheorem{lemma}[definition]{Lemma}
\newtheorem{proposition}[definition]{Proposition}
\newtheorem{theorem}[definition]{Theorem}
\newtheorem{corollary}[definition]{Corollary}
\newtheorem{observation}[definition]{Observation}


\renewcommand{\topfraction}{0.9}	\renewcommand{\bottomfraction}{0.8}	\renewcommand{\floatpagefraction}{0.8}	\renewcommand{\dblfloatpagefraction}{0.8}	



\usepackage{bm}
\usepackage{enumitem}
\usepackage{mathabx}

\let\boxminusnew\boxminus
\let\boxplusnew\boxplus

\usepackage{MnSymbol}

\graphicspath{{figures/}{}}

\newcommand{\complain}[1]{\marginpar{{\color{red} ****}}{\color{red}\sf ***~~#1 ***}}



\newcommand*{\boxangle}{\mathrel{
\vcenter{\offinterlineskip
\hbox{\scalebox{.76}[0.85]{}}\vskip-.9ex\hbox{}}}}

\newcommand{\boxanglerota}{\rotatebox[origin=c]{90}{}}
\newcommand{\boxanglerotb}{\rotatebox[origin=c]{270}{}}

\newcommand{\boxtriangledown}{\rotatebox[origin=c]{180}{}}
\newcommand{\boxtriangleright}{\rotatebox[origin=c]{270}{}}
\newcommand{\boxtriangleleft}{\rotatebox[origin=c]{90}{}}

\newcommand{\boxvertnew}{\rotatebox[origin=c]{90}{}}




\begin{document}

\title{Finding a largest empty convex subset in space is W[1]-hard}

\author{Panos Giannopoulos\thanks{Institut f{\"u}r Informatik, Universit{\"a}t Bayreuth, 
			  \allowbreak Universi\-t{\"a}tsstra{\ss}e, 30, D-95447 Bayreuth, Germany,
			  \textsf \{christian.knauer, panos.giannopoulos\}@uni-bayreuth.de}.
\footnote{Research supported by the German Science Foundation (DFG) under grant Kn~591/3-1.}
\and
Christian Knauer\footnotemark[1]
}

\index{Panos Giannopoulos}
\index{Christian Knauer}


\maketitle

\begin{abstract}
We consider the following problem: Given a point set in space find a largest subset that is in convex position and whose convex hull is empty. We show that the (decision version of the) problem is W[1]-hard.
\end{abstract}




\section{Introduction}
\label{sec:introduction}

\noindent
{\textbf {Problem definition.}} Let  be a set of  points in  and . In the \textsc{Largest-Empty-Convex-Subset} problem we want to decide whether there is a set  of  points in convex position whose convex hull does not contain any other point of .  

\subsection{Results}

We show that \textsc{Largest-Empty-Convex-Subset} is W[1]-hard with respect to the solution size , under the extra condition that the solution set is \emph{strictly} convex, i.e., the interior of the convex hull of any of its subsets is empty. This means that (under standard complexity-theoretic assumptions) the problem is not fixed-parameter tractable with respect to , i.e., it does not admit an -time algorithm for any computable function  and any constant . See~\cite{FG06} for basic notions of parameterized complexity theory and~\cite{GKW08}  for survey of parameterized complexity results on geometric problems.

\subsection{Related work}

\textsc{Largest-Empty-Convex-Subset} has been shown to be NP-hard in last year's EuroCG \cite{KW12}. (In that paper, NP-hardness has been shown also for the more general version where the emptiness condition is dropped.) Several interesting questions were also raised such as whether the problem is fixed-parameter tractable with respect to the solution size and whether it admits a polynomial -approximation algorithm. Here, we give a negative answer to the first question.
Note that in the plane, the problem is solvable in polynomial time; see, for example, \cite{AR85},~\cite{DEO90}.

From a combinatorial point of view, there is a long history of results starting with the famous Erd\"os-Szekeres theorem \cite{ES35}, which states that for every  there is a number  such that every planar set of  points in general position contains  points in convex position. Horton \cite{Ho83} showed that this is not true when the emptiness condition is imposed: There are arbitrarily large sets that do not contain empty -gons. Results of this type exist also for higher dimensions; see~\cite{MS00}.



\section{Reduction}
\label{sec:reduction}

We show that \textsc{Largest-Empty-Convex-Subset} is W[1]-hard by an \emph{fpt}-reduction from the W[1]-hard -\textsc{Clique} problem \cite{FG06}: Given a
graph  and , decide whether  contains a clique of size .

\begin{figure}
\centering
	\includegraphics[width=.47\textwidth]{high_level_construction}
	\caption{High level schematic of the construction. The zoomed-in area shows points shared among gadgets on their common boundaries and some pairs of points inside each gadget that take part in a choice of an empty convex set, (partially) marked with dashed segments. }
	\label{fig:high_level_construction}
\end{figure}

\subsection{High level description}

We begin with a high-level description of the construction, see Fig.~\ref{fig:high_level_construction}. Initially, the construction will lie on the plane; later on, it will be lifted to the elliptic paraboloid with a (more or less) standard transform. The construction is organized as the upper diagonal part of a grid with  rows and  columns. The th row and column represent a choice for the th vertex of a clique in  and are made of  and  gadgets respectively. There are  choices and each choice is represented by a collection of empty convex subsets of points -- one subset with a constant number of points from each gadget.

Each gadget consists of  points within a rectangular region, which are organized in sets (of pairs) of collinear points. There is a constant number of such sets and, since we are looking for strictly convex subsets, only one pair of consecutive points per set can be chosen at any time. Certain choices are rendered invalid by additional points. Neighboring gadgets share the points on their common rectangle edge, see the zoomed-in area in Fig~\ref{fig:high_level_construction}. Through these common points, the choice of subsets is made consistent among the gadgets. In particular, the choice in the th row is made consistent with the choice in the th column via the `diagonal',  gadget in their intersection corner; consistency here means that they both correspond to the same choice of a vertex of . On the other hand, in the intersection of the th column with the th row, for every , there is a `cross',  gadget, which ensures that the choice in the column is propagated independently of the choice in the row and vice versa. 
Finally, the th column is `connected' to the th row, for every , by three additional gadgets. One of them, the `star',  gadget, encodes graph , i.e., it allows only for combinations of choices (in the column and the row) that are consistent with the edges in .

Locally, every valid subset from a gadget consists of points that are in strictly convex position and whose convex hull is empty. By lifting the whole construction to the paraboloid appropriately, we make sure that this property is true globally, i.e., for any set constructed from the local choices in a consistant manner. 

In total, the construction consists of a set  of  points  in , such that there exists a set  with the desired property and , for some function , if and only if  has a clique of size .

\subsection{Gadgets}
\label{sec:gadgets}

There are five different types of gadgets, and each type has a specific function, which is explained below.

\medskip
\noindent
{ \textbf {gadget.}} This gadget propagates a choice of pairs of points horizontally, see Fig.~\ref{fig:horizontal_gadget}. 
\begin{figure}[h]
\centering
	\includegraphics[width=.7\textwidth]{horizontal_gadget}
	\caption{The  gadget. Dashed parallelograms represent choices cancelled by points on . Parallelograms in full are not cancelled. There are  choices of convex empty -gons.}
	\label{fig:horizontal_gadget}
\end{figure}
It has  pairs of points ,  on the left and right side of its rectangle respectively, which correspond to the vertices of ; pairs corresponding to the same vertex are aligned horizontally. It also has  points inside the rectangle on a vertical line  as follows. For every two pairs  and , with , a point is placed such that it is inside the parallelogram  formed by the pairs but outside the parallelogram formed by any other two pairs. Effectively, this point cancels a choice of pairs that correspond to different vertices of . One point is placed above  on  and close to its boundary; similarly one point is placed below .  Due to the strict convexity condition, at most one pair per rectangle side and at most one pair on  can be chosen. Thus, there are  maximum size empty convex subsets. Each subset contains six points and is formed by three pairs: , , for some , and the pair of points on  that are closest to the parallelogram ; this latter pair is formed by the point that cancels  and the point that cancels . The  gadget is just a -rotated copy of the  gadget.

\medskip
\noindent
{ \textbf{gadget.}} This gadget has basically the same structure as the  gadget but propagates information diagonally. For completeness, it is shown in Fig \ref{fig:diagonal_simple_gadget}.
\begin{figure}[h]
\centering
	\includegraphics[width=.7\textwidth]{diagonal_simple_gadget}
	\caption[]{The  gadget.}
	\label{fig:diagonal_simple_gadget}
\end{figure}

\medskip
\noindent
{ \textbf{gadget.}} This gadget propagates information both horizontally and diagonally, see Fig.~\ref{fig:T_gadget}. It has  pairs of points , , and , on the left, right, and bottom side of its rectangle respectively. As before, an th pair corresponds to the th vertex of . There will be only  valid choices and three pairs per choice, namely, the th pair from each side. This is enforced by the strict convexity condition and by placing, for every , four additional points on two vertical lines  and  inside the rectangle. These points are placed \emph{outside} the convex -gon  that is formed by the points in the corresponding pairs. (The points are also outside every other -gon for .) See the example for  in Fig.~\ref{fig:T_gadget}. 

More specifically, looking at the gadget from top to bottom and from left to right, a point is placed on the intersection of  and the line through the second point of  and the second point of ; for the case of , the point is placed just below the -gon. A second point is placed on  just above the -gon. At the right side, two points are placed on  as follows. One point is placed on the intersection of  and the line through the second point of  and the first point of ; for , the point is placed just above the -gon. A second point is placed on the intersection of  and the line through the second point of  and the first point of ; for , the point is placed just below the -gon.

There are  maximum size empty convex subsets with 10 points each. A subset is formed by the points in the pairs , , , and the four points on  and  that are closest to the corresponding -gon. 

(Note that gadgets  and  (used later on) are just rotated copies of gadget .)
\begin{figure}
\centering
	\includegraphics[width=.7\textwidth]{T_gadget_corrected}
	\caption{The  gadget. There are  choices of convex empty -gons. An empty convex -gon, as described in the text, is shown only for .}
	\label{fig:T_gadget}
\end{figure}

\medskip
\noindent
{ \textbf{gadget.}} This gadget consists of eight subgadgets, which are very similar to the ones we have already described above. See Fig.~\ref{fig:high_level_cross_gadget}.
\begin{figure}[h]
\centering
	\includegraphics[width=.5\textwidth]{high_level_cross_gadget}
	\caption{The  gadget: a high level schematic.}
	\label{fig:high_level_cross_gadget}
\end{figure}
It can be thought of as having two inputs (at the upper and lower left corner) and two outputs (at the upper and lower right corner). It propagates the inputs (choices) independently from each other, one horizontally and one vertically. The input subgadgets  and  are respectively connected (through their left and bottom rectangle sides) to the  and  gadgets of the th row and th column of the global construction (Fig.~\ref{fig:high_level_construction}). The output subgadgets  and  are similarly connected to a  and  gadget. Note that the subgadgets  and  in the middle of the  gadget (Fig.~\ref{fig:high_level_cross_gadget}) as well as the subgadgets  and  are \emph{not} connected directly to any row or column of the global construction. Roughly speaking, the  gadget has the following function: it multiplexes the two inputs, then it mirrors them (vertically and horizontally), and then demultiplexes them. 
Next, we describe the subgadgets in some more detail.

\begin{figure}[h]
\centering
	\includegraphics[width=.5\textwidth]{horizontal_for_cross_gadget}
	\caption[]{The  subgadget.}
	\label{fig:horizontal_for_cross_gadget}
\end{figure}

The  subgadget is shown in Fig.~\ref{fig:horizontal_for_cross_gadget}. It is similar to the previously described  gadget in Fig.~\ref{fig:horizontal_gadget}. It has again  pairs of points  on the left side. The difference now is that there are  pairs of points , , on the right side of the gadget. The second index  basically encodes the choice coming from the input subgadget  at the lower left corner, which is communicated through the  and  subgadgets inbetween. Only the  pairs  and  constitute valid choices. The rest are cancelled by additional points as usually. Together with pairs of canceling points (which are also chosen as before) there are exactly  empty convex -gons, and these are of maximum size.

The input  subgadget propagates information vertically and is defined similarly to the  subgadget. It has  pairs of points  on the bottom side and  pairs  on the top side. The difference now is that only the  pairs  and  are valid.



The output subgadgets  and  are just mirrored images of their input counterparts. The  and  subgadgets are constructed in the same way as the  gadget in Fig.~\ref{fig:diagonal_simple_gadget} but have  valid -gons, while the  and  subgadgets are constructed in the same way as the  gadget in Fig.~\ref{fig:T_gadget} and have  valid -gons. 

\medskip
\noindent
{ \textbf{gadget.}} This gadget encodes the edges of the input graph . See Fig.~\ref{fig:star_gadget}. It is similar to gadget  (Fig.~\ref{fig:horizontal_gadget}) and allows only combinations of pairs that correspond to edges of the graph: for every \emph{non}-edge  of , a point is placed inside the parallelogram . 


\begin{figure}[h]
\centering
	\includegraphics[width=.65\textwidth]{star_gadget}
	\caption{The  gadget. Several examples of cancelled choices (in dashed) and of empty convex -gons (in bold) are shown.}
	\label{fig:star_gadget}
\end{figure}

\subsection{Lifting to }
Every corner of a gadget rectangle is lifted to the paraboloid with the map . See Fig~\ref{fig:lifting}. The images of the corners of each rectangle lie on one distinct plane (since the corners lie on a circle). The points in a gadget are projected orthogonally on the corresponding plane. This is an affine map and thus colinearity and convexity within a gadget is preserved. Each gadget now lies on a distinct facet (a parallelogram) of a convex polyhedron.

\begin{figure}[h]
\centering
	\includegraphics[width=.5\textwidth]{lifting}
	\caption{A -dimensional analogue of the lifting procedure.}
	\label{fig:lifting}
\end{figure}

\subsection{Correctness}
The total number of (sub)gadgets of each type together with the size of a largest valid (i.e., empty and convex) subset in a gadget of the type is
\begin{itemize}[leftmargin=2.2cm]
\setlength{\itemsep}{-\parsep}
\item[] : , ;
\item[] : , ;
\item[, , ] : , ;
\item[, , ] : , ;
\item[, , , ] : , ;
\item[] : , .
\end{itemize}

A global valid subset is formed by locally choosing one valid subset from every gadget in a consistent manner. When a largest locally possible subset (as given above) can be chosen, the global subset has size . As we will now prove, such a global subset corresponds to a -size clique of . Let  be the set of all the points in our construction.


\begin{lemma}
There exists an empty convex subset of  with  points if and only if  has a clique of size .
\end{lemma}
\begin{proof}
Suppose there exists a global valid subset of size . Then, a largest locally possible valid subset must be chosen from every gadget. Consider such a choice of subsets and let  be the vertex of  corresponding to the choice from the leftmost gadget of the th grid row. By construction, the subset corresponding to the same vertex  must be chosen from every other gadget in this row as well as every gadget in the th column. Consider the th row, for some . Through the  gadget that connects the th column to the th row, when , or the th column to the  row, when , the subset chosen from the th row must correspond to a vertex  such that  is an edge of . Hence  is a clique in . The converse is obvious.
\end{proof}

Therefore, we have shown the following

\begin{theorem}
\textsc{Largest-Empty-Convex-Subset} is W[1]-hard, under the condition of strict convexity.
\end{theorem}

\bibliographystyle{abbrv}


\begin{thebibliography}{1}

\bibitem{AR85}
D.~Avis and D.~Rappaport.
\newblock Computing the largest empty convex subset of a set of points.
\newblock In {\em Proc. of SCG '85}, 1985. ACM.

\bibitem{DEO90}
\mbox{D.~Dobkin, H.~Edelsbrunner, and M.~Overmars.}
\newblock \allowbreak Sear\-ching for empty convex polygons.
\newblock {\em Algorithmica 5}, 1990.

\bibitem{ES35}
P.~Erd\"os and G.~Szekeres.
\newblock A combinatorial problem in geometry.
\newblock {\em Compositio Math. 2}, 1935.

\bibitem{FG06}
J.~Flum and M.~Grohe.
\newblock {\em Parameterized Complexity Theory}.
\newblock Texts in Theoretical Computer Science. Springer, 2006.

\bibitem{GKW08}
P.~Giannopoulos, C.~Knauer, and S.~Whitesides.
\newblock {Parameterized Complexity of Geometric Problems}.
\newblock {\em Computer Journal}, 51(3):372--384, 2008.

\bibitem{Ho83}
J.~D. Horton.
\newblock Sets with no empty convex 7-gons.
\newblock {\em C. Math. Bull. 26}, 1983.

\bibitem{KW12}
C.~Knauer and D.~Werner.
\newblock Erd\"os-szekeres is {NP}-hard in  dimensions - and what now?
\newblock In {\em Abstracts of 28th EuroCG}, pages 61--64, 2012.

\bibitem{MS00}
W.~Morris and V.~Soltan.
\newblock The {Erd\"os-Szekeres} problem on points in convex position -- a
  survey.
\newblock {\em Bull. Amer. Math. Soc. 37}, 2000.

\end{thebibliography}


\end{document}
