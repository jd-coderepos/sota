\documentclass[number]{llncs}
\usepackage{amsmath}
\usepackage{amssymb}
\usepackage{tikz}
\usepackage[strings]{underscore}
\usepackage{url}
\usetikzlibrary{calc}


\newcommand{\all}{\forall}
\newcommand{\eq}{\Leftrightarrow}
\newcommand{\ex}{\exists}
\newcommand{\impl}{\Rightarrow}

\newcommand{\ama}{}
\newcommand{\set}[1]{\left\{#1\right\}}
\newcommand{\abs}[1]{\left|#1\right|}
\newcommand{\ceil}[1]{\left\lceil#1\right\rceil}
\newcommand{\floor}[1]{\left\lfloor#1\right\rfloor}
\newcommand{\setc}[2]{\left\{\left.#1\ \right|\ #2\right\}}
\newcommand{\eps}{\varepsilon}
\newcommand{\N}{\mathbb{N}}
\newcommand{\Z}{\mathbb{Z}}
\newcommand{\Q}{\mathbb{Q}}
\newcommand{\R}{\mathbb{R}}
\newcommand{\C}{\mathbb{C}}
\newcommand{\setOT}{\set{0,1,\dots,T-1}}
\newcommand{\rev}[1]{\overleftarrow{#1}}

\newcommand{\zerot}{\mathbb{T}}
\newcommand{\zerotp}{\mathbb{T}'}

\newenvironment{lp}[2]{
 \begin{array}{ll@{\ \ }c@{\ \ }l@{\quad}l}
  \textrm{#1}  & #2, &      &     & \\label{eq:capacities}
\sum_{i\in K} f^i_a(\theta) \leq u_a\qquad \text{for all , .}
\label{eq:weak-flow-conservation}
\begin{split}
\sum_{a = (\cdot,v) \in A} \int_0^{\theta-\tau_a} f^i_a(\xi)\ d\xi  - \!\!\!\!\sum_{a = (v,\cdot) \in A} &\int_0^\theta f^i_a(\xi)\ d\xi \geq 0\\
&\text{for all , , .}	
\end{split}
\label{eq:fulfill-demands}
\begin{split}
\sum_{a = (\cdot,v) \in A} \int_0^{T-\tau_a} f^i_a(\xi)\ d\xi  - \!\!\!\!\sum_{a = (v,\cdot) \in A} \int_0^T f^i_a(\xi)\ d\xi =&
\begin{cases}
d_i & \text{if ,}\\
-d_i & \text{if ,}\\
0 & \text{otherwise,}	
\end{cases}\\
&\qquad\qquad\text{for all .}
\end{split}

\sum_{j=0}^{k-1}\sum_{i\in K}\int_{k-2}^{k-1}f^i_{a_j}(\xi)\ d\xi \geq \sum_{i\in K}d_i = k+1\enspace.

\sum_{j=0}^{k-1}\sum_{i\in K}\int_{k-2}^{k-1}f^i_{a_j}(\xi)\ d\xi \leq k\enspace.

This contradiction concludes the proof.
\end{proof}
 
Together, Lemma~\ref{lemma:lowerbound} and Lemma~\ref{lemma:upperbound} yield the following theorem.
 
\begin{theorem}
There is a family of quickest multi-commodity flow instances for which the ratio between the optimal time horizon without and with intermediate storage, respectively, converges to~.
\end{theorem}

We mention another interesting consequence of the family of instances discussed above. If we reduce the demand of commodity~ to~, then there is a quickest flow with time horizon~ which does not use storage at intermediate nodes. However, as soon as we increase~ to~ for any~, the same arguments as in the proof of Lemma~\ref{lemma:upperbound} show that, if storage is prohibited, the optimal time horizon jumps above~. Thus, in contrast to the setting where storage is permitted (see Lemma~4.8 in~\cite{FleischerSkutella07}), the optimal time horizon is no longer a continuous function of the positive demand values if storage is prohibited.

\bibliographystyle{elsarticle-num} 
\bibliography{literature} 
\end{document}