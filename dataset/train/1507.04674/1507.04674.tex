\documentclass[11pt]{article}
\usepackage{amsthm,amsfonts,amsmath,times}
\usepackage{enumerate}
\usepackage{algo, extras, xspace, color}
\usepackage{algorithm,algorithmic}
\usepackage{boxedminipage, wrapfig}
\usepackage{hyperref,url}
\usepackage{tikz}
\usepackage{graphicx}
\usetikzlibrary{arrows}
\def\Comment#1{\textsl{#1\/}}

\textheight 9.1in
\advance \topmargin by -1.0in
\textwidth 6.7in
\advance \oddsidemargin by -0.8in
\newcommand{\myparskip}{3pt}
\parskip \myparskip

\newtheorem{fact}{Fact}[section]
\newtheorem{lemma}{Lemma}[section]
\newtheorem{theorem}[lemma]{Theorem}
\newtheorem{assumption}[lemma]{Assumption}
\newtheorem{definition}[lemma]{Definition}
\newtheorem{corollary}[lemma]{Corollary}
\newtheorem{prop}[lemma]{Proposition}
\newtheorem*{claim}{Claim}
\newtheorem{remark}[lemma]{Remark}
\newtheorem{prob}{Problem}
\newtheorem{conjecture}{Conjecture}
\newtheorem{question}{Question}



\newcommand{\opt}{\textsc{OPT}}
\newcommand{\etal}{{\em et al.}~}
\DeclareMathOperator*{\Ex}{\mathbb{E}}

\renewenvironment{proof}{\vspace{-0.1in}\noindent{\bf Proof:}}{\hspace*{\fill}\par}
\newenvironment{proofof}[1]{\smallskip\noindent{\bf Proof of #1:}}{\hspace*{\fill}\par}
\newenvironment{proofsketch}{\vspace{-0.1in}\noindent{\bf Proof Sketch:}}{\hspace*{\fill}\par}

\def\eps{\varepsilon}
\def\bar{\overline}
\def\floor#1{\lfloor {#1} \rfloor}
\def\ceil#1{\lceil {#1} \rceil}
\def\script#1{\mathcal{#1}}
\def\bx{\textbf{x}}
\def\tbx{\tilde{\textbf{x}}}
\def\sp{\;|\;}

\def\UndirMC{\textsc{MC}\xspace}
\def\MC{\textsc{Edge-wt-MC}\xspace}
\def\DirMC{\textsc{Dir-MC}\xspace}
\def\2DirMC{\text{-Bi-Cut}\xspace}
\def\NodeMC{\textsc{Node-wt-MC}\xspace}
\def\DirMCRel{\textsc{Dir-MC-Rel}\xspace}
\def\UndirMCRel{\textsc{MC-Rel}\xspace}
\def\NodeMC{\textsc{Node-wt-MC}\xspace}
\def\NodeMCRel{\textsc{Node-MC-Rel}\xspace}
\def\DirMulticut{\textsc{Dir-Multicut}\xspace}
\def\MCut{\textsc{Multiway Cut}\xspace}

\def\CR{\textsc{LE-Rel}\xspace}
\def\MCSAfull{Minimum Submodular-Cost Allocation\xspace}
\def\MCSA{\textsc{MSCA}\xspace}
\def\monMCSA{\textsc{Monotone-MSCA}\xspace}
\def\SubMPfull{\textsc{Submodular Multiway Partition}\xspace}
\def\SubMP{\textsc{Sub-MP}\xspace}
\def\kSubMP{\textsc{k-way-Sub-MP}\xspace}
\def\kHMC{\textsc{k-way-Hypergrap-Cut}\xspace}
\def\SubMPRel{\textsc{SubMP-Rel}\xspace}
\def\SymSubMPfull{\textsc{Symmetric Sub-MP}\xspace}
\def\SymSubMP{\textsc{Sym-Sub-MP}\xspace}
\def\kSymSubMP{\textsc{k-way-Sym-Sub-MP}\xspace}
\def\MMCSA{\textsc{Monotone MSCA}\xspace}
\def\HMCfull{Hypergraph Multiway Cut\xspace}
\def\HMC{\textsc{Hypergraph-MC}\xspace}
\def\AHMCfull{\textsc{Hypergraph Multiway Cut}\xspace}
\def\AHMC{\textsc{Hypergraph-MC}\xspace}
\def\HMPfull{\textsc{Hypergraph Multiway Partition}\xspace}
\def\HMP{\textsc{Hypergraph-MP}\xspace}
\def\ML{\textsc{ML}\xspace}
\def\SubMLfull{\textsc{Submodular Cost Labeling}\xspace}
\def\SubML{\textsc{Sub-Label}\xspace}
\def\MSP{\textsc{Monotone MSCA}\xspace}
\def\CKR{\textsc{CKR-Rounding}\xspace}
\def\HR{\textsc{Half-Rounding}\xspace}
\def\KT{\textsc{KT-Rounding}\xspace}
\def\SymMPR{\textsc{SymSubMP-Rounding}\xspace}
\def\SubMPRH{\textsc{SubMP-Half-Rounding}\xspace}
\def\SymMLR{\textsc{SymSubLabel-Rounding}\xspace}
\def\optcr{{\textsc{OPT}_{\textsc{frac}}}\xspace}
\def\lovasz{Lov\'{a}sz\xspace}

\newcommand{\cP}{\mathcal{P}}

\begin{document}
\title{Simple and Fast Rounding Algorithms for Directed and Node-weighted Multiway Cut}

\author{
Chandra Chekuri
\thanks{
Dept. of Computer Science, University of Illinois, Urbana, IL 61801.
Supported in part by NSF grant CCF-1319376. {\tt chekuri@illinois.edu}}
\and
Vivek Madan
\thanks{
Dept. of Computer Science, University of Illinois, Urbana, IL 61801.
Supported in part by NSF grant CCF-1319376.
{\tt vmadan2@illinois.edu}}
}

\date{\today}

\maketitle

\thispagestyle{empty}
\begin{abstract}
  We study the multiway cut problem in {\em directed} graphs and one
  of its special cases, the {\em node-weighted} multiway cut problem
  in {\em undirected} graphs.  In {\sc Directed Multiway Cut} (\DirMC)
  the input is an edge-weighted directed graph  and a set of
   terminal nodes ; the goal is
  to find a min-weight subset of edges whose removal ensures that there
  is no path from  to  for any . In {\sc
    Node-weighted Multiway Cut} (\NodeMC) the input is a node-weighted
  undirected graph  and a set of  terminal nodes
  ; the goal is to remove a
  min-weight subset of nodes to disconnect each pair of
  terminals. \DirMC admits a -approximation \cite{NaorZ01}
  and \NodeMC admits a -approximation
  \cite{GargVY04}, both via rounding of LP relaxations.
  Previous rounding algorithms for these problems, from nearly twenty
  years ago, are based on careful rounding of an {\em
    optimum} solution to an LP relaxation. This is particularly true
  for \DirMC for which the rounding relies on a custom LP formulation
  instead of the natural distance based LP relaxation \cite{NaorZ01}.

  In this paper we describe extremely simple and near linear-time
  rounding algorithms for \DirMC and \NodeMC via a natural distance
  based LP relaxation. The dual of this relaxation is a special case
  of the maximum  multicommodity flow problem. Our
  algorithms achieve the same bounds as before but have the
  significant advantage in that they can work with {\em any feasible}
  solution to the relaxation. Consequently, in addition to obtaining
  ``book'' proofs of LP rounding for these two basic problems, we also
  obtain significantly faster approximation algorithms by taking
  advantage of known algorithms for computing near-optimal solutions
  for maximum multicommodity flow problems. We also investigate 
  lower bounds for \DirMC when  and in particular prove that
  the integrality gap of the LP relaxation is  even in directed
  planar graphs.
\end{abstract}

\newpage
\section{Introduction}
We study several variants of the multiway cut problem in graphs (also
referred to as the mult-terminal cut problem). In the classical
- cut problem the input consists of a graph  and two
distinct nodes ; the goal is to separate  from  by removing
a minimum cost set of edges and/or nodes. In the multiway cut problem
the input is a graph  and a set 
of  nodes from  called terminals; the goal is to separate the terminals
from each other at minimum cost by removing edges and/or nodes. We
describe the three main variants that are of interest to us.

\medskip 
\noindent {\sc Multiway Cut} (\MC): The input is an undirected
graph  along with non-negative edge weights 
and a set  of terminals. The goal
is to find a min-cost set of edges  such that in  there is no path from  to  for .

\medskip 
\noindent {\sc Node-Weighted Multiway Cut} (\NodeMC): The input is an
undirected graph  along with non-negative node weights  and a set  of terminals. The
goal is to find a min-cost set of nodes  such that
in  there is no path from  to  for .\footnote{
In this definition terminals are allowed to be removed. If they are not allowed
to be removed we can simply make their weight .}



\medskip 
\noindent {\sc Directed Multiway Cut} (\DirMC): The input is a
directed graph  along with non-negative edge weights  and a set  of terminals. The
goal is to find a min-cost set of edges  such that
in  there is no path from  to  for .

\begin{remark}
  \DirMC with  is \emph{not} the same as the - cut problem.
  The goal is to separate  from  \emph{and}  from .
  In fact \DirMC with  is NP-Hard~\cite{GargVY94}. 
\end{remark}

The complexity of the multiway cut problem and its variants have been
extensively studied since the paper of Dahlhaus \etal
\cite{DahlhausJPSY92}. They showed that \MC with  is NP-Hard; it
was later observed that the problem is also APX-hard to approximate. This
is in contrast to the case of  which can be solved in 
polynomial-time in undirected graphs via a reduction to the
- minimum-cut problem.

\MC reduces in an approximation preserving fashion to \NodeMC which in
turn reduces in an approximation preserving fashion to \DirMC
\cite{GargVY04}; it is also easy to see that in the directed case,
node-weighted and edge-weighted versions are equivalent. The current
best approximation ratio for \MC stands at  due to Sharma and
Vondr\'ak \cite{SharmaV14}. For \NodeMC a  approximation is
known from the work of Garg, Vazirani and Yannakakis \cite{GargVY04},
and for \DirMC a  approximation is known from the work of Naor and
Zosin \cite{NaorZ01}.  {\sc Vertex Cover} reduces to \NodeMC and
\DirMC in an approximation preserving
fashion~\cite{GargVY04}. Assuming  {\sc Vertex Cover} is
hard to approximate to within a factor of  \cite{DinurS05}, and
assuming the Unique Games Conjecture it is hard to approximate to
within a factor of  for any fixed 
\cite{KhotR08}. These hardness results apply to \NodeMC and \DirMC and
show that \MC is provably easier to approximate than them.

Our focus in this paper is on approximation algorithms for \NodeMC and
\DirMC. The known algorithms are based on rounding suitable LP
relaxations for the problems. For both problems there is a simple and
natural LP relaxation based on distance variables on nodes/edges; see
Section~\ref{sec:dir-mc} and \ref{sec:node-mc}. (We note that a
similar relaxation applies to the more general {\sc Multicut} problem
and that dual of the LP relaxation corresponds to the LP for maximum
multicommodity flow.)  \iffalse In fact the approximation bounds of
 and  also establish the same upper bounds on the
integrality gap of this natural relaxation.  However, the rounding
schemes are far from trivial.  \fi For \NodeMC the algorithm of Garg,
Vazirani and Yannakakis \cite{GargVY04} shows that any {\em optimum}
solution to the relaxation can be converted to a half-integral optimum
solution which can then be rounded easily.  The situation for \DirMC
is much more involved. Unlike the case of \NodeMC, half-integral
optimum solutions may not exist for the relaxation even for .
Garg \etal \cite{GargVY94} obtained an -approximation via
the relaxation using ideas from approximation algorithms for multicut
\cite{GVY}.  Naor and Zosin obtained a -approximation for \DirMC in
an elegant, surprising and somewhat mysterious fashion. They write a
different LP relaxation called the {\em relaxed multiway flow}
relaxation which is within a factor of  of the natural relaxation,
and show that an {\em optimum} solution to this new relaxation can be
rounded without any loss in the approximation. This gives an indirect
proof that the natural relaxation has an integrality gap of at most
. The proof of correctness crucially relies on complementary
slackness properties of the optimum solution and is partly inspired by
the ideas in \cite{GargVY04}. The idea of using a relaxed multiway
flow is inspired by earlier work on the subset feedback vertex problem
\cite{EvenNZ00}.

The algorithms of \cite{GargVY04} and \cite{NaorZ01} are from almost
twenty years ago. During this intervening years no alternative
algorithms or rounding schemes have been obtained for these basic
problems. We observe that for the case of \MC there is an extremely
simple rounding scheme that converts any fractional feasible solution
to a multiway cut with a loss of a factor of  (see
\cite{Vazirani-book}). The algorithm picks a random  and for each terminal  removes the edges leaving the
ball  of nodes contained within a radius 
around  (with respect to distances given by the LP solution);
more formally the output is .

In this paper we show that very simple algorithms which are essentially
similar in spirit to the above scheme also work for \DirMC and \NodeMC!
\begin{itemize}
\item The rounding algorithms are extremely simple and natural to
  describe, and in retrospect also to analyze. 
\item The algorithms only require a feasible solution to the natural
  LP relaxation and not necessarily an optimum solution.
\item Given a feasible fractional solution, the rounding algorithms
  can be implemented in time that is similar to what is required for
  one single-source shortest path computation. The deterministic
  version requires an additional logarithmic factor.
\end{itemize}

In addition to algorithmic results we also obtain some lower bound
results for \DirMC with ; the goal is to separate  from 
{\em and}  from  in a directed graph ; subsequently 
we refer to this special case as \2DirMC. We prove that the
natural LP relaxation has an integrality gap of  for
\2DirMC even in {\em planar} directed graphs.

We believe that our algorithms and analysis will be useful for related
problems. Indeed one of our motivations for simplifying the rounding
schemes for \DirMC and \NodeMC came from attempts to obtain algorithms
for a problem with applications to network information theory
\cite{KKCV15}. A significant consequence of our rounding algorithms
are much faster approximation algorithms for \NodeMC and \DirMC in
both theory and practice. Solving the LP relaxations for \NodeMC and
\DirMC to optimality is quite challenging. The options are to use the
Ellipsoid method or to use a compact formulation with a very large
number of variables and constraints. As we remarked earlier, the dual
of the natural LP relaxation for these problems is the maximum
multicommodity flow problem. Combinatorial fully-polynomial time
approximation schemes for solving these multicommodity flow problems
have been extensively investigated in theoretical computer science and
mathematical programming with a number of techniques developed over
the years; we refer the reader to
\cite{PST95,GrigoriadisK94,Young95,Bienstock-book,GargK,Fleischer,BienstockI06,Madry10}. Thus,
a fast -approximation for the LP relaxation for \NodeMC and
\DirMC can be obtained using these methods. The fastest theoretical
algorithms run in time  \cite{Fleischer,GargK}
or in even faster  time \cite{Madry10} under
some mild conditions; here  is the number of edges and  is the
number of nodes in  and  suppresses poly-logarithmic
factors. Note that these running times are independent of .  Our
rounding algorithms can convert such an approximate feasible solution
to an integral cut in near-linear time with a factor of  loss in
the cost.  Thus, we can obtain provably fast -approximation
algorithms. Since our focus is on the rounding algorithms we do not go
into further details of specific algorithms or running times for
solving the relaxation.

We refer the interested reader to quickly jump to
Section~\ref{sec:dir-mc} to see the simplicity of the rounding scheme
and its analysis for \DirMC that achieves a bound of . This also
applies to \NodeMC via a simple reduction to \DirMC. We also discuss
some new observations on the hardness of the problem when . In
Section~\ref{sec:node-mc} we give a slightly different rounding scheme
for \NodeMC that achieves an improved bound of , matching
the known ratio from \cite{GargVY04}.

\subsection{Other related work}
\label{subsec:related-work}

The natural LP relaxation for \MC has an integrality gap of .
Approximation algorithms for \MC received substantial attention
following the breakthrough work of Calinescu, Karloff and Rabani
\cite{CalinescuKR98}. They developed a new ``geometric'' LP relaxation
(henceforth referred to as the CKR-relaxation) which they used to
obtain a -approximation. The integrality gap of the
CKR-relaxation, and consequently the approximation ratio, was improved
subsequently to  by Karger \etal \cite{KargerKSTY99}, to
 by Buchbinder \etal \cite{BuchbinderNS13}, and to 
the currently best known bound of  by Sharma and Vondr\'ak
\cite{SharmaV14}. For  a tight bound of  is known
\cite{CheungCT06,KargerKSTY99}. It is also known that assuming the
Unique Games Conjecture, for any fixed , the approximability
threshold for \MC coincides with the integrality gap of the
CKR-relaxation \cite{ManokaranNRS08}.

The CKR-relaxation makes use of the observation that \MC can be viewed
as a partition problem where the goal is to partition the node set
 into  parts  to minimize  subject to the constraint that for ,
.  {\sc Submodular Multiway Partition} (\SubMP) is a
generalization from the setting of graphs to arbitrary submodular
functions. Here we are given a non-negative submodular function  over the ground set  along with terminals
.  The goal is to partition  into
 to minimize  subject to the
constraint that  for .  If  is
symmetric, as in the case of the undirected graph cut function, we
obtain the {\sc Symmetric Submodular Multiway Partition} (\SymSubMP)
problem. These problems were considered by Zhao, Nagamochi and Ibaraki
\cite{ZhaoNI05} who analyzed greedy-splitting algorithms, and more
recently by Chekuri and Ene \cite{ChekuriE11b} who used a
Lov\'asz-extension based convex relaxation. Interestingly, the convex
relaxation when specialized to \MC yields the CKR-relaxation.Chekuri and Ene
\cite{ChekuriE11b} obtained a -approximation for \SymSubMP
and -approximation for \SubMP. Ene, Vondr\'ak and Wu \cite{EneVW13}
improved the bound for \SubMP to  and also obtained
lower bound results in the oracle model.

\NodeMC cannot be viewed as a partition problem
directly. Nevertheless, it can be seen that \NodeMC
is equivalent to \AHMCfull problem (\AHMC) which is a generalization
of \MC from graphs to hypergraphs. \AHMC can be
cast as a special case of \SubMP (note that the reduction uses a
non-symmetric submodular function ) and thus \NodeMC can be
indirectly reduced to a partition problem. This leads to an
alternative -approximation for \NodeMC based on the
Lov\'asz-extension based relaxation for \AHMC. This relaxation does
not result in a better worst-case approximation than the
distance-based relaxation, however, it appears to be strictly stronger
in that it improves the approximation ratio in special some cases as
observed in \cite{ChekuriE11}. No fast approximation
algorithms are known to solve this convex relaxation.

Finally we mention the {\sc Multicut} problem where the goal is to
separate a given set of  node-pairs  in
a given graph at minimum-cost. One can consider undirected graphs with
edge weights, undirected graphs with node weights and directed graph
with edge weights.  These versions generalize the corresponding
multiway cut problems. The best known approximation ratio for {\sc
  Multicut} in undirected graphs is  \cite{GVY,GargVY94}
while the best known bounds in directed graphs is  \cite{AgarwalAC07}. Moreover, it is known from
the work of Chuzhoy and Khanna \cite{ChuzhoyK09} that the problem in
directed graphs is inapproximable to a factor better than
.


\section{LP Relaxation and rounding for \DirMC}
\label{sec:dir-mc}

\DirMC can be naturally formulated as an integer linear program with
variables ,  which indicate whether  is
cut or not. Let  be the set of all directed paths from 
to  in .  The constraint that  is separated from  by
the cut can be enforced by requiring that 
for each .  This leads to the following LP relaxation
where the integer constraint  is replaced by . We can without loss of generality drop the constraint .

\begin{figure}[htb]
  \centering
\begin{boxedminipage}{0.5\linewidth}
\vspace{-0.2in}

\end{boxedminipage}
  \caption{LP Relaxation for \DirMC}
  \label{fig:dirmc-lp}
\end{figure}

The main result of the paper is the following theorem.

\begin{theorem}\label{thm:directed_cut_approximation}
  There is a randomized algorithm that given a
  feasible solution  to {\sc \DirMCRel} returns a feasible
  integral solution of expected cost at most , and
  runs in  time. The algorithm can be 
  derandomized to yield a deterministic -approximation algorithm 
  that runs in  time. Here, .
\end{theorem}

We now describe the simple randomized ball-cutting algorithm that
achieves the properties claimed by the theorem. Let  be a
feasible solution to . For any two nodes  we
define  be the shortest path length from  to  using
edge lengths given by . For notational simplicity we omit the
subscript  since there is little chance of confusion. The algorithm
adds new nodes  and adds the edge set  and sets the  value of each of these new edges to
. Note that, this is in effect a reduction of the \DirMC for the
given instance to a \DirMulticut instance which requires us to
separate the pairs , . The solution 
augmented with the extra nodes and edges leads to a feasible
fractional solution for this \DirMulticut instance. Our algorithm,
formally described below, is very simple. We pick a random  and take the union of the cuts defined by balls of radius
 around each . More formally let  be the set of
all nodes at distance at most  from .  Then the algorithm simply
outputs  where 
denote the set of outgoing edges from .

\begin{algorithm}
	\caption{Rounding for \DirMC}
	\label{alg:directed_cut_rounding_scheme}
	\begin{algorithmic}[1]
		\STATE Given a feasible solution 
                to \DirMCRel
		\STATE Add new vertices , edges  
              for all  and set  
		\STATE Pick  uniformly at random
\STATE 
		\STATE Return 
	\end{algorithmic}
\end{algorithm}

Note that  is a random set of edges that depends on the choice of
. We denote by  the set of edges output by the
algorithm for a given .

\begin{lemma}
  \label{lem:dir-feasibility}
  If  is a feasible fractional solution to \DirMCRel, 
  is a feasible multiway cut for  for any . Thus, Algorithm \ref{alg:directed_cut_rounding_scheme}
  always returns a feasible integral solution given a feasible .
\end{lemma}
\begin{proof}
  Fix any  and . Since
   for all , we have that  for all . Moreover, by feasibility of
  , we have  for otherwise there will be a path
  of length less than  from some  to  where .
  Therefore  because .
  Therefore,  has no path from  to
   for any . Since , it follows that there is no path in  from  to  for any .
\end{proof}

We now bound the probability that any fixed edge  is cut by the
algorithm, that is, .  Note that  may be {\em
  simultaneously} cut by several  for the same value of 
but we are only interested in the probability that it is included in
.

\begin{lemma}
  \label{lem:prob-e-cut}
  For any edge , .
\end{lemma}
\begin{proof}
  Let . Rename the terminals such that . This implies that 
  
  and 
  
  Edge  if and only if ; we have that . 
  Defining the interval  as , we see
  that  only if . 
  However, from the property that ,
  . Thus, 
   only if  or  and
  since  and  are both at most  long and 
  is chosen uniformly at random from ,
  
\end{proof}

\begin{corollary}
  , the expected cost of , is at most .
\end{corollary}




\iffalse
\begin{lemma}
Expected cost of a cut returned by algorithm \ref{alg:directed_cut_rounding_scheme} is at most twice the fractional solution cost\Big(\Big).
\end{lemma}
\begin{proof}
Fix an edge  and rename the terminals such that . This implies that . Edge  is included in  if for some ,  i.e. . Note that .

 If , then . If , then . So, an edge  is cut only if  or . 

Let  be an indicator variable denoting if  belongs to  or not. 

\end{proof}


\begin{proof}(Theorem \ref{thm:directed_cut_approximation})

\end{proof}
\fi

\paragraph{Running time analysis and derandomization:}
A natural implementation of Algorithm
\ref{alg:directed_cut_rounding_scheme} would first choose  and
then compute  for each .  This can be easily
accomplished via  executions of Dijkstra's single-source shortest
path algorithm, one for each , leading to a running time of
 where  and . However, by taking
advantage of our analysis in Lemma~\ref{lem:prob-e-cut}, we can obtain
a run time that is equivalent to a single execution of Dijkstra's
algorithm.

Consider a slight variation of Algorithm
\ref{alg:directed_cut_rounding_scheme}. For each edge ,
define two intervals  and , where  are the two terminals
from which  is the closest in terms of distance. We pick  uniformly at random and include  in  iff  or . The analysis in Lemmas
\ref{lem:dir-feasibility} and \ref{lem:prob-e-cut} shows that even
this modified algorithm outputs a feasible cut whose expected cost is at
most . Note that the edges cut by this modified
algorithm may be a strict superset of the edges cut by Algorithm
\ref{alg:directed_cut_rounding_scheme}. The advantage of the modified
algorithm is that we only need to calculate  and  for
each edge . To do this, for each node , we need to find
the two terminals from which  is the closest and their
corresponding distances. More formally, consider the following
-nearest-terminal problem.

\begin{prob}
  Given a directed graph  with non-negative edge-lengths, a set  of  terminals, and an integer , for each
  vertex , find the  terminals from which  is the closest
  among the terminals and their corresponding distances. In other
  words for each  find the  smallest values in 
   where .
\end{prob}

The above problem can be solved via a randomized algorithm using
hashing that runs in expected time , which
corresponds to  executions of Dijkstra's algorithm.  It can also be
solved in  time via a deterministic
algorithm. See \cite{HarPeled15} who refers to this as the
-nearest-neighbors problem.

Using the algorithm for the -nearest-terminal problem with ,
we can calculate  and  for each  in  time\footnote{One can easily derive the  case from first
  principles also.}.  We then chose  uniformly at random from
 and cut  if  lies in one of the range  or
. This gives us a -approximate randomized algorithm with
running time .


\iffalse
\begin{prob}
  Given a directed  with non-negative edge-weights, a subset
   of terminals, for each vertex , find two closest
  vertices(and distances) in  i.e. find  and
   such that , we have . Here,
   denotes the shortest path length from  to  using edge
  lengths given by .
\end{prob}
This problem, referred to as the can be solved in 
time \cite{HarPeled15}.
\fi


We can derandomize the algorithm by computing the cheapest cut among
all  as follows. Once  and  are
computed for each  we sort the  end points of these 
intervals; let them be .  We observe that it suffices to evaluate the cut value
at each of these values of .  A simple scan of these 
points while updating the cut-value at each end point can be
accomplished in  time. Sorting the end points takes  time.  This leads to a deterministic -approximation algorithm
with running time .

\iffalse
\paragraph{Running time analysis and derandomization:}
The randomized algorithm can be easily implemented. The main step is
to find  for each  once  is chosen.  We can
accomplish this by computing  for all 
and all . This takes  single-source shortest path
computations on a graph with  nodes and  edges.  Using the
-time implementation of Dijkstra's algorithm for
single-source shortest paths, we obtain a running time of  for the randomized algorithm.  However, we can, in fact
obtain a running time of  by taking advantage of the
analysis in Lemma~\ref{lem:prob-e-cut}. We observe that, to obtain, a
-approximation it suffices to compute for each node , only the
minimum and second minimum from .
This can be accomplished in time that is essentially two single-source
computations. We give details in the appendix.

The algorithm can be easily derandomized by computing  for
each  and outputting the cheapest cut among all of them.
There are only a finite number of distinct cuts that need to be considered.
Again, taking advantages of the analysis in  Lemma~\ref{lem:prob-e-cut},
we can find a -approximation is  time. Details in appendix.
\fi

\subsection{\DirMC with }
\label{sec:2-terminal-dir-mc}
In this section we address \DirMC with  which we refer to as
\2DirMC. We believe this is an interesting problem on its own as it is
related closely to the classical - cut problem.  As we remarked
earlier, \2DirMC is NP-Hard and APX-Hard to approximate. This was
shown in \cite{GargVY94,GargVY04} via a simple approximation preserving
reduction from \MC with .  Another consequence of the reduction
is that the integrality gap of \DirMCRel for \2DirMC is at least
. On the other hand no ratio better than  is known for
\2DirMC. This naturally raises the following question.

\begin{question}
  What is the integrality gap of \DirMCRel for \2DirMC? What is
  the approximability of \2DirMC?
\end{question}

We obtain two theorems. The first one shows that the integrality gap 
for \2DirMC is .

\begin{theorem}\label{thm:2-terminal-directed-integrality-gap}
  Integrality gap of \DirMCRel for \2DirMC is  even in planar directed
  graphs.
\end{theorem}

The second theorem slightly extends a result in \cite{GargVY04}.

\begin{theorem}
\label{thm:2-terminal-hardness}
  There is an approximation preserving reduction from -terminal
  \NodeMC to \2DirMC.
\end{theorem}

We raise the following question.

\begin{question}
  Can we prove a factor  hardness of approximation for \DirMC under
  the assumption that ? Does a factor of  hardness hold
  for \2DirMC even under the Unique Games conjecture?
\end{question}

\paragraph{Integrality gap construction:} 
Proof of Theorem~\ref{thm:2-terminal-directed-integrality-gap}
is based on recursively defined sequence of graphs
 with increasing integrality gap; we will use
 to denote the integrality gap (we also refer to this
as the flow-cut gap) in .  The two terminals
will be denoted by . The symmetry in the construction will ensure
that in  the - cut value will be equal to the - cut
value; we refer to these common values as the one-way cut value and
the optimum value of a cut that separates  from  and  from
 as the two-way cut value. The graph  is shown in
Fig~\ref{fig:gap} and it is easy to see that .

\begin{figure}[htb]
\centering
\includegraphics[scale=0.8]{2-terminal-directed-integrality-gap.pdf}
\caption{ on the left and constructing  from  shown
on the right.}
\label{fig:gap}
\end{figure}

The iterative construction of  from  is shown at a
high-level in figure \ref{fig:gap}. A formal description is as
follows. To obtain  with terminals  we start with two
copies of  with terminals  and  (denoted 
by ) and two new vertices . We set ,  and
identify  and  as the center vertex  shown in the
figure. We add edges  and  with weight  and four
other edges  each with weight
infinity. Finally we scale the weights of the edges of  and 
such that the two-way cut value in each of them is
. It is easy to observe inductively that
the each graph in the sequence is planar and moreover the graph can be
embedded such that  and  are on the outer face.  The analysis of
the integrality gap of this construction can be found in the appendix.

Subsequent to our construction, Julia Chuzhoy obtained
an alternative non-recursive construction with an integrality gap of
 for \2DirMC. 

\paragraph{Reduction from -terminal \NodeMC to \2DirMC:} 
Given a \NodeMC instance with graph  and set of terminals
, Figure~\ref{fig:node_to_directed_reduction}
shows the ingredients of a reduction to \DirMC instance with graph
 and terminals . This is a slight modification of the
reduction from three-terminal \MC to \2DirMC given in \cite{GargVY04}.
It is convenient to consider the node-weighted
version of \DirMC which is equivalent to the edge-weighted version.
Formally  is obtained from  by the addition of two new nodes
 which are connected to the terminals via directed edges of
infinite weight as shown in the figure. Each edge  is
replaced by two directed edges  and  and the weights of
the nodes of  remain the same. We will assume without loss of generality
that the terminals  have infinite weight.
A relatively simple case analysis shows that 
 is a feasible node-multiway cut for the terminals
 in  iff  is a feasible node-multiway cut in 
 for . This type of reduction does not seem to generalize
beyond four terminals.

\iffalse
Given a \NodeMC instance with graph  and set of terminals
, figure \ref{fig:node_to_directed_reduction}
shows a reduction to \DirMC instance with graph  and terminals
. For each non-terminal vertex  in , we have two
terminals  and an edge  with weight . For
each edge  we have two new edges  and
 of infinite weight. We also have two new vertices 
and edges:
. Weights
of all of these edges are infinite. It is easy to argue that  is a
feasible solution to \MC in G iff  is a
feasible solution to \NodeMC in . Infinite weight edges are never
picked in optimum solution for \NodeMC. Thus, minimum cut for \MC in
 is of same value as the minimum cut for \NodeMC in .
\fi

\begin{figure}[htb]
\centering
\includegraphics[scale=1]{node_directed_reduction}
\caption{Reduction from 4-terminal \NodeMC to \2DirMC. Non-terminal
vertices are not shown.}
\label{fig:node_to_directed_reduction}
\end{figure}

Garg \etal \cite{GargVY04} showed that \DirMCRel does not necessarily
have half-integral opitmum solutions. In
Section~\ref{sec:fractionality} we extend their example to show that
for every non-negative integer  there exist instances for which 
there is no optimum solution to \DirMCRel that is  integral.

\section{LP Relaxation and rounding for \NodeMC}
\label{sec:node-mc}

The LP relaxation for the \NodeMC
is similar to the one for \MC. We
have a variable  for each  which indicates
whether to remove  or not. We can assume without loss of generality
that we cannot remove the terminals  and moreover
that they form an independent set. This can be accomplished by adding
to each original terminal  a new dummy terminal  and
adding the edge . Let  be the set of all 
paths between  and  in . Note that in the undirected graph
case we do not need to distinguish  from . Let  be the set of terminals.
\begin{figure}[htb]
  \centering
\begin{boxedminipage}{0.5\linewidth}
\vspace{-0.2in}

\end{boxedminipage}
  \caption{LP Relaxation for \NodeMC}
  \label{fig:nodemc-lp}
\end{figure}


\begin{theorem}\label{thm:node_cut_approximation}
  There is a polynomial-time randomized algorithm that given a
  feasible solution  to {\sc \NodeMCRel} returns a feasible
  integral solution of expected cost at most , and runs in  time.  The algorithm can be
  derandomized to yield a deterministic -approximation algorithm
  that runs in  time.
\end{theorem}

Let  be a feasible fractional solution to \NodeMCRel.  For nodes
 and  we define  to be the length of the shortest path
between  and  according to the node weights given by ; we
count the weights of the end points  and  in . We omit
the subscript  in subsequent discussion.  For a given radius 
and node  let  be the set of all nodes  such that
;  is the ball of radius  around . 
We define the ``boundary'' of radius  from , denoted by
 to be the set of all nodes that are not in  but
have an edge to some node in .

\begin{prop}
  \label{prop:node-boundary}
  A node  iff . Further, 
  if  for  then .
\end{prop}

Our rounding algorithm first picks an index  uniformly at random
from . It then picks a  uniformly at random
from . For each  it includes in the final cut 
all nodes  that are in the ``boundary'' of the ball of radius
 around .  The formal description is given in
Algorithm~\ref{alg:node_cut_rounding_scheme}.

\begin{algorithm}[htb]
	\caption{Rounding for \NodeMC}
	\label{alg:node_cut_rounding_scheme}
	\begin{algorithmic}[1]
		\STATE Given feasible fractional solution  to \NodeMCRel
		\STATE Chose  uniformly at random
\STATE Pick  uniformly at random
\STATE 
		\STATE Return 
	\end{algorithmic}
\end{algorithm}

Let  be the output of the algorithm for fixed 
and .  We first argue that the algorithm always returns a
feasible multiway cut.
\begin{lemma}
For all ,  
is a feasible multiway cut for the given instance. That is, 
has no path from  to  for .
\end{lemma}
\begin{proof}
  Consider any pair  where .
  Assume , the case when  is similar.
  The ball  does not contain  since 
   and  by feasibility of .
   contains all nodes from , thus, in   there
  cannot be a path from  to any node in ,
  and hence to .
\end{proof}

We say that  is cut by the algorithm if .  The key to the
performance guarantee of the algorithm is the following lemma.
\begin{lemma}
  \label{lem:node-cut-prob}
  .
\end{lemma}
\begin{proof}
  Fix a node  and rename the terminals such that . Define the interval  as
  . From the algorithm
  description and Proposition~\ref{prop:node-boundary}, we can see
  that  iff  such that  and .

  Note that  is an empty interval if  or . Hence we can assume that  and , otherwise  is empty for all  and . We now consider two cases depending on
  whether  is greater than  or not.

  First, consider the case when . Interval
   is empty. Since , intervals  are also empty. Hence,  iff  and . Interval  has length at
  most  and  is chosen uniformly at random from
  . Therefore,


In the preceding equation we used independence in the choice of  and 
.

Next, consider the case when . From the
feasibility of , we have that 
(recall that  and  include the length of
). This implies that . Since,  for all , we have  which implies that
for all , . Easy to see that . Therefore,  iff 
and  or  and . Length of
interval  and  are  and  respectively. Hence,

In the penultimate inequality above, we use the fact that  if . The final inequality follows from already stated
observation,  due to feasibility of
.
\end{proof}


\begin{corollary}
  .
  Thus, the expected cost of the cut output by the algorithm is at most
   times the cost of the fractional solution .
\end{corollary}

\paragraph{Running time:} 
Algorithm \ref{alg:node_cut_rounding_scheme} can be implemented in
 time, in a fashion very similar to the
implementation of the modified version of Algorithm
\ref{alg:directed_cut_rounding_scheme}. First, we pick 
uniformly at random from  and  uniformly at
random from . Then, for each vertex  we find the closest
terminal  in the set  and cut vertex  if
. Finding nearest terminal for each
vertex can be done in  time. Hence, we get a
randomized -approximation rounding scheme in time .

To derandomize, we consider for each  intervals  and
 as in the proof of Lemma~\ref{lem:node-cut-prob}. Using the
-nearest terminal algorithm for  with  as the set of
terminals, in  time, we can compute  and
 for all . We sort the  end points of these 
intervals and let them be .  It
suffices to find the cost of the cut for each  from this 
values and for each .  We process these
sorted values in order and for each , we calculate
 for all .  The proof of
Lemma~\ref{lem:node-cut-prob} shows that this can be done by using
only  and  for all .  As we process the end points
in the sorted order the time to update the cut for each  per end
point is .  Thus, in  time we can obtain a
deterministic algorithm that gives a -approximation.

\paragraph{Acknowledgments:} CC thanks Sudeep Kamath, Sreeram Kannan and
Pramod Viswanath for extensive discussions on the problems considered
in \cite{KKCV15} which inspired us to revisit the rounding schemes
for multiway cut problems. CC also thanks Anupam Gupta
for discussion on some of the problems considered in \cite{KKCV15} during
an Oberwolfach workshop. 

\bibliographystyle{plain}
\bibliography{soda}

\appendix
\section{Proof of Theorem~\ref{thm:2-terminal-directed-integrality-gap}}
Here we prove the correctness of the integrality gap construction
described in Section~\ref{sec:2-terminal-dir-mc}.

The following proposition is easy to establish based on the symmetry in
the construction of the graphs.
\begin{prop}
  The - cut value and the - cut value in  are the same.
\end{prop}

Now, we calculate  in terms of . We refer to the
copy of  containing  and  with scaled capacities as , and the one 
containing  and  as .

\begin{lemma}
  For , . 
  For , the ratio of of
  the one-way cut value to the two-way cut value in  is
  .
\end{lemma}
\begin{proof}
  Proof by induction on . For the base case we see that  and in  the one-way cut value and two-way cut value are both
   and hence the ratio is equal to .
  
  We now prove the induction step. For this purpose we estimate the
  one-way cut value and the two-way cut value in .

  \noindent
  \textbf{Minimum two-way cut:} Any finite value cut that separates
   from  has to cut at least one of the two edges .
  We consider two cases.
  
  \noindent\textbf{Case 1:} Both  are cut. 
  To separate  and  it is best to pick a two-way cut between 
  and  in  (or symmetrically between  and  in ).
  Thus the total cost is .
  
  \noindent\textbf{Case 2:} Only one of the edges   is cut.
  Without loss of generality this edge is . Since 
  is not cut  and  can reach  via . Thus any two-way cut
  in  needs to use a one-way cut in  to separate  from 
  and a one-way cut in  to separate  from . The cost of each
  of these one-way cuts is, by induction, . Thus the total
  cost is .

  In both cases the cost is the same and hence the optimal two-way cut in
   is .
  
 
  \noindent\textbf{Minimum one-way cut:} We now calculate one-way cut
  from  to . At least one of the edges  has to
  be cut. Also, either there is no path from  to  or no path from  to
  . Thus, the cost of the one-way cut from  to  is at least . Moreover it
  is easy to see that this is achievable by removing  and
  one-way cut from  to  in .

  \noindent\textbf{Optimum fractional solution value:} We now 
  calculate the optimum for \DirMCRel on . We consider the
  following feasible solution . Assign  to the infinite weight
  edges and  to each of edges  and . For the
  edges in the graphs  and  we take an optimum solution  to
  \DirMCRel on  and scale it down by  and assign these
  values to the edges of  and . Feasibility of  for 
  implies that distance from  to  and  to  in 
  according to  is  (since we scaled down by ). It is
  easy to verify that distance of  to  and from  to  is
   in the fractional solution  in . Now we analyze the
  cost of this solution .  We have a
  total contribution of  from the two edges  and
  .  We claim that  since
  the cost of the two-way cut in  is chosen to be
  , the integrality gap is
   and we scaled down  by  to obtain 
  in . Same holds for .  Thus the total fractional cost of this
  solution is . We can see that this is an optimum solution by
  exhibiting a multicommodity flow of the same value for the pairs
   and  in . Route one unit of flow from  to
   along the path . In  there exists a feasible flow of total value
  . Let  and  be the amount of
  flow from  to  and  to  respectively. By duplicating this
  flow in  we see that a flow of value 
  exists between  and  in  via  and . Thus there
  is a total flow of value at least  in 
  and this is optimal.
  

\iffalse
  \noindent\textbf{Maximum flow:} We now calculate the optimum for 
  \DirMCRel on  which by duality is the same as the maximum
  multicommodity flow for the pairs  and  in .
  It is easy to argue that there is an optimum flow in which one unit
  of flow is sent from  to  along the path . We can thus focus on
  the maximum flow in . In this graph all flow from
  has to go through . Thus if  is flow from  to  and
   from  to  then we see that this is feasible if and
  only if there is a feasible flow in  that simultaneously routes
   flow from  to  and  flow from  to . In
  other words, the maximum flow possible in  is
  exactly the same as the maximum flow in  which is a scaled copy
  of .  By induction this is . Thus the maximum flow in  is
   .
\fi
 
  \medskip We can now put together the preceding bounds to prove the
  lemma.  The flow-cut gap in  is seen to be the ration of
  the two-way cut value  and the
  maximum flow value . Hence
   as desired.  The
  ratio of one-way cut value  and the
  two-way cut value  in  is
   which is equal to
  . This completes the inductive proof.
\end{proof}

\medskip
We have a sequence of numbers  where  and
. It is easy to argue
that this sequence converges to . This proves that the integrality
gap of \DirMCRel is in the limit equal to .

\section{Fractionality of the LP solutions}
\label{sec:fractionality}
It was shown in \cite{GargVY04} that there is a half-integral optimum
solution for the natural LP relaxation for node-weighted multiway cut
(\NodeMC) which was then exploited to obtain a
-approximation.  \cite{GargVY04} also showed that the
half-integral property does not hold for \2DirMC. Here we generalize
their example to observe that for any positive integer  there are
examples where there may not exist an optimum solution to \DirMCRel on
instances with two terminals that is  integral. More generally,
there does not exists an edge with length more than .

Consider the generalization of the example in \cite{GargVY04} as shown
in Fig~\ref{fig:2-terminal-non-half-integral-example}.  Each flow path
from  to  or  to  has to use at least  edges of the
type  or . Since, there are only
 such edges, flow is upper bounded by . To see that
this flow is also achievable, consider the following sets of
paths. For , path  and path . Send  unit of flow along
each of these paths. Each of the edge  is part of 
for  and part of  for . Hence, capacity
used for edge  is . Similarly for each
edge . Flow value is equal to . So, optimum
solution has value .


\begin{figure}[hbt]
\centering
\includegraphics{2-terminal-non-half-integral-example.pdf}
\caption{Edges of the form  or  have capacity 
  and rest have infinite capacity. Optimal fractional cut/flow is .}
\label{fig:2-terminal-non-half-integral-example}
\end{figure}


By strong duality, optimal value of \DirMCRel is equal to maximum flow which is
equal to . Let  be an optimal solution to the
\DirMCRel. By feasibility of the solution, each of the paths  and
 has length at least . Summing up the lengths of path 
and , we get . By
optimality of the solution first term is equal to
. Therefore, . Since, this inequality holds for all , and
, we get that all
the inequalities are tight and . Since, all lengths are non-negative, . By taking , we get an instance where
optimal solution has no edge having length at least .



\end{document}