Reading order detection, aiming to capture the word sequence which can be naturally comprehended by human readers, is a fundamental task for visually-rich document understanding. 
Current off-the-shelf methods usually directly borrow the results from the Optical Character Recognition (OCR) engines \cite{10.1145/3394486.3403172} while most OCR engines arrange the recognized tokens or text lines in a top-to-bottom and left-to-right way~\cite{6628706}.
Apparently, as shown in Figure~\ref{fig:1}, this heuristic is not optimal for certain document types, such as multi-column templates, forms, invoices, and many others. 
An incorrect reading order will lead to unacceptable results for document understanding tasks such as the information extraction from receipts/invoices. Therefore, an accurate reading order detection model is indispensable to the document understanding tasks.





In the past decades,  some conventional machine learning based or rule based methods~\cite{aiello2003bidimensional,4377050,malerba2007learning,malerba2008machine,10.1145/2644866.2644883} have been proposed.
However, 
these approaches are usually trained with only a small number of samples within a restricted domain or resort to unsupervised methods with empirical rules, because it is too laborious to annotate a large enough dataset.
These models can barely show case studies of certain reading order scenarios and cannot be easily adapted for real-world reading order problems. 
Recently, deep learning models \cite{10.1007/978-3-030-58595-2_6} have been applied to address the reading order issues for images from E-commerce platforms. Although good performance has been achieved, it is time-consuming and labor-intensive to produce an in-house dataset, while they are still not publicly available to compare with other deep learning approaches. Therefore, to facilitate the long-term research of reading order detection, it is inevitable to leverage automated approaches to create a real-world dataset in general domains, not only with high quality but also of larger magnitude than the existing datasets.




\begin{figure*}[htbp]
\centering
\subfigure[]{
\includegraphics[width=0.20\linewidth]{figs/example_1_new.png}
}
\subfigure[]{
\includegraphics[width=0.20\linewidth]{figs/example_1_seq_new.png}
}
\subfigure[]{
\includegraphics[width=0.20\linewidth]{figs/example_2_new.png}
}
\subfigure[]{
\includegraphics[width=0.20\linewidth]{figs/example_2_seq_new.png}
}
\caption{Document image examples in ReadingBank with the reading order information. The colored areas show the paragraph-level reading order.}
\label{fig:1}
\end{figure*}

To this end, we propose ReadingBank, a benchmark dataset with 500,000 real-world document images for reading order detection. 
Distinct from the conventional human-labeled data, the proposed method obtains high-quality reading order annotations in a simple but effective way with automated metadata extraction. 
Inspired by existing document layout annotations~\cite{Siegel2018ExtractingSF, Zhong2019PubLayNetLD, li2020tablebank, li2020docbank}, there are a large number of Microsoft WORD documents with a wide variety of templates that are available on the internet. 
Typically, the WORD documents have two formats: the binary format (Doc files) and the XML format (DocX files). 
In this work, we exclusively use WORD documents with the XML format
as the reading order information is embedded in the XML metadata.
Furthermore, we convert the WORD documents into the PDF format so that the 2D bounding box of each word can be easily extracted using any off-the-shelf PDF parser. 
Finally, we apply a carefully designed coloring scheme to align the text in the XML metadata with the bounding boxes in PDFs. 


With the large-scale dataset, it is possible to take advantage of deep neural networks to solve reading order detection task. 
We further propose LayoutReader, a novel reading order detection model in which the seq2seq model is used by encoding the text and layout information and generating the index sequence in the reading order. Ablation studies on the input modalities show that both text and layout information are essential to the final performance. The LayoutReader with both modalities surpasses other comparative methods and performs almost perfectly in reading order detection.
In addition, we also adapt the results of LayoutReader to open-source and commercial OCR engines in ordering text lines. 
Experiments show that the line ordering of both open-source and commercial OCR engines can be greatly improved. 
We believe that ReadingBank and LayoutReader will empower more deep learning models in the reading order detection task and foster more customized neural architectures to push the new SOTA on this task.

The contributions are summarized as follows:

\begin{itemize}[nosep, leftmargin=*]
    \item We present ReadingBank, a benchmark dataset with 500,000 document images for reading order detection. To the best of our knowledge, this is the first large-scale benchmark for the research of reading order detection.
    \item We propose LayoutReader for reading order detection and conduct experiments with different parameter settings. The results confirm the effectiveness of LayoutReader in detecting reading order of documents and improving line ordering of OCR engines.
    \item The ReadingBank dataset and LayoutReader models will be publicly available to support more deep learning models on reading order detection.
\end{itemize}

\noindent\textbf{Reproducibility.} We will release the code and datasets at \url{https://aka.ms/layoutreader}.