\documentclass[fleqn,11pt]{article}
\usepackage{a4}
\usepackage{graphicx}
\usepackage{psfrag}
\usepackage{beton}
\usepackage{concrete}
\usepackage{dsfont}




\newtheorem{definition}{Definition}
\newtheorem{lemma}[definition]{Lemma} 
\newtheorem{observation}[definition]{Observation} 
\newtheorem{theorem}[definition]{Theorem} 
\newtheorem{fact}[definition]{Fact} 
\newtheorem{corollary}[definition]{Corollary} 
\newtheorem{remark}[definition]{Remark} 
\newtheorem{result}[definition]{Result} 
\newtheorem{problem}[definition]{Problem} 
\newtheorem{example}[definition]{Example} 
\newtheorem{assumption}[definition]{Assumption}
\newtheorem{proposition}[definition]{Proposition} 
\newtheorem{algorithm}[definition]{Algorithm}
\newtheorem{question}[definition]{Question}
\newtheorem{game}[definition]{Game}


\newsavebox{\proofsymbol}   
\savebox{\proofsymbol}                         
{
    \begin{picture}(10,10)                       
    \put(0,0){\framebox(9,9){}}                 
    \put(0,3){\framebox(6,6){}}                  
    \end{picture}                               
} 

\newcommand{\proof}{\noindent{\bf Proof.} \hspace{1mm}}
\newcommand{\qed}{\hfill\usebox{\proofsymbol}}

\newcommand{\R}{\mathds{R}}
\newcommand{\lex}{\rm lex}

\begin{document}

\title{Ranking Unit Squares with Few Visibilities} 
\author{Bernd G\"artner\thanks{
Institute of Theoretical Computer Science, 
ETH Z{\"u}rich, CAB G32.2, CH-8092 Z{\"u}rich, Switzerland. Email:
gaertner@inf.ethz.ch}}
\date{August 4, 2008} 
\maketitle

\begin{abstract}
  Given a set of  unit squares in the plane, the goal is to rank
  them in space in such a way that only few squares see each other
  vertically.  We prove that ranking the squares according to the
  lexicographic order of their centers results in at most 
  pairwise visibilities for .  We also show that this bound is
  best possible, by exhibiting a set of  squares with at least
   pairwise visibilities under any ranking.
\end{abstract}

\section{Problem statement}
The unit square with center  is the set
 Given a set  of  unit squares (simply called squares in the sequel), a
\emph{ranking} of  is a sequence
 such that . For ,  \emph{sees}
 under  if there exists a point  such that


The graph on  formed by all pairs  such that
 sees  under  is called the \emph{visibility graph}
of  and will be denoted by . 

The goal is to find a ranking  such that  has as
few edges as possible. We do not know how to find the best ranking for
a given set , but we prove that there is always a ranking 
 under which  has no more than  edges. For
some sets , this is the best bound that can be achieved.

This research is motivated by similar questions for intervals in
 \cite{intervals}.

\section{Main result}
Given a ranking , the center of square
 will be denoted by .  We will
repeatedly use the following simple fact.

\begin{observation}\label{obs:help}
  Let  be a ranking, and suppose that
  there are centers  with , such that  is
  contained in the (axis-parallel) rectangle spanned by  and
  . Then  is not an edge of~.
\end{observation}

\proof 
Center  being contained in the rectangle spanned by 
and  is easily seen to be equivalent to .
It follows that  does not see  under .  
\qed

The size (number of edges) of the visibility graph may be
 for a ``bad'' ranking, see Figure~\ref{fig:n2} (left).
The right part of the figure depicts the lexicographic ranking.
According to the next lemma, this ranking always results in 
visibilities.

\begin{figure}[htb]
\begin{center}
\includegraphics[width=12cm]{n2.eps}
\end{center}
\caption{Upward view on a ranked set of squares. Left: All squares in the 
lower half see all squares in the
upper half. Right: the lexicographic ranking incurs only linearly many 
visibilities.\label{fig:n2}}
\end{figure}

\begin{lemma}\label{lem:main}
  Let  be the lexicographic ranking,
  meaning that  if and only if the center of  is
  lexicographically smaller than the center of . Then 
  is a planar graph. More precisely, the straight-line embedding of
   obtained by mapping each square  to its center
   is a plane graph.
\end{lemma}

\proof We show that if two segments  and
 cross, then  contains at most
one of the edges  and . 

Since deletion of other squares can only add visibilities between the
four squares involved in the crossing, we may w.l.o.g.\
assume that . By 
lexicographic order of the centers, there are only two
cases.\smallskip

\emph{Case (a): Segments  and 
  cross.} If  is between  and , we get that
 is in the rectangle spanned by  and , so
 is not an edge of  by 
Observation~\ref{obs:help}.


If  is not between  and , we either have
 and thus  (otherwise, there
would be no crossing), or  and thus . In both cases,  is between  and  which
means that  is in the rectangle spanned by  and . This
in turn shows that  is not an edge of .
\smallskip

\emph{Case (b): Segments  and 
  cross.} An easy case occurs if one of
 and  is in the rectangle spanned by  and
, since Observation~\ref{obs:help} then implies that
 is not an edge.

Otherwise, we have 

and thus 

(because of the crossing), or  and
. Let us only treat the first case; the second one
is symmetric under exchange of indices~ and~.

We now show that every point in  is in , given that . Therefore, if
 is an edge of , then
 is not an edge.

Let  be any point in . 
From 
and lexicographic order it follows that

Using  together with (\ref{eq:visa}) and
(\ref{eq:visb}), we also conclude that

If  in addition satisfies , (\ref{eq:vis1}) and
(\ref{eq:vis2}) imply that . But if , we can
use the assumption  to conclude that 
 and hence . With (\ref{eq:vis1}) and
(\ref{eq:vis2}), we then get .
\qed

Now we are ready to prove our main theorem.

\begin{theorem}\label{thm:main}
If ,  has at most  edges.
\end{theorem}

\proof An upper bound of  already follows from Lemma
\ref{lem:main}. In order for this bound to be tight, the outer face of
's straight-line embedding would have to be a
triangle  spanned by three centers
, , and with
all other centers inside . Indeed, for  this is possible,
but for , we get a contradiction: Let  be any center
distinct from . From , it easily follows that
 (comparison being lexicographically). 

If  is between  and ,  is in the rectangle spanned by
 and , so  can't be an edge of
 by Observation \ref{obs:help}.

If  is not between  and , then  must be between  and
, \emph{and} between  and . Depending on whether
 or , we get that either 
 or  is not an edge of .
\qed

\section{Lower Bound}
The bound of  derived in Theorem \ref{thm:main} is best possible
in the worst case, not only under the lexicographic ranking, but under
\emph{every} ranking. For a proof by picture see Figure \ref{fig:best}.

\begin{figure}[htb]
\begin{center}
\includegraphics[width=5cm]{best.eps}
\end{center}
\caption{A set of  unit squares such that the visibility graph
   has at least  edges for every ranking : a
  square from the middle bunch of  squares always sees the next
  square above it in the bunch (these are  visibilities), and it
  sees (or is seen by) both of the two special squares (
  visibilities).\label{fig:best}}
\end{figure}
 
\section*{Acknowledgment}
The problem of ranking unit squares was given to me as an
undergraduate student by Emo Welzl in early 1989.  After some initial
programming tasks, this was the first actual research assignment
during my time as ``Forschungstutor'' (research assistant) with Emo.

My initial manuscript from 1989, typeset in bumpy \LaTeX\ and
containing manually drawn figures, is lost. But throughout the almost
twenty years that have passed since then, I could never forget this
first (simple) result of mine. 

While writing it up now, many memories of all these years have come
back to me with surprising strength. I thank Emo for bringing the
problem to my attention, and for so much more.
\bibliographystyle{plain} \bibliography{biblio}

\end{document}
