\documentclass{llncs}
\usepackage{amssymb}
\usepackage{amsmath,fancyhdr}
\usepackage{amsfonts}
\usepackage{epsfig}
\usepackage{enumerate}
\usepackage{verbatim}
\usepackage{graphics}
\usepackage{hyperref}
\usepackage{makeidx}
\usepackage{microtype}
\usepackage{subfig}
\usepackage{wrapfig}
\usepackage{pifont}



\usepackage{microtype}

\def\cro{\mathop{\rm cr}\nolimits}
\def\pcr{\mathop{\rm pcr}\nolimits}
\def\ocr{\mathop{\rm ocr}\nolimits}
\def\iocr{\mathop{\rm iocr}\nolimits}
\def\iocro{\mbox{iocr-}}

\def\qed{ \ \vrule width.2cm height.2cm depth0cm\smallskip}
\def\res{\mid^{\mskip-4.5mu\scriptscriptstyle\backslash}}


\newtheorem{observation}{Observation}

\newcommand{\rephrase}[3]{\noindent\textbf{#1 #2}.~\emph{#3}}
\def\marrow{{\marginpar[\hfill]{}}}
\def\rado#1{{{\sc Rado says: }\bf{\marrow\sf #1}}}





\def\cNP{\hbox{\rm \sffamily NP}}
\def\cP{\hbox{\rm \sffamily P}}
\def\inst#1{}

\def\wn{\mathrm{wn}}
\def\length{\mathrm{height}}
\def\index{\mathrm{index}}

\newif\iflong

\longfalse














\iflong
\pagestyle{fancy}
\fi

\begin{document}


\title{C-planarity of Embedded Cyclic c-Graphs}

\author{Radoslav Fulek\inst{1}\thanks{The research leading to these results has received funding from the People Programme (Marie Curie Actions) of the European Union's Seventh Framework Programme (FP7/2007-2013) under REA grant agreement no [291734].}
\iflong\else\thanks{The omitted parts of the proof are in the appendix.}\fi}

\institute{
IST Austria, Am Campus 1, Klosterneuburg 3400, Austria\\
\email{radoslav.fulek@gmail.com}
}

\maketitle



\begin{abstract}
We show that c-planarity is solvable in quadratic time for 
flat clustered graphs with three clusters if the combinatorial embedding
of the underlying  graph is fixed. In simpler graph-theoretical terms our result can be viewed as follows. Given a graph  with the vertex set  partitioned into three parts embedded on a 2-sphere, our algorithm decides if we can augment  by adding edges 
without creating an edge-crossing so that in the resulting spherical graph the vertices of each part induce a  connected sub-graph.
We proceed by a reduction to the problem of testing the existence of  a perfect matching in planar
bipartite graphs.
We formulate our result in a slightly more general setting of cyclic clustered graphs,
i.e., the simple graph  obtained by contracting each cluster, where we disregard loops and multi-edges, is a cycle. 
\end{abstract}






\setcounter{page}{1}
\section{Introduction}

Testing planarity of graphs with additional constraints is a popular theme in
the area of graph visualizations.
One of most the prominent such planarity variants, c-planarity, raised in 1995 by Feng, Cohen and Eades~\cite{FCEa95,FCEb95} 
asks for a given planar graph  equipped with a hierarchical structure on its vertex 
set, i.e., clusters, to decide if a planar embedding  with the following property exists:
the vertices in each cluster are drawn inside a disc so that the discs
form a laminar set family corresponding to the given hierarchical structure
and the embedding has the least possible number of edge-crossings with the boundaries of the discs.
Shortly after, several groups of researchers tried to settle
the main open problem formulated by Feng et al. asking to decide its complexity
status, i.e., either provide
a polynomial/sub-exponential-time  algorithm for c-planarity or show its \cNP-hardness.
First, Biedl~\cite{B98} gave
 a polynomial-time algorithm for c-planarity with two clusters. A different approach
for two clusters was considered by Hong and Nagamochi~\cite{HN16}
and quite recently in~\cite{FKMP15}.
 The result also follows from a work by Gutwenger et al.~\cite{GJL+02}.
Beyond two clusters a polynomial time algorithm for c-planarity was obtained only in special cases,
e.g.,~\cite{BFPP08,GLS05,GJL+02,JJK+09,JKK+09}, and most recently in~\cite{BR14+,CBFK14+}. Cortese et al.~\cite{CDPP05} shows that c-planarity is solvable in polynomial
time if the underlying  graph is a cycle and the number of clusters
is at most three.

 In the present work we generalize the result of Cortese et al. to the class of all planar graphs
with a given combinatorial embedding. In a recent pre-print~\cite{F14+} we established 
a strengthening for trees, where we do not fix the embedding.
In the general case (including already the case of three clusters) of so-called flat clustered graphs a similar result was obtained
only in very limited cases. Specifically, either when every face of  is incident
to at most five vertices~\cite{BF07,FKMP15}, or when there exist at most two vertices of a cluster incident to a single face~\cite{CBFK14+}.
We remark that the techniques of the previously mentioned papers do not give
a polynomial-time algorithm for the case of three clusters, and also do not seem to be adaptable
to this setting. Our result and the technique used to achieve it suggest that, for a fairly general class of clustered graphs, c-planarity could be tractable/solvable in  sub-exponential time at least with a fixed combinatorial embedding. 

 {\bf Notation.}
Let  denote a connected planar graph possibly with multi-edges.
For  standard graph theoretical definitions such as path, cycle, walk etc.,
we refer reader to~\cite[Section 1]{D05}. 
A \emph{drawing} of  is a representation of  in the plane where every vertex
 in  is represented by a unique point and every
edge  in  is represented by a Jordan arc joining the two points that represent  and . 
We assume that in a drawing no edge passes through a vertex,
no two edges touch and every pair of edges cross in finitely many points.
An \emph{embedding} of  is an  edge-crossing free drawing.
If it leads to no confusion, we do not distinguish between
a vertex or an edge and its representation in the drawing and we use the words ``vertex'' and ``edge'' in both
 contexts.
A  \emph{face} in an embedding is a connected component of the complement of the embedding 
of  (as a topological space) in the plane.
 The \emph{facial walk} of  is the closed walk in  with a fixed orientation that we obtain by traversing the boundary of  counter-clockwise.
In order to simplify the notation we sometimes denote the facial walk of a face  by . 
  A pair of consecutive edges  and  in a facial walk  creates a \emph{wedge} incident to  at their common vertex.
  A vertex or an edge is \emph{incident} to a face , if it appears on its facial walk.
The \emph{rotation} at a vertex is the counter-clockwise cyclic order of the end pieces of its incident edges
in a drawing of .
An embedding of  is up to an isotopy and the choice of an \emph{outer} (unbounded) face described by the rotations at its vertices. We call such a description of an embedding of  a \emph{combinatorial embedding}. Remaining faces are \emph{inner faces}.
The \emph{interior} and \emph{exterior} of a cycle in an embedded graph is the bounded and unbounded, respectively, connected component
of its complement in the plane. 
Similarly, the \emph{interior} and \emph{exterior} of an inner face in an embedded graph is the bounded and unbounded, respectively, connected component
of the complement of its facial walk in the plane, and vice-versa for the outer face.
When talking  about interior/exterior or area of a cycle  
in a graph  with a combinatorial embedding and a \emph{designated} outer face  we mean it with respect to an embedding in the isotopy class that  defines.
For  we denote by  the sub-graph of  induced by . 









 


A \emph{flat clustered graph}, shortly  \emph{c-graph}, is a pair , where  is a graph and , , is a partition of the
vertex set into \emph{clusters}. See Figure~\ref{fig:treeEx} for an illustration.
A  c-graph  is \emph{clustered planar} (or briefly \emph{c-planar}) if  has an
 embedding in the plane such that (i)
for every  there is a topological disc , where , if ,
 containing all the vertices of  in its interior, and (ii)
 every edge of  intersects the boundary of  at most once for every .
A c-graph   with a given combinatorial embedding of  is \emph{c-planar} 
if additionally the embedding is combinatorially described as given.
 A \emph{clustered drawing and embedding} of a flat clustered graph  is a drawing and embedding, respectively,
 of  satisfying (i) and (ii).
In 1995
 Feng, Cohen and Eades~\cite{FCEa95,FCEb95} introduced the notion of clustered planarity for clustered graphs, shortly c-planarity, (using, a more general, hierarchical clustering)
as a natural generalization of graph planarity. (Under a different name
Lengauer~\cite{L89} studied a similar concept in 1989.)


\begin{wrapfigure}{r}{.5\textwidth}
  \centering
\centering
\subfloat[]{
\includegraphics[scale=0.4]{treeEx}
    	}
\subfloat[]{
\includegraphics[scale=0.4]{treeEx+}
		}
\caption{A c-graph that is not c-planar (left); and a c-planar c-graph (right).}
\label{fig:treeEx}
\end{wrapfigure}

By slightly abusing the notation for the rest of the paper  denotes  a flat c-graph  with  clusters
 and , and a given combinatorial embedding, and we assume that  is \emph{cyclic}~\cite[Section 6]{FKMP15}. Thus, every  of  is such that
 and  where  and for every
 there exists an edge in  between  and .
In the case of three clusters, the first condition is redundant.
If the second condition is violated, the problem was essentially solved for three clusters as discussed in
Section~\ref{sec:wnwn}.
We assume that  is connected, since in the problem that we are studying, the connected components of  can be treated separately. Indeed, \iflong as we show in Section~\ref{sec:fan} \fi without loss of generality  we  assume throughout the paper that in a clustered embedding of  the clusters are unbounded wedges defined by pairs of rays emanating from the origin (see Figure~\ref{fig:wedges}) that is disjoint from all the edges (see Appendix). We call such a clustered drawing  a \emph{fan drawing}. \\









Thus, a connected component in a clustered embedding 
can be drawn so that it is disjoint from a ball  centered at the origin of radius 
for any . The rest of the graph is then embedded inductively inside .
The aim of the present work is to prove the following.

\begin{theorem}
There exists a quadratic-time algorithm in  to test if a cyclic c-graph  is c-planar.
\end{theorem}




{\bf Further research directions.}
We think that our technique should be extendable by means of Euler's formula to resolve the c-planarity in more general situations than the one treated in the present paper. In particular, we suspect that
the technique should yield a generalization of the characterization of strip planar clustered graphs~\cite[Section 5]{F14+}. 
That would allow us to work with graphs without a fixed embedding. We mention that the tractability in a special case 
of our problem known as cyclic level planarity, when the embedding is not fixed, follows from a recent work of Angelini et al.~\cite{angelini2015beyond}.


{\bf Organization.} In Section~\ref{sec:pre} we introduce concepts used in the proof of
our result. We give an outline of our approach in Section~\ref{sec:out}.
A more detailed description and a proof of correctness of our algorithm is in Section~\ref{sec:alg}.



 
\section{Preliminaries}
\label{sec:pre}

\iflong
\subsection{Fan drawings}
\label{sec:fan}
We show that the clusters can be drawn as regions, each bounded by a pair of rays emanating from the origin.
Suppose that  is given by a clustered embedding
living in the  plane of .
We assume that boundries of discs representing clusters do not touch.
Consider a stereographic projection from the north pole of a two-dimensional sphere  
sitting at the origin of .
Let  be a stereographical pre-image of the embedding of  on .
Let  denote the union of  (as a topological space) with the boundaries of the clusters in .
Let  and  be a connected component of the complement of  in , respectively, containing the north pole and south pole.
If necessary, we apply an isotopy  to  (a continuous deformation keeping  to be a clustered embedding all the time)  so that in the resulting embedding  of  on  every boundary of a cluster intersects (in fact touches) the closure of  and  the closure of . 

We show that a desired isotopy exists. We contract every cluster to a point thereby
treating clusters as vertices in an embedding  of a cycle  of length  having multi-edges. 
Formally, this can be viewed as a quotient , where  iff
 and  belong to the same cluster.
In  there must be a pair of faces  and  whose facial walk is  since any cycle in the corresponding multi-graph is obtained as a symmetric difference of facial walks. Apply an isotopy to  such that  contains
the north pole in its interior and  contains the south pole in its interior. Finally, we decontract clusters in the end. The above procedure can be easily turned into an isotopy of . 

By projecting the resulting spherical embedding back to the plan we can also assume that we have a clustered embedding of 
such that clusters are represented by small discs of diameter  each drawn in a close vicinity
of a different vertex of a regular convex -gon with the center at the origin, and the edges
between clusters  and , for every , are closely following the edge
of the -gon between the corresponding pair of vertices. 
The desired rays bounding clusters are those from the origin orthogonal to the sides of the -gon. \\
\fi




\subsection{Outline of the approach}
\label{sec:out}
By~\cite[Theorem 1]{FCEb95} deciding c-planarity of instances  in which all 's are connected amounts to 
checking if an outer face of  can be chosen so that every  is embedded in the outer face 
of . On the other hand, once we have a clustered embedding of  we can augment  by adding edges drawn inside clusters without creating an edge-crossing so that clusters become connected.
These observations suggest that c-planarity of  could be viewed as a connectivity augmentation problem, for example as in~\cite{CBFK14+,FKMP15},
in which we want to decide if it is possible to make clusters connected  while maintaining the planarity of .
One minor problem with this viewpoint is the fact that if  is c-planar we do not allow
a cluster  to induce a cycle such that clusters  and , , are drawn on its opposite 
sides. However, this cannot happen  if  is cyclic.
Following the above line of thought our algorithm tries to augment  by subdividing its faces with
paths and edges. We proceed in two steps. In the first step, Section~\ref{sec:norm}, we either 
detect that  is not c-planar or similarly as in~\cite{ADDF13} and~\cite{F14+} by 
turning clusters into independent sets and adding certain paths we normalize the instance. In the second step, Section~\ref{sec:const}, we decide if the normalized instance can be further augmented by  edges as desired.

In order to prove the correctness of the second step of the algorithm
we use the notion of the \emph{winding number}  of a walk  of , as defined
in Section~\ref{sec:wnwn}. The parameter  says how many times and in which sense 
a walk  of  winds around the origin in a clustered drawing of .
Thus,  is not c-planar if there exists a face  such that for its facial walk  or
if there exists at least two inner faces  with .
However, it can be easily seen that this necessary condition of c-planarity is not sufficient
except when  is a cycle~\cite{CDPP05}.
The  necessary condition allows us to reduce the c-planarity testing problem of a normalized instance to the problem
of finding a perfect matching in an auxiliary face-vertex incidence graph which is polynomially solvable.
The novelty of our work lies in the use of the winding number in the context of connectivity augmentation guided
by the flow and matching in the auxiliary face-vertex incidence graph \`a la~\cite{ADDF13} and~\cite{F14+}, respectively. 

We remark that the approach of~\cite{ADDF13} via a variant of upward embeddings
for directed graphs in our settings has several problems that seem quite hard to overcome,
the main one being the fact that the result of Bertolazzi et al.~\cite{BBLM94} does not extend, at least not in a natural way, to the drawings on the rolling cylinder, see e.g.,~Auer et al.\cite{Auer201536} for the definition of these drawings.
We are not aware of a polynomial-time algorithm for the corresponding  problem,
nor a corresponding \cNP-hardness result, and
find the corresponding algorithmic question interesting and related to our problem. 

\begin{figure}
  \centering
\centering
\subfloat[]{
\label{fig:wedges}
\includegraphics[scale=0.5]{3ex}
    	}
    	\hspace{10px}
    	\subfloat[]{
    	\label{fig:semiSimple}
    	\includegraphics[scale=0.5]{semiSimple}
    	}
\caption{(a) A clustered graph  with clusters represented by wedges bounded by rays meeting at the origin. The highlighted wedge at  is concave and at  convex. (b) A semi-simple face 
and the outer face  with an incident concave wedge.}
\end{figure}


\subsection{Winding number} 
\label{sec:wnwn1}
We define the winding number  of a closed oriented walk  in a drawing disjoint from the origin of a graph  (possibly with crossings). In what follows  facial walks are understood with the orientations as in
  an embedding of  with the given rotations and a face  being a designated outer face.
By viewing a closed walk  in the drawing  as a continuous  function  from the unit circle  to ,
the winding number  corresponds to the element of the fundamental group of  ~\cite[Chapter 1.1]{Hatch02}  represented by .
Let  and  denote a pair of oriented closed walks meeting in a vertex .
Let  denote the closed oriented walk from  to  obtained by concatenating  and .
By the definition of  we have .
Let  and , , denote a pair of faces of  whose walks intersect in a single walk.
Let  denote a graph we get from  by deleting edges incident to both  and .
Let  denote the new face thereby obtained. Since  and  intersect in a single walk, the boundary of  is connected.
In the drawing of  inherited from the drawing of  we have , since common edges of  and  are traversed
in opposite directions by  and .
A face  or a vertex is in the interior of a closed walk  in  if it is in the interior of 
a cycle induced by the edges of  in an embedding of  with the given rotations and   as the outer face.
The previous observation is easily generalized by a simple inductive argument 
as follows \\
 \\
 where we sum over all  faces~ of  in the interior of the
closed walk  in . In particular, , where we sum over all 
 faces   of .

\subsection{Labeling vertices} 
\label{sec:wnwn}
Let  be a labeling of the vertices  by integers
such that  if .
 Let  denote an oriented closed  walk in a clustered drawing of  . We put ,
 where  and .
We have the following.

\begin{lemma}
\label{lemma:wn}
For a walk  in a fan drawing of  we have .
\end{lemma}

\begin{proof}
The number of times
the walk  crosses the ray between  and  from right
to left w.r.t. to the direction of the ray is , 
where we sum over the edges  in the walk , where
  immediately precedes  in the walk.
 Similarly, we define \\  , 
where we sum over the edges  in , where
  immediately precedes  in the walk.
We have, 
which in turn implies .
\qed\end{proof}

 
By the previous lemma  is determined already by the c-graph  and is the same in all  clustered drawings of , and hence, putting , for a walk  with a fixed orientation, allows us to speak about  without
referring to a particular drawing of .
 Thus,  tells us the winding number of  in any clustered 
drawing.
By Jordan-Sch\"onflies theorem  the following holds. 

\begin{lemma}
\label{lem:2}
 is not c-planar if there exists a face  such that  or if there exists more than one inner face  with .
\end{lemma}
\begin{proof}
In a crossing free drawing   for every face .
If  the origin  lies in the interior of  since
otherwise the facial walk is null-homotopic, i.e., homotopic to a constant map, in  (contradiction). However, interiors of faces are disjoint.
\qed\end{proof}
 If  for all faces ,~\cite[Lemma 1.2]{F14+} extends easily to this case,
reducing the problem to the work of Angelini et al.~\cite{ADDF13}.
Thus, by Lemma~\ref{lem:2} and for the sake of simplicity of the presentation, throughout the paper we assume that there exists a pair  of faces ,   (by
~ there cannot be just one such face) one of which, let's say ,
we designate as an \emph{outer face}.  The roles of  and  are, in fact, interchangeable.
Also such a restriction is by no means crucial in our problem, and alternatively, it is always possible
to choose and subdivide the outer face in the normalized instance (defined later) by a path so that the restriction is satisfied.


Viewing a facial walk  as a sequence of vertices and edges , where ,
let  be the set  of \emph{vertex occurrences} along~.
We treat  also as a multi-set of vertices, and thus,  is defined on its elements.
Let , for , be a labeling of the elements of  by integers defined as follows.
We mark all the vertex occurrences  in  as unprocessed.
We pick an arbitrary vertex occurrence  , set 
and mark  as processed.
We repeatedly pick an unprocessed vertex occurrence   that has its predecessor or successor  along the boundary walk of  in   processed.
 We put  .
 Intuitively,  records 
 the distance  in terms of ``winding around origin'' of vertex occurrences 
 along the boundary walk of  from a single chosen vertex occurrence.
 Since  the function   is completely determined by
the choice of the  first occurrence of a vertex we processed. 
This choice is irrelevant for our use of  as we see later.
Also notice that  for all vertices incident to .

A normalized instance allows only the faces of the types defined next.
An element  in  is a \emph{local minimum} (\emph{maximum}) of a face  if in the  facial walk   the value of  is not bigger (not smaller)
with respect to the  relation   than the value of its successor and predecessor.
A walk  in  is \emph{(strictly) monotone with respect to } if the labels of the occurrences of vertices on  form a (strictly) monotone sequence
with respect to the  relation  when ordered
in the correspondence with their appearance on  .
The  face  is \emph{simple} if  has at most one local minimum. It follows 
that a simple face  has also at most one local maximum.
The inner face    is \emph{semi-simple}  if  has exactly two local minima and maxima and these minima and maxima, respectively,  have the same  value.



\section{Algorithm}
 \label{sec:alg}

A cyclic c-graph  is \emph{normalized } if 

\noindent
(i)  is connected; \\
(ii) each cluster  induces an independent set; and \\
(iii) each face of  is simple or semi-simple, and  and  are both simple. 

Suppose that (i)--(iii) are satisfied. 
By~(ii) we  put directions on all the edges in  as follows.
Let  denote the directed c-graph obtained from  by orienting every edge
 from the vertex with the smaller label  to the vertex with the bigger label  with respect to the relation .
A \emph{sink} and \emph{source} of  is
a vertex with no outgoing and incoming, respectively, edges.


Let  denote an edge of  not contained in a single 
cluster.
Given a clustered embedding  of  let  denote the intersection point of  with a ray separating a pair of clusters.
Let  be  the  edges
 incident to a sink or source .
By Jordan curve theorem it is not hard to see that (i)--(iii) imply that a clustered embedding  of  is ``combinatorially'' determined once we order the set 
  of intersection points
  along
 rays separating clusters for every sink and sources  in . Moreover, the set of intersection points corresponding to a sink or source  admits in
 an embedding only orders that are cyclic shifts of
 one another, since we have the rotations at vertices 
 of  fixed.
 The wedge in  formed by a pair of edges  and  incident to a face  at its local extreme  is \emph{concave} (see Figure~\ref{fig:wedges} 
 for an illustration) if 
 is a sink or source of  and
the line segment  contains all the other points  or
in other words the order of intersection points
corresponding to  is .
A non-concave wedge is \emph{convex}.\iflong\footnote{The terminology comes from the fact that in an embedding that is ``combinatorially the same'' with straight line edges, wedges become convex and concave in the usual sense.}\fi
Note that in  every sink or source is
incident to exactly one concave wedge that in
turn determines the order of intersection points.
Thus, combinatorially  is also determined
by a prescription of concave wedges at sink and sources.

Let   be the set of sinks and sources of . Let  denote the union of the set of 
semi-simple faces of  with a subset of  containing faces incident to a sink and a source.
We construct a planar bipartite graph  with parts  and ,
 where  and 
is joined by an edge if  is incident to . 
Given that (i)--(iii) are satisfied, the existence of a perfect matching  in  is a necessary condition for  being c-planar. Indeed, as we just said,
in a clustered embedding, each source or sink has exactly one of its wedges concave.
On the other hand, by Jordan curve theorem  it  can be easily checked that in the clustered embedding  \\
{\bf (A)} every semi-simple face is incident to exactly one concave wedge \\
{\bf (B)} faces  and  are incident to one concave wedge if they are incident to a sink and source, and \\ {\bf (C)} all the other faces are not incident to any concave wedges at the minimum and maximum. \\
This is fairly easy to see if  is vertex two-connected, see Figure~\ref{fig:semiSimple} for an illustration.
The cycle  \emph{corresponding to a closed walk} is obtained
by traversing the walk and introducing a new vertex for each
vertex occurrence in the walk.
For a face  incident to cut-vertices, {\bf (A)--(C)} follows by considering the cycle corresponding to the facial walk of  (treated as a face) embedded in a close vicinity of the boundary of . 
 Thus, a desired matching  is obtained by matching each source or sink with
the face incident to its concave wedge.

 
We show in Section~\ref{sec:const}  that if  exists   is c-planar by augmenting 
 with edges as described in Section~\ref{sec:out}.
Testing the existence, but even counting perfect matchings in a planar bipartite graph can be carried out in a polynomial time~\cite[Section 8]{L09}.



The running time of our algorithm is  since finding the perfect matching 
can be done in   time, due to , and the pre-processing step including the construction of  and the normalization will be easily seen to have this time complexity. Also computing the winding number for all the faces can be performed in a linear time by Lemma~\ref{lemma:wn}.
First, we explain and prove the correctness of the algorithm for
 instances satisfying~(i)--(iii). In Section~\ref{sec:norm},
 we show a polynomial-time reduction of the general case  to instances  satisfying~(i)--(iii). We often use Jordan--Sch\"onflies theorem without
 explicitly mentioning~it.


\iflong\else
\fi
\subsection{Constructing a clustered embedding}
 \label{sec:const}
 
\begin{wrapfigure}{r}{.5\textwidth}
\centering
\includegraphics[scale=0.65]{fuk}
\caption{Subdividing a semi-simple face   (left). Subdividing a simple face  (right).}

\label{fig:figfig}
\end{wrapfigure}

 
 Given a normalized instance  and a matching  between sources and sinks in , and  faces in  of  we construct a  clustered embedding of  as follows. Recall that we assume that  does not have a face 
  with  besides  and .
  We start with  defined
above and add edges to it thereby eliminating all the sinks and sources, see Figure~\ref{fig:figfig}.
Let  be a source matched in  with . If  is a semi-simple inner face
let  denote another local minimum  incident to . We add to  
  an edge  embedded in the interior of . 
  If  or  we join  by  with the vertex in the same cluster 
so that we subdivide  into two simple faces  and  such that
 and . If   face  is the new outer face. By Lemma~\ref{lemma:wn}, such a vertex  exists and it is unique.




We proceed with  that are sinks analogously thereby eliminating all the sinks and source in the resulting graph , where by  we denote its underlying undirected graph.
By Lemma~\ref{lemma:wn}, there still exists exactly one inner face  with a non-zero winding number in the resulting graph .



\begin{lemma}
\label{lemma:keyFact0}
 has exactly one inner face  such that .
\end{lemma}

Since  for every face  and  incident to , 
every edge we added joins a pair of vertices in the same cluster.


    \begin{lemma}
\label{lemma:keyFact1}
The induced sub-graph  of (undirected)  does not contain a cycle for .
\end{lemma}
\begin{proof}
For the sake of contradiction suppose that a cycle  is contained in .
Let us choose  such that the area of its interior is minimized.
Since  is an independent set all the edges of  are newly added.
Thus, by looking at the rotation of an arbitrary vertex  of  we see that  is incident to a vertex 
  from  , ,  in the interior of . Indeed, no two edges of  subdivide the same face of .
 
 



 
Using the fact that  does not contain any source or sink, we show that 
a vertex  in the interior of  belongs to an oriented cycle  (by chance also directed in ), whose interior is contained in the interior of    
such that .
The cycle  is obtained by following a directed path in  (from which it inherits its orientation)
 passing through  .
Either both ends of the path meet each other, they both meet , or the path meet itself in the interior. 
In the first two cases we can take  in the last case it can happen that the directed path gives rise to a cycle  not containing . However,  is not induced by a single cluster by the choice of , and thus,  by Lemma~\ref{lemma:wn} and  contains a vertex  from .
Let  denote the set of faces
in the interior of  and not in the interior of . 
In all cases it can be seen   by Lemma~\ref{lemma:wn} that .

Indeed, as we proved in  the proof of Lemma~\ref{lemma:wn} 
.
Since  follows a directed path and is not induced by a single cluster we have  and .
Hence, . 



 By~(*) it follows that  contains the unique
inner face with a non-zero winding number in its interior.
Then Lemma~\ref{lemma:keyFact0} with~(*) yields the following  contradiction

\qed\end{proof}

Let  such that each edge in  can be added to 
the embedding of  without creating a crossing or increasing the number of inner faces with a non-zero winding number.
We do not put any direction on the edges in .
Since every inner face  in  is simple, and its outer face and the face  are not adjacent
to a source or sink, all the edges in  can be 
introduced simultaneously without creating a crossing. 
In particular, no edge of  subdivides  or the outer face.
Let  denote a maximal subset of 
that does not introduce a cycle in  for every  (see Figure~\ref{fig:simpleFace}), where .
By Lemma~\ref{lemma:keyFact1},  is well-defined.


\begin{figure}\centering
\includegraphics[scale=0.65]{simple_face}
\caption{A simple face  of  (left). The face  subdivided with edges of  (right). Labels at vertices are their  values (or indices of their clusters).}
\label{fig:simpleFace}
\end{figure}





\begin{lemma}
\label{lemma:keyFact2}
    is a tree for .
  \end{lemma}
\begin{proof}
Suppose for the sake of contradiction that  for some  is not a tree, and thus, it is just a  forest with more than one connected component.
It follows that either (1) there exists a cycle in 
containing a vertex  of  in its interior
or (2) a pair of vertices of  in different
connected components of   are incident to the same face of .  
The claim (1) or (2) implies that there exists a cycle
 in  containing a vertex  of  in its interior.
Similarly as in the proof of Lemma~\ref{lemma:keyFact1}, by following
a directed path through  we obtain 
an oriented cycle  (this time not necessarily directed) in ,  whose interior is contained in the interior of  with  yielding a contradiction.

Indeed, as we proved in  the proof of Lemma~\ref{lemma:wn} 
.
Since  is not induced by a single cluster and follows in the interior of  a directed path, and  does not have any vertex in  we have
  and .
Hence, . 
\qed\end{proof}






By Lemma~\ref{lemma:keyFact2}, every  is a tree. Taking a close neighborhood of each such  as a disc representing
the cluster  we obtain a desired clustered embedding of . In the obtained embedding we just delete edges not belonging to  and that concludes the proof of the correctness of our algorithm. \\







\subsection{Normalization}
 \label{sec:norm}

 
 
 In the present section we normalize the instance so that~(i)-(iii) are satisfied.
We argued the connectedness in Introduction, and hence, (i) is taken care of.
\iflong 

A \emph{contraction} of an  edge  in a topological graph is an operation that turns
 into a vertex
by moving  along  towards  while dragging all the other edges incident to  along .
By a contraction we can introduce multi-edges or loops at the vertices.
We will also  use the following operation which can be thought of as the inverse operation of the edge contraction
in a topological graph.
A \emph{vertex split} in a drawing of a graph  is an operation that replaces a vertex  by two vertices  and 
drawn in a small neighborhood of  joined by a short crossing free edge so that the neighbors of  are partitioned into two parts
according to whether they are joined with  or  in the resulting drawing, the rotations at  and  are inherited from the
rotation at , and the new edges are drawn in the small  neighborhood of the edges they correspond to in .



Regarding~(ii), by a series of successive edge contractions we contract each connected component of 's to a vertex.
We delete any created loop.
If a loop at a vertex from  contains a vertex from a different cluster , , in its interior we know that the 
instance is not c-planar, since for every  all the vertices in  must be contained in the outer face of  in a positive instance. This all can be easily checked in  polynomial time.
Otherwise, a contraction preserves c-planarity of , since
deleted empty loops can be introduced in a c-planar embedding of the reduced graph,
and contracted edges recovered via vertex splits.
From now on we assume that clusters of  form independent sets. 
It remains to satisfy (iii).
\else
To achieve~(ii) is fairly standard by contracting components induced by clusters to vertices. \fi
Thus, it remains to satisfy (iii).


We want to sub-divide a non-simple face   into 
a pair of faces one of which is
semi-simple by a monotone path  w.r.t. .
Let  denote an oriented monotone  sub-walk of  w.r.t.  joining a local minimum  and maximum  of  minimizing  .
Let  denote the oriented  monotone walk with  immediately following  on the facial walk of , and let  be such walk  immediately preceding  on the facial walk of . Note that  and  exists due to the minimality of  and that we have
.
Similarly as in~\cite{F14+} we subdivide  into two faces  and 
by a strictly monotone path   w.r.t. . Hence,  .
We have .
Thus, by Lemma~\ref{lemma:wn} if  with  is semi-simple we obtain a simple face  with 
 and a semi-simple face  with  as desired.
Indeed, 
and . It remains to show the following lemma, since both  and  
are incident to less local minima and maxima than  if  is not semi-simple.
Hence, after  facial subdivisions we obtain a desired instance, since .

\begin{lemma}
\label{lemma:norm}
If the c-graph  is c-planar then by subdividing  of  by  into
a pair of faces  and , where  is semi-simple we obtain a c-planar c-graph. Moreover, 
 and .
\end{lemma}
\begin{proof}
The second statement is proved above.
Hence, we deal just with the first one.
Let  and  denote the first edge on  and the last edge on , respectively. Let  and 
denote the last edge on  and the first edge on , respectively.
Let  and  denote the last edge on  and the first edge on .
Let    and 
 denote the intersection of the edges  and , respectively,  with a ray separating a pair of clusters.
Let  and  denote the wedge between  and , respectively, in .


We presently show that subdividing  with   preserves c-planarity, since a clustered
embedding without  can be deformed so that  can be added to a clustered planar embedding without creating a crossing, while keeping the embedding clustered. 
This is not hard to see if, let's say , is convex and the line segment  is not crossed by an edge. Since  is
convex, the relative interior of  is contained in the interior of . Note that  is a sub-walk of  since  is not simple. We draw a curve  joining  with  following the walk  in its small neighborhood in the interior ; we cut  at its (two) intersection points with  and reconnected the severed ends on both sides by a curve following  in its small neighborhood thereby obtaining a closed curve, and a curve  joining   and . Finally,  can be subdivided by vertices thereby 
yielding a desired embedding of .
Otherwise, if  is concave
or  is crossed by an edge of  we need to deform the clustered
embedding of  so that this is not longer the case.

 \begin{figure}[h]
  \centering
\centering\textbf{•}
{
\includegraphics[scale=0.6]{stork}\textbf{•}
    	}

\caption{A pair of deformations of the clustered embedding of  so that  can be subdivided by . For the sake of clarity clusters are drawn as horiznotal strips rather than wedges.}
 \label{fig:stork}
\end{figure}


By a \emph{spur} with the \emph{tip}  we understand a closed curve  obtained  as  a concatenation of a line segment contained in a ray separating clusters
and a curve contained in the boundary of  passing through exactly one extreme 
of  such that the curve is longest possible. The \emph{length} is the spur
is one plus the number of its crossings with rays separating clusters divided by two.
 If  is concave, the vertex  is a tip of a spur whose length is the distance of  to a closest other extreme along the face. Note that both  and  must be paths in this case.
The rough idea in the omitted part of the proof is that shortest spurs have room around them to be deformed while maintaining the embedding
clustered such that  can be added.
Spurs  are deformed  as illustrated in Fig.~\ref{fig:stork}. (see Appendix for the rest of the proof)
\iflong


 \begin{figure}[h]
  \centering
\centering
{
\includegraphics[scale=0.65]{deform1}
    	}

\caption{Deformation in the case when both  and  have concave wedges incident to  that is indicated by grey.
The dashed curve represents the path  subdividing .
On the left,  point . In the middle, 
   point . On the right, the corresponding
 deformation.}
\label{fig:deform1}
\end{figure}


 First, we suppose that  is concave.
 W.l.o.g. we assume that . This holds when  is convex,
 Figure~\ref{fig:deform2}. Otherwise,
we exchange the roles of  and , see Figure~\ref{fig:deform1}. Combinatorially, there are two
cases depending on whether  is concave, but we treat them
simultaneously.
We isolate a part of the embedding of  inside a spur represented
by a topological disc . In order to get a desired deformed clustered embedding of  we define a homeomorphism from  that we use to redraw the corresponding part of   thereby disconnecting some edges
that are reconnected in the end.
 Let  denote the topological disc bounded by the closed curve
obtained by concatenating the line segment  with the parts
of  and  connecting endpoints of   with . 
We assume that  which holds automatically when  is concave due to . 

 
 

 \begin{figure}[h]
  \centering
\centering
\subfloat[]{
\label{fig:deform3}
\includegraphics[scale=0.65]{deform3}
    	}
    	\hspace{10px}
\subfloat[]{
\label{fig:deform5}
\includegraphics[scale=0.65]{deform5}	
    	}
\caption{(a) A ``covering map''  from a disc onto a spur. (b) Identifying parts of the boundary of the discs  
according to .}

\end{figure}
 

  If  does not have a concave wedge incident to  it could happen that the boundary of  crosses itself. As we will see later due to this reason we cannot simply put .
We use a ``covering map'' , see Figure~\ref{fig:deform3}.
In the light of Jordan-Sch\"onflies theorem,
  let a topological disc  be a pre-image of a continuous  map   such that the map  is injective when restricted (in the target) to the embedding of ;
 maps the boundary of  to the concatenation of 
with the parts of  and  connecting endpoints of   with ;
and the pre-image (of  parts) of the rays separating clusters
 consists of a union of a connected part of the boundary of  
 (contained in the pre-image of ) and
a set of pairwise disjoint diagonals of .
Treating  as a topological space let .  

 If  contains cut-vertices  can have pairs of boundary points identified, see Figure~\ref{fig:deform5}.
Let  denote  a line segment contained inside the intersection of the interior of  with a ray separating clusters containing , whose end vertex is very close to , and  if and only if  is convex. Let  denote a disc bounded by  and a curve 
joining the endpoints of  following  towards  and back in its small neighborhood in the interior of .
Let  be the connected component on the boundary of  in .
Let  for , denote a connected component of a pre-image by  of (a part of) a ray separating clusters. The segments 
's are indexed by the order of appearance of their endpoints along the boundary of .
A line segment  is just a point if it joins identified boundary points in .
Similarly, let  for , denote a connected component of the 
intersection of  with rays separating clusters.
We contract  to a point iff  is a single point
and we  contract  interior parts of  in the correspondence with  and 's as illustrated in Figure~\ref{fig:deform5}. Since , we have .

 
 

 \begin{figure}[h]
  \centering
\centering
\subfloat[]{
\includegraphics[scale=0.65]{deform2}
\label{fig:deform2}    	}
    	\hspace{3px}
\subfloat[]{
\includegraphics[scale=0.65]{deform4}
\label{fig:deform4}    	}

\caption{(a) Deformation in the case when only  has a concave wedge incident to  that is indicated by grey.
The dashed curve represents the desired path subdividing .
On the left, the line segment  crosses  edges of . 
On the right, the corresponding deformation. (b) The wedge at 
in , that is indicated by grey, is convex.  A sink   in the interior of a spur having the vertex 
as the tip.}
\end{figure}




\begin{figure}
 \centering
\includegraphics[scale=0.7]{surgery}
\caption{Re-routing  past  along the boundary of .}
\label{fig:deform8} 
\end{figure}

We map by a homeomorphism  the disc  to  so that 
is mapped to  for all 
and so that the endpoint  of  for which  is closer to  is mapped to the point of  closest to (farthest from) 
 if  is concave (convex), see Figure~\ref{fig:stork} for an illustration.
We alter the embedding of  by deleting  and replacing it
by .
Finally, we  reconnect the severed end pieces of edges intersecting  by curves inside the cluster containing  without
creating any edge crossing. 



If , we redraw the portion of  contained in the disc  (we override the previous ) bounded by the line segment 
and the parts of  and  joining end points of   with , see Figure~\ref{fig:deform6}.
Note that the boundary of the disc  is non-self intersecting.
If  does not cross  we make  crossing free by mapping
the part of  in  to a long skinny disc   (we override the previous ) in the vicinity of  and reconnecting the severed edges similarly 
as above.

 \begin{figure}
  \centering
\centering
\subfloat[]{
\includegraphics[scale=1]{3rdcaseT}
\label{fig:deform6}  }
    	\hspace{1px}
\subfloat[]{
\includegraphics[scale=1]{3rdcase2T}
\label{fig:deform7}}
\caption{The face  is indicated by grey. (a) The vertex   corresponding to the spur with the tip . (b) Corresponding  deformation involving  and the dashed path  subdividing .}
\end{figure}



If  crosses  we cannot apply the previous argument since ,
and thus, we need a different approach.
We temporarily add to  a path  starting at , following closely  in the interior of 
and ending in a vertex . Let  denote the crossing point of the last edge on 
separating clusters with a ray separating clusters. 
We cut edges at  by removing a small  neighborhood of their crossing points.
We reroute the paths   (see Figure~\ref{fig:deform8}) and  without crossing an edge of  from their severed ends outside of  past  and along  in the interior of  so that we closely follow the boundary of .
We  easily avoid creating crossings since we cut edges at .
Let   be the disc bounded by rerouted parts of  and  from their first intersections on the way from  and , respectively, with  to their common vertex ; and by  connecting the ends of those parts.
Let , , denote the line segments in the intersection of  with rays separating clusters listed in the order of appearance along the boundary of .
We deform the embedding of  by a mapping  from , see Figure~\ref{fig:deform7} for an illustration, such that  maps the relative interior of  to the relative interior of  , the parts of  and  in  to their rerouted counterparts, and for the open line segments , , in the intersection of  with the rays separating clusters we have .
The mapping  is then extended to the whole  such that (i)  is a homeomorphism when restricted to the interior
of the slab between  and , for all , where  is the 
final piece of the boundary of , and (ii) no edge crossings are introduced, i.e.,  is injective
and . It is not hard to see that  exists. Finally, we reconnect the severed end pieces of edges inside the cluster
containing  and remove .
The previous deformation is perhaps easier seen as follows. The clustered embedding of  can be made  arbitrarily skinny. Thus, for the purpose of  deformation, we can picture that  consists just of the part of  and  in  and a strictly monotone 
path starting at   as in Figure\ref{fig:deform6}. We map  as indicated in Figure~\ref{fig:deform7}.
 
   
Second, if both  and  are convex we can subdivide  by  unless  is intersected by edge(s) of  (we still assume that  ).
   However, if this is the case  (defined above) contains a sink or source  in its interior, see Figure~\ref{fig:deform4}. Note that we can assume that  is also a tip of a shortest spur of , and hence, the previous case applies.\fi
\qed\end{proof}



	










{\bf Acknowledgment} I would like to thank Jan Kyn\v{c}l and D\"om\"ot\"or P\'alv\"olgyi for many comments and suggestions that helped to improve the presentation of the result.


	
	
	




 


 
 

 
   



\bibliographystyle{plain}



\bibliography{bib}

\newpage


\section*{Appendix}


\subsection*{Fan drawings}
\label{sec:fan}
We show that the clusters can be drawn as regions, each bounded by a pair of rays emanating from the origin.
Suppose that  is given by a clustered embedding
living in the  plane of .
We assume that boundries of discs representing clusters do not touch.
Consider a stereographic projection from the north pole of a two-dimensional sphere  
sitting at the origin of .
Let  be a stereographical pre-image of the embedding of  on .
Let  denote the union of  (as a topological space) with the boundaries of the clusters in .
Let  and  be a connected component of the complement of  in , respectively, containing the north pole and south pole.
If necessary, we apply an isotopy  to  (a continuous deformation keeping  to be a clustered embedding all the time)  so that in the resulting embedding  of  on  every boundary of a cluster intersects (in fact touches) the closure of  and  the closure of . 

We show that a desired isotopy exists. We contract every cluster to a point thereby
treating clusters as vertices in an embedding  of a cycle  of length  having multi-edges. 
Formally, this can be viewed as a quotient , where  iff
 and  belong to the same cluster.
In  there must be a pair of faces  and  whose facial walk is  since any cycle in the corresponding multi-graph is obtained as a symmetric difference of facial walks. Apply an isotopy to  such that  contains
the north pole in its interior and  contains the south pole in its interior. Finally, we decontract clusters in the end. The above procedure can be easily turned into an isotopy of . 

By projecting the resulting spherical embedding back to the plan we can also assume that we have a clustered embedding of 
such that clusters are represented by small discs of diameter  each drawn in a close vicinity
of a different vertex of a regular convex -gon with the center at the origin, and the edges
between clusters  and , for every , are closely following the edge
of the -gon between the corresponding pair of vertices. 
The desired rays bounding clusters are those from the origin orthogonal to the sides of the -gon. \\


\section*{Normalization}


\begin{proof}[the omitted part of the proof from Section~\ref{sec:norm}]
 \begin{figure}[h]
  \centering
\centering
{
\includegraphics[scale=0.65]{deform1}
    	}

\caption{Deformation in the case when both  and  have concave wedges incident to  that is indicated by grey.
The dashed curve represents the path  subdividing .
On the left,  point . In the middle, 
   point . On the right, the corresponding
 deformation.}
\label{fig:deform1}
\end{figure}


 First, we suppose that  is concave.
 W.l.o.g. we assume that . This holds when  is convex,
 Figure~\ref{fig:deform2}. Otherwise,
we exchange the roles of  and , see Figure~\ref{fig:deform1}. Combinatorially, there are two
cases depending on whether  is concave, but we treat them
simultaneously.
We isolate a part of the embedding of  inside a spur represented
by a topological disc . In order to get a desired deformed clustered embedding of  we define a homeomorphism from  that we use to redraw the corresponding part of   thereby disconnecting some edges
that are reconnected in the end.
 Let  denote the topological disc bounded by the closed curve
obtained by concatenating the line segment  with the parts
of  and  connecting endpoints of   with . 
We assume that  which holds automatically when  is concave due to . 

 
 

 \begin{figure}[h]
  \centering
\centering
\subfloat[]{
\label{fig:deform3}
\includegraphics[scale=0.65]{deform3}
    	}
    	\hspace{10px}
\subfloat[]{
\label{fig:deform5}
\includegraphics[scale=0.65]{deform5}	
    	}
\caption{(a) A ``covering map''  from a disc onto a spur. (b) Identifying parts of the boundary of the discs  
according to .}

\end{figure}
 

  If  does not have a concave wedge incident to  it could happen that the boundary of  crosses itself. As we will see later due to this reason we cannot simply put .
We use a ``covering map'' , see Figure~\ref{fig:deform3}.
In the light of Jordan-Sch\"onflies theorem,
  let a topological disc  be a pre-image of a continuous  map   such that the map  is injective when restricted (in the target) to the embedding of ;
 maps the boundary of  to the concatenation of 
with the parts of  and  connecting endpoints of   with ;
and the pre-image (of  parts) of the rays separating clusters
 consists of a union of a connected part of the boundary of  
 (contained in the pre-image of ) and
a set of pairwise disjoint diagonals of .
Treating  as a topological space let .  

 If  contains cut-vertices  can have pairs of boundary points identified, see Figure~\ref{fig:deform5}.
Let  denote  a line segment contained inside the intersection of the interior of  with a ray separating clusters containing , whose end vertex is very close to , and  if and only if  is convex. Let  denote a disc bounded by  and a curve 
joining the endpoints of  following  towards  and back in its small neighborhood in the interior of .
Let  be the connected component on the boundary of  in .
Let  for , denote a connected component of a pre-image by  of (a part of) a ray separating clusters. The segments 
's are indexed by the order of appearance of their endpoints along the boundary of .
A line segment  is just a point if it joins identified boundary points in .
Similarly, let  for , denote a connected component of the 
intersection of  with rays separating clusters.
We contract  to a point iff  is a single point
and we  contract  interior parts of  in the correspondence with  and 's as illustrated in Figure~\ref{fig:deform5}. Since , we have .

 
 

 \begin{figure}[h]
  \centering
\centering
\subfloat[]{
\includegraphics[scale=0.65]{deform2}
\label{fig:deform2}    	}
    	\hspace{3px}
\subfloat[]{
\includegraphics[scale=0.65]{deform4}
\label{fig:deform4}    	}

\caption{(a) Deformation in the case when only  has a concave wedge incident to  that is indicated by grey.
The dashed curve represents the desired path subdividing .
On the left, the line segment  crosses  edges of . 
On the right, the corresponding deformation. (b) The wedge at 
in , that is indicated by grey, is convex.  A sink   in the interior of a spur having the vertex 
as the tip.}
\end{figure}




\begin{figure}
 \centering
\includegraphics[scale=0.7]{surgery}
\caption{Re-routing  past  along the boundary of .}
\label{fig:deform8} 
\end{figure}

We map by a homeomorphism  the disc  to  so that 
is mapped to  for all 
and so that the endpoint  of  for which  is closer to  is mapped to the point of  closest to (farthest from) 
 if  is concave (convex), see Figure~\ref{fig:stork} for an illustration.
We alter the embedding of  by deleting  and replacing it
by .
Finally, we  reconnect the severed end pieces of edges intersecting  by curves inside the cluster containing  without
creating any edge crossing. 



If , we redraw the portion of  contained in the disc  (we override the previous ) bounded by the line segment 
and the parts of  and  joining end points of   with , see Figure~\ref{fig:deform6}.
Note that the boundary of the disc  is non-self intersecting.
If  does not cross  we make  crossing free by mapping
the part of  in  to a long skinny disc   (we override the previous ) in the vicinity of  and reconnecting the severed edges similarly 
as above.

 \begin{figure}
  \centering
\centering
\subfloat[]{
\includegraphics[scale=1]{3rdcaseT}
\label{fig:deform6}  }
    	\hspace{1px}
\subfloat[]{
\includegraphics[scale=1]{3rdcase2T}
\label{fig:deform7}}
\caption{The face  is indicated by grey. (a) The vertex   corresponding to the spur with the tip . (b) Corresponding  deformation involving  and the dashed path  subdividing .}
\end{figure}



If  crosses  we cannot apply the previous argument since ,
and thus, we need a different approach.
We temporarily add to  a path  starting at , following closely  in the interior of 
and ending in a vertex . Let  denote the crossing point of the last edge on 
separating clusters with a ray separating clusters. 
We cut edges at  by removing a small  neighborhood of their crossing points.
We reroute the paths   (see Figure~\ref{fig:deform8}) and  without crossing an edge of  from their severed ends outside of  past  and along  in the interior of  so that we closely follow the boundary of .
We  easily avoid creating crossings since we cut edges at .
Let   be the disc bounded by rerouted parts of  and  from their first intersections on the way from  and , respectively, with  to their common vertex ; and by  connecting the ends of those parts.
Let , , denote the line segments in the intersection of  with rays separating clusters listed in the order of appearance along the boundary of .
We deform the embedding of  by a mapping  from , see Figure~\ref{fig:deform7} for an illustration, such that  maps the relative interior of  to the relative interior of  , the parts of  and  in  to their rerouted counterparts, and for the open line segments , , in the intersection of  with the rays separating clusters we have .
The mapping  is then extended to the whole  such that (i)  is a homeomorphism when restricted to the interior
of the slab between  and , for all , where  is the 
final piece of the boundary of , and (ii) no edge crossings are introduced, i.e.,  is injective
and . It is not hard to see that  exists. Finally, we reconnect the severed end pieces of edges inside the cluster
containing  and remove .
The previous deformation is perhaps easier seen as follows. The clustered embedding of  can be made  arbitrarily skinny. Thus, for the purpose of  deformation, we can picture that  consists just of the part of  and  in  and a strictly monotone 
path starting at   as in Figure\ref{fig:deform6}. We map  as indicated in Figure~\ref{fig:deform7}.
 
   
Second, if both  and  are convex we can subdivide  by  unless  is intersected by edge(s) of  (we still assume that  ).
   However, if this is the case  (defined above) contains a sink or source  in its interior, see Figure~\ref{fig:deform4}. Note that we can assume that  is also a tip of a shortest spur of , and hence, the previous case applies.
\qed\end{proof}




\end{document}