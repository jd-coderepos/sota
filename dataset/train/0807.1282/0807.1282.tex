\documentclass[envcountsect, envcountsame]{llncs}

\usepackage{textcomp}
\usepackage{amsmath}
\usepackage{amssymb}
\usepackage{graphics}
\usepackage{psfrag}

\newtheorem{prop}{Proposition}[section]



\newcommand{\E}{\ensuremath{\mathbb{E}}}
\newcommand{\forig}{\ensuremath {F_{\rm orig}}}
\newcommand{\fminus}{\ensuremath{F^-}}
\newcommand{\fplus}{\ensuremath{F^+}}
\newcommand{\fdreiplus}{\ensuremath{F^{3+}}}
\newcommand{\fzweiplus}{\ensuremath{F^{2+}}}
\newcommand{\feinsplus}{\ensuremath{F^{1+}}}
\newcommand{\safe}{\ensuremath{S}}
\newcommand{\vbl}{\ensuremath{{\rm vbl}}}
\newcommand{\sat}{\ensuremath{{\rm sat}}}
\newcommand{\poly}{\ensuremath{{\rm poly}}}
\newcommand{\ex}{\ensuremath{{\rm ex}}}
\newcommand{\C}{\ensuremath{{\mathcal{C}}}}
\newcommand{\wrt}{w.\,r.\,t.\ }
\newcommand{\obda}{w.\,l.\,o.\,g.\ }
\newcommand{\Obda}{w.\,l.\,o.\,g.\ }
\newcommand{\Fk}{\ensuremath{\mathbb{F}_k}}
\newcommand{\Cke}{\ensuremath{C_{k,\epsilon}}}
\addtolength{\oddsidemargin}{-.4in}
\addtolength{\evensidemargin}{-.4in}
\addtolength{\textwidth}{.5in}


\addtolength{\textheight}{1in}

\begin{document}
\frontmatter
\pagestyle{headings}
\title{Satisfiability of Almost Disjoint CNF Formulas}

\author{Dominik Scheder}
\institute{Theoretical Computer Science, ETH Z\"urich\\
CH-8092 Z\"urich, Switzerland\\
\email{dscheder@inf.ethz.ch}}
\maketitle
\begin{center}
 \today
\end{center}

\begin{abstract}
  We call a CNF formula {\em linear} if any two clauses have at most
  one variable in common. Let  be the largest integer  such that
  any linear -CNF formula with  clauses is satisfiable.
  We show that .

  More generally, a -CSP is a constraint satisfaction problem
  in conjunctive normal form where each variable can take on one of  
  values, and each constraint contains  variables and 
  forbids exacty one of the  possible assignments to these variables.
  Call a -CSP -disjoint if no two distinct constraints have
   or more variables in common. Let  denote the
  largest integer  such that any -disjoint -CSP with	
  at most  constraints is 
  satisfiable. We show that

  
  for some constant .  This means for constant , upper and
  lower bound differ only in a polynomial factor in  and .
\end{abstract}

\section{Introduction}

How difficult is it to come up with an unsatisfiable CNF formula?
Stupid question, you might think: , here is
one. Two clauses, each containing one literal, and unsatisfiable.
Well, yes, but what if we want a -CNF formula, i.e., we require
that every clause contains exactly  literals? Now it's a little bit
less trivial, but still easy: Take a clause ,
then , ,
until you have exhausted all  combinations of negative and
positive literals. Each assignment to the  variables is ruled out
by exactly one clause: Your formula has  clauses, and it is
unsatisfiable. This formula is the ``simplest'' unsatisfiable -CNF
formula, in a sense as  is the simplest non--colorable
graph. What if we impose further restrictions? For example, what if no
variable can occur in more than one clause? This restriction is surely
too strong: One can satisfy each clause individually, hence such
a formula is always satisfiable, unless it contains the empty clause.\\

Let us consider two weaker restrictions. First, what if each variable
may occur in several clauses of our -CNF formula, but in at most
? Let us call such a formula a {\em -bounded -CNF formula}.
Second, what if we allow every {\em pair} of variables to occur in at
most one clause, or, equivalently, allow any two clauses to have at
most one variable in common? Such a formula is called, in analogy to
hypergraph terminology, a {\em linear} -CNF formula.\\

The first problem has been introduced by Tovey~\cite{Tovey84}, who
showed, using Hall's Marriage Theorem, that every -bounded -CNF
formula is satisfiable. This has been improved by Kratochv\'{i}l,
Savick\'{y} and Tuza~\cite{KST93}, who proved that there is some
threshold function  such that any -bounded -CNF formula
is satisfiable, but deciding satisfiability of -bounded
-CNF formulas is already NP-complete, and further, that .  For an upper bound on how often we can allow a
variable to occur while still guaranteeing satisfiability, Hoory and
Szeider~\cite{HS06} show how to construct unsatisfiable -bounded
-CNF formula for -CNF formulas.
Thus,  is known up to a logarithmic factor.\\

For the second question, let us give an unsatisfiable linear
-CNF formula: 

This formula has  clauses, which is as few as possible for
unsatisfiable linear -CNF formulas. Finding an unsatisfiable
-CNF formula is already much harder. Hence we may ask the following
question: For which  do unsatisfiable linear -CNF formulas
exist, and if they exist, how many clauses do they have?  The
existence question has been answered by Porschen, Speckenmeyer and
Zhao~\cite{PSZ08}, who give an explicit construction of unsatisfiable
linear -CNF formulas, for any .  However, the size of their
formulas (i.e., the number of clauses), is gigantic: Let  be the
size of the unsatisfiable linear -CNF formula obtained by the
construction in ~\cite{PSZ08}.  Then  and . In this paper, we prove that much smaller unsatisfiable
linear -CNF formulas exist, namely of size , and
complement this by proving a lower bound of .
Since the smallest non-linear unsatisfiable -CNF formula has
exactly  clauses, this shows that unsatisfiable linearity
formuals require
significantly more clauses than non-linear ones.\\

A similar problem has been investigated, and to large extent solved,
for hypergraphs: An -hypergraph  is a hypergraph where
every edge has  vertices, and a proper -coloring of
 is a coloring of the vertices such that no edge is
monochromatic. A hypergraph is called {\em linear} if  for any two distinct edges  of . It is
easy to construct a non--colorable -hypergraph, for any  and
.  However, it is not obvious whether non--colorable {\em
  linear} -hypergraphs exist. For , this has been positively
answered by Abbott~\cite{Abbott65}. For general , existence follows
from the Hales-Jewett theorem~\cite{HJ63}.  Using Ramsey-like
theorems, the obtained bounds on the size of  have been
quite poor.  Tight bounds --- up to a constant factor --- have later
been given by Kostochka, Mubayi, R\"odl and
Tetali~\cite{KMRT01}, usign probabilistic techniques.\\

\subsection{Notation and Terminology}

Though we are primarily interested in linear -CNF formulas, our
methods apply to a much more general class, namely -constraint
satisfaction problems, or short -CSPs. This is basically the
same as a -CNF formula, only that each variable can take on one
of  different values, not just  as in the binary case.  In this
context, a literal is an inequality , where  is a variable
and . A -constraint is a set of set of 
literals, and a -CSP is a set of -constraints.  An {\em
  assignment} is a mapping from variables to .  An
assignments  {\em satisfies} a literal  if, well,
. It satisfies a constraint if it satisfies {\em at
  least one} literal in it, and it satisfies a CSP if it satisfies
every constraint of it. An issue that sometimes causes confusion is
whether one allows a constraint to contain several literals involving
the same variable. We do not. However, this is not important, since
such a constraint, e.g.,  would be satisfied by
every assignment anyway. \\

We say variable  {\em occurs} in constraint  if  contains the
literal  for some . For a CSP ,
we denote by  the number of constraints  in which
 occurs, and by  the set of all variables occurring in
constraint . For example . A CSP  is called {\em -disjoint} if there are no
two distinct constraints  with .  Thus, a linear -CNF formula is a -disjoint
-CSP.

\subsection{Results}

Let  be the largest integer  such that any linear -CNF
formula with  clauses is satisfiable.  For CSPs, let
 denote the largest integer  such that any
-disjoint -CSP with at most  constraints is
satisfiable. Clearly . Our main result is

\begin{theorem}
  There is some constant  such that
  
\label{main-result}
\end{theorem}

To understand these bounds, suppose  is constant.
Then the dominating term is  in both the
upper and lower bound, and the two bounds differ only by a polynomial
factor in  and .  For linear -CNF formulas, we obtain



Compare this with the bound for general -CSPs: The smallest
unsatisfiable -CSP has exactly  constraints. 


\section{A Lower Bound}
\label{section-lower}

Our main tool to prove a lower bound is the symmetric version of the
Lov\'asz Local Lemma (see e.g.~\cite{AS00}):

\begin{lemma}[Lov\'asz Local Lemma]
  Let  be events in a probability
  space with  for every . If each event
   is independent of all other events except at most
   many, and , then .
  \label{lemma-local}
\end{lemma}

The following corollary states that any CSP is satisfiable unless some
variable occurs ``too often''. This has been shown by~\cite{KST93} for
, and their proof directly generalizes to general .

\begin{corollary}
  If  is a -CSP and 
  for every variable , then  is satisfiable.
  \label{low-deg}
\end{corollary}

\begin{proof}
  Assign each variable uniformly at random a value from
  .  Write  and let
   be the event that constraint  is not satisfied.
  Clearly .  Event  is
  independent of all other events except those events 
  where , i.e. those
  constraints sharing a variable with . Since 
  contains  variables, and each occurs in at most
   other clauses,  shares a variable with at
  most  other
  clauses. By Lemma~\ref{lemma-local}, with positive probability none
  of the events  occurs, i.e.,  is satisfiable.
  \qed
\end{proof}

Let  be a -CSP. We call  {\em frequent in } if
. Our idea is that an
-disjoint -CSP with few frequent variables can be
transformed into a -CSP  having no frequent
variable. By Corollary~\ref{low-deg},  is satisfiable, and the
transformation is such that  is satisfiable, too.

\begin{theorem}
  Any -disjoint -CSP with  frequent variables is
  satisfiable.
\label{few-frequent}
\end{theorem}

\begin{proof}
  We obtain a new formula  by removing certain literals from
  certain clauses: For each constraint , we distinguish two
  cases: If  contains less than  variables that are frequent
  in , let  by  minus all literals involving one of these
  frequent variables. Otherwise, let  just be .  We define . Observe that  contains constraints of
  different sizes, ranging from  to . Further, for
  each constraint in , the number of variables in 
   that are frequent in  is either  or .
  
  We claim that  for any
  variable . If  is not frequent in , this is obvious, since
  .  If  is frequent in , let ,  be the clauses of  containing .
  Clearly, each  contains , which is frequent in .  For
  each  containing ,  contains at least 
  variables besides  which are frequent in . We pick  of
  them arbitrarily and call this set .  Clearly  for
  , otherwise the -set  would occur in
   and , contradicting -disjointness of .  Let 
  be the number of frequent variables in . There are at most  choices for an -set of frequent
  variables, thus
    
  We would now like to apply Corollary~\ref{low-deg} for
  -CSPs.  However,  is not a -CSP,
  because it may still contain larger constraints. This is no problem,
  as we can further delete literals until every constraint has size
  exactly .  This process clearly does not increase any
  . Hence, by Corollary~\ref{low-deg},  is
  satisfiable, and so is .  \qed
\end{proof}

\textbf{Proof of the lower bound in Theorem~\ref{main-result}.}
Assume  is an unsatisfiable -disjoint -CSP. Then by
Theorem~\ref{few-frequent}, we have
 frequent
variables. Since

and  for all , it follows that  has more than
 constraints.

\section{The Upper Bound}

In this section we complement our lower bound by an upper bound. The
ratio of upper and lower bound will be polynomial in  and , but
the degree of the polynomial will depend on .  \\

The proof of the upper bound uses the first moment method and proceeds
in two steps. First, we show that for given  and , we
can find an -disjoint -CSP  over  variables with
``many'' clauses. In a second step, we replace each literal 
in each constraint of  by , where  is each time
chosen independently uniformly at random from ,
resulting in a random -disjoint -CSP .  We will show
that for the right values of ,  is unsatisfiable with positive
probability.\\

As long as we do not care about the values  in the literals, a CSP
is basically nothing more than a hypergraph.

\begin{lemma}
  Let . There exists an -disjoint
  -uniform hypergraph with

edges.
\label{large-hypergraph}
\end{lemma}

\begin{proof}
  We will actually prove something stronger. Let  be the
  set of all -sets of . We claim that any {\em
    maximal} -disjoint subfamily  has at least  sets. Suppose  is maximal. For , we say  is
  {\em incompatible} with  . Note that by
  this definition,  is incompatible with itself.  By maximality of
  , each  is incompatible with some . For each , there are at most
  
  sets  incompatible with : Each fixed
  -subset of  is contained in 
  subsets of , and  contains  such
  -subsets.  Hence , and the claim follows after a short
  calculation.  \qed
\end{proof}

We bound , the size of the -disjoint -hypergraph on
 vertices, from below by a formula that will be easier to work with:


We can obtain a -CSP over variable set 
from a -uniform hypergraph over vertex set  by
simply replacing each edge  by a constraint
, where we sample each 
independently and uniformly at random from . We
obtain a random CSP . Any fixed assignment  has a chance of
 to satisfy a random constraint, and each random constraints
is chosen independently. Hence  satisfies  with probability
, where  is the number of constraints.
The expected number of satisfying assignments of  is



If we can choose  and  such that the latter term is ,
then with positive probability,  is not satisfiable. We re-write
this condition:

Combining this with (\ref{formula-m}), we see that it suffices
to choose  such that

and we choose

Hence there is some constant 
such that


With these values of  and , the rightmost term in
(\ref{number-sat}) is , and thus with positive probability,
the random -CSP  has  satisfying assignments.
This finishes the proof of Theorem~\ref{main-result}.\\

\section{Conclusions and Open Problems}

We determined the value of  up to a factor that is,
for constant , polynomial in  and . Can one eliminate
the exponential factor  in the lower bound? \\

Further, we do not have any good {\em explicit} construction of
unsatisfiable linear -CNF formulas. Can one derandomize our
randomized construction? Our lower bound suffers from a similar
problem: Given an -disjoint  -CSP formula  with  frequent
variables, we know that  is satisfiable, but we do not know how to
find a satisfying assignment in polynomial time.\\

Last, can one obtain any good lower bound on  that does
not use the Lov\'asz Local Lemma? 


\bibliographystyle{splncs}
\bibliography{refs}


\end{document}
