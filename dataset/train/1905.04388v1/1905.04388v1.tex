

\typeout{Multi-Pass Q-Networks for Deep Reinforcement Learning with Parameterised Action Spaces}



\documentclass{article}
\pdfpagewidth=8.5in
\pdfpageheight=11in
\usepackage{ijcai19}

\usepackage{times}
\usepackage{soul}
\usepackage{url}
\usepackage[hidelinks]{hyperref}
\usepackage[utf8]{inputenc}
\usepackage[small]{caption}
\usepackage{graphicx}
\usepackage{amsmath}
\usepackage{booktabs}
\usepackage{algorithm}
\usepackage{algorithmic}
\urlstyle{same}

\usepackage[list=false]{subcaption}
\usepackage{amssymb}
\usepackage{tablefootnote}
\usepackage{siunitx} \usepackage{multicol}				\usepackage{multirow}
\usepackage{hhline}

\usepackage{mathtools}  \usepackage{xfrac}   

\usepackage{arydshln}

\graphicspath{{./images/}}

\newcommand{\Expec}{\displaystyle \mathbb{E}}
\newcommand{\Real}{\mathbb{R}}

\def\PDQN*{P\nobreakdash-DQN}
\def\SPDQN*{SP\nobreakdash-DQN}
\def\MPDQN*{MP\nobreakdash-DQN}
\def\QPAMDP*{Q\nobreakdash-PAMDP}
\def\PADDPG*{PA\nobreakdash-DDPG}

\sisetup{
	table-align-uncertainty=true,
	separate-uncertainty=true,
}

\newcommand{\citet}[1]
{\citeauthor{#1}~\shortcite{#1}}
\newcommand{\citep}{\cite}
\newcommand{\citealp}[1]
{\citeauthor{#1}~\citeyear{#1}}

\makeatletter
\def\adl@drawiv#1#2#3{\hskip.5\tabcolsep
	\xleaders#3{#2.5\@tempdimb #1{1}#2.5\@tempdimb}#2\z@ plus1fil minus1fil\relax
	\hskip.5\tabcolsep}
\newcommand{\cdashlinelr}[1]{\noalign{\vskip\aboverulesep
		\global\let\@dashdrawstore\adl@draw
		\global\let\adl@draw\adl@drawiv}
	\cdashline{#1}
	\noalign{\global\let\adl@draw\@dashdrawstore
		\vskip\belowrulesep}}
\makeatother

\newrobustcmd{\B}{\fontseries{b}\selectfont}
\renewrobustcmd{\boldmath}{}



\title{Multi-Pass Q-Networks for Deep Reinforcement Learning with~\\Parameterised Action Spaces}

\author{
Craig J. Bester\and Steven D. James\and
George D. Konidaris\\
\affiliations
University of the Witwatersrand, Johannesburg\\
Brown University, Providence RI\\
}

\begin{document}

\maketitle

\begin{abstract}
Parameterised actions in reinforcement learning are composed of discrete actions with continuous action-parameters. This provides a framework for solving complex domains that require combining high-level actions with flexible control. The recent \PDQN* algorithm extends deep Q-networks to learn over such action spaces. However, it treats all action-parameters as a single joint input to the Q-network, invalidating its theoretical foundations. We analyse the issues with this approach and propose a novel method---multi-pass deep Q-networks, or \MPDQN*---to address them. We empirically demonstrate that \MPDQN* significantly outperforms \PDQN* and other previous algorithms in terms of data efficiency and converged policy performance on the Platform, Robot Soccer Goal, and Half Field Offense domains.
\end{abstract}


\section{Introduction}

Reinforcement learning (RL) and deep RL in particular have demonstrated remarkable success in solving tasks that require either discrete actions, such as Atari \citep{mnih2015}, or continuous actions, such as robot control \citep{schulman2015,lillicrap2015}. Reinforcement learning with parameterised actions \citep{masson2016} that combine discrete actions with continuous action-parameters has recently 
emerged as an additional setting of interest,  
allowing agents to learn flexible behavior in tasks such as 2D robot soccer
\citep{hausknecht2016,hussein2018}, simulated human-robot interaction \citep{khamassi2017}, and terrain-adaptive bipedal and quadrupedal locomotion \citep{peng2016}.



There are two main approaches to learning with parameterised actions: alternate between optimising the discrete actions and continuous action-parameters separately \citep{masson2016,khamassi2017}, or collapse the parameterised action space into a continuous one \citep{hausknecht2016}. Both of these approaches fail to fully exploit the structure present in parameterised action problems. The former does not share information between the action and action-parameter policies, while the latter does not take into account which action-parameter is associated with which action, or even which discrete action is executed by the agent. More recently, \citet{xiong2018} introduced \PDQN*, a method for learning behaviours directly in the parameterised action space. This leverages the distinct nature of the action space and is the current state-of-the-art algorithm on 2D robot soccer and King of Glory, a multiplayer online battle arena game. However, the formulation of the approach is flawed due to the dependence of the discrete action values on all action-parameters, not only those associated with each action. In this paper, we show how the above issue leads to suboptimal decision-making. We then introduce a novel multi-pass method to separate action-parameters, and demonstrate that the resulting algorithm---\MPDQN*---outperforms existing methods on the Platform, Robot Soccer Goal, and Half Field Offense domains.

\section{Background}

Parameterised action spaces \citep{masson2016} consist of a set of discrete actions, , where each  has a corresponding continuous action-parameter \mbox{} with dimensionality . This can be written as

We consider environments modelled as a Parameterised Action Markov Decision Process (PAMDP) \cite{masson2016}. For a PAMDP \mbox{}:  is the set of all states,  is the parameterised action space,  is the Markov state transition probability function,  is the reward function, and  is the future reward discount factor. An action policy  maps states to actions, typically with the aim of maximising Q-values , which give the expected discounted return of executing action   in state  and following the current policy thereafter. 

The \QPAMDP* algorithm \citep{masson2016} alternates between learning a discrete action policy with fixed action-parameters using Sarsa() \citep{sutton1998} with the Fourier basis \citep{konidaris2011} and optimising the continuous action-parameters using episodic Natural Actor Critic (eNAC) \citep{peters2005} while the discrete action policy is kept fixed.
\citet{hausknecht2016} apply artificial neural networks and the Deep Deterministic Policy Gradients (DDPG) algorithm \citep{lillicrap2015} to parameterised action spaces by treating both the discrete actions and their action-parameters as a joint continuous action vector. This can be seen as relaxing the parameterised action space (Equation~\ref{eq:parameterised_action_spaces}) 
into a continuous one:

where  are continuous values in . An -greedy or softmax policy is then used to select discrete actions. However, not only does this fail to exploit the disjoint nature of different parameterised actions, but optimising over the joint action and action-parameter space can result in premature convergence to suboptimal policies, as occurred in experiments by \citet{masson2016}. We henceforth refer to the algorithm used by \citeauthor{hausknecht2016} as \PADDPG*. 


\begin{figure}
	\centering
\includegraphics[height=6cm]{PDQNsmall17.pdf}
	\caption[The \PDQN* network architecture]{The \PDQN* network architecture~\protect\citep{xiong2018}. Note that the joint action-parameter vector  is fed into the Q-network.}
	\label{fig:pdqn}
\end{figure}

\subsection{Parameterised Deep Q-Networks}
Unlike previous approaches, \citet{xiong2018} introduce a method that operates in the parameterised action space directly by combining DQN and DDPG. Their \PDQN* algorithm achieves state-of-the-art performance using a Q-network to approximate Q-values used for discrete action selection, in addition to providing critic gradients for an actor network that determines the continuous action-parameter values for all actions.
By framing the problem as a PAMDP directly, rather than alternating between discrete and continuous action MDPs as with \QPAMDP*, or using a joint continuous action MDP as with \PADDPG*, \PDQN* necessitates a change to the Bellman equation to incorporate continuous action-parameters:

To avoid the computationally intractable calculation of the supremum over , \citet{xiong2018} state that when the  function is fixed, one can view  as a function  for any state  and . This allows the Bellman equation to be rewritten as:

\PDQN* uses a deep neural network with parameters  to represent , and a second deterministic actor network with parameters  to represent the action-parameter policy , an approximation of . With this formulation it is easy to apply the standard DQN approach of minimising the mean-squared Bellman error to update the Q-network using minibatches sampled from replay memory  \cite{mnih2015}, replacing  with :

where  is the update target derived from Equation~\eqref{eq:parameterised_bellman}. Then, the loss for the actor network in \PDQN* is given by the negative sum of Q-values:

Although this choice of loss function was not motivated by \citet{xiong2018}, it resembles the deterministic policy gradient loss used by \PADDPG* where a scalar critic value is used over all action-parameters \citep{hausknecht2016}. During updates, the estimated Q-values are backpropagated through the critic to the actor, producing gradients indicating how the action-parameters should be updated to increase the Q-values.

\section{Problems with Joint Action-Parameters}
\begin{figure*}[ht!]
	\begin{subfigure}[t]{0.32\textwidth}
		\centering
		\includegraphics[width=\textwidth]{PlatformStatev2.pdf}
\caption{Agent in the Platform domain.}
		\label{fig:platform_state}
	\end{subfigure}\hspace{0.1cm}
	\begin{subfigure}[t]{0.67\textwidth}
		\centering
\includegraphics[width=\textwidth]{qvalues_paper_platformv3.pdf}
\caption{Predicted Q-values versus leap action-parameter. Vertical lines indicate the leap value actually chosen. Ideally, the run and hop Q-values should remain constant since their action-parameters are fixed.}
		\label{fig:platform_state_qvalues}
	\end{subfigure}
	\caption[Example of sensitivity to unrelated action-parameters affecting discrete action selection]{Example of dependence on unrelated action-parameters affecting discrete action selection on the Platform domain. Three parameterised actions are available: \emph{run}, \emph{hop}, and \emph{leap}. In a particular state (\subref{fig:platform_state}), the optimal action is to run forward to be able to traverse a gap, while choosing to leap would cause the agent to fall and die. The Q-value of the leap action should change with its action-parameter, but (\subref{fig:platform_state_qvalues}) shows that varying the leap action-parameter while the others are kept fixed changes the Q-values predicted by \PDQN* for \emph{all} actions. Near the start of training, this can alter the discrete policy such that a suboptimal action is chosen. After  episodes, \PDQN* correctly learns to choose the optimal action regardless of the unrelated leap action-parameter, although the other Q-values still vary.}
	\label{fig:example_sensitivity}
\end{figure*}
The \PDQN* architecture inputs the joint action-parameter vector over all actions to the Q-network, as illustrated in Figure~\ref{fig:pdqn}. This was pointed out by \citet{xiong2018} but they did not discuss it further. While this may seem like an inconsequential implementation detail, it changes the formulation of the Bellman equation used for parameterised actions (Equation~\ref{eq:parameterised_bellman}) since each Q-value is a function of the joint action-parameter vector \mbox{}, rather than only the action-parameter  corresponding to the associated action:

This in turn affects both the updates to the Q-values and the action-parameters. Firstly, we consider the effect on the action-parameter loss, specifically that each Q-value produces gradients for all action-parameters. Consider for demonstration purposes the action-parameter loss (Equation~\ref{eq:action_parameter_loss}) over a single sample with state :

The policy gradient is then given by:

Expanding the gradients with respect to the action-parameters gives

where . Theoretically, if each Q-value were a function of just  as the \PDQN* formulation intended, then \mbox{} and  simplifies to:

However this is not the case in \PDQN*, so the gradients with respect to other action-parameters  are not zero in general. This is a problem because each Q-value is updated only when its corresponding action is sampled, as per Equation~\ref{eq:pdqn_qnetwork_loss}, and thus has no information on what effect other action-parameters  have on transitions or how they should be updated to maximise the expected return. They therefore produce what we term \emph{false gradients}. This effect may be mitigated by the summation over all Q-values in the action-parameter loss, since the gradients from each Q-value are summed and averaged over a minibatch.

The dependence of Q-values on all action-parameters also negatively affects the discrete action policy. Specifically, updating the continuous action-parameter policy of any action perturbs the Q-values of \emph{all} actions,  not just the one associated with that action-parameter. This can lead to the relative ordering of Q-values changing, which in turn can result in suboptimal greedy action selection. We demonstrate a situation where this occurs on the Platform domain in Figure~\ref{fig:example_sensitivity}.


\section{Multi-Pass Q-Networks}

The na\"ive solution to the problem of joint action-parameter inputs in \PDQN* would be to split the Q-network into separate networks for each discrete action. Then, one can input only the state and relevant action-parameter  to the network corresponding to . However, this drastically increases the computational and space complexity of the algorithm due to the duplication of network parameters for each action. Furthermore, the loss of the shared feature representation between Q-values may be detrimental.

We therefore consider an alternative approach that does not involve architectural changes to the network structure of \PDQN*. While separating the action-parameters in a single forward pass of a single Q-network with fully connected layers is impossible, we can do so with multiple passes. We perform a forward pass once per action  with the state  and action-parameter vector  as input, where  is the standard basis vector for dimension . Thus  is the joint action-parameter vector where each  is set to zero. This causes all false gradients to be zero, , and completely negates the impact of the network weights for unassociated action-parameters  from the input layer, making  only depend on . That is, 
 
Both  problems are therefore addressed without introducing any additional neural network parameters. We refer to this as the \emph{multi-pass Q-network} method, or \MPDQN*.

A total of  forward passes are required to predict all Q-values instead of one. However, we can make use of the parallel minibatch processing capabilities of artificial neural networks, provided by libraries such as PyTorch and Tensorflow, to perform this in a single parallel pass, or \emph{multi-pass}. A multi-pass with  actions is processed in the same manner as a minibatch of size :

where  is the Q-value for action  generated on the \textsuperscript{th} pass where  is non-zero. Only the diagonal elements  are valid and used in the final output . This process is illustrated in Figure~\ref{fig:mpdqn}.
\begin{figure}[ht]
	\centering
\includegraphics[height=8.2cm]{MPDQN8.pdf}
	\caption{Illustration of the multi-pass Q-network architecture.}
	\label{fig:mpdqn}
\end{figure}

Compared to separate Q-networks, our multi-pass technique introduces a relatively minor amount of overhead during forward passes. Although minibatches for updates are similarly duplicated  times, backward passes to accumulate gradients are not duplicated since only the diagonal elements  are used in the loss function. The computational complexity of this overhead scales linearly with the number of actions and minibatch size during updates. Unlike separate Q-networks (and even when a larger Q-network with more hidden layers and neurons is used) if the number of actions does not change, then the overhead of multi-passes would be the same as with a smaller Q-network, provided the minibatch is of a reasonable size and can be processed in parallel.

\section{Experiments}

We compare the original \PDQN* algorithm with a single Q-network against our proposed multi-pass Q-network (\MPDQN*), as well as against separate Q-networks (\SPDQN*). We also compare against \QPAMDP* and \PADDPG*, the former state-of-the-art approaches on their respective domains. We are unable to use King of Glory as a benchmark domain as it is closed-source and proprietary.

Similar to \citet{mnih2015} and \citet{hausknecht2016}, we add target networks to \PDQN* to compute the update targets  for stability. Soft updates (Polyak averaging) are used for the target networks. Adam \citep{kingma2014} with  is used to optimise the neural network parameters for \PDQN* and \PADDPG*. Layer weights are initialised following the strategy of \citet{he2015} with rectified linear unit (ReLU) activation functions. We employ the inverting gradients approach to bound action-parameters for both algorithms, as \citet{hausknecht2016} claim \PADDPG* is unable to learn without it on Half Field Offense. Action-parameters are scaled to , as we found this increased performance for all algorithms. 

We perform a hyperparameter grid search for Platform and Robot Soccer Goal over: the network learning rates ; Polyak averaging factors ; minibatch size ; and number of hidden layers and neurons in . The hidden layers are kept symmetric between the actor and critic networks as in previous works. Each combination is tested over  random runs for \PDQN* and \PADDPG* separately on each domain. The same hyperparameters are used for \PDQN*, \SPDQN* and \MPDQN*. 

To keep the comparison with \PADDPG* fair, we do not use dueling networks \citep{wang2016} nor asynchronous parallel workers as \citet{xiong2018} used for \PDQN*. For each algorithm and domain, we train  agents with unique random seeds and evaluate them without exploration for another  episodes. Our experiments are implemented in Python using PyTorch \citep{paszke2017} and OpenAI Gym \citep{openaigym}, and run on the following hardware: Intel Core i7-7700, 16GB DRAM, NVidia GTX 1060 GPU. Complete source code is available online.\footnote{\url{https://github.com/cycraig/MP-DQN}}

\subsection{Platform}
The Platform domain \citep{masson2016} has three actions---run, hop, and leap---each with a continuous action-parameter to control horizontal displacement. The agent has to hop over enemies and leap across gaps between platforms to reach the goal state. The agent dies if it touches an enemy or falls into a gap. A -dimensional state space gives the position and velocity of the agent and local enemy along with features of the current platform such as length. 


We train agents on this domain for  episodes, using the same hyperparameters for \QPAMDP* as \citet{masson2016}, except we reduce the learning rate for eNAC () to , and exploration noise variance () to , to account for the scaled action-parameters. For \PDQN*, shallow networks with one hidden layer  were found to perform best with , , , , and . \PADDPG* uses two hidden layers  with , , , , and . A replay memory size of  samples is used for both algorithms, update gradients are clipped at , and .

We introduce a \emph{passthrough layer} to the actor networks of \PDQN* and \PADDPG* to initialise their action-parameter policies to the same linear combination of state variables that \citet{masson2016} use to initialise the \QPAMDP* policy. The weights of the passthrough layer are kept fixed to avoid instability; this does not reduce the range of action-parameters available as the output of the actor network compensates before inverting gradients are applied. We use an -greedy discrete action policy with additive Ornstein-Uhlenbeck noise for action-parameter exploration, similar to \citet{lillicrap2015}, which we found gives slightly better performance than Gaussian noise.

\subsection{Robot Soccer Goal}
The Robot Soccer Goal domain \citep{masson2016} is a simplification of RoboCup 2D \citep{kitano1997} in which an agent has to score a goal past a keeper that tries to intercept the ball. The three parameterised actions---kick-to, shoot-goal-left, and shoot-goal-right---are all related to kicking the ball, which the agent automatically approaches between actions until close enough to kick again. The state space consists of  continuous features describing the position, velocity, and orientation of the agent and keeper, and the ball's position and distance to the keeper and goal.

\begin{figure}[ht!]
	\begin{subfigure}[t]{\linewidth}
		\centering
		\includegraphics[width=0.95\textwidth]{pdqn_paper_comparison_platform.pdf}
		\caption{Platform}
		\label{fig:learning_curves_platform}
	\end{subfigure}
	\vspace*{0.15cm}
	\begin{subfigure}[t]{\linewidth}
		\centering
		\includegraphics[width=0.95\textwidth]{pdqn_paper_comparison_goal.pdf}
		\caption{Robot Soccer Goal}
		\label{fig:learning_curves_robot_soccer_goal}
	\end{subfigure}
	\vspace*{0.15cm}
	\begin{subfigure}[t]{\linewidth}
		\centering
		\includegraphics[width=0.95\textwidth]{pdqn_paper_comparison_soccer.pdf}
		\caption{Half Field Offense}
		\label{fig:learning_curves_hfo}
	\end{subfigure}
\caption{Learning curves on Platform (\subref{fig:learning_curves_platform}), Robot Soccer Goal (\subref{fig:learning_curves_robot_soccer_goal}), and Half Field Offense (\subref{fig:learning_curves_hfo}). The running average scores---episodic return for Platform and HFO, and goal scoring probability for Robot Soccer Goal---are smoothed over  episodes and include random exploration, higher is better. Shaded areas represent standard error of the running averages over the different agents. The oscillating behaviour of \QPAMDP* on Platform is a result of re-exploration between the alternating optimisation steps. \MPDQN* clearly performs best overall. }
	\label{fig:learning_curves}
\end{figure}

Training consisted of  episodes, using the same hyperparameters for \QPAMDP* as \citet{masson2016} except we set  and . \PDQN* uses a single hidden layer , with , , , , and . Two hidden layers  are used for \PADDPG*, with , , , , and . Both algorithms use a replay memory size of , , gradients clipping at , and the same action-parameter policy initialisation as \QPAMDP* with additive Ornstein-Uhlenbeck noise.

\begin{table*}[ht!]
	\centering
	\begin{tabular}{@{}
			l
			S[table-format=1.3(4),detect-weight,mode=text]
			S[table-format=1.3(4),detect-weight,mode=text]
			S[table-format=1.3(4),detect-weight,mode=text]
			S[table-format=3(2),detect-weight,mode=text]
			@{}}
		\toprule
		& \multicolumn{1}{c}{{\textbf{Platform}}} & \multicolumn{1}{c}{\textbf{Robot Soccer Goal}} & \multicolumn{2}{c}{\textbf{Half Field Offense}} \\& {\textbf{Return}} & {\textbf{P(Goal)}} & {\textbf{P(Goal)}} & {\textbf{Avg. Steps to Goal}} \\ \midrule
		\QPAMDP* & 0.789 \pm 0.188 & 0.452 \pm 0.093 &{} & {n/a}  \\
		\PADDPG* & 0.284 \pm 0.061 & 0.006 \pm 0.020 & 0.875\pm 0.182 & \B 95 \pm 7 \\
		\PDQN*   & 0.964 \pm 0.068 & 0.701 \pm 0.078 & 0.883\pm 0.085 & 111 \pm 11 \\
		\SPDQN* & 0.941 \pm 0.164 & 0.752 \pm 0.131 & 0.718 \pm 0.131 & 99 \pm 7 \\
		\MPDQN*  & \B 0.987 \pm 0.039 & \B 0.789 \pm 0.070 & \B 0.913\pm 0.070 & 99 \pm 12 \\ 
		\cdashlinelr{1-5}
		\PADDPG*\footnotemark[2] &{-}&{-}& 0.923 \pm 0.073 & 112 \pm 5 \\
		Async. \PDQN*\footnotemark[3] &{-}&{-}& 0.989 \pm 0.006 & 81 \pm 3 \\ \bottomrule 
	\end{tabular}
	\caption[Table]{Mean evaluation scores over  random runs for each algorithm, averaged over  episodes after training with no random exploration. We include previously published results from \citet{hausknecht2016} and \citet{xiong2018} on HFO, although they are not directly comparable with ours as we use a longer training period and have a much larger sample size of agents--- versus  and  respectively---and asynchronous \PDQN* uses  parallel workers to implement -step returns rather than the mixing strategy we use.}
	\label{tab:results_comparison}
\end{table*}

\subsection{Half Field Offense}
The third and final domain, Half Field Offense (HFO) \citep{hausknecht2016}, is also the most complex. It has  state features and three parameterised actions available: dash, turn, and kick. Unlike Robot Soccer Goal, the agent must first learn to approach the ball and then kick it into the goals, although there is no keeper in this task. 

We use  episodes for training on HFO. This is more than the  episodes (or roughly  million transitions) used by \citet{hausknecht2016} and \citet{xiong2018} so that ample opportunity is given for the algorithms to converge in order to fairly evaluate the final policy performance. We use the same network structure as previous works with hidden layers of  neurons for \PDQN* and  neurons for \PADDPG*. The leaky ReLU activation function with negative slope  is used on HFO because of these deeper networks. \citet{xiong2018} use  asynchronous parallel workers for -step returns on HFO. For fair comparison and due to the lack of sufficient hardware, we instead use mixed -step return targets \cite{hausknecht2016betamixing} with a mixing ratio of  for both \PDQN* and \PADDPG*, as this technique does not require multiple workers. The  value was selected after a search over . We otherwise use the same hyperparameters as \citet{hausknecht2016betamixing} apart from the network learning rates: ,  for \PDQN* and ,  for \PADDPG*. In the absence of an initial action-parameter policy, we use the same -greedy with uniform random action-parameter exploration strategy as the original authors. In general we kept as many factors consistent between the two algorithms as possible for a fair comparison.

We select  of the most relevant state features for \QPAMDP* to avoid intractable Fourier basis calculations. These features include: player orientation, stamina, proximity to ball, ball angle, ball-kickable, goal centre position, and goal centre proximity. Even with this reduced selection, we found at most a Fourier basis of order  could be used. We use an adaptive step-size \citep{dabney2012} for Sarsa() with an eNAC learning rate of . The Q-PAMDP agent initially learns with Sarsa() for a period of  episodes before alternating between  eNAC updates of  rollouts each, and  episodes of discrete action re-exploration.
\footnotetext[2]{Average over  runs \citep{hausknecht2016}.}
\footnotetext[3]{Average over  runs with  workers \citep{xiong2018}.}

\section{Results}

The resulting learning curves of \MPDQN*, \SPDQN*, \PDQN*, \PADDPG*, and \QPAMDP* on the three parameterised action benchmark domains are shown in Figure~\ref{fig:learning_curves}, with mean evaluation scores detailed in Table~\ref{tab:results_comparison}.  

Our results show that \MPDQN* learns significantly faster than baseline \PDQN* with joint action-parameter inputs and achieves the highest mean evaluation scores across all three domains. \SPDQN* similarly shows better performance than \PDQN* on Platform and Robot Soccer Goal but to a slightly lesser extent than \MPDQN*. Notably, \SPDQN* exhibits fast initial learning on HFO but plateaus at a lower performance level than \PDQN*. This is likely due to the aforementioned lack of a shared feature representation between the separate Q-networks and the duplicate network parameters which require more updates to optimise.

In general, we observe that \PDQN* and its variants outperform \QPAMDP* on Platform and Robot Soccer Goal, while \PADDPG* consistently converges prematurely to suboptimal policies. \citet{wei2018} observe similar behaviour for \PADDPG* on Platform. This highlights the problem with updating the action and action-parameter policies simultaneously and was also observed when using eNAC for direct policy search on Platform \citep{masson2016}. On HFO, \QPAMDP* fails to learn to score any goals---likely due to its reduced feature space and use of linear function approximation rather than neural networks. Unexpectedly, baseline \PDQN* appears to learn slower than \PADDPG* on HFO. This suggests that the dueling networks and asynchronous parallel workers used by \citet{xiong2018} were major factors improving \PDQN* in their comparisons. 







\section{Related Work}

Many recent deep RL approaches follow the strategy of collapsing the parameterised action space into a continuous one. \citet{hussein2018} present a deep imitation learning approach for scoring goals on HFO using long-short-term-memory networks with a joint action and action-parameter policy. \citet{agarwa2018} introduces skills for multi-goal parameterised action space environments to achieve multiple related goals; they demonstrate success on robotic manipulation tasks by combining \PADDPG* with hindsight experience replay and their skill library.

One can alternatively view parameterised actions as a 2-level hierarchy: \citet{klimek2017} use this approach to learn a reach-and-grip task using a single network to represent a distribution over macro (discrete) actions and their lower-level action-parameters. The work most relevant to this paper is by \citet{wei2018}, who introduce a parameterised action version of TRPO (PATRPO). They also take a hierarchical approach but instead condition the action-parameter policy on the discrete action chosen to avoid predicting all action-parameters at once. While their preliminary results show the method achieves good performance on Platform, we omit comparison with PATRPO as it fails to learn to
score goals on HFO.

\section{Conclusion}
We identified a significant problem with the \PDQN* algorithm for parametrised action spaces: the dependence of its Q-values on all action-parameters causes false gradients and can lead to suboptimal action selection. We introduced a new algorithm, \MPDQN*, with separate action-parameter inputs which demonstrated superior performance over \PDQN* and former state-of-the-art techniques \QPAMDP* and \PADDPG*. We also found that \PADDPG* was unstable and converged to suboptimal policies on some domains. Our results suggest that future approaches should leverage the disjoint nature of parameterised action spaces and avoid simultaneous optimisation of the policies for discrete actions and continuous action-parameters.

\section*{Acknowledgments}
This work is based on the research supported in part by the National Research Foundation of South Africa (Grant Number: 113737).

\bibliographystyle{named}
\bibliography{references}

\end{document}
