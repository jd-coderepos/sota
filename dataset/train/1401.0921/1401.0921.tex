

\documentclass{elsartNoFoot}
\usepackage{amssymb}
\usepackage[pdftex]{color}






\setlength{\unitlength}{1cm}		



\definecolor{red  }{rgb}{0.8,0.0,0.0}
\definecolor{green}{rgb}{0.0,0.8,0.0}

\newcommand{\1}{\color{red}}
\newcommand{\2}{\color{green}}




\renewcommand{\leq}{\leqslant}		\renewcommand{\geq}{\geqslant}		

\newcommand{\set}[1]{\{#1\}}		

\newcommand{\X}[1]{x_{#1}}		\renewcommand{\S}[1]{s_{#1}}		

\newcommand{\+}[3]{{\renewcommand{\i}{{#1}}{#3},\ldots,\renewcommand{\i}{{#2}}{#3}}}

\newcommand{\ti}{{\tt i}}		\newcommand{\tk}{{\tt k}}		

\renewcommand{\exp}[1]{\gcd(N,#1)}	









\newcommand{\rf}[2]{\stackrel{(\ref{#1})}{#2}}



\newcommand{\SUM}[1]{\Sigma_{#1}}





\begin{document}



\begin{frontmatter}

\title{Maintaining partial sums in logarithmic time}
\author{Jochen Burghardt}
\address{Berlin}
\begin{keyword}
	Partial sums;
	Data structures;
	Algorithms
\end{keyword}

\begin{abstract}
        We present a data structure that allows to maintain in
        logarithmic time all partial sums of elements of a linear array
        during incremental changes of element's values. 
\end{abstract}

\end{frontmatter}



\section{Motivation}
\label{Motivation}



Assume you have a linear array  of numbers
which are frequently updated, and you need to maintain all partial
sums , where .
We present a data structure that allows to access each  and to
compute any partial sum in time .



As an application, think of the 
as integer numbers indicating the probabilities of certain
events; 
by chosing a uniformly distributed random number  in the range

and selecting the unique 

with
,
event  is selected with probability
.

If the probability distribution of events changes frequently,
the partial sums need to be recomputed every time,
which takes time  using the naive algorithm.


\begin{figure}
\begin{center}
\begin{picture}(8.3,4.5)
\put(0.242,1.210){\makebox(0.000,0.000){}}
\put(0.847,1.210){\makebox(0.000,0.000){}}
\put(1.331,1.210){\makebox(0.000,0.000){}}
\put(1.815,1.210){\makebox(0.000,0.000){}}
\put(2.299,1.210){\makebox(0.000,0.000){}}
\put(2.783,1.210){\makebox(0.000,0.000){}}
\put(3.267,1.210){\makebox(0.000,0.000){}}
\put(3.751,1.210){\makebox(0.000,0.000){}}
\put(4.235,1.210){\makebox(0.000,0.000){}}
\put(4.719,1.210){\makebox(0.000,0.000){}}
\put(5.203,1.210){\makebox(0.000,0.000){}}
\put(5.687,1.210){\makebox(0.000,0.000){}}
\put(6.171,1.210){\makebox(0.000,0.000){}}
\put(6.655,1.210){\makebox(0.000,0.000){}}
\put(7.139,1.210){\makebox(0.000,0.000){}}
\put(7.623,1.210){\makebox(0.000,0.000){}}
\put(8.107,1.210){\makebox(0.000,0.000){}}
\put(0.242,0.726){\makebox(0.000,0.000){}}
\put(0.847,0.726){\makebox(0.000,0.000){}}
\put(1.331,0.726){\makebox(0.000,0.000){}}
\put(1.815,0.726){\makebox(0.000,0.000){}}
\put(2.299,0.726){\makebox(0.000,0.000){}}
\put(2.783,0.726){\makebox(0.000,0.000){}}
\put(3.267,0.726){\makebox(0.000,0.000){}}
\put(3.751,0.726){\makebox(0.000,0.000){}}
\put(4.235,0.726){\makebox(0.000,0.000){}}
\put(4.719,0.726){\makebox(0.000,0.000){}}
\put(5.203,0.726){\makebox(0.000,0.000){}}
\put(5.687,0.726){\makebox(0.000,0.000){}}
\put(6.171,0.726){\makebox(0.000,0.000){}}
\put(6.655,0.726){\makebox(0.000,0.000){}}
\put(7.139,0.726){\makebox(0.000,0.000){}}
\put(7.623,0.726){\makebox(0.000,0.000){}}
\put(8.107,0.726){\makebox(0.000,0.000){}}
\put(0.242,0.242){\makebox(0.000,0.000){}}
\put(0.847,0.242){\makebox(0.000,0.000){}}
\put(1.331,0.242){\makebox(0.000,0.000){}}
\put(1.815,0.242){\makebox(0.000,0.000){}}
\put(2.299,0.242){\makebox(0.000,0.000){}}
\put(2.783,0.242){\makebox(0.000,0.000){}}
\put(3.267,0.242){\makebox(0.000,0.000){}}
\put(3.751,0.242){\makebox(0.000,0.000){}}
\put(4.235,0.242){\makebox(0.000,0.000){}}
\put(4.719,0.242){\makebox(0.000,0.000){}}
\put(5.203,0.242){\makebox(0.000,0.000){}}
\put(5.687,0.242){\makebox(0.000,0.000){}}
\put(6.171,0.242){\makebox(0.000,0.000){}}
\put(6.655,0.242){\makebox(0.000,0.000){}}
\put(7.139,0.242){\makebox(0.000,0.000){}}
\put(7.623,0.242){\makebox(0.000,0.000){}}
\put(8.107,0.242){\makebox(0.000,0.000){}}
\put(8.107,1.573){\circle*{0.182}}
\put(7.139,1.573){\circle*{0.182}}
\put(6.171,1.573){\circle*{0.182}}
\put(5.203,1.573){\circle*{0.182}}
\put(4.235,1.573){\circle*{0.182}}
\put(3.267,1.573){\circle*{0.182}}
\put(2.299,1.573){\circle*{0.182}}
\put(1.331,1.573){\circle*{0.182}}
\put(7.623,2.178){\circle*{0.182}}
\put(5.687,2.178){\circle*{0.182}}
\put(3.751,2.178){\circle*{0.182}}
\put(1.815,2.178){\circle*{0.182}}
\put(6.655,2.783){\circle*{0.182}}
\put(2.783,2.783){\circle*{0.182}}
\put(4.719,3.388){\circle*{0.182}}
\put(0.847,4.174){\circle*{0.182}}
\put(8.107,1.573){\line(-4,5){0.484}}
\put(7.139,1.573){\line(-4,5){0.484}}
\put(6.171,1.573){\line(-4,5){0.484}}
\put(5.203,1.573){\line(-4,5){0.484}}
\put(4.235,1.573){\line(-4,5){0.484}}
\put(3.267,1.573){\line(-4,5){0.484}}
\put(2.299,1.573){\line(-4,5){0.484}}
\put(1.331,1.573){\line(-4,5){0.484}}
\put(7.623,1.573){\line(0,1){0.605}}
\put(6.655,1.573){\line(0,1){1.210}}
\put(5.687,1.573){\line(0,1){0.605}}
\put(4.719,1.573){\line(0,1){1.815}}
\put(3.751,1.573){\line(0,1){0.605}}
\put(2.783,1.573){\line(0,1){1.210}}
\put(1.815,1.573){\line(0,1){0.605}}
\put(0.847,1.573){\line(0,1){2.662}}
\put(7.623,2.178){\line(-5,3){0.967}}
\put(5.687,2.178){\line(-5,3){0.967}}
\put(3.751,2.178){\line(-5,3){0.967}}
\put(1.815,2.178){\line(-5,3){0.967}}
\put(6.655,2.783){\line(-3,1){1.936}}
\put(2.783,2.783){\line(-3,1){1.936}}
\put(4.719,3.388){\line(-5,1){3.920}}
\end{picture}
\end{center}
\caption{Data structure}
\label{tree}
\end{figure}








\section{Data structure and access algorithms}
\label{Data structure and access algorithms}



Our solution is to store a mix of individual values  and partial
sums in the array, thus realizing a binary tree where each node
represents the sum of all leafs below it.
Figure~\ref{tree}
sketches an example for , the partial sums
corresponding to the nodes
indicated by solid circles are stored as .

In some respect, this idea is similar to that of heap sort
\cite[Sect.~3.4]{Aho.Hopcroft.Ullman.1974}, which also uses a mix of
representations (sorted along a path and unsorted within a level) to
combine the advantages of both. Our data structure combines the
advantages of storing single values (easily updatable) and sums (no need
to recompute them). 



Formally, let  be a power of 2;
let an array  of size  be given.
Instead of this original array, 
we maintain the array ,
where



Here,  is 
the greatest common divisor of  and , i.e.,
the largest power of 2 dividing .
It corresponds to the least 1 bit in the 2--complement
representation of , which can be computed as bitwise {\tt and}
of  and .



\begin{figure}
\begin{center}
\begin{verbatim}
int s[M];

#define S(i)              (i<M ? s[i] : 0)

#define gcdN(k)           ((N+k) & (N-k))

int sumN(int k) {
    int i, sm = 0;
    for (i=k; i<M; i+=gcdN(i))
        sm += s[i];
    return sm;
}

#define sum(j,k)          (sumN(j) - sumN(k+1))

int get(int k) {
    int i, x = s[k];
    for (i=1; i<gcdN(k) && k+i<M; i*=2)
        x -= s[k+i];
    return x;
}

void inc(int k,x) {
    int i;
    for (i=k; i>=0; i-=gcdN(i))
        s[i] += x;
}

#define set(k,x)          inc(k,x-get(k))

int find(int x) {
    int i, k = 0, pv = s[N/2];
    for (i=N/2; i>0; i/=2)
        if (x < pv) {
            pv += S(k+i*3/2) - s[k+i];
            k  += i;
        } else {
            pv += S(k+i/2);
        }
    return k;
}
\end{verbatim}
\caption{Algorithms}
\label{Algorithms}
\end{center}
\end{figure}



\begin{figure}\begin{center}{\small\tt \begin{tabular}[t]{@{}*{12}{|r}|@{}}\multicolumn{6}{@{}l@{}}{\rule{0cm}{0.5cm}{\tt sumN(3)}}     
	& \multicolumn{1}{l}{} &
	\multicolumn{5}{@{}l@{}}{{\tt get(12)}}       \\
\cline{1-6}
	\cline{8-12}
i & & 3 & 4 & 8 & 16    &&
	i & & 1 & 2 & 4 \\
\cline{1-6}
	\cline{8-12}
sm & 0 & 3 & 20 & 71 &	&&
	x & 17 & 15 & 6 &   \\
\cline{1-6}
	\cline{8-12}
\multicolumn{4}{@{}l@{}}{\rule{0cm}{0.5cm}{\tt inc(12,...)}}
	& \multicolumn{1}{l}{} &
	\multicolumn{7}{@{}l@{}}{\rule{0cm}{0.5cm}{\tt find(69)}}     \\
\cline{1-4}
	\cline{6-12}
i & 12 & ~8 & 0        &&
	i   &    & 8  & 4  & 2  & 1  & ~0	\\
\cline{1-4}
	\cline{6-12}
\multicolumn{4}{@{}l@{}}{\rule{0cm}{0.5cm}{\tt inc(3,...)}}
	&&
	pv  & 51 & 68 & 77 & 71 & 71 &	\\
\cline{1-4}
	\cline{6-12}
i & 3 & 2 & 0	&&
	k   & 0  &    &    & 2  & 3  &	\\
\cline{1-4}
	\cline{6-12}
\end{tabular}}\end{center}\caption{Sample runs}\label{Sample runs}\end{figure}











The following algorithms, given in C code
in Fig.~\ref{Algorithms}, maintain our data structure.



\begin{itemize}
\item \verb|int  sumN(int k) |
	returns ;
\item \verb|int  sum(int j,k)|
	returns ;
\item \verb|int  get(int k)  |
	retrieves ;
\item \verb|void inc(int k,x)|
	adds {\tt x} to ;
\item \verb|void set(int k,x)|
	assigns {\tt x} to ; and
\item \verb|int  find(int x) |
	returns some  such that
	,
	\\
	provided 
	;
	~
	 is unique if no  is negative.
\end{itemize}

Figure \ref{Sample runs} shows some sample runs on the data in
Fig.~\ref{tree}.


In order to deal with arrays whose size is not a
power of 2, assume  for all , where
. At two places it is neccessary to test
the index boundary explicitely, using the function {\tt int S(int i)}.

The algorithms can immediately be generalized to deal with arbitrary
(non--abelian) group elements instead of integers; if {\tt find} is to
be used, ordered groups are neccessary.








\section{Complexity}
\label{Complexity}


All algorithms take  time due to the implicit tree
structure.
For {\tt sumN} and {\tt inc}, note that the value of 
grows in every loop cycle, since



In the following sections \ref{Correctness of get} to
\ref{Correctness of find}, we give correctness proofs of the main
algorithms in the Hoare calculus \cite{Hoare.1969}.






\section{Correctness of {\tt get}}
\label{Correctness of get}

To see the correctness of {\tt get}, show



by induction on ;
note that commutativity of  is not required for the proof.

If , we have  for ,
and therefor 



We define the abbreviation
 .

By equation (\ref{recurs}), we obtain 
, 
justifying the step in lines 4.--5.

We have  if  or
; this justifies lines 13.--14.

We can now apply the Hoare calculus to the code of
\verb|int get(int k)|:

\begin{enumerate}
\setlength{\itemsep}{-0.0cm}
\renewcommand{\labelenumi}{\arabic{enumi}.}
 \item{\1\verb|int get(int k) {					|
}\item{\1\verb|    int i, x;					|
}\item{\1\verb|    x = s[k];					|
}\item{\1\verb|    i = 1;					|
}\item{\2\verb|    |
}\item{\1\verb|    while (i < gcdN(k) && k+i < M) {		|
}\item{\2\verb|        |
}\item{\1\verb|        x = x - s[k+i];				|
}\item{\2\verb|        |
}\item{\1\verb|        i = i * 2;				|
}\item{\2\verb|        |
}\item{\1\verb|    }						|
}\item{\2\verb|    |
}\item{\1\verb|    return x;					|
}\item{\1\verb|}						|
}
\end{enumerate}





\section{Correctness of {\tt inc}}
\label{Correctness of inc}

Next, we show that {\tt inc} makes sufficiently many updates.
By (\ref{dataInv}),
 depends on , 
iff
.

Hence, if  depends on ,
then so does , since



But no  for  depends on :
\begin{quote}
Let  and  
for odd numbers .
\\
Then  since .
\\
And 
implies .
\\
Hence,

\end{quote}






\section{Correctness of {\tt sumN}}
\label{Correctness of sumN}

The loop in {\tt sumN} satisfies the invariant
,
since



This justifies the step in lines 7.--9.
For lines 13.--14.\ note that  for .

\begin{enumerate}
\setlength{\itemsep}{-0.0cm}
\renewcommand{\labelenumi}{\arabic{enumi}.}
 \item{\1\verb|int sumN(int k) {				|
}\item{\1\verb|    int i, sm;					|
}\item{\1\verb|    sm = 0;					|
}\item{\1\verb|    i  = k;					|
}\item{\2\verb|    |
}\item{\1\verb|    while (i < M) {				|
}\item{\2\verb|        |
}\item{\1\verb|        sm = sm + s[i];				|
}\item{\2\verb|        |
}\item{\1\verb|        i  = i  + gcdN(i);			|
}\item{\2\verb|        |
}\item{\1\verb|    }						|
}\item{\2\verb|    |
}\item{\1\verb|    return sm;					|
}\item{\1\verb|}						|
}
\end{enumerate}







\section{Correctness of {\tt find}}
\label{Correctness of find}

The loop in {\tt find} satisfies the invariant



To show this, note that for , we have



and similarly



hence, we get



in case of  and ,
respectively.

We transform the program to make the Hoare verification rules
applicable and unfold the last loop cycle () to avoid
confusing case distinctions.
We omit the computation of the pivot element
{\tt pv} in the last cycle, since its value
isn't used any more.

We define the abbreviations
~

~
and
~


Observe that 

implies both  and ;
~
this is used in lines 13.--15.\ and 21.--23., respectively.

Equations (\ref{pvHoare}) justify the steps in lines 13.--15.\ and
19.--21.;
equation (\ref{dataInv}) justifies step 7.--9.


\begin{enumerate}
\setlength{\itemsep}{-0.0cm}
\renewcommand{\labelenumi}{\arabic{enumi}.}
 \item{\2\verb||	
}\item{\1\verb|int find(int x) {				|
}\item{\1\verb|    int i, k, pv;				|
}\item{\2\verb|    |	
}\item{\1\verb|    k  = 0;					|
}\item{\1\verb|    i  = N/2;					|
}\item{\2\verb|    |	
}\item{\1\verb|    pv = s[N/2];				|
}\item{\2\verb|    |	
}\item{\1\verb|    while (i >= 2) {				|
}\item{\2\verb|        |	
}\item{\1\verb|        if (x < pv) {				|
}\item{\2\verb|            |	
}\item{\1\verb|            pv = pv + S(k+i*3/2) - s[k+i];	|
}\item{\2\verb|            |	
}\item{\1\verb|            k  = k + i;			|
}\item{\2\verb|            |	
}\item{\1\verb|        } else {				|
}\item{\2\verb|            |	
}\item{\1\verb|            pv = pv + S(k+i/2);		|
}\item{\2\verb|            |	
}\item{\1\verb|        }					|
}\item{\2\verb|        |	
}\item{\1\verb|        i = i/2;				|
}\item{\2\verb|        |	
}\item{\1\verb|    }						|
}\item{\2\verb|    |	
}\item{\1\verb|    if (x < pv) {				|
}\item{\2\verb|        |	
}\item{\1\verb|        k  = k + 1;				|
}\item{\2\verb|        |	
}\item{\1\verb|    } else {					|
}\item{\2\verb|        |	
}\item{\1\verb|    }						|
}\item{\2\verb|    |	
}\item{\1\verb|    return k;					|
}\item{\1\verb|}						|
}
\end{enumerate}



This completes the verification proofs of the algorithms given in
Fig.~\ref{Algorithms}.
\\
A short version of this paper (without proofs) was published in
\cite{Burghardt.2001}.

\bibliographystyle{alpha}
\bibliography{lit}
















\end{document}
