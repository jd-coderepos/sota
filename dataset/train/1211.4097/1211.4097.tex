In this section we will prove that the non-deterministic reduction enjoys a standardization property.
As we recalled already in the introduction, the standardization property is based on an order relation between redexes.
We can define it formally as follows:
\begin{definition}
Let $\prec$ be an order on positions in terms (which is extended to an order on
subterms of a given term). Suppose $\rho$ is a reduction chain, and let
$M_i$ and $R_i$ be the $i$-th term and fired redex in $\rho$ respectively.
We say that $\rho$ is \emph{$\prec$-standard} if for every $i$ we have that
$R_{i+1}$ is not the residual of a redex $R'$ in $M_i$ such that $R'\prec R_i$.
\end{definition}

We will prove that non deterministic reduction in $\Lamr$ enjoys the standardization property
with respect to the order $\prec_r$, which is the partial order on positions in $\Lambda^r$ terms that, intuitively,
gives precedence to linear positions over non-linear ones, and then orders
linear positions left-to-right, with the proviso that positions inside the same bag
be not comparable. The formal definition follows.

\begin{definition}[Linear left-to-right order]\label{def:order}
For two subterms $S_1$ and $S_2$ inside the expression $\Sum A$, we say that
$S_1\prec_r S_2$ in $\Sum A$ if and only if any of the following happens:
\begin{itemize}
 \item $S_2$ is a subterm of $S_1$;
 \item $S_1$ is linear in $\Sum A$ while $S_2$ is not;
 \item $S_1$ and $S_2$ are both linear in $\Sum A$, $\Sum A=MP$, $S_1$ is in $M$ and $S_2$ is in $P$.
 \item $S_1$ and $S_2$ are subterms of the same proper subexpression $\Sum B$ of $\Sum A$, and
 $S_1\prec_r S_2$ in $\Sum B$;
\end{itemize}
\end{definition}
\begin{example}
 $S_1\prec_r S_2$ in both $\lambda x.x\bag{\bang S_2}\bag{S_1}$ and
 $\lambda x.x\bag{S_1}\bag{S_2}$,
 while they are incomparable in $\lambda x.x\bag{S_1,S_2}$.
\end{example}

Our starting point is the division of redexes in two classes, outer and inner.

\begin{definition}[\cite{PaganiTranquilli09}]\label{def:outerreduction}
Let $\epsilon\in\{\red{b}, \red{g}, \red{nd}\}$. The \textdef{outer $\epsilon$-reduction} $\outbeta[\epsilon]$ is the 
 \emph{linear} context closure of the $\epsilon$-steps given in
Definitions~\ref{def:giantbaby}. 
A \textdef{non-outer $\epsilon$-reduction}, called \textdef{inner} is
defnoted by $\nonoutbeta[\epsilon]$.
\end{definition}
In other words, an outer reduction does not reduce inside reusable resources, so
an outer redex (\emph{i.e.}\ a redex for $\outbeta[\epsilon]$) is a redex not under the scope of a 
$\bang{(\cdot)}$ constructor. In particular a term corresponding to a 
$\lambda$-term has at most one outer-redex, which coincides with the head-redex.
Pagani and Tranquilli stated in some sense a weak standardization property for the giant reduction,
proving that inner redexes can always be postponed. Their result
can easily be extended to other reductions, in particular to the non-deterministic one.

\begin{theorem}[\cite{PaganiTranquilli09}]\label{the:patra}
Let $\epsilon\in\{\red{b}, \red{g}, \red{nd}\}$. $M \redto[\epsilon^*]\Sum A $ implies $M \redto[o\epsilon^{*}]\Sum A' $ and $\Sum A' \redto[i\epsilon^{*}]\Sum A$.
\end{theorem}

We introduce now a further classification between outer redexes.
\begin{definition}
The set of \emph{leftmost} redexes $\L(M)$ of a term $M$ or a bag $P$ are defined
  inductively by:
  $$\begin{aligned}
    \L(x) &\ass  \emptyset,\\ \L(\lambda x.M) &\ass\L(M)
  \end{aligned}
  \;\;
    \L(MP) \ass
    \begin{cases}
      \{MP\} & \text{if $M=\lambda x.M'$,}\\
      \L(M) & \text{otherwise, if $\L(M)\neq \emptyset$}\\
      \L(P) & \text{otherwise}
    \end{cases}
    \;\;
    \begin{aligned}
      \L(1) &\ass \emptyset,\\
      \L([\bang M]\cdot P &\ass \L(P),\\
      \L([M]\cdot P) &\ass \L(M) \cup \L(P)
    \end{aligned}
  $$
\end{definition}

In regular $\lambda$-calculus, the set $\L(M)$ is at most a singleton, and
$\prec_r$-standardness collapses to the regular notion of left-to-right order of redexes.

\begin{fact}
  Redexes in $\L(M)$ are exactly the $\prec_r$-minimal elements among
  all redexes of $M$.
\end{fact}

In the following, we will consider in particular the non-deterministic reduction. So, let us introduce some notation.

\begin{notation}
Let $M \redto[ndo] N$.
$M\redto[lm]N$ denotes that the reduction fires a
redex in $\L(M)$, while we write $M\xredto[\neg lm]N$ if the redex
is not a leftmost one. Moreover $M\redto[o]N$ and $M\redto[i]N$ will be short for 
for $M\redto[ndo]N$ and $M\redto[ndi]N$ respectively.

\end{notation}

\begin{lemma}\label{lem:rightpushing}
We have the following facts on non-leftmost reduction.
\begin{itemize}
\item $\rho:\lambda x.M \xredto[\neg lm*] N$ if and only if $N = \lambda x.M'$
  and $\rho':M \xredto[\neg lm*] M'$ with $|\rho| = |\rho'|$;
\item $\rho:MP \xredto[\neg lm*] N$ if and only if $N = M'P'$, $\rho':M
  \xredto[\neg lm*] M'$ and $\rho'':P \xredto[o*] P'$ with $|\rho| =
  |\rho'| + |\rho''|$;
\item $\rho: [M]\cdot P \xredto[\neg lm*] Q$  if and only if$Q = [M']\cdot P'$,
  $\rho':M \xredto[\neg lm*] M'$ and $\rho'':P \xredto[\neg lm*] P'$
  with $|\rho| = |\rho'| + |\rho''|$;
\item $\rho: [\bang M]\cdot P \xredto[\neg lm*] Q$  if and only if$Q = [\bang
  M]\cdot P'$ and $\rho'':P \xredto[\neg lm*] P'$ with $|\rho| =
  |\rho''|$.
\end{itemize}
\end{lemma}
The proof of standardization is based on an inversion property between outer redexes,
saying that a not-leftmost reduction followed by a leftmost one can always be replaced by a leftmost followed by 
an outer. This is the upcoming \autoref{lem:inversion}. In order to get it we first prove the following intermediate properties.

\begin{lemma} \label{lem:triangolo}
If $O \redto[o] O'$ then
$\forall L' \in O'\gen{Q}{x}\expo{0}{x}
\exists L \in O\gen{Q}{x}\expo{0}{x}$
such that $L \redto[o] L'$.
\end{lemma}
\begin{proof}
We will prove that $\forall L' \in O'\gen{Q}{x}
\exists L \in O\gen{Q}{x}$
such that $L \redto[o] L'$. Then the statement of the lemma follows easily.
By induction on $O$. 
\begin{enumcases}
\item $O=x$ and $O=y$ are not possible.
\item $O=\lambda y.M$.
By inductive hypothesis.
\item $O=(\lambda y.M)P$.
There are three cases: $\lambda y.M  \redto[o]  \lambda y.M'$, $ P \redto[o] P'$,
$(\lambda y.M)P \redto[o] O' \in M\gen{P}{y}\expo{0}{y}$.
Let $M  \redto[o]  M'$.
$(\lambda y.M)P \gen{Q}{x}= \sum_{Q_{1},Q_{2}}((\lambda y.M) \gen{Q_{1}}{x})(P\gen{Q_{2}}{x})$,where
$Q_{1},Q_{2}$ range over all the possible decomposition of $Q$ into two parts, counting the reusable resources
with all the possible multiplicities. This means that in case $Q_{1},Q_{2}$ are considered two different subterms also
in case they are syntactically equal.
By inductive hypothesis, for all $L' \in (\lambda y.M') \gen{Q_{1}}{x}$ there is
$L \in (\lambda y.M) \gen{Q_{1}}{x}$ such that $L \redto[o] L'$, and the result follows by transitivity of $\in$.
The case $ P \redto[o] P'$ is similar.

Let $(\lambda y.M)P \redto[o] O' \in M\gen{P}{y}\expo{0}{y}$.
Then we have that the substitution $(\lambda y.M)P\gen{Q}{x}$ is equal to the sum $\sum_{Q_{1},Q_{2}}(\lambda y.M \gen{Q_{1}}{x}) (P\gen{Q_{2}}{x}) $, where
$Q_{1},Q_{2}$ range as before. Since each component of this sum is a redex (the substitutions do not modify the
external shape of the terms), we can reduce each redex, so obtaining that for all
$L\in (\lambda y.M \gen{Q_{1}}{x}) (P\gen{Q_{2}}{x})$, $L \redto[o] L' \in M  \gen{Q_{1}}{x}
\gen{P\gen{Q_{2}}{x}}{y} \expo{0}{y}$.
On the other side, $M\gen{P}{y}\expo{0}{y}\gen{Q}{x}$ is equal to the sum
$\sum_{Q_{1},Q_{2}}M \gen{Q_{1}}{x}
\gen{P\gen{Q_{2}}{x}}{y} \expo{0}{y} $, and the proof is done.



\item $O=MP$ and $O'= M'P$ or $O=MP$ and $O'= MP'$.
All by inductive hypothesis.\qedhere
\end{enumcases}


\end{proof}

\begin{lemma} \label{lem:quadrato}
If $Q \redto[o]  Q'$ then
$\forall L' \in O\gen{Q'}{x} \expo{0}{x}
\exists L \in O\gen{Q}{x} \expo{0}{x}$
such that $L \redto[o] L'$.
\end{lemma}

\begin{proof}
By induction on $Q$. $Q$ cannot be $1$ as it would be normal.

If $Q=\lbrack H \rbrack \cdot P$ then
$O \gen{Q}{x} \expo{0}{x} =
O \linear{H}{x} \gen{P}{x} \expo{0}{x}$.
We proceed by cases:
\begin{enumcases}
\item The reduction is on $P$, \emph{i.e.}\ $ [H] \cdot P \redto[o] [H] \cdot P' $.
For all $N \in O \linear{H}{x}$, $N\gen{P}{x} \redto[o]N\gen{P'}{x}$. Then we have by induction that
for all $L \in N\gen{P}{x} \expo{0}{x}$ there is $L' \in N \gen{P'}{x} \expo{0}{x}$ such that $L \redto[o] L'$.
So the result follows.

\item The reduction is on $H$, \emph{i.e.}\ $ [H] \cdot P \redto[o] [H'] \cdot P $).
Let us set $
O\linear{H}{x}\gen{P}{x}\expo{0}{x} =
(O_1 + ... + O_k) \gen{P}{x}\expo{0}{x}$,
where $H$ occurs in all $O_{j}$ ($1 \leq j \leq k$), since the substitution is linear.
Let $O_{j} \redto[og] O^{j}_{1}+...+O^{j}_{m}$ by reducing the occurrence of $H$ in it.
So $O_{j} \redto[o] O^{j}_{i}$ ($1 \leq i \leq m_{j}$), and, by \autoref{lem:triangolo},
for all $L' \in O^{j}_{i} \gen{P}{x} \expo{0}{x}$, there is $L\in O_{j}\gen{P}{x} \expo{0}{x}$
such that $L\redto[o] L' $. Since $O_{j}\in O\linear{H}{x}$ and $O^{j}_{i}\in O\linear{H'}{x}$, the proof follows.\
\end{enumcases}

If $Q= \lbrack H^! \rbrack \cdot P$
the reduction on $P$, and the case is similar to the first case of the previous point.\qedhere





\end{proof}

\begin{lemma}[Inversion] \label{lem:inversion}
$M \redto[\neg lm] M'  \redto[lm] N $ implies $M  \redto[lm] M'' \redto[ndo] N$, for some $M''$.
\end{lemma}
\begin{proof}
We proceed by induction on $M$.\\
Let $M = \lambda \vec{y} .(\lambda x.O)Q P_1...P_j...P_n$, so ${\cal L}(M)=\{(\lambda x.O)Q\}$.
Non leftmost reductions on $M$ can be done in $O$, in $Q$ or in $P_j$ ($1 \leq j \leq n$). We procede by cases:
\begin{enumcases}
\item The reduction is on $P_j (1 \leq j \leq n)$.].
Let $P_j \redto[g] P'_j + \mathbb{S}$.
We have that
$$M \redto[\neg lm] \lambda \vec{y}. (\lambda x.O) Q P_1 ... P'_j ... P_n \redto[g]
\lambda \vec{y}.O \gen{Q}{x}\expo{0}{x} P_1 ... P'_j ... P_n.$$
Moreover, by reducing the leftmost redex
$$M \redto[g] \lambda \vec{y}.O \gen{Q}{x} \expo{0}{x} P_1 ... P_j ... P_n =
\lambda \vec{y}.(O_1 + ... +O_k) P_1 ... P_j ... P_n,
$$
so that
$$M \redto[lm] \lambda \vec{y}.O_h P_1 ... P_j ... P_n \redto[o]
\lambda \vec{y}.O_h P_1 ... P'_j ... P_n$$
for all $1\leq h \leq k$.

\item The reduction is on $Q$.
Let $Q \redto[o] Q'$ and let $M \redto[\neg lm] \lambda \vec{y}.(\lambda x.O)Q' P_1 ... P_n  \redto[lm]
\lambda \vec{y}. \overline{M} P_1 ... P_n$, where $\overline M$ is such that
$\overline{M} \in O\gen{Q'}{x} \expo{0}{x} $.
Moreover, by reducing the leftmost redex, we also have the reduction
$M \redto[g] \lambda \vec{y}. O\gen{Q}{x} \expo{0}{x} P_1 ... P_n $.
By \autoref{lem:quadrato}, $Q \redto[o]Q'$ implies that for all $L'\in O\gen{Q'}{x} \expo{0}{x}$,
there exists $
L \in O\gen{Q}{x} \expo{0}{x}$ such that $L \redto[o] L'$.
So there is $\overline{\overline{M}}\in O\gen{Q}{x}\expo{0}{x}$ such that
$M \redto[lm] \lambda \vec{y}. \overline{\overline{M}}  P_1 ... P_n \redto[o] \lambda \vec{y}. \overline{M} P_1 ... P_n.$

\item The reduction is in O.
Let $O \redto[o] O' $, and let
$M \redto[\neg lm] \lambda \vec{y}.(\lambda x.O')Q P_1...P_n \redto[lm] \lambda \vec{y}. \overline{M}P_1...P_n$,
where $\overline{M}$ is such that $\overline M \in O'\gen{Q}{x} \expo{0}{x}$.
Again if we reduce the leftmost redex, we have the reduction
$M \redto[g] \lambda \vec{y}. O \gen{Q}{x} \expo{0}{x} P_1 ... P_n$.
$O \redto[o]O'$ implies, by \autoref{lem:triangolo}, $\forall L' \in O'\gen{Q}{x} \expo{0}{x},
 \exists L \in O \gen{Q}{x} \expo{0}{x}$ such that
$L \redto[o] L'$. So there is $\overline{\overline{M}}\in O \gen{Q}{x} \expo{0}{x}$ such that we
can compose the reductions
$M \redto[lm] \lambda \vec{y}. \overline{\overline{M}}  P_1 ... P_n \redto[o] \lambda \vec{y}.\overline{M}
 P_1 ... P_n $.
\end{enumcases}

Let $M=\lambda \vec{y}.x P_1 ... P_n$, and let $P_{i} \redto[\neg lm]P'_{i}$ and $P_{j} \redto[lm]P'_{j}$.
In case $i \not = j$, the proof is trivial. In case $i=j$ the proof is by induction on $P_{i}$.\qedhere

\end{proof}
\begin{corollary}\label{cor:reordering}
  If $\rho:M\xredto[o*]M'$ then there are $\sigma:M\xredto[lm*]M''$
  and $\pi:M''\xredto[\neg lm*]M'$ with $|\sigma| +
  |\pi| = |\rho|$.
\end{corollary}


\begin{lemma}\label{lem:divide}\
\begin{enumerate}[(i)]
 \item Given $\rho : M \redto[lm*] N$ and $\sigma: N \xredto[o*] L$,
then $\rho\sigma: M\xredto[o*] L$ is
 $\prec_r$-standard if and only if $\sigma$ is.
 In particular every chain of leftmost reductions is $\prec_r$-standard.
 \item Given $\rho : M\xredto[o*] N$ and $\sigma: N \redto[i*] L$,
then $\rho\sigma: M \redto[nd*] L$ is
 $\prec_r$-standard if and only if both $\rho$ and $\sigma$ are.
\end{enumerate}
\end{lemma}
 \begin{proof}
 The result follows easily from the definition of $\prec_r$.
 \end{proof}
Now we can prove that the non-deterministic outer reduction is $\prec_{r}$-standard.
\begin{lemma}[Non-deterministic outer standard reduction]
  \label{lem:stdouter}
  If $M \xredto[o*] N$, then there is a $\prec_r$-standard
  non-deterministic outer reduction from $M$ to $N$.
\end{lemma}
\begin{proof}
   We reason by induction on the pair $(p,s)$, where $p=|\rho|$ is the
  length of the reduction sequence $\rho:M \xredto[o*] N$, and $s$ is
  the number of symbols in $M$.  By \autoref{cor:reordering},
  there is a reduction $\sigma_l:M \redto[lm*] M'$ and $\sigma_r: M'
  \xredto[\neg lm*] N$ with $|\sigma_l| + |\sigma_r| = |\rho| = p$.
  If $|\sigma_l|>0$ then inductive hypothesis applies to $\sigma_r$,
  giving $\prec_r$-standard $\sigma'_r: M' \xredto[o*] N$, which
  gives that $\sigma_l\sigma'_r: M\xredto[o*] N$ is
  $\prec_r$-standard by \autoref{lem:divide}.
  In case $\sigma_r: M \xredto[\neg lm*] N$ is the
  whole reduction, the proof is by cases on $M$.
 The only non-obvious case is when $M=LP$: by \autoref{lem:rightpushing} we have $N=L'P'$ and
    $\rho': L \xredto[\neg lm*] L'$ and $\rho'': P \xredto[o*]
    P'$. We can apply inductive hypothesis to both as $|\rho'| +
    |\rho''| = |\rho_r|$, and get $LP \xredto[\neg lm*] L'P
    \xredto[o*] L'P'$. Now assuming that this is not
    $\prec_r$-standard leads to a contradiction to the definition at
    the seam, since all linear positions in $L''$ are $\prec_r$ with respect to
    those in $P$.
\end{proof}
In order to prove that also inner reductions can be standardized, we need to introduce the notion of 
{\em outer shape} of a term.
\begin{definition}
  The \emph{outer shape} $\ell(M)\hole\cdot$ of a term $M$ is a context that
  is $M$ with holes replacing all exponential arguments of $M$'s
  bags.

  Formally, extending the definition to bags, we define
  $\ell(\,.\,)\hole\cdot$ inductively as follows.
  
  $\begin{array}{lll}
    \ell(x)\hole\cdot = x, &
    \ell(\lambda x.M) \hole\cdot= \lambda x.\ell(M)\hole\cdot,&
    \ell(MP)\hole\cdot = \ell(M)\hole\cdot\ell(P)\hole\cdot,\\
    \ell(1)\hole\cdot  = 1,&
    \ell(\bag M\cdot P)\hole\cdot = \bag{\ell(M)\hole\cdot}\cdot\ell(P)\hole\cdot,&
    \ell(\bag{\bang M}\cdot P)\hole\cdot= \bag{\bang{\hole\cdot }}\ell(P)\hole\cdot.
  \end{array}$
\end{definition}
\begin{property}\label{lem:inner_outer}\
\begin{enumerate}[(i)]
\item \label{lem:inner_shape} $M\xredto[i*]N$ if and only if $\ell(M)\hole\cdot=\ell(N)\hole\cdot$, and there are $k$
  terms $M'_i$ and $k$ terms $N'_i$ such that
  $M=\ell(M)_a\hole{\vec M'_i}$, $N=\ell(M)_a\hole{\vec N'_i}$ and
  $M'_i\xredto[nd*]N'_i$ for each $i$.
\item \label{lem:outer_shape_std}
  If $M=\ell(M)_{a}\hole{\vec M'_i}$ and
  $\rho_i:M'_i\xredto[nd*]M''_i$
  are standard, then there is a standard 
  $\rho':M\xredto[i*]\ell(M)_{a}\hole{\vec M''_i}$.
\end{enumerate}
\end{property}
\begin{proof}\
\begin{itemize}
\item[i)] The if direction is a direct consequence of how $\red{i}$ is
  defined and of context closedness of the reduction. We thus move to
  the only if direction.

  First, let us show that the property to prove is preserved by
  composition of reduction chains.

  Suppose
  $M\xredto[i*]N\xredto[i*]O$ with $M=\ell(M)[\vec M'_i]$,
  $N=\ell(M)_{a_1}[\vec N'_i]=\ell(M)_{a_2}[\vec N''_i]$
  and $O=\ell(M)_{a_3}[\vec O'_i]$. We can suppose
  $a_1=a_2$ by re-indexing (namely using
  $\ell(N)_{a_1}[\vec N''_{a_2^{-1}(a_1(i)}]$ and
  $\ell(O)_{a_3'}[\vec N''_{a_2^{-1}(a_1(i)}]$ with
  $a_3'=a_3\circ a_2^{-1}\circ a_1$). So we just
  forget the bijections employed, and then we have by hypothesis
  $M'_i\xredto[nd*]N'_i=N''_i\xredto[nd*]O'_i$, which is what is
  needed.

  Now, we can prove the property by reducing to the case of a single
  inner reduction, as composing multiple ones of them preserves the
  property.

  Take $M\redto[i]N$: the result follows by a straightforward
  induction on how the reduction is defined.

\item[ii)] The idea is that the reductions in the subterms can be freely
  rearrenged.

  Let us reason by generalizing to expressions and by structural
  induction on $\Sum A$.
  \begin{enumcases}
    \item $\Sum A=x$ or $\Sum A=1$: nothing to prove.
    \item $\Sum A=\lam x.N$: straightforward application of inductive
      hypothesis.
    \item $\Sum A=NP$, with $\ell(A)\hole\cdot=\ell(N)\hole\cdot\ell(P)\hole\cdot$: we can partition
      $M'_i$ into what goes in $\ell(N)\hole\cdot$ and what goes in
      $\ell(P)\hole\cdot$. We can suppose that
      $\Sum A=(\ell(N)[M'_1,\dots,M'_h])(\ell(P)[M'_{h+1},\dots,M'_k])$
      without loss of generality,
      and by inductive hypothesis get standard $\sigma:N\xredto[i*]N'$ and
      $\rho:P\xredto[i*]P'$ (with $N'$ and $P'$ the correct pluggings of
      $\ell(N)$ and $\ell(P)$).

      Now, if we reduce $\Sum A=NP\xredto[i*]N'P\xredto[i*]N'P'$
      following first $\sigma$ and then $\rho$, the resulting
      reduction must be standard as all positions in $P$ are greater
      than those in $N$ according to $\prec_r$.
    \item $\Sum A=[N]\cdot P$: exactly as above, but without any constraint
      on the order in which the reductions are composed.
    \item $\Sum A=[\bang N]\cdot P$, with $\ell(\Sum A)=\bag{\bang{\hole\cdot}}\cdot
      \ell(P)\hole\cdot$: suppose that $M'_1=N$ and
      $P=\ell(P)[M'_2,\cdots,M'_k]$. By inductive hypothesis we have a
      standard $\rho:P\xredto[i*]P'=\ell(P)[M''_i]_{i=2}^k$, and as
      positions in $[\bang N]$ and \emph{non-linear} positions in $P$
      are incomparable, we can freely combine the reductions on $M'_1$
      and $P$ to get a standard one.\qedhere
  \end{enumcases}
 \end{itemize}
\end{proof}
Now we are able to show the desired result.
\begin{theorem}[Standardization]\label{thm:standard-inner}
If $M\redto[nd*]M'$, then there is a $\prec_r$-standard chain from $M$ to $M'$.
\end{theorem}
\begin{proof}
  By structural induction on $M'$, the term where the reduction ends.
     First, applying \autoref{the:patra}, we get
  $\sigma:M\xredto[o*]M''$ and $\rho:M''\xredto[ndi*]M'$. Now we
  strive to obtain two standard chains $\sigma':M\xredto[o*]M''$ and
  $\rho':M''\xredto[ndi*]M'$ to obtain the chain $\sigma'\rho'$ which
  is standard by \autoref{lem:divide}. The existence of a standard
  $\sigma'$ is assured directly by \autoref{lem:stdouter}, so we
  need to concentrate on finding $\rho'$.
  By using \hyperref[lem:inner_outer]{\autoref*{lem:inner_outer}(\ref*{lem:inner_shape})}, we get
  $M''=\ell(M')\hole{N_1,\dots,N_k}$, $M'=\ell(M')\hole{N'_1,\dots,N'_k}$ and
  $\rho_i:N_i\xredto[nd*]N'_i$. As all $N'_i$ are structurally
  strictly smaller than $M'$, we can apply inductive hypothesis on
  each $\rho_i$ and get standard $\rho_i':N_i\xredto[nd*]N'_i$. Then
  using \hyperref[lem:outer_shape_std]{\autoref*{lem:inner_outer}(\ref*{lem:outer_shape_std})} we can glue back those
  reductions into the standard reduction $\rho':M''\xredto[i*]M'$.
\end{proof}
\begin{example}
Let $I=\lambda x.x$, $M_{1}=I \bag{\bang{ ((\lambda xy.x)\bag{\bang{I}}\bag{\bang{I}})}}$,
$M_{2}= I \bag{\bang{I}}$, and
let $M=\lambda x.x \bag{\bang{M_{1}},\bang{M_{2}}}$.
The following reduction is standard:
$M_{1}=I \bag{ \bang{ ((\lambda xy.x)\bag {\bang{I}}\bag{\bang{I}})   } } \redto[lm]
(\lambda xy.x)\bag{\bang{I}}\bag{\bang{I}} \redto[lm] (\lambda y.I)\bag{\bang{I}} \redto[lm] I.$
As $M_{2} \redto[lm] I$, the following is standard too
\begin{multline*}
\lambda x.x \bag{  \bang   {    (I \bag{\bang{ ((\lambda xy.x)\bag{\bang{I}}\bag{\bang{I}})}}     )},\bang{(I \bag{\bang{I}})}    } \redto[i]
\lambda x.x \bag{  \bang   {   ((\lambda xy.x)\bag{  \bang{I} }\bag{\bang{I} })   },\bang{(I \bag{\bang{I}})}    } \redto[i]
\\
\lambda x.x \bag{   \bang{((\lambda xy.x)\bag{\bang{I}}\bag{\bang{I}})   },
\bang{I}} \redto[i]
\lambda x.x \bag{\bang{((\lambda y.I)\bag{\bang{I}})},\bang{I}} \redto[i] 
\lambda x.x \bag{\bang{I},\bang{I}}.\end{multline*}
\end{example}

Let us notice that, as opposed to the weak form of standardization given in \autoref{the:patra}, the
$\prec_{r}$-standardization does not hold for parallel reduction. A counterexample is the following.
\begin{example}
Let $I_{0}$ and $I_{1}$ denote two occurrences of the identity $\lambda x.x$, and let
$M=I_0[I_1[x^!,y^!]]\redto[g] I_0[x]+I_0[y]\redto[g] x + I_0[y]$ by reducing the inner redex first.
But reducing the leftmost redex first we obtain
$M \redto[g] I_1[x^!,y^!] \redto[g] x+y$.
So the previous result cannot be obtained by a standard reduction. 
 \end{example}




