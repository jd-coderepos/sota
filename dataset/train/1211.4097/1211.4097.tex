In this section we will prove that the non-deterministic reduction enjoys a standardization property.
As we recalled already in the introduction, the standardization property is based on an order relation between redexes.
We can define it formally as follows:
\begin{definition}
Let  be an order on positions in terms (which is extended to an order on
subterms of a given term). Suppose  is a reduction chain, and let
 and  be the -th term and fired redex in  respectively.
We say that  is \emph{-standard} if for every  we have that
 is not the residual of a redex  in  such that .
\end{definition}

We will prove that non deterministic reduction in  enjoys the standardization property
with respect to the order , which is the partial order on positions in  terms that, intuitively,
gives precedence to linear positions over non-linear ones, and then orders
linear positions left-to-right, with the proviso that positions inside the same bag
be not comparable. The formal definition follows.

\begin{definition}[Linear left-to-right order]\label{def:order}
For two subterms  and  inside the expression , we say that
 in  if and only if any of the following happens:
\begin{itemize}
 \item  is a subterm of ;
 \item  is linear in  while  is not;
 \item  and  are both linear in , ,  is in  and  is in .
 \item  and  are subterms of the same proper subexpression  of , and
  in ;
\end{itemize}
\end{definition}
\begin{example}
  in both  and
 ,
 while they are incomparable in .
\end{example}

Our starting point is the division of redexes in two classes, outer and inner.

\begin{definition}[\cite{PaganiTranquilli09}]\label{def:outerreduction}
Let . The \textdef{outer -reduction}  is the 
 \emph{linear} context closure of the -steps given in
Definitions~\ref{def:giantbaby}. 
A \textdef{non-outer -reduction}, called \textdef{inner} is
defnoted by .
\end{definition}
In other words, an outer reduction does not reduce inside reusable resources, so
an outer redex (\emph{i.e.}\ a redex for ) is a redex not under the scope of a 
 constructor. In particular a term corresponding to a 
-term has at most one outer-redex, which coincides with the head-redex.
Pagani and Tranquilli stated in some sense a weak standardization property for the giant reduction,
proving that inner redexes can always be postponed. Their result
can easily be extended to other reductions, in particular to the non-deterministic one.

\begin{theorem}[\cite{PaganiTranquilli09}]\label{the:patra}
Let .  implies  and .
\end{theorem}

We introduce now a further classification between outer redexes.
\begin{definition}
The set of \emph{leftmost} redexes  of a term  or a bag  are defined
  inductively by:
  
\end{definition}

In regular -calculus, the set  is at most a singleton, and
-standardness collapses to the regular notion of left-to-right order of redexes.

\begin{fact}
  Redexes in  are exactly the -minimal elements among
  all redexes of .
\end{fact}

In the following, we will consider in particular the non-deterministic reduction. So, let us introduce some notation.

\begin{notation}
Let .
 denotes that the reduction fires a
redex in , while we write  if the redex
is not a leftmost one. Moreover  and  will be short for 
for  and  respectively.

\end{notation}

\begin{lemma}\label{lem:rightpushing}
We have the following facts on non-leftmost reduction.
\begin{itemize}
\item  if and only if 
  and  with ;
\item  if and only if ,  and  with ;
\item   if and only if,
   and 
  with ;
\item   if and only if and  with .
\end{itemize}
\end{lemma}
The proof of standardization is based on an inversion property between outer redexes,
saying that a not-leftmost reduction followed by a leftmost one can always be replaced by a leftmost followed by 
an outer. This is the upcoming \autoref{lem:inversion}. In order to get it we first prove the following intermediate properties.

\begin{lemma} \label{lem:triangolo}
If  then

such that .
\end{lemma}
\begin{proof}
We will prove that 
such that . Then the statement of the lemma follows easily.
By induction on . 
\begin{enumcases}
\item  and  are not possible.
\item .
By inductive hypothesis.
\item .
There are three cases: , ,
.
Let .
,where
 range over all the possible decomposition of  into two parts, counting the reusable resources
with all the possible multiplicities. This means that in case  are considered two different subterms also
in case they are syntactically equal.
By inductive hypothesis, for all  there is
 such that , and the result follows by transitivity of .
The case  is similar.

Let .
Then we have that the substitution  is equal to the sum , where
 range as before. Since each component of this sum is a redex (the substitutions do not modify the
external shape of the terms), we can reduce each redex, so obtaining that for all
, .
On the other side,  is equal to the sum
, and the proof is done.



\item  and  or  and .
All by inductive hypothesis.\qedhere
\end{enumcases}


\end{proof}

\begin{lemma} \label{lem:quadrato}
If  then

such that .
\end{lemma}

\begin{proof}
By induction on .  cannot be  as it would be normal.

If  then
.
We proceed by cases:
\begin{enumcases}
\item The reduction is on , \emph{i.e.}\ .
For all , . Then we have by induction that
for all  there is  such that .
So the result follows.

\item The reduction is on , \emph{i.e.}\ ).
Let us set ,
where  occurs in all  (), since the substitution is linear.
Let  by reducing the occurrence of  in it.
So  (), and, by \autoref{lem:triangolo},
for all , there is 
such that . Since  and , the proof follows.\
\end{enumcases}

If 
the reduction on , and the case is similar to the first case of the previous point.\qedhere





\end{proof}

\begin{lemma}[Inversion] \label{lem:inversion}
 implies , for some .
\end{lemma}
\begin{proof}
We proceed by induction on .\\
Let , so .
Non leftmost reductions on  can be done in , in  or in  (). We procede by cases:
\begin{enumcases}
\item The reduction is on .].
Let .
We have that

Moreover, by reducing the leftmost redex

so that

for all .

\item The reduction is on .
Let  and let , where  is such that
.
Moreover, by reducing the leftmost redex, we also have the reduction
.
By \autoref{lem:quadrato},  implies that for all ,
there exists  such that .
So there is  such that


\item The reduction is in O.
Let , and let
,
where  is such that .
Again if we reduce the leftmost redex, we have the reduction
.
 implies, by \autoref{lem:triangolo},  such that
. So there is  such that we
can compose the reductions
.
\end{enumcases}

Let , and let  and .
In case , the proof is trivial. In case  the proof is by induction on .\qedhere

\end{proof}
\begin{corollary}\label{cor:reordering}
  If  then there are 
  and  with .
\end{corollary}


\begin{lemma}\label{lem:divide}\
\begin{enumerate}[(i)]
 \item Given  and ,
then  is
 -standard if and only if  is.
 In particular every chain of leftmost reductions is -standard.
 \item Given  and ,
then  is
 -standard if and only if both  and  are.
\end{enumerate}
\end{lemma}
 \begin{proof}
 The result follows easily from the definition of .
 \end{proof}
Now we can prove that the non-deterministic outer reduction is -standard.
\begin{lemma}[Non-deterministic outer standard reduction]
  \label{lem:stdouter}
  If , then there is a -standard
  non-deterministic outer reduction from  to .
\end{lemma}
\begin{proof}
   We reason by induction on the pair , where  is the
  length of the reduction sequence , and  is
  the number of symbols in .  By \autoref{cor:reordering},
  there is a reduction  and  with .
  If  then inductive hypothesis applies to ,
  giving -standard , which
  gives that  is
  -standard by \autoref{lem:divide}.
  In case  is the
  whole reduction, the proof is by cases on .
 The only non-obvious case is when : by \autoref{lem:rightpushing} we have  and
     and . We can apply inductive hypothesis to both as , and get . Now assuming that this is not
    -standard leads to a contradiction to the definition at
    the seam, since all linear positions in  are  with respect to
    those in .
\end{proof}
In order to prove that also inner reductions can be standardized, we need to introduce the notion of 
{\em outer shape} of a term.
\begin{definition}
  The \emph{outer shape}  of a term  is a context that
  is  with holes replacing all exponential arguments of 's
  bags.

  Formally, extending the definition to bags, we define
   inductively as follows.
  
  
\end{definition}
\begin{property}\label{lem:inner_outer}\
\begin{enumerate}[(i)]
\item \label{lem:inner_shape}  if and only if , and there are 
  terms  and  terms  such that
  ,  and
   for each .
\item \label{lem:outer_shape_std}
  If  and
  
  are standard, then there is a standard 
  .
\end{enumerate}
\end{property}
\begin{proof}\
\begin{itemize}
\item[i)] The if direction is a direct consequence of how  is
  defined and of context closedness of the reduction. We thus move to
  the only if direction.

  First, let us show that the property to prove is preserved by
  composition of reduction chains.

  Suppose
   with ,
  
  and . We can suppose
   by re-indexing (namely using
   and
   with
  ). So we just
  forget the bijections employed, and then we have by hypothesis
  , which is what is
  needed.

  Now, we can prove the property by reducing to the case of a single
  inner reduction, as composing multiple ones of them preserves the
  property.

  Take : the result follows by a straightforward
  induction on how the reduction is defined.

\item[ii)] The idea is that the reductions in the subterms can be freely
  rearrenged.

  Let us reason by generalizing to expressions and by structural
  induction on .
  \begin{enumcases}
    \item  or : nothing to prove.
    \item : straightforward application of inductive
      hypothesis.
    \item , with : we can partition
       into what goes in  and what goes in
      . We can suppose that
      
      without loss of generality,
      and by inductive hypothesis get standard  and
       (with  and  the correct pluggings of
       and ).

      Now, if we reduce 
      following first  and then , the resulting
      reduction must be standard as all positions in  are greater
      than those in  according to .
    \item : exactly as above, but without any constraint
      on the order in which the reductions are composed.
    \item , with : suppose that  and
      . By inductive hypothesis we have a
      standard , and as
      positions in  and \emph{non-linear} positions in 
      are incomparable, we can freely combine the reductions on 
      and  to get a standard one.\qedhere
  \end{enumcases}
 \end{itemize}
\end{proof}
Now we are able to show the desired result.
\begin{theorem}[Standardization]\label{thm:standard-inner}
If , then there is a -standard chain from  to .
\end{theorem}
\begin{proof}
  By structural induction on , the term where the reduction ends.
     First, applying \autoref{the:patra}, we get
   and . Now we
  strive to obtain two standard chains  and
   to obtain the chain  which
  is standard by \autoref{lem:divide}. The existence of a standard
   is assured directly by \autoref{lem:stdouter}, so we
  need to concentrate on finding .
  By using \hyperref[lem:inner_outer]{\autoref*{lem:inner_outer}(\ref*{lem:inner_shape})}, we get
  ,  and
  . As all  are structurally
  strictly smaller than , we can apply inductive hypothesis on
  each  and get standard . Then
  using \hyperref[lem:outer_shape_std]{\autoref*{lem:inner_outer}(\ref*{lem:outer_shape_std})} we can glue back those
  reductions into the standard reduction .
\end{proof}
\begin{example}
Let , ,
, and
let .
The following reduction is standard:

As , the following is standard too

\end{example}

Let us notice that, as opposed to the weak form of standardization given in \autoref{the:patra}, the
-standardization does not hold for parallel reduction. A counterexample is the following.
\begin{example}
Let  and  denote two occurrences of the identity , and let
 by reducing the inner redex first.
But reducing the leftmost redex first we obtain
.
So the previous result cannot be obtained by a standard reduction. 
 \end{example}




