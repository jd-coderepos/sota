\section{Experiments}
\label{sec:experiments}
This section presents the results of our compact representation for image retrieval, evaluate the localization accuracy \deeploc, and finally employ it for retrieval re-ranking.

\textbf{Experimental setup.}
We evaluate the proposed methods on Oxford Buildings dataset~\citep{PCISZ07} and Paris dataset~\citep{PCISZ08}, which are composed of 5063 and 6412 images, respectively.
We refer to these datasets as Oxford5k and Paris6k.
We additionally use 100k Flickr images~\citep{PCISZ07} to compose Oxford105k and Paris106k, respectively. A distractor set of 1 million images from Flickr~\citep{JDS10a} is additionally used to go at larger scale.
Retrieval performance is measured in terms of mean Average Precision (mAP).
We follow the standard protocol and use the bounding boxes defined on the query images\footnote{The query regions are cropped and then used as input to the CNN.}.
These bounding boxes are also employed to evaluate localization accuracy. 
PCA is learned on Paris6k when testing on Oxford5k and vice versa. In order to be fair, we directly compare our results only to previous methods that do not perform learning on the test set.

The focus of our work is not to train a CNN, but to extract visual descriptors from its convolutional layers.
We use networks widely used in the literature: AlexNet by~\cite{KSH12} and the very deep network (VGG16) by~\cite{SZ14}. 
We choose VGG16 instead of VGG19 because we observe that the latter does not always attain better performance while it has higher feature extraction cost.
Our representation is extracted from the last pooling layer, which has 256 feature channels for AlexNet and 512 for VGG16.
MatConvNet~\citep{VL14} is used to extract the features.

\begin{table}
\caption{Left: Comparison between the exhaustive sliding window and our alternative of window sampling and refinement. We report the average IoU \wrt the globally optimal window and the average percentage of windows evaluated \wrt to the exhaustive search (noted by \%W). Measurements are conducted on all pairs of Oxford5k query images and their corresponding positive images. Right: Performance (mAP) of \gfv and \rfv on Oxford5k. Resol. corresponds to the input image resolution (maximum dimension).  \label{tab:both}}
\vspace{1.5ex}
\setlength\extrarowheight{1pt}
\small
\centering
\begin{minipage}{0.49\textwidth}
\begin{tabular}{|@{\sssp}c@{\sssp}|@{\sssp}r@{\msp}r@{\sssp}|@{\sssp}r@{\msp}r@{\sssp}|@{\sssp}r@{\msp}r@{\sssp}|} \hline
\multirow{3}{*}{Search step }  & \multicolumn{6}{c|}{Aspect ratio change threshold }\\ \cline{2-7}
		   														&  \multicolumn{2}{c|}{1.1} & \multicolumn{2}{c|}{1.5} & \multicolumn{2}{c|}{2.0} \\  \cline{2-7}
		   														& IoU  & \%W 								& IoU  & \%W               &           IoU  & \%W     \\ \hline \hline
		 														1 & 81.8 & 8.9 								& 88.7 & 27.5 						 &           93.7 & 46.3    \\
		 											      2 & 79.9 & 0.5                & 83.8 & 2.0               &           86.6 & 3.6     \\
		 											      3 & 78.7 & 0.2                & 81.2 & 0.5               &           83.6 & 0.8     \\
		 											      4 & 77.0 & 0.1                & 79.5 & 0.2               &           81.5 & 0.3     \\
		 											      5 & 75.8 & 0.1                & 79.0 & 0.1               &           80.9 & 0.2     \\ \hline
\end{tabular}
\end{minipage}
\begin{minipage}{0.49\textwidth}
\begin{tabular}{|@{\ssp}c@{\ssp}|@{\ssp}c@{\ssp}|@{\ssp}c@{\ssp}|c@{\bsp}c@{\bsp}c@{\bsp}c|} \hline
\multirow{2}{*}{Network} & \multirow{2}{*}{Resol.}  & \multirow{2}{*}{\gfv} & \multicolumn{4}{c|}{\rfv} 														\\ \cline{4-7}
												 & 			&								   					            		&	 			=1   &		 	=2  & 		 =3 &	 		 =4 	\\ \hline \hline
\multirow{2}{*}{AlexNet} &					 1024					 & 		 								 			44.9 						&       47.9    &     54.6   &     \textbf{56.1}  &       55.6   \\
												 &            724 			   & 										       44.8            &       48.4    &     54.4   &     54.3  &       52.6   \\
\hline
\multirow{2}{*}{VGG16}   & 					 1024          &                           55.2            &       57.3    &     64.5   &     \textbf{66.9}  &       \textbf{67.4}   \\
		                     &            724          &                        52.2            &       54.8    &     58.0   &     60.9  &       60.3   \\
\hline
\end{tabular}
\end{minipage}
\vspace{2ex}
\end{table}

\textbf{Localization accuracy.}
To evaluate the accuracy of \deeploc, we employ pairs of Oxford5k query images and their corresponding positive images.
We first perform exhaustive search to detect the globally optimal window.
Then, we apply our speeded-up detector that evaluates fewer regions and in the end refines the best one.
In both cases the approximate max-pooling is used for each window evaluation.
We report Intersection over Union (IoU) with the optimal window and the percentage of windows evaluated compared to the exhaustive case.
Results are shown in Table~\ref{tab:both} (left).
We provide a large speed-up while maintaining a high overlap with the optimal detection.
Recall that our purpose is to apply this detector for fast re-ranking.
Measuring IoU provides evidence for localization accuracy, however we observed that it does not directly impact retrieval performance.
We finally set  and  for re-ranking usage.

In order to evaluate the localization accuracy with respect to ground-truth annotation we cross-match all 5 query images that exist per building.
One of them is used as a query (cropped bounding box), while for the other we compare the detected region to the ground-truth annotation.
Exhaustive evaluation achieves an IoU equal to 52.6\% (52.9\%) and the speeded-up approach achieves 51.3\% (51.4\%) on Oxford5k (Paris6k) datasets.
The accuracy loss is limited, while the localization is approximately 180 times faster. 
\deeploc provides a rough localization at low computational cost.
Such a setup results in an average re-ranking query time of 2.9 sec using AlexNet, when re-ranking 1000 images with a single threaded implementation. 
\begin{figure*}
\centering
  \includegraphics[height=0.25\textwidth]{figs/rerank_map_oxford_global}
\hfill
  \includegraphics[height=0.25\textwidth]{figs/rerank_map_oxford_vgg}
  \hfill
 \includegraphics[height=0.25\textwidth]{figs/rerank_map_flickr1m}
\caption{Performance of retrieval with re-ranking by \deeploc versus number of re-ranked images on Oxford105k and Oxford5k combined with 1M distractor images.
\label{fig:rerank_map}}
\end{figure*}

\textbf{Retrieval and re-ranking.}
We evaluate retrieval performance using \gfv and \rfv compact representations.
The \gfv vectors are \l2-normalized, PCA-whitened and \l2-normalized once more, while the corresponding processing of the \rfv is as described in Section~\ref{sec:vector}. Table~\ref{tab:both} (right) presents the results on Oxford5k.
We evaluate different input image resolutions and observe that the original image size (1024) provides higher performance.
Note that \gfv is similar to the one proposed by~\cite{ARSM+14}, however their process remains constrained by  standard input size and aspect ratio.
The proposed \rfv gives a large performance improvement at no extra cost, as both feature vectors have exactly the same dimensionality.
Regions of different scales are aggregated together, meaning that  combines regions at scales , , and~.
We set  in the following. 
In order to decompose the components of \rfv, we construct \rfv by aggregating only regions of . It achieves mAP equal to 63.0 on Oxford5k with VGG16. 
Aggregating both regions of  and  improves to 65.4.
Finally, adding  (original \rfv) performs 66.9 (see Table \ref{tab:both} right).
Filtering time is 12 ms on average for Oxford105k. 

Next, we employ \deeploc for image re-ranking and conduct performance evaluation on Oxford105k by re-ranking up to 1000 images.
The performance is consistently improved as shown in Figure~\ref{fig:rerank_map}.
\rfv brings a larger benefit and VGG16 performs better than AlexNet.
Query expansion, as described in Section~\ref{sec:retrieval}, improves the performance at low extra cost, since similarity is re-computed only for the re-ranked short-list.
Finally, we carry out experiment at larger scale with 1M distractor images and present results in Figure~\ref{fig:rerank_map}. \deeploc improves the performance by 13\% mAP.

Examples of ranking using \gfv and re-ranking using \deeploc are presented in Figure~\ref{fig:rerankexample}.
Recall that we only provide a rough object localization, since our main goal is to obtain improved image similarity.
Furthermore, the provided localization is accurate enough for re-ranking.

\begin{figure*}[t]
\scriptsize
\newcommand{\figh}{10ex}
\centering

\begin{tabular}{@{\sssp}c@{\sssp}}\includegraphics[height=\figh]{figs/rerank/23//q23_5865_7105.pdf}\end{tabular} 
\begin{tabular}{@{\sssp}c@{\sssp}}\includegraphics[height=\figh]{figs/rerank/23///q23_init1.pdf}\end{tabular}
\begin{tabular}{@{\sssp}c@{\sssp}}\includegraphics[height=\figh]{figs/rerank/23///q23_init2.pdf}\end{tabular}
\begin{tabular}{@{\sssp}c@{\sssp}}\includegraphics[height=\figh]{figs/rerank/23///q23_init3.pdf}\end{tabular}
\begin{tabular}{@{\sssp}c@{\sssp}}\includegraphics[height=\figh]{figs/rerank/23///q23_init4.pdf}\end{tabular}
\begin{tabular}{@{\sssp}c@{\sssp}}\includegraphics[height=\figh]{figs/rerank/23///q23_init5.pdf}\end{tabular}
\begin{tabular}{@{\sssp}c@{\sssp}}\includegraphics[height=\figh]{figs/rerank/23///q23_init6.pdf}\end{tabular}
\begin{tabular}{@{\sssp}c@{\sssp}}\includegraphics[height=\figh]{figs/rerank/23///q23_init7.pdf}\end{tabular}
\begin{tabular}{@{\sssp}c@{\sssp}}\includegraphics[height=\figh]{figs/rerank/23///q23_init8.pdf}\end{tabular}

\begin{tabular}{@{\sssp}c@{\sssp}}\includegraphics[height=\figh]{figs/rerank/23//q23_5865_7105.pdf}\\Query\\ \end{tabular} 
\begin{tabular}{@{\sssp}c@{\sssp}}\includegraphics[height=\figh]{figs/rerank/23///q23_rr1_init1.pdf}\\1  1\\ \end{tabular} 
\begin{tabular}{@{\sssp}c@{\sssp}}\includegraphics[height=\figh]{figs/rerank/23///q23_rr2_init21.pdf}\\21  2\\ \end{tabular} 
\begin{tabular}{@{\sssp}c@{\sssp}}\includegraphics[height=\figh]{figs/rerank/23///q23_rr3_init19.pdf}\\19  3\\ \end{tabular} 
\begin{tabular}{@{\sssp}c@{\sssp}}\includegraphics[height=\figh]{figs/rerank/23///q23_rr4_init13.pdf}\\13  4\\ \end{tabular} 
\begin{tabular}{@{\sssp}c@{\sssp}}\includegraphics[height=\figh]{figs/rerank/23///q23_rr5_init3.pdf}\\3  5\\ \end{tabular} 
\begin{tabular}{@{\sssp}c@{\sssp}}\includegraphics[height=\figh]{figs/rerank/23///q23_rr6_init25.pdf}\\25  6\\ \end{tabular} 
\begin{tabular}{@{\sssp}c@{\sssp}}\includegraphics[height=\figh]{figs/rerank/23///q23_rr7_init2.pdf}\\2  7\\ \end{tabular} 
\begin{tabular}{@{\sssp}c@{\sssp}}\includegraphics[height=\figh]{figs/rerank/23///q23_rr8_init8.pdf}\\8  8\\ \end{tabular} 
  
\vspace{2ex}

\begin{tabular}{@{\sssp}c@{\sssp}}\includegraphics[height=\figh]{figs/rerank/42//q42_0932_2625.pdf}\end{tabular} 
\begin{tabular}{@{\sssp}c@{\sssp}}\includegraphics[height=\figh]{figs/rerank/42///q42_init1.pdf}\end{tabular}
\begin{tabular}{@{\sssp}c@{\sssp}}\includegraphics[height=\figh]{figs/rerank/42///q42_init2.pdf}\end{tabular}
\begin{tabular}{@{\sssp}c@{\sssp}}\includegraphics[height=\figh]{figs/rerank/42///q42_init3.pdf}\end{tabular}
\begin{tabular}{@{\sssp}c@{\sssp}}\includegraphics[height=\figh]{figs/rerank/42///q42_init4.pdf}\end{tabular}
\begin{tabular}{@{\sssp}c@{\sssp}}\includegraphics[height=\figh]{figs/rerank/42///q42_init5.pdf}\end{tabular}
\begin{tabular}{@{\sssp}c@{\sssp}}\includegraphics[height=\figh]{figs/rerank/42///q42_init6.pdf}\end{tabular}
\begin{tabular}{@{\sssp}c@{\sssp}}\includegraphics[height=\figh]{figs/rerank/42///q42_init7.pdf}\end{tabular}
\begin{tabular}{@{\sssp}c@{\sssp}}\includegraphics[height=\figh]{figs/rerank/42///q42_init8.pdf}\end{tabular}
\begin{tabular}{@{\sssp}c@{\sssp}}\includegraphics[height=\figh]{figs/rerank/42///q42_init9.pdf}\end{tabular}
\begin{tabular}{@{\sssp}c@{\sssp}}\includegraphics[height=\figh]{figs/rerank/42///q42_init10.pdf}\end{tabular}
\begin{tabular}{@{\sssp}c@{\sssp}}\includegraphics[height=\figh]{figs/rerank/42///q42_init11.pdf}\end{tabular}

\begin{tabular}{@{\sssp}c@{\sssp}}\includegraphics[height=\figh]{figs/rerank/42//q42_0932_2625.pdf}\\Query\\ \end{tabular} 
\begin{tabular}{@{\sssp}c@{\sssp}}\includegraphics[height=\figh]{figs/rerank/42///q42_rr1_init1.pdf}\\1  1\\ \end{tabular} 
\begin{tabular}{@{\sssp}c@{\sssp}}\includegraphics[height=\figh]{figs/rerank/42///q42_rr2_init4.pdf}\\4  2\\ \end{tabular} 
\begin{tabular}{@{\sssp}c@{\sssp}}\includegraphics[height=\figh]{figs/rerank/42///q42_rr3_init220.pdf}\\220  3\\ \end{tabular} 
\begin{tabular}{@{\sssp}c@{\sssp}}\includegraphics[height=\figh]{figs/rerank/42///q42_rr4_init52.pdf}\\52  4\\ \end{tabular} 
\begin{tabular}{@{\sssp}c@{\sssp}}\includegraphics[height=\figh]{figs/rerank/42///q42_rr5_init15.pdf}\\15  5\\ \end{tabular} 
\begin{tabular}{@{\sssp}c@{\sssp}}\includegraphics[height=\figh]{figs/rerank/42///q42_rr6_init212.pdf}\\212  6\\ \end{tabular} 
\begin{tabular}{@{\sssp}c@{\sssp}}\includegraphics[height=\figh]{figs/rerank/42///q42_rr7_init10.pdf}\\10  7\\ \end{tabular} 
\begin{tabular}{@{\sssp}c@{\sssp}}\includegraphics[height=\figh]{figs/rerank/42///q42_rr8_init159.pdf}\\159  8\\ \end{tabular} 
\begin{tabular}{@{\sssp}c@{\sssp}}\includegraphics[height=\figh]{figs/rerank/42///q42_rr9_init26.pdf}\\26  9\\ \end{tabular} 
\begin{tabular}{@{\sssp}c@{\sssp}}\includegraphics[height=\figh]{figs/rerank/42///q42_rr10_init860.pdf}\\860  10\\ \end{tabular}  

\vspace{2ex}

\begin{tabular}{@{\sssp}c@{\sssp}}\includegraphics[height=\figh]{figs/rerank/4//q4_3004_7522.pdf}\end{tabular} 
\begin{tabular}{@{\sssp}c@{\sssp}}\includegraphics[height=\figh]{figs/rerank/4///q4_init1.pdf}\end{tabular}
\begin{tabular}{@{\sssp}c@{\sssp}}\includegraphics[height=\figh]{figs/rerank/4///q4_init2.pdf}\end{tabular}
\begin{tabular}{@{\sssp}c@{\sssp}}\includegraphics[height=\figh]{figs/rerank/4///q4_init3.pdf}\end{tabular}
\begin{tabular}{@{\sssp}c@{\sssp}}\includegraphics[height=\figh]{figs/rerank/4///q4_init4.pdf}\end{tabular}
\begin{tabular}{@{\sssp}c@{\sssp}}\includegraphics[height=\figh]{figs/rerank/4///q4_init5.pdf}\end{tabular}
\begin{tabular}{@{\sssp}c@{\sssp}}\includegraphics[height=\figh]{figs/rerank/4///q4_init6.pdf}\end{tabular}
\begin{tabular}{@{\sssp}c@{\sssp}}\includegraphics[height=\figh]{figs/rerank/4///q4_init7.pdf}\end{tabular}
\begin{tabular}{@{\sssp}c@{\sssp}}\includegraphics[height=\figh]{figs/rerank/4///q4_init8.pdf}\end{tabular}
\begin{tabular}{@{\sssp}c@{\sssp}}\includegraphics[height=\figh]{figs/rerank/4///q4_init9.pdf}\end{tabular}
\begin{tabular}{@{\sssp}c@{\sssp}}\includegraphics[height=\figh]{figs/rerank/4///q4_init10.pdf}\end{tabular}

\begin{tabular}{@{\sssp}c@{\sssp}}\includegraphics[height=\figh]{figs/rerank/4//q4_3004_7522.pdf}\\Query\\ \end{tabular} 
\begin{tabular}{@{\sssp}c@{\sssp}}\includegraphics[height=\figh]{figs/rerank/4///q4_rr1_init120.pdf}\\120  1\\ \end{tabular} 
\begin{tabular}{@{\sssp}c@{\sssp}}\includegraphics[height=\figh]{figs/rerank/4///q4_rr2_init118.pdf}\\118  2\\ \end{tabular} 
\begin{tabular}{@{\sssp}c@{\sssp}}\includegraphics[height=\figh]{figs/rerank/4///q4_rr3_init753.pdf}\\753  3\\ \end{tabular} 
\begin{tabular}{@{\sssp}c@{\sssp}}\includegraphics[height=\figh]{figs/rerank/4///q4_rr4_init467.pdf}\\467  4\\ \end{tabular} 
\begin{tabular}{@{\sssp}c@{\sssp}}\includegraphics[height=\figh]{figs/rerank/4///q4_rr5_init631.pdf}\\631  5\\ \end{tabular} 
\begin{tabular}{@{\sssp}c@{\sssp}}\includegraphics[height=\figh]{figs/rerank/4///q4_rr6_init82.pdf}\\82  6\\ \end{tabular} 
\begin{tabular}{@{\sssp}c@{\sssp}}\includegraphics[height=\figh]{figs/rerank/4///q4_rr7_init594.pdf}\\594  7\\ \end{tabular} 

  \vspace{2ex}

\begin{tabular}{@{\sssp}c@{\sssp}}\includegraphics[height=\figh]{figs/rerank/9//q9_4081_5270.pdf}\end{tabular} 
\begin{tabular}{@{\sssp}c@{\sssp}}\includegraphics[height=\figh]{figs/rerank/9///q9_init1.pdf}\end{tabular}
\begin{tabular}{@{\sssp}c@{\sssp}}\includegraphics[height=\figh]{figs/rerank/9///q9_init2.pdf}\end{tabular}
\begin{tabular}{@{\sssp}c@{\sssp}}\includegraphics[height=\figh]{figs/rerank/9///q9_init3.pdf}\end{tabular}
\begin{tabular}{@{\sssp}c@{\sssp}}\includegraphics[height=\figh]{figs/rerank/9///q9_init4.pdf}\end{tabular}
\begin{tabular}{@{\sssp}c@{\sssp}}\includegraphics[height=\figh]{figs/rerank/9///q9_init5.pdf}\end{tabular}
\begin{tabular}{@{\sssp}c@{\sssp}}\includegraphics[height=\figh]{figs/rerank/9///q9_init6.pdf}\end{tabular}
\begin{tabular}{@{\sssp}c@{\sssp}}\includegraphics[height=\figh]{figs/rerank/9///q9_init7.pdf}\end{tabular}
\begin{tabular}{@{\sssp}c@{\sssp}}\includegraphics[height=\figh]{figs/rerank/9///q9_init8.pdf}\end{tabular}
\begin{tabular}{@{\sssp}c@{\sssp}}\includegraphics[height=\figh]{figs/rerank/9///q9_init9.pdf}\end{tabular}
\begin{tabular}{@{\sssp}c@{\sssp}}\includegraphics[height=\figh]{figs/rerank/9///q9_init10.pdf}\end{tabular}

 \begin{tabular}{@{\sssp}c@{\sssp}}\includegraphics[height=\figh]{figs/rerank/9//q9_4081_5270.pdf}\\Query\\ \end{tabular} 
 \begin{tabular}{@{\sssp}c@{\sssp}}\includegraphics[height=\figh]{figs/rerank/9///q9_rr1_init1.pdf}\\1  1\\ \end{tabular} 
 \begin{tabular}{@{\sssp}c@{\sssp}}\includegraphics[height=\figh]{figs/rerank/9///q9_rr2_init2.pdf}\\2  2\\ \end{tabular} 
 \begin{tabular}{@{\sssp}c@{\sssp}}\includegraphics[height=\figh]{figs/rerank/9///q9_rr3_init7.pdf}\\7  3\\ \end{tabular} 
 \begin{tabular}{@{\sssp}c@{\sssp}}\includegraphics[height=\figh]{figs/rerank/9///q9_rr4_init5.pdf}\\5  4\\ \end{tabular} 
 \begin{tabular}{@{\sssp}c@{\sssp}}\includegraphics[height=\figh]{figs/rerank/9///q9_rr5_init3.pdf}\\3  5\\ \end{tabular} 
 \begin{tabular}{@{\sssp}c@{\sssp}}\includegraphics[height=\figh]{figs/rerank/9///q9_rr6_init4.pdf}\\4  6\\ \end{tabular} 
 \begin{tabular}{@{\sssp}c@{\sssp}}\includegraphics[height=\figh]{figs/rerank/9///q9_rr7_init8.pdf}\\8  7\\ \end{tabular} 
 \begin{tabular}{@{\sssp}c@{\sssp}}\includegraphics[height=\figh]{figs/rerank/9///q9_rr8_init6.pdf}\\6  8\\ \end{tabular} 
 \begin{tabular}{@{\sssp}c@{\sssp}}\includegraphics[height=\figh]{figs/rerank/9///q9_rr9_init43.pdf}\\43  9\\ \end{tabular} 
 \begin{tabular}{@{\sssp}c@{\sssp}}\includegraphics[height=\figh]{figs/rerank/9///q9_rr10_init9.pdf}\\9  10\\ \end{tabular} 

\caption{Examples of top retrieved images before (top) and after (bottom) re-ranking with \deeploc. 
On the left we show the query image and depict the bounding box in blue color.
When re-ranking is used, we present the top ranked images and report for each image its initial and final ranking.
The localization window is shown in magenta, while positive/negative/junk images are depicted with green/red/yellow border.
\label{fig:rerankexample}}
\vspace{0pt}
\end{figure*}
 \begin{table}[t]
\caption{Performance comparison with state of the art. We report results for compact vector representations (left) and for retrieval approaches employing geometry, re-ranking, query expansion, or vector approximations (right). D = dimensionality.
Our approaches are identified with bullets .
\label{tab:soa}}
\vspace{2ex}
\footnotesize
\centering
\begin{tabular}{|@{\sssp}l@{\sssp}|@{\sssp}l@{\sssp}|@{\sssp}c@{\sssp}|@{\sssp}c@{\sssp}|@{\sssp}c@{\sssp}|@{\sssp}c@{\sssp}|@{\sssp}l@{\sssp}|@{\sssp}c@{\sssp}|@{\sssp}c@{\sssp}|@{\sssp}c@{\sssp}|@{\sssp}c@{\sssp}|} \hline
Method	       			   & D 	  & Oxf5k & Par6k & Oxf105k & Par106k &  Method	      	           & 		Oxf5k & 		Par6k & 		Oxf105k &  Par106k    \\ \hline \hline
{\scriptsize\cite{JZ14}}   & 1024 & 56.0  &	-    &  50.2    & -        & {\scriptsize\cite{CMPM11}} &       82.7   & 		80.5  &     76.7     &    71.0  	  \\
{\scriptsize\cite{JZ14}}   & 128  & 43.3  & -    &  35.3    & -        & {\scriptsize\cite{DGBQG11}}& 	   81.4   & 	    80.3  &		76.7    &  	  - 	      \\
{\scriptsize\cite{BSCL14}} & 128  & 55.7  & -    &  52.3    & -        & {\scriptsize\cite{MPCM13}} &\textbf{84.9} &	    82.4  &\textbf{79.5} &    77.3      \\
{\scriptsize\cite{RSMC14}} & 256  & 53.3  & 67.0 &  48.9    & -        & {\scriptsize\cite{SLBW14}} &        75.2  &      74.1  &     72.9     &     -        \\
{\scriptsize\cite{BL15}}   & 256  & 53.1  & -    &  50.1    & -        & {\scriptsize\cite{TGSS14}} &        77.8  &       -    &       -      &       -       \\
{\scriptsize\rfv} 		   & 256  & 56.1  & 72.9 &  47.0    & 60.1     & {\scriptsize\cite{TAJ15}}  & 	    80.4  & 	    77.0  &		75.0     & 	  -	   	   \\
{\scriptsize\rfv} 		   & 512  &\textbf{66.9}&\textbf{83.0}&\textbf{61.6} &\textbf{75.7} & {\scriptsize\rfv\hspace{-2pt}+\deeploc\hspace{-2pt}+QE} & 		77.3     &\textbf{86.5} &   73.2    &\textbf{79.8}  \\ \hline
\end{tabular}
\vspace{1ex}
\end{table}

\textbf{Comparison to the state of the art.}
We compare the proposed methods to state-of-the-art performance of compact representations and approaches based on local features that perform precise descriptor matching, re-ranking or query expansion.
Results are shown in Table~\ref{tab:soa}\footnote{Small differences of scores compared to the first version of the manuscript on arxiv are due to a slightly different evaluation protocol used before. Now, the evaluation protocol is the standard one for these datasets.}. AlexNex and VGG16 are used to produce the 256D and 512D vectors for \rfv, respectively.
Regarding the compact representations, our short-sized \rfv outperforms all other approaches.
The better performance on Paris is inherited by the nature of the pre-trained networks; the baseline \gfv with VGG achieves 55.2 on Oxford5k and 74.7 on Paris6k.

Unlike previous description schemes derived from CNN layers, our approach compete with the best
approaches based on local features for geometric matching and query expansion.
Our 
\deeploc can even outperform them: while our results are lower
on Oxford, we achieve the best performance on Paris and, to the best of our knowledge, outperform all published results on this benchmark.
Higher scores on Paris6k are reported by~\cite{AZ12} (91.0) and by~\cite{ZJS15} (91.5). These are achieved by learning the codebook on Paris6k itself and by performing pre-processing of the indexed dataset.


\textbf{Discussion about other CNN-based approaches.}
\cite{RSMC14} propose to perform region cross-matching and accumulate the maximum similarity per query region.
We evaluate this cross-matching process on the collection of regional vectors used in \rfv; we simply skip the final aggregation process and keep the regional vectors individually.
The cross-matching achieves 75.2\% mAP on Oxford5k as a filtering stage, while re-ranking with \deeploc on top of this acts in a complementary way and increases the performance up to 78.1\%. 
However, cross-matching has two drawbacks.
Firstly, the region vectors have to be stored individually and increase the memory requirements by a factor of , where  is the number of extracted regions.
Secondly, the complexity cost is linear in the number of indexed images and quite high since it requires to compute  (\eg 1024~\citep{RSMC14}) inner products per image.
The work of~\cite{RSMC14} follows a non-standard evaluation protocol by enlarging the provided query bounding boxes.
In addition, the cost of their feature extraction is extremely high since they feed 32 images of resolution  to the CNN.
The recent work of \cite{XTHZ15} is quite similar to theirs and is applied on both retrieval and classification.

\cite{BL15} show that global sum-pooling on convolutional layer activations is better than max-pooling when the final image vectors are PCA-whitened. 
When whitening is not employed, then the latter is better.
In the context of object localization we efficiently evaluate a large number of candidate regions on query time with \deeploc.
Performing whitening on each candidate region vector significantly increases the cost and is prohibitive for this task.
We switch max-pooling to sum-pooling for both our proposed \rfv and \deeploc and test performance. Note that sum-pooling is a special case of our integral max-pooling with .
Switching to sum-pooling makes \rfv perform  and \rfv+\deeploc+QE perform  on Paris106k. These scores are directly comparable to our scores in Table~\ref{tab:soa} and reveal that our choice is consistently better in all cases within our pipeline.
