\def\rightmark{Appendices}
\setcounter{page}{1}\renewcommand\thepage{\roman{page}}
\begin{vshort}
\noindent The following appendices contain material that will not be part of the
final paper, if accepted.
\end{vshort}

\section{Unary Alphabet: Prop.~\ref{prop-1ep}}\label{ax-1ep}

We prove here Prop.~\ref{prop-1ep}: -EP[Rat] is in
\textsc{NLogSpace}.

\paragraph{Parikh Images.}  The proof employs the semilinear view of
unary rational relations: a \emph{semilinear set}  is a subset of
 described by a finite union of \emph{linear sets}
 defined as  where  is a base in
 and  is a set of  periods
, each in .  It is well known that
the \emph{Parikh image} (aka commutative image)  of a regular
language  over an alphabet  is a semilinear set in
 telling for each symbol of  how many times it
occurs in a word of .  Formally, let ,
then a vector  is in  iff there exists a word  in
 s.t.\ for all ,  the number
of occurrences of  in .

\begin{proof}[Proof of Prop.~\ref{prop-1ep}]
Given a rational relation  over the unary alphabet ,
we can view its normalized transducer
 as a nondeterministic finite
automaton  over the two-letters
alphabet .  The Parikh
image of  is then a semilinear set
 verifying



Assume , i.e.\
there exists a pair  in  with .  Then, there exists
a linear set  in  s.t.\ either
 with , or  but there exists
a period  in  verifying ---and
then there exists  in  s.t.\ .

It therefore suffices to check in \textsc{NLogSpace} for the existence
of such a non-decreasing basis  or such an increasing period
 in the normalized transducer  for .  This is rather
straightforward:
\begin{itemize}
\item a basis  is read along a simple accepting
  run in , hence a run of length at most , while
\item a period  is read along a simple loop on some state 
  of ; we have to check that  is both accessible and
  co-accessible, thus  should lie on an accepting run of length at
  most  and exhibit a loop of length at most .
\end{itemize}
In both cases it suffices to guess a suitable accepting run to find
such a  or .
\end{proof}

\section{Codes and Hardy Computations}

\subsection{Pure Codes: Lem.~\ref{lem-pure}}\label{app-pure}
\begin{proof}[Proof of Lem.~\ref{lem-pure}]
  Remember that ordinals are supplied with a \emph{left
  subtraction} operation, because if , then
   can be defined as the unique ordinal verifying
  .

  We define an inverse  by induction on the CNF of ordinals;
  this function yields pure codes exclusively:
  
\end{proof}



\subsection{Computing with Rational Relations:
  Lem.~\ref{lem-corr}}\label{ax-rules}

\paragraph{Forward and Backward Rules.}  We present here the two
relations  and  under the understanding that they are
suitably restricted to sequences in .  The relations below are
rational and even 1-bld.  It suffices to give the relations for ,
as  just reverses their direction and uses states ,
, and  instead of , , and
.  For  given in \eqref{eq-rz0},

for ,  in ,  in , and
 in .

We define  and analogously for .
The first thing to check is that the reflexive transitive closures of
 and  implement those of  and  as advertised.  A
helpful notion is that of a \emph{phase} of a state , which is a
sequence of rewrites of form

for some s in  and s in
, where  is  or  and thus  is
the corresponding state  or , and
 is an \emph{intermediate} state among
.  The idea is that a
phase ought to simulate exactly the effect of a single step
 or .
\begin{lemma}[Correctness of  and ]\label{lem-FwBwcor}
  Let  be in  and  be in .  Then
   iff  iff .
\end{lemma}
\begin{proof}
  The proof is conducted by a case analysis.  Because  is the
  inverse of  with substituted state names, it suffices to show
  the equivalence of (q_\Fw\sep\sep
    c) with . For , the correctness of
   is immediate.

  For , it suffices to consider a single step of , i.e.\ a
  pair of  and
   with  in ,  in
  , and  in .  Then we have a
  rewrite sequence
  
  Conversely, it suffices to consider a phase of .  It
  is necessarily of the form above, because for \eqref{eq-rFw13} to be
  applicable, the counter segment must be in , but the
  first step \eqref{eq-rFw10} puts an  at the end of the
  segment.  Thus  has to go through the appropriate number of
  applications of \eqref{eq-rFw11}.  Therefore, a phase of
   implies a rewrite of .

  We leave the case of  as an exercise for the reader, as it is
  very similar to that of .
\end{proof}

\subsubsection{Proof of Lem.~\ref{lem-corr}}  The lemma contains
several statements.  The fact that  and  are rational 1-bld
is by definition.  That they are terminating is because they check
that their counters are larger than  in limit steps.
Regarding weak implementation, thanks to
Lem.~\ref{lem-robust} and Lem.~\ref{lem-crct}, we know that
computations using  are weak implementations in the sense of
Lem.~\ref{lem-corr}.  Therefore, it remains to prove that the small steps
defined for  and  (i)~correctly implement the rules of
 and (ii)~are ``robust'' to losses.

Point~(i) was the topic of Lem.~\ref{lem-FwBwcor}, which in
combination with Lem.~\ref{lem-crct} proves the existence of rewrites 

with .

Turning to point~(ii), in order to prove that a rewrite of
form \eqref{eq-Fw} implies , we want to transform
it into a rewrite according to , which is known to
imply  thanks to Lem.~\ref{lem-robust} and
Lem.~\ref{lem-crct}.  We conduct a similar proof later
for \eqref{eq-Bw} with ,
proving \eqref{eq-Bw} to imply .
\begin{claim}
  If , then
  .
\end{claim}
Note that this holds trivially for , thus as in the proof of
Lem.~\ref{lem-FwBwcor}, we can focus on \emph{lossy phases} of form

for some intermediate state .  We will deal with lossy phases of
 here; the proof of the claim for  is similar.
\begin{proof}[Proof of Claim for ]
  First observe that in a lossy phase like \eqref{eq-lphase} for
  , because \eqref{eq-rFw10} and \eqref{eq-rFw13} are used
  at the beginning and the end of the phase, each intermediate  is necessarily in the language
  

  Define the \emph{atomic embedding} relation over an alphabet  as
  
  Moreover, if  and  are
  two sequences in  with , then we can find
   all in  s.t.\ , i.e.\ we can find suitable atomic embeddings
  while remaining in .

  Write  for the subrelation of  defined by
  \eqref{eq-rFw11} and  for that of
  \eqref{eq-rFw13}.  Let us show that these subrelations verify
  
  over .  This is immediate in most cases, but there
  is a non-trivial case that justifies the use of transitive
  closures in \eqref{eq-commutation} for \eqref{eq-rFw13}:
  
  
  To wrap up the proof of the claim, observe that we can apply
  repeatedly \eqref{eq-commutation} to a lossy phase
  like \eqref{eq-lphase} until we have obtained a proper phase of the
  form
  
  Therefore, by Lem.~\ref{lem-FwBwcor},
   as desired.
\end{proof}

Let us turn to the backward version of the claim:
\begin{claim}
  If , then
  .
\end{claim}
\begin{proof}
  We proceed as in the proof of the previous claim, by considering
  lossy phases and transforming them into reliable ones.  Focusing on
  , the crux of
  the argument mirrors \eqref{eq-commutation} with
  
  over  for  in .
  The cases can be solved rather easily thanks to the restriction to
   defined in \eqref{eq-def-L1}.  For instance,
  
  Similar arguments can be used to complete the proof.  
\end{proof}

\section{Complexity Bounds}

\subsection{Upper Bounds: Prop.~\ref{prop-up}}\label{ax-upb}
\paragraph{Coverability Algorithm.}  
The algorithm for deciding coverability in WSTS is known as
the \emph{backward coverability} algorithm: given an instance
 with , the algorithm starts with the
upward-closure  of  as initial set of potential
targets .  The algorithm then builds the set of predecessors
: any
sequence that covers  has to go through .  This process is
repeated with  until
stabilization, which occurs since upward-closed subsets of a wqo
display the
\emph{ascending chain condition}: there exists 
s.t.\ .  As this  contains all the words in
 that can cover , it remains to check whether 
belongs to the set or not.
This algorithm is effective because, although the sets  are
infinite, they can be represented by their -minimal elements,
which are in finite number thanks to the wqo.

\paragraph{Controlled Sequence.}
When moving from decidability issues to complexity ones, we need to
measure the complexity of basic operations in the previous algorithm.
The key computation here is that of a minimal element  in 
given a minimal element  of .  Since  is minimal,
it is produced from some  s.t.\
, i.e.\  and
 for some  in  and
 in .

Given  a normalized transducer
for , we know the accepting run with  as image is of form

with  in ,  in , and  as input.  Then none of the segments
 can have length greater than ,
or  would not be a minimal element of .  Therefore,
, and any  can be computed
in \textsc{NLogSpace}.

\begin{proof}[Proof of Prop.~\ref{prop-up}] The idea of our
  \emph{combinatory algorithm} is to derive an upper bound on the
  length of a sequence proving reachability.  A nondeterministic
  algorithm can then explore this search space for a witness.

  Assume the -LR[Rat] instance to be positive.  We consider now
  a sequence of upward-closed sets  such that  is in 
  but not in  for any , i.e.\ we do not wait for
  saturation of the 's but stop as soon as  appears.  We can
  extract a particular minimal element  in each
  .  Let ; by the previous
  analysis, , and of course .  The
  sequence  is a \emph{bad sequence}: for all
  , .  By the Length Function
  Theorem~\citep{SS-icalp2011}, the length  is bounded by the
  \emph{Cich\'on function} 
  relativized with .

  A nondeterministic algorithm can then set  and guess one by
  one a sequence of  words  in  with  in
   until .  The space
  required at each step is logarithmic in , which is bounded
  overall by the Hardy function 
  for the same relativized .

  Finally, given an instance  of -LR[Rat] of size
  , as  we can use  instead and bound the
  required space of each step by . The
  space requisites of this algorithm place it in
  , as the function  is polynomial.
\end{proof}

\section{Applications}

\subsection{Lossy Termination: Prop.~\ref{prop-term}}\label{ax-term}

We prove in this section Prop.~\ref{prop-term}: -LT[Rat] is 
-hard.

\begin{proof}[Proof Sketch of Prop.~\ref{prop-term}]
  We need for this proof to examine more carefully the construction
  in \autoref{sub-low}.  The following facts are decisive:
  \begin{enumerate}
  \item both  in Phase~\ref{c-forward} and  in
    Phase~\ref{c-backward} are terminating relations,
  \item the simulation of the semi-Thue system  in
    Phase~\ref{c-simul} can be
    carried instead with a ``time budget'': it employs sequences of the form
    , where  encodes a sequence of  and
     tells how many steps are still allowed---initially the same
    budget allocated by Phase~\ref{c-forward}, but
    decremented by  at each rewrite.  This allows to simulate a
    time-bounded semi-Thue system instead of a space-bounded one, but
    they are equivalent as far as  is concerned.
  \end{enumerate}

  Let us detail a bit further the changes we carry.  The new relation
   has to be modified to work on words in
  .  The relation  for
  Phase~\ref{c-forward} needs
  to duplicate its counter increments on both sides of the last separator
   in \eqref{eq-rFw0}, which becomes
  
  {The simulation relation  for Phase~\ref{c-simul}
  now decrements the time budget:}

  The reader can check that the defined relation  is 1-bld and
  rational, and that the constructed instance  terminates
  iff the R[Rewr] instance  was positive.
\end{proof}

\subsection{Lossy Channel Systems: Prop.~\ref{prop-lcs}}\label{ax-lcs}

We prove in this section Prop.~\ref{prop-lcs}: -LCS is
-hard.
\begin{proof}
We reduce from a -LR[Rat] instance  and use
Prop.~\ref{prop-lrr}.  Let \?T=\tup{Q,\Sigma,\Sigma,\delta,I,F}R\?C=\tup{Q\uplus\{q_i,q_f\},\Sigma\uplus\{\ that cycles
through its channel content: it starts with \?T(q,u,v,q')\?Tuv(q,?u!v,q')\delta'\ in
.  As in the proof of Prop.~\ref{prop-ratep}, we can tighten this
construction by reusing  for $.
\end{proof}


