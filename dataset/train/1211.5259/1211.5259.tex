\def\rightmark{Appendices}
\setcounter{page}{1}\renewcommand\thepage{\roman{page}}
\begin{vshort}
\noindent The following appendices contain material that will not be part of the
final paper, if accepted.
\end{vshort}

\section{Unary Alphabet: Prop.~\ref{prop-1ep}}\label{ax-1ep}

We prove here Prop.~\ref{prop-1ep}: $1$-EP[Rat] is in
\textsc{NLogSpace}.

\paragraph{Parikh Images.}  The proof employs the semilinear view of
unary rational relations: a \emph{semilinear set} $S$ is a subset of
$\+Z^k$ described by a finite union of \emph{linear sets}
$(\vec{b},\vec{P})$ defined as $\{\vec{b}+\sum_{i=1}^m
x_i\vec{p}_i\mid x_1,\dots,x_m\in\+N\}$ where $\vec{b}$ is a base in
$\+Z^k$ and $\vec{P}$ is a set of $m$ periods
$\vec{p}_1,\dots,\vec{p}_m$, each in $\+Z^k$.  It is well known that
the \emph{Parikh image} (aka commutative image) $\Psi(L)$ of a regular
language $L$ over an alphabet $\Sigma$ is a semilinear set in
$\+N^{|\Sigma|}$ telling for each symbol of $\Sigma$ how many times it
occurs in a word of $L$.  Formally, let $\Sigma=\{a_1,\dots,a_n\}$,
then a vector $\vec{u}$ is in $\Psi(L)$ iff there exists a word $u$ in
$L$ s.t.\ for all $1\leq i\leq n$, $\vec{u}(i)=|u|_{a_i}$ the number
of occurrences of $a_i$ in $u$.

\begin{proof}[Proof of Prop.~\ref{prop-1ep}]
Given a rational relation $R$ over the unary alphabet $\Sigma=\{a\}$,
we can view its normalized transducer
$\?T=\tup{Q,\Sigma,\Sigma,\delta,I,F}$ as a nondeterministic finite
automaton $\?A=\tup{Q,\Delta,\delta,I,F}$ over the two-letters
alphabet $\Delta=\{(a,\varepsilon),(\varepsilon,a)\}$.  The Parikh
image of $L(\?A)$ is then a semilinear set
$S\subseteq\+N^2$ verifying
\begin{equation}
 S=\{(m,n)\mid (a^m,a^n)\in R\}\;.
\end{equation}


Assume $R\cap{\embeds}\neq\emptyset$, i.e.\
there exists a pair $(m,n)$ in $S$ with $m\leq n$.  Then, there exists
a linear set $(\vec{b},\vec{P})$ in $S$ s.t.\ either
$\vec{b}=(b_1,b_2)$ with $b_1\leq b_2$, or $b_1>b_2$ but there exists
a period $\vec{p}=(p_1,p_2)$ in $\vec{P}$ verifying $p_1<p_2$---and
then there exists $x$ in $\+N$ s.t.\ $b_1+x_1p_1\leq b_2+xp_2$.

It therefore suffices to check in \textsc{NLogSpace} for the existence
of such a non-decreasing basis $\vec{b}$ or such an increasing period
$\vec{p}$ in the normalized transducer $\?T$ for $R$.  This is rather
straightforward:
\begin{itemize}
\item a basis $\vec{b}$ is read along a simple accepting
  run in $\?T$, hence a run of length at most $|Q|$, while
\item a period $\vec{p}$ is read along a simple loop on some state $q$
  of $Q$; we have to check that $q$ is both accessible and
  co-accessible, thus $q$ should lie on an accepting run of length at
  most $2|Q|$ and exhibit a loop of length at most $|Q|$.
\end{itemize}
In both cases it suffices to guess a suitable accepting run to find
such a $\vec{b}$ or $\vec{p}$.
\end{proof}

\section{Codes and Hardy Computations}

\subsection{Pure Codes: Lem.~\ref{lem-pure}}\label{app-pure}
\begin{proof}[Proof of Lem.~\ref{lem-pure}]
  Remember that ordinals are supplied with a \emph{left
  subtraction} operation, because if $\beta\leq \beta'$, then
  $\beta'-\beta$ can be defined as the unique ordinal verifying
  $\beta+(\beta'-\beta)=\beta'$.

  We define an inverse $\pi^{-1}$ by induction on the CNF of ordinals;
  this function yields pure codes exclusively:
  \begin{align}
    \pi^{-1}(0)&\eqdef \varepsilon\;,
  & \pi^{-1}\left(\sum_{i=1}^p\omega^{\beta_i}\right)&\eqdef \beta^{-1}(\beta_p)\tally\pi^{-1}\left(\sum_{i=1}^{p-1}\omega^{\beta_i-\beta_p}\right)\,.
  \qedhere
  \end{align}
\end{proof}



\subsection{Computing with Rational Relations:
  Lem.~\ref{lem-corr}}\label{ax-rules}

\paragraph{Forward and Backward Rules.}  We present here the two
relations $\Fw$ and $\Bw$ under the understanding that they are
suitably restricted to sequences in $\Seqs$.  The relations below are
rational and even 1-bld.  It suffices to give the relations for $\Fw$,
as $\Bw$ just reverses their direction and uses states $q_\Bw$,
$q_{\Bw_1}$, and $q_{\Bw_2}$ instead of $q_\Fw$, $q_{\Fw_1}$, and
$q_{\Fw_2}$.  For $R_0$ given in \eqref{eq-rz0},
\begin{align}
  q_\Fw\sep\sep\tally x\sep\tally^n
  &\;\mathbin{\Fw_0}\;q_\Fw\sep\sep x\sep\tally^{n+1}\label{eq-rFw0}
  \intertext{for all $n$ in $\+N$ and $x$ in $\Sigma_{k\tally}^\ks$.
  Let us repeat the rules for $R_1$ given in
  (\ref{eq-rFw10}--\ref{eq-rFw13}):}
  q_\Fw\sep \sep wa_0\tally x\sep \tally^{n+2}
  &\;\mathbin{\Fw_1}\;q_{\Fw_1}\sep
  w\tally\sep\purify(a_0x)\sep\tally^{n+1}a_0\tag{\ref{eq-rFw10}}\\
  q_{\Fw_1}\sep w\tally^m\sep x\sep\tally^{n+1}a_0^{p+1}
  &\;\mathbin{\Fw_1}\;q_{\Fw_1}\sep w\tally^{m+1}\sep x\sep\tally^{n}a_0^{p+2}
  \tag{\ref{eq-rFw11}}\\
  q_{\Fw_1}\sep w\tally^{m+1}\sep x\sep a_0^{n+2}
  &\;\mathbin{\Fw_1}\;q_{\Fw_1}\sep\sep w\tally^{m+1} x\sep\tally^{n+2}
  \tag{\ref{eq-rFw13}}
  \intertext{where $m,n,p$ range over $\+N$, $w$ over $\Sigma_k^\ks$,
  and $x$ over $\Sigma_{k\tally}^\ks$.  For $R_2$ defined in \eqref{eq-rz2},} 
  q_\Fw\sep \sep wa_i\tally x\sep \tally^{n+2}
  &\;\mathbin{\Fw_2}\;q_{\Fw_2}\sep
  wa_{i-1}\sep\purify(a_ix)\sep\tally^{n+1}a_0\label{eq-rFw20}\\
  q_{\Fw_2}\sep wa_{i-1}^{m}\sep x\sep\tally^{n+1}a_0^{p+1}
  &\;\mathbin{\Fw_2}\;q_{\Fw_2}\sep wa_{i-1}^{m+1}\sep x\sep\tally^{n}a_0^{p+2}
  \label{eq-rFw21}\\
  q_{\Fw_2}\sep wa_{i-1}^{m+1}\sep x\sep a_0^{n+2}
  &\;\mathbin{\Fw_2}\;q_{\Fw_2}\sep\sep wa_{i-1}^{m+1}\tally x\sep\tally^{n+2}
  \label{eq-rFw23}
\end{align}
for $i>0$, $m,n,p$ in $\+N$, $w$ in $\Sigma_k^\ast$, and
$x$ in $\Sigma_{k\tally}^\ks$.

We define $\Fw = \Fw_0 \cup \Fw_1 \cup \Fw_2$ and analogously for $\Bw$.
The first thing to check is that the reflexive transitive closures of
$\Fw$ and $\Bw$ implement those of $R_H$ and $R_H^{-1}$ as advertised.  A
helpful notion is that of a \emph{phase} of a state $q$, which is a
sequence of rewrites of form
\begin{equation}  
(q_R\sep\sep c_0)\mathbin{R}(q\sep x_1\sep
  c_1)\mathbin{R}\cdots\mathbin{R}(q\sep x_m\sep
  c_{m})\mathbin{R}(q_R\sep\sep c_{m+1})
\end{equation}
for some $c_i$s in $\mathrm{Confs}$ and $x_i$s in
$\Sigma_{k\tally}^\ks$, where $R$ is $\Fw$ or $\Bw$ and thus $q_R$ is
the corresponding state $q_\Fw$ or $q_\Bw$, and
$q$ is an \emph{intermediate} state among
$\{q_{\Fw_1},q_{\Fw_2},q_{\Bw_1},q_{\Bw_2}\}$.  The idea is that a
phase ought to simulate exactly the effect of a single step
$c_0\mathbin{R_H}c_m$ or $c_0\mathbin{R_H^{-1}}c_m$.
\begin{lemma}[Correctness of $\Fw$ and $\Bw$]\label{lem-FwBwcor}
  Let $j$ be in $\{0,1,2\}$ and $c,c'$ be in $\mathrm{Confs}$.  Then
  $(q_\Fw\sep\sep c)\mathbin{\Fw_j^\trc}(q_\Fw\sep\sep c')$ iff $(q_\Bw\sep\sep c')\mathbin{\Bw_j^\trc}(q_\Bw\sep\sep c)$ iff $c \mathbin{R_j^\trc} c'$.
\end{lemma}
\begin{proof}
  The proof is conducted by a case analysis.  Because $\Bw$ is the
  inverse of $\Fw$ with substituted state names, it suffices to show
  the equivalence of $\mbox{$(q_\Fw\sep\sep
    c)$}\mathbin{\Fw_j^\trc}(q_\Fw\sep\sep c')$ with $c \mathbin{R_j^\trc}
  c'$. For $j=0$, the correctness of
  $\Fw_0=\{(q_\Fw{\sep\sep},q_\Fw{\sep\sep})\}\cdot R_0$ is immediate.

  For $j=1$, it suffices to consider a single step of $R_1$, i.e.\ a
  pair of $c=wa_0\tally x\sep\tally^{n+2}$ and
  $c'=w\tally^{n+2}\purify(a_0x)\sep\tally^{n+2}$ with $n$ in $\+N$, $w$ in
  $\Sigma_k^\ks$, and $x$ in $\Sigma_{k\tally}^\ks$.  Then we have a
  rewrite sequence
  \begin{align*}
    (q_\Fw\sep\sep c)
    &\mathbin{\Fw_1}(q_{\Fw_1}\sep w\tally\sep\purify(a_0x)\sep\tally^{n+1}a_0)
    \tag{by \eqref{eq-rFw10}}\\
    &\mathbin{\Fw_1^{n+1}}(q_{\Fw_1}\sep w\tally^{n+2}\sep\purify(a_0x)\sep a_0^{n+2})
    \tag{by \eqref{eq-rFw11}}\\
    &\mathbin{\Fw_1}(q_{\Fw}\sep\sep w\tally^{n+2}\purify(a_0x)\sep\tally^{n+2})
    \tag{by \eqref{eq-rFw13}}\\
    &=(q_{\Fw}\sep\sep c')\;.
  \end{align*}
  Conversely, it suffices to consider a phase of $q_{\Fw_1}$.  It
  is necessarily of the form above, because for \eqref{eq-rFw13} to be
  applicable, the counter segment must be in $\{\tally\}^\ks$, but the
  first step \eqref{eq-rFw10} puts an $a_0$ at the end of the
  segment.  Thus $\Fw_1$ has to go through the appropriate number of
  applications of \eqref{eq-rFw11}.  Therefore, a phase of
  $q_{\Fw_1}$ implies a rewrite of $R_1$.

  We leave the case of $j=2$ as an exercise for the reader, as it is
  very similar to that of $j=1$.
\end{proof}

\subsubsection{Proof of Lem.~\ref{lem-corr}}  The lemma contains
several statements.  The fact that $\Fw$ and $\Bw$ are rational 1-bld
is by definition.  That they are terminating is because they check
that their counters are larger than $1$ in limit steps.
Regarding weak implementation, thanks to
Lem.~\ref{lem-robust} and Lem.~\ref{lem-crct}, we know that
computations using $R_H$ are weak implementations in the sense of
Lem.~\ref{lem-corr}.  Therefore, it remains to prove that the small steps
defined for $\Fw$ and $\Bw$ (i)~correctly implement the rules of
$R_H$ and (ii)~are ``robust'' to losses.

Point~(i) was the topic of Lem.~\ref{lem-FwBwcor}, which in
combination with Lem.~\ref{lem-crct} proves the existence of rewrites 
\begin{align}
  (q_\Fw\sep\sep\pi^{-1}(\alpha)\sep\tally^n)\:&\Fw_\embedd^\trc\:(q_\Fw\sep\sep\sep\tally^m)\label{eq-Fw}\\
  (q_\Bw\sep\sep\sep\tally^m)\:&\Bw_\embedd^\trc\:(q_\Bw\sep\sep\pi^{-1}(\alpha)\sep\tally^n)\label{eq-Bw}
\end{align}
with $m=H^\alpha(n)$.

Turning to point~(ii), in order to prove that a rewrite of
form \eqref{eq-Fw} implies $m\leq H^\alpha(n)$, we want to transform
it into a rewrite according to $(R_H)_\embedd^\trc$, which is known to
imply $m\leq H^\alpha(n)$ thanks to Lem.~\ref{lem-robust} and
Lem.~\ref{lem-crct}.  We conduct a similar proof later
for \eqref{eq-Bw} with $(R_H^{-1})_\embedd^\trc$,
proving \eqref{eq-Bw} to imply $m\geq H^\alpha(n)$.
\begin{claim}
  If $(q_\Fw\sep\sep c)\mathbin{\Fw_\embedd^\trc}(q_\Fw\sep\sep c')$, then
  $c\mathbin{(R_H)_\embedd^\trc}c'$.
\end{claim}
Note that this holds trivially for $\Fw_0$, thus as in the proof of
Lem.~\ref{lem-FwBwcor}, we can focus on \emph{lossy phases} of form
\begin{equation}\label{eq-lphase}
  (q_\Fw\sep\sep c_0)\mathbin{\Fw}(q\sep w_1\sep
  c_1)\mathbin{\Fw_\embedd}\cdots\mathbin{\Fw_\embedd}(q\sep w_m\sep
  c_{m})\mathbin{\Fw}(q_\Fw\sep\sep c_{m+1})
\end{equation}
for some intermediate state $q$.  We will deal with lossy phases of
$q_{\Fw_1}$ here; the proof of the claim for $q_{\Fw_2}$ is similar.
\begin{proof}[Proof of Claim for $\Fw_1$]
  First observe that in a lossy phase like \eqref{eq-lphase} for
  $q_{\Fw_1}$, because \eqref{eq-rFw10} and \eqref{eq-rFw13} are used
  at the beginning and the end of the phase, each intermediate $q\sep
  w_i\sep c_i$ is necessarily in the language
  \begin{equation}\label{eq-def-L1}\begin{array}{rl}
    L_1\eqdef \{q_{\Fw_1}\sep w\tally^{\ell+1}\sep x\sep
    \tally^{n}a_0^{p}\:&\mid w\in\pure(\Sigma_k^\ks),x\in\pure(\Sigma_{k\tally}^\ks),\ell\geq 0,\\&\; p>0,n+p\geq 2\}\;.
  \end{array}\end{equation}

  Define the \emph{atomic embedding} relation over an alphabet $\Gamma$ as
  \begin{align}\label{eq-def-atomic}
    \atomi&\eqdef\{(ww',waw')\mid a\in\Gamma,w,w'\in\Gamma^\ks\}\;.
    \shortintertext{Clearly,}
    \label{eq-atomic}
    {\embeds}&={\atomi^\trc}\;.
  \end{align}
  Moreover, if $y$ and $y'$ are
  two sequences in $L_1$ with $y\embeds y'$, then we can find
  $y_0,y_1,\dots,y_n$ all in $L_1$ s.t.\ $y=y_0\atomi
  y_1\atomi\cdots\atomi y_n$, i.e.\ we can find suitable atomic embeddings
  while remaining in $L_1$.

  Write $\Fw_{\ref{eq-rFw11}}$ for the subrelation of $\Fw$ defined by
  \eqref{eq-rFw11} and $\Fw_{\ref{eq-rFw13}}$ for that of
  \eqref{eq-rFw13}.  Let us show that these subrelations verify
  \begin{align}\label{eq-commutation}
    ({\atoml}\comp\Fw_{\ref{eq-rFw11}})&\subseteq
    (\Fw_{\ref{eq-rFw11}}\comp{\atoml})
    &({\atoml}\comp\Fw_{\ref{eq-rFw13}})&\subseteq
    (\Fw_{\ref{eq-rFw11}}\comp\Fw_{\ref{eq-rFw13}}\comp{\embedd})
  \end{align}
  over $L_1\times L_1$.  This is immediate in most cases, but there
  is a non-trivial case that justifies the use of transitive
  closures in \eqref{eq-commutation} for \eqref{eq-rFw13}:
  \begin{align*}
    q_{\Fw1}\sep w\tally^{m+1}\sep x\sep \tally a_0^{n+2}
  &\atoml q_{\Fw_1}\sep w\tally^{m+1}\sep x\sep a_0^{n+2}\\
  &\mathrm{\Fw_{\ref{eq-rFw13}}}\;q_\Fw\sep\sep w\tally^{m+1} x\sep\tally^{n+2}
  \shortintertext{should be rewritten into}
    q_{\Fw_1}\sep w\tally^{m+1}\sep x\sep \tally a_0^{n+2}
  &\mathrm{\Fw_{\ref{eq-rFw11}}}\;q_{\Fw_1}\sep w\tally^{m+2}\sep x\sep a_0^{n+3}\\
  &\mathrm{\Fw_{\ref{eq-rFw13}}}\;q_{\Fw}\sep\sep w\tally^{m+2} x\sep\tally^{n+3}\\
  &\embedd q_{\Fw}\sep\sep w\tally^{m+1} x\sep\tally^{n+2}\;.
  \end{align*}
  
  To wrap up the proof of the claim, observe that we can apply
  repeatedly \eqref{eq-commutation} to a lossy phase
  like \eqref{eq-lphase} until we have obtained a proper phase of the
  form
  \begin{equation}
   (q_\Fw\sep\sep c_0)\mathbin\Fw_{\ref{eq-rFw10}}(q_{\Fw_1}\sep w_1\sep
   c_1)\mathbin\Fw_{\ref{eq-rFw11}}\cdots\mathbin\Fw_{\ref{eq-rFw11}}(q_{\Fw_1}\sep
  w'_{m'}\sep c'_{m'})\mathbin\Fw_{\ref{eq-rFw13}}\comp{\embedd}\:(q_\Fw\sep\sep
  c_{m+1})\;.
  \end{equation}
  Therefore, by Lem.~\ref{lem-FwBwcor},
  $c_0\mathbin{(R_1)_\embedd}c_m$ as desired.
\end{proof}

Let us turn to the backward version of the claim:
\begin{claim}
  If $(q_\Bw\sep\sep c)\mathbin{\Bw_\embedd^\trc}(q_\Bw\sep\sep c')$, then
  $c\mathbin{(R^{-1}_H)_\embedd^\trc}c'$.
\end{claim}
\begin{proof}
  We proceed as in the proof of the previous claim, by considering
  lossy phases and transforming them into reliable ones.  Focusing on
  $\Bw_1$, the crux of
  the argument mirrors \eqref{eq-commutation} with
  \begin{align}
    (\Bw_{j}\comp{\atoml})
    &\subseteq({\atoml}\comp\Bw_{j})\;,
  \end{align}
  over $L_1\times L_1$ for $j$ in $\{\ref{eq-rFw11},\ref{eq-rFw13}\}$.
  The cases can be solved rather easily thanks to the restriction to
  $L_1$ defined in \eqref{eq-def-L1}.  For instance,
  \begin{align*}
    q_\Bw\sep\sep w\tally^{m+1}x\sep\tally^{n+2}
    &\mathbin{\Bw_{\ref{eq-rFw13}}}\;q_{\Bw_1}\sep w\tally^{m+1}\sep
      x\sep a_0^{n+2}\\
    &\atoml q_{\Bw_1}\sep w\tally^{m}\sep x\sep a_0^{n+2}
    \shortintertext{necessarily has $m>0$ in order to belong to $L_1$,
      thus can be rewritten into}
    q_\Bw\sep\sep w\tally^{m+1}x\sep\tally^{n+2}
    &\atoml q_\Bw\sep\sep w\tally^{m}x\sep\tally^{n+2}\\
    &\mathbin{\Bw_{\ref{eq-rFw13}}}\;q_{\Bw_1}\sep w\tally^{m}\sep x\sep a_0^{n+2}\;.
  \end{align*}
  Similar arguments can be used to complete the proof.  
\end{proof}

\section{Complexity Bounds}

\subsection{Upper Bounds: Prop.~\ref{prop-up}}\label{ax-upb}
\paragraph{Coverability Algorithm.}  
The algorithm for deciding coverability in WSTS is known as
the \emph{backward coverability} algorithm: given an instance
$\tup{R,w,w'}$ with $w\neq w'$, the algorithm starts with the
upward-closure $\embeds(\{w'\})$ of $w'$ as initial set of potential
targets $I_0$.  The algorithm then builds the set of predecessors
$I_1=I_0\cup R_\embedd^{-1}(I_0)=I_0\cup\embeds(R^{-1}(I_0))$: any
sequence that covers $w'$ has to go through $I_1$.  This process is
repeated with $I_{i+1}=I_i\cup\embeds(R^{-1}(I_{i}))$ until
stabilization, which occurs since upward-closed subsets of a wqo
display the
\emph{ascending chain condition}: there exists $n$
s.t.\ $I_{n+1}=I_n$.  As this $I_n$ contains all the words in
$\Sigma^\ks$ that can cover $w'$, it remains to check whether $w$
belongs to the set or not.
This algorithm is effective because, although the sets $I_i$ are
infinite, they can be represented by their $\embeds$-minimal elements,
which are in finite number thanks to the wqo.

\paragraph{Controlled Sequence.}
When moving from decidability issues to complexity ones, we need to
measure the complexity of basic operations in the previous algorithm.
The key computation here is that of a minimal element $u_{i+1}$ in $I_{i+1}$
given a minimal element $u_i$ of $I_i$.  Since $u_{i+1}$ is minimal,
it is produced from some $v_i\embedd u_i$ s.t.\
$u_{i+1}\mathbin{R}v_i$, i.e.\ $u_i=a_1\cdots a_m$ and
$v_i=v'_0a_1v_1\cdots v'_{m-1}a_mv'_m$ for some $a_j$ in $\Sigma$ and
$v'_j$ in $\Sigma^\ks$.

Given $\?T=\tup{Q,\Sigma,\Sigma,\delta,I,F}$ a normalized transducer
for $R$, we know the accepting run with $v_i$ as image is of form
\begin{equation}
  q_0\xrightarrow{(u'_0,v'_0)}q'_0\xrightarrow{(\varepsilon,a_1)}q_1\xrightarrow{(u'_1,v'_1)}q'_1\cdots
  q_{m-1}\xrightarrow{(u'_{m-1},v'_{m-1})}q'_{m-1}\xrightarrow{(\varepsilon,a_m)}q_m\xrightarrow{(u'_m,v'_m)}q'_m
\end{equation}
with $q_0$ in $I$, $q'_m$ in $F$, and $u_{i+1}=u'_0u'_1\cdots
u'_{m-1}u'_m$ as input.  Then none of the segments
$q_j\xrightarrow{(u'_j,v'_j)}q'_j$ can have length greater than $|Q|$,
or $u_{i+1}$ would not be a minimal element of $I_{i+1}$.  Therefore,
$|u_{i+1}|\leq |Q|\cdot(|u_i|+1)$, and any $u_{i+1}$ can be computed
in \textsc{NLogSpace}.

\begin{proof}[Proof of Prop.~\ref{prop-up}] The idea of our
  \emph{combinatory algorithm} is to derive an upper bound on the
  length of a sequence proving reachability.  A nondeterministic
  algorithm can then explore this search space for a witness.

  Assume the $(k+2)$-LR[Rat] instance to be positive.  We consider now
  a sequence of upward-closed sets $\embeds(\{w'\})=I_0\subsetneq
  I_1\subsetneq \cdots\subsetneq I_\ell$ such that $w$ is in $I_\ell$
  but not in $I_i$ for any $i<\ell$, i.e.\ we do not wait for
  saturation of the $I_i$'s but stop as soon as $w$ appears.  We can
  extract a particular minimal element $u_{i+1}$ in each
  $I_{i+1}\setminus I_i$.  Let $g(x)=|Q|\cdot (x+1)$; by the previous
  analysis, $|u_{i+1}|\leq g(|u_i|)$, and of course $|u_0|=|w'|$.  The
  sequence $u_0,u_1,\dots,u_\ell$ is a \emph{bad sequence}: for all
  $i<j$, $u_i\not\embeds u_j$.  By the Length Function
  Theorem~\citep{SS-icalp2011}, the length $\ell$ is bounded by the
  \emph{Cich\'on function} $h_{\omega^{\omega^{k+1}}}((k-1)|w'|)$
  relativized with $h(x)=x\cdot g(x)=|Q|x^2+|Q|x$.

  A nondeterministic algorithm can then set $w_0=w'$ and guess one by
  one a sequence of $\ell$ words $w_i$ in $I_i$ with $w_{i+1}$ in
  $\embeds(R^{-1}(\{w_i\}))$ until $w_\ell\embeds w$.  The space
  required at each step is logarithmic in $|w_i|$, which is bounded
  overall by the Hardy function $h^{\omega^{\omega^{k+1}}}((k-1)|w'|)$
  for the same relativized $h$.

  Finally, given an instance $\tup{R,w,w'}$ of $(k+2)$-LR[Rat] of size
  $n$, as $n\geq|Q|$ we can use $h(x)=x^3+x^2$ instead and bound the
  required space of each step by $h^{\omega^{\omega^{k+1}}}(n)$. The
  space requisites of this algorithm place it in
  $\F_{\omega^{\omega^{k+1}}}$, as the function $h$ is polynomial.
\end{proof}

\section{Applications}

\subsection{Lossy Termination: Prop.~\ref{prop-term}}\label{ax-term}

We prove in this section Prop.~\ref{prop-term}: $(k+2)$-LT[Rat] is 
$\F_{\omega^k}$-hard.

\begin{proof}[Proof Sketch of Prop.~\ref{prop-term}]
  We need for this proof to examine more carefully the construction
  in \autoref{sub-low}.  The following facts are decisive:
  \begin{enumerate}
  \item both $\Fw$ in Phase~\ref{c-forward} and $\Bw$ in
    Phase~\ref{c-backward} are terminating relations,
  \item the simulation of the semi-Thue system $\Upsilon$ in
    Phase~\ref{c-simul} can be
    carried instead with a ``time budget'': it employs sequences of the form
    $\gamma\sep\tally^t$, where $\gamma$ encodes a sequence of $\Seqs$ and
    $t$ tells how many steps are still allowed---initially the same
    budget allocated by Phase~\ref{c-forward}, but
    decremented by $1$ at each rewrite.  This allows to simulate a
    time-bounded semi-Thue system instead of a space-bounded one, but
    they are equivalent as far as $\F_{\omega^k}$ is concerned.
  \end{enumerate}

  Let us detail a bit further the changes we carry.  The new relation
  $R'$ has to be modified to work on words in
  $\Seqs\cdot\{\sep\}\cdot\{\tally\}^\ks$.  The relation $\Fw'$ for
  Phase~\ref{c-forward} needs
  to duplicate its counter increments on both sides of the last separator
  $\sep$ in \eqref{eq-rFw0}, which becomes
  \begin{align}
  q_\Fw\sep\sep\tally x\sep\tally^n\sep\tally^{n}
  &\;\mathbin{\Fw'_0}\;q_\Fw\sep\sep x\sep\tally^{n+1}\sep\tally^{n+1}\;.
  \intertext{The other cases of $\Fw'$ are based on those of
  $\Fw$ and additionally duplicate the contents after
  the last ``$\sep$'': for instance, for \eqref{eq-rFw11}:}
  q_{\Fw_1}\sep w\tally^m\sep x\sep\tally^{n+1}a_0^{p+1}\sep z
  &\;\mathbin{\Fw'_1}\;q_{\Fw_1}\sep w\tally^{m+1}\sep
  x\sep\tally^{n}a_0^{p+2}\sep z\;.
\end{align}
  {The simulation relation $\Sim'$ for Phase~\ref{c-simul}
  now decrements the time budget:}
\begin{align}
  \Sim'&\eqdef \Sim\cdot\begin{bmatrix}\sep\\\sep\end{bmatrix}\cdot\begin{bmatrix}\tally\\\tally\end{bmatrix}^{\!\ks}\cdot\begin{bmatrix}\tally\\\varepsilon\end{bmatrix}\;.
  \intertext{The other relations can be taken to simply preserve the
  time budget:}
  \Bw'&\eqdef\Bw\cdot\begin{bmatrix}\sep\\\sep\end{bmatrix}\cdot\begin{bmatrix}\tally\\\tally\end{bmatrix}^{\!\ks}\;,\\
  \Init'&\eqdef\Init\cdot\begin{bmatrix}\sep\\\sep\end{bmatrix}\cdot\begin{bmatrix}\tally\\\tally\end{bmatrix}^{\!\ks}\;,\\
  \Fin'&\eqdef\Fin\cdot\begin{bmatrix}\sep\\\sep\end{bmatrix}\cdot\begin{bmatrix}\tally\\\tally\end{bmatrix}^{\!\ks}\;.
  \shortintertext{We add a new relation $\mathsf{End}$ that enters an
  infinite loop if the full simulation has been carried:}
  \mathsf{End}&\eqdef\left(\begin{bmatrix}q_\Bw\sep\sep
  a_{k-1}^n\tally\sep\tally^n\sep\\q_{\mathsf{End}}\sep\sep
  a_{k-1}^n\tally\sep\tally^n\sep\end{bmatrix}\cdot\begin{bmatrix}\tally\\\tally\end{bmatrix}^{\!\ks}\right)
  + \left(\begin{bmatrix}q_{\mathsf{End}}\\q_{\mathsf{End}}\end{bmatrix}\cdot\mathrm{Id}_{\Sigma_{k\tally}\uplus\{\sep\}}^\ks\right)\,.
  \intertext{Finally, the source sequence becomes}
  w&\eqdef q_\Fw\sep\sep a_{k-1}^n\tally\sep\tally^n\sep\tally^n\;.
  \end{align}
  The reader can check that the defined relation $R'$ is 1-bld and
  rational, and that the constructed instance $\tup{R',w}$ terminates
  iff the R[Rewr] instance $\tup{\Upsilon,y,y'}$ was positive.
\end{proof}

\subsection{Lossy Channel Systems: Prop.~\ref{prop-lcs}}\label{ax-lcs}

We prove in this section Prop.~\ref{prop-lcs}: $(k+2)$-LCS is
$\F_{\omega^k}$-hard.
\begin{proof}
We reduce from a $(k+2)$-LR[Rat] instance $\tup{R,w,w'}$ and use
Prop.~\ref{prop-lrr}.  Let $\$$ be a fresh symbol and
$\?T=\tup{Q,\Sigma,\Sigma,\delta,I,F}$ the normalized finite
transducer for $R$.

We construct a LCS
$\?C=\tup{Q\uplus\{q_i,q_f\},\Sigma\uplus\{\$\},\delta'}$ that cycles
through its channel content: it starts with $w\$$ as initial channel
contents in some initial state of $\?T$, applies the transitions
$(q,u,v,q')$ of $\?T$ by reading $u$ from the channel and writing $v$
through a transition $(q,?u!v,q')$ of $\delta'$, and cycles back upon
reading $\$$ by transitions $(q,?\$!\$,q')$ in $\delta'$ for all
initial states $q'\in I$ and final states $q\in F$ of $\?T$.  Adding
to $\delta'$ the transitions $(q_i,\varepsilon,q)$ for $q$ in $I$ and
$(q,\varepsilon,q_f)$ for $q$ in $F$, then $(w,w')$ belongs to
$R_{\embedd}^\trc$ iff $q_i,w\$\rightarrow_\embedd^\trc q_f,w'\$$ in
$\?C$.  As in the proof of Prop.~\ref{prop-ratep}, we can tighten this
construction by reusing $\sep$ for $\$$.
\end{proof}


