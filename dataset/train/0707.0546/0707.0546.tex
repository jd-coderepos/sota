\documentclass[11pt]{article}

\usepackage[letterpaper,left=1in,right=1in,top=1in,bottom=1in]{geometry}
\usepackage{picins}
\usepackage{floatflt}
\usepackage{newalg}
\usepackage{xspace,pst-node}

\newcommand{\A}{A}
\newcommand{\J}{J}
\newcommand{\first}{f}
\newcommand{\second}{s}
\newcommand{\critical}{\mathrm{critical}}
\newcommand{\minlabel}{\ensuremath{\lambda_{\mathrm{min}}}}
\newcommand{\equivlabel}{\ensuremath{\lambda_{\mathrm{equiv}}}}
\newcommand{\lab}{\ensuremath{\lambda}}
\newcommand{\wrt}{{w.r.t.}\xspace}
\newcommand{\etal} {{\it et al. }} 

\usepackage{amsfonts,amsmath,amsthm}
\newtheorem{theorem}{Theorem}
\newtheorem{lemma}{Lemma}
\newtheorem{definition}{Definition}
\newtheorem{invariant}{Invariant}
\newtheorem{observation}{Observation}

\begin{document}

\title{Weighted Popular Matchings \thanks{Research supported by NSF Awards CCR 0113192 and CCF 0430650} }

\author{Juli\'{a}n Mestre \\begin{array}{lll|rrrr}
w(x_1) = 7 & \hspace{0.3cm} & x_1 \hspace{0.1cm}&\hspace{0.1cm} A & B & C  \0.05cm]
w(x_3) = 2 & & x_3 & C & A & D & E  \

& 
\vspace{-1em}
0.1cm]
\cnode*{1.5pt}{x1} \rput[B](-0.5,0){x_1}  \hspace{2.75em} & \cnode*{1.5pt}{b} \rput[B](0.5,0){B} \0.1cm]
\cnode*{1.5pt}{x3} \rput[B](-0.5,0){x_3} & \cnode*{1.5pt}{d} \rput[B](0.5,0){D} \
\psset{linewidth=1pt, linecolor=red}
\ncline{-}{x1}{a}
\ncline{-}{x2}{c}
\ncline{-}{x3}{d}
\ncline{-}{x4}{d}
\psset{linewidth=1pt,linecolor=blue,linestyle=dashed}
\ncline{-}{x1}{b}
\ncline{-}{x2}{d}
\ncline{-}{x3}{e}
\ncline{-}{x4}{e}
\1em]

0.1cm]
\cnode*{1.5pt}{x1} \rput[B](-0.5,0){x_1}  \hspace{2.75em} & \cnode*{1.5pt}{b} \rput[B](0.5,0){B} \0.1cm]
\cnode*{1.5pt}{x3} \rput[B](-0.5,0){x_3}   & \cnode*{1.5pt}{d} \rput[B](0.5,0){D} \
\psset{linewidth=1pt, linecolor=black}
\ncline{-}{x1}{a}
\ncline{-}{x2}{c}
\ncline{-}{x3}{d}
\ncline{-}{x4}{e}
&
0.1cm]
\cnode*{1.5pt}{x1} \rput[B](-0.5,0){x_1}  \hspace{2.75em}  & \cnode*{1.5pt}{b} \rput[B](0.5,0){B} \0.1cm]
\cnode*{1.5pt}{x3} \rput[B](-0.5,0){x_3}   & \cnode*{1.5pt}{d} \rput[B](0.5,0){D} \
\ncline{-}{x1}{a}
\ncline{-}{x2}{c}
\ncline{-}{x3}{e}
\ncline{-}{x4}{d}
\\begin{array}{l@{\hspace{3.5em}}l}
 \cnode*{1.5pt}{z} \rput[r](-0.25,0){z}  & \cnode*{1.5pt}{fy} \rput[l](0.25,0){\first(y)} \0.35cm]
\cnode*{1.5pt}{x} \rput[r](-0.25,0){x}  \
\ncline{-}{z}{fy}
\ncline[linestyle=dashed,dash=4pt 2pt,arrowinset=0.2]{->}{y}{fy}
\ncline{-}{y}{p}
\ncline[linestyle=dashed,dash=4pt 2pt,arrowinset=0.2]{->}{x}{p}
\end{floatingfigure}

\newcommand\myproof{ \noindent {\it Proof.}~}
\myproof By induction on . For the base case let  and . For the sake of contradiction assume that  is matched to  and . If , then promote  to  and demote  to . The swap improves the satisfaction by , but this cannot be since  is popular. If  then promote  to  and  to , and demote applicant  as depicted on the right. Thus, the satisfaction improves by . 

For the inductive case let  and . Assume like before that  and . If , then promote  and demote  to get a change in satisfaction of . If  then by induction  is matched to , promoting  to  and   to  while demoting  changes the satisfaction by . Finally, suppose . Let . If  then the usual promotions change the satisfaction by at least . Note that if  then  by definition of . Letting  play the role of  in the above argument handles this case.
In every case we have reached the contradiction that  is not popular, therefore the lemma follows.
\qed \\

\begin{lemma} \label{lemma:second} Let  be a popular matching, then every  is matched either to  or .
\end{lemma}

\begin{floatingfigure}[r]{3.75cm}
\vspace{1ex}0.35cm]
\cnode*{1.5pt}{y} \rput[r](-0.25,0){y}  & \cnode*{1.5pt}{p} \rput[l](0.25,0){\second(x) = p } \7em]
\end{array}
\ncline{-}{z}{fy}
\ncline[linestyle=dashed,dash=4pt 2pt,arrowinset=0.2]{->}{y}{fy}
\ncline{-}{y}{p}
\ncline[linestyle=dashed,dash=4pt 2pt,arrowinset=0.2]{->}{x}{p}
\end{floatingfigure}

\myproof
As a corollary of Lemma~\ref{lemma:first} no  can be matched to a job which is strictly better than  or in between  and . Hence we just need to show that  cannot be matched to a job which is strictly worse than . For the sake of contradiction let us assume this is the case.


Let  and . Note that  must be matched to some applicant , otherwise we get an immediate improvement by promoting  to . If  then promoting  and demoting  gives us a more popular matching because . Otherwise  belongs to  in which case . By Lemma~\ref{lemma:first} there exists  matched to . Promoting  to  and  to  while demoting  improves the satisfaction by . A contradiction.
\qed \\

Let  be a subgraph of  having only those edges between applicants and their first and second jobs. See the graph in Figure~\ref{figure:instance}.b. Theorem~\ref{theorem:characterization} tells us that every popular matching must be contained in . Ideally we would like every well-formed matching in  to be popular; unfortunately, this is not always the case. To remedy this situation, we will prune some edges from  that cannot be part of any popular matching. Then we will argue that every well-formed matching in the pruned graph is popular. In order to understand the intuition behind the pruning algorithm we need the notion of .

\begin{definition} A promotion path \wrt a well-formed matching  is a sequence , such that ,  , and for all , applicant  prefers  over .
\end{definition}

Such a path can be used to free  by promoting  to , for all , and demoting . We say the cost (in terms of satisfaction) of the path is , as everyone gets a better job except . To illustrate this consider the instance in Figure~\ref{figure:instance}, and the well-formed matching . The sequence  is a promotion path with cost  that can be used to free .

To see how promotion paths come into play, let  be a well-formed matching and  be any other matching. Suppose  prefers  over , we will construct a promotion path starting at . Note that  is an -job and must be matched in  to  such that . Thus, our path starts with . To extend the path from , check if   prefers  over , if that is the case,  and , otherwise the path ends at . Notice that if  then , which means the path must be simple. Coming back to , the applicant who induced the path, note that if  is greater than the cost of the path, then  cannot be popular because using the promotion path and promoting  to  gives us a more popular matching. On the other hand, it is easy to see that if for every applicant , the cost of the path induced by  is at least , then  cannot be more popular than . For example, if the weights are sufficiently spread apart then any promotion path out of an -job will have cost at least , and as a result, any well-formed matching would be popular.

\begin{observation} \label{obs:all-popular} If  for all  then every well-formed matching is popular.
\end{observation}

The pruning procedure keeps a label  for every -job . Based on these labels we will decide which edges to prune. The following invariant states the meaning these labels carry.

\begin{invariant} \label{invariant:label} Let  be an -job and  be any well-formed matching contained in the pruned graph. A minimum cost promotion path out of  \wrt  has cost exactly~.
\end{invariant}

We now describe the procedure {\sc pruned-strict} whose pseudo-code is given in Figure~\ref{figure:pruning-strict}. The algorithm works in iterations. The th iteration consists of two steps. First, we prune some edges incident to , making sure that these edges do not belong to any popular matching. Second, we label all the -jobs so that Invariant~\ref{invariant:label} holds for them. Note that later pruning cannot break the invariant for -jobs as promotion paths out of these jobs only use edges incident to applicants in .

In the first iteration we do not prune any edges. Notice that a promotion path out of an -job must end in its  mate, therefore line 1 sets the label of all -jobs to .

At the beginning of the th iteration we know that the invariant holds for all \mbox{-jobs}. Consider an applicant . Let  be a job  prefers over . Note that  must be an -job, therefore, in any well-formed matching included in the pruned graph the minimum cost promotion out  has cost . We can use the path to free  and then promote  to it, the total change in satisfaction is . Therefore, if  then no popular matching exists. Lines 3--5 check for this. The expression  is a shorthand notation for  where  is a job  prefers over ; if there is no such job, we define .

\begin{figure}[t]
\center
\psframebox[boxsep=true,framearc=0.05,framesep=8pt]{
\begin{minipage}{2cm}
\begin{algorithm}{prune-strict}{G}
\mbox{All -jobs get a label of .} \\
\begin{FOR}{i = 2 \TO k}
   \begin{FOR}{x \in C_i}
      \begin{IF}{ \minlabel(x,\first(x)) < w_i }
          \RETURN \mbox{``no popular matching exists''}
      \end{IF}
   \end{FOR} \\
   \begin{FOR}{p \in \mbox{-job}}
      S \= \{ x \in \A\, |\, \first(x) = p \} \\
      \begin{IF}{S = \{x\} }
          \lab(p) \= \min (w_i, \minlabel(x, \first(x) ) -w_i)
      \ELSE
          \lab(p) \= w_i \\
          \begin{FOR}{x \in S \mbox{ such that } \minlabel(x,p) < 2 w_i}
                \mbox{prune the edge } (x,p)
          \end{FOR}
      \end{IF}
   \end{FOR}
\end{FOR} \\
\begin{FOR}{x \in \A \mbox{ such that } \minlabel(x,\second(x)) < w(x)}
   \mbox{prune the edge } (x, \second(x))
\end{FOR} \end{algorithm}
\end{minipage} }
\caption{Pruning the graph with strict preference lists. \label{figure:pruning-strict}}
\end{figure}

Let  be an -job and  be the set of applicants in  whose first job is , also let  be a well-formed matching contained in the pruned graph. Suppose  consists of just one applicant , then  must belong to . A promotion path out of  either ends at  or continues with another job which  prefers over . Therefore , which must be non-negative. On the other hand, if  then only one of these applicants will be matched to  while the rest must get their second job. Suppose . Invariant~\ref{invariant:label} tells us that there exists a promotion path \wrt  out of  with cost  that can be used to free , which in turn allows us promote one of the other applicants in  to . Therefore if ,  is not popular, which means the edge  cannot belong to any popular matching and can safely be pruned. We set  because in the pruned graph  can only be matched to  such that . Lines 6--12 capture exactly this.

Finally, Lines 13--14 prune edges  that cannot be part of any popular matching because of promotion paths out of jobs between  and  on 's list, with cost .

To exemplify how {\sc prune-strict} works let us run the algorithm on the instance in Figure~\ref{figure:instance}. The jobs are labeled , , and . The only edge pruned is  because both  and  have  as their first job and . Hence, the only well-formed matching included in the pruned graph is . The next theorem states that  must be popular.

\begin{theorem} Let  be the resulting pruned graph after running {\sc prune-strict}. Then a matching is popular if and only if it is a well-formed matching in .
\end{theorem}

We have argued that no pruned edge can be present in any popular matching, let us now show that every well-formed matching  in the pruned graph is indeed popular. Let  be any other matching, our goal is to show that  is not more popular than . Suppose  prefers  over , this induces a promotion path at  with respect to . If  gets  in  then the cost of such a path is at least  by Lines 4-5. Otherwise,  and Lines 13-14 make sure the cost at the promotion path is at least . Since this holds for every applicant ,  cannot be more popular than . 

It is entirely possible that the pruned graph does not contain any well-formed matching. In this case we know that no popular matching exists.

\subsection{Implementation}

Let  be the graph with edge set , . Assuming the applicants are already partitioned into categories , we can compute  in  time. The pruning procedure also takes linear time, since the th iteration takes  time. Let  be the pruned graph. Finding a popular matching reduces to finding a well-formed matching in .

Abraham \etal \cite{AIKM05} showed how to build a well-formed matching for unweighted instances (), if one exists, in linear time. The unweighted setting is slightly simpler than ours. More specifically, the set of second jobs is disjoint from the set first jobs and every applicant in  has degree exactly 2. These two issues can be easily handled: First, for every edge  in  if  happens to be someone else's first job then prune the edge . Second, iteratively find an applicant  with degree 1, let  be 's unique neighbor, add  to the matching, and remove  and  from . All these modifications can be carried out in  time. If at the end some applicant has degree 0 there is no well-formed matching, and consequently no popular matching. Otherwise every applicant has degree 2 and the set of -jobs is disjoint from the set of -jobs, thus we can apply directly the linear time algorithm of Abraham \etal \cite{AIKM05}.

\begin{theorem} In the case of strict preferences lists, we can find a weighted popular matching, or determine that none exists, in  time. \label{theorem:strict}
\end{theorem}

Recall that at the beginning we modified the instance by adding dummy jobs at the end of every applicant's list. A natural objective would be to find a popular matching that minimizes the number of applicants getting a dummy job. The cited work also shows how to do this in  time; thus, it carries over to our problem.

\section{Preference lists with ties}

Needless to say, if ties are allowed in the preference lists, the solution from the previous section does not work anymore. We will work out an alternative definition for first and second jobs which will lead to a new definition of well-formed matchings. Like in the case without ties if a matching is popular then it must be well-formed, but the converse does not always hold. We will show how to prune some edges that cannot be part of any popular matching to arrive at the goal that every well-formed matching in this pruned graph is popular.

Let us start by revising the notion of first job. For , let  be the set of jobs on 's list with the highest rank. Let  be the graph with edges between applicants in  and their first jobs. We say a job/applicant is \emph{critical} in  if it is matched in every maximum matching of , otherwise we say it is \emph{non-critical}. For , define  as the highest ranked jobs on 's list which are non-critical in all . The graph  includes  and edges between applicants in  and their first jobs. We note that a critical node in  may be non-critical in some .

If  is non-critical in  we define  as the highest ranked set of jobs on 's list which are non-critical in all . If  is critical in  then  is the empty set.

\begin{observation} For every applicant  we have .
\end{observation}

Essentially, when  is non-critical in  we can show that all the jobs in  are critical, therefore  and  are always disjoint.

\begin{definition}
A matching  is \emph{well-formed} if, for all , the matching  is maximum in , and every applicant  is matched within . 
\end{definition}

Notice that when there are no ties all these definitions are identical to the ones given in the previous section. Before proceeding to prove Theorem~\ref{theorem:characterization} in the context of ties we review some basic notions of matching theory.

The following definitions are all with respect to a given matching . An \emph{alternating path} is a simple path that alternates between matched and free edges. An \emph{augmenting path} is an alternating path that starts and ends with a free vertex. An \emph{exchange path} is an alternating path that start with a matched edge and ends with a free vertex. We can update  along an augmenting or exchange path  to get the matching , the symmetric difference of  and . 

In our proofs we will make use of the following property of non-critical nodes, which is a part of the Gallai-Edmonds decomposition \cite{S03}.

\begin{lemma} \label{lemma:critical} Let  be a bipartite graph and let  be a non-critical vertex. Then, in every maximum matching  of  there exists an alternating path starting at  and ending with a free vertex.
\end{lemma}

\begin{proof}
If  is free in , the lemma is trivially true, so assume that  is matched in .
Since  is non-critical there is a maximum matching  in which  if free. In  there must be an alternating path \wrt  of even length that starts at  and ends with a vertex free in .
\end{proof}

The next two lemmas prove Theorem~\ref{theorem:characterization} under the new definition of well-formed.

\begin{lemma} \label{lemma:first-ties} Let  be a popular matching. Then, for all ,  is maximum in .
\end{lemma}

\begin{proof} 
By induction on . For the base case, suppose that  is not maximum, then there must be an augmenting path in  \wrt  starting at  and ending at . If  is free in  or  then we can update  along the path\footnote{While  is augmenting \wrt , it may not be augmenting \wrt  since  could be matched in . This can be easily fixed by removing  from  before doing the update. For the sake of succinctness, from now on we assume that such implicit fix always occurs when updating along a path that ends with a free edge leading to a matched node.} to improve the satisfaction by , so let us assume that there exists . If  then updating  along  gives us an improvement in satisfaction of . Suppose then that , and let  be a job in . Since , applicant  must prefer  over . If the  belongs to  then we can create an alternating cycle by replacing the section of  before  with the edge . Updating the matching along the cycle improves the satisfaction by . If  does not belong to  then appending  to  and updating along the resulting path changes the satisfaction by . In every case we reach the contraction that  is not popular, thus  must be maximum in .

For the inductive step, if  is not maximum we can find like before an augmenting path  starting at  and ending at a job . If  is free in  or  updating along  improves the satisfaction, so assume that  is matched in  to . Let  be a job in , if  belongs to  then we can construct an alternating cycle to improve the satisfaction, so assume that . Since , we know that ; by inductive hypothesis  must be strictly worse than . We update  along  to get ; note that  is free in . By inductive hypothesis  is maximum in all , therefore  is maximum in  as well. There are three cases to consider. First, if  then we can promote  to  and demote whoever is matched to , the total change in satisfaction is . Second, consider the case . Note that  cannot be free in , as this would contradict the maximality of . If  we are done since promoting  to  gives a total change in satisfaction of   \wrt , so assume . By definition of ,  is non-critical in . Thus we can find an exchange path~ \wrt  starting at  and ending at a job  free in . Note that  cannot be free in , otherwise  would not be maximum in , thus . Updating  along  to free  and promoting  to  gives us a new matching . The satisfaction of  \wrt  is , thus  cannot be popular.

Finally, we need to consider the most involved case, namely, . Note that we cannot use the argument given above because  need not be maximum in . In order to fix this let us forget about  and consider a matching  maximum in ; furthermore, assume  minimizes . The set  is made up of paths , each of which is augmenting \wrt . By inductive hypothesis  is maximum in  so each  starts at  and end at some , both free in . Let  be  and  be a job in . Assume  and , otherwise we fall in one of the cases we have already covered. Now suppose that  belongs to some path  for ; then we can replace the portion of the path of  before  with the edge  update  along the resulting path and then update along  to improve the satisfaction by . Thus we can assume that  for all . At this point we can safely update  along all paths  to get a matching , which is maximum in . Finally, we can use the argument above on . Namely, find an exchange path  \wrt  from  to ; if  for all  then the same argument applies. On the other hand, if  for some  then give  its original job , update the matching along  and promote  to , which improves the satisfaction by . Notice that in the last exchange we assumed  was free in , or equivalently, that  for all  and . Indeed, if  for some  and  then we can join together  and  using the edge  and update  along the resulting path to improve the satisfaction by .

In every case we have reached a contradiction, thus the lemma follows.
\end{proof}

\begin{lemma} \label{lemma:second-ties} Let  be a popular matching, then every applicant  is matched within .
\end{lemma}

\begin{proof} Recall that  is undefined only if every job in 's preference list is critical in some . But , 's the last resort job, is critical in  if and only if , in which case  by Lemma~\ref{lemma:first-ties}. Let us assume then that  is well defined.

For the sake of contradiction assume that the lemma does not hold for . Note that all jobs which  prefers over  are critical in , among these, only  have an edge to  in . Thus if Lemma~\ref{lemma:first-ties} is to hold,  must be strictly worse than any job .

Consider the applicant . If  then  can be demoted and  promoted to  to improve the satisfaction by . If , the job  is strictly worse than any job . Using the fact that  is non-critical in  we find an alternating path in  to a free (in ) job  which must be matched in  to . Updating along the path and promoting  to  improves the satisfaction by . The case where  is simpler as we can promote  to  and demote whoever was matched to ; the change in satisfaction is .
\end{proof}

This finishes the proof of Theorem~\ref{theorem:characterization} under the new definition of well-formed matching. Thus every popular matching is contained in , the graph consisting of those edges between applicants and their first and second jobs. Because the new definition of well-formed matching generalizes the one for strict preferences, we again encounter the problem that not every well-formed matching is popular. We proceed as before, pruning certain edges which are not part of any popular matching. Finally, we show that every well-formed matching in the pruned graph is popular.

It is time to revise the definition of promotion path. Let  be a well-formed matching. Our promotion path starts at , a job critical in , but non-critical in all . We find an alternating path in  \wrt  from  to  which starts and ends with a matched edge; we augment along the path to get . Let  be a job which according to  is better than  (or as good, but not in ), moreover let  be critical in , but non-critical in all . Since , the matching  is still maximum in . Find a similar alternating path in  \wrt  from  to , update , and so on. Finally, every applicant  is assigned to , except , the last applicant in the path, who is demoted. The cost of the path is defined as the satisfaction of  with respect to , or equivalently,  minus the weight of those applicants  who find  strictly better than  (recall that  may be as good as, but not in ). This is the price to pay, in terms of satisfaction, to free  using the path.

To see why this is the right definition, let  be a well-formed matching and  be any other matching. Suppose  prefers  over , we will construct a promotion path starting at . Since  is well-formed,  must be critical; let  be the smallest  such that  is critical in . Taking  we can find an alternating path that starts with  and ends at  which is free in ---the path cannot end in a job that is free in  because  is critical. Either  gets a worse job under , in which case the promotion path ends, or gets a job  which is better than , or just as good but does not belong to . We continue growing the path until we run into an applicant  who prefers  over , notice that since  we  are bound to find such an applicant. Now, if the cost of the path is less than  then we know the well-formed matching  is not popular. On the other hand, if the cost of the path induced by  is at least , for all such , we can claim that  is not more popular than .

We are ready to discuss the algorithm {\sc prune-ties} for pruning the graph in the presence of ties, which is given in Figure~\ref{figure:pruning-ties}. In the th iteration we prune some edges incident to applicants in  making sure these edges do not belong to any popular matching, and label those jobs that became critical in  such that Invariant~\ref{invariant:label-ties} holds.

\begin{invariant} \label{invariant:label-ties} Let  be a critical job in , and let  be any matching in the pruned graph and maximum in all , i.e.,  is maximum in  for all . A minimum cost promotion path out of  \wrt  has cost exactly .
\end{invariant}

In the first iteration we do not prune any edges from . Let  be a critical job in , and  be a maximum matching in . Every alternating path \wrt  out of  must end in some applicant in , therefore, Line 1 sets .

Recall that  where  ranges over jobs \emph{strictly better} than  in 's preference list. In addition, let us define  where  ranges over jobs not in  that have the same rank as other jobs in .

For the th iteration we assume Invariant~\ref{invariant:label-ties} holds for those jobs critical in some . Suppose there exists an applicant  such that . Then in every well-formed matching in the pruned graph we can find a promotion path to free a job  that  prefers over , and then promote  to . This improves the satisfaction by . Therefore, no popular matching exists. Lines 3--5 check for this.

Consider a vertex  non-critical in . We claim that if  or  then the edges  cannot be part of any popular matching and can thus be pruned. Indeed, let  be a matching maximum in all  and included in the pruned graph such that ; we will show that  cannot be a subset of any popular matching. Because  is non-critical in  we know there is an exchange path \wrt  from  to some applicant  such that , for otherwise  would not be maximum in . Augment along the path to get . While the matching  may not be maximum in  (in case ), it is still maximum in all . Invariant~\ref{invariant:label-ties} tells us we can find a promotion path to free a job  that  can be promoted to; note that because  the changes needed to free  do not affect . This improve the satisfaction of the matching, therefore  cannot be included in any popular matching. Since the edges  cannot be part of any popular matching they can safely be pruned. Lines 6--8 check this.

\begin{figure}
\center
\psframebox[boxsep=true,framearc=0.05,framesep=8pt]{
\begin{minipage}{2cm}
\begin{algorithm}{prune-ties}{G}
\mbox{All critical jobs in  get a label of .} \\
\begin{FOR}{i = 2 \TO k}
   \begin{FOR}{x \in C_i}
      \begin{IF}{ \minlabel(x, \first(x) )  < w_i }
          \RETURN \mbox{``no popular matching exists''}
      \end{IF}
   \end{FOR} \\
   \begin{FOR}{x \in C_{j\leq i} \mbox{ non-critical in } G_i} 
     \begin{IF}{ \minlabel(x, \first(x) )  < w_j + w_i \mbox{ or } \equivlabel(x) < w_i }
       \mbox{prune the edges between  and }
      \end{IF}
   \end{FOR} \\
   \begin{FOR}{p \mbox{ critical in } G_i \mbox{, but non-critical in } G_{<i}}
      \mbox{let } S \= \{ x \, | \, \exists \mbox{ alternating path from  to } \} \\
      \lab(p) \= \min_{x \in S} \big\{ w_i, \minlabel(x, \first(x) ) - w(x), \equivlabel(x) \big\}
   \end{FOR}
\end{FOR} \\
\begin{FOR}{x \in \A \mbox{ such that } \minlabel(x,\second(x)) < w(x)}
   \mbox{prune the edges between  and }
\end{FOR}
\end{algorithm}
\end{minipage}
}
\caption{Pruning the graph with ties.\label{figure:pruning-ties}}
\end{figure}

Finally, we must compute  for jobs  that are critical in , but non-critical in all . A promotion path out of  must begin with an alternating path starting and ending with a matched edge, going from  to some applicant . Since  is non-critical in  there must be alternating path in  to some applicant in , thus . Note that if  is non-critical then  and , otherwise the edges  would have been pruned earlier. We shall explore alternating paths out of  into  that start and end with a matched edge in some arbitrary matching  maximum in . In fact we only care about reaching applicants  critical in . Since  is an arbitrary matching, we must argue that a similar path can always be found in any matching  included in the pruned graph, maximum in all . To show this, augment along the path to get , the resulting matching is not maximum in  any more. Take , and consider the alternating path out of . This path must end at an applicant , matched in , but free in ---otherwise, if it ends in a job free in  then  is non-critical. For the sake of contradiction suppose that . Since  is critical, there must be a path in  from  to , such that  is free in ; which contradicts the fact that  is critical in . Thus we set  to the minimum of ,  and , for those applicants  that can be reached from  with an alternating path starting and ending with a matched edge. Lines 9--11 do this.

The last thing to consider are non-critical applicants  who may get their second job. We can promote them to a job  strictly better than  and start a promotion path from there. If such exchange improves the satisfaction then the edges  must be pruned. This is done in Lines 12--13.

\begin{theorem} Let  the resulting graph after running {\sc prune-ties}. Then a matching  is popular if and only if  is well-formed and .
\end{theorem}

We have shown that if there exists a popular matching it must be well-formed and be contained in . The proof that every well-formed matching  in the pruned graph is popular is similar to that for strict preferences. Let  be any other matching, we argue that  is not more popular than . Suppose  prefers  over , this induces a promotion path out of  with respect to . If  gets  in  then the cost of such a path is at least . Otherwise,  and Lines 12-13 make sure the cost of such promotion path is at least . Since this holds for every applicant ,  cannot be more popular than . 

So far we have been concerned with showing the correctness of the algorithm, in the next section we show how to implement these ideas efficiently.

\subsection{Implementation}

First we need to compute  and  for every applicant ; we do so in iterations. For  computing  is trivial. Now build  and find a maximum matching  in . Using the algorithm of Hopcroft and Karp \cite{HK73} this can be done in  time. The set of critical jobs in  can be computed in  time by growing a Hungarian tree \cite{CombOpt-book} from those jobs that are free in : By Lemma~\ref{lemma:critical} those jobs that are reachable from a free job by an alternating path must be non-critical and those jobs that are not reachable from any free job must be critical. Using this information compute  for all  and  for all . Now construct , augment  to get a maximum matching  in , and so on. Using Hopcroft-Karp to compute  from  takes  time. Adding up over all categories we get overall  time.

The next lemma argues that {\sc prune-ties} can be implemented to run in  time. The procedure makes use of the matchings  found while computing  and  and the list of critical jobs in each .

\begin{lemma} Given a matching  maximum in , the th iteration of {\sc prune-ties} can be carried out in  time.
\end{lemma}

\begin{proof}
At the beginning of the th iteration we have available  for all jobs that are critical in some . Using this information it is easy to compute  and  in  time for each . With this information, Lines 3--8 can be done in  time.

Note that for each  critical in , but non-critical in all , Lines 10-11 can be implemented in linear time: Grow a Hungarian tree \wrt   out of , keep track of the applicants  that can be reached from , and find the one minimizing . But we would like to carry out this computation \emph{for all} such jobs within the same time bounds. This can be done provided the applicants  are sorted in non-decreasing value of : Instead of growing Hungarian trees from the jobs we grow Hungarian trees from the matched applicants in sorted order. When growing a tree out of applicant  we mark the nodes we visit and do not explore edges that lead to nodes that have already been marked. Suppose that job  critical in  was marked by applicant  then clearly . Because a node is never explored after it has been marked, the overall work is . If we have a sorted list of applicants in  adding the applicants  takes  time if we maintain the list using a balanced search tree.
\end{proof}

Finally, after  is computed and pruned we must find a well-formed matching in it. This problem can be reduced to finding a rank-maximal matching which can be done in time  \cite{IKMMP06}. Edges between  and  get a rank of , and edges from applicants to their second job get a rank of . If the resulting rank-maximal matching is well-formed, i.e., applicant complete and maximum in all  graphs, then we have a popular matching, otherwise no popular matching exists.

\begin{theorem} In the presence of ties we can find a weighted popular matching or determine that none exists in  time. \label{theorem:ties}
\end{theorem}

Finding a popular matching of maximum cardinality, i.e., one that minimizes the assignment of dummy last-resort jobs, can the done within the same time bounds. Note that  then the pair  will be in every well-formed matching so there is no point in minimizing these edges. If  we can give the edge  a rank of . Finding a rank-maximal matching in the new instance gives us a popular matching with maximum cardinality.

\section{Conclusion}

We have developed efficient algorithms for finding weighted popular matchings, a natural generalization of popular matchings. It would be interesting to study other definitions of the \emph{more popular than} relation. For example, define the satisfaction of  over  to be the sum (or any linear combination) of the differences of the ranks of the jobs each applicant gets in  and . Finding a popular matching under this new definition can be reduced to maximum weight matching, and vice versa. Defining the satisfaction to be a positive linear combination of the sign of the differences we get weighted popular matchings. We leave as an open problem to study other definitions that use a function ``in between'' these two extremes. Ideally, we would like to have efficient algorithms that can handle any odd step function.

\hspace{0.5cm}

{\bf Acknowledgment}: Thanks to David Manlove and Elena Zotenko for useful comments. Special thanks to Samir Khuller for suggesting the notion of weighted popular matchings and providing comments on earlier drafts.

\bibliographystyle{abbrv}
\bibliography{references,conferences}

\end{document}
