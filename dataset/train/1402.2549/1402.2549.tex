\documentclass[11pt,a4paper]{article}
\pdfoutput=1



\usepackage[vmargin=25mm,hmargin=25mm]{geometry}
\usepackage{amsmath,amssymb,amsthm,booktabs,color,enumitem,graphicx,latexsym,url,xspace}
\usepackage[numbers,sort&compress]{natbib}
\usepackage[basic,small]{complexity}
\usepackage[small]{caption}
\usepackage[outercaption]{sidecap}
\usepackage{microtype}
\usepackage{hyperref}

\newtheorem{theorem}{Theorem}
\newtheorem{corollary}[theorem]{Corollary}
\newtheorem{lemma}[theorem]{Lemma}

\theoremstyle{definition}
\newtheorem{definition}{Definition}

\newclass{\LOCAL}{LOCAL}
\newcommand{\A}{\mathcal{A}}

\hypersetup{
    colorlinks=true,
    linkcolor=black,
    citecolor=black,
    filecolor=black,
    urlcolor=[rgb]{0,0.1,0.5},
    pdfauthor={Miikka Hilke, Christoph Lenzen, Jukka Suomela},
    pdftitle={Local approximability of minimum dominating set on planar graphs},
}

\begin{document}

\begin{flushleft}
{\Large\bfseries Local Approximability of \\ Minimum Dominating Set on Planar Graphs\par}
\bigskip
\bigskip

\textbf{Miikka Hilke}

{\sffamily\small miikka.hilke@helsinki.fi}

{\footnotesize
Helsinki Institute for Information Technology HIIT,
Department of Computer Science, University of Helsinki, Finland\par}

\medskip
\textbf{Christoph Lenzen}

{\sffamily\small clenzen@csail.mit.edu}

{\footnotesize
Computer Science and Artificial Intelligence Laboratory, MIT\par}

\medskip
\textbf{Jukka Suomela}

{\sffamily\small jukka.suomela@aalto.fi}

{\footnotesize
Helsinki Institute for Information Technology HIIT,
Department of Information and Computer Science, Aalto University\par}
\end{flushleft}

\bigskip
\noindent\textbf{Abstract.}
We show that there is no deterministic local algorithm (constant-time distributed graph algorithm) that finds a -approximation of a minimum dominating set on planar graphs, for any positive constant . In prior work, the best lower bound on the approximation ratio has been ; there is also an upper bound of .

\medskip


\section{Introduction}

This work studies one of the last uncharted corners in the area of deterministic local algorithms: planar graphs.

A \emph{local algorithm} is a distributed graph algorithm that runs in  communication rounds, independently of the size of the network. While the theory of \emph{randomised} local algorithms is still in its infancy, we have nowadays a good understanding of the capabilities of \emph{deterministic} local algorithms.

For many classical graph problems, there are exactly matching upper and lower bounds on the best possible approximation ratio that can be achieved by a deterministic local algorithm \cite{suomela13survey}. In many cases, we can apply a straightforward two-step procedure to derive tight lower bounds:
\begin{enumerate}[noitemsep]
    \item Prove tight bounds for anonymous networks (without unique identifiers).
    \item Apply a simulation argument \cite{goos13local-approximation} to show that unique identifiers do not help.
\end{enumerate}
However, there are some isolated examples of natural questions in which the above two-step procedure fails badly. Perhaps the most intriguing example is \emph{dominating sets on planar graphs}:
\begin{enumerate}[noitemsep]
    \item We do not have tight bounds for this problem in anonymous networks.
    \item Planar graphs are not closed under lifts, and therefore the simulation argument~\cite{goos13local-approximation} cannot be applied.
\end{enumerate}
In this work we are interested in the smallest  such that there is a deterministic local algorithm that finds an -approximation of a minimum dominating set in any planar graph. The current bounds are very far from being tight:
\begin{itemize}[noitemsep]
    \item  for anonymous networks \cite{czygrinow08fast,wawrzyniak13mds-brief},
    \item  in the  model \cite{czygrinow08fast,lenzen11phd,lenzen13mds,wawrzyniak14mds-analysis}.
\end{itemize}
In this work we give the first improvement on the lower bounds in six years: we prove a lower bound  for both models, for any positive constant .


\section{Proof Overview}

Let  be a deterministic distributed algorithm with running time  in the  model. Assume that  finds a dominating set  in any planar graph~.

\begin{SCfigure}
\includegraphics[page=1]{figs.pdf}
\caption{(a)~Construction of graph  for , , and . There are  blocks. In each block there are  nodes:  internal nodes (white area), surrounded by a boundary area of width  (shaded). (b)~A dominating set  of  that contains only a fraction  of internal nodes. (c)~The local output of an internal node  (black node) only depends on its radius- neighbourhood (white nodes, here ). In particular, if we know the unique identifiers in the  region  around  (shaded area), we know the local output of node~.}\label{fig:G}
\end{SCfigure}

Pick sufficiently large  and . Let . We will construct a planar graph  with  nodes as shown in Figure~\ref{fig:G}a. There are  \emph{blocks} with  nodes in each block. The nodes of each block are partitioned to \emph{internal nodes} and \emph{boundary nodes}: there are  internal nodes, and they are surrounded by boundary areas of width . Let  be the set of nodes in block , and let  be the set of internal nodes in block . We will prove the following lemma.
\begin{lemma}\label{lem:main}
    For any  and any sufficiently large , we can assign unique identifiers in  so that  for all , for some .
\end{lemma}
In other words, all internal nodes of blocks  are in the dominating set  produced by algorithm . Now if we choose large enough  and , we can make the contributions of the boundary nodes and the contributions of the remaining  blocks arbitrarily small. In particular, for any positive constant , we can pick  and  such that .

On the other hand, there is a dominating set  which contains only a fraction  of the internal nodes; see Figure~\ref{fig:G}b. Therefore , and the claim follows: for any positive constant  we can show that algorithm  cannot find a factor  approximation of a minimum dominating set on planar graphs.


\section{Proof of Lemma~\ref{lem:main}}

The proof uses the strategy of repeated applications of Ramsey's theorem; cf.\ Czygrinow et al.~\cite[Lemma~4]{czygrinow08fast}. We will use the notation  to refer the \emph{local output} of node  when we apply algorithm  to graph ; we have  if node  is in the dominating set computed by algorithm~. By definition,  only depends on the radius- neighbourhood of  in .

Let , , and . Consider any internal node  of any block . The structure of graph~ in the radius- neighbourhood does not depend on the choice of . Hence the local output of node  only depends on the unique identifiers in the local neighbourhood. The local neighbourhood is contained within a rectangular  region ; see Figure~\ref{fig:G}c.

Let  be the set of unique identifiers. Consider any -subset of identifiers , . We will associate a \emph{colour}  with each such set, as follows:
\begin{enumerate}[noitemsep]
    \item Pick an internal node .
    \item Assign the identifiers from  to region  in an increasing order by rows: the smallest  identifiers to the bottom row from left to right, etc. Assign the identifiers from  to the remaining nodes arbitrarily.
    \item Apply algorithm , and set .
\end{enumerate}

Now we have defined a colouring of all -subsets of ; by restriction, we also have a colouring of all -subsets of any . We say that  is \emph{monochromatic} if  for any -subsets  and  of . By Ramsey's theorem \cite{ramsey30problem} there exists an integer  such that the following holds: if  is any -subset of , then there always exists a monochromatic subset  of size .

Now we will pick  and  so that  and . Let . For each , we define the identifiers of block  as follows.
\begin{enumerate}[noitemsep]
    \item As , we can find a monochromatic subset  of size .
    \item Assign the identifiers from  to block  in an increasing order by rows: the smallest  identifiers to the bottom row from left to right, etc.
    \item Set .
\end{enumerate}
Finally, assign the remaining  identifiers from  to blocks  arbitrarily.

To complete the proof, consider a block , where . Let  be an internal node of the block. Consider the  region  around , and let  be the set of unique identifiers assigned to region . Observe that the identifiers of  are assigned in an increasing order by rows. It follows that , i.e., the local output of the internal node  is simply the colour of subset . Furthermore,  and  was monochromatic. Hence all internal nodes of block  produce the same output. The common output cannot be ; otherwise there would be nodes that are not dominated. Hence .


\paragraph{Acknowledgements.}

Many thanks to Wojciech Wawrzyniak for discussions. This work was supported in part by the Deutsche Forschungsgemeinschaft (DFG, reference number Le 3107/1-1), by the Academy of Finland, Grant 252018, and by the Research Funds of the University of Helsinki.


\def\UrlFont{\sf\footnotesize}
\setlength{\bibsep}{3pt}
\bibliographystyle{abbrvnat}
\bibliography{planar-mds}

\end{document}
