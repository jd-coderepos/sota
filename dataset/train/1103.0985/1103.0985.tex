\documentclass[11pt,twoside,a4paper]{article}

\usepackage{epsfig}
\usepackage{amsfonts}
\usepackage{amssymb}
\usepackage{amstext}
\usepackage{amsmath}
\usepackage{xspace}
\usepackage{shadow,times}
\usepackage{algorithm,algorithmic}
\usepackage{fullpage}

\newcommand{\ignore}[1]{}







\allowdisplaybreaks

\newtheorem{theorem}{Theorem}
\newtheorem{lemma}[theorem]{Lemma}
\newtheorem{claim}[theorem]{Claim}



\newenvironment{proof}{

\noindent{\bf Proof:}} {\hfill


}


\newenvironment{Myquote}{\par\begingroup
\addtolength{\leftskip}{1em} \rightskip\leftskip }{\par
\endgroup
}


\newcommand{\initOneLiners}{\setlength{\itemsep}{0pt}
    \setlength{\parsep }{0pt}
    \setlength{\topsep }{0pt}
}

\newenvironment{OneLiners}[1][\ensuremath{\bullet}]
    {\begin{list}
        {#1}
        {\initOneLiners}}
    {\end{list}}

\newenvironment{proofof}[1]{

\bigskip\noindent{\bf Proof of {#1}:}}
{\hfill


}



\newcommand{\sse}{\subseteq}
\def\lrp{-LocVRP\xspace}
\def\opt{{\sf Opt}\xspace}
\newcommand{\kmed}{{\ensuremath\textrm{kmed}}}
\newcommand{\MST}{\ensuremath{\textrm{MST}}}
\def\kmf{ median forest\xspace}

\title{Locating Depots for Capacitated Vehicle Routing}
\author{
Inge Li G{\o}rtz\thanks{Technical University of Denmark.} \and Viswanath Nagarajan\thanks{IBM T.J. Watson Research
Center. } }
\date{}





\begin{document}


\maketitle
\begin{abstract}
We study a location-routing problem in the context of capacitated vehicle routing. The input to \lrp is a set of demand
locations in a metric space and a fleet of  vehicles each of capacity . The objective is to {\em locate} 
depots, one for each vehicle, and {\em compute routes} for the vehicles so that all demands are satisfied and the total
cost is minimized. Our main result is a constant-factor approximation algorithm for \lrp. To achieve this result, we
reduce \lrp to the following generalization of  median, which might be of independent interest. Given a metric
, bound  and parameter , the goal in the {\em  median forest} problem is to find
 with  minimizing:

where  and  is a minimum spanning tree in the graph obtained by
contracting  to a single vertex. We give a -approximation algorithm for  median forest, which leads
to a -approximation algorithm for \lrp, for any constant . The algorithm for   median
forest is -swap local search, and we prove that it has locality gap ; this generalizes the corresponding
result for  median~\cite{AGKMMP04}.

Finally we consider  the  median forest problem when there is a different cost function  for the MST part, i.e.
the objective is . We show that the locality gap for this problem
is unbounded even under multi-swaps, which contrasts with the  case. Nevertheless,  we obtain a constant-factor
approximation algorithm, using an LP based approach along the lines of~\cite{KKNSS11}.

\end{abstract}

\section{Introduction}\label{sec:intro}
In typical facility location problems, one wishes to locate centers and connect clients directly to centers at minimum
cost. On the other hand, the goal in vehicle routing problems (VRPs) is to compute routes for vehicles originating from
a given set of depots. Location routing problems represent an integrated approach, where we wish to  make combined
decisions on facility location and vehicle routing. This is a widely researched area in operations research, see eg.
surveys~\cite{BWW87,L88,L89,BJS95,MJS98,NS07}. Most of these papers deal with exact methods or heuristics, without any
performance guarantees. In this paper we present an approximation algorithm for a location routing problem in context
of capacitated vehicle routing.

Capacitated vehicle routing (CVRP) is an extensively studied vehicle routing problem~\cite{TV02} which involves
distributing identical items to a set of demand locations. Formally we are given a metric space  on vertices 
with distance function  that is symmetric and satisfies triangle inequality. Each
vertex  demands  units of the item. We have available a fleet of  vehicles, each having capacity 
and located at specified depots. The goal is to distribute items using the  vehicles at minimum total cost. There
are two versions of CVRP depending on whether or not the demand at a vertex may be satisfied over multiple visits. We
focus on the  {\em unsplit delivery} version in the paper, while noting that this also implies the result under
split-deliveries.


We consider the question ``where should one locate the  depots so that the resulting vehicle routing solution has
minimum cost?'' This is called {\em -location capacitated vehicle routing} (\lrp). The \lrp problem bears obvious
similarity to the well-known {\em  median} problem, where the goal is to choose  centers to minimize the sum of
distances of each vertex to its closest center. The difference is that our problem also takes the routing aspect into
account.  Not surprisingly, our algorithm for \lrp builds on approximation algorithms for the  median problem.

In obtaining an algorithm for \lrp we introduce the {\em   median forest} problem, which might be of some
independent interest. The objective here is a combination of -median and minimum spanning tree. Given metric
, bound  and parameter , the goal is to find  with  minimizing
.  Here  is the
minimum distance between  and an -vertex;  is a minimum spanning tree in the graph obtained by
contracting  to a single vertex. Note that when  we have the -median objective, and  being very
large reduces to MST.
\subsection{Our Results}
The main result is the following.
\begin{theorem}\label{th:lcvrp}
There is a -approximation algorithm for \lrp, for any constant .
\end{theorem}
Our algorithm first reduces \lrp to  median forest, at the loss of a constant approximation factor of four. This
step is fairly straightforward and makes use of
known lower-bounds~\cite{HK85} for the CVRP problem. We present this reduction in Section~\ref{sec:redn}.


Then we prove the following result in Section~\ref{sec:kmf} which implies Theorem~\ref{th:lcvrp}.
\begin{theorem}\label{th:kmed-forest}
There is a -approximation algorithm for  median forest, for any constant .
\end{theorem}
This is the technically most interesting part of the paper. The algorithm is straightforward: perform local search
using multi-swaps. It is well known that (single swap) local search is optimal for the minimum spanning tree problem.
Moreover, Arya et al.~\cite{AGKMMP04} showed that -swap local search achieves exactly a -approximation
ratio for the -median objective (this proof was later simplified by Gupta and Tangwongsan~\cite{GT08}). Thus one can
hope that local search performs well for  median forest, which is a combination of both MST and -median
objectives. However, the local moves used in proving the quality of local optima are different for the MST and
-median objectives. Our proof shows we can {\em simultaneously} bound both MST and -median objectives using a
common set of local moves. In fact we prove that the locality gap for \kmf under -swaps is also .
Somewhat surprisingly, it suffices to consider exactly the same set of swaps from~\cite{GT08} to establish
Theorem~\ref{th:kmed-forest}, although~\cite{GT08} does not take into account any MST contribution. The interesting
part of the proof is in bounding the change in MST cost due to these swaps--- this makes use of non-trivial exchange
properties of spanning trees and properties of the potential swaps from~\cite{GT08}. We remark that the -median,
-tree (i.e. choose  centers  to minimize ), and \kmf objectives are incomparable in general:
Appendix~\ref{app:example} gives an instance where near-optimal solutions to these three objectives are mutually far
apart.

Finally we consider the {\em non-uniform   median forest} problem in Section~\ref{sec:non-unif-kmf}. This is an
extension of  median forest where there is a different cost function  for the MST part in the objective. Given
vertices  with two metrics  and , and bound , the goal is to find  with  minimizing
. Here  is a minimum spanning tree in the
graph obtained by contracting  to a single vertex, {\em under metric }. In contrast to the uniform case , we
show that the locality gap here is unbounded even for multi-swaps. In light of this, Theorem~\ref{th:kmed-forest}
appears a bit surprising. Still, we show that a different LP-based approach yields:
\begin{theorem}\label{th:gen-kmed-forest}
There is a 16-approximation algorithm for non-uniform  median forest.
\end{theorem}
This algorithm follows closely that for the matroid median problem~\cite{KKNSS11}. We consider the natural LP
relaxation and round it in two phases. The first phase sparsifies the solution (using ideas from~\cite{CGTS99}) and
allows us to reformulate a new LP-relaxation using fewer variables; this is identical to~\cite{KKNSS11}. The second
phase solves the new LP-relaxation, which we show  to be integral.


\subsection{Related Work}
The basic capacitated vehicle routing problem involves a single fixed depot. There are two versions of CVRP: {\em split
delivery} where the demand of a vertex may be satisfied over multiple visits; and {\em unsplit delivery} where the
demand at a vertex must be satisfied in a single visit (in this case we also assume ). Observe
that the optimal value under split-delivery is at most that under unsplit-delivery. The best known approximation
guarantee for split-delivery is ~\cite{HK85,AG90} and for unsplit-delivery is ~\cite{AG87}, where
 denotes the best approximation ratio for the Traveling Salesman Problem. We make use of the following known
lower bounds for CVRP with single depot : the minimum TSP tour on all demand locations, and . Similar constant factor approximation algorithms~\cite{LS90} are also known for the CVRP with
multiple depots which was defined in the introduction.

The  median problem is a widely studied location problem and has many constant factor approximation algorithms.
Starting with the LP-rounding algorithm of~\cite{CGTS99}, the primal-dual approach was used in~\cite{JV01}, and also
local search~\cite{AGKMMP04}. A simpler analysis of the local search algorithm was given in~\cite{GT08}; we make use of
this in our proof for the  median forest problem. Several variants of  median have also been studied. One that is
relevant to us is the matroid median problem~\cite{KKNSS11}, where the set of open centers are constrained to be
independent in some matroid; our approximation algorithm for the non-uniform  median forest problem is based on this
approach.

Recently~\cite{HKM10} studied (among other problems) a facility-location variant of CVRP: there are opening costs for
depots and the goal is to open a set of depots and find vehicle routes so as to minimize the sum of opening and routing
costs. The \lrp problem in this paper can be thought of as the -median variant of~\cite{HKM10}. In~\cite{HKM10} the
authors give a 4.38-approximation algorithm for facility-location CVRP. Following an approach similar to~\cite{HKM10}
one can obtain a bicriteria approximation algorithm for \lrp, where more than  depots are opened. However more work
is needed to obtain a true approximation, and this is where we need an algorithm for the  median forest problem.




\section{Reducing \lrp to  median forest} \label{sec:redn}
\def\flow{{\sf Flow}}
\def\tree{{\sf Tree}}
\def\med{{\sf Med}}

Here we show that the \lrp problem can be reduced to  median forest at the loss of a constant approximation factor.
This makes use of known lower bounds for CVRP~\cite{HK85,LS90,HKM10}.

For any , let  and  be the
length of the minimum spanning tree in the metric obtained by contracting . The following theorem is implicit in
previous work~\cite{HK85,LS90,HKM10}; this uses a natural MST splitting algorithm.
\begin{theorem}[\cite{HKM10}] \label{th:cvrp}
Given any instance of CVRP on metric  with demands , vehicle capacity  and depots ,
\begin{OneLiners}
 \item The optimal value (of the split-delivery CVRP) is at least .
 \item There is a polynomial time algorithm that computes an unsplit-delivery solution of length at most .
\end{OneLiners}
\end{theorem}


Based on this it is clear that the optimal value of the CVRP instance given depot positions  is roughly given by
, which is similar to the \kmf objective. The following lemma formalizes this reduction. We will
assume an algorithm for the  median forest problem with vertex-weights , where the objective becomes
.


\begin{lemma}\label{lem:lrp2kmed}
If there is a -approximation algorithm for  median forest then there is a -approximation algorithm
for \lrp.
\end{lemma}
\begin{proof} Let \opt denote the optimal value of the \lrp instance. Using the lower bound in Theorem~\ref{th:cvrp},

where  is any value; this will be fixed later. Consider the instance of  median forest on metric
, vertex weights  and parameter . For any
 the objective is:
 Thus the optimal value of
the \kmf instance is at most . Let  denote the solution found by the
-approximation algorithm for \kmf. It follows that  and:
 For the \lrp
instance, we locate the depots at . Using Theorem~\ref{th:cvrp}, the cost of the resulting vehicle routing
solution is at most  where we set . From Inequality~\eqref{eq:lrp2kmed} it follows
that our algorithm is a -approximation algorithm for \lrp.
\end{proof}

\medskip

We remark that this reduction already gives us a constant factor {\em bicriteria} approximation algorithm for \lrp as
follows. Let  denote an approximate solution to -median on metric  with vertex-weights , which can be obtained by directly using a -median algorithm~\cite{AGKMMP04}. Let  denote the optimal
solution to , which can be obtained using the greedy MST algorithm. We output
 as a solution to \lrp, along with the vehicle routes obtained from
Theorem~\ref{th:cvrp} applied to . Note that , so we open at most  depots. Moreover, if
 denotes the location of depots in the optimal solution to \lrp then:
\begin{OneLiners}
\item  since we used a -approximation algorithm for
-median~\cite{AGKMMP04}.
\item  since   is an optimal solution to the MST part of the objective.
\end{OneLiners}
Clearly  and , so:
 Using Theorem~\ref{th:cvrp} the cost of the CVRP solution
with depots  is at most . So this gives a  bicriteria approximation
algorithm for \lrp, where  is any fixed constant. We note that this approach combined with algorithms for
facility-location and Steiner tree immediately gives a constant factor approximation for the facility location CVRP
considered in~\cite{HKM10}. The algorithm in that paper~\cite{HKM10} has to do  some more work in order to get a
sharper constant. For \lrp this approach clearly does not give any true approximation ratio, and for this purpose we
give an algorithm for \kmf.





\section{Multi-swap local search for \kmf} \label{sec:kmf}
\def\swap{\ensuremath{\mathcal{S}}\xspace}
\def\p{\ensuremath{\mathcal{P}}\xspace}
\def\f{\ensuremath{\mathcal{F}}\xspace}
\def\C{\ensuremath{\mathcal{C}}\xspace}


The input to {\em  median forest} consists of a metric , vertex-weights  and bound . The
goal is to find  with  minimizing:

where  and  is a minimum spanning tree in the graph obtained by contracting 
to a single vertex. Note that this is slightly more general than the definition in Section~\ref{sec:intro} (which is
the special case when  for all ).

We analyze the natural -swap local search for this problem, for any constant . Starting at an arbitrary solution
 consisting of  centers do the following until no improvement is possible: if there exists  and  with  and  then .
Clearly each local step can be performed in  time which is polynomial for fixed . The number of iterations
to reach a local optimum may be super-polynomial; however this can be made polynomial by the standard
method~\cite{AGKMMP04} of performing a  local move only if the cost  reduces by some  factor.
Here we omit this (minor) detail and bound the local optimum under the swaps as defined above.

Let  denote the local optimum solution (under -swaps) and  the global optimum. Note that
. Define map  as  for all . For any
, let , and  be the length of the minimum
spanning tree in the metric obtained by contracting ; so . For any  and  with  we refer to the swap  as a `` swap''. We use the following swap
construction from~\cite{GT08} for the -median problem.

\begin{theorem}[\cite{GT08}] \label{th:GT-swap}
For any  with , there are partitions  of  and 
of  such that  ; and there is a unique  (for each )
with  for all  and  for all .  Define set
 of -swaps with multipliers  as:
\begin{itemize}
 \item For any , if  then swap  with .
 \item For any , if  then for each  and  swap  with .
\end{itemize}
Then we have:
\begin{itemize}
 \item  .
 \item For each , the extent to which  is added .
\item For each , the extent to which  is dropped .
\end{itemize}
\end{theorem}

We use the same set  of swaps for the \kmf problem and will show the following:


Combined with the similar inequality in Theorem~\ref{th:GT-swap} (for ) and using local optimality of , we
would obtain the main result of this section:
\begin{theorem} The -swap local search algorithm for \kmf is a -approximation.\end{theorem}

It remains to prove~\eqref{eq:tree-swap}, which we do in the rest of the section. Consider a graph  which is the
complete graph on vertices  (for a new vertex ). If  denotes the edges in the metric,
 has edges ; the edges  are called {\em root-edges} and edges
 are {\em true-edges}. Let  denote the {\em spanning tree} of  consisting of edges ; similarly  is the spanning tree . For ease of notation, for
any subset , when it is clear from context we will use  to also denote the set  of
root-edges. We start with the following exchange property (which holds more generally for any matroid), see
Equation~(42.15) in Schrijver~\cite{Schr-book}.
\begin{theorem}[\cite{Schr-book}]\label{th:mat-exch}
Given two spanning trees  and  in a graph  and a partition  of the edges of ,
there exists a partition  of edges of  such that  is a
spanning tree in  for each . (This also implies  for all ).
\end{theorem}


\begin{figure}
\begin{center}
\includegraphics[scale=0.75]{partn.pdf}
\caption{The partitions used in local search proof (eg. has  and ).\label{fig:partn}}
\end{center}
\end{figure}


We will apply Theorem~\ref{th:mat-exch} on trees  and . Throughout,  and  represent the corresponding
edge-sets. Recall the partition  of  from Theorem~\ref{th:GT-swap}; we refine
 by splitting parts of size larger than  into singletons, and let  denote the resulting partition (see
Figure~\ref{fig:partn}). The reason behind splitting the large parts of  is to ensure the
following property (recall swaps \swap from Theorem~\ref{th:GT-swap}).
\begin{claim}\label{cl:fstar-partn}
For each swap ,  appears as a part in . Moreover, for each part  in  there is
some swap .
\end{claim}


Consider the partition  of  with parts , i.e. each true edge lies
in a singleton part  and the root edges form the partition  defined above. Let     denote the partition of  obtained by applying Theorem~\ref{th:mat-exch} with partition  of ; note also
that there is a {\em pairing} between parts of  and . Let  denote the true edges of  that
are paired with true edges of ; and  are the remaining true edges of  (see also
Figure~\ref{fig:partn}). We will bound the cost of  and  separately.

\begin{claim}\label{cl:kmf-m'}
.
\end{claim}
\begin{proof} Fix any . By the definition of  it follows that there is an 
such that part  in  is paired with part  in . In particular,   is a spanning tree in
. Note that the root edges in  are exactly , and so  is a spanning tree in the original metric
graph  when we contract vertices . Since  is the minimum such tree, we have  or . Summing over all  and observing that each edge  can be paired with at
most one , we obtain the claim.\end{proof}

Consider the connected components (in fact a forest) induced by true-edges of : for each  let 
denote the vertices connected to . Note that  partitions .

Now consider the forest induced by true edges of  (i.e. ) and direct each edge towards an -vertex
(note that each tree in this forest contains exactly one -vertex). Observe that each vertex 
has exactly one out-edge , and -vertices have none.

For each , define  the set of out-edges from .
\begin{claim}\label{cl:kmf-m*} .
\end{claim}
\begin{proof} It is clear that  partitions .\end{proof}

\medskip
We are now ready to bound the increase in the  cost under swaps \swap. By Claim~\ref{cl:fstar-partn} it follows
that for each swap ,  is a part in  (and so in ); define  as the true-edges of 
(possibly empty) that are paired with the part  of .

\begin{claim}\label{cl:kmf-Ea}  is a partition of .\end{claim}
\begin{proof}
Consider the partition  of  given by Theorem~\ref{th:mat-exch} applied to . By definition,  are the true edges of  paired (by  and ) with true edges of ; and  are
paired with parts from  (i.e. root edges of ). For each part  (and also ) let  denote the -edges paired with . It follows that  partitions . Using the
second fact in Claim~\ref{cl:fstar-partn} and the definition s, we have , a partition of .
\end{proof}

We prove the following key lemma.


\begin{lemma}\label{lem:kmf-main}
For each swap , .
\end{lemma}
\begin{proof}
By Claim~\ref{cl:fstar-partn},  is a part in . Recall that  denotes the true-edges of  paired
with ; let  denote the root-edges of  paired with . Then using Theorem~\ref{th:mat-exch} it follows that
 is a spanning tree in . Hence  is a forest
with each component containing some vertex from ; for any  let  denote vertices in the
component containing .
In other words,  connects connects each vertex to some vertex of .



Consider the edge set . We will add some edges  so that  connects
each -vertex to some vertex of . Since  already connects all vertices to , it would follow that
 connects all vertices to , i.e.

To prove the lemma it now suffices to construct a set  with , such that  connects each -vertex to . Below, for any  we use  to denote the edges of 
between  and .

\paragraph{Constructing } Consider any minimal  such that ; recall that s are
the connected components of . By minimality of , it follows that  is connected
in . We now prove two simple claims:



\begin{claim}\label{cl:tree-loc2} For any  we have .
\end{claim}
\begin{proof} By construction of the swaps \swap in Theorem~\ref{th:GT-swap}.
\end{proof}

\begin{claim}\label{cl:tree-loc3}
There exists  and  such that 
contains a path between  and .
\end{claim}
\begin{proof} Let any . Consider the directed path  from  obtained by following {\em out-edges } until the {\em first
occurrence} of  a vertex  or . Since -vertices are
the only ones with no out-edge , and  is acyclic, there must exist such a
vertex . Observe that  for
all ; recall that s (resp. s) are the connected components in  (resp. ). So
.
Suppose that vertex , then  which implies
 since path  leaves . So we
have  and  is a path from  to .
\end{proof}

Consider  and  as given Claim~\ref{cl:tree-loc3}. If  then the component  of
 is already connected to . Otherwise by Claim~\ref{cl:tree-loc2} we have ; in this
case we add edge  to  which connects component  to . Now using Claim~\ref{cl:tree-loc3}, .\footnote{This is the only place in the proof where we use uniformity in the metrics for -median and MST.}
In either case,  is connected to  in , and cost of  increases by at most .

We apply the above argument to {\em every minimal}  with .
The increase in cost of  due to each such  is at most . Since such minimal sets s are
disjoint, we have . Clearly  connects each -vertex to .
\end{proof}
\medskip

Using this lemma for each  weighted by  (from Theorem~\ref{th:GT-swap}) and adding,

Above~\eqref{eq:kmf-ls1} is by Lemma~\ref{lem:kmf-main},~\eqref{eq:kmf-ls2} is by interchanging summations using the
fact that  (for all ) from Claim~\ref{cl:kmf-Ea}. The first term in~\eqref{eq:kmf-ls3} uses
the property in Theorem~\ref{th:GT-swap} that each  is dropped (i.e. ) to extent at most ;
the second term uses the property in Theorem~\ref{th:GT-swap} that each  is added to extent one in \swap
and Claim~\ref{cl:kmf-Ea}. Finally~\eqref{eq:kmf-ls4} is by Claim~\ref{cl:kmf-m*}.


Adding the inequality  from Claim~\ref{cl:kmf-m'} yields:
 since  and  partition the true edges . Thus we obtain
Inequality~\eqref{eq:tree-swap}.















\section{Non-uniform  median forest}\label{sec:non-unif-kmf}
In this section we study the following extension of  median forest. There is a set of vertices  with weights
, two metrics  and  defined on , and a bound . The goal is to find  with 
minimizing . Here  is a minimum
spanning tree in the graph obtained by contracting  to a single vertex under metric . The difference from the
\kmf problem is that the cost functions for the -median and MST parts in the objective are different.

It is natural to consider the local search algorithm in this setting as well, since local search achieves good
approximations for both -median and MST. However the next lemma shows that the locality gap is unbounded even if we
allow multiple swaps. The example is similar to the locality gap in~\cite{KKNSS11}.
\begin{lemma}
The locality gap of non-uniform \kmf with multi-swaps is unbounded.
\end{lemma}
\begin{proof}
Fix values . Let , so . Define vertex-weights
as follows:  and all other vertices have weight . The metric  for the -median part is:

The second metric  for the MST part of the objective is:

Observe that for any  with , we have  iff  for all
. So the non-uniform \kmf objective is smaller than  only if .

We claim that the optimal value is at most one. Consider the solution . It is clear that
. Moreover,  with vertex  being the only contributor.

We now claim that the solution  is locally optimal under even -swaps. First, observe that
 and  with vertex  being the only contributor. So  has
objective value of . Secondly, notice that every solution  obtained by some -swap of  has either
MST-objective of  or median-objective of . Thus  is a local optimum and the locality gap is .
\end{proof}

\medskip

We remark that the near-optimality proof of local search in the previous section only requires the following
consistency property between the two metrics: for any pair  of edges . In spite of
the large locality gap, we show that non-uniform \kmf admits a constant factor approximation algorithm via an LP
approach.

\def\spp{\mathbb{SP}}
\newcommand{\pri}{{\mathcal{P}}}
\newcommand{\lpo}{\ensuremath{\mathsf{LP_{med}}\xspace}}
\newcommand{\LP}{\ensuremath{\mathsf{LP}\xspace}}
\newcommand{\I}{\ensuremath{{\mathcal{I}}}\xspace}
\newcommand{\J}{\ensuremath{{\mathcal{M}}}\xspace}

\paragraph{The algorithm.} We make use of the following natural LP relaxation for non-uniform \kmf. The variables  denote the probability
of locating a depot at ;  denotes the extent to which vertex  is connected to a depot at  (for the
-median part); and  denotes the extent to which edge  is used in the MST part of the objective. Also
 is the set of all edges in the metric. Define  to be the complete graph on vertices 
(for a new vertex ) with edges .

Above  denotes the spanning tree polytope of graph , which admits a linear description in terms of its edge
variables; see eg.~\cite{Schr-book}. Also  corresponds to the fractional spanning tree in  with
values  on edges  and value  on each edge . It can be checked directly that this is a valid
relaxation of non-uniform \kmf. Moreover this LP can be solved exactly in polynomial time to obtain solution
 using the Ellipsoid algorithm.

We now describe the rounding procedure. Let \I denote the instance of \kmf and  denote the median part of the optimal LP solution. Apply Stage I of the rounding algorithm
in~\cite{KKNSS11} to modify variables  to  (here  and  remain unchanged), with the
following properties:
\begin{OneLiners}
 \item Set  of representatives with weights  for each , which defines a new instance \J of non-uniform \kmf (the weights of
 vertices in  are zero).
 \item Any solution to the new instance \J with objective  is a solution to the original instance \I having
 objective at most .
 \item  is feasible for .
 \item Disjoint collection of subsets  with  for all .
 \item Collection of pseudoroots  with each representative in at
most one pseudoroot.
 \item Map  where  lies in a pseudoroot for each .
 \item Each  is connected (under ) only to .
 \item .
\end{OneLiners}

\medskip
Now apply the LP reformulation from Stage II in~\cite{KKNSS11} to eliminate -variables in \LP, using the above
structure of , and obtain: \begin{small}\end{small}

Based on the above properties, it follows that  is a feasible solution to  with objective
at most , i.e. at most four times the optimal value of . The advantage of the new LP
is:
\begin{lemma}
Any basic feasible solution to  is integral.
\end{lemma}
\begin{proof}
Let  denote any basic feasible solution. The constraints from  \eqref{nlp:1}-\eqref{nlp:3} define a laminar
family on just  variables. By a standard uncrossing argument, we can choose a maximum linearly independent set of
tight rank constraints in~\eqref{nlp:4} to be a chain on  variables. Thus a maximum linearly independent set of
tight constraints in  can be described as the intersection of two laminar families-- this is always
a totally unimodular matrix, and hence  must be integral.
\end{proof}

 can be solved exactly in polynomial time to obtain an extreme point solution using the Ellipsoid
algorithm and the approach in Jain~\cite{J01}; by the above lemma this solution is integral. Finally using Lemma 3.3
in~\cite{KKNSS11}, any integral solution to  of value  is also a valid solution to the \kmf
instance \J of value at most . Altogether we obtain an integral solution  to \J of value at most 12
times the optimum of . Combined with the relation between instances \I and \J, we have  is a valid
solution to \I of objective at most 16 times the optimum of \I, thereby proving Theorem~\ref{th:gen-kmed-forest}.




\bibliography{lvrp}
\bibliographystyle{plain}

\appendix

\section{Example comparing -median, -tree and \kmf}\label{app:example}
We give an example which shows that near-optimal solutions to the -median, -tree and \kmf problems can be very
far from each other. This implies that an approximation algorithm for \kmf must simultaneously take into account both
the median and tree parts of its objective. (For eg. we cannot merely solve -median and -tree separately and take
the better of those solutions.)

The underlying metric consists of six vertices . Let  be a parameter that
will be set to be arbitrarily large. The distance between any  and  (for all ) is infinite;
, ; and , . The weights
of vertices are  and . The bound  and parameter 
for the \kmf problem. Let ,  and  denote solutions that are -approximately optimal
for the -median, -tree and \kmf objectives respectively. We claim that ,  and  are
mutually disjoint.

It can be checked directly that the optimal -median value is . Moreover the only
solution of value  is ; so  consists of just this solution.

The optimal -tree value is . For any solution  (i.e. having value
), we must have ,  and . So  consists
of these 4 solutions.

For the \kmf objective it can be seen that the optimal value is ; from
the solutions  and . Moreover, any other solution has value ;
so  consists of the above two solutions. Clearly ,  and  are disjoint.



\end{document}
