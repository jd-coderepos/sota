\section{Experimental Evaluation}
\label{sec:evaluation}
We perform extensive experiments on a controlled testbed, in order to evaluate 
the properties of the proposed protocol. 

\subsection{Methodology}
\label{sec:methodology}
All of our experiments are performed on the Planetlab experimental 
platform~\cite{planetlab04}, which utilizes the nodes (machines) located across 
the globe.
We execute all the experiments consecutively in time on the same set of 
nodes. Unless otherwise specified, the default implementations
of leecher and seed in regular BitTorrent systems are deployed. 

The upload capacities of the nodes are artificially set according to the 
bandwidth distribution of typical BitTorrent leechers~\cite{piatek07}.
The distribution was estimated based on empirical measurements of BitTorrent 
swarms including more than 300,000 unique BitTorrent IPs.
Since several nodes are incapable to match the target upload capacities 
determined by the bandwidth distribution, 
we scale the upload capacity and other relevant experimental parameters such 
as file size by 1/20th. 
However, we have not set limitations on download bandwidth.

All peers begin the download process simultaneously, which emulates a flash 
crowd scenario. 
The initial seeds have stayed connected through out the entire experiment.
To provide synthetic churn with constant capacity, leechers disconnect 
immediately after completion of downloading the entire video file, and
reconnect as new comers immediately while requesting the entire video file 
again. This enables our experiments to have the same upload bandwidth 
distribution for the duration of the experiment.

Unless otherwise specified, our experiments host 104 Planetlab nodes, 100 
leechers and 4 seeds with a combined capacity of 128 KB/s, serving a 99 MB video 
file. 

\subsection{Experiment Results: Single RL Leecher in a Network}
\begin{figure}[t]
\centering
\includegraphics[width=0.8\columnwidth]{download_time_single.eps}
\caption{Leecher download time} 
\label{fig:single}
\end{figure}

We start with the experiment where only a single leecher adopts the 
\emph{RL-enhanced} protocol, while the rest of the leechers in the network run with 
the regular BitTorrent, and there are no \FR s in the network (note that this 
is a common scenario that was tested by other proposed protocols such 
as~\cite{levin08, piatek07}). 
Fig.~\ref{fig:single} compares the download time of a single 
leecher, while adopting the \emph{RL-enhanced} protocol and the regular BitTorrent 
protocol as a function of the leecher's upload capacity over 7 trials.

In Fig.~\ref{fig:single}, as in~\cite{mcgill78} separate boxplots are depicted 
for the different scenarios. The top and the bottom of the boxes represent the 
75th and the 25th percentile sample of download time, respectively, over all 7 
runs of the experiments. The markers inside the boxes represent the median, 
while the vertical lines extending above and below the boxes represent the 
maximum and minimum of samples of download time within the ranges of 1.5 time 
the box height from the box boarder. Outliers are marked individually with 
``" mark.

The results in Fig.~\ref{fig:single} provide several insights into the 
operation of our RL-based proposed protocol.
High and Low capacity leechers benefit from the \emph{RL-enhanced} with 12\%-27\%
improvement of their download time performance as indicated by the median.
This improvement provides leechers with an incentive to adopt the proposed 
protocol. Moreover, the RL-based strategy does not simply improve 
performance; it also provides more consistent performance across multiple 
trials. By selecting to unchoke peers based on historical behavior information, 
our proposed protocol avoids the randomization present in the regular BitTorrent 
tit-for-tat and optimistic unchoke implementations, which cause to unstable 
peer selections and results in slow convergence.\\
\\{\bfseries Peer Selection Mechanism Stability}
\begin{figure}[t]
\centering
\includegraphics[width=0.8\columnwidth]{Comparing_dynamic.eps}
\caption{Peer selection mechanism dynamics} 
\label{fig:dynamics}
\end{figure}
We further study the peer selection mechanism stability.
The stability of the peer selection mechanism affects directly the performance 
of the system since once a peer starts to upload to another peer it takes time 
till the peer reaches its full capacity. In the BitTorrent 
protocol~\cite{cohen03} the author suggests allowing 30 seconds for a peer to 
reach its full capacity. 
Thus, a system that has a high fluctuation in peer selection will have many 
occurrences of peers that do not reach their full capacity.

We compare the peer selection fluctuations of the two protocols.
A stable peer selection mechanism should minimize the peer selection 
fluctuations. We measured peer selection fluctuations by comparing the peer 
selection decisions during two consecutive rechoke periods and measuring the 
difference between the two decisions, e.g., replacing an unchoked peer by a 
different peer counts as one change.
Fig.~\ref{fig:dynamics} indicates the average number of peer selection 
changes as a function of time (rechoke period units) for a single peer. 
It shows that the average number of peer selection changes is lower in the 
\emph{RL-enhanced} network for the majority of the time, with an average of 2.1 
changes in the regular \BT~network as compared to 0.9 average changes in the 
\emph{RL-enhanced} network. Thus, the \emph{RL-enhanced} peer selection 
mechanism is more stable than the peer selection mechanism in the regular \BT, reducing the peer selection fluctuations by an average of 57\%.

Note that the optimistic unchoke mechanism contributes about 1 change 
every 3 rechoke periods, thus contributing an average change of about 
 per time unit in the regular \BT~network. Therefore, the decrease 
in optimistic unchokes is not the main reason for the stability improvement of 
the peer selection mechanism. 
Instead, replacing the tit-for-tat mechanism, which relies on short-term 
history of associated peers with the RL-based mechanism that 
relies on a long history and performs foresighted unchoking decisions is the 
main contributor for this stability.

\subsection{Experiment Results: Performance of Leechers in Network without
Free-Riders}


We compare a system consisting of all leechers adopting the regular BitTorrent 
protocol, to a system consisting of all leechers running in \emph{RL-enhanced} mode, adopting the RL-based strategy.  
In this section, we assume that there are no free-riders in the network. 
Note that this experiment hosted only 50 leechers. 
Fig.~\ref{fig:DLtime} shows the download completion time of leechers.
\begin{figure}[t]
\centering
\includegraphics[width=0.8\columnwidth]{DLtime.eps}
\caption{Download completion time for leechers.} 
\label{fig:DLtime}
\end{figure}
For each group of leechers having the same upload capacity, separate boxplots 
are depicted for the different scenarios.


The results show the clear performance difference among high-capacity leechers, 
which are the fastest 20\% leechers, and low-capacity leechers, which are the 
slowest 80\% leechers. 
High-capacity leechers can significantly improve their download
completion time -- leechers having the upload capacity of at least 18kB/sec 
improve their download completion time by up to 33\% in median. 
Unlike in the regular BitTorrent system, where leechers determine their peer 
selection decisions based on the myopic tit-for-tat that uses only the last 
reciprocation history, the \emph{RL-enhanced} leechers determine their peer 
selection decisions based on the long term history. This enables the leechers 
to estimate the behaviors of their associated peers more accurately. Moreover, 
since part of the peer selection decisions is randomly determined in the 
regular BitTorrent, there is a high probability that high capacity leechers 
need to reciprocate with the low-capacity leechers~\cite{piatek07}. 
However, the randomly determined peer selection decisions are significantly 
reduced in the proposed approach, as the random decisions are taken only in the 
initialization phase or in order to collect the reciprocation history of newly 
joined peers. 
As a result, the high capacity leechers increase theie probability to 
reciprocate resources with other high capacity leechers.

This is confirmed in the results of Fig.~\ref{fig:Unchoke_dist}, which shows 
the unchoking percentage among the  high capacity leechers, 
comparing the two different systems.
\begin{figure}[t]
\centering
\includegraphics[width=0.8\columnwidth]{Unchoke_dist.eps}
\caption{Unchokes among the  fastest peers.} 
\label{fig:Unchoke_dist}
\end{figure}
It is clearly observed that the collaboration among high capacity leechers 
improves when leechers adopt the RL-based strategy. 
Thus, we can conclude that the RL-based strategy improves the 
incentive mechanisms in BitTorrent networks: as a leecher contributes more to 
the network, it achieves higher download rate. 

Recent studies~\cite{piatek07,Buddies,bharambe06,guo05} show that 
the regular BitTorrent protocol suffers from unfairness particularly for high 
capacity leechers.
In Fig.~\ref{fig:dl_vs_ul}, we compare the upload rates and the average 
download rates of the leechers. The ratio of these values can indicate the 
degree of fairness in the system.
\begin{figure}[t]
\centering
\includegraphics[width=0.8\columnwidth]{dl_vs_ul.eps}
\caption{Download rates versus upload rates.} 
\label{fig:dl_vs_ul}
\end{figure}
The results in Fig.~\ref{fig:dl_vs_ul} show that fairness is improved in 
the \emph{RL-enhanced} network, since high-capacity leechers increased their 
download rate getting closer to their upload rate, in spite of the restriction 
of limited seeds' upload rate. On the other hand, in the \emph{RL-enhanced} network, 
the download rates of low-capacity leechers decrease, getting close to their 
upload rates by at most , compared to the regular BitTorrent system.
However, all the peers that are slowed down by the RL-based 
strategy still  download faster than their upload rate. \\


\subsection{Experiment Results: Performance of Leechers in Network with
Free-Riders}

\begin{figure}[t]
\centering
\includegraphics[width=0.8\columnwidth]{FR_DLtime.eps}
\caption{Download completion time for free-riders.} 
\label{fig:FR_DLtime}
\end{figure}

In this section, we investigate how effectively the proposed protocol can 
prevent selfish behaviors such as free-ridings. Note that the RL-based strategy shows a similar performance for the leechers that upload 
their content in a network that includes free-riders (i.e., shows the 
improved fairness, etc.). Hence, in this section, our focus is on studying how 
the free-riders are punished due to their selfish behaviors. 
Fig.~\ref{fig:FR_DLtime} shows the time that the free-riders complete 
downloading 99MB video file in a network consisting of 50 contributing 
leechers, and increasing number of free-riders (i.e., 5, 10, and 15 
free-riders). It compares the results of the \emph{RL-enhanced} network to the 
regular BitTorrent network. Fig.~\ref{fig:FR_DLtime} confirms that in the 
\emph{RL-enhanced} network the leechers are able to effectively penalize the 
free-riders, as it takes longer time for the free-riders to complete their 
downloads (requires 8\%-20\% more time as measured by the median, in 
comparison to the regular BitTorrent protocol).

\begin{figure}[t]
\centering
\includegraphics[width=0.8\columnwidth]{percentage_fr.eps}
\caption{Percentage of free-riders' downloads from contributing leechers.} 
\label{fig:upload_to_fr}
\end{figure}

The \emph{RL-enhanced} leechers can efficiently capture the selfish behaviors of 
the free-riders. Thus, they unchoke the free-riders with a significantly lower 
probability. Hence, the free-riders can download their content mainly from 
seeds and not from the leechers. The results shown in 
Fig.~\ref{fig:upload_to_fr} also confirm that the leechers in the regular 
BitTorrent network upload approximately 2.8-3.7 times more data to the 
free-riders compared to the \emph{RL-enhanced} network.
This also shows that the \emph{RL-enhanced} networks are more robust to the selfish 
behaviors of peers than the networks operating with the regular BitTorrent 
protocol. For example, in the network with 15 free-riders, the leechers in the 
regular BitTorrent systems upload 4.5\% of their total upload capacity to 
free-riders, while they only upload 1.6\% of their total upload capacity in the 
\emph{RL-enhanced} network. Thus reducing by 64\% their upload capacity to \FR s.

Therefore, our experiment results confirm that the RL-based strategy provides incentives for adoption because it improves the peer's download rate, improves 
the stability of the peer selection mechanism, improves collaboration among 
high capacity peers, improves fairness in the system, and discourages 
non-cooperative behaviors such as free-riding.





