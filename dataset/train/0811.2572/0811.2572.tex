\documentclass[12pt,oneside]{article}

\usepackage[margin=2cm]{geometry}

\usepackage{cite}
\usepackage{amsmath}
\usepackage{amsthm}
\usepackage{amssymb}
\usepackage{subfigure}
\usepackage{graphicx}
\usepackage{color}
\newcommand{\rmj}[1]{{\color{red} #1}}
\newcommand{\comrj}[1]{\begin{tt}[RJ: #1]\end{tt}}

\newtheorem{lemma}{Lemma}
\newtheorem{theorem}{Theorem}
\newtheorem{definition}{Definition}
\newtheorem{corollary}{Corollary}

\newcommand{\partsort}{{\sc Partial Order Production}}
\newcommand{\psort}{{\sc Partial Sorting}}
\newcommand{\topsort}{{\sc Topological Sorting}}
\newcommand{\sortwpi}{{\sc Sorting with Partial Information}}
\newcommand{\STAB}{\mathrm{STAB}}
\newcommand{\GAP}{\mathrm{GAP}}



\author{Jean Cardinal, Samuel Fiorini, Gwena\"el Joret\thanks{Universit\'e Libre de Bruxelles (ULB), Brussels, Belgium. {E-mail: \tt\small \{jcardin,sfiorini,gjoret\}@ulb.ac.be}
},\\ Rapha\"el M. Jungers\thanks{Universit\'e catholique de Louvain (UCL), Louvain-la-Neuve, Belgium. {E-mail: \tt\small raphael.jungers@uclouvain.be}  }, J. Ian Munro\thanks{University of Waterloo, Waterloo, Ontario, Canada. {E-mail: \tt\small imunro@uwaterloo.ca}}
}

\title{An Efficient Algorithm for Partial Order Production\footnote{This work was supported by the ``Actions de Recherche Concert\'ees'' (ARC) fund of the ``Communaut\'e fran\c{c}aise de Belgique'', NSERC of Canada, and the Canada Research Chairs Programme. G.J.\ and R.J.\ are Postdoctoral Researchers of the ``Fonds National de la Recherche Scientifique'' (F.R.S.--FNRS). A preliminary version of the work appeared in \cite{CFJJM09-stoc}.}}
\date{}
\begin{document}

\maketitle

\begin{abstract} We consider the problem of {\em partial order production}: arrange the elements of an unknown totally ordered set  into a target partially ordered set , by comparing a minimum number of pairs in . Special cases include sorting by comparisons, selection, multiple selection, and heap construction.

We give an algorithm performing  comparisons in the worst case. Here,  denotes the size of the ground sets, and  denotes a natural information-theoretic lower bound on the number of comparisons needed to produce the target partial order.

Our approach is to replace the target partial order by a weak order (that is, a partial order with a layered structure) extending it, without increasing the information theoretic lower bound too much. We then solve the problem by applying an efficient multiple selection algorithm. The overall complexity of our algorithm is polynomial. This answers a question of Yao (SIAM J.\ Comput.\ 18, 1989).

We base our analysis on the entropy of the target partial order, a quantity that can be efficiently computed and provides a good estimate of the information-theoretic lower bound.
\end{abstract}
\textbf{Keywords: }{Partial order, graph entropy}

\section{Introduction}

We consider the \partsort\ problem:\smallskip

{\it Given a set  partially ordered by a known partial order  and a set  totally ordered by an unknown linear order , find a permutation  of  such that , by asking questions of the form: ``is ?".}\smallskip

The \partsort\ problem generalizes many fundamental problems (see Figure~\ref{fig-special}), corresponding to specific families of posets . It amounts to sorting by comparisons when  is a chain. The selection~\cite{H61} and multiple selection~\cite{C71} problems are special cases in which  is a weak order\footnote{Most of the terms that are not defined in the introduction are defined in Sections \ref{sec-perfect-graphs} and \ref{sec-algorithm}}, that is, has a layered structure (with  iff  is on a lower layer than ). When the Hasse diagram of  is a complete binary tree, the problem boils down to heap construction~\cite{CC92}.

\begin{figure}
\begin{center}
\subfigure[Sorting]{\hspace{1cm}\includegraphics[scale=.15, angle=-90]{sorting.pdf}\hspace{1cm}}\hskip .5cm
\subfigure[Selection]{\hspace{1cm}\includegraphics[scale=.15, angle=-90]{selection.pdf}\hspace{1cm}}\hskip .5cm
\subfigure[Multiple selection]{\hspace{1cm}\includegraphics[scale=.15, angle=-90]{multiselection.pdf}\hspace{1cm}}\hskip .5cm
\subfigure[Heap construction]{\hspace{1cm}\includegraphics[scale=.15, angle=-90]{heap.pdf}\hspace{1cm}}
\end{center}
\vspace{-.5cm}
\caption{\label{fig-special}Special cases of the \partsort\ problem.}
\end{figure}

We assume that the target poset  is part of the input and represented by its Hasse diagram. Hence the size of the input can be , whereas sorting the  elements of  takes  time. In other words, reading the input could take more time than necessary to solve problem, provided a topological sorting of  is known.

To cope with this paradoxical situation, we consider algorithms that proceed in two phases: a {\em preprocessing phase\/} during which an ordering strategy is determined (for instance, in the form of a decision tree, or any more efficient description, if possible), on the basis of the structure of , and an {\em ordering phase\/} during which all comparisons between elements of  are performed. Accordingly, we distinguish the {\em preprocessing complexity\/} and the {\em ordering complexity\/} of the algorithm, the latter being essentially proportional to the number of comparisons performed.

As noted before, we expect the overall complexity of the algorithm to be dominated by its preprocessing complexity. Thus it is desirable to perform the preprocessing phase only once, and then use the resulting ordering strategy on several data sets.



\paragraph{Lower bound on the number of comparisons}
We denote by  the number of linear extensions of the target poset . Feasible permutations  are in one-to-one correspondence with the linear extensions of , thus the number of feasible permutations is exactly . On the other hand, the total number of permutations is . We have thus the following information-theoretic lower bound (logarithms are base 2):

\begin{theorem}[\hspace{-.01em}\cite{S76, A81, Y89}]
Any algorithm solving the \partsort\ problem for an -element poset P requires

comparisons between elements of  in the worst case and on average.
\end{theorem}

Note that we can assume without loss of generality that  is connected, hence we also have a lower bound of .

\paragraph{Problem history and contribution}
The \partsort\ problem was first proposed in 1976 by Sch\"onhage~\cite{S76}. It was studied five years after by Aigner~\cite{A81}. Another four years passed and the problem simultaneously appeared in two survey papers: one by Saks~\cite{S85} and the other by Bollob\'as and Hell~\cite{BH85}. In his survey, Saks conjectured that the \partsort\ problem can be solved by performing  comparisons in the worst case.

Four years later, in 1989, Yao proved Saks' conjecture~\cite{Y89}. He gave an algorithm solving the \partsort\ problem in at most  comparisons, for some constants  and . However, the preprocessing phase of Yao's algorithm seems difficult to implement efficiently. In fact, in the last section of his paper~\cite{Y89}, Yao asked whether, assuming  is part of the input (as is the case here), there exists a polynomial-time algorithm for the problem that performs  comparisons.

Our main contribution is an algorithm that solves the \partsort\ problem and performs at most  comparisons in the worst case. The preprocessing complexity of our algorithm is . Hence we answer affirmatively the question of Yao \cite{Y89} mentioned above. Moreover, we also significantly improve the ordering complexity, since Yao's constants  and  are quite large.

Further references, focussing mainly on lower bounds for the problem and generalizations of it include Culberson and Rawlins \cite{CR88}, Chen \cite{C94} and Carlsson and Chen \cite{CC94}.

\paragraph{Main ideas underlying our approach} We reduce the \partsort\ problem to the multiple selection problem. Instead of solving the problem for the given target poset  we solve it for a larger (more constrained) poset that has a simpler structure, namely, a weak order  extending  (a weak order is a set of antichains with a total ordering between these antichains). This approach works because, as we show below, it is possible to find such a weak order  whose corresponding information-theoretic lower bound  is not too large compared to that of .

Unfortunately, computing  exactly is -hard, because computing the number of linear extensions of a poset is -complete, a result due to Brightwell and Winkler \cite{BW91}. The analysis is made possible because there exists a quantity, depending on the structure of the target poset, that can be computed in polynomial time and provides a good estimate of . This quantity is , where  denotes the entropy of the considered target poset. (The entropy of a graph is defined in the next section, and the entropy of a poset is defined as the entropy of its comparability graph.) It was K\"orner who introduced the notion of the entropy of a graph, in the context of source coding~\cite{K73}. The idea of estimating an information-theoretic lower bound by means of the entropy of a poset was used before by Kahn and Kim in their inspiring work on sorting with partial information \cite{KK95}, see below.

\paragraph{Related problems}
In 1971 Chambers~\cite{C71} proposed an algorithm for the \psort\ problem, defined as follows: given a vector  of  numbers and a set  of indices, rearrange the elements of  so that for every , all elements with indices  are smaller or equal to , and elements with indices  are bigger or equal to . For the indices , the elements  in the rearranged vector have rank exactly , hence this problem is also called {\em multiple selection}. The \psort\ problem is a special case of \partsort\ in which the partial order is also a weak order.

The algorithm proposed by Chambers is similar to Hoare's ``find" algorithm~\cite{H61}, or QuickSelect. It has been refined and analyzed by Dobkin and Munro~\cite{DM81}, Panholzer~\cite{Pa03}, Prodinger~\cite{Pr03}, and Kaligosi, Mehlhorn, Munro, and Sanders~\cite{KMMS05}. For our purposes, the key result is that of Kaligosi {\em et al.\/}~\cite{KMMS05} in which it is shown that multiple selection can be done within a lower order term of the information theoretic lower bound, plus a linear term.\medskip

Another generalization of the sorting problem, called \sortwpi, was studied by Kahn and Kim~\cite{KK95}:\smallskip

{\it Given an unknown linear order  on a set , together with a subset  of the relations  forming a partial order, determine the complete linear order  by asking questions of the form: ``is ?".}\smallskip

This problem is equivalent to sorting by comparisons if  is empty. The information-theoretic lower bound for that problem is , where . The problem is complementary to the \partsort\ problem in the sense that sorting by comparisons can be achieved by first solving a \partsort\ problem, then solving the \sortwpi\ problem on the output.

A proof that there exists a decision tree achieving the lower bound up to a constant factor has been known for some time (see in particular Kahn and Saks~\cite{KS84j}). This is related to the -- conjecture of Fredman~\cite{F76} and Linial~\cite{L84}. Kahn and Kim~\cite{KK95} provided a polynomial time algorithm that finds the actual comparisons. They show that choosing the comparison that causes the entropy of  to increase the most leads to a decision tree that is near-optimal in the above sense.

\paragraph{Overview}
In Section~\ref{sec-perfect-graphs}, we study the entropy of perfect graphs. We show that it is possible to approximate the entropy of a perfect graph  using a simple greedy coloring algorithm. More precisely, we prove that any such approximation is at most , where  denotes the entropy of graph .\smallskip

Section~\ref{sec-algorithm} explains how to apply this result to solve the \partsort\ problem algorithmically. We begin the section by remarking that entropy is bound to play a central role for the problem since , where  denotes the entropy of poset .

The preprocessing phase of our algorithm starts by applying the greedy coloring algorithm studied in Section~\ref{sec-perfect-graphs} to the comparability graph of .  We then modify this coloring (we ``uncross" the colors) in order to obtain an extension of  which is an interval order . Another application of the greedy coloring algorithm, this time on the comparability graph of , yields a weak order  extending . Using our result on perfect graphs, we prove that the entropy of  is not much larger than that of , that is, .

The ordering phase of the algorithm simply runs then a multiple selection algorithm based on the weak order . We use a multiple selection algorithm from Kaligosi {\it et al.}~\cite{KMMS05} that performs a number of comparisons close to the information-theoretic lower bound.

We conclude the section by proving that the preprocessing complexity of our algorithm is .\smallskip

Finally, in Section~\ref{sec-discussion}, we discuss {the number of comparisons and study} the existence of an algorithm solving the \partsort\ problem in  comparisons. We give an example showing that such an algorithm cannot always reduce the problem to the case where the target poset is a weak order. More specifically, we exhibit a family of interval orders with entropy at most , any weak order extension of which has entropy at least .

\section{Entropy of Perfect Graphs}
\label{sec-perfect-graphs}
We recall that a subset  of vertices of a graph is a \emph{stable set} (or {\em independent set}) if the vertices in  are pairwise nonadjacent. Also, a graph  is {\em perfect} if  holds for every induced
subgraph  of , where  and  denote the
clique and chromatic numbers of , respectively.

Let us recall similarly that the {\em stable set polytope\/} of an arbitrary graph  with vertex set  and order  is the -dimensional polytope

where  is the characteristic vector of the subset , assigning the value  to every vertex in , and  to the others. The {\em entropy\/} of  is defined as (see~\cite{K73,CKLMS90})

For example, if  is the graph with  and , then  and the minimum in (\ref{def-H}) is attained for .

Note that graph entropy was originally defined with respect to a given probability distribution on . However, for our purposes we can take the uniform distribution, as in \cite{KK95}. In this case we obtain Equation (\ref{def-H}).

An upper bound on  can be found as follows: First, use the greedy coloring algorithm that removes iteratively a maximum stable set from , giving a sequence  of stable sets of . If  is perfect, this can be done in polynomial time (see, e.g., Gr\"otschel, Lov\'asz and Schrijver \cite{GLS93}). Next, let  be defined as

By definition, . We call any such point  a {\em greedy point}. The value of the objective function in the definition of  for  is . We refer to the latter quantity simply as the entropy of .
It turns out that this gives a good approximation of  when  is a perfect graph.

\begin{theorem}
\label{th-greedy-perfect-graphs}
Let  be a perfect graph on  vertices and denote
by  the entropy of an arbitrary greedy point . Then

for all , and in particular

\end{theorem}

A key tool in our proof of Theorem~\ref{th-greedy-perfect-graphs} is a min-max relation of Csisz{\'a}r, K{\"o}rner, Lov{\'a}sz, Marton, and Simonyi~\cite{CKLMS90} relating the entropy of a perfect graph  to the entropy of its complement :

\begin{theorem}[\hspace{-.01em}\cite{CKLMS90}]
\label{th-sum-log-n}
If  is a perfect graph on  vertices, then .
\end{theorem}

We now turn to the proof of Theorem~\ref{th-greedy-perfect-graphs}.

\begin{proof}[Proof of Theorem~\ref{th-greedy-perfect-graphs}]
Let  be the sequence of stable sets of  selected by the greedy
algorithm (in the order the algorithm removes them).
So  is a maximum stable set in ,  is a maximum stable set in , and so on.
The outline of the proof is as follows: We first use the sets
  to define a point , where  is the vertex set of . We then
show that   belongs to the stable set polytope of the {\em complement}  of , that
is, . Finally, we derive the desired inequality by combining the
upper bound on  implied by  with Theorem~\ref{th-sum-log-n}.

Fix . For each vertex  we let  be the unique index in 
such that . We define  by letting, for each vertex  of ,

We claim that for every stable set  of :


Write the stable set  as , where  is the th subset of  taken by the greedy algorithm during its execution.
For every , we have   and , since the greedy algorithm could have
selected the set  when it took .
More generally, for every  and , we have
. It follows in particular that we can enumerate the points of  as , , \ldots,  in such a way that

We thus have

Equation~\eqref{eq-clique-n} follows.

Two classical results on perfect graphs are that the stable set polytope is completely described
by the non-negativity and clique inequalities, that is,

(see Chv\'atal~\cite{C75}), and that the complement  of  is also a
perfect graph (Lov\'asz~\cite{L72}). Combining these two results with~\eqref{eq-clique-n}
shows that .
Using Theorem~\ref{th-sum-log-n}, we then deduce

Hence, , for all .
By choosing  if , and  otherwise, we obtain
.
\end{proof}

\section{An Algorithm for Partial Order Production}
\label{sec-algorithm}

We denote by  the comparability graph of a poset , and let . Note that a stable set in  is an antichain in , that is, a set of mutually incomparable elements. Note also that  is perfect, a basic result that is dual to Dilworth's theorem, see, e.g., \cite{G04}. The relevance of the notion of graph entropy in the context of sorting was first observed by Kahn and Kim~\cite{KK95}. Using the fact that the volume of  equals  (see Stanley~\cite{S86}), they proved the following result.

\begin{lemma}[\hspace{-.01em}\cite{KK95}]
For any poset  of order ,

\end{lemma}

When written as

the above inequalities become intuitively clear, since  is the (maximum) volume of a box contained in ,  is the volume of , and  is the volume of a simplex containing . The lemma directly implies the following equality for every poset :


We recall that a poset is said to be a {\em weak order\/} whenever its comparability graph is a complete
-partite graph, for some . Such a poset  can be partitioned into  maximal
antichains , \ldots, , the {\em layers\/} of , such that  whenever there exist indices 
and  such that ,  and . When restricted to weak orders, the \partsort\ problem
resembles the \psort\ problem, with .

Our key idea is to show that, using (twice) the greedy coloring algorithm presented in the previous section, we can efficiently extend\footnote{A poset  {\em extends\,} a poset  if
they have the same ground set  and  implies , for all .}
the given poset  to a weak order  whose entropy is close to that of .  The reason why we have to use twice the greedy algorithm is that the obtained coloration might not be "ordered" (might not represent the stable sets of a weak order).  However, we describe below how to uncross this coloring in order to extend  to an interval order without increasing too much the entropy.  We show that applying our greedy coloring to an interval order provides an "ordered" coloring, which allows us to run a second time our greedy algorithm, providing an extension which is a weak order.

We then simply run an efficient multiple selection procedure, with  as input.
We show that, because replacing  by  does not increase the entropy too much, the resulting number of comparisons is close to .

The preprocessing phase is composed of three steps, each of which can be performed in polynomial time. In the first step, we apply the greedy coloring procedure to , to obtain a greedy point . This step makes use of an auxiliary network defined from . Then, in the second step, using again the auxiliary network, we extend  to an interval order  whose entropy is not larger than that of . This allows us to ``uncross" the antichains used in . (An alternative way of obtaining the interval order  is to apply Kahn and Kim's \cite{KK95} laminar decomposition lemma to .) Finally, in the third step, we apply the greedy coloring procedure again, this time on , to obtain the weak order . See Figure \ref{fig-weakextension} for an illustration of steps 1 and 2.

\begin{figure}
\begin{center}
\subfigure[\label{fig-coloring} Possible greedy point .]{\includegraphics[scale=.25, angle=-90]{coloring.pdf}}\hskip 2.5cm
\subfigure[\label{fig-network} Network  and potential .]{\includegraphics[scale=.25, angle=-90]{network.pdf}}\\
\subfigure[\label{fig-intervals} Interval representation of .]{\includegraphics[scale=.25, angle=-90]{intervals.pdf}}\hskip 2cm
\subfigure[\label{fig-intervalgraph} Interval order .]{\includegraphics[scale=.25, angle=-90]{intervalgraph.pdf}}
\end{center}
\caption{\label{fig-weakextension} Obtaining an interval order extension of the poset .}
\end{figure}

\paragraph{Auxiliary network}
Let  be any poset. We say that  is {\em covered\/} by  in  if ,  and  implies  or . The {\em Hasse diagram\/} of  is the network with node set , and arc set  is covered by  in . An element  of  is {\em minimal\/} (resp.\ {\em maximal\/}) if  (resp.\ ) implies .

We construct a network  from the Hasse diagram of  by first uncontracting each element  to an arc  and then adjoining a source node  sending an arc to each minimal element, and a sink node  receiving an arc from each maximal element. The resulting network has node set

and arc set
.5ex]
&&\{(v^+,w^-) : v \textrm{ is covered by } w \textrm{ in } P\} \cup \{(v^+,t) : v \in V,\ v \textrm{ maximal in } P\}.
\end{array}

\STAB(G) = \{x \in \mathbb{R}^V_+ : \sum_{v \in K} x_v \le 1\quad \forall K \textrm{ clique in } G\}.

\sum_{v \in C} x_v &= & (y_{v_1^+}-y_{v_1^-}) + \cdots + (y_{v_c^+}-y_{v_c^-})\\
&\le &(y_{v_1^-}-y_{s}) + (y_{v_1^+}-y_{v_1^-}) +
(y_{v_2^-} - y_{v_1^+}) + \cdots + (y_{v_c^+}-y_{v_c^-}) + (y_{t} - y_{v_c^+})\
It follows that .

For necessity, consider . For , we let  be the maximum total weight of a chain of  whose maximum with respect to  is ,
when each vertex  is given the weight , and . Then we let  and . As is easily verified,  is a potential for .
\end{proof}

It follows that  is the optimum value of the following convex minimization problem with a polynomial number of variables and constraints:
.5ex]
&                  &y_{p}   &\leqslant &y_{q} &\forall (p,q) \in A(D)\\
&                  &y_{s} &= &0\\
&                  &y_{t}  &= &1.
\end{array}

H(W) \le \frac{1}{1-\delta} \left(H(P) + 2\log \frac{1}{\delta} + 2\right)

H(W) \le H(P) + 2\log (H(P)+1) + O(1).

H(P) &\geq (1-\delta')H(I) - \log(1/\delta') \\
&\geq(1-\delta')\big( (1-\delta')H(W) - \log(1/\delta') \big) - \log(1/\delta') \\
&\geq (1-\delta) H(W) - 2 \log(1/\delta) - 2.

\label{ub-exp}
ITLB + o(ITLB) + O(n)

B  & =    & n H(W) + O(n) \hspace{7.2em} {\textrm{ (from\ Eqn.~(\ref{eq-H}))}}\\
   & \leq & n H(P) + 2 n \log (H(P)+1) + O(n) \hspace{1em} \textrm{ (from Lemma~\ref{lem-greedy})} \\
   & = & ITLB + 2 n \log \left( \frac{ITLB}{n} + 1\right) + O(n)\hspace{.9em} {\textrm{ (from\ Eqn.~(\ref{eq-H}))}}  \\
   & = & ITLB + o(ITLB) + O(n).

A_{i} := \{v\in V \mid (v^{-}, v^{+}) \in \delta^{+}(Y),\ \phi_{(v^{-}, v^{+})} = 1\}

x^{(k)} := \sum_{i=1}^{2^{k}-1}  \frac{1}{2^{k}-1}   \chi^{S_{i}},

H(I_{k})
\leq -\frac 1{k2^{k-1}}\sum_{\ell=0}^{k-1} 2^{k-1}\log{\frac{ 2^{\ell}}{2^{k}-1} }=\log{(2^k-1)}-\frac{k-1}{2}\leq (k+1)/2.

-\sum_{i=1}^{\ell} \frac{|C_{i}|}{n} \log \frac{|C_{i}|}{n}.

\tilde  g_{k} &= - \sum_{i=1}^{k} 2^{i-1} \cdot \frac{2^{k-i}}{k2^{k-1}} \log \frac{2^{k-i}}{k2^{k-1}}\\
&= \frac{1}{k} \sum_{i=1}^{k} \log \frac{k2^{k-1}}{2^{k-i}} \\
&= \frac{1}{k} \sum_{i=1}^{k} (\log k + (i-1))\\
&= \log k + \frac{k-1}{2},

H(W) - H(I_{k})
&\geq H_{\chi}(\bar G_{k}) - H(I_{k})  \\
&\geq \left(\frac{k-1}{2} + \log k - \log e \right) - \frac{k+1}{2}\\
&= \log k - \log e - 1 \\\
&= \Omega(\log \log n),

as claimed.
\end{proof}

\paragraph{Acknowledgments}
The authors wish to thank S\'ebastien Collette, Fran\c{c}ois Glineur and Stefan Langerman for useful discussions, and the anonymous referees for their comments on an earlier version of the paper.

\bibliographystyle{plain}

\begin{thebibliography}{10}

\bibitem{A81}
M.~Aigner.
\newblock Producing posets.
\newblock {\em Discrete Math.}, 35:1--15, 1981.

\bibitem{BH85}
B.~Bollob{\'a}s and P.~Hell.
\newblock Sorting and graphs.
\newblock In {\em Graphs and order, Banff, Alta., 1984}, volume 147 of {\em
  NATO Adv. Sci. Inst. Ser. C Math. Phys. Sci.}, pages 169--184, Dordrecht,
  1985. Reidel.

\bibitem{BW91}
G.~Brightwell and P.~Winkler.
\newblock Counting linear extensions.
\newblock {\em Order}, 8(3):225--242, 1991.

\bibitem{CFJ08-ALGO}
J.~Cardinal, S.~Fiorini, and G.~Joret.
\newblock Tight results on minimum entropy set cover.
\newblock {\em Algorithmica}, 51(1):49--60, 2008.

\bibitem{CFJJM09-stoc}
J.~Cardinal, S.~Fiorini, G.~Joret, R.~M.~Jungers, and J.~I.~Munro.
\newblock An Efficient Algorithm for Partial Order Production.
\newblock To appear in {\em Proceedings of STOC 09, Bethesda (Maryland), United States}, 2009.

\bibitem{CC92}
S.~Carlsson and J.~Chen.
\newblock The complexity of heaps.
\newblock In {\em Proceedings of the third annual ACM-SIAM symposium on
  discrete algorithms (SODA '92), Orlando (Florida), United States}, pages
  393--402, Philadelphia, PA, USA, 1992. Society for Industrial and Applied
  Mathematics.

\bibitem{CC94}
S.~Carlsson and J.~Chen.
\newblock Some lower bounds for comparison-based algorithms.
\newblock In {\em Proc. 2nd European Sympositum on Algorithms (ESA '94),
  Utrecht, The Netherlands}, volume 855 of {\em Lecture Notes in Computer Science},
  pages 106--117. Springer-Verlag, 1994.

\bibitem{C71}
J.~M. Chambers.
\newblock Partial sorting (algorithm 410).
\newblock {\em Commun. {ACM}}, 14(5):357--358, 1971.

\bibitem{C94}
J.~Chen.
\newblock Average cost to produce partial orders.
\newblock In {\em Proc. 5th International
  Symposium on Algorithms and Computation (ISAAC '94), Beijing, P. R. China},
  volume 834 of {\em Lecture Notes in Computer Science},
  pages 155--163. Springer-Verlag, 1994.

\bibitem{C75}
V.~Chv{\'a}tal.
\newblock On certain polytopes associated with graphs.
\newblock {\em J. Combinatorial Theory Ser. B}, 18:138--154, 1975.

\bibitem{CKLMS90}
I.~Csisz{\'a}r, J.~K{\"o}rner, L.~Lov{\'a}sz, K.~Marton, and G.~Simonyi.
\newblock Entropy splitting for antiblocking corners and perfect graphs.
\newblock {\em Combinatorica}, 10(1):27--40, 1990.

\bibitem{CR88}
J.~C.~Culberson and G.~J.~E.~Rawlins.
\newblock On the comparison cost of partial orders.
\newblock Technical Report TR88-01, Department of Computing Science, University
  of Alberta, Edmonton, Alberta, Canada T6G 2E8, 1988.

\bibitem{DM81}
D.~P.~Dobkin and J.~I.~Munro.
\newblock Optimal time minimal space selection algorithms.
\newblock {\em J. ACM}, 28(3):454--461, 1981.

\bibitem{F76}
M.~L.~Fredman.
\newblock How good is the information theory bound in sorting?
\newblock {\em Theor. Comput. Sci.}, 1(4):355--361, 1976.

\bibitem{G04}
M.~C.~Golumbic.
\newblock {\em Algorithmic graph theory and perfect graphs. 2nd ed.}, volume~57 of the {\em Annals of Discrete Mathematics}.
\newblock Elsevier, Amsterdam, 2004.

\bibitem{GLS93}
M.~Gr\"otschel, L.~Lov\'asz and A.~Schrijver.
\newblock {\em Geometric algorithms and combinatorial optimization. 2nd corr. ed.}, volume 2 of {\em Algorithms and Combinatorics}.
\newblock Springer-Verlag, Berlin, 1993.

\bibitem{H61}
C.~A.~R.~Hoare.
\newblock Find (algorithm 65).
\newblock {\em Commun. {ACM}}, 4(7):321--322, 1961.

\bibitem{KK95}
J.~Kahn and J.~H.~Kim.
\newblock Entropy and sorting.
\newblock {\em J. Comput. Syst. Sci.}, 51(3):390--399, 1995.

\bibitem{KS84j}
J.~Kahn and M.~E.~Saks.
\newblock Balancing poset extensions.
\newblock {\em Order}, 1:113--126, 1984.

\bibitem{KMMS05}
K.~Kaligosi, K.~Mehlhorn, J.~I. Munro, and P.~Sanders.
\newblock Towards optimal multiple selection.
\newblock In {\em Proc. International Conference on Automata, Languages, and
  Programming ({ICALP'05})}, {\em Lecture Notes in Computer Science}, pages 103--114.
  Springer-Verlag, 2005.

\bibitem{K73}
J.~K\"orner.
\newblock Coding of an information source having ambiguous alphabet and the
  entropy of graphs.
\newblock In {\em Transactions of the 6th {P}rague {C}onference on
  {I}nformation {T}heory}, pages 411--425, 1973.

\bibitem{L84}
N.~Linial.
\newblock The information-theoretic bound is good for merging.
\newblock {\em SIAM J. Comput.}, 13(4):795--801, 1984.

\bibitem{L72}
L.~Lov{\'a}sz.
\newblock Normal hypergraphs and the perfect graph conjecture.
\newblock {\em Discrete Math.}, 2(3):253--267, 1972.

\bibitem{NN94}
Y.~Nesterov and A.~Nemirovskii.
\newblock {\em Interior-point polynomial algorithms in convex programming},
  volume~13 of {\em SIAM Studies in Applied Mathematics}.
\newblock Society for Industrial and Applied Mathematics (SIAM), Philadelphia,
  PA, 1994.

\bibitem{Pa03}
A.~Panholzer.
\newblock Analysis of multiple quickselect variants.
\newblock {\em Theor. Comput. Sci.}, 302(1-3):45--91, 2003.

\bibitem{Pr03}
H.~Prodinger.
\newblock Multiple quickselect -- {H}oare's find algorithm for several
  elements.
\newblock {\em Inf. Process. Lett.}, 56:123--129, 1995.

\bibitem{S85}
M.~E.~Saks.
\newblock The information theoretic bound for problems on ordered sets and
  graphs.
\newblock In {\em Graphs and order, Banff, Alta., 1984}, volume 147 of {\em
  NATO Adv. Sci. Inst. Ser. C Math. Phys. Sci.}, pages 137--168, Dordrecht,
  1985. Reidel.

\bibitem{S76}
A.~Sch{\"o}nhage.
\newblock The production of partial orders.
\newblock In {\em Journ\'ees algorithmiques, \'{E}cole {N}orm. {S}up., {P}aris,
  1975}, pages 229--246. Ast\'erisque, No. 38--39. Soc. Math. France, Paris,
  1976.

\bibitem{S86}
R.~P.~Stanley.
\newblock Two poset polytopes.
\newblock {\em Discrete Comput.\ Geom.}, 1:9--23, 1986.

\bibitem{Y89}
A.~C.~Yao.
\newblock On the complexity of partial order productions.
\newblock {\em SIAM J. Comput.}, 18(4):679--689, 1989.

\end{thebibliography}


\end{document}
