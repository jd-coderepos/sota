\documentclass[final,journal,letterpaper]{IEEEtran}

\usepackage{amsmath}
\usepackage{graphicx}
\usepackage{array}
\usepackage{hyperref}
\usepackage{microtype}
\usepackage{multicol}
\usepackage[caption=false,font=footnotesize]{subfig}

\newcounter{MYtempeqncnt}

\begin{document}

\title{Performance of Opportunistic Epidemic Routing on Edge-Markovian Dynamic Graphs}

\author{John Whitbeck, Vania Conan, and Marcelo Dias de Amorim

\thanks{A poster of this work was presented at the \emph{ACM SIGCOMM Workshop on Networking, Systems, Applications on Mobile Handhelds (Mobiheld 2009)}. This version is more detailed and contains many more results.}

\thanks{This work has been partially supported by the ANR project Crowd under contract ANR-08-VERS-006.}

\thanks{John Whitbeck is with both UPMC Sorbonne Universit{\'e}s and Thal{\`e}s Communications, France. Email: john.whitbeck@lip6.fr. }

\thanks{Vania Conan is with Thal{\`e}s Communications, France. Email: vania.conan@fr.thalesgroup.com.}

\thanks{Marcelo Dias de Amorim is with LIP6/CNRS~-- UPMC Sorbonne Universit{\'e}s, France. Email: marcelo.amorim@lip6.fr.}}

\IEEEpubid{0000--0000/00\N\tau\tauN\uparrow\downarrowp_\uparrowp_\downarrow\uparrow\downarrowT_\uparrowT_\downarrowE(T_\uparrow) = \frac{\tau}{p_\downarrow}E(T_\downarrow) = \frac{\tau}{p_\uparrow}\uparrow\downarrowp_\downarrowp_\uparrow\pi_\uparrow\pi_\downarrow\uparrow\downarrow\pi_\uparrow = \frac{p_\uparrow}{p_\uparrow+p_\downarrow}\pi_\downarrow = \frac{p_\downarrow}{p_\uparrow+p_\downarrow}(N-1)\pi_\uparrowInit(1,0)(1,1)(2,0)SuccInit(1,0)(1,1)(2,0)Succ\phi\phi \tau\phi \tau\tau\tau\alpha \phi \tau\alpha2\alpha=220.5dd\alpha = 1\alpha < 1\alpha > 1\alpha = 1ab1VkbJ_kk-1I_{k}S_k = V \setminus (I_k \cup J_k \cup \{b\})k+1I_kkI_k \cup J_kS_kk+1J_kS_kk+1bdijI_kJ_k2+\frac{N(N-1)}{2}InitaSuccb(i,j)1 \le i \le N-10 \le j \le   N-1-iUWUWpmW\textrm{pdf}_{\mathcal{B}}(m,p,n)np\in J_k\pi_\uparrow\in I_kp_\uparrow|I_k|=i|J_k|=jbSuccInit(0,1)(i,j)SuccP_{succ}(i,j)(i+j,j')\mathbf{T}\mathbf{i}\mathbf{s}1Succ0SuccdP_{deliv}( d, \alpha = 1 ) = \mathbf{i} \mathbf{T}^d \mathbf{s}35Init = (0,1)(1,0)(1,1)(2,0)Succs = [0 \: 0 \: 0 \: 0 \: 1]^Ti = [1 \: 0 \: 0 \: 0 \: 0]\mathbf{T}\alpha < 1\left \lfloor \frac{1}{\alpha} \right \rfloor\mathbf{R}P_{succ}^{static}(i,j) = 1 - \pi_\downarrow^j(i,j)(i+j,j')\left( 1-P_{succ}^{static}(i,j) \right) P_{inf}(j',\pi_\uparrow,j,N-1-i-j)SuccdP_{deliv}(d,\alpha < 1) = \mathbf{i} \left( \mathbf{T} \cdot \mathbf{R}^{\lfloor \frac{1}{\alpha} \rfloor -1} \right)^d \mathbf{s}\alpha > 1\lceil \alpha \rceil1,2, \ldots, \lceil \alpha \rceil\lceil \alpha \rceil\lceil \alpha \rceil\pi_\uparrow\pi_\uparrow (1-p_\downarrow)^{\lceil \alpha \rceil - 1}p_\uparrowp_\uparrow p_\downarrow^{\lceil \alpha \rceil - 1}\alpha = 2\pi_\uparrow\left(\pi_\uparrow+\pi_\downarrow (1-(1-p_\uparrow)^{\lceil \alpha \rceil - 1})\right) (1-p_\downarrow)^{\lceil \alpha \rceil - 1}p_\uparrow\left(1-(1-p_\uparrow)^{\lceil \alpha \rceil}\right)(1-p_\downarrow)^{\lceil \alpha \rceil - 1}\mathbf{T_l}\mathbf{T_u}d\mathbf{i} \mathbf{T_l}^{\frac{d}{\lceil \alpha \rceil}} \mathbf{s} \le P_{deliv}(d,\alpha > 1) \le \mathbf{i} \mathbf{T_u}^{\frac{d}{\lceil \alpha \rceil}} \mathbf{s}ddd=4\tau\tauN=20p_\downarrow=1/2p_\uparrow=1/20d=51N\uparrow\downarrow\alpha \le 1E(T_\uparrow)\alpha > 1\alpha \le 1126.184.75p_\uparrow=0.05p_\downarrow=0.57p_\uparrowp_\downarrow\alpha \le 2115110.95d=4$ curve.

\noindent\textbf{Tight delays require smaller bundles.} The sharp delivery ratio drop in Fig.~\ref{exp_results} occurs later for more relaxed delay constraints. A tight time constraint (less than a couple of minutes for example) forces the use of smaller bundles in order to obtain an acceptable delivery ratio. On the other hand, looser time constraints allow for more flexibility regarding bundle size. It is therefore possible to determine the maximum bundle size for any given target delivery ratio. 

\vfill \break

\section{Conclusion}
In this paper, we proposed a new model for epidemic propagation on edge-Markovian dynamic graphs which capture the correlation between successive connectivity graphs. We find a closed-form expression of delivery ratio as a function of bundle size, maximum tolerated delay, and the dynamics of the underlying evolving graph. In particular, we have shown that, given a certain maximum delay and node mobility, bundle size has a major impact on the delivery ratio. Our theoretical insights on the interaction between these parameters are corroborated by experimental results on the Rollernet dataset.

\bibliographystyle{IEEEtran} 
\bibliography{whitbecktoc}

\end{document}
