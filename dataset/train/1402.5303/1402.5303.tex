\documentclass[12pt]{article}
\usepackage{fullpage,amsthm}
\usepackage{amsmath,amssymb}
\usepackage{color}
\usepackage{graphicx}
\usepackage{booktabs}
\usepackage{enumerate}
\usepackage{caption}
\usepackage{times}

\newcommand{\abs}[1]{\left| #1 \right|}
\newcommand{\set}[1]{\{#1\}}
\newcommand{\ep}{\varepsilon}
\newcommand{\spread}{{\rm spread}}

\newcommand{\ceil}[1]{\lceil #1 \rceil}
\newcommand{\floor}[1]{{\lfloor #1 \rfloor}}

\newcommand{\RR}{\mathbb{R}}

\newtheorem{lemma}{Lemma}
\newtheorem{proposition}{Proposition}
\newtheorem{theorem}{Theorem}
\newtheorem{definition}{Definition}\newtheorem{corollary}{Corollary}
\newtheorem{remark}{Remark}
\begin{document}

\title{Diffuse Reflection Radius in a Simple Polygon}
\author{
  Eli Fox-Epstein\thanks{Department of Computer Science, Brown University, Providence, RI. \texttt{ef@cs.brown.edu}} \and
  Csaba D. T\'oth\thanks{Department of Mathematics, California State University, Northridge, Los Angeles, CA. \texttt{cdtoth@acm.org}} \and
  Andrew Winslow\thanks{D\'{e}partement d'Informatique, Universit\'{e} Libre de Bruxelles, Brussels, Belgium. \texttt{awinslow@ulb.ac.be}}
}\date{}
\maketitle

\begin{abstract}
It is shown that every simple polygon in general position with  walls can be illuminated from a single point light source  after at most  \emph{diffuse reflections}, and this bound is the best possible. A point  with this property can be computed in  time.
It is also shown that the minimum number of diffuse reflections needed to illuminate a given simple polygon from a single point can be approximated up to an additive constant in polynomial time.
\end{abstract}

\section{Introduction}
When light diffusely reflects off of a surface, it scatters in all directions.
This is in contrast to specular reflection,
  where the angle of incidence equals the angle of reflection.
We are interested in the minimum number of diffuse reflections needed to illuminate all points in the interior of a simple polygon  with  vertices from a single light source  in the interior of .
A \emph{diffuse reflection path} is a polygonal path  contained in  such that every interior vertex of  lies in the relative interior of some edge of , and the relative interior of every edge of  lies in the interior of  (see Fig.~\ref{fig:diffuse-ex1} for an example). Our main result is the following.

\begin{theorem}\label{thm:radius}
For every simple polygon  with  vertices in general position (i.e., no three collinear vertices), there is a point  such that for all ,
  there is a diffuse reflection path from  to  with at most  internal vertices.
This lower bound is the best possible.
A point  with this property can be computed in  time.
\end{theorem}

\begin{figure}[htb]
\centering
\includegraphics[width=\textwidth]{diffuse-ex1}
\caption{(a) A diffuse reflection path between  to  in a simple polygon . (b)--(d) The regions of a polygon illuminated by a light source  after 0, 1, and 2 diffuse reflections. The diffuse reflection radius of a zig-zag polygon with  vertices is .}
\label{fig:diffuse-ex1}
\end{figure}

This result is, in fact, a tight bound on the worst-case diffuse reflection radius (defined below) for simple polygons.
Denote by  the part of the polygon illuminated by a light source  after at most  diffuse reflections. Formally,  is the set of points  such that there is a diffuse reflection path from  to  with at most  interior vertices. Hence,  is the visibility polygon of point  within the polygon  if . The \emph{diffuse reflection depth} of a point  is the smallest integer  such that . The \emph{diffuse reflection radius}  of a simple polygon  is the minimum diffuse reflection depth over all points , and \emph{diffuse reflection center} of  is the set of points  that attain this minimum. With this terminology, Theorem~\ref{thm:radius} implies that  for every simple polygon  with  vertices in general position. A family of zig-zag polygons (e.g.\ the polygon in Fig.~\ref{fig:diffuse-ex1}) shows that this bound is the best possible for all . We note here that the \emph{diffuse reflection diameter}  of  is the \emph{maximum} diffuse reflection depth over all .

No polynomial-time algorithm is known for computing  for a given polygon  with  vertices. We show, however, that  can be approximated up to a constant additive error in polynomial time.
\begin{theorem}\label{thm:apx-compute-radius}
 Given a simple polygon  with  vertices in general position,
 one can compute in time polynomial in :
\begin{enumerate}
\item an integer  such that , and
\item a point  such that .
\end{enumerate}
\end{theorem}

\paragraph{\bf Motivation and Related Work.}
Diffuse reflection paths are special cases of \emph{link paths}, which have been studied extensively due to applications in motion planning, robotics, and curve compression~\cite{G07,MSD00}.
The \emph{link distance} between two points,  and , in a simple polygon  is the minimum number of edges in a polygonal path between  and  that lies entirely in .
In a polygon  with  vertices, the link distance between two points can be computed in  time~\cite{S86}.
The \emph{link depth} of a point  is the smallest integer  such that all other points in  are within link distance  of .
The \emph{link radius} is the minimum link depth over all points in , and the \emph{link center} is the set of points with minimum link depth.
It is known that the link center is a convex region and can be computed in  time~\cite{DLS92}.
The \emph{link diameter} of , the maximum link depth over all points in , can also be computed in  time~\cite{S90}.

The \emph{geodesic center} of a simple polygon is a point inside the polygon which minimizes the maximum shortest-path distance  (also known as geodesic distance) to any point in the polygon. Asano and Toussaint~\cite{AT85} proved that the geodesic center is unique. Pollack et al.~\cite{PSR89} show how to compute the geodesic center of a simple polygon with  vertices in  time; this was recentpy improved to  time by Ahn et al.~\cite{ABB+15}. Hershberger and Suri~\cite{HS97} give an  time algorithm for computing the \emph{geodesic diameter}. Schuirer~\cite{Sch94} gives  time algorithms for the geodesic center and diameter under the  metric in rectilinear polygons. Bae et al.~\cite{BKOW14} show that the -geodesic diameter and center can be computed in  time in every simple polygon with  vertices.

Diffuse reflection paths have received increasing attention since the mid-1990s when Tokarsky~\cite{Tok95} answered a question of Klee~\cite{Kle69,Kle79}, proving that a light source may not cover the interior of the simple polygon using \emph{specular} reflection (where the angle of incidence equals the angle of reflection in the reflection path).
He constructed a simple polygon  and two points  such that there is no specular reflection path from  to . It is not difficult to see that all points  can be reached from any  on a diffuse reflection path. However, the maximum number of reflections, in terms of the number of vertices, have been determined only recently. Barequet et al.~\cite{Us} proved, confirming a conjecture by Aanjaneya et al.~\cite{ABP08}, that  for all simple polygons with  vertices, and this bound is the best possible.

The link distance, geodesic distance and the -geodesic distance are all metrics; but the minimum number of edges on a diffuse reflection path between two points is \emph{not} a metric. Specifically, the triangle inequality need not hold (note that for , the concatenation of an two diffuse reflection paths, -to- and -to-, need not be a diffuse reflection path since it may have an interior vertex at ). This explains, in part, the difficulty of handling diffuse reflections.
Brahma et al.~\cite{BPS04} constructed examples where  (the set of points reachable from  after at most two diffuse reflections) is not simply connected, and where  has  holes. In general, the maximum complexity of  is known to be  and ~\cite{ADI+06}. In contrast to link paths, the best known algorithm for computing a minimum diffuse reflection path (one with the minimum number of reflections) between two points in a simple polygon with  vertices takes  time~\cite{ADI+06,G07}.
Ghosh et al.~\cite{GGM+12} give a 3-approximation for this problem that runs in  time. Bishnu et al.~\cite{BGG+14} define a \emph{constrained} version of diffuse reflection paths that can be computed in  time.

Khan et al.~\cite{KPA+13} study two weaker models of diffuse reflections, in which some edges of a diffuse reflection path may overlap with the boundary of the polygon . They establish upper and lower bounds for
the diffuse reflection radius under these weaker models for simple polygons that can be decomposed into convex quadrilaterals. No previous bound has been known for the diffuse reflection radius under the standard model that we use in this paper.

\paragraph{\bf Proof Technique.}
The regions  are notoriously difficult to handle. Instead of , we rely on the simply connected regions  defined by Barequet et al.~\cite{Us} and show that  for some point . In Section~\ref{sec:prelim}, we establish a simple sufficient condition (Lemma~\ref{lem:condition}) for  in terms of the visibility polygon  that can be verified in  time. Except for two extremal cases that are resolved directly (Section~\ref{ssec:double}), we prove that there \emph{exists} a point satisfying these conditions in Section~\ref{sec:center}.

The two main geometric tools we use are a generalization of a kernel of a simple polygon (Section~\ref{ssec:kernel}) and the weak visibility polygon for a line segment (Section~\ref{ssec:witness}). Finally, the existential proof can be turned into an efficient algorithm: the generalized kernel can be computed in  time, and the visibility polygon for a point moving along a line segment can be maintained with a persistent data structure. The combination of these methods helps finding a witness point  with   in  time.


\section{Preliminaries}
\label{sec:prelim}


For a set  in the plane, let  denote the interior,  the boundary, and  the closure of . Let  be a simply connected closed polygonal domain (for short, \emph{simple polygon}) with  vertices. A \emph{chord} of  is a closed line segment  such that  and the relative interior of  is in .

We assume that the vertices of  are in general position (that is, no three collinear vertices), and we only consider light sources  that do not lie on any line spanned by two vertices of . Recall that  is the visibility polygon of the point  with respect to .
The \emph{pockets} of  are the connected components of .
See Fig.~\ref{fig:pockets}(a) for examples.
The common boundary of  and a pocket is a chord  of  (called a \emph{window}) such that  is a reflex vertex of  that lies in the relative interior of segment . We say that a pocket with a window  is \emph{induced by} the reflex vertex . Note that every reflex vertex induces at most one pocket of . We define the \emph{size} of a pocket as the number of vertices of  on the boundary of the pocket. Since the pockets of  are pairwise disjoint, the sum of the sizes of the pockets is at most , the number of vertices of .

\begin{figure}[ht]
  \centering
  \includegraphics[width=.9\textwidth]{fig-pocketsxy}
  \caption{\label{fig:pockets}
(a) A polygon  where  has three pockets ,  and ,
    of size 4, 4, and 5, respectively. The left pockets are  and ,
   the only right pocket is . Pocket  is independent of
   both  and ; but  and  are dependent.
(b) The construction of region  from  in~\cite{Us}.
Pocket  is saturated, and pockets  and  are unsaturated.}
\end{figure}

A pocket is a \emph{left} (resp., \emph{right}) pocket if it lies on the left (resp., right) side of the directed line . Two pockets of  are \emph{dependent} if some chord of  crosses the window of both pockets; otherwise they are \emph{independent}.
One pocket is called independent if it is independent of all other pockets.

\begin{proposition} \label{prop:independent}
All left (resp., right) pockets of  are pairwise independent.
\end{proposition}


\begin{proof}
Consider two left pockets of , lying on the left side of the windows  and , respectively (see Fig.~\ref{fig:pockets}(a)). Suppose, for contradiction, that some chord  of  intersects both windows. Let  be the segment of  between  and . Segment  lies in the right halfplane of both  and . The intersection of these two halfplanes is a wedge with the apex at , and either  or  is not incident to this wedge.
This contradiction implies that no chord  can cross both windows  and .
\end{proof}
The main result of Section~\ref{sec:prelim} is a sufficient condition (Lemma~\ref{lem:condition})
for a point  to fully illuminate  within 
diffuse reflections.
The proof of the lemma is postponed to the end of Section~\ref{sec:prelim}.
It relies on the techniques developed by Barequet et al.~\cite{Us} and the bound
 on the diffuse reflection diameter.

\begin{lemma} \label{lem:condition}
We have 
for a point  if the pockets of 
satisfy these conditions:
\begin{enumerate}\itemsep -2pt
  \item[] every pocket has size at most ; and
  \item[] the sum of the sizes of any two dependent pockets is at most .
\end{enumerate}
\end{lemma}

\subsection{Review of regions .}

We briefly review the necessary tools developed by Barequet et al.~\cite{Us}. Let  be a point in general position with respect to the vertices of . Recall that , the set of points reachable from  with at most  diffuse reflections, is not necessarily simply connected when ~\cite{BPS04}. Instead of tackling  directly, Barequet et al.~\cite{Us} recursively define simply connected regions , where , for all . For , we have . We now review how  is constructed from . Each region  is bounded by chords of  and segments along the boundary . The connected components of  are the \emph{pockets} of . Each pocket  of  is bounded by a chord  such that  is a reflex vertex of ,  is an interior point of an edge of , and the two edges of  incident to  are on the same side of the line  (these properties are maintained recursively for ).

A pocket  of  is \emph{saturated} if every chord of  that crosses  has one endpoint in  and the other endpoint in . Otherwise,  is \emph{unsaturated}. Recall that for a point ,  is the set of points in  visible from . We also introduce an analogous notation for a line segment : let  denote the set of points in  visible from any point in .

For a given point , the regions  are defined as follows (refer to Fig.~\ref{fig:pockets}(b)). Let . If ,
then let . If , then  has at least one pocket.
For each pocket , we define a set :
If  is saturated, then let .
If  is unsaturated, then let  be a point close to 
such that no line determined by two vertices of  separates  and ;
and then let . Let  be the union of 
and the sets  for all pockets  of . Barequet et al.~\cite{Us} prove
that  for all .

\begin{remark}\label{remark:1}{\rm
Note that when a pocket  is unsaturated, then  is an interior point of
some edge  of . Since light does not propagate along the edge , the regions  and  do not contain . Consequently, there is a fine difference between \emph{independent} and \emph{saturated} pockets. Every saturated pocket of  is independent from all other pockets (by definition), but an independent pocket of  is not necessarily saturated. In Fig.~\ref{fig:remark} (a),  and  are dependent
pockets of ; region  covers the interior of , but not its boundary, and it
has a pocket . Even though  is independent of all other pockets of ,
it is unsaturated: a chord between  and the uncovered part of  crosses .
Since  is in general position, this phenomenon does not occur for , and
every independent pocket of  is saturated.}
\end{remark}

\begin{figure}[htp]
  \centering
  \includegraphics[width=.9\textwidth]{fig-remark}
  \caption{\label{fig:remark}
(a) A polygon  where  has four unsaturated pockets: .
(b) The white lines on the boundary of  are not part of .
Consequently, pocket  of  is unsaturated, although it is independent of all other pockets.
Pockets  and  of  are independent and saturated.}
\end{figure}

We say that a region  \emph{weakly covers} an edge of  if the boundary 
intersects the relative interior of that edge. On the boundary of every pocket  of ,
there is an edge of  that  does not weakly cover, namely, the edge of  incident to .
We call this edge the \emph{lead edge} of . The following observation follows from
the way the regions  are constructed in~\cite{Us}.

\begin{proposition}[\cite{Us}]\label{pp:Us}
For every pocket  of region , ,
the lead edge of  is weakly covered by region 
and \emph{is not} weakly covered by .
\end{proposition}

\begin{proposition}\label{pp:Us+}
If a pocket  of  has size , then 
weakly covers at least  edges of  on the boundary of .
\end{proposition}

\begin{proof}
  For every , let  denote the number of edges of  on the boundary of  that are weakly covered by .
 is bounded by a chord and  edges of .
One of these edges (the edge that contains ) is weakly covered by ,
  hence .
Since  for all ,
   is monotonically increasing, and every pocket of 
  that intersects  is contained in .
In each pocket of , by Proposition~\ref{pp:Us}, region  weakly covers at least one new edge of .
Consequently, we have  for all .
Induction on  yields .
\end{proof}

\subsection{Incrementally covering the pockets of }

In this subsection, we present three technical lemmas that yield upper
bounds on the minimum  for which  contains the interior of
a given pocket of .
The following lemma is a direct consequence of Proposition~\ref{pp:Us+}.
It will be used for unsaturated pockets of .

\begin{lemma}\label{lem:allpockets}
If  is a size- pocket of , then .
\end{lemma}

\begin{proof}
By Proposition~\ref{pp:Us+},  weakly covers all edges of  on the boundary of .
Consequently,  cannot contain any pocket of  (otherwise  would weakly
cover at least  edges by Proposition~\ref{pp:Us}).
Thus , as claimed.
\end{proof}

For saturated pockets, the diameter bound~\cite{Us} allows a better result.

\begin{lemma} \label{lem:saturated}
If  is a size- saturated pocket of ,
then .
\end{lemma}

\begin{proof}
Let  be the window of .
Since  is a reflex vertex of , it is a convex vertex of the pocket .
Refer to Fig.~\ref{fig:pprime}. Since  is saturated, every chord that crosses 
is part of a diffuse reflection path that starts at 
and enters the interior of  after at most  reflections.

\begin{figure}[htp]
  \centering
  \includegraphics[width=.8\textwidth]{fig-pocketsxz}
  \caption{\label{fig:pprime}
(a) A polygon  from Fig.~\ref{fig:pockets} with saturated pocket .
(b) Polygon  for the pocket .}
\end{figure}

We construct a polygon  with  vertices and a point  such that  is a pocket of  in ,
and every chord of  that crosses  is part of a diffuse reflection path that starts at  and enters the interior of  after one reflection in .
The polygon  is bounded by the common boundary  and a polygonal path , where  is in a small neighborhood of  such that  and  lie on the same side of line , and  lies on the edge of  that contains  in the exterior of . Place  on the line  such that  is in the relative interior of .

Polygon  has  vertices (since  lies in the interior of an edge of ). The diffuse reflection diameter of a polygon with  vertices is  from~\cite{Us}. Consequently, every point  can be reached from  after at most  diffuse reflections in .
Since a reflection path from  to any point  in  corresponds to an -to- reflection path in the original polygon 
with at most  more reflections, every  can be reached from  after at most  diffuse reflections in .
\end{proof}

Lemmas~\ref{lem:allpockets} and~\ref{lem:saturated} yield the following for dependent pockets of~.

\begin{lemma}\label{lem:dependent}
Let  be a pocket of  of size .
If each pocket dependent on  has size at most ,
  then .
\end{lemma}

\begin{proof}
For every , let  denote the number of edges of  on the boundary of  that are weakly covered by . We have , and if , then .
By Proposition~\ref{pp:Us+},   (i.e., at most  more edges
have to be weakly covered).

By Lemma~\ref{lem:allpockets},  contains the interior of all pockets of  that depend on . Consequently, if  has only one pocket inside , it must be independent (but not necessarily saturated, cf. Remark~\ref{remark:1}). By definition, , and so  also contains the boundaries of all pockets of  that depend on . Consequently, if  has exactly one pocket inside , it must be saturated.

We distinguish between two possibilities. First assume  for all  (that is, at least two more edges in  get weakly covered until all edges in  are exhausted). Then .

Otherwise, let  be the first index such that . Since  by assumption and  by Proposition~\ref{pp:Us}, we have . This means that  has exactly one pocket in , say , and  weakly covers only one new edge of  (e.g., pocket  in Fig.~\ref{fig:remark}). This is possible only if  is unsaturated. Then the region  is extended by  for a point  close to . Since  weakly covers only one new edge, the lead edge of , which incident to . Therefore,  is a triangle bounded by , the lead edge of , and the edge the contains . It follows that  also has exactly one pocket in , say , where the window  is collinear with the edge of  that contains . Hence the pocket  is \emph{saturated}:
of every chord that crosses , one endpoint is either in  or in . By Lemma~\ref{lem:saturated}, the interior of this pocket is contained in , as claimed.
\end{proof}


\subsection{Proof of Lemma~\ref{lem:condition}}
We prove a slightly more general statement than Lemma~\ref{lem:condition}.

\begin{lemma}\label{lem:condition+}
We have  if the pockets of  satisfy these conditions:
\begin{enumerate}\itemsep -2pt
    \item every pocket has size at most ; and
    \item the sum of the sizes of any two dependent pockets is at most .
\end{enumerate}
\end{lemma}
\begin{proof}
Consider the pockets of . By Lemma~\ref{lem:allpockets},
the interior of every pocket of size at most  is contained in .
It remains to consider the pockets  of size  for .
We distinguish between two cases.

\noindent{\bf Case~1: a pocket  of size  is independent of all
other pockets of .} Then  is saturated (cf. Remark~\ref{remark:1}).
By Lemma~\ref{lem:saturated}, the interior of  is
contained in .

\noindent{\bf Case~2: a pocket  of size  is dependent on some other
pockets of .}
Any other pocket dependent on  has size at most
   by our assumption.
Lemma~\ref{lem:dependent} implies that
  the interior of  is contained in .
\end{proof}

\begin{proof}[of Lemma~\ref{lem:condition}]
Invoke Lemma~\ref{lem:condition+} with , and note that .
\end{proof}

\subsection{Double Violators}
\label{ssec:double}

Recall that the sum of sizes of the pockets of  is at most , the number of vertices of .
It is, therefore, possible that several pockets or dependent pairs of pockets violate
conditions  or  in Lemma~\ref{lem:condition}.
We say that a point  is a
\emph{double violator} if  has either (i) two disjoint pairs of dependent pockets, each
pair with total size at least , or (ii) a pair of dependent pockets of total size at
least  and an independent pocket of size at least . (We do not worry
about the possibility of two independent pockets, each of size at least .)
In this section, we show that if there is a double violator
, then there is a point  (possibly ) for which
, and such an  can be found in  time.

The key technical tool is the following variant of Lemma~\ref{lem:dependent} for
a pair of dependent pockets that are adjacent to a common edge (that is, \emph{share} an edge).

\begin{lemma}\label{lem:double}
Let  and  be two dependent pockets of  such that neither is dependent on any other pocket, and points  and  lie in the same edge of . Let the size of  be  and  be .
Then  contains the interior of both  and .
\end{lemma}

\begin{proof}
For every , let  (resp., ) denote the number of edges of  on the boundary of  (resp., ) that are weakly covered by . We have  and  (the edge containing  and  is weakly covered by ). Proposition~\ref{pp:Us} guarantees . If , then the proof of Lemma~\ref{lem:dependent} readily implies that  contains the interior of both  and .

Assume now that . This means that  weakly covers precisely one new edge from each of  and . Recall that  and  are unsaturated, and  covers the part of  (resp., ) visible from a point near  (resp., ). See Fig.~\ref{fig:double1}(a). It follows that 
has exactly one pocket in each of  and , and both pockets are on the same
side of the line . Hence these pockets are saturated. They have size  and , respectively.
By Lemma~\ref{lem:saturated}, the interiors of both  and  are covered by .
\end{proof}

\begin{figure}[htp]
  \centering
  \includegraphics[width=.95\textwidth]{fig-double1}
  \caption{\label{fig:double1}
A polygon  with  vertices where  has four pockets: two pairs of dependent pockets, the
sum of sizes of each pair is .
(a) One extra vertex lies on  between two independent pockets.
(b) One extra vertex lies on  between two dependent pockets.}
\end{figure}

\begin{lemma}\label{lem:violate1}
Suppose that  has two disjoint pairs of dependent pockets, each
pair of total size at least . Then there is a point  such that
, and such a point  can be computed in  time.
\end{lemma}

\begin{proof}
The sum of the sizes of these four pockets is at least . If  is even, then the two dependent pairs each have size , they use all  vertices of , and both dependent pairs share an edge. If  is odd, then either (i) the two dependent pairs have sizes  and , resp., using all  vertices of , and both dependent pairs share an edge; or (ii) the two dependent pairs each have size , leaving one extra vertex, which may lie on the boundary between two independent pockets (Fig.~\ref{fig:double1}(a)), or between two dependent pockets (Fig.~\ref{fig:double1}(b)). In all cases, there is at least one dependent pair with joint size  that share an edge.

If the two dependent pairs each have size  and each share an edge (Fig.~\ref{fig:double1}(a)), then their interiors are covered by  for  by Lemma~\ref{lem:double}.
This completely resolves that case that  is even.

Assume now that  is odd. Denote the four pockets by , induced by the reflex vertices  in counterclockwise order along , such that  and  are dependent with joint size  and share an edge; and  and  are dependent but either has joint size  or do not share any edge. Refer to Fig.~\ref{fig:double1}(b). Note that  and  are edges of . Let  be the wedge bounded by the rays  and  (and disjoint from both  and ). For every point  in this wedge,  and  induce pockets  and , respectively, such that  and , and they also share an edge. Compute the intersection of region  with the two lines containing the lead edges of  and .
Let  be a closest point to  on these segments, and let  be a point close to  in general position such that it can see all of the lead edge for  or . By construction, vertex  or  is not incident to any pocket of . Consequently, the total size of all pockets of  in  and  is at most . By Lemmas~\ref{lem:condition} and~\ref{lem:double},  contains the interiors
of all pockets of , as claimed.
\end{proof}

\begin{figure}[htp]
  \centering
  \includegraphics[width=.85\textwidth]{fig-double2}
  \caption{\label{fig:double2}
A polygon  with  vertices where  has three pockets: two dependent pockets of total size  and an independent pocket of size .
(a) One extra vertex lies on  between two independent pockets.
(b) One extra vertex lies on  between two dependent pockets}
\end{figure}

\begin{lemma}\label{lem:violate2}
Suppose that  has a pair of dependent pockets of total size at least  and
an independent pocket of size at least . Then there is a point  with
, and  can be computed in  time.
\end{lemma}

\begin{proof}
The sum of the sizes of these three pockets is at least .
This implies that  has no other pocket, and so the independent pocket is saturated (cf. Remark~\ref{remark:1}). If  is even, then the two dependent pockets have total size  and share an edge, and the independent pocket has size .
If  is odd, then either (i) the three pockets use all  vertices of , and the dependent pockets share an edge (Fig.~\ref{fig:double2}(a)); or (ii) the dependent pair and the independent pocket each have size , leaving one extra vertex, which may lie on the boundary between two independent pockets, or between two dependent pockets (Fig.~\ref{fig:double2}(b)).
Denote the three pockets by , , and , induced by the reflex vertices , , and  in counterclockwise order along , such that  and  are dependent; and  is independent.
By Proposition~\ref{prop:independent},  and  have opposite orientation, so we may assume without loss of generality that  and  have opposite orientation (say, left and right).

First suppose that  has size . Refer to Fig.~\ref{fig:double2}(a). Then  and  have joint size  and share an edge, and by Lemma~\ref{lem:double},  contains the interior of both  and . Since all  vertices are incident to pockets,  is an edge of , and  is contained in an edge of , say . Since  is a window of , the supporting line of  intersects segment . Let  be a point close to the intersection of line  and segment . Then  and  induce pockets  and , respectively, such that , , and they share an edge. Both vertex  and the lead edge of  are directly visible from , they are not part of any pocket of . Consequently, the total size of pockets of  inside  is at most . By Lemma~\ref{lem:condition},  contains the interiors of all pocket of .

Now suppose that  is odd and  has size . Refer to Fig.~\ref{fig:double2}(b).
Denote the edge of  that contains  by  such that  (and possibly ).
Let  be a point in a small neighborhood of . Then  directly sees ,
and similarly to the previous case,  contains the interior of all pockets of  inside . If  and  jointly have size , then they share an edge and .
In this case  can see the lead edge of , the total size of all pockets of  inside  and  is at most , and if it equals , then two of those pockets are dependent and share an edge. If  and  jointly have size , then  has one ``unaffiliated'' vertex that does not belong to any pocket of  (Fig.~\ref{fig:double2}(b)). If , then  can see the lead edge of , and thus the total size of all pockets of  inside  and  is at most . If , then the unaffiliated vertex is , hence  and  share an edge. Consequently, the total size of all pockets of  inside  and  is at most , and if it equals , then two of those pockets are dependent and share an edge. By Lemmas~\ref{lem:condition} and~\ref{lem:double}, .
\end{proof}

\section{Finding a Witness Point}
\label{sec:center}

In Section~\ref{ssec:kernel}, we show that in every simple polygon  in general position, there is a point  that satisfies condition .
In Section~\ref{ssec:witness}, we pick a point  that satisfies condition , and move it continuously until either (i) it satisfies both conditions  and , or (ii) it becomes a double violator. In both cases, we find a witness point for Theorem~\ref{thm:radius} (by Lemmas~\ref{lem:condition},~\ref{lem:violate1}, and~\ref{lem:violate2}).

\subsection{Generalized Kernel}
\label{ssec:kernel}

Let  be a simple polygon with  vertices. Recall that the set of points from which the entire polygon  is visible is the \emph{kernel} of , denoted , which is the intersection of all halfplanes bounded by a supporting line of an edge of  and facing towards the interior of . Lee and Preparata~\cite{LeePreparata79} designed an optimal  time algorithm for computing the kernel of simple polygon with  vertices. We now define a generalization of the kernel. For an integer , let  denote the set of points 
such that every pocket of  has size at most . Clearly, , and  for all . The set of points that satisfy condition  is .

\begin{figure}[htp]
  \centering
  \includegraphics[width=\textwidth]{fig-kernel}
  \caption{\label{fig:kernel}
(a) Polygon .
(b) Polygon .
(c) Polygon .}
\end{figure}

For every reflex vertex , we define two polygons  and :
let  (resp. ) be the set of points  such that  does not induce a left (resp., right) pocket of size more than  in . We have



We show how to compute the polygons  and . Refer to Fig.~\ref{fig:kernel}. Denote the vertices of  by , and use arithmetic modulo  on the indices.
For a reflex vertex , let  be the first edge of the shortest (geodesic) path from  to  in . If the chord  and  meet at a reflex angle, then  is on the boundary of the \emph{smallest} left pocket of size at least  induced by  (for any source ). In this case, the ray  enters the interior of , and we denote by  the first point hit on . The polygon  is the part of  lying on the left of the chord . However, if the chord  and  meet at convex angle, then every left pocket induced by  has size less than , and we have .
Similarly, let  be the first edge of the shortest path from  to . Vertex  can induce a right pocket of size more than  only if  and  make a reflex angle. In this case,  is the boundary of the \emph{largest} right pocket of size at most  induced by , the ray   enters the interior of , and hits  at a point , and  is the part of  lying on the right of the chord . if  and  meet at a convex angle, then .

Note that every set  (resp., ) is \emph{-convex} (a.k.a.  \emph{geodesic convex}),
that is,  contains the shortest path between any two points in  with respect to ~\cite{BKOW14,DEH04,Tou86}. Since the intersection of -convex polygons is -convex,  is also -convex for every .
There exists a point  satisfying condition  if and only if  is nonempty. We prove  using a Helly-type result by Breen~\cite{Breen} (cf.~\cite{Breen98,Molnar57}).

\begin{theorem}[\cite{Breen}]\label{thm:Breen}
Let  be a family of simple polygons in the plane. If every three (not necessarily distinct)
members of  have a simply connected union and every two members of 
have a nonempty intersection, then .
\end{theorem}
\begin{figure}[htp]
  \centering
  \includegraphics[width=.85\textwidth]{fig-kernel2}
  \caption{\label{fig:kernel2}
(a) A simple polygon  with  vertices, and the generalized kernel .
(b) A schematic picture of a triangular hole in the union of three polygons in .}
\end{figure}
\begin{lemma}\label{lem:kernel}
  For every simple polygon  with  vertices,  has nonempty interior.
\end{lemma}
\begin{proof}
When , we have .
Now assume .

We apply Theorem~\ref{thm:Breen} for the polygons  and
 for all reflex vertices  of . By definition,
 is incident to  on or left of , and similarly ) is incident to  or or right or .
Furthermore,  (resp., )
is incident to at least one additional vertex right of 
(resp., left of ). Thus, each of these sets is incident to
at least  vertices of . Since ,
any two of these sets are incident to a common vertex of  by the pigeonhole principle.

Recall that  and  are each bounded by part of the boundary of 
and possibly a chord incident to . Consequently, if two of these sets are incident
to two or more common vertices of  then their interiors intersect. If they are incident
to precisely one common vertex of , then the common vertex, say , is incident to
both boundary chords, hence the two sets are  and . In this case, however,
 is a reflex vertex of both  and , and so their interiors intersect.

It remains to show that the union of any three of them is simply connected.
Suppose, to the contrary, that there are three sets whose union has a hole. Since each set is bounded by a chord of , the hole must be a triangle bounded by the three chords on the boundary of the three polygons. Refer to Fig.~\ref{fig:kernel2}(b).
Each of these chords is incident to a reflex vertex of  and is collinear with \emph{another} chord of  that weakly separates the vertices
 or
 from the hole.
Figure~\ref{fig:kernel2}(b) shows a schematic image.
The latter three chords together weakly separate disjoint sets of vertices
  of total size at least  from the hole,
contradicting the fact that  has  vertices altogether.
\end{proof}

By Lemma~\ref{lem:kernel},  has nonempty interior,
so there is a light source  
that satisfies condition .

\begin{lemma}\label{lem:kernel2}
For every ,  can be computed in  time.
\end{lemma}

\begin{proof}
With a shortest path data structure~\cite{GH89} in a simple polygon , the first edge of the shortest path
between any two query points can be computed in  time after  preprocessing time. A ray shooting data structure~\cite{HS95} can answer ray shooting queries in  time after  preprocessing time. Therefore, any chord  or  can be computed in  time.

The generalized kernel , can be constructed by incrementally maintaining the intersection  of some sets from  is reflex. In each step, we compute the intersection of  with  or . Recall that all these sets are -convex (the intersection of -convex sets is -convex). A chord of  intersects the boundary of a -convex polygon  in at most two points, and the intersection points can be computed in  time using a ray-shooting query in  (shoot a ray along the chord, and find the intersection points with binary search along the boundary of ). Thus  can be updated in  time. Altogether, we can compute  in  time.
\end{proof}


\subsection{Finding a Witness}
\label{ssec:witness}

In this section, we present an algorithm that, given a simple polygon  with  vertices in general position, finds a witness  such that .

Let  be an arbitrary point in . Such a point exists by Lemma~\ref{lem:kernel}, and can be computed in  time by Lemma~\ref{lem:kernel2}. We can compute the visibility polygon 
and its pockets in  time~\cite{GHL+87}. The definition of  ensures that  satisfies condition  of Lemma~\ref{lem:condition}. If it also satisfies , then  is a desired witness.

Assume that  does not satisfy , that is,  has two dependent pockets of total size at least , say a left pocket  and (by Proposition~\ref{prop:independent}) a right pocket . We may assume that  is at least as large as , by applying a reflection if necessary, and so the size of  is at least . Refer to Fig.~\ref{fig:line}(a). Let  be a point sufficiently close to  such that segment  is disjoint from all lines spanned by the vertices of , segment  is disjoint from the intersection of any two lines spanned by the vertices of , and . In Lemma~\ref{lem:line} (below), we find a point on segment  that is a witness, or double violator, or improves a parameter (spread) that we introduce now.

For a pair of dependent pockets, a left pocket  and (by Proposition~\ref{prop:independent}) a right pocket , let  be the part of  clockwise from  to  (inclusive), and let the \emph{size} of  be the number of vertices of  along . Note that
 is at least the sum of the sizes of the two dependent pockets, as all vertices incident to the two pockets are counted. For a pair of pockets of total size at least , we have .

\begin{figure}[htbp]
  \centering
  \includegraphics[width=0.8\textwidth]{fig-line}
  \caption{\label{fig:line}
(a) A polygon with  vertices where  violates  a pair of dependent
    pockets  and .
(b) Point  satisfies both  and .
(c) A polygon with  vertices where  violates  with
    a pair of pockets  and  of .
(d) Point  also violates  with a pair of pockets of .}
\end{figure}

We introduce some terminology to trace effects of moving a point  continuously in the interior of .
The visibility polygons of two points are \emph{combinatorially equivalent} if there is a bijection between their pockets such that corresponding pockets are incident to the same sets of vertices of .
The combinatorial changes incurred by a moving point  have been thoroughly analysed in \cite{AGTZ02,BLM02,CW12}.
The set of points  that induces combinatorially equivalent visibility polygons  forms a \emph{cell} in the \emph{visibility decomposition}  of polygon . It is known that each cell is convex and there are  cells, but a line segment in  intersects only  cells~\cite{BLM02,CD98}.
A combinatorial change in  occurs if  crosses a \emph{critical line} spanned by two vertices of , and the circular order of the rays from  to the two vertices is reversed.
The possible changes are: (1) a pocket of size 2 appears or disappears; (2) the size of a pocket increases or decreases by one; (3) two pockets merge into one pocket or a pocket splits into two pockets.
Importantly, the combinatorics of  does not contain enough information to decide whether two pockets are dependent or independent. Proposition~\ref{prop:sliding-pockets} (below) will be crucial for checking whether two dependent pockets become independent when a point  moves along a straight-line trajectory from  to .


\begin{proposition} \label{prop:sliding-pockets}
Let  be a line segment in . Then
\begin{enumerate}[(i)]
\item Every left (resp., right) pocket of  induced by a
vertex on the left (right) of 
is contained in a left (right) pocket of .
\item Let  and  be independent pockets of .
Then every two pockets of  contained in  and , respectively, are also independent.
\end{enumerate}
\end{proposition}

\begin{proof}
(i) Let  be a left pocket of  induced by vertex  on the left of . If  is directly visible from  (i.e., ),
then  is clearly contained in the left pocket of  induced by .
Otherwise, consider the geodesic path from  to  in . It is homotopic to the path , and so it contained in the triangle~. The first
internal vertex of this geodesic induces a left pocket of  that contains .

(ii) Since  and  are independent, no chord of  crosses the window of both pockets. Therefore no chord of  can cross the windows of two pockets lying in
 and , respectively.
\end{proof}


\begin{lemma}\label{lem:line}
Let  be an arbitrary point in  and
 as defined above.
Then there is a point  such that one of the following statements holds:
\begin{itemize}
\item[]  satisfies both  and ;
\item[]  is a double violator;
\item[]  satisfies  but violates  due
                  to two pockets of whose spread is contained in 
                  and has size at most .
\end{itemize}
\end{lemma}
\begin{proof}
We move a point  from  to  and trace the combinatorial changes
of the pockets of , and their dependencies. Initially, when , all pockets
have size at most ; and there are two dependent pockets, a left pocket  on the left of  and, by Proposition~\ref{prop:independent}, a right pocket  on the right of , of total size at least . When , every left pocket of  on the left of  is independent of any right pocket on the right of .

Consequently, when  moves from  to , there is a critical change from  to  such that  still has two dependent pockets of size at least  where the left (resp., right) pocket is on the left (right) of ; but  has no two such pockets. (See Fig.~\ref{fig:line} for examples.) Let  and  denote the two violator pockets of . The critical point is either a combinatorial change (i.e., the size of one of these pockets drops), or the two pockets become independent. By Proposition~\ref{prop:sliding-pockets}, we have  and , and the spread of  and  is contained in . We show that one of the statements in Lemma~\ref{lem:line} holds for  or .

If  satisfies both  and , then our proof is complete (Fig.~\ref{fig:line}(a-b)).
If  violates , i.e.,  has a pocket of size ,
then  also has a combinatorially equivalent pocket (which is independent of  and ), and so  is a double violator. Finally, if  violates , i.e.,  has two dependent pockets of total size , then the left pocket of this pair is not contained in  by the choice of point . We have two subcases to consider: (i) If the right pocket of this new pair is contained in  (or it is ), then we know that their spread is contained in  which has size at most   (Fig.~\ref{fig:line}(c-d)). (ii) If the right pocket of the new pair is disjoint from , then  also has a combinatorially equivalent pair of pockets, which is different from  and , and so  is a double violator.
\end{proof}

\begin{lemma}\label{lem:line2}
A point  described in Lemma~\ref{lem:line} can be found in  time.
\end{lemma}
\begin{proof}
It is enough to show that the critical positions,  and , in the proof of Lemma~\ref{lem:line} can be computed in  time. We use the persistent data structure developed
by Chen and Daescu~\cite{CD98} for maintaining the combinatorial structure of  as  moves along the line segment . The pockets (and pocket sizes) change only at  points
along , and each update can be computed in  time.

However, the data structure in~\cite{CD98} does not track whether two pockets on opposite sides
of  are dependent or not. The main technical difficulty is that  dependent pairs might become independent as  moves along  (even if we consider only pairs of total  size at least ), in contrast to only  combinatorial changes. We reduce the number of relevant events by focusing on only the ``large'' pockets (pockets of size at least ), and maintaining at most one pair that violates  for each large pocket. (In a dependent pair of size , one of the pockets has size .)

We augment the persistent data structure in~\cite{CD98} as follows. We maintain the list of all left (resp., right) pockets of  lying on the left (right) of , sorted in counterclockwise order along . We also maintain the set of \emph{large} pockets of size at least  from these two lists. There are at most 4 large pockets for any .
For a large pocket  of , we maintain one possible other pocket  of  such that they together violate . If there are several such pockets , we maintain only the one where  (the reflex vertex that induces ) is farthest from  along . Thus, we maintain a set  of at most 4 pairs . Finally, for each of pair , we maintain the positions  where the pair  becomes independent assuming that neither  nor  goes through combinatorial changes before  reaches . We use~\cite{CD98} together with these supplemental structures, to find critical points  such that  but .

We still need to show that  can be maintained in  time as  moves from  to . A pair   has to be updated if  or  undergoes a combinatorial change, or if they become independent (i.e., ). Each large pocket undergoes  combinatorial changes by Proposition~\ref{prop:sliding-pockets}. Note also that there are  reflex vertices along the boundary  between  and  (these vertices are candidates to become ).
No update is necessary when  or  changes but  remains large and the total size of the pair is at least . If the size of  drops below , we can permanently eliminate the pair from . In all other cases, we search for a new vertex , by testing the reflex vertices that induce pockets from the current  towards  along  until we either find a new pocket  or determine that  is not dependent of any other pocket with joint size . We can test dependence between  and a candidate for  in  time (test  by a ray shooting query). Each update of  decreases the size of the large pocket  or moves the vertex  closer to . Therefore, we need to test dependence between only  candidate pairs of pockets. Overall, the updates to  take  time.
\end{proof}

We are now ready to prove Theorem~\ref{thm:radius}.
\begin{proof}[of Theorem~\ref{thm:radius}]
Let  be a simple polygon with  vertices.
Compute the generalized kernel ,
and pick an arbitrary point ,
which satisfies . If  satisfies , too,
then  by Lemma~\ref{lem:condition}.
Otherwise, there is a pair of dependent pockets,  and ,
of total size at least  and .
Invoke Lemma~\ref{lem:line} up to three times to find a point 
that either satisfies both  and , or is a double violator.
If  satisfies  and  then Lemma~\ref{lem:condition}
completes the proof. If  is a double violator, apply Lemma~\ref{lem:violate1} or Lemma~\ref{lem:violate2} as appropriate to complete the proof. The overall running time
of the algorithm is  from the combination of Lemmas~\ref{lem:violate1},~\ref{lem:violate2},~\ref{lem:kernel2}, and~\ref{lem:line2}.

For every , the diffuse reflection diameter of the zig-zag polygon (cf. Fig.~\ref{fig:diffuse-ex1}) with  vertices is . By introducing up to 3 dummy vertices on the boundary of a zig-zag polygon, we obtain -vertex polygons  with  for all . Finally, every simple polygon with , 4, or 5 vertices is star-shaped, and so its diffuse reflection radius is .
\end{proof}

\section{Approximate Diffuse Reflection Radius}

In this section, we prove Theorem~\ref{thm:apx-compute-radius} and show how to \emph{approximate} the diffuse reflection radius  of a given polygon  up to an additive error of at most 1 in polynomial time (a similar strategy works for approximating the diffuse reflection diameter , as well).

\begin{proof}[of Theorem~\ref{thm:apx-compute-radius}]
Let  be a simple polygon with vertex set  in general position, where .
We wish to compute an integer  such that ,
and a point  such that  in polynomial time.
We prove the claim by analyzing the following algorithm:
\begin{enumerate} \itemsep2pt
\item[]\texttt{ApproxDiffuseRadius}
  \item For each vertex  of , find two points  and  in the relative interior of the two edges of  incident to  such that no line through a pair of vertices in
        separates them from . Let .
  \item Find the minimum integer  such that 
       is nonempty by binary search over .
\item Return  and an arbitrary point .
\end{enumerate}

We first show that \texttt{ApproxDiffuseRadius} runs in polynomial time in~.
We can find a suitable set  in  time by computing, for each ,
the intersection points between the  lines through a pair of vertices in  and
the two edges of  incident to .
Then,  and  can be picked as points on the relative interiors of the two edges incident
to  between  and the closest intersection point.
The combinatorial complexity of a region 
is at most , but the set of the boundary points  consists
of only~ line segments~\cite{ADI+06}.  Given ,
we can compute  by taking the union of the visibility regions
for~ line segments in  time~\cite{BLM02,CW12}. Instead of computing the
regions~, we iteratively maintain~ for all
 and , in  time.

For each , we find  as follows.
First compute the intersection of the boundary segments
, which consists of 
line segments, in  time. Then compute  as the set of points in 
visible from any point in  in  time~\cite{ADI+06}. The binary search
tries  values of , and so the total running time is .

Next we show that the minimum integer  for which  approximates~.
First, we prove that there is no  for which .
By the choice of , there is no  for which  or  (since ).
Then by~\cite{Us} and Proposition~\ref{prop:independent},  implies  for any .

Let  be an arbitrary point in . By the choice of , we have
. We now show that .
Let  be an arbitrary point in the interior of .
In any triangulation of , point  lies in some triangle~, and so  is directly visible from
 or  for . Since ,
there is a diffuse reflection path from  to these boundary points
with at most~ interior vertices. By appending one new segment to this
path, we obtain a diffuse reflection path from  to  with at most
 interior vertices.
\end{proof}


\section{Conclusions}
\label{sec:con}

Theorem~\ref{thm:radius} establishes the upper bound of  for the diffuse refection radius  of a simple polygon  with  vertices. This bound is the best possible. For a given instance , we can approximate  up to an additive error of~2 (Theorem~\ref{thm:apx-compute-radius}). However, no polynomial-time algorithm is known for computing  for a given polygon , or for computing the diffuse reflection center of . Similarly, we know that the diffuse reflection diameter~ of a simple polygon with  vertices is at most , and this bound is the best possible~\cite{Us}, but no polynomial-time algorithm is known for computing~ or a diametric pair of points for a given polygon .

We believe the general position assumptions about  and choice of light sources
  can be avoided at the cost of more complicated analysis taking
  caution to properly handle collinear chords.

In the remainder of this section, we show that the diffuse reflection center of a polygon  may not be connected or -convex, and that in general there is no containment relation between the geodesic center and the diffuse reflection center. These constructions explain, in part, why it remains elusive to efficiently compute the diffuse reflection center and radius.

 \begin{figure}[htbp]
    \centering
    \includegraphics[scale=1.0]{disconnected-and-nonconvex-center}
    \caption{(a) A polygon whose diffuse reflection center is disconnected.
    (b) A polygon whose diffuse reflection center is not geodesic convex.}
    \label{fig:disconnected-center}
  \end{figure}

\smallskip
\noindent{\bf Shape of the diffuse reflection center.}
While the link center is geodesic convex and connected~\cite{LPS+88}, it turns out
that we have no such guarantees on the shape of the diffuse reflection center.
There are polygons with disconnected diffuse reflection centers (Fig.~\ref{fig:disconnected-center}(a)),
and there are polygons whose diffuse reflection centers are connected but not geodesic convex (Fig.~\ref{fig:disconnected-center}(b)).

\begin{figure}[hb!]
    \centering
    \includegraphics[scale=1.0]{diffuse-vs-link}
    \caption{Examples of four inclusion relationships between the diffuse reflection and link centers.
    The diffuse center, link centers, and their intersection are colored yellow, blue, and green, respectively.
    The diffuse and link radii for the polygons in clockwise order from the upper left are 2, 2, 4, 4 and 2, 1, 3, 3, respectively.}
    \label{fig:diffuse-vs-link}
\end{figure}

Furthermore, there is no clear relationship between the two centers;
Fig.~\ref{fig:diffuse-vs-link} illustrates that there exists simple polygons
with each of the following properties:
\begin{enumerate}[(a)] \itemsep2pt
\item the diffuse reflection center is strictly contained in the link center;
\item the diffuse reflection center strictly includes the link center;
\item neither center contains the other but they are not disjoint;
\item the diffuse reflection center and the link center are disjoint.
\end{enumerate}


\bibliographystyle{spmpsci}
\bibliography{radius-arxiv}

\end{document}
