\documentclass[11pt]{article}


\usepackage{microtype}

\usepackage{verbatim}
\usepackage{amsmath}
\usepackage{amssymb}
\usepackage{amsthm}
\usepackage{amsthm}
\usepackage{amssymb}
\usepackage{amsmath}
\usepackage{wasysym}
\usepackage{amsmath}
\usepackage{subfig}
\usepackage{graphicx}
\usepackage{caption}
\usepackage{epstopdf}
\usepackage{enumerate}
\usepackage{color}
\usepackage{floatflt}
\usepackage{tabulary}
\usepackage{booktabs}




\usepackage[margin=1in]{geometry}


    \newtheorem{theorem}{Theorem}
    \newtheorem{assumption}{Assumption}
    \newtheorem{lemma}{Lemma}
    \newtheorem{definition}{Definition}
    \newtheorem{proposition}{Proposition}
    \newtheorem{remark}{Remark}
    \newtheorem{corollary}{Corollary}
    \newtheorem{property}{Property}
    
    


\newcommand{\Sites}{S}
\newcommand{\Vor}{\text{Vor}}
\newcommand{\diag}{\mathop{\mathrm{diag}}}
\newcommand{\D}[2]{D\left(#1 \parallel #2\right)}
\newcommand{\CHS}{{conv}\left\{\Sites\right\}}
\newcommand{\CH}[1]{\text{conv} #1 }
\title{On the Embeddability of Delaunay Triangulations in Anisotropic, Normed, and Bregman Spaces}



\author{Guillermo D. Canas\\Massachusetts Institute of Technology\\guilledc@mit.edu
\and Steven J. Gortler\\Harvard University\\sjg@seas.harvard.edu}








\date{}


\begin{document}
\newcounter{foo}
\maketitle


\begin{abstract}






Given a two-dimensional space endowed with a divergence function that is convex in the first argument, 
	continuously differentiable in the second, 
	and satisfies suitable regularity conditions at Voronoi vertices, 
we show that orphan-freedom (the absence of disconnected Voronoi regions) is sufficient 
	to ensure that Voronoi edges and vertices are also connected, and 
	that the dual is a simple planar graph. 
We then prove that the straight-edge dual of an orphan-free Voronoi diagram 
		(with sites as the first argument of the divergence) is always an embedded triangulation. 






Among the divergences covered by our proofs are Bregman divergences, 
	anisotropic divergences, 
	as well as all distances derived from strictly convex   norms 
	(including the  norms with ). 
While Bregman diagrams of the {first kind} are simply affine diagrams, 
	and their duals ({weighted} Delaunay triangulations) are always embedded, 
	we show that duals of orphan-free Bregman diagrams of the \emph{second kind} are always embedded. 	
	


\end{abstract}

\newpage

\section{Introduction}


Voronoi diagrams and their dual Delaunay triangulations are fundamental constructions with
numerous associated guarantees, and extensive application in
practice (for a thorough review consult~\cite{Aurenhammer13} and references therein).
At their heart is the use of a distance between points, which in the original
version is taken to be Euclidean. 
This suggests that, by considering 
distances other than Euclidean, 
it may be possible to obtain variants which can be well-suited to
a wider range of applications.  

Attempts in this direction have been met with some success. 
Power diagrams~\cite{power} generalize Euclidean distance by associating 
a {bias-term}
to each 
site. The duals of these diagrams
are  
guaranteed to be embedded triangulations, in any number of
dimensions. 
Although this is a strict generalization of Euclidean distance, it is a somewhat 
limited one. The effect of the bias term is to locally enlarge or shrink the
region associated to each site, loosely-speaking ``equally in every
direction". It allows some freedom in choosing local scale, with no
preference for specific directions. 





Two related, and relatively recent generalizations of Voronoi diagrams and Delaunay triangulations have been proposed, 
independently, by Labelle and Shewchuk~\cite{LS}, and Du and Wang~\cite{DW}. 
Although their associated anisotropic Voronoi diagrams are, in general, no longer orphan-free
(i.e.~they may have disconnected Voronoi regions), 
Labelle and Shewchuk show that a set of sites exists 
with an orphan-free diagram, whose dual is embedded, in two dimensions. 
They accomplish this by proposing an iterative site-insertion algorithm
that, for any given metric, constructs one such set of sites. 
Note that this is a property of the output of the algorithm, and not a general condition for obtaining embedded triangulations. 


The recent work of~\cite{Bregman} discusses Voronoi diagrams and their duals with respect to Bregman divergences. 
They show that Bregman Voronoi diagrams of the \emph{first kind} are simply power diagrams, 
	whose duals are known to always be embedded~\cite{powerdiag}. 
Bregman diagrams of the \emph{second kind} are power diagrams in the dual (gradient) space, 
	but, prior to this work, no results for them were available in the primal space. 


In this paper we discuss properties of Voronoi diagrams and Delaunay triangulations for 
	a general class of divergences, 
	including Bregman, quadratic, and all distances derived from strictly convex  norms. 
We show that, 
given a divergence  that is convex in the first argument and continuously differentiable in the second, 
	and under a \emph{bounded anisotropy} assumption on the divergence, 
 if a set of sites produces an orphan-free 
Voronoi diagram with respect to ,
then its dual is always an embedded triangulation 
	(or an embedded polygonal mesh with convex faces in general), 
in two dimensions (theorem~\ref{th:main}). 
This effectively states that, regardless of the sites' positions, if the primal 
is well-behaved, then the dual is also well-behaved. 
Further, in a way that parallels the ordinary Delaunay case, the dual has no
degenerate elements (proposition~\ref{prop:ECB}), its elements (vertices, edges, faces) 
are unique (Cor.~\ref{cor:VorI}), and the dual is guaranteed to cover the convex hull of the sites (theorem~\ref{th:main}). 











\section{Voronoi diagrams with respect to divergences}\label{sec:setup}

The class of divergences that we consider in this work are non-negative functions 
	 
	which are 	strictly convex in the first argument and continuously differentiable in the second, 
	and such that  for all . 
Following~\cite{Bregman}, we let

be, respectively, balls of the \emph{first} and \emph{second kind}, centered at  of radius . 
Note that balls of the first kind are necessarily convex since  is convex. 
We also assume that  satisfies what we term a \emph{bounded anisotropy} condition, 
	defined in assumption~\ref{ass:BAA} below. 

Given a set  of 
 distinct sites on the plane, and a divergence , 
the Voronoi regions 
of the first and second kinds~\cite{Bregman} are:  

respectively, and are indexed by the site its points are closest to.  
Of course, the two kinds of Voronoi diagrams are different because  is in general not symmetric. 
In the sequel, and whenever not otherwise specified, we will assume that balls are of the \emph{first kind} (convex), 
	and Voronoi diagrams, and their dual Delaunay triangulations are of the \emph{second kind}. 
For instance, we will use the convexity of balls (of the first kind) to prove 
	that every face in a Delaunay triangulation (of the second kind) 
	satisfies an \emph{Empty Circum-Ball} property (proposition~\ref{prop:ECB}) 
	that parallels the empty circumcircle property of Euclidean Delaunay triangulations. 



Consider the following definition of Voronoi element:
\begin{definition}\label{def:VorI}
	For each subset , the set  is a Voronoi element of order . 
Elements of orders , , and  are denoted regions, edges, and vertices, respectively. 
\end{definition}
\begin{remark}
The set of all Voronoi elements  forms a partition of the plane. 
\end{remark}




\begin{table}[htbp]\caption{Notation}\label{table:notation}
\begin{center}
\begin{tabular}{r c p{11cm} }
\toprule
 && A non-negative divergence strictly convex in its first argument and continuously differentiable in the second. \\
 && Bregman divergence (section~\ref{sec:DF}). \\
 && Csisz\'ar divergence (section~\ref{sec:Df}).\\
 && Quadratic divergence (seciton~\ref{sec:DQ}).\\
 && Global lower bound on the ratio of eigenvalues 
	of metric  (quadratic divergence, lemma~\ref{lem:DQgamma}) 
	or of the Hessian of  (Bregman divergence, lemma~\ref{lem:DFgamma}). \\
 && Set of  sites. \\
 && The supporting line of sites , . \\
 && Convex hull of . \\
 && Subset of sites on the boundary of , in clock-wise order. \\  && Convex ball of the first kind (equation~\eqref{eq:defball}). \\
 && The ball (of the first kind)   centered at  with  in its boundary. \\
 && Voronoi region of the second kind corresponding to site  (equation~\eqref{eq:defvor2}). \\
 && Voronoi element of order  (Voronoi region),  (Voronoi edge), or  (Voronoi vertex).\\
 && The straight-edge dual triangulation with vertices at the sites. \\
 && The edges in the topological boundary of  (incident to one face). \\
 && The edges in the boundary of . \\
 && Projection from  onto  (section~\ref{sec:boundary}). \\
 && Projection function onto a circle of radius  (section~\ref{sec:boundary}). \\
 && The half-spaces on either side of , chosen so  (fig.~\ref{fig:pinu}). \\
 && The origin-centered circle of radius  (with respect to the natural metric). \\
\bottomrule
\end{tabular}
\end{center}
\end{table}




The following ``bounded anisotropy" condition is assumed to hold. 
It is written in its most general (but very technical) form below, 
	but it becomes much simpler in particular cases, as shown in Section~\ref{sec:summary}. 
Typically, it can be rewritten as a simple regularity condition on a symmetric positive definite matrix, 
	such that its ratio of minimum to maximum eigenvalues
	(a measure of anisotropy) is globally bounded away from zero. 


\begin{figure}[thb]
\centering
\includegraphics[width=2.1in]{boundedanisotropy.pdf}
\caption{The \emph{bounded anisotropy asumption} ensures that balls of the first kind are globally well-behaved. }
\label{fig:gamma}
\end{figure}


\begin{assumption}[Bounded anisotropy]\label{ass:BAA}
For every two points  with supporting line , and every point , 
	there is a sufficiently large value  such that 
	for every point  lying on the same side of  as , 
		such that , and 
		whose closest point  in  lies in the segment , 
	it is . 
\end{assumption}
\begin{remark}
Note that the condition  depends on the (arbitrary) choice of origin. 
Assumption~\ref{ass:BAA} is, however, independent of this choice. 
\end{remark}
Loosely speaking, this condition ensures that balls of the first kind are 
	not just convex, but also ``sufficiently round". 
For instance, it is satisfied by all the  distances with , 
	but not for , since (aside from not being strictly convex) the corresponding balls have ``kinks". 



\begin{assumption}[Extremal gradients]\label{ass:EGA}
	For each Voronoi vertex  with , 
		the gradients , at  are distinct and extremal, 
		i.e.\ they are vertices of the convex hull:  . 
\end{assumption}

\begin{remark}
	In the ``typical" case that , the above simply means that  are not colinear. 
	Given two distinct gradients , requiring  not to be colinear only constraints it 
		to be outside a line.  
	If  is the  distance  (or any other non-spatially-varying divergence), 
	the extremal gradient assumption can be shown to be always automatically satisfied at Voronoi vertices. 
	Finally, the extremal gradient assumption will be shown to imply that Voronoi vertices are composed of isolated points, 
		and therefore, when satisfied, the assumption only needs to be enforced at a discrete set of points. 
\end{remark}



	

\subsection{Orphan-free Voronoi diagrams and dual triangulations}\label{sec:simpleplanar}



As described in the classic survey by Aurenhammer~\cite{Aurenhammer:1991}, 
	planar Voronoi diagrams and Delaunay triangulations are duals in a graph theoretical sense. 
Associated to the ordinary Voronoi diagram is a simple, planar (primal) graph with vertices at 
	points equidistant to three or more sites (Voronoi vertices), 
	and edges composed of line segments 
	equidistant to two sites (Voronoi edges). 
Because edges are always line segments, the graph is simple (has no multi-edges or self-loops), 
	and this construction provides an embedding of the graph, which must therefore be planar. 

	
	
For Voronoi diagrams defined by divergences, the situation is markedly different. 
The incidence relations between Voronoi elements cannot be so easily established. 
For instance, Voronoi edges may be disconnected and incident to any number of Voronoi vertices. 
For this reason, we begin our proof by constructing an embedding of a primal graph 
	from the incidence relations of the Voronoi diagram (definition~\ref{def:incidence}), 
	in a way that generalizes ordinary Voronoi diagrams, 
	and show that this graph is simple and planar (section~\ref{sec:planar}).
This primal graph is then dualized into a simple, planar graph. 
The dual graph is denoted the Delaunay \emph{triangulation} because, as will be shown, 
	it is composed of convex faces which can be triangulated without breaking any of 
	its important properties, such as embeddability or the \emph{empty circum-ball} property (property~\ref{cor:VorI}). 	


The rest of the paper makes heavy use of the following trivial lemmas, which we include here for convenience. 
The first follows directly from the properties of , while the second
	is a direct consequence of the strict convexity of  and the continuity of 
	(note that  is globally continuous since it is continuous in the second argument 
	and convex in the first, and therefore it is also continuous in the first argument~\cite{rockafellar1997convex}). 



\begin{lemma}\label{lem:site}
Every site  is an interior point of its corresponding Voronoi region . 
\end{lemma}



\begin{lemma}\label{lem:midpoint}
Given two sites  with supporting line , 
	all points  that are equidistant to  and  
	belong to the segment . 
Furthermore, there is always at least one such point. \end{lemma}





\section{Summary of results}\label{sec:summary}


\begin{figure}[ht]
\begin{center}
\subfloat{\includegraphics[height=3.5cm]{c09.png}}\quad
\subfloat{\includegraphics[height=3.5cm]{c10.png}}
\caption{If all sites are colinear, the dual is always a chain connecting consecutive sites along their supporting line.
This structure is independent of the divergence, and doesn't require assumption~\ref{ass:BAA}. }
\label{fig:colinear}
\end{center}
\end{figure}

Consider first the special case that all sites in  are colinear. The structure of the Voronoi diagram and the Delaunay triangulation is very simple in this case. 
If we order the sites  sequentially along their supporting line, 
lemma~\ref{lem:midpoint} shows that there must be Delaunay edges between successive sites, 
	while the strict convexity of the balls implies that these are the only edges
	(all points in  are strictly closer to  than to any other site), 
	and that there are no Delaunay faces
	(since three colinear points cannot be in the boundary of a strictly convex ball). 
The following proposition does not require assumption~\ref{ass:BAA} nor~\ref{ass:EGA}. 


\begin{proposition}
{For all divergences} , 
	the Delaunay triangulation of a set of colinear sites is a chain connecting successive sites ,  
	along their supporting line. 
\end{proposition}

With the colinear site case covered, we assume in the remainder 
	that \emph{not all} sites are colinear, 
	and that  satisfies assumptions~\ref{ass:BAA} and~\ref{ass:EGA}. 


We begin, in section~\ref{sec:planar}, by constructing a primal graph from the incidence relations between Voronoi elements, 
	and dualize it to obtain a simple, planar graph. 

\begin{theorem}\label{th:simpleplanar}
The dual of the primal Voronoi graph of an orphan-free Voronoi diagram 
	is a simple, connected, planar graph. 
\end{theorem}

\begin{remark}
Note that the differentiability of  with respect to the second argument is only used in (a small neighborhood around) Voronoi vertices 
	(a set of isolated points). 
Everywhere else, it suffices that  is continuous in its second argument. 
\end{remark}


While this dual graph is an embedded planar graph with curved edges, we then show
	that it is also an embedded planar graph with vertices at the sites and straight edges. 



\begin{theorem}\label{th:main}
	The straight-edge dual of a primal Voronoi graph
	(obtained from an orphan-free Voronoi diagram of a set of sites )
is embedded with vertices at the sites, 
		has (non-degenerate) strictly convex faces, and covers the convex hull of . 
\end{theorem}


As described in Section~\ref{sec:simpleplanar}, lemmas~\ref{lem:regionSC} and~\ref{lem:SCedges} 
	can be used in conjunction with theorem~\ref{th:main} to 
	conclude that orphan-freedom 
	is a sufficient condition 
	for the well-behavedeness of not just the 
dual, but also of the primal Voronoi diagram. 
Note that this excludes isolated Voronoi edges (those not incident to any Voronoi vertex), 
	which are shown to be contained in Voronoi regions, 
	and are considered part of their containing regions (section~\ref{sec:propedges}). 


\begin{corollary}\label{cor:VorI}
	All the elements of an orphan-free Voronoi diagram are connected, 
	with the exception of isolated Voronoi edges. 
\end{corollary}
\begin{remark}
	Isolated edges are connected components of a Voronoi edge 
	which are incident to a single Voronoi region. 
	Since they do not affect the construction of the primal Voronoi graph, 
they can be safely discarded, as shown in section~\ref{sec:propedges}.
\end{remark}

Perhaps the most fundamental property of the diagrams that we use in the proofs is that 
	every dual face has an ``empty" circumscribing \emph{convex} ball. 
This empty circum-ball (ECB) property
	is analogous to the empty circumcircle property of ordinary Voronoi diagrams:
\begin{proposition}[Empty Circum-Ball property]\label{prop:ECB}
	For every dual face with vertices  there is a convex ball that circumscribes  and contains no site in its interior. 
\end{proposition}

Indeed, since to every dual face  with vertices  () corresponds 
	a Voronoi element , 
	any point  serves as center of an empty circumscribing ball of . 
To see that this ball must be ``empty", note that 
no site  can be strictly inside the circumscribing ball (certainly not , since they are in the boundary), 
	or  would be closer to  than to , 
	and therefore it would not be . 

Notice that, although we consider Voronoi diagrams of the second kind, 
	it is the convexity of balls of the \emph{first kind} that establishes the ECB condition. 
The ECB property is, in general, not satisfied by Delaunay triangulations of the first kind. 

\begin{figure}[htbp]
\begin{center}
\includegraphics[width=2.5in]{outline.pdf}
\caption{
	We prove that the Delaunay triangulation is embedded (theorem~\ref{th:main}) 
	by showing that its boundary is simple and convex (corollary~\ref{cor:boundary}), 
		and its interior is a ``flat sheet": it has no edge fold-overs (green edge) (lemma~\ref{lem:ef}).  
	We use a discrete version of the Poincar\'e-Hopf index theorem (lemma~\ref{lem:ph}) to prove that  
		an edge fold-over would create a ``wrinkle" (circled) somewhere in the triangulation (lemma~\ref{lem:index-1}), 
		which in turn would force some vertex (blue) to ``invade" a face (red) (lemma~\ref{lem:non-negative}), 
		breaking the face's empty circum-ball (grey, dotted) condition (proposition~\ref{prop:ECB}).}
\label{fig:outline}
\end{center}
\end{figure}

After establishing that a Voronoi diagram can be associated with an embedded planar primal graph 
	which can be dualized into a planar dual graph (section~\ref{sec:planar}), 
	the rest of the paper is concerned with the proof of our main claim (theorem~\ref{th:main}), 
	whose structure is outlined in figure~\ref{fig:outline}. 
The proof of embeddability of the straight-edge dual is divided in two parts. 
In the first part (section~\ref{sec:boundary}), we use the bounded anisotropy assumption (assumption~\ref{ass:BAA})
	to show that the 
	\empty{topological} boundary of the straight-edge dual Delaunay triangulation 
	(the set of edges shared by only one face) coincides with the boundary of the convex hull of the sites, 
	and therefore is a simple, closed polygonal chain, a fact necessary for the second part of the proof to proceed. 
	Section~\ref{sec:boundary} is the more technical part of the proof; at its heart it is an application of Brouwer's fixed point theorem. 
In section~\ref{sec:interior}, we use the theory of discrete one-forms~\cite{1form} 
	to show that the Delaunay triangulation has no fold-overs (is a ``flat sheet") 
	and is therefore a single-cover of the convex hull of . 
Note that these two results, along with the ECB property, mirror similar properties of ordinary Delaunay triangulations. 


The above results can be particularized to a number of existing divergences and metrics. 
We briefly discuss next a few of them, as well as simple conditions for assumption~\ref{ass:BAA} to hold
	for some of them (with proofs in Appendix A). 







\subsection{Bregman divergences}\label{sec:DF}
Given a strictly convex, everywhere differentiable function , 
	the Bregman divergence

	is the (non-negative) difference between  and the first-order Taylor approximation of  around 
	(the first order Lagrange remainder). 
Bregman divergences are widely used in statistics and 
	include the Kullback-Leibler divergence. By the (strict) convexity of , and the definition of  it 
	it is clear that, whenever  is twice continuously differentiable, 
	 is (strictly) convex in the first argument and continuously differentiable in the second. 


From the definition of , it is clear that Bregman Voronoi diagrams of the first kind are composed of regions 
	
which are intersections of half-spaces of the form .
Furthermore, 
	Bregman Voronoi diagrams of the first kind are simply power diagrams~\cite{Bregman}, 
	and thus their dual Delaunay triangulations of the first kind are always embedded~\cite{powerdiag,DMG}. 


On the other hand, Bregman diagrams of the \emph{second} kind can be shown to be affine diagrams 
	only in the dual (gradient) space~\cite{Bregman}. 
In the original space, the cells  are not simple intersections of half-spaces and, in general, they have curved boundaries. 
Prior to this work, no guarantees concerning Bregman Delaunay triangulations of the second kind were available. 


\begin{lemma}[Bounded anisotropy for Bregman divergences]\label{lem:DFgamma}
If  and there is  such that the Hessian of  has ratio of eigenvalues bounded by ,
	then assumption~\ref{ass:BAA} holds. 
\end{lemma}



\subsection{Quadratic divergences}\label{sec:DQ}

As is well known, the approximation efficiency of a 
	piecewise-linear function supported on a triangulation can be greatly improved by 
	adapting the shape and orientation of its elements to the target function~\cite{triangle,DAzevedo,DBLP:conf/imr/CanasG06}. 
An effective way to construct such anisotropic triangulations is to dualize a Voronoi diagram 
	derived from an anisotropic divergence~\cite{LS,DW}. 

By considering a  metric 
	(in coordinates: a function  that is symmetric, positive definite), 
	we define the quadratic divergence as:

which is clearly strictly convex in the first argument and continuously differentiable in the second. 
Voronoi diagrams and Delaunay triangulations with respect to , of the first and seconds kinds, have been considered in the literature. 
The diagram and the dual triangulation of the \emph{first kind} were proposed by Labelle and Shewchuk~\cite{LS}, 
	while those of the second kind were discussed by Du and Wang~\cite{DW}. 
While the work of Du and Wang does not provide theoretical guarantees, 
	that of Labelle and Shewchuk provides an algorithm that is guaranteed to output \emph{a} set of sites for which the 
	Voronoi diagram of the \emph{first kind} is orphan-free, and whose corresponding Delaunay triangulation is embedded. 







\begin{lemma}[Bounded anisotropy for quadratic divergences]\label{lem:DQgamma}
If there is  such that  has ratio of eigenvalues bounded by ,
	then assumption~\ref{ass:BAA} holds. 
\end{lemma}

Note that the above condition on the bounded anisotropy of  may commonly hold in practice, 
for instance if the metric is sampled on a compact domain and continuously extended to the plane by reusing sampled values only. 

In the case of quadratic divergences, there already exists sufficient conditions to generate orphan-free Voronoi diagrams. 
In particular, it has been shown that if  is a bound on a certain measure of variation of , 
	then any (asymmetric) -net with respect to  that satisfies  
	(corresponding to a roughly  variation of eigenvalues between Voronoi-adjacent sites)
	is guaranteed to be orphan-free~\cite{avd}. 


\subsection{Normed spaces}\label{sec:Lp}

Our results also cover all normed spaces with a continuously differentiable, strictly convex norm, 
	including the  spaces, 
	but excluding the cases  and . 
\begin{lemma}[Bounded anisotropy for normed spaces]\label{lem:Lpgamma}
	Distances derived from strictly convex  norms satisfy assumption~\ref{ass:BAA}.
\end{lemma}


\subsection{Csisz\'ar f-divergences}\label{sec:Df}

Given a convex real function  with  and two measures  over a probability space , 
	Csisz\'ar's f-divergence~\cite{CsiszarTutorial} is

where  is absolutely continuous with respect to , 
	and therefore has a Radon-Nikodym derivative . 

If  is strictly convex, then the f-divergence is strictly convex in the first argument and continuously differentiable in the second 
(in this case it is also jointly convex). 
For instance, the strictly convex function  generates the Hellinger distance. 
F-divergences are functions of measures, and thus often in practice restricted to the probability simplex. 

\begin{remark}The limitation of our work to two dimensions implies that results for f-divergences 
	are limited to probability measures supported on just three atoms. 
Their applicability is thus somewhat limited, and  are only included for completeness. 
\end{remark}







\section{Primal Voronoi diagram and dual Delaunay triangulation}\label{sec:planar}




In this section we use the definition of Voronoi diagram (definition~\ref{def:VorI})
	to construct an embedded simple planar graph whose incidence relations match 
	those of the Voronoi diagram. 
We then dualize this graph to obtain an embedded simple planar graph 
	with vertices at the sites and curved edges. 
Section~\ref{sec:dual} will then show that the dual graph is also embedded when replacing curved edges by straight segments. 
Recall that we have assumed that not all sites are colinear (the colinear case is described in section~\ref{sec:summary}). 


\subsection{Assumptions}\label{sec:assumptions}

We begin by making the following two technical assumptions. 


\vspace*{0.1in}\noindent{\bf Path-connectedness.}
Assume that all connected components of Voronoi elements are also path-connected. 
In fact, given the assumption below, as well as assumptions~\ref{ass:BAA} and~\ref{ass:EGA}, 
	we only need to further assume that connected components of Voronoi \emph{edges} are path-connected. 
Indeed, Voronoi regions are open and Voronoi vertices will be shown to be composed of isolated points, 
	and therefore their connected components are automatically path-connected~\cite[p.\ 158]{munkres2000topology}.


\vspace*{0.1in}\noindent{\bf Boundaries of Voronoi regions.}
Further assume that the boundary of bounded, simply-connected Voronoi regions are simple, closed (Jordan) curves. 
For unbounded regions , we assume that they can be first mapped 
	through a continuous transformation 
	onto a bounded set , for instance through an appropriate M{\"o}bius transformation. Bounded simply-connected sets whose boundary is a Jordan curve 
	are those that are uniformly connected \emph{im kleinen}~\cite{Moore1918}\footnote{
	A space  is uniformly connected \emph{im kleinen} if for every  there is  
	such that for every pair of points  with 
		there is a connected subset  with  and . 
	}.





\subsection{Properties of Voronoi elements}\label{sec:properties}


Before constructing an appropriate primal graph from the connectivity relations of the Voronoi diagram, 
	we first establish some relevant properties of the diagram's elements. 

We say that Voronoi element  is incident to Voronoi element  
	(denoted )
	if their closures overlap  
	and . 

From this incidence relation we build a primal Voronoi graph, whose dual is the Delaunay triangulation with respect to . 
Since ``planar graphs, and graphs embeddable on the sphere are one and the same''~\cite[p.\ 247]{bondy2008graph}, 
	we consider incidence relations on the Riemann sphere (by stereographically projecting the plane onto ), 
	where the added vertex at infinity is defined to be incident to unbounded elements on the plane. 
Geometric constructions will, however, typically be carried out on the plane for convenience. 







\subsubsection{Incident elements}\label{sec:incidence}


\begin{figure}[htbp]
   \centering
\includegraphics[width=2.5in]{incidence.pdf}
\caption{A portion of a Voronoi diagram, with highlighted incidence relations between Voronoi elements. 
   		The incidence relation (definition~\ref{def:incidence}) forms a directed acyclic graph. }
   \label{fig:incidence}
\end{figure}

Consider the following definition of incidence between Voronoi regions (or between connected components of Voronoi regions):

\begin{definition}\label{def:incidence}
Given , we say that  is incident to  (written ) iff 
 
	\emph{and} . 
\end{definition}



\begin{remark}
By orphan-freedom, and lemma~\ref{lem:connectededges}, both Voronoi regions and edges are connected
	(except for isolated edges, which are defined in section~\ref{sec:propedges}).
For simplicity, in the sequel we refer to connected components of Voronoi vertices simply as ``Voronoi vertices",
	except for the statement of lemma~\ref{lem:vertexincidence}, which makes this distinction explicit. 
\end{remark}


Note that this definition and the one in section~\ref{sec:properties} are equivalent since, 
	for distinct sets , and 	by the continuity of , 
	 implies 
	(and viceversa). 







Given the following substitution rules:

the following are the incidence relations depicted in figure~\ref{fig:incidence}: 

where we often write  instead of  for simplicity. 




\begin{property}\label{prop:boundaryincidence}
	All points in the boundary of a Voronoi element  belong to either , 
		or to an element that  is incident to. \end{property}
\begin{proof}
	Let , and  be the set of sites that  is equidistant to. 
	Since , by the continuity of ,  is equidistant to all sites in , 
		and therefore . 
	The property follows from the definition of incidence. 
\end{proof}


\begin{property}
	From the properties of strict set containment, 
		it follows that the incidence relation  forms a directed acyclic graph  
	(a cycle  would imply , a contradiction). 
\end{property}



From property~\ref{prop:boundaryincidence} it follows that closed Voronoi elements 
	are those with zero out-degree in the incidence graph (e.g.\  in figure~\ref{fig:incidence}), 
and that open Voronoi elements (i.e.\ Voronoi regions) are those with zero in-degree 
	(e.g.\  in figure~\ref{fig:incidence}). 






\subsubsection{Properties of Voronoi vertices}


The main properties at Voronoi vertices are derived from the two assumptions in section~\ref{sec:setup}. 
Assumptions~\ref{ass:BAA} and~\ref{ass:EGA} are useful when deriving properties of the vertex at infinity, 
	and bounded vertices (all other vertices), respectively. 

Given the set negated gradients  at a bounded vertex point (eq.~\ref{eq:neggrad}), 
	by assumption~\ref{ass:EGA} they are distinct vertices of their convex hull. 
It is then possible to define ``outward" vectors  (eq.~\ref{eq:dkdef}) 
	such that eq.~\ref{eq:extremal1} holds. 
This is because, for each , 
	eq.~\ref{eq:extremal1} simply requires all gradients other than 
	to be below the (red dotted) line orthogonal to  passing through 
	(as shown in fig.~\ref{fig:EGA.a} for ), 
	which is possible because  are the distinct vertices of .
	
Figure~\ref{fig:EGA.b} shows that eq.~\ref{eq:extremal2} holds for the same reason as above. 
Given two gradients that are adjacent vertices of 
	(for instance ), 
	eq.~\ref{eq:extremal2} (in this case with )
	is possible whenever all gradients different from 
	are simultaneously below two lines, both passing through  and orthogonal to  and 
	(the gray area). 
This holds because the outward directions  can be chosen to form an obtuse angle with 
	both segments  and . 
The same argument applies to eq.~\ref{eq:extremal3}. 


\begin{figure}[htbp]
   \centering
	\subfloat[]{\label{fig:EGA.a}\includegraphics[width=2.2in]{EGA.pdf}}\quad\quad\quad
	\subfloat[]{\label{fig:EGA.b}\includegraphics[width=2.4in]{EGA2.pdf}}
   \caption{Diagrams used in the proof of lemma~\ref{lem:vertexincidence}. 
   		Assumption~\ref{ass:EGA} ensures that for all  there is a vector  
				with  (a), 
			and such that all intermediate direction vectors  between  and  
				satisfy  (blue lines)
				and  (red lines) (b). }
   
\end{figure}


\begin{lemma}[Incidence at Voronoi vertices]\label{lem:vertexincidence}
	A Voronoi vertex  is a collection of discrete points, 
		at each of which there is an ordered set of indices  
		such that  and the following incidence relations hold:
	
	Additionally, if an edge  is incident to a vertex , then .\\
	If  is the vertex at infinity (), then  
		are the indices of the sites in the boundary of the convex hull , 
		in either clockwise or counter-clockwise order. \end{lemma}
\begin{proof}
\noindent{\bf [Bounded vertices, ]}.
Let  be a Voronoi vertex not at the point at infinity and 
	 be a point in . 
By the extremal gradient assumption (assumption~\ref{ass:EGA}), 
	the negated gradients 
	
are distinct vertices of their convex hull. 
Let  be the indices in  ordered (for instance clockwise) 
	around , as shown in figure~\ref{fig:EGA.a}. 



Since , with  are distinct vertices of their convex hull, 
	it is easy to show that there are direction (unit) vectors , with , 
	such that for all  it holds:
	
For instance 
	



By the multivariate version of Taylor's theorem~\cite[p.\ 68]{konigsberger2006analysis}, 
	for each , and , we may write:
	
For each , and  with , let , with , 
	and let . 
It then follows that:
	
where . 
Note that, crucially,  depends on  but not on the direction . 



Since  with , 
	we can pick constants  sufficiently small so that 
	for all  it holds , 
	and therefore . 
Let  be the minimum of all , with , and . 



Since  is strictly closest to sites , 
	let  be small enough so all points  with  
	are closest only to sites in  (which is possible since  is continuous). 
Consider the set of points in a small circle of radius  around . 
From the above, we have that at the point , it holds:
	
from which it follows that  is strictly closer to  than to any other site. 
Since this is true for all  and for all sufficiently small , the incidence relations
	
follow. 

Because  are vertices of , 
	it is clear, as shown in figure~\ref{fig:EGA.b}, 
	that for each  there are constants , with  and , 
	such that, for every unit vector  intermediate between  and , 
it holds:
	
Let  be small enough such that for all , 
	it holds . 
Let , 
	then for all , 
	and every point  it holds:
	
	and therefore  is closest to either , or to both. 
For each such , and for each  , 
	by the intermediate value theorem, 
	there is a direction vector  between  such that  
	is in . Note that, by the above construction, for every such sufficiently small , 
	, with , are the only Voronoi edges inside the ball of radius  around . 
From this it directly follows that: 
\begin{enumerate}
	\item since all points , with unit vector  and sufficiently small  
			have been shown to be in a Voronoi region or edge, 
			 is an isolated point of ;
		since  is a generic point of , it follows that  is composed of isolated points;
	\item it holds ; and 
	\item if a Voronoi edge  is incident to , then , 
		since the only edges incident to  are , with . 
\end{enumerate}


\vspace*{0.1in}\noindent{\bf [Vertex at infinity, ]}. 
Incidence to the vertex at infinity is dealt with in section~\ref{sec:boundary}, 
	where lemma~\ref{lem:VW} shows that the only unbounded elements 
	are of the form  where all  are vertices of , 
and lemmas~\ref{boundary_easy} and~\ref{lem:hard} show that, 
	if  are the vertices on the boundary of  (whether on an edge or vertex of ), 
	ordered around , 
	then  and  are the only unbounded elements
	(and therefore incident to ). 
In this sense we can say that the vertex at infinity  is the Voronoi vertex . 
The proofs in section~\ref{sec:boundary} show that points  in any circle of sufficiently large radius are incident 
	only to sites in , 
	that  cannot be incident to more than two sites simultaneously (lemma~\ref{lem:contrad}), 
	and therefore  cannot belong to a Voronoi vertex, 
	and finally that  can \emph{only} be simultaneously closest to two consecutive sites of the form 
	(page~\pageref{text:boundary}). 
Note that the relevant proofs of section~\ref{text:boundary} use the bounded anisotropy assumption (assumption~\ref{ass:BAA}), 
	but do not use any result from this section. 
\end{proof}

From the proof of lemma~\ref{lem:vertexincidence}, 
	it is clear that 
	the bounded anisotropy assumption (assumption~\ref{ass:BAA}) 
	is constructed so that lemma~\ref{lem:vertexincidence} holds for the vertex at infinity, 
	while the extremal gradient assumption (assumption~\ref{ass:EGA}) 
	is meant to ensure that  lemma~\ref{lem:vertexincidence} holds for regular (bounded) vertices.




\subsubsection{Properties of Voronoi edges}\label{sec:propedges}



We begin by considering (isolated) Voronoi edges that are bounded and not incident to any Voronoi vertex. 
Since, as will be shown in lemma~\ref{lem:SCedges}, 
	Voronoi edges are simply connected, 
	it is easy to see that for any Voronoi edge 
		that is not incident to any bounded Voronoi vertex,
	it can only be  or , 
	and  cannot be involved in any other incidence relation. 
To see this, first note that an isolated component of  has, 
	by definition, zero out-degree, and therefore it is closed. 
Because  is not incident to the vertex at infinity, it is bounded. 
Since  implies 
	that their common boundary belongs to vertex  (where it may be ), 
	 is not incident to any Voronoi edge. 
 cannot be incident to a region  with , 
	or else their common boundary would belong to vertex . 
Finally, we show that it cannot be 
	both  and . 
Because  is closed, simply connected, and bounded, 
	by the continuity of , 
	we can consider a sufficiently small  
	such that every -offset of its outer boundary cannot be 
	closest to any site  with . 
If , then 
	there must be  
	such that the -offset  of 's outer boundary 
	has at least one point closest to , and one point closest to , 
and therefore, by continuity of , at least one point equally close to . 
Since all points in  are closest to  only, then  has 
	been shown to have a point in , 
	contradicting the fact that  is a -offset of 's outer boundary, 
	and therefore outside . 


Let  be an bounded isolated Voronoi edge 
	such that . 
Because they are not incident to any Voronoi vertex, 
	bounded isolated edges will not be considered part of the primal Voronoi graph.
For simplicity, we consider all points of an isolated edge  to be part of its containing
	Voronoi region (say ), 
	and therefore to be (by definition) strictly closer to  than to any other site. 
This is not just a simplification (which does not affect the final Voronoi graph), 
	but will allow us to prove that Voronoi regions are simply connected. 
	








We begin by proving the following technical lemma. 





\begin{figure}[htbp]
   \centering
   	\subfloat[]{\label{fig:RSC.a}\includegraphics[width=2.5in]{SC1.pdf}}
	\quad\quad\quad
	\subfloat[]{\label{fig:RSC.b}\includegraphics[width=2.5in]{SC2.pdf}}
   \caption{Diagrams used in the proof of lemma~\ref{lem:RSC}. }
   \label{fig:RSC}
\end{figure}



\begin{lemma}\label{lem:RSC}
	Let the boundary  of  be a simple, closed path, 
		and  be a Voronoi element of an orphan-free diagram. 
	If , then . 		
\end{lemma}
\begin{proof}
Let , and . 
We begin by showing that  does not contain any site  whenever  or . 


Let , and  with , as in figure~\ref{fig:RSC.a}. 
Let  be the ray starting from  in the direction of  
	(note that  may be inside or outside ). 
Since  is unbounded and  is bounded, then part of  is outside  and, 
	by the Jordan curve theorem, it must intersect  at some point . 
Since , 
	 is closest to , while  is closest to 
		(since  and  is non-negative and convex). 
By the continuity of , there is an intermediate point  between  and  
	that is equidistant to  and , contradicting lemma~\ref{lem:midpoint}. 

Let , and let  be any site (figure~\ref{fig:RSC.b}). 
Pick  among , 
	which is always possible because . 
The argument is identical in this case, except that, 
	because , then  is closest and equidistant to , 
	and therefore closer to  than to , and the same argument holds. 

\vspace*{0.05in}\noindent{\bf [Voronoi regions]}.
We now prove that no point  belongs to a Voronoi region  if  or . 
Let  belong to , with  or , we show that this leads to a contradiction. 

We first show that . 
Assume otherwise. Since  is open and connected (by the orphan-freedom property), it is path connected. 
Let  be a simple path from  to a point  outside . 
By the Jordan curve theorem,  intersects , 
	which leads to a contradiction whenever  or . 

Since  and, 
	by lemma~\ref{lem:site}, 
	, then , contradicting the fact that  does not contain any site  if  or . 

\vspace*{0.05in}\noindent{\bf [Voronoi vertices]}.
If  contains a point  that belongs to a Voronoi vertex  with , 
	then  must be in the interior of , since its boundary  is in . 
By lemma~\ref{lem:vertexincidence},  is incident to , 
	where  and . 
Since  is in the interior of , then there are points 
	 that belong to , respectively. 
If , then this contradicts the fact that  does not have any point in a Voronoi region. 
If , since , then one of  must be different from , 
	contradicting the fact that  does not have any point in a Voronoi region different from . 

\vspace*{0.05in}\noindent{\bf [Voronoi edges]}.
Let  be a connected component of a Voronoi edge, with . 
If some point  is in , 
	then , or else since, by the assumption in section~\ref{sec:assumptions}, 
	 is path connected, there would be a path  connecting 
	 to a point of  outside . 
By the Jordan curve theorem  would intersect , a contradiction. 

Since we have already discarded isolated Voronoi edges that are not incident to any Voronoi vertex, 
	a Voronoi edge is always incident to a Voronoi vertex and, 
	since  is in the interior of , then its incident Voronoi vertex is in , 
	a contradiction. 

Finally, since we have shown that there cannot be any Voronoi vertices, edges, or regions 
	with  in , then it must be . 
\end{proof}






























\begin{figure}[htbp]
   \centering
	\subfloat[]{\label{fig:connectededges.a}\includegraphics[width=2.7in]{conedges.pdf}}
	\quad\quad\quad
	\subfloat[]{\label{fig:connectededges.b}\includegraphics[width=2.7in]{Vinf.pdf}}
   \caption{Diagrams used in the proof of lemmas~\ref{lem:connectededges} (a), 
   			and~\ref{lem:val_ge2} (b).}
\end{figure}


\begin{lemma}\label{lem:connectededges}
	Voronoi edges of an orphan-free diagram are connected. 
\end{lemma}
\begin{proof}



Let  be two disconnected pieces of a Voronoi edge ,
	as shown in figure~\ref{fig:connectededges.a}.
Since we have discarded (bounded) isolated edges, we assume that  
	 is incident to at least one vertex, and therefore
	by lemma~\ref{lem:vertexincidence},
	 it is  and . 



Since  are incident to both , 
	the boundaries of  and  overlap
	(and likewise ). 
Let  
	be a point in the common boundary between  and , 
	and  be a point in the common boundary between  and , 
	and define the points  analogously.
Since  are disjoint, 
	it holds  and , 
	and therefore by lemma~\ref{lem:regionpath}
		 there are non-crossing simple paths  
		 	from  to , respectively, 	
		 and non-crossing simple paths  
		 	from  to , respectively. 
Additionally, since by the assumption in section~\ref{sec:assumptions}  are path connected, 
	there are simple paths  and 
	connecting  to , and  to , respectively. 
Let  be the concatenation of paths , 
	and  be the concatenation of paths . 
By construction, and since  are disjoint, 
	the simple paths  only meet at their endpoints . 



Let  be the simple closed curve resulting from concatenating . 
By the Jordan curve theorem,  divides the plane into an interior () and exterior regions, bounded by . 
We first show that  does not contain any sites (other than ). 


\vspace*{0.1in}\noindent{\bf [ contains no sites]}.
We first divide  in three parts, as shown in figure~\ref{fig:connectededges.a}: 
\begin{enumerate}
	\item the region  bounded by , , and ,
	\item the region  bounded by , , and , 
	\item and .
\end{enumerate}
We begin by observing that if , then the triangle  
	cannot contain any site (other than )
	because 1)  is closest and equidistant to ,
	and 2) the ball of the first kind  
	centered at  with  in its boundary (see table~\ref{table:notation}) 
	is convex and therefore contains . 
Since the sides of  are line segments, and  is strictly convex, 
	the only points of  touching the boundary of 
	are , and therefore a site at any other point in  would be strictly closer to  than , 
	a contradiction. 



Since  can be written as the union of triangles with vertices  with , 
	then  does not contain any site other than . 
An analogous argument proves that  does not contain any site other than .



We split the remaining region  into four parts . Let  be the part of  bounded by the segment  and the curve . 
Let  be the union of segments connecting  to points in . 
Clearly, it is . We show that  cannot contain any site other than , 
	and thus the same is true of . 



Let  be a site, and let  be the point such that . 
Because ,  is closest and equidistant to  (and possibly also to ), 
	that is:  for all . 
Since  and , we can write , with , 
	and therefore by the strict convexity of  it holds:
	
where the last equality follows from , and the last inequality follows from the non-negativity of . 
This shows that the site  is \emph{strictly} closer to  than , a contradiction. 
Therefore there are no sites in , and thus no sites in  either. 
Applying an identical argument to  shows that  cannot contain any sites other than . 


\vspace*{0.05in}\noindent{\bf [Points in  can only be closest to  and/or ]}.
We begin by showing that there is no point  that is 
	\emph{strictly} closer to a site  than to any other site (). 
If  is closest to , then we first show that  is wholly contained in . 
Assume otherwise, and pick a point  outside . 
Since Voronoi regions are path-connected, let  be a path connecting . 
By the Jordan curve theorem,  crosses the boundary , 
	contradicting the fact that . 
Since  is completely inside  then, by lemma~\ref{lem:site}, it is , 
	contradicting the fact the  contains no sites other than , and therefore  with . 


We now show that no point  can be closest to , even if it is also simultaneously closest to 
	 and/or . 
Since  is closest to , 
	and the boundary of  is , then  belongs to the interior of . 
By definition,  belongs to a Voronoi edge or vertex. If it belongs to a Voronoi vertex and is closest to  then, 
	by lemma~\ref{lem:vertexincidence}, and since Voronoi vertices are composed of isolated points, 
	 is incident to , a contradiction since  whenever . 
Therefore  does not contain any Voronoi vertices. 



Finally, we show that no point  can be closest to a site  
	and belong to a Voronoi edge . 
Since  is in the interior of , the connected component  of  that  belongs to 
	must be fully contained in , 
	or else by the Jordan curve theorem  would be separated by the boundary  of . 
Since we have discarded connected components of Voronoi edges not incident to any Voronoi vertex, 
	then  is incident to some vertex . 
Since  is in the interior of , then  must be contained in . 
As we have shown above,  does not contain any Voronoi vertex, and therefore  cannot be closest to . 





\vspace*{0.1in}\noindent{\bf [ is connected]}.
Finally, we show that there is a path in  connecting  to , 
	and therefore  is connected. 
Recall that all points in  can only be closest to  and/or , 
	that  are simple paths from  to , 
	and that, by construction, they do not meet except at their endpoints. 
Clearly,  are path homotopic~\cite[p.\ 323]{munkres2000topology}, 
	for instance via the straight-line homotopy. 



We begin by constructing a path homotopy  between  and 
	(a continuous function  
		such that  and )
	contained in . 
Since  is a Jordan curve, and  is simply connected, by Carath\'eodory's theorem~\cite{conformal}, 
	there is a homeomorphism  from  to the closed unit disk  that maps  to the unit circle. 
Since  and  is convex, the straight-line homotopy  
	between   and  is contained in . 
We can now inversely map this homotopy through  to obtain a path homotopy  
	between  and  which is contained in  
	(i.e.\  with ). 



Since every path  with  starts at  and ends at , 
	and  is continuous, there is  such that  is equidistant to . 
Since we have shown above that all points in  are closest to  and/or , 
	then  for . 
By the continuity of  and , is it possible to choose  to be continuous with , 
	and such that the path  
	with  is . 
Since the path  is defined to start at  and end at , then 
	 and  are connected, 
	and therefore  must be connected. 
\end{proof}


\begin{lemma}\label{lem:SCedges}
	Voronoi edges of orphan-free diagrams are simply connected. 
\end{lemma}

\begin{proof}

Recall that, by the assumption in section~\ref{sec:assumptions}, 
	connected Voronoi edges are also path connected.


Let  be a Voronoi edge, and  be a simple path not contractible to a point. 
By the Jordan curve theorem,  divides the plane into an exterior (unbounded), 
	and an interior (bounded) region . 
By lemma~\ref{lem:RSC}, , and therefore  is contractible to a point. 

\end{proof}





\subsubsection{Properties of Voronoi regions}



\begin{lemma}\label{lem:regionSC}
Voronoi regions of orphan-free diagrams are simply connected.
\end{lemma}

\begin{proof}
Let  be a Vornoi region, which must be connected since the diagram is orphan-free. 
Since  is open, it is path connected~\cite[p.\ 158]{munkres2000topology}.



Assume that  is not simply connected, 
	and therefore has a closed simple path  that is not contractible to a point. 
By the Jordan curve theorem the path  separates the plane into an exterior and an interior region . 
By lemma~\ref{lem:RSC}, , and therefore  is contractible to a point. 
\end{proof}




\begin{lemma}\label{lem:regionpath}
For every Voronoi region  of an orphan-free Voronoi diagram, 
there is a collection of simple paths connecting the site  to each point in the boundary of , 
	 such that:
\begin{enumerate}
	\item all paths are contained in ,
	\item paths intersect the boundary  only at the final endpoint, and 
	\item two paths meet only at the starting point .  
	\end{enumerate}
\end{lemma}
\begin{proof}
By the assumption in section~\ref{sec:assumptions}, the boundary of Voronoi regions are simple closed paths. 
Since a Voronoi region  is also simply connected (lemma~\ref{lem:regionSC}), we may use Carath\'eodory's theorem~\cite{conformal}
	to map  to the closed unit disk  through a homeomorphism   
	that maps the boundary  to the unit circle. 
Since, by lemma~\ref{lem:site},  is an interior point of , 
	then  is an interior point of . 
We now simply construct a set of straight paths from  to each point in the unit circle. 
These paths are contained in , and meet only at the starting point. 
We map them back through  to obtain the desired set of paths. 
\end{proof}



\subsection{Voronoi edges are incident to two and only two Voronoi vertices}



\begin{lemma}\label{lem:val_ge2}
No Voronoi edge is incident to just one Voronoi vertex.
\end{lemma}
\begin{proof}
Let  be a Voronoi edge incident to just one Voronoi vertex . 
By lemma~\ref{lem:vertexincidence}, it is , 
	and therefore  has a common boundary with . 
Recall from property~\ref{prop:boundaryincidence} that the boundary  belongs 
	to Voronoi edges and vertices to which  is incident. 
Since, by lemma~\ref{lem:vertexincidence}, Voronoi vertices are isolated points, 
	and two Voronoi edges  can only meet at a Voronoi vertex  (with ), 
	we can enumerate an alternating sequence of Voronoi edges and vertices 
	 in clockwise order around , 
	in which every edge is incident to the previous and next vertices in the sequence. 
Therefore, a Voronoi edge can only be incident to one Voronoi vertex if the sequence is . 

If  is not the vertex at infinity, then we can show that the above is not possible 
	with an argument identical to the proof of lemma~\ref{lem:RSC} (figure~\ref{fig:RSC}). 
Note that  implies , 
	and therefore all points in  are equidistant to . 
Let , and consider the ray  from  in the direction  
	which, since  is unbounded and  is bounded (since it is not incident to ), 
	it must cross  at some point . 
Since ,  is equidistant to , contradicting lemma~\ref{lem:midpoint}. 

If  is the vertex at infinity, then  is not incident to any Voronoi vertex, and is unbounded. 
Therefore,  does not cross any Voronoi edge, 
	or else  would be incident to their intersection point (a Voronoi vertex). 
Recall from lemma~\ref{lem:midpoint} 
	that 	 can never intersect the supporting line  of  outside the segment . 
Let  () be the ray starting at  () with direction  (), 
as shown in figure~\ref{fig:connectededges.b}. 
It can be easily shown that every point in  () is strictly closer to  () than to  (). 
Since, regardless of the choice of origin, 
	\emph{every} origin-centered circle  of sufficiently large radius  intersects  at 
	exactly one point  in , and one point  in ,
	the following holds. 
Let  divide  into two half spaces , 
	and let  and
		    . 
Since  () is closer to  () than to  (), 
	and  are the endpoints of , 
	by the continuity of , 
	there are points  and  equidistant to . 
Since  does not intersect any Voronoi element, then  are also closest to . 
Because this holds for all sufficiently large , 
	then both  and  are unbounded, 
	contradicting lemma~\ref{lem:contrad}, 
	which states that every point  that is sufficiently far from the origin
	and equidistant to  (and therefore its closest point in  lies in ) 
	is closer to a site in  than to . 
\end{proof}






\begin{figure}[htbp]
   \centering
   	\subfloat[]{\label{fig:closureedge.a}\includegraphics[width=2.7in]{W.pdf}}\quad\quad
	\subfloat[]{\label{fig:closureedge.b}\includegraphics[width=2.7in]{closureedge.pdf}}
\caption{From assumption~\ref{ass:EGA}, it follows that every Voronoi edge , 
   			in the close vicinity of a Voronoi vertex can be written as the graph of a function  with an endpoint at  (a). 
		Figure (b) shows a hypothetical Voronoi edge  that breaks assumption~\ref{ass:EGA}, 
			for which lemma~\ref{lem:closureedge} does not hold. }
   \label{fig:closureedge}
\end{figure}



\begin{lemma}\label{lem:closureedge}
Let  be a Voronoi edge. 
For every  and  there is a simple path  
	such that , , and . 
\end{lemma}

\begin{proof}
\noindent{{\bf [Case ]}}. 
Recall that connected components Voronoi edges are assumed to be path-connected (section~\ref{sec:assumptions}). 
Since Voronoi edges are connected (lemma~\ref{lem:connectededges}), they are path-connected. 
Therefore, if , there is always a path  connecting . \\

\noindent{{\bf [Case ]}}. 
In this case, by property~\ref{prop:boundaryincidence},  must belong to a Voronoi element of higher order than 
	(a Voronoi vertex ), to which  is incident (with ). 
Since, by lemma~\ref{lem:vertexincidence}, Voronoi vertices are composed of isolated points, 
	then  is a connected component of  (possibly the vertex at infinity). 
Consider separately whether  is the vertex at infinity. \\

\noindent{{\bf [Case  and  is not the vertex at infinity]}}.
Recall that the proof of lemma~\ref{lem:vertexincidence}
	defines an ordering of , and a set of associated direction vectors . 
Let , with , 
	and let  be the unit vector orthogonal to 
in the direction outgoing from 
	(which exists since, by assumption~\ref{ass:EGA}, it is ). 
We assume, without loss of generality, that the coordinate representation of  
	is . 
Since  and , by the implicit function theorem, 
	there is an open  ball  around  in which the implicit equation 
	can be written as , with , 
	as shown in figure~\ref{fig:closureedge.a}. 
	
Since  is incident to  at , 
	there is  such that . 
Choose  to be sufficiently small for the conditions of the proof of lemma~\ref{lem:vertexincidence} to apply
	(in particular , as defined in the proof). 
Let  be a circular wedge contained in the  ball , 
	and bounded by the rays  and  which, 
	aside from , only contains points strictly closer to  than to all other sites.  
From the definition of  it is clear that the segment  
	is contained in .

Since , and inside  all points with the exception of  are closest only to , 
	the implicit equation  represents the set of points in . 
Since  can be written in coordinates as  inside , 
	it is clear that, inside ,  is a simple curve, and that this is the only part of  incident to . 


Given , find any point  that is closer to  than . 
Because , there is a simple path  connecting  to  and, 
	because  is in , there is also a simple path 
	 from  to 
	(part of the curve  of figure~\ref{fig:closureedge}). 
Finally, because  is closer to  than  is, 
	the paths  and  do not cross, and therefore the 
	concatenation of  and  
	meets the requirements of the lemma. 
	







\noindent{{\bf [Case  and  is the vertex at infinity]}}.
Since  then, by definition,  is unbounded. 
Let  and, for each , let  be at distance . 
One can always find such a sequence of points because  is unbounded and path-connected
	(if there is no  at distance  then the circle with center at  and radius  would disconnect ). 
Let  be paths connecting  to , 
	and  be the concatenation of , 
	where , with  and . 

Define  as . 
Consider  on the Riemann sphere, transformed through a stereographic projection. 
Since  is continuous and  has an accumulation point at the point at infinity (north pole on the sphere), 
	it is continuous on the sphere. 
If  is not simple, it can be appropriately cut and reparametrized until it is
(i.e.\ by tracing the path and, upon arrival to a point  where the path crosses itself, 
		cutting out the next portion up to the highest  for which , 
		and proceeding this way to the end of the path). 




\end{proof}


Note that for lemma~\ref{lem:closureedge} to hold it is crucial that edges  are incident to vertices  
	as a curve arriving at  from a single direction, 
	as illustrated in figure~\ref{fig:closureedge.a}.  
To see that assumption~\ref{ass:EGA} is required, 
	consider figure~\ref{fig:closureedge.b}, 
	which depicts an edge  incident to two vertices 
	which do not satisfy assumption~\ref{ass:EGA}, 
	in which \emph{every} path connecting the two disks passes through either  or , 
	and therefore for which lemma~\ref{lem:closureedge} does not hold. 

	

\begin{figure}[htbp]
   \centering
	\subfloat[]{\label{fig:tree.a}\includegraphics[width=3.0in]{tree1.pdf}}
	\quad
	\subfloat[]{\label{fig:tree.b}\includegraphics[width=3.0in]{tree2.pdf}}
   \caption{The construction of a tree (blue) inside an edge  (green region), with root  and leafs 
   			at its incident Voronoi vertices .}
   \label{fig:tree}
\end{figure}


\begin{lemma}\label{lem:tree}
In an orphan-free diagram, 
	for every Voronoi edge  that is incident to Voronoi vertices , 
	there is an embedded tree graph in  whose leafs are .
\end{lemma}

\begin{proof}

Unless otherwise specified, we assume in this proof that all paths are simple, contained in , 
	parametrized over the unit interval , 
	and that, using lemma~\ref{lem:closureedge},
	there is a path connecting any two points in  that does not intersect a Voronoi vertex
	(expect perhaps at the endpoints). 
We use throughout the fact that Voronoi edges are path connected (lemma~\ref{lem:SCedges} and section~\ref{sec:assumptions}). 



If , pick a point  as root and, 
	using lemma~\ref{lem:regionpath},
	 consider a simple path  connecting  to , 
	then the tree with vertex set , and edge set  meets the requirements of the lemma. 





For each , assume that there is an embedded tree graph  
	with  as leafs. 
We construct a new embedded tree  as follows (figure~\ref{fig:tree}). Let  be the root of , 
	and let  be a simple path connecting  to  which, 
	making use of lemma~\ref{lem:closureedge}, is chosen such that it does not intersect 
	any Voronoi vertex (other than the final endpoint). 
Let 
	
	which always exists because  is closed 
	and . 
Let  be the ``last" point along  that belongs to .
Because  
	then it must be . 
Additionally,  cannot be a Voronoi vertex, 
	since  doesn't intersect Voronoi vertices except at the final endpoint . 



Let  be the path , 
	that is, the part of  from  to . 
We construct a new tree graph  as follows. 
Begin by setting  equal to . 
We then insert a new vertex  into . 
Next, we proceed differently depending on whether  
	is a vertex, or it belongs to an edge of 
(note that, since  is not a Voronoi vertex, it cannot be a leaf vertex of ). 




If  is an internal vertex of , 
	as in figure~\ref{fig:tree.a},
	then we add a new edge  to  connecting vertices  and . 
Since, by construction,  does not cross any edge in , the tree graph remains embedded. 



If, on the other hand,  belongs to an edge  of  connecting vertices , 
	as shown in figure~\ref{fig:tree.b}, 
	then:
	\begin{enumerate}
	\item we insert a new (internal) vertex  into ;
	\item we split  into two edges: 
		 and , connecting , and , respectively; 
	\item we insert a new edge  connecting vertices  and . 
	\end{enumerate}
Note that 
	the edge  is split into two edges that represent the same set of points, 
	and therefore, since  didn't cross any edges of , 
	then  does not cross any edge of . 
Hence, since  is an embedded tree graph, 
	the new tree  is also embedded
	 and has  as leafs. 

The lemma follows by induction on .  
\end{proof}


\begin{figure}[!h]
   \centering
\includegraphics[width=3.5in]{K3_graph2.pdf}
   \caption{By assuming that a Voronoi edge  is incident to three Voronoi vertices , 
   		we can construct a planar embedding of the non-planar graph , a contradiction. 
		The more general figure~\ref{fig:planarity.a} 
			 further illustrates the proof of lemma~\ref{lem:val_le2}. }
\label{fig:K33}
\end{figure}

The final lemma of this section can be used in conjunction with lemma~\ref{lem:val_ge2} 
	to establish that Voronoi edges are incident to exactly two Voronoi vertices. 
We sketch here the argument that shows that a Voronoi edge  cannot 
	be incident to three vertices  (figure~\ref{fig:K33}). 
The general case in the proof of lemma~\ref{lem:val_le2} follows a similar argument. 
We first use lemma~\ref{lem:tree} to build a tree inside  with leafs at , 
	and show that it can be collapsed into a star-graph with a vertex , and non-crossing edges 
		, as shown in the figure. 
The incidence rules of lemma~\ref{lem:vertexincidence}, as well as lemma~\ref{lem:regionpath} allows us 
	to construct six non-crossing edges from  and , to , respectively. 
We have just constructed an embedding of a graph which can be easily shown to be the non-planar graph , 
	thereby reaching a contradiction. 







\begin{figure}[!h]
   \centering
	\subfloat[]{\label{fig:planarity.a}\includegraphics[width=2.8in]{graph1.pdf}}
	\quad\quad
	\subfloat[]{\label{fig:planarity.b}\includegraphics[width=2.8in]{graph2.pdf}}
   \caption{Every Voronoi edge  (green region) incident to  Voronoi vertices 
   			allows the construction of an embedded planar graph  connecting 
   			a tree  inside , to the sites  (a). 
		This graph has a minor  obtained from  by contracting edges of . 
		 can be shown not to be planar for , and therefore Voronoi edges are incident to no more than two Voronoi vertices. 
		 }
   \label{fig:planarity}
\end{figure}



\begin{lemma}\label{lem:val_le2}
Voronoi edges of an orphan-free diagram are incident to no more than two Voronoi vertices.
\end{lemma}

\begin{proof}
	Let  be a Voronoi edge incident to Voronoi vertices . 
Since , 
		and Voronoi vertices are of higher order () than Voronoi edges, 
		by the definition of incidence (definition~\ref{def:incidence}), it is , with . 
	We prove the lemma on the sphere , where any of the Voronoi vertices may be the vertex at infinity. 
	Note also that some of the sets  with  may be equal, 
	since Voronoi vertices have not yet been shown to be connected. 
	

By lemma~\ref{lem:vertexincidence}, 
	the vertices  are isolated points
	(possibly the point at infinity), 
	and . 
We begin by assuming that , and build an embedded planar graph  (figure~\ref{fig:planarity.a}). 
We then show that  can only be planar if , reaching a contradiction. 


By lemma~\ref{lem:tree}, there is an embedded tree graph 
	with  as leafs. 
We begin by setting  equal to . 
We then insert the vertices  and  in  (as shown in figure~\ref{fig:planarity.a}). 
Since , 
	by lemma~\ref{lem:regionpath}, 
	there are non-crossing paths , with , 
	connecting  to  and non-crossing paths , with , 
	connecting  to . We insert the above paths , , as edges of . 
Aside from all paths  () only crossing
	 at their starting point,
all paths  () are, by lemma~\ref{lem:regionpath},
	 contained (except for their final endpoint) 
	in the interior of  (), 
	and therefore they can only cross an edge of  at an endpoint. 
 is therefore embedded in , and so it is a planar graph. 


Recall that the minors of a graph are obtained by erasing vertices, erasing edges, or contracting edges, 
	and that minors of planar graphs are themselves planar~\cite[p.\ 269]{bondy2008graph}. 
We now construct an appropriate minor  of the planar graph , 
	shown in figure~\ref{fig:planarity.b}, 
	and prove that it is non-planar whenever , creating a contradiction. 




Clearly, every tree  satisfying the conditions of lemma~\ref{lem:tree} has a minor  
	directly connecting the root to each leaf ,  (see figure~\ref{fig:planarity.b}), 
	which is obtained by successively contracting every edge of  that connects two internal vertices. 
We apply the same sequence of edge contractions to obtain  from , as shown in figure~\ref{fig:planarity}. 



Let  be the root of , and  be edges from  to , with . 
The minor  has vertex set
	
and edge set
	
and therefore  has  vertices and  edges.  
Since (as is easily verified) every cycle in  has length four or more, and  is planar, 
	then it holds , where  is the number of faces. 
Using Euler's identity for planar graphs, ~\cite{bondy2008graph}, 
	and the fact that , , and , it follows that , 
	and therefore  is not planar whenever 
(for instance,  is the utility graph ). 




Since  leads to a contradiction, it follows that every Voronoi edge is incident to at most two Voronoi vertices.
\end{proof}

\subsection{Primal Voronoi graph and dual Delaunay triangulation}\label{primaldual}

We use the results in this section to construct a graph from the 
	incidence relations of an orphan-free Voronoi diagram, 
	and dualize it into a planar embedded graph. 

Let the \emph{primal Voronoi graph} 
	of an orphan-free Voronoi diagram be defined as follows. 
The vertices  are the connected components of Voronoi vertices. 
Since, by lemma~\ref{lem:vertexincidence}, 
	Voronoi vertices are composed of isolated points, then  is a collection of isolated points. 
By lemmas~\ref{lem:val_ge2} and~\ref{lem:val_le2}, 
	Voronoi edges that are incident to some Voronoi vertex are incident to exactly two Voronoi vertices. 
For each Voronoi edge  incident to some Voronoi vertex, 
	we include in  an edge connecting the vertices in  
	corresponding to the connected components of Voronoi vertices that  is incident to. 
By lemma~\ref{lem:closureedge}, for each such Voronoi edge  there is a simple path in 
	connecting the two Voronoi vertices incident to , and therefore  is an embedded planar graph. 



\begin{figure}[htbp]
   \centering
	\subfloat[]{\label{fig:planar.a}\includegraphics[width=2.3in]{planar1.pdf}}
	\quad\quad
	\subfloat[]{\label{fig:planar.b}\includegraphics[width=2.3in]{planar2.pdf}}
   \caption{Diagrams used in the proof of theorem~\ref{th:simpleplanar}. }
\end{figure}



\vspace*{0.1in}\noindent{\bf Theorem~\ref{th:simpleplanar}}. \emph{
Let  be the dual of the primal Voronoi graph corresponding to an orphan-free Voronoi diagram, then  is a simple, connected, planar graph. 
}
\begin{proof}
The dual graph  is constructed by dualizing  
	and using the natural embedding described in~\cite[p.\ 252]{bondy2008graph}, 
	in which dual vertices are placed inside primal faces (at the sites in this case), 
	and dual edges cross once their corresponding primal edges. 
From this construction,  is an embedded planar graph~\cite[p.\ 252]{bondy2008graph}, 
	and is connected by virtue of being the dual of a planar graph~\cite[p.\ 253]{bondy2008graph}. 

We show that  is simple (edges have multiplicity one, and there are no loops: edges incident to the same vertex). 
Edges of  are one-to-one with edges of . 
In turn, edges of  correspond to Voronoi edges, and these are, 
	by lemma~\ref{lem:connectededges}, 
	connected. 
Therefore the edges of  have multiplicity one. 

Since loops and cut edges (those whose removal disconnects the graph) 
	are duals of each other~\cite[p.\ 252]{bondy2008graph}, 
	we now show that  has no cut edges, and therefore  has no loops. 

By~\cite[p.\ 86]{bondy2008graph}, 
	an edge of  is a cut edge iff it belongs to no cycle of . 
To every edge of  corresponds an Voronoi edge  that is incident to two Voronoi vertices. 
By lemma~\ref{lem:vertexincidence}, 
	 is incident to at least one Voronoi region . 
We next show that the Voronoi elements in the boundary of every Voronoi region  form a cycle, 
	and therefore  belongs to a cycle, so it cannot be a cut edge. 

Clearly, the boundary  of  is composed of Voronoi edges and Voronoi vertices, 
	since  is not possible because  (see section~\ref{sec:incidence}). 
Let  be the sequence of elements around the boundary of , 
	with . 
	We show that  is a cycle. 

\vspace*{0.05in}\noindent{\bf [ has no repeated Voronoi vertices]}. 
By the assumption of section~\ref{sec:assumptions}, 
	Voronoi regions have boundaries that are simple closed curves (in ). 
Note that, because vertices are isolated points, there are no repeated vertices in  since the boundary of  is a simple curve. 


\vspace*{0.05in}\noindent{\bf [ has no repeated Voronoi edges]}. \\
Let  appear twice in  as , 
where  are Voronoi vertices, as in figure~\ref{fig:planar.a}. 
Let  be two points in each of the two common boundaries between  and . 
By lemma~\ref{lem:regionpath}, there are simple paths  from  to , respectively, 
	which only meet at the initial endpoint (figure~\ref{fig:planar.a}). 
Since  is simply connected, we can consider a simple path  connecting . 
Let  be the simple closed path obtained by concatenating  
	which, by the Jordan curve theorem divides the plane into a bounded region , and an unbounded region. 
Since it must be  or , 
	assume without loss of generality that , and note that it cannot be , 
	since  is bounded. 
We show that  is not possible, 
	and therefore that  has no repeated elements.

Let  and let  be , which always exists because . 
By lemma~\ref{lem:vertexincidence}, there is a point . 
Since  is path connected, and the boundary of  is , then , 
	and therefore . 
We show that  cannot contain any sites other than , reaching a contradiction.  

Recall that the boundary  of  is the concatenation of 
	, , and , 
	and that , as in figure~\ref{fig:planar.b}. 
Let  be the union of segments from  to every point in :

Since it is clearly , it suffices to show that  does not contain any site  different from .
Every segment of the form  
	with  or  cannot contain a site  
	or else, by the convexity of ,  would be closer to  than to . 
Similarly, every segment of the form  with 
	 cannot contain a site , or else by the convexity of ,  would be closer to  than to . 
	

Since every Voronoi edge  is part of a cycle, it cannot be a cut edge, and therefore its dual has no loops. 
\end{proof}









\section{Embeddability of the Delaunay triangulation}\label{sec:dual}





Let  be the dual of the primal Voronoi graph corresponding to an orphan-free Voronoi diagram, 
	as defined in section~\ref{sec:planar}. 
By theorem~\ref{th:simpleplanar},  is simple and planar with vertices at the sites. 
Let  be the planar graph obtained by replacing curved edges by straight segments. 
Recall from section~\ref{sec:planar} that, 
	while Voronoi regions and edges are connected, Voronoi vertices may have multiple connected components, 
	and therefore  can have duplicate faces in . 
We only show after this section that faces have multiplicity one by virtue of  being embedded. 


\vspace*{0.08in}\noindent{\bf Faces with more than three vertices}. 
Every face  is dual to a Voronoi element  of order , 
	to which corresponds (proposition~\ref{prop:ECB}) a convex ball , with , 
	that circumscribes the sites  incident to . 
Due to the planarity of , we can assume the sites  to be ordered around . 
In order to find whether a point  belongs to , 
	we simply triangulate  in a fan arrangement: 
		, 
	and consider that  iff it lies in any of the resulting . 
Note that this arrangement does not interfere with the original edges in  (other than creating new ones), 
	all new edges are incident to two faces (they are not in the topological boundary of ), 
	and most importantly, 
every , with  satisfies the empty circum-ball property with the same {witness} ball  
	as . 
We assume in the sequel that  has been triangulated in this way. 
The fact that this triangulated  will be shown to be embedded will imply that every face  is in fact convex. 



For convenience in the remainder of this section 
we name  the sites that are part of the boundary of
the convex hull , and order them in clock-wise order around . 



\subsection{Boundary}\label{sec:boundary}

In this section, we assume that the divergence  satisfies the bounded anisotropy assumption~\ref{ass:BAA},
 and conclude that the boundary of the dual triangulation of an orphan-free diagram is the same as the boundary of the convex hull of the sites (and in particular it is simple and closed). 

The vertices in the \emph{topological boundary} of  are those 
	whose corresponding primal regions are unbounded, 
while topological boundary edges are those connecting topological boundary vertices. 
For convenience, we call  the set of topological boundary edges of . 

The boundary  of the convex hull is a simple circular chain 
. We prove that it is  
(loosely speaking: the topological, and geometric boundaries of  are the same and coincide with the boundary of ), 
which implies that  covers
the convex hull of the sites, and its topological boundary edges form 
a simple, closed polygonal chain. All the proofs of this section are in Appendix B. 

\begin{lemma}[]\label{boundary_easy}
 To every topological boundary edge of  corresponds a segment in the boundary of .
\end{lemma}



We now turn to the converse claim: that to every segment 
in  corresponds one in . Since  is the set of boundary edges of , whose corresponding primal edges 
are unbounded, the claim is equivalent to proving that, to every segment
in  corresponds a boundary edge  of  whose corresponding primal edge  is unbounded. 

The proof proceeds as follows. First, assume without loss of generality  that the origin is in the interior of . 
Let   be an origin-centered circle 
	of radius  large enough so that lemmas~\ref{lem:VW} and~\ref{lem:contrad} hold in . 
We define two  functions: 

  simply projects points in the boundary of  out to their closest point in  
(using the natural metric; note that  can always be chosen large enough so this projection is unique). 
 is constructed as follows. 


\begin{figure}[htbp]
   \centering
   	\includegraphics[width=2.7in]{pi.pdf} \caption{The construction of the projection function . 
   Note that in this case the region-to-site-index function is simply , but this cannot be assumed in general. }
   \label{fig:pi}
\end{figure}

\refstepcounter{foo}\thefoo\label{text:boundary}
Consider the situation illustrated in figure~\ref{fig:pi}. 
By lemma~\ref{lem:VW}, all points in  are closer to  than to any interior site . 
We split  into 
	a sequence  of \emph{connected} parts closest 
	to the same boundary site  (the function  is used to map part indices to the index of their closest site). 
By the convexity of balls, 
	adjacent regions \emph{must} be closest to (circularly) consecutive sites in  
	(e.g.~if regions  had  and , 
	 by the continuity of , the point  where  meet would be closest to ; 
	 however, since the sites  are in cyclic order around , 
 would be closer to  than to , a contradiction). 
Pick one point  for each region , and let . 
For each pair of consecutive regions  meeting at , 
	let  (the midpoint of two consecutive boundary sites). 
The remaining values of  are filled using simple linear interpolation. 
By construction, the following holds:
\begin{property}\label{prop:pi}
  is continuous. \\
\hspace*{0.91in} Given  and consecutive boundary sites , 
	then  iff . 
\end{property}



By the convexity of ,  is continuous in . 
Note that, because  is assumed to contain the origin then, 
	as shown in figure~\ref{fig:pinu},  projects 
every point  lying on a segment of , 
\emph{outwards} from the convex hull (and on the \emph{empty} side of
); that is, so that 
(i.e.~ is in the empty half-space of ). 

\begin{figure}[htbp]
   \centering
   	\includegraphics[width=2.7in]{pinu.eps} \caption{The construction for the proof of lemma~\ref{lem:hard}.}
   \label{fig:pinu}
\end{figure}



The claim now reduces to showing that for each segment 
	 of , and for \emph{every} sufficiently large , 
there is  with  (i.e.~). 
Since this implies that  is unbounded, 
	it means that the corresponding edge  is in  (the topological boundary of ).

The proof is by contradiction. 
Lemma~\ref{lem:Sn} uses Brouwer's fixed point theorem to show that, for every segment  of , 
if there were no  closest to , then the function  
(in fact a slightly different but related function) 
would have a point  such that
, that is, such that 
 is ``behind" the segment  to which it is
closest (). 
On the other hand, lemma~\ref{lem:contrad} shows that, for all sufficiently large circles , no
point  can be closest to a segment
 it is \emph{behind} of, creating a contradiction. 



The next Lemma is used to create a contradiction, and relies on assumption~\ref{ass:BAA}. 
Lemma~\ref{lem:Sn} is the key lemma in this section, and is a simple application of Brouwer's fixed point theorem. 

\begin{lemma}\label{lem:contrad}
There is  such that, for any segment  with supporting line , 
	 every  with  whose closest point in  belongs to  is 
	closer to a site in  than to .
\end{lemma}



\begin{lemma}\label{lem:Sn}
	Every continuous function  that is not onto has a fixed point. 
\end{lemma}




\begin{lemma}[]\label{lem:hard}
 To every segment  in the boundary of
 corresponds a boundary edge of .
\end{lemma}


Finally, since we have shown that the topological boundary of the dual triangulation is the 
	same as the boundary of the convex hull of the sites, we can conclude that:


\begin{corollary}\label{cor:boundary}
The topological boundary of the dual of an orphan-free Voronoi diagram 
	is the boundary of the convex hull , and is therefore simple and closed. 
\end{corollary}




\subsection{Interior}\label{sec:interior}


This section concludes the proof of Theorem~\ref{th:main} by showing that, 
	if the topological boundary of  is simple and closed, then  must be embedded.
The main argument in the proof 
uses proposition~\ref{prop:ECB} and~\ref{cor:boundary}, as well as the theory of discrete one-forms on graphs, 
to show that there are no
``edge fold-overs" in  (edges whose two incident faces are on the same side of its supporting line), 
and uses this to conclude that the interior of  is a single ``flat sheet", and therefore it is embedded. 




The following definition, from~\cite{1form},  assumes that, for each edge  of , 
we distinguish the two opposing half-edges  and . 


\begin{definition}[Gortler et al.\ \cite{1form}]\label{def:1form}
A non-vanishing (discrete) one-form   is an assignment of a real value
 to each half edge  in , such that 
. 
\end{definition}

We can construct a non-vanishing one-form
over  as follows. 
Given some unit direction vector 
(in coordinates ), 
we assign a real
value  to each vertex  in , and define 
, which clearly satisfies 
. The one-form, denoted by , 
is non-vanishing if, for all edges , 
it is , 
that is, if  is not orthogonal to any edge. 
The set of edges has finite cardinality ,  
so \emph{almost all} directions  generate a non-vanishing one-form . 


Since  is a planar graph with a well-defined face structure,
there is, for each face , a cyclically ordered set
 of half-edges around the face. 
Likewise, for each vertex , the set  of cyclically ordered
(oriented) half-edges emanating from each vertex is well-defined. 

\begin{definition}[Gortler et al.\ \cite{1form}]
Given non-vanishing one-form ,
the index of vertex  with respect to  is 
	where  is the number of sign changes of  
	when visiting the half-edges of  in order. 
The index of face  is  where  is the number
of sign changes of  as one visits the half-edges of  in order. 
\end{definition}

Note that, by definition, it is always . 
A discrete analog of the Poincar\'e-Hopf index theorem relates 
the two indices above:

\begin{theorem}[Gortler et al.\ \cite{1form}]\label{lem:ph}
For any non-vanishing one-form , it is 

\end{theorem}

Note that this follows from Theorem 3.5 of~\cite{1form} because the unbounded, 
outside face, which is not in , is assumed in this section to be closed and simple
(corollary~\ref{cor:boundary}), and therefore has null index. Note that the machinery from~\cite{1form} to deal with degenerate cases
isn't needed here because vertices, by definition, cannot coincide ( is not a multiset). 
All  proofs in this section, except for that of theorem~\ref{th:main}, are  in Appendix C. 

The one-forms  constructed above satisfy the following property:

\begin{lemma}\label{lem:non-negative}
	Given a non-vanishing one-form , the sum of indices of interior vertices () of  is non-negative. 
\end{lemma}




The next two lemmas relate the presence of edge fold-overs and 
the ECB property (proposition~\ref{prop:ECB})  to the indices of vertices in . 


\begin{lemma}\label{lem:index-1}
If  has an edge fold-over, then there is  and non-vanishing one-form  such
that  for some interior vertex . \end{lemma}


\begin{lemma}\label{lem:index1}
Given  and non-vanishing one-form , if  has an interior vertex  with index
, then there is a face  of
 that does not satisfy the empty circum-ball property (proposition~\ref{prop:ECB}). 
\end{lemma}

The above provides the necessary tools to prove the following key lemma. 


\begin{lemma}\label{lem:ef}
 has no edge fold-overs. 
\end{lemma}



Finally, the absence of edge fold-overs, together with a simple and closed boundary, 
is sufficient to show that  is 
embedded.

\begin{lemma}\label{lem:interior}
If its (topological) boundary is simple and closed, 
	then the straight-line dual of an orphan-free Voronoi diagram, 
	with vertices at the sites, 
	is an embedded triangulation. 
\end{lemma}



\section{Proof-of-concept implementation}\label{sec:implementation}

\begin{figure}[ht]
\centering
\subfloat[]{\includegraphics[height=2.6cm]{c07.png}\label{fig:img_a}}
\subfloat[]{\includegraphics[height=2.6cm]{c08.png}\label{fig:img_b}}
\subfloat[]{\includegraphics[height=2.6cm]{c03.png}\label{fig:img_c}}
\subfloat[]{\includegraphics[height=2.6cm]{c04.png}\label{fig:img_d}}\quad
\subfloat[]{\includegraphics[height=2.6cm]{c00.png}\label{fig:img_e}}
\subfloat[]{\includegraphics[height=2.6cm]{c01.png}\label{fig:img_f}}\subfloat[]{\includegraphics[height=2.6cm]{c05.png}\label{fig:img_g}}
\subfloat[]{\includegraphics[height=2.6cm]{c06.png}\label{fig:img_h}}
\caption{
Anisotropic Voronoi diagrams, and their duals generated by our
proof-of-concept implementation. 
Voronoi vertices are marked as red dots, while dual vertices (sites) and edges are drawn
in black.}
\label{fig:test}
\end{figure}



Though not aiming for an efficient implementation, 
we tested a simple proof-of-concept that constructs anisotropic Voronoi diagrams
(using a quadratic divergence  of the type discussed in section~\ref{sec:DQ}) 
and their duals
(figure~\ref{fig:test}). 
A closed-form metric, which has bounded ratio of eigenvalues 
(and therefore by lemma~\ref{lem:DQgamma} satisfies assumption~\ref{ass:BAA}), 
is discretized on a fine regular grid, and linearly interpolated inside grid elements, resulting in a
continuous metric. The sites are generated randomly (figures~\ref{fig:img_a}
and~\ref{fig:img_b}), or using a combination of random, and equispaced
points forming an (asymmetric) -net~\cite{avd} (remaining figures). 

The primal diagram was obtained using front propagation from the sites
outwards, until fronts meet at Voronoi edges. 
The runtime is proportional to the grid size, since every grid-vertex is visited exactly six times (equal to their valence), 
	and so linear in the resolution of the sampled divergence . 

The implementation does not guarantee the correctness of the diagram unless it \emph{is} orphan-free, and serves to verify the claims of the paper since well-behave-ness of the dual is predicated on that of the primal. 

The two main claims of the paper (that orphan-freedom is sufficient to ensure well-behavedeness of 
both the dual and the primal) are clearly illustrated in these examples. 
In all examples, the dual covers the convex hull of the vertices
(corollary~\ref{cor:boundary}), is a
single cover, embedded with straight edges without edge crossings
(lemma~\ref{lem:interior}), 
and has no degenerate faces 
(since, by proposition~\ref{prop:ECB}, the vertices of a face lie on the boundary of a strictly convex ball). 
By focusing on the primal diagrams (second and fourth column), further claims in
the paper become apparent, namely that Voronoi regions 
(Voronoi elements of order one according to definition~\ref{def:VorI}) are simply connected (lemma~\ref{lem:regionSC}), 
and Voronoi edges (order two), and vertices (order three or higher) are connected (corollary~\ref{cor:VorI}). 


\section{Conclusion and open problems}

We studied the properties of duals of orphan-free Voronoi diagrams with respect to divergences, for the
purposes of constructing triangulations on the plane. 
The main result (Theorems~\ref{th:main}) is that
the dual, with straight edges and vertices at the sites, is embedded
and covers the convex hull of the sites, mirroring similar results for
ordinary Voronoi diagrams and their duals.
Additionally, the primal is composed of connected elements (corollary~\ref{cor:VorI}). 




\begin{figure}[ht]
\centering
\includegraphics[width=5.6cm]{3dbreaks.pdf}
\caption{
	The main proof of this paper does not work as is in higher dimensions. 
This arrangement of tetrahedra is not embedded: 
	the red tetrahedra has been ``inverted" 
	(the green dotted edge is \emph{behind} the solid blue edge), 
	``invading" the two front tetrahedra (closest to the viewer), 
	as well as the two back tetrahedra (farthest from the viewer). 
However, it does not violate the ECB condition (proposition~\ref{prop:ECB}). 
}
\label{fig:3dbreaks}
\end{figure}
Perhaps the most important outstanding question is whether these results
extend to higher dimensions. The proofs in Secs.~\ref{sec:boundary} and~\ref{sec:interior},
except for lemma~\ref{lem:regionSC}, can be trivially extended to n dimensions. 
Section~\ref{sec:boundary} has been written only for the two-dimensional case,
but a similar construction, and the same argument would work in higher
dimensions (lemma~\ref{lem:Sn} being a hint of this). 
It is the argument in section~\ref{sec:interior}, and described in figure~\ref{fig:outline}, that becomes problematic. 
While the ECB property is shown to be sufficient to prevent fold-overs in the
triangulation, it is not sufficient in 
higher dimensions. In particular, fixing the boundary to be simple and convex, 
there are simple arrangements of tetrahedra in  that contain 
face fold-overs but do not break the ECB property. 
In particular, the arrangement of tetrahedra of figure~\ref{fig:3dbreaks} is not embedded: 
	the red tetrahedra has been ``inverted" 
	(indicated by the green dotted edge being \emph{behind} the solid blue edge); 
	its interior overlaps that of the two front tetrahedra (closest to the viewer), 
	as well as the two back tetrahedra (those farthest from the viewer). 
However, this arrangement does not break the ECB condition (proposition~\ref{prop:ECB}, which holds in any dimension), 
	and therefore the same argument used in this work would not create a 
	contradiction in higher dimensions. 






\bibliographystyle{plain}
\bibliography{vddw8}






\section*{Appendix A: Bounded anisotropy condition}\label{app:gamma}


\noindent{\bf lemma~\ref{lem:DFgamma}} (Bounded anisotropy for Bregman divergences).
\emph{
If  and there is  such that the Hessian of  has ratio of eigenvalues bounded by ,
	then assumption~\ref{ass:BAA} holds. 
}\begin{proof}
Consider the situation described in figure~\ref{fig:gamma}, 
	in a coordinate system with the y-axis along . 

Let . 
Because  and the ball  it tangent to the y-axis at , 
	it is 

Since , 
we can obtain the value of  by integration from , first 
	along the y-axis from  to , 
	then along the x-axis from  to . 

Let  and , 
	and assume that  and  without loss of generality, 
	since ,  is on the same side of  as , 
	and we have freedom in choosing the sign of the axis. 

For assumption~\ref{ass:BAA} to hold it must be . This holds whenever

or equivalently

which reduces to 

and is always satisfied whenever . 
Note that this bound is finite because  and , . 
\end{proof}



\noindent{\bf lemma~\ref{lem:DQgamma}} (Bounded anisotropy for quadratic divergences).
\emph{
If there is  such that  has ratio of eigenvalues bounded by ,
	then assumption~\ref{ass:BAA} holds. 
}\begin{proof}
The proof of this lemma can be reduced to that of lemma~\ref{lem:DFgamma}.
Given , we let 
whose Hessian is . Since  has eigenvalues bounded from below by , 
	the conditions of the proof of lemma~\ref{lem:DFgamma} hold. 
Note that this definition of  is \emph{per choice of} , 
	and therefore we are not defining a real Bregman divergence this way, but simply choosing 
	a different  for each  as to satisfy the conditions of the proof. 

\end{proof}







\begin{figure}[htbp]
   \centering
	\subfloat[]{\label{fig:norm.a}\includegraphics[width=2.3in]{norm1.pdf}}\quad\quad
	\subfloat[]{\label{fig:norm.b}\includegraphics[width=2.5in]{norm2.pdf}}
   \caption{Diagrams used in the proof of lemma~\ref{lem:Lpgamma}. 
   	By making  large enough, we can ensure that  falls in the blue shaded region, and 
		therefore , where . 
   }
\end{figure}



\noindent{\bf Lemma~\ref{lem:Lpgamma}} (Bounded anisotropy for normed spaces)\emph{
	Distances derived from strictly convex  norms satisfy assumption~\ref{ass:BAA}.
}
\begin{proof}
Let  be a strictly convex  norm, whose unit ball is the symmetric convex body . 
Let  with supporting line , and  be given. 
For any  with closest point  in , define . 
Defining  to be the origin, let  be a linear transformation that maps the  direction into the -axis, 
	and  into the  axis. 
The fact that  implies that  is non-singular. 
Choose the sign of the -axis so that  
	are the maximum and minimum eigenvalues of , 	respectively. 

Consider the following statements:
\begin{enumerate}[i]
\item For all pairs , there is a sufficiently large  such that whenever 
	 then . 
\item For all pairs , there is a sufficiently large  such that whenever
	 then . 
\end{enumerate}
	
\vspace*{0.05in}\noindent{\bf [Reducing assumption~\ref{ass:BAA} to statement (i)]}. 	
Given (i), and since both  and  are compact, we can define:
	
from which it follows that whenever , it holds
	
	and therefore , thereby satisfying assumption~\ref{ass:BAA}. 


\vspace*{0.05in}\noindent{\bf [Reducing statement (i) to statement (ii)]}. 	
Assume (ii) is true and let . 
	Whenever , it holds:

	and therefore by (ii) it is . 


\noindent{\bf [Proof of statement (ii)]}. 
Consider the situation depicted in figure~\ref{fig:norm.a}, which shows a portion of the plane transformed by . 
Given , consider the set of points at distance  from  (red line). 
First note that, because we have temporarily chosen  as the origin, then , and . 
Because , there is an open interval  and 
	a function  such that , with  
	are the coordinates of the points (in -space) at distance  from . 

Because  is the point closest to  in , then, 
	in -space,  is tangent to the -axis at , 
	and therefore , from which it follows that
	

By a simple calculation, it is simpe to show that moving  further down along the  axis to 
	(figure~\ref{fig:norm.b}), 
	scales the red curve of figure~\ref{fig:norm.a} by a factor , so that it becomes 
	 with . 

Given  in coordinates, with  and , without loss of generality, 
	then, from the figure and the expression for the curve , it is clear that it is 
	possible to choose  large enough so that  is below the curve , 
	and therefore  is closer (with respect to ) to  then . By setting , statement (ii) follows. 

In particular, we simply choose  large enough so that
\begin{itemize}
	\item ; 
	\item  is far enough from  so the line  crosses the -axis between 
		, which is clearly possible for sufficiently large ; 
	\item if , then it is a simple calculation to show that 
		we can ensure that  is ``below" the curve  as follows:
		1) choose a small enough  such that , 
		which is always possible because , 
		and 2) enforcing . 
\end{itemize}









\end{proof}












\section*{Appendix B: dual triangulation (boundary)}\label{app:boundary}

Let  be two sites, we denote by  
	the two open half-spaces on either side of their supporting line . 
The set  is therefore a disjoint partition of .
Whenever the two sites we consider are on the boundary of , 
	they are denoted by , 
	and we always choose  to be the ``empty" half space of the two 
	(such that ). 


\begin{lemma}\label{lem:halfspace}
    Given a Voronoi edge  
    	corresponding to neighboring sites , if  () is unbounded, then it is 
 (), 
where  are open half spaces on either side of the
supporting line of . 
\end{lemma}
\begin{proof}
Assume . 
Since  is unbounded, 
we choose  of sufficiently large norm, 
	so that assumption~\ref{ass:BAA} implies that . 
By the convexity of , this means that . 
Since  is closer to  than to , it is , a
contradiction. 
\end{proof}



\noindent{\bf Lemma~\ref{boundary_easy}}
\emph{	To every topological boundary edge of  corresponds a segment in the boundary of .
}
\begin{proof}
By the definition of , 
to every boundary edge  corresponds a primal edge 
that is unbounded. 


Consider the two open half-planes  and  on either
side of the supporting line  of . 
We split  in three parts: , , and , 
	at least one which must be unbounded. 
	
Since, by lemma~\ref{lem:midpoint}, it is  (and therefore bounded), 
	then it must be that either  or  are unbounded. 
By lemma~\ref{lem:halfspace}, they cannot both be, or else 
 and , and therefore all sites
would be in  (all colinear). 
Assume w.l.o.g.\  that  is unbounded. 

By lemma~\ref{lem:halfspace},  unbounded implies 
, and 
so  must lie in the boundary of  
( and ). 


It only remains to show that  are consecutive in the sequence . 
We prove this by contradiction. 
If they were not, then since , 
there must be some site , . 
However, this is not possible. 
To see this, simply pick some point , by definition closest to ;
 by the convexity of , it must be , a contradiction. 

Since  are consecutive vertices in , then 
. 
\end{proof}







\begin{figure}
\centering
\subfloat[]{\includegraphics[height=3.0cm]{wijcap.png}\label{fig:wijcap}}
\subfloat[]{\includegraphics[height=3.0cm]{vHm.png}\label{fig:vH-}}
\quad\quad
\subfloat[]{\includegraphics[height=3.0cm]{vHp.png}\label{fig:vH+}}
\label{fig:VW}
\caption{Diagram for the proof of lemma~\ref{lem:VW}.}
\end{figure}




\begin{lemma}\label{lem:VW}
	There is  such that all  with   are closer to  than to . 
\end{lemma}
\begin{proof}


Let  be the sites that lie in the boundary of the convex hull . 
We prove that there is a sufficiently large value  such that all  with 
	are strictly closer (in the sense of ) to  than to the remaining sites . 

Pick any pair of consecutive sites  along the boundary of , and any third site  from . 
We first show that there are values  such that all  with 
	are strictly closer to  than to . 
By letting
	
	lemma 6.5 of~\ref{ass:BAA}. 


For a triple  with  consecutive vertices in , and , 
	we consider the supporting line  of , 
		which divides space into two half-spaces  and , 
	where we pick  so as to contain . 
Note that  or else it would be . 

First, let  be large enough so that all  with  are outside the convex hull . 
The divide the proof in three cases, 
depending on whether 
	point  belongs to , , or , respectively. 
Each case will result in a different constraint , , , 
	
	

\vspace*{0.1in}\noindent{\bf [Case ]}.
Since not all sites are colinear, then it is , and therefore the segment  
	is different from . 
If , 
	we can consider the ``next" segment , 
	for which, since   is outside , it must hold 
	, 
	and therefore we can simply let . 
Note that there is no circular dependency in this definition, since we can resolve it by simply 
	letting 
	
	




\vspace*{0.1in}\noindent{\bf [Case ]}.
By assumption~\ref{ass:BAA}, there is  
	such that all  with  
	are closer to  than to  or .


\vspace*{0.1in}\noindent{\bf [Case ]}.
Let , and consider the ball  of the first kind centered at  with radius . 
Since  is convex, we can find a line  passing through  that ``separates" , 
	that is,  lies in the half-space  associated to . 
It follows that  is the closest point to  in the line . 
Note that, because , and  passes through , 
	then it must be either  or  
	(otherwise, if  then it would be , a contradiction). 
Without loss of generality, let  be in . 

Pick two point  along  such that  is between  and . 
We are now ready to apply assumption~\ref{ass:BAA}, using the substitution
	, , , and , 
	from which it follows that there is a sufficiently large  
		such that if  and , 
		then  is closer (in the sense of ) to  than to . 



\end{proof}





\noindent{\bf lemma~\ref{lem:contrad}}
\emph{
There is  such that, for any segment  with supporting line , 
	 every  with  whose closest point in  belongs to  is 
	closer to a site in  than to . }
\begin{proof}
For each edge  with supporting line , 
	pick a site  that isn't in  (which always exists since not all sites are colinear). 
By assumption~\ref{ass:BAA}, there is a sufficiently large  such that every point  
	whose closest point  to  satisfies  is closer to  than to . 
Since , then  is closer to  than to either , and 
thus  is closer to  than to either . 

Letting  be the maximum of  over all edges   completes the proof. 
\end{proof}






\noindent{\bf lemma~\ref{lem:Sn}}
\emph{
	Every continuous function  that is not onto has a fixed point. 
}
\begin{proof}
	Assume  misses , and let
 be a diffeomorphism
between the punctured sphere and the open unit disk. 
Since  is continuous and  is compact,
then the set  is compact.

The function  with  
is continuous and therefore, by Brouwer's fixed point theorem~\cite{Milnor}, has a fixed point . 
The fact that  implies  
and thus  is a fixed point of . 
\end{proof}




\noindent{\bf lemma~\ref{lem:hard}}
()
\emph{
To every segment  in the boundary of  corresponds a boundary edge of .  
}
\begin{proof}  
Let  be a segment in the boundary of , 
	as shown in figures~\ref{fig:pinu} and~\ref{fig:pi}. 
Pick a sufficiently large  such that every  with
 is outside  and such that 
	Lemmas~\ref{lem:VW} and~\ref{lem:contrad}hold. 
For any , if  is the antipodal map , 
then, by continuity of  (property~\ref{prop:pi}.i) and by the continuity of , the function  is
continuous. 
By lemma~\ref{lem:VW} and property~\ref{prop:pi}.ii, 
	if for some  it is 
 with
, then  is (strictly) closest to
, and therefore belongs to the primal edge , which implies
that . 

Showing that  now reduces to showing that for all 
, for all , there is  such that
. 

Assume otherwise. The function 
 is not onto and therefore, 
by lemma~\ref{lem:Sn} (and using the fact that  is isomorphic to ), 
it must have a fixed point . 

Since  then . 
Since  is the closest point to  in , 
	there is a segment  such that
. Consider two open half spaces
 and  on either side of the supporting line of
. Since not all sites are colinear, we can choose these half spaces so that 
 and . 
By the definition of , and recalling that the chosen origin of
 is in the interior  of the convex hull, 
it is , and . 
To see this note that the outward-facing normal  is defined so that
 and so .
On the other hand, since the origin is in
, the fact that  
implies . 

Since  was chosen sufficiently large for lemma~\ref{lem:contrad} to
hold, and ,  is closer to some site  than to . 
Since , this contradicts the fact
that  is the closest point to  in
. \end{proof}





\section*{Appendix C: dual triangulation (interior)}\label{app:interior}


\noindent{\bf lemma~\ref{lem:non-negative}}
\emph{
	Given a non-vanishing one-form , the sum of indices of interior vertices () of  is non-negative. 
}
\begin{proof}
Given non-vanishing , 
the index of a face  is . 
To see this, assume otherwise: a face with vertices  around it, and index one satisfies, by the
definition of index and of , 
 
(or ), a
contradiction. 

Because, by corollary~\ref{cor:boundary}, the boundary edges of  form a convex, 
	simple polygonal chain, then, given any non-vanishing , all the boundary vertices have
index zero, except for the ``topmost" () 
and ``bottommost" () vertices, which have
index one (note that the topmost and bottommost vertices are unique because  is chosen not to be orthogonal to any edge in the triangulation). 

Since face indices are non-positive, and the sum of indices of
boundary vertices is two then, by lemma~\ref{lem:ph},
the sum of indices of {interior} vertices must be non-negative. 
\end{proof}






\noindent{\bf lemma~\ref{lem:index-1}}
\emph{
If  has an edge fold-over, then there is  and non-vanishing one-form  such
that  for some interior vertex . }
\begin{proof}
If edge fold-over  is a non-boundary edge, then at least one
of its incident vertices, say  is an interior vertex . 

Consider the two faces  incident to , which, by definition of
edge fold-over, are on the same side of its
supporting line, and the two edges  in  respectively,
incident to . 
Taking the half-line  from  towards  as reference, consider 
the (open) set  of directions ranging from  to
. 
The set  is not empty since, by proposition~\ref{prop:ECB},  are not degenerate, and therefore neither  are parallel
to .  is also uncountable, since it is a range of the form

where  are the direction vectors of , and
 is one of the two orthogonal directions to , 
chosen to fit the definition. 

Because  is not empty, and is uncountable, and because the set of edges  is finite,
then there is always some direction  that is not orthogonal to any edge in
. 
Pick any such . 
We prove that the non-vanishing one-form  is such that
. 

The (cyclic) sequence of oriented half-edges {around}  is, without loss of generality,
, and therefore the values of the
one-form around  are , ,
, . 
By the definition of , it is , ,
and , and therefore 
the number of sign changes in 
the subsequence
 is four. 
Since the number of sign changes in the full sequence  cannot
be less
than that of its subsequence , 
it is  and therefore . 
\end{proof}




\noindent{\bf lemma~\ref{lem:index1}}
\emph{
Given  and non-vanishing one-form , if 
has an interior vertex  with index
, then there is a face  of
 that does not satisfy the empty circum-ball
property (proposition~\ref{prop:ECB}). 
}
\begin{proof}
We must prove that there is a face  all of whose circumscribing balls
contain some vertex in its interior. 

Consider the vertex  with
, and thus
with . If  is the cyclic sequence of vertices neighboring
, then 
 implies either , , or , . 
Assume the former w.l.o.g. The line , passing through
, strictly separates  from the convex hull of its neighbors. 

Consider the mesh , with the same structure as  but in which all the incident faces to 
are eliminated. 
We show that, in , the face count of  (the number of faces in which  lies) is at
least one. 
Since  separates  from its neighbors, it also separates all the faces
incident to  from  (except for  itself, which lies on ). 
Pick any direction  with . 
The half-line  starting at  with direction  does not intersect any face in  that is incident to . 
Since there is only a finite number of edges and vertices, it is always
possible to choose  not to contain any vertex other than , and not to be parallel to any edge. 
Since  is bounded and  isn't, there is
some point  outside , whose face count must be zero. 
Moving from  toward ,  crosses  only once 
(since  is convex), incrementing the face count to one. 
Because every interior edge is incident to exactly two faces, 
every subsequent edge cross (which is transversal because  is not
parallel to any edge) modifies the face count by either zero, two, or
minus two. Since the face count cannot be negative, and it is one at 
, then it must be at least one at . 
Since  does not contain any face incident to , 
this implies that there is
some face  not incident to  such that . 

We prove that the face  above cannot satisfy the ECB property. 
Since  is in  but is not incident to it, and  is convex
then, by Carath\'eodory's theorem~\cite{matousek2002lectures},  can be written as a 
convex combination ,
,  of vertices 
incident to  
(note that this is slightly more general than required since we have 
already made sure in the beginning of section~\ref{sec:dual} that  is a triangle). 
Given a ball circumscribing the vertices incident to , 
because it is strictly convex, and  lie in its boundary, 
then any convex combination of them with  must be in the
interior of the circumscribing ball, and therefore  does not satisfy the ECB property. 
\end{proof}







\noindent{\bf lemma~\ref{lem:ef}}
\emph{
 has no edge fold-overs. }
\begin{proof}
Assume  has an edge fold-over. 
By lemma~\ref{lem:index-1}, there is a non-vanishing one-form
 such that some interior vertex  has . 
Since, by lemma~\ref{lem:non-negative}, the sum of indices of interior vertices is
non-negative,
then there must be
at least one interior vertex  with positive index
. In that case, by lemma~\ref{lem:index1}, there is a face of  that does not satisfy the
ECB property, raising a contradiction. 
Therefore  has no edge fold-overs. 
\end{proof}







\noindent{\bf lemma~\ref{lem:interior}}
\emph{
If its (topological) boundary is simple and closed, 
	then the straight-line dual of an orphan-free Voronoi diagram, 
	with vertices at the sites, 
	is an embedded triangulation. 
}
\begin{proof}


Given a point  in
the interior of the convex hull of , we show that its \emph{face count}
(the number of straight-edge faces that contain it) is one. 
Consider a line  passing through  that does not pass through any vertex of
, and is not parallel to any (straight) edge. 
It is always possible to find such a line since the set of vertices and edges is finite. 
Because the line is unbounded and  is bounded, there is a
point  that is outside . At this point clearly the
face count is zero. 
Moving from  toward ,  crosses the boundary of 
(and therefore, by corollary~\ref{cor:boundary}, the boundary of ) 
only once, since it is a simple convex polygonal chain, incrementing the face
count by one. At every edge crossing (which is transversal by the choice of
line), the face count remains one since, by lemma~\ref{lem:ef} there are no
edge fold-overs, and thus every non-boundary edge is incident to two faces
that lie on either side of its supporting plane. Therefore the face count at
 must be one. 


Since every point inside  is covered once by faces in
,
and the boundaries of  and  coincide, then
 is a single-cover
of . 
Because two straight edges that cross at a non-vertex always generate points with
face count higher than one, then the edges of  can only meet at vertices, and
therefore  is embedded. 
\end{proof}





\end{document}
