\documentclass[11pt,a4paper]{article}
\usepackage{amsfonts,amssymb,amscd}
\usepackage{latexsym, graphicx}
\usepackage[affil-it]{authblk}
\usepackage{algorithmic}
\usepackage{algorithm}

\usepackage[margin=3cm]{geometry}

\usepackage{enumerate}

\newcommand{\proofBox}{\hfill }
\def\qed{\hfill \vspace{2ex}}

\newtheorem{theorem} {Theorem}[section]
\newtheorem{lemma}[theorem]{Lemma}
\newtheorem{observation}[theorem]{Observation}
\newtheorem{corollary}[theorem]{Corollary}
\newtheorem{problem}{Problem}[section]

\usepackage{xcolor}




\newenvironment{proof}{\noindent {\it Proof:~}}{\hfill \smallskip\par}

\def\inst#1{}
\begin{document}

\title{On the structure of (pan, even hole)-free graphs\thanks{Research
support by Natural Sciences and Engineering Research Council of
Canada.}}
\author{Kathie Cameron\inst{1}
 \and Steven Chaplick\inst{2}\thanks{Research partially supported by the European Science Foundation project EUROGIGA GraDR.}
  \and Ch\'inh T. Ho\`ang\inst{3}
}
\date{}
\maketitle
\begin{center}
{\footnotesize

\inst{1} Department of Mathematics, Wilfrid Laurier University,
Waterloo, Ontario,  Canada, N2L 3C5\\
\texttt{kcameron@wlu.ca}

\inst{2} Lehrstuhl f\"ur Informatik I
Universit\"at W\"urzburg, Am Hubland
D-97074 W\"urzburg, Germany\\
\texttt{steven.chaplick@uni-wuerzburg.de}

\inst{3} Department of Physics and Computer Science, Wilfrid
Laurier University,
Waterloo, Ontario,  Canada, N2L 3C5\\
\texttt{choang@wlu.ca} }

\end{center}

\begin{abstract}
A hole is a chordless  cycle with at least four vertices. A pan is
a graph which consists of a hole and a single vertex with
precisely one neighbor on the hole. An even hole is a hole with
an even number of vertices. We prove that a (pan, even hole)-free
graph can be decomposed by  clique cutsets into essentially unit
circular-arc graphs. This structure theorem is the basis of our
-time certifying algorithm for recognizing (pan, even
hole)-free graphs and for our -time algorithm to optimally
color them. Using this structure theorem, we show that the tree-width
of a (pan, even hole)-free graph is at most 1.5 times the clique
number minus 1, and thus the chromatic number is at most 1.5 times the
clique number.
\end{abstract}

\section{Introduction}\label{sec:introduction}
A {\em hole} is a chordless cycle with at least four vertices. A
graph is  {\em chordal} if it does not contain a hole as an
induced subgraph. Chordal graphs are well-studied and have a
number of interesting structural properties (see
\cite{BerChv1984,Gol1980}). For example, it is known
\cite{Dir1961} that every chordal graph contains a {\it
simplicial} vertex; i.e., a vertex whose neighborhood induces a
clique. Based on this, a largest clique, a minimum coloring, a
largest stable set, and a minimum partition into cliques of a
chordal graph can be found in polynomial time \cite{Gavril1972}.



An {\em even hole} is a hole with an even number of vertices. A
graph is {\em even-hole-free} if it does not contain an even hole
as an induced subgraph. Even-hole-free graphs generalize chordal
graphs and analogous properties have been found.
A largest clique of an even-hole-free graph can be found in
polynomial time \cite{actv, AddChu2008, daSV}. However, it is not known whether even-hole-free
graphs can be optimally colored in polynomial time.

The {\it claw} is the graph with vertices  and edges
. As usual,  (respectively, ) denotes the number of vertices
(respectively, edges) of the input graph . We give an -time
algorithm to color (claw, even hole)-free graphs,
providing a contrast to the well-known result \cite{Hol1981} that it is NP-hard
to optimally color claw-free graphs. Our techniques actually apply to a
larger class of graphs which we will now define. An {\it atom  }
is a connected graph without a clique cutset. A {\it pan} is a graph
which consists of a hole and a single vertex with precisely one
neighbor on the hole. Let  denote the class of graphs
 such that each atom of  is (pan, even hole)-free. In this
paper, we will give an -time algorithm to color a graph in
, and an -time certifying algorithm for
recognition of graphs in  and recognition of (pan, even
hole)-free graphs. Pan-free graphs have been studied previously
regarding: establishing the perfectness of (pan, odd hole)-free graphs \cite{Ola1989},
and providing a polynomial-time algorithm to find a largest weight stable
set first on a subclass of pan-free graphs \cite{Des1993} and then the whole
class of pan-free graphs \cite{BraLoz2010}.
The latter two
papers use the term ``apple" for ``pan".

In Section~\ref{sec:definitions}, we will cover the relevant background
and state our main results. In
Section~\ref{sec:properties}, we will prove that a (pan, even
hole)-free graph can be decomposed by the well-studied clique
cutset decomposition into, essentially, ``unit circular-arc
graphs". This structural result is the foundation of our
polynomial-time algorithms. In Section \ref{sec:coloring}, we give
our -time algorithm for coloring the graphs in .
In Section ~\ref{sec:recognition}, we discuss our -time
algorithm to recognize if a graph is in . In
Section~\ref{sec:tw}, we show that the tree-width of a (pan, even
hole)-free graph is at most 1.5 times the clique number minus 1,
and thus the chromatic number is at most 1.5 time the
clique number. In Section~\ref{sec:conclusions}, we discuss open problems
related to our work.
\section{Background and results}\label{sec:definitions}
In this section, we discuss the relevant background and give
the definitions necessary to state our main results.  Let  be a
graph. For a subset  of the vertices of , we use 
to denote the subgraph of  induced by .  A {\em clique cutset} of  is a set  of vertices where
 is a clique whose
removal increases the number of components of . The following theorem is well known.
\begin{theorem}\label{thm:dirac}{\rm \cite{Dir1961}}
Every chordal graph is either a clique or contains a clique
cutset.
\end{theorem}
Recall that a vertex is \emph{simplicial} if its neighborhood
induces a clique,  and a vertex is \emph{bi-simplicial} if its
neighborhood can be partitioned into two cliques (i.e., its
neighborhood induces the complement of a bipartite graph). It
follows from Theorem~\ref{thm:dirac} that every chordal graph
contains a simplicial vertex.

Let  denote the degree of a vertex  in a graph .
When the context is clear, we will write  to mean .
Let  (respectively, ) denote the chromatic
number (respectively, the clique number, i.e., the number of
vertices in a largest clique) of . If  is a simplicial
vertex of , then  and
. An analogous property
was established for even-hole-free graphs. (Recall a hole is {\it
even} ({\it odd}) if it has an even (odd) number of vertices.)
\begin{theorem}\label{thm:A}{\rm \cite{AddChu2008}} Every
even-hole-free graph contains a bi-simplicial vertex.
\end{theorem}
Theorem~\ref{thm:A} implies that for even-hole-free graphs ,
a largest clique can be found in polynomial time and that . However,
it is not known if the coloring problem can be solved in polynomial time for even-hole-free
graphs.

The clique (respectively, chordless cycle, chordless path) on 
vertices is denoted by  (respectively, , ). Recall that the
{\it claw} is the graph with vertices  and edges
;  vertex  is the {\em center} of the claw.
Let   be a graph and let  be a family of graphs. We say that a graph
 is {\em -free} if  does not contain
an induced subgraph isomorphic to  and  is {\em -free}
if  does not contain an induced subgraph isomorphic to any graph in~.
In particular,  
is (claw, even hole)-free if  does not contain a claw or an
even hole as an induced subgraph.

Theorem~\ref{thm:A} implies that
an even-hole-free graph contains a vertex that is not the center
of a claw. This suggests that even-hole-free graphs such that no vertex
is the center of a claw (i.e., (claw, even hole)-free graphs)
might have interesting structure. Indeed, our results show that
(claw, even hole)-free graphs can be decomposed by the clique
cutset decomposition into (essentially) unit circular-arc graphs.
Our results actually apply to a larger class of graphs that we
will define later in this section (see Theorem \ref{thm:structure}).
\begin{figure}[h]
\centering
\includegraphics[scale=1.5]{claw.pdf}
\caption{The \emph{claw} with center .}
\label{fig:small}
\end{figure}

Consider  the following procedure to decompose a graph  . If
 has a clique cutset , then  can be decomposed into
subgraphs  and  where  and . Note: there is no edge between ) and .  Given minimum colorings of  and , we
can obtain a minimum coloring of  by identifying the coloring
of  in  with that of  in . In particular, we have
. If  ()
has a clique cutset, then we can recursively decompose  in
the same way.  This decomposition can be represented by a binary
tree  whose root is  and where the two children of  are
 and , which are in turn the roots of subtrees
representing the decompositions of  and . Each leaf of
 corresponds to an induced subgraph of G that contains no
clique cutset; such an induced graph is called an {\em atom} of
. Algorithmic aspects of the clique cutset decomposition are
studied in \cite{Tar1985} and \cite{Whi1984}. In particular, the
decomposition tree  can be constructed in  time such
that the total number atoms is at most  \cite{Tar1985}. We
have seen in the discussion above that the clique cutset decomposition
can be used to color a graph. With , , and  defined as
above,  contains an even hole (or odd hole) if and only if
 or  does. Thus, the clique cutset decomposition can
also be used to find an even hole (or odd hole), if one exists.

Recall that a {\it pan} is the graph obtained from taking a  with
, adding another vertex , and joining  to a vertex
 of the  by an edge. The edge  is called the {\em
handle} of the pan.  We will show that (pan, even hole)-free atoms
have very special structure. This structure allows us to solve the
recognition and coloring problems. To describe this structure, we will need to
introduce more definitions.

A graph  is a {\em circular-arc} graph if there is a bijection
between its vertices and a set  of arcs on a circle such that
two vertices of  are adjacent if and only if the two
corresponding arcs of  intersect. A circular-arc graph is {\it
proper}  if no arc contains another. Additionally,  is a {\em
unit} circular-arc graph if every arc of  has the same length.
It is easy to see that unit circular-arc graphs are proper and
that proper circular-arc graphs are claw-free and hence pan-free.

Let  and  be two disjoint sets of vertices. We say  is {\em
-null} if there is no edge between  and , and  is {\em
-complete} if every possible edge between  and  is
present. For a vertex ,  denotes the set of neighbors
of  in . When the context is obvious, we use  for .  For a set  of vertices,  denotes the set of
vertices outside  that have neighbors in . A vertex  {\em
dominates} a vertex  if .
Vertex  {\em strictly dominates} vertex  if . Two vertices are {\em comparable} if one dominates
the other. The domination relation is transitive, that is, if 
dominates  and  dominates , then  dominates . Thus,
given a set  of vertices such that any two vertices in  are
comparable, there is a total order  on  such that  whenever  dominates . We call such order a {\it
domination order}.  Two vertices  and  are comparable in  if
they are comparable in the subgraph induced by .



Let  and  be two vertex-disjoint graphs. The {\em join} of  and
 is the graph  obtained from  and  by adding every
edge between the vertices of  and those of ; thus, in 
the vertices of  are -complete and vice versa.

We now state our decomposition theorem for the class .

\begin{theorem}\label{thm:structure}
If  is a connected graph in  (i.e. every atom of 
is (pan, even hole)-free), then
\begin{enumerate}[(i)]
  \item  is a clique, or
  \item  contains a clique cutset, or
  \item  is  a unit circular-arc graph, or
  \item  is the join of a unit circular-arc graph and a
clique.
\end{enumerate}
\end{theorem}


In \cite{CamEsc2007}, a polynomial-time algorithm is given for
finding an even hole (or, odd hole) in a circular-arc graph. In
\cite{OrlBon1991} or combining \cite{LS2008} and \cite{ShihHsu},
polynomial-time algorithms are given for
finding an optimal coloring of a unit circular-arc graph. In
Sections~\ref{sec:coloring} and~\ref{sec:recognition}, we discuss
how these results can be used to color and recognize
both the class  and the class of (pan, even hole)-free graphs.
In Section~\ref{sec:properties}, we will
establish structural properties of (pan, even hole)-free graphs.
Specifically, we will prove Theorem 2.3.

Our main result is:
\begin{theorem}\label{MainResult}
Given a graph , a pan or even hole of , if one exists,
can be found in  time.
\end{theorem}

We end this section with a discussion on the relationship between
even-hole-free graphs and -perfect graphs, which were
introduced in \cite{MarGas1996} and are defined as follows. Let   denote the
minimum degree of a vertex of a graph . Order the vertices of
 as  where  has minimum degree in
. Greedily color  starting from ,
i.e.,  is given the smallest color distinct from its
neighborhood in . The number of colors used
is at most maximum is an induced subgraph of
, which is denoted .  Thus for any graph , we
have . A graph is defined to be
-perfect, if for every induced subgraph  of ,
. An even hole  has  and
.  It follows that -perfect graphs
are even-hole-free.  A diamond is the complete graph on four
vertices with an edge removed. In \cite{KloMul2009}, it is proved
that (diamond, even hole)-free graphs are -perfect.  In
\cite{MarGas1996}, the authors gave the graph of
Figure~\ref{fig:non-beta-perfect} as an example of an
even-hole-free graph which is not -perfect. This graph is
claw-free and hence pan-free, so it follows that (claw, even
hole)-free and thus (pan, even hole)-free graphs need not be
-perfect.
\begin{figure}[h]
\centering
\includegraphics[scale=1.2]{not-Beta-Perfect.pdf}
\caption{A claw-free, non--perfect graph}
\label{fig:non-beta-perfect}
\end{figure}
\section{Properties of (pan, even hole)-free graphs}
\label{sec:properties}
In this section we will prove our structure theorem (Theorem
\ref{thm:structure}). We will actually prove a stronger but
more technical result which implies Theorem
\ref{thm:structure} (see Theorem \ref{thm:structure_buoy}).
We separate the discussion into two
subsections. In the first we consider a special substructure of a
graph which generalizes holes: we call this substructure a
``buoy". In the second we prove that (pan, even hole)-free
graphs decompose into buoys via clique cutsets.

\subsection{Buoys}
\label{sec:buoy}
To motivate our buoys we start with a key observation
regarding the structure around a hole in a pan-free graph.
\begin{observation}\label{obs:neighbors}
Let  be a pan-free graph. Let  be a hole of  of length
at least five and let  be a vertex outside~.
\begin{enumerate}[(i)]
 \item If  has a neighbor  in , then some  is
 adjacent to both  and .
\item If  has exactly three neighbors  in
 , then  forms a path in .
 \item If  has exactly four neighbors in , then 
 contains an even hole.
 \item If  has at least five neighbors in , then  is
 -complete.
  \item If  is even-hole-free, then  has 2, 3, or  neighbors in , where
   is the length of~.
\end{enumerate}
\end{observation}

\begin{proof}
Enumerate the vertices of  in the cyclic order as . Suppose that (i) is false for a neighbor 
of . We may assume , i.e.,  is adjacent to ,
and non-adjacent to  and . Vertex  must have
another neighbor in , for otherwise  and  form a pan. Let
 be the smallest subscript, different from , such that 
is a neighbor of . If , then  induces a pan. If , then
 induces an pan. We
have established (i).
Now (ii) follows immediately from (i). For (iii) suppose  is
an odd hole (otherwise, we are done). Now, by (i), these four
vertices either form single sub-path 
of  or two non-adjacent paths  and 
such that  can be written as 
(for non-empty paths ). In the former case, an even hole
is induced by . In the latter
case, one of  and  is an
even hole. We now prove (iv). Suppose  has at least five
neighbors in  but is not -complete (this implies ). We may assume  is adjacent to  and non-adjacent to
. Let  be the smallest subscript, different from , such
that  is a neighbor of . By (i),  is adjacent to
 and . Since  has at least five neighbors
in ,  is adjacent to a vertex  with . But now  induces a pan.
Part (v) follows from (i)--(iv).
\end{proof}
For the purpose of finding a forbidden induced subgraph for
recognition of  class , we will now reformulate the
results of Observation~\ref{obs:neighbors} into their algorithmic
counter-parts. From the proof  of Observation~\ref{obs:neighbors},
we can extract a linear-time algorithm to find a pan or even hole
of an input graph when one of the conditions (i)-(v) fails.
\begin{observation}\label{obs:find-neighbors}
Let  be a graph, let  be a hole of  of length at least five,
and let  be a vertex outside . If  fails to satisfy (i)--(v)
of Observation~\ref{obs:neighbors}, then  contains a pan or an
even hole, and such an induced graph can be found in linear time.
\hfill 
\end{observation}



We generalize the presence of a length  hole (and
Observation \ref{obs:neighbors}) in a graph to the presence of an
\emph{-buoy} (defined as follows). For , an
{\em -buoy}  is a collection of sets  of vertices of  such that each  induces a
clique, each vertex in  has a neighbor in  and one
in , and there are no edges between  and , with subscripts taken modulo
 (see Figure~\ref{fig:5-buoy} for an example); the sets
 are called the {\em bags} of the buoy; a buoy is \emph{odd} or
\emph{even} depending on whether the number of bags () is odd or even.
We also refer to  as a buoy. A {\em skeleton} of
 is a hole containing one vertex of each , . A buoy  in a graph  is said to be
\emph{full} when it includes every vertex of . Due to the
cyclic structure of -buoys, when we refer to a bag  of
an -buoy, we always mean the bag .
We will see that when  is -free its buoys are circular-arc
graphs (see Theorem \ref{thm:buoy-square-circarc}). Similarly,
when  is (pan, even hole)-free its buoys are unit
circular-arc  graphs (see Theorem \ref{thm:buoy-unit-circ}).
\begin{figure}[h]
\centering
\includegraphics[scale=1.2]{5-buoy.pdf}
\caption{An example of a 5-buoy.}
\label{fig:5-buoy}
\end{figure}
\begin{observation}\label{obs:hole}
Let  be an even-hole-free graph having an odd -buoy with bags . Consider a path  with  where  for . Then where
,  belongs to a skeleton of .
\end{observation}
\begin{proof}
By definition of the buoy, there is an induced path  such that  for . We may assume  is not adjacent to , for
otherwise we are done. Let  be a neighbor of  in
. We have , for otherwise
 induced an even
hole. If , then  induces a
skeleton; if , then  induces a skeleton.
\end{proof}
\begin{figure}[h]
\centering
\includegraphics[scale=1.2]{buoy-skeleton.pdf}
\caption{The skeleton containing the path 
where  as in the proof of Observation
\ref{obs:hole}. Note: the bold edges connecting  to
 and  to  correspond to the paths
connecting these vertices.}
\end{figure}


\begin{corollary}\label{cor:hole}
Let  be an even-hole-free graph having an odd -buoy with bags 
 . Let  be a path with
vertices  where  and  for all  (with the subscripts taken
modulo ), then  belongs to a skeleton of .
\hfill
\end{corollary}
From the proof of Observation~\ref{obs:hole}, we can extract a
linear-time algorithm to establish the following observation.
\begin{observation}\label{obs:find-hole}
Let  be a graph having an odd -buoy with bags . Consider a path  with  where  for . Then there is a linear-time algorithm that either
finds an even hole, or a skeleton containing the set   (for a given ). \hfill

\end{observation}

\begin{observation}\label{obs:buoy-square}
Let  be a -free graph. Let  be an -buoy of 
with bags . Then any two vertices  and 
in the same  are comparable in . By symmetry, 
and  are comparable in .
\end{observation}
\begin{proof}
Let  and  be two vertices in . Suppose they are not
comparable in . Then there are vertices 
with  and . Now, the four
vertices  form a .
\end{proof}
From the proof of Observation~\ref{obs:buoy-square}, we can
extract a linear-time algorithm to establish the following
observation.
\begin{observation}\label{obs:find-buoy-square}
Let  be a  graph. Let  be an -buoy of  with bags
 . If two vertices  and  in the same 
are not comparable in , then  contains a , and
this  can be found in linear time. \hfill 
\end{observation}

\begin{theorem}\label{thm:buoy-square-circarc}
If  is an -buoy of a -free graph , then  is a
circular-arc graph.
\end{theorem}
\begin{proof}
Let  be the bags of . We construct a
circular-arc representation of  as follows. First we partition
the circle into  arcs of equal length and label the boundary
points of these arcs as  in clockwise
order. By Observation~\ref{obs:buoy-square} the vertices of 
can be partitioned and ordered by neighborhood inclusion with
respect to .

That is, we let  such that for
,  and
for  and , the neighborhood of 
is a strict subset of that of  with respect to  (i.e.,
  ).
That is,  can be partitioned into  subsets  where: (1) if , then they
have the same neighbors in  (i.e., ), and (2) if  and , then  in  the neighborhood of  is a strict
subset of the of the neighborhood of   (i.e., ).
From this partitioning of  and  we can easily
construct arcs between  and  to capture the edges
between vertices of  and . This is depicted in
Figure~\ref{fig:circular_arc} and described as follows.

For every , we place  equally spaced points  on the arc from  to  and:
\begin{itemize}
\item for a vertex  in  we use the arc from 
to .
\item for a vertex  in  we use the arc from
 to .
\end{itemize}
Clearly, the arc from  to  precisely intersects all
arcs from vertices in  and the arcs of 's neighbors in
. Similarly, the arc from  to  precisely
intersects all arcs from vertices in  and the arcs of
's neighbors in .

Notice that, for each  there is a unique triple
 of indices where . Thus,
by performing this construction for each , each  will be mapped to an arc from
 to ; i.e., we have a circular-arc representation
for .
\end{proof}
\begin{figure}[h]
\centering
\includegraphics[scale=1.2]{circular_arc.pdf}
\caption{The partial arcs between  and  where the
arc  represents the vertices from  and the arc
 represents the vertices from .}
\label{fig:circular_arc}
\end{figure}
Let  be an -buoy of a graph  with bags . Consider a vertex  of some bag . We say 
is  a {\it dominant} vertex of  if (in ) it dominates
every other vertex of .
\begin{observation}\label{obs:buoy-comparable}
Let  be an odd -buoy of an even-hole-free graph  with bags
. For every    and every pair of vertices  in
,  and  are comparable in . In particular, each
 contains a dominant vertex.
\end{observation}
\begin{proof}
Figure~\ref{fig:comparable} depicts the structure we observe
in this proof.
Suppose some pair of vertices  in  are incomparable.
Then by Observation~\ref{obs:buoy-square}, there are
vertices  with 
and . Now, by Observation~\ref{obs:hole}
and for the edges  and ,  has skeletons
 and
 where .

Notice that if  is an edge, then  is an even hole.
Thus,  and each  is distinct from
each . Moreover,  is an edge (otherwise
 is an induced ). But now  is an even hole
since , , and
, a contradiction. Since the
domination relation is transitive, every bag  contains a
vertex  that dominates every other vertex of , i.e.,
 is a dominant vertex of 
\end{proof}
\begin{figure}[h]
\centering
\includegraphics[scale=1.5]{buoy-comparable.pdf}
\caption{The -buoy from the proof of Observation~
\ref{obs:buoy-comparable}. Note: we have duplicated bag
 for ease of presentation. Also, the bold edges
connecting  with  and  with 
correspond the to the paths  and  respectively.}
\label{fig:comparable}
\end{figure}
From the proof of Observation~\ref{obs:buoy-comparable}, we can
extract a linear-time algorithm to establish the following
observation.
\begin{observation}\label{obs:find-buoy-comparable}
Let  be an odd -buoy of a graph  with bags . If there are vertices  and  in
some  such that  and  are incomparable in , then 
contains an even hole, and this even hole can be found in linear
time. \qed
\end{observation}
Later (Lemma~\ref{lem:find-buoy-domination}) we will show that
the domination property of Observation~\ref{obs:buoy-comparable}
can be verified in linear time.
\begin{observation}\label{obs:buoy-pan}
Let  be a (pan, even hole)-free graph. Let  be an
-buoy of  with bags . Let  and 
be two vertices in some . If  strictly dominates  in
, then  dominates  in .
\end{observation}
\begin{proof}
Let  and  be two vertices in . Suppose  strictly dominates
 in , but  does not dominate  in .
Thus, there are vertices  such that
 and . By
Corollary~\ref{cor:hole}, there is a skeleton  containing the
vertices . Now  together with  induces a pan in
.
\end{proof}


From the proof of Observation~\ref{obs:buoy-pan}, we can extract
an algorithm to establish the following observation.
\begin{observation}\label{obs:find-buoy-pan}
Let  be a graph. Let  be an -buoy of  with bags
. Let  and  be two vertices in some
. If  strictly dominates  in , and  does
not dominate  in , then  contains a pan or an even
hole, and such an induced subgraph can be found in linear time.
\hfill 
\end{observation}


Observations~\ref{obs:buoy-pan} and \ref{obs:buoy-comparable} tell
us that the structure of a buoy in a (pan, even hole)-free graph
is very restricted (see Corollary \ref{cor:buoy-clique} below).
Additionally, this structure allows us to prove that a buoy in a
(pan, even hole)-free graph is a unit circular-arc graph (see
Theorem \ref{thm:buoy-unit-circ} below).
\begin{corollary}\label{cor:buoy-clique}
Let  be a (pan, even hole)-free graph and let  be an
-buoy of  with bags . For every
 either  or  is a clique.
\end{corollary}
\begin{proof}
Consider a bag .  By Observation~\ref{obs:buoy-comparable},
we can order the vertices of  as  such that  dominates  whenever .
In particular, the vertex  is a dominant vertex of ,
and  is adjacent to all of . We may
suppose  is not a clique, for otherwise we are
done. Consider a vertex  that is not adjacent to some
vertex  in . Thus,  strictly dominates
, and therefore . By
Observation~\ref{obs:buoy-pan},  dominates  in
, and so  is adjacent to all of . By
the domination order of , every other vertex in  is
adjacent to all of . Thus  induces a
clique.
\end{proof}

\begin{theorem}\label{thm:buoy-unit-circ}
If  is an -buoy in a (pan, even hole)-free graph ,
then  is a unit circular-arc graph.
\end{theorem}
\begin{proof}
This proof is an easy adaptation of the construction from the
proof of Theorem~\ref{thm:buoy-square-circarc}.
By Corollary~\ref{cor:buoy-clique}, when  is
not a clique, both  and  must be cliques. In particular, when 
is not a clique we use nearly the same construction as
before and exploit the fact that  and  are cliques to ensure all of our arcs have the
same length. Figure~\ref{fig:unit_circular_arc} depicts our
construction and we will refer to it as we describe the
details.

\begin{figure}[h]
\centering
\includegraphics[width=\linewidth]{unit_circular_arc.pdf}
\caption{The unit circular-arc construction from the proof of
Theorem~\ref{thm:buoy-unit-circ} for the case when  is not a clique.} \label{fig:unit_circular_arc}
\end{figure}

The arcs we construct will have length . As in the
previous case we first partition the circle into arcs. We then
use these arcs to place the endpoints of the arcs for the
vertices of .

We partition the circle into arcs as follows. For each bag
 we allocate an arc  of length . The
midpoint of  will be the point  from
the proof of Theorem~\ref{thm:buoy-square-circarc}.
For each  we allocate an arc  such that:
\begin{itemize}
\item When  is a clique, the length of  is two.
\item When  is not a clique, the length of  is one.
\end{itemize}
These arcs are arranged as  around the circle so that the circle
is covered and consecutive arcs intersect in precisely one
point.

As we mentioned, when  is not a clique we
use the previous construction subject to the constraint that
the length of the arc from  to  is one (note:
 since  is not a clique).

In the first half of  we insert a copy of the points
of .
In particular, the left endpoint of  is a copy of
 and this is followed by ,
 with precisely the same spacing as in .
With these points we can now create the arcs for the vertices
in . Specifically, using the same partition  as
before, each  is represented by the arc
from the copy of  in  to the original 
in . It is important to note that each such arc
includes the midpoint of , has length ,
and includes the same points between  and  as in
our previous construction.

We similarly, insert a copy of the points of  in the
second half of . Specifically, the midpoint of
 is a copy of , which is followed by ,  with precisely the same spacing
as in .
With these points we can now create the arcs for the vertices
in . Specifically, using the partition 
as before, each  is represented by the arc
from the original  in  to the copy of 
in . It is important to note that each such arc
includes the midpoint of , has length ,
and includes the same points between  and  as in
our previous construction.

We need only consider one special case to complete our
construction, namely, when both  and
 are cliques. In this case we simply map
each vertex of  to an arc from the midpoint of 
to the midpoint of . This again provides arcs of length
.

Now, when  is not a clique, this construction
properly represents the edges between  and  since we
simply have the same representation as before. Additionally, when
 is a clique, the arcs of  and 
always include the midpoint of . This again properly
represents the edges between  and . Thus, we have
produced a unit circular-arc representation of .
\end{proof}
\subsection{Neighbors of Buoys}
\label{sec:buoy-neighbor}
We now generalize the results of Observation 3.1 to buoys. We
examine the different \emph{types} of adjacencies between vertices
outside a buoy  in a (pan, even hole)-free graph and vertices
inside . Let  be a vertex of  outside of . We say 
is of type  with respect to  if  has neighbors in exactly
 distinct bags . It is easy to see that in a pan-free
graph,  cannot be of type 1. The following lemma describes
possible adjacencies between  and .
\begin{lemma}\label{lem:buoy-outside}
Let  be a (pan, even hole)-free graph. Let  be an odd -buoy
of  with bags  and let  be a vertex of  that has some neighbors in
.

\begin{enumerate}[(i)]
\item If , then , or  .
\item If  and  are indices such that   ,  has a neighbor in each of  and , and 
has no neighbors in   , then
 is odd (i.e., the number of bags  is
even).
\item If  has neighbors in  and neighbors in
, then  is -complete.
\item Vertex  is of types 2, 3, or . If  is of type
, then  is -complete.
\item If  is of type 3, then  has neighbors in three
consecutive s.
\item Suppose  is of type 2 and has neighbors in  and in
. Then  is a clique.
\end{enumerate}
\end{lemma}
\begin{proof}
Let  be a fixed bag. Let  be a dominant vertex of ,
. The vertices  exist for all  by
Observation \ref{obs:buoy-comparable}.

\textit{Proof of (i)}. Suppose  is adjacent to some  and is -null. Let 
be a neighbor of  in  and let 
be a neighbor of  in . By Corollary~\ref{cor:hole}, there is a
skeleton  containing . But then part (i)
of Observation~\ref{obs:neighbors} is contradicted.  \qed

\textit{Proof of (ii)}. Suppose (ii) is false. Let  and  be
neighbors of  in  and  respectively. Since  is
a dominant vertex of  for all ,
     induces an even hole.
\qed

\textit{Proof of (iii)}. Suppose there is  which is
not a neighbor of . We will distinguish among three cases: (1)
 and  have common neighbors  and
; (2)   and  have no common neighbors
in ; and (3)   and  have a common
neighbor , but no common neighbor in
. In case (1),  induces a
. For case (2), let  be a neighbor of  in
 and let  be a neighbor of  in .
Note that  must have neighbors  in   and
 in . Thus, in
this case, 
induces a .

Now we handle case (3). Let  be a neighbor of  in
 and  be a neighbor of  in . The
dominant vertex  of  is adjacent to both  and
. Thus,  is adjacent to , for otherwise  induces a .

Suppose  is not adjacent to . By
Corollary~\ref{cor:hole}, there is a skeleton  containing
. But then  and  contradict
Observation~\ref{obs:neighbors} (i). Thus  is  adjacent to
. Let  be the path .
Corollary~\ref{cor:hole} implies there is a skeleton 
containing . Since  is a dominant vertex of ,
we may assume  ( may replace the vertex of
). Since  has at least three neighbors and one
non-neighbor () on , Observation~\ref{obs:neighbors} (v)
implies that  has exactly three neighbors on . In
particular, we have . Let . Then  is a hole. But now, 
has at least four neighbors and one non-neighbor () on
, a contradiction to Observation~\ref{obs:neighbors} (v). \qed




\textit{Proof of (iv)}. Suppose  is of a type different from 2,
3, or . There are indices  such that  has neighbors
in each of  and  and no neighbors in  and  . By (i), 
has neighbors in  and in . By (ii), the number
of sets  is even. So the number of
sets  is odd. Let  be a
neighbor (if one exists) of  in , for all . Consider
the path . Vertex 
must be adjacent to an interior vertex  of this path, for
otherwise  and  form an even hole. But now there is a pan
formed by the vertices .

If  is of type , then by (iii),  is -complete for
all . \qed

\textit{Proof of (v)} Let  be of type 3. Assume  has a
neighbor in  for some . By (i) we may assume  has a
neighbor in . Let  be the third bag such that 
has neighbors in . If , then (i) is
contradicted. \qed

\textit{Proof of (vi)}. Suppose  is of type 2 and has neighbors
in  and  and let  and  be neighbors of . If  and  are not
adjacent, then     
 induces an even hole. Thus, the neighbors of  in
 form a clique.

We now show that  is adjacent to every neighbor  of . Suppose  is not adjacent to .
Consider the skeleton  formed by the vertices . Vertex  has only one
neighbor on this hole, a contradiction to part (v) of
Observation~\ref{obs:neighbors}. By symmetry,  is adjacent to
every neighbor  of .

From the previous paragraph  must be adjacent to both  and
 since they are neighbors  of  and 
respectively. Thus,  is adjacent to all of 
and as such  form a clique.
\end{proof}
From the proof of Lemma~\ref{lem:buoy-outside}, we can extract a
linear-time algorithm to establish the following lemma.
\begin{lemma}\label{lem:find-buoy-outside}
Let  be a graph. Let  be an odd -buoy of  with bags
, and let  be a
vertex of  that has some neighbors in . If  fails to
satisfy (i)--(vi) of Lemma~\ref{lem:buoy-outside}, then 
contains a pan or an even hole, and such an induced subgraph can
be found in linear time. \hfill 
\end{lemma}

\subsection{Structure Theorem}
\label{sec:structure}

Now that we understand the structure of buoys (see Section
\ref{sec:buoy}) and their neighbors (see Section
\ref{sec:buoy-neighbor}), we are ready to prove the structure
theorem introduced in Section~\ref{sec:definitions}. We prove the
following  theorem which together with
Theorem~\ref{thm:buoy-unit-circ} implies
Theorem~\ref{thm:structure}. Recall that  is the class
of graphs  such that every atom of  is (pan, even
hole)-free.
\begin{theorem}\label{thm:structure_buoy}
If  is a connected graph in  then
\begin{enumerate}[(i)]
 \item  is a clique, or
 \item  contains a clique cutset, or
 \item For every maximal buoy  of , either  is a full
 buoy of , or  is the join of  and a clique.
\end{enumerate}
\end{theorem}
\begin{proof}
We may assume  is connected and contains an odd hole , for
otherwise  is chordal and the theorem holds. Let  be the
length of . Since  contains ,  contains a maximal buoy
 with bags  and skeleton 
(here, as usual, ``\emph{maximal}'' is meant with respect to set
inclusion, not size). If , then  is a full
buoy and we are done. Let  be the set of vertices in 
with some neighbor in , and  be the set of vertices in 
with no neighbor in . Consider a vertex  in . By
Lemma~\ref{lem:buoy-outside},  is of types 2, 3, or . If
 is of type 3, then, by  and  of
Lemma~\ref{lem:buoy-outside},  has neighbors in three
consecutive bags  and is complete to .
In particular,  is a larger buoy with bags , a contradiction to our choice of . Also, when  is of
type 2, then, by  and   of
Lemma~\ref{lem:buoy-outside}, there is an index  such that
 is a clique. Thus,  can be
partitioned into sets  such that
\begin{itemize}
\item  if and only if ;
and \item  if and only if  is -complete.
\end{itemize}

Note that all type- vertices are in . The set  (if
non-empty) induces a clique for otherwise,  two non-adjacent
vertices of  and  two non-adjacent vertices of  form a
. We may assume there is a non-empty , for otherwise 
is the join of  and  (if ), or   is a
clique cutset of  separating  and  (if ).

Consider a non-empty set  and let . The set  is a clique by Lemma~\ref{lem:buoy-outside}. We
will show that  is a clique cutset. Suppose it is not a clique
cutset. Then, in , there is a shortest path  from a vertex
 to a vertex .
Enumerate the vertices of  as  with  and . Since the path is shortest,  for some , ,  and  for  (when ). There are two induced paths
whose endpoints are  and , and whose interior
vertices are disjoint and lie in . We may enumerate one as , and the second one as  with  for all .
Since  is odd,   and  have different parities. Let
. One of the two holes induced by 
and  has to be even, a contradiction.
\end{proof}
An algorithm can be extracted from the proof of
Theorem~\ref{thm:structure_buoy} to prove the following theorem.
\begin{theorem}\label{thm:find-structure-buoy}
Let  be a maximal buoy of a graph . If  is not a full buoy of  and
if  is not the join of B and a clique, then  contains an even hole or a pan,
and such an induced subgraph can be found in linear time. \hfill 
\end{theorem}

\section{A coloring algorithm for (pan, even hole)-free graphs}
\label{sec:coloring}
In this section, we discuss a polynomial-time algorithm to color a
graph in . Consider a graph   with a clique cutset
decomposition tree . From the discussion in
Section~\ref{sec:definitions}, if we can color the atoms of  in
polynomial time, then we can also color . The purpose of this
section is to show that  can indeed be colored in polynomial
time.

In \cite{OrlBon1991}, an -time algorithm is given for coloring
proper circular-arc graphs. For unit circular-arc graphs, this can be improved.
First, we use the -time algorithm of \cite{LS2008} for recognizing unit circular-arc
graphs to construct a unit circular-arc representation. Then we use the -time
algorithm of \cite{ShihHsu} to find a minimum coloring of a unit circular-arc graph
given the representation.  This gives an -time algorithm to color unit
circular-arc graphs.

Thus, from
Theorems~\ref{thm:buoy-unit-circ} and \ref{thm:structure_buoy}, we have the following two results.
\begin{theorem}\label{thm:color-join}
There is an -time algorithm to find a minimum coloring of
a (pan, even hole)-free graph that is either a buoy or the join of  a buoy and a clique.
\end{theorem}
\begin{proof}
Let  be a (pan, even hole)-free graph that is the join of a
clique  and a buoy . Then we have .
Thus, we only need to establish the theorem for (pan, even
hole)-free buoys. Now the result follows from
Theorem~\ref{thm:buoy-unit-circ} and the -time
algorithm to color unit circular-arc graphs.
\end{proof}

\begin{theorem}\label{thm:color-main}
There is an -time algorithm to find a minimum coloring of
a graph in .
\end{theorem}
\begin{proof}
By the discussion above and the fact that the clique cutset
decomposition provides at most  atoms, we only need show
there is an -time algorithm to color a  (pan, even
hole)-free atom . By Theorem~\ref{thm:structure},  is one of
the following: a clique, a buoy, or the join of a clique and a
buoy. Thus, by Theorem~\ref{thm:color-join},  can be optimally
colored in  time.
\end{proof}

\section{Recognition algorithms for (pan, even hole)-free graphs}
\label{sec:recognition}
In this section, we give two polynomial-time algorithms to recognize
(pan, even hole)-free graphs. We note that a polynomial-time algorithm
for recognizing (pan, even hole)-free graphs can easily be
converted to a polynomial-time algorithm for recognizing graphs in~.

There exist several polynomial-time algorithms (\cite{ChaLu2013,
ChuKaw2005, ConCor2002}) for finding an even hole in a graph. But
the fastest such algorithm \cite{ChaLu2013} runs in time
. A straight-forward algorithm to
recognize a (pan, even hole)-free graph is to test for a pan using
Theorem~\ref{thm:find-pan} below, and then to test for an even
hole. In particular, we can recognize (pan, even hole)-free graphs
in  time. We will design faster algorithms for (pan,
even hole)-free graph recognition. We provide two recognition
algorithms. The first uses the fact that the (pan, even hole)-free
atoms are unit circular-arc graphs and recognizes (pan, even
hole)-free graphs in  time. The
second uses the fact that the atoms are essentially very special
buoys and runs in  time.

Similar to our coloring algorithm, we note that detecting an
even hole in a graph  is easily reduced to checking for an
even hole in an atom. That is, suppose a graph  has a clique
cutset  and consider the subgraphs  and  where  and . Then 
contains an even hole if and only if  or  does; i.e.,
when testing for even holes one need only consider atoms.

As we have mentioned previously, the clique cutset decomposition
tree  can be computed in  time such that there are
fewer than  atoms \cite{Tar1985}.



\subsection{Recognition via testing for pans and then testing for even holes in unit circular-arc atoms}

We will first describe an algorithm to find a pan in
a graph.

\begin{lemma}\label{lem:hole-in-atom}
Let a graph  be an atom. Then every vertex  of  is
universal, or lies in a hole of . Furthermore, there is a
linear-time algorithm to find a hole containing  when  is
not universal.
\end{lemma}
\begin{proof}
Let  be an atom and  be a vertex of . Let . If , then  is universal.
Compute the components  of . For each
, compute the set  of vertices in  that have some
neighbors in . If some  is a clique, then  is a
clique cutset separating  from , a contradiction. Thus,
none of the s are cliques. Choose an arbitrary set
. Consider two non-adjacent vertices  and  in . Find a
chordless path  from  to  whose interior vertices lie
entirely in . Then  and  induce a hole in .
\end{proof}
\begin{lemma}\label{lem:find-hole}
Given a graph  and a vertex  in , there is an 
time algorithm to find a hole containing , if such a hole
exists.
\end{lemma}
\begin{proof}
Construct in  time the clique cutset decomposition 
of . Consider all the atoms of  containing . If  is a universal
vertex in all such atoms, then  does not lie on any hole of .
Suppose  is not universal in some atom . By
Lemma~\ref{lem:hole-in-atom}, we can find a hole
containing  in linear time.
\end{proof}
\begin{theorem}\label{thm:find-pan}
There is an -time algorithm to find a pan in a graph, if
one exists.
\end{theorem}
\begin{proof}
For an edge  we can check, by Lemma~\ref{lem:find-hole}, in
 time whether  is the handle of a pan by finding a
hole containing  (respectively, ) in the subgraph of 
induced by  (respectively, .) Since  has  edges, the time bound of the theorem
follows.
\end{proof}





Now, to recognize whether  is a (pan, even hole)-free graph, we
first use Theorem~\ref{thm:find-pan} to test for a pan. If  has
no pan, find the clique cutset decomposition.

For an atom , Theorem~\ref{thm:structure} implies that  is
either a unit circular-arc graph or the join of a clique  and a
unit circular-arc graph . In the latter case,   is
even-hole-free if and only if  is even-hole-free. One can test
whether a graph is a unit circular-arc graph in linear time
\cite{LS2008}. In particular, if  is not unit circular-arc,
then we know  must have an even hole (by
Theorem~\ref{thm:buoy-unit-circ}). Additionally, an -time algorithm is known for finding an even (or odd) hole
in a circular-arc graph \cite{CamEsc2007}. That is, via the clique
cutset decomposition, we can test whether a graph in 
contains an even hole in  time (since the
decomposition can be computed in  time and has 
atoms).



Thus, for a given graph , we can recognize whether  is
(pan, even hole)-free in  time.



\subsection{Recognition via buoy construction}

We now present an algorithm which relies on the buoy structure of
a (pan, even hole)-free graph to test whether an atom is (pan,
even hole)-free, and if it is not, to find a pan or even hole.
Recall that, by Theorem \ref{thm:structure_buoy}, in a (pan, even
hole)-free atom  either every maximal buoy is a full buoy or
 is the join of a clique and a buoy. With this approach, we do
not attempt to directly find a pan. Instead, a pan (if it exists)
can be found by examining the buoys and their neighborhoods.

An atom  of graph  is {\it maximal} if any induced subgraph
 of  that properly contains  is not an atom, i.e., if 
has a clique cutset. The atoms produced by the clique cutset
decomposition are maximal.

The algorithm will produce a forbidden induced subgraph, if one
exists. The algorithm has three steps.
\begin{enumerate}[(1)]
 \item Find a clique cutset decomposition tree  of .
 \item For each (maximal) atom  of ,  (i) extract a forbidden induced
subgraph (if one exists) from , or (ii) show that  is a buoy, or (iii)  find a partition of
the vertices of  into the
join of a buoy and a clique. The involved buoy will
satisfy Observation~\ref{obs:buoy-comparable}.
 \item For each atom  of , verify that no holes of 
    form a pan with a vertex outside~.
\end{enumerate}
We will show that steps (2) and (3) can be done in linear time for an
atom. This shows the algorithm runs in  time.



The correctness of step (2) follows from the following theorem.
\begin{theorem}\label{thm:test-atom}
Let  be an atom. There is a linear-time algorithm to output
\begin{enumerate}[(i)]
   \item a pan, or
   \item an even hole, or
   \item a certificate that  is (pan,
   even hole)-free, and  either a certificate that  is a buoy or
   a partition of  into sets  and
     such that  is a buoy,   is a clique, and  is the
   join of  and .
\end{enumerate}
\end{theorem}
To prove Theorem~\ref{thm:test-atom}, we will need the following
three lemmas.
\begin{lemma}\label{lem:buoy-length}
If  is an -buoy where each  can be ordered by
neighborhood inclusion, then every hole in  has length .
\end{lemma}
\begin{proof}
Consider a hole  of . No two vertices of  have comparable
neighborhoods. Thus, each bag of  contains at most one vertex
of . If  has fewer than  vertices, then   is not a
hole (a contradiction) since vertices in a bag can only have
neighbors in the bag preceding it and the bag following it in the
cyclic order. So  has length .
\end{proof}
\begin{lemma}\label{lem:clique}
Let  be an odd -buoy where each bag
 can be ordered by neighborhood inclusion. The following
three statements are equivalent
\begin{enumerate}[(i)]
 \item There are  two vertices  in some  such that
        strictly dominates  in , but  does not
       dominate  in .
 \item  has a pan.
 \item There is a subscript  such that  and
        are
        both not cliques.

\end{enumerate}
\end{lemma}
\begin{proof}
First, note that Lemma~\ref{lem:buoy-length} implies that  has no
even hole.  Now the fact that  (i)  (ii) follows
from Observation~\ref{obs:find-buoy-pan}. Next, we prove the
implication (ii)  (iii). Suppose  contains a
pan. By Lemma~\ref{lem:buoy-length}, the hole of this pan must
contain exactly one vertex of each bag. Let the pan consist of
vertices  where 
and the vertices  form a hole. Without loss of generality, we
may assume . So, (iii) is satisfied with .
Finally, we prove the implication (iii)  (i).
Suppose  and  are both not
cliques. Let  be the dominant vertex of , and  be the
vertex in  that is dominated by every other vertex of .
If  is adjacent to every vertex in , then every vertex in
 is adjacent to every vertex in , a contradiction. So
 is non-adjacent to some vertex of , i.e.,  strictly
dominates  in . A symmetric argument shows that  strictly
dominates  in .
\end{proof}
\begin{lemma}\label{lem:find-buoy-domination}
Let  be an -buoy  with bags  for
some . There is a linear-time algorithm to verify that the
bags of  admit a domination order, i.e., the vertices of each
 are pairwise comparable.
\end{lemma}
\begin{proof}
For each bag , we order its vertices by non-decreasing size
of their neighborhoods; i.e.,  is ordered as , where  with . This can be
done in  time via bucket-sort.
That is, sorting all of the bags can be done in  time. We
then check that for every , every
neighbor of  is a neighbor of  (if this is not
the case, then  is incomparable with ). For each
bag , this neighborhood checking can be performed in
 time. In particular, all such
checking can be performed in  time. Thus, we can check that
the bags of  admit a domination order in  time.
\end{proof}

Now we can prove Theorem~\ref{thm:test-atom}.

\begin{proof}
Suppose that  is an atom.  Using the linear-time algorithm in
\cite{TarYan1984}, we either confirm that  is chordal (and
hence is (pan, even hole)-free) or obtain a hole . We may
assume  is an odd hole. We first briefly describe the
algorithm. We will construct a maximal buoy  with  as its
skeleton in  time. During this process, we verify that 
has the domination property of
Observation~\ref{obs:buoy-comparable} or  contains a pan or
even hole. If  and  is not the join of  and a
clique, then we will find a pan or even hole.

We now describe our construction of a maximal -buoy  in
an atom  from a hole  of .
We start with the initial -buoy  with bags , , .

Let  be the set of universal vertices of . We remove
vertices in  from  since these cannot be part of an even
hole or a pan. Since  is not a clique cutset of , removing
 does not make the resulting graph disconnected. Also, if 
is a clique cutset of , then  is a clique cutset of
. Thus, the graph we obtain by removing  is still an atom.

Consider a vertex  in  with some neighbors in . If
 is of type  with , then by
Lemma~\ref{lem:buoy-outside}, we know  has a pan or even hole,
and we can find this forbidden induced subgraph by
Lemma~\ref{lem:find-buoy-outside}. So,  is of type 2, 3 or
. The only candidates to be added to  are type 3
vertices.  Suppose  is of type 3. By
Lemma~\ref{lem:buoy-outside}, either  is adjacent to three
consecutive bags  of , or  contains a
pan or an even hole. Suppose  is adjacent to three consecutive
bags . If  is not adjacent to all
vertices of , then by Lemmas~\ref{lem:buoy-outside}
and~\ref{lem:find-buoy-outside}, we will find a forbidden induced
subgraph. Now  is adjacent to all vertices of . We then
add  to . We summarize the operations described in the
above paragraph with Algorithm 1, named ENGLARGE and given below.
\begin{algorithm}
\caption{ENLARGE} \label{alg:enlarge}
\begin{algorithmic}

    \STATE
    \STATE Iterating over the edges from  to ,  label
    each vertex of  with the bags of its
neighbors in .
    \FOR{every vertex  of  with a label}
       \IF { has  labels with }
             \STATE output a pan or even hole, and stop
       \ENDIF
       \IF{ is labelled with three non-consecutive indices}
             \STATE   output a pan or even hole, and stop
       \ENDIF
       \IF { is labelled with three consecutive indices (say, )}
            \IF{ is not adjacent to all of }
                       \STATE output a pan or even hole, and stop
            \ELSE
            \STATE add  to 
            \ENDIF
        \ENDIF
    \ENDFOR

\end{algorithmic}
\end{algorithm}


Starting with our first buoy  which is an odd hole, we call
ENLARGE on  twice. We will show that after two calls to
ENLARGE, we can decide whether  is (pan, even hole)-free. After
the first (respectively, second) call to ENLARGE, let 
(respectively, ) be the resulting buoy, and let the bags of
  (respectively, ) be  (respectively, ). Note
that  for all . Using
Lemma~\ref{lem:find-buoy-domination}, we verify in linear time
that both  and  have the desired domination property, or
else we find a pan or even hole.

Suppose there is a vertex of  not belonging to any .
Consider a vertex  in  with neighbors in some of the
bags. If  is of type  with , then
by Lemma~\ref{lem:find-buoy-outside}, we can produce a pan or even
hole in linear time.

We will prove that  is of type 2 or . Suppose  is of
type 3. If  does not have neighbors in three consecutive bags,
then by Lemma~\ref{lem:find-buoy-outside}, we can produce a pan or
even hole in linear time. So  has neighbors in three
consecutive bags, say, . Vertex 
is adjacent to all of  , for otherwise, by
Observation~\ref{lem:find-buoy-outside} we can find a pan or even
hole. So,  is a buoy with bags  . We may assume  has the domination property of
Observation~\ref{obs:buoy-comparable}, for otherwise by
Observation~\ref{obs:find-buoy-comparable}, we will find a pan or
even hole. Thus, each bag  of  has a dominant vertex
. Since the vertex  is adjacent to three vertices of
,  is added to the buoy  in the first iteration.
Vertex  is adjacent to , so  would have
been added to  in the second iteration, a contradiction.


So  is of type 2 or . (From now on, we only refer to the
buoy produced after the second call; so to simplify notation, we
will let .) When  is of type 2, then, by   of
Lemma~\ref{lem:buoy-outside}, there is an index  such that
 is a clique, or else we can produce
a pan or even hole.   Thus, by  and  of
Lemma~\ref{lem:buoy-outside},  can be partitioned into
sets  such that
\begin{itemize}
 \item  if and only if ,
 \item  if and only if  is -complete,
 \item  if and only if  is -null.
\end{itemize}
We may now use the proof of Theorem~\ref{thm:structure_buoy} to
find a pan or even hole. The set  (if non-empty) induces a
clique for otherwise,  two non-adjacent vertices of  and two
non-adjacent vertices of  form a . If all sets  are
empty, then  is a clique cutset of . So, some  is
non-empty. Since  is an atom,  is not
a clique cutset separating  from .
Thus, there is a shortest path  from a vertex  to
a vertex  () whose interior vertices lie
entirely in . Find an induced path  with the
same parity as  from  to  whose interior vertices
belongs to . Then  is an even hole.

Thus, after two calls to ENLARGE, we have constructed a full buoy
 of the graph , where  is the set of universal
vertices we remove before the first call to ENLARGE.



To complete the proof, we only need to find a pan in , if one
exists. By Lemma~\ref{lem:clique},  has no pan if and only if
for every ,  or  is a
clique. This condition can be checked in  time. If the
condition fails for some , then the proof of
Lemma~\ref{lem:clique} shows that the dominant vertex  of 
strictly dominates the vertex  with the smallest
degree; and so we can find a pan using
Observation~\ref{obs:find-buoy-pan}. If  (the set of universal
vertices of ) is non-empty, then  is the join of the buoy  and ; otherwise,  is a full buoy of .
\end{proof}

Now we show that step (3) of our algorithm can be implemented in
 time. At this point, we know that the (maximal) atom 
of  under consideration is (pan, even hole)-free, and that
 is either a buoy or the join of a buoy and a clique. We need
to determine that no hole of  forms a pan with a vertex in ; we call such a pan {\em straddling}. We need to find a
straddling pan with respect to  (if one exists). An atom  of
a graph  is {\it maximal} if for any induced subgraph  of
 containing , either  has a clique cutset, or  is a
component of . The atoms produced by the clique cutset
decomposition are maximal. We will need the following observation.
\begin{observation}\label{obs:maximal-atom}
If  is a maximal atom of a graph , then for every vertex 
in ,  is either empty or a clique.
\end{observation}
\begin{proof}
Suppose  is not empty but is not a clique. Write . Since  is a maximal atom,  is not an atom,
i.e.,  has a clique cutset . If , then 
is a clique cutset of , a contradiction. So,  belongs to a
component  of . Vertex  cannot be the only vertex of
, for otherwise,  is a subset of , and therefore a
clique, a contradiction. But now  is a clique cutset of , a
contradiction.
\end{proof}
Now consider
an atom  of  that is either a buoy 
or the join of a
buoy  and a clique . We are going to
describe a way to  find a straddling pan (if one exists) whose
hole belongs to . (Vertices of  do not belong to a hole in .)
Let the bags of  be .
Remember that  is (pan, even hole)-free, so by Observation~\ref{obs:buoy-comparable},
each bag can be ordered by neighborhood inclusion, so by Lemma~~\ref{lem:buoy-length},
every hole in  has length .
Compute the set  of vertices of 
that have neighbors in . The set  can be computed in 
time. Let  be the set of vertices  of  with . Since  is a maximal atom, the graph
 contains a clique cutset  such that . It follows from Observation~\ref{obs:maximal-atom} that, with
respect to the buoy , every vertex in  is of type 1 or 2.
Furthermore, since  is (pan, even hole)-free, it follows from Lemma~\ref{lem:buoy-outside}(vi) 
that every vertex of type 2 belongs to some . If
some  is of type 1, then clearly a straddling pan can be
found in linear time.
Now, every vertex  in  is such that   is a clique, and we conclude there is no straddling
pan.

Thus, for a maximal atom, we can determine in  time whether
a straddling pan exists. Since there are at most  atoms of
, we can implement step 3 in  time. This completes the proof
of Theorem \ref{MainResult}:
\\
\\
\noindent\textbf{Theorem \ref{MainResult}} \textit{Given a graph , a pan or even hole of , if one exists, can be found in  time.}
\\


Note that for an input graph , if  is not (pan, even hole)-free, our algorithm produces a pan or an even hole.  If  is (pan, even hole)-free, the algorithm produces a clique cutset decomposition tree which satisfies Theorem \ref{thm:certifying} below; furthermore, the set of atoms of every clique cutset decomposition tree will satisfy (i) and (ii) below.

\begin{theorem}\label{thm:certifying}
A graph  is (pan, even hole)-free if and only if there is a clique cutset decomposition tree with most  atoms  such that
\begin{itemize}
\item[(i)] Each atom  is either a clique or consists of a buoy  and a possibly empty set  of universal vertices; the buoy  has an odd number of bags; each bag can be ordered by neighborhood inclusion; and, for each consecutive triple of bags either the first two or second two form a clique.
\item[(ii)] Further, for each atom  which is not a clique, the neighborhood of  in  can be partitioned into sets , some of which may be empty, where  is universal to the th and st bags of the buoy  and  has no other neighbors in~.
\end{itemize}
\end{theorem}

The correctness of the algorithm proves the ``only if" part of the theorem. To see that the ``if" part holds, note that a graph  satisfying property (i) of Theorem \ref{thm:certifying} is (pan, even hole)-free.  Since any hole of  must lie in some atom , property (ii) then guarantees that there is no straddling pan whose hole is in .

It follows from Theorem \ref{thm:certifying} that our algorithm is certifying.  The certificate given by Theorem \ref{thm:certifying} has size .

\section{Tree-width and -boundedness}
\label{sec:tw}

In this section we bound the \emph{tree-width} of (pan,
even hole)-free graphs in terms of their clique number (see
Theorem \ref{thm:tw}). This bound immediately provides a bound on
the chromatic number (see Corollary~ \ref{cor:chi-boundedness}).

A \emph{tree decomposition}  of a graph  is defined to
be a tree  together with a function  such that:
\begin{itemize}
\item For every ,  induces a subtree of .
\item For every , .
\end{itemize}
Notice that a simple tree decomposition of any graph can be
obtained by choosing  to be a single vertex and mapping every
vertex of  to this vertex. The vertices of  are often
treated as sets, referred to as the \emph{bags} of the tree
decomposition, and the elements of a bag  are the vertices 
of  where . For a tree decomposition  of a
graph , the \emph{tree-width} of , denoted , is
the size of the largest bag of  minus 1; i.e., ). For a graph , the \emph{tree-width} of ,
denoted , is the smallest tree-width of any tree
decomposition of .

We use the following three well-known  and easy results regarding
tree-width.

\begin{observation}\label{obs:tw-chi}
For a graph , .
\end{observation}
\begin{proof}
Let  be a tree decomposition of  with .
We create a supergraph  by completing the bags of  to
cliques. This resulting graph is chordal, and thus perfect. So we now
have .
\end{proof}

\begin{lemma}\label{lem:tw-cliquedecomp}
If  contains a clique cutset  where  are
the components of , then:

\end{lemma}

\begin{lemma}\label{lem:tw-join}
Let  be a graph that is the join of a graph  and a clique . Then

where  denotes the number of vertices of .
\end{lemma}

For further information on tree-width, see \cite{Reed}.

\begin{theorem}\label{thm:tw}
A (pan, even hole)-free graph  has .
\end{theorem}
\begin{proof}
By Lemmas~\ref{lem:tw-cliquedecomp} and~\ref{lem:tw-join} and
Theorem~\ref{thm:structure_buoy}, we only need  show that  for any buoy in . Recall that, for every bag
 of , either  is a clique or 
is a clique. In particular, we can build a tree representation
 of  where  is path using the unit circular-arc
construction from the proof of Theorem \ref{thm:buoy-unit-circ}.
To do this we choose the smallest , and ``split'' the unit
circular-arc representation at the point  and ``unroll'' it
onto a line. We now have a path where every point from our unit
circular-arc representation is a bag, and the extreme bags are
copies of the bag corresponding to the point . Thus, by
adding the vertices of  to every bag on this path, we obtain
a tree representation of . It is easy to see that the largest
bag in this representation has size .
\end{proof}

Theorem \ref{thm:tw} is tight since odd cycles  have tree-width two
and clique number two. Similarly, by making a buoy with an odd
number of bags such that each bag has  vertices and  is a clique for every , we have a graph whose
tree-width is  and whose clique number is . (See
Figure~\ref{fig:non-beta-perfect} for an example with .)
Moreover, by Observation \ref{obs:tw-chi}, we obtain the following
corollary.

\begin{corollary}\label{cor:chi-boundedness}
A (pan, even hole)-free graph  has .
\hfill 
\end{corollary}

\section{Conclusion and open problems}\label{sec:conclusions}
In this paper, we studied the structure of (claw, even hole)-free
graphs. It turned out that our results apply to the larger class of
(pan, even hole)-free graphs. From the structure results, we
obtained fast recognition and coloring algorithms for (pan, even
hole)-free graphs. The complexity of coloring even-hole-free
graphs is unknown.  It follows from Corollary~1 in
\cite{KraKra2001} that coloring odd-hole-free graphs is
NP-Complete. Thus, the following problem, analogous to our result,
is of interest to us.
\begin{problem}\label{pro:odd-hole}
What is the complexity of coloring (pan, odd hole)-free graphs?
\end{problem}
Observation~\ref{obs:tw-chi} shows the tree-width of a (pan, even
hole)-free graphs is bounded by a function in the clique number.
It is conceivable that a more general statement holds.
\begin{problem}\label{pro:tree-width}
Is the tree-width of an even-hole-free graph bounded by a function
of its clique number?
\end{problem}

The {\it clique-width} of a graph , denoted by , is the
minimum number of labels needed to construct  using the
following four operations:
\begin{description}
 \item[(i)] Creation of a new vertex  with label .
 \item[(ii)] Disjoint union of two labeled graphs.
 \item[(iii)] Joining each vertex with
label  to each vertex with label .
 \item[(iv)] Changing label  to .

\end{description}
It is known \cite{CorRot2005} that for any graph ,  and that \cite{CouOla2000}  where  is the complement of .

In \cite{CouMak2000}, it is shown that every problem definable in
a certain kind of Monadic Second Order Logic, called
LinEMSOL() is linear-time
solvable on any graph class with bounded clique-width for which a
-expression can be constructed in linear time. In
\cite{CouMak2000}, it is mentioned that, roughly speaking,
MSOL() is Monadic Second Order Logic with quantification
over subsets of vertices but not of edges; MSOL() is
the restriction of MSOL() with the addition of labels
added to the vertices, and LinEMSOL() is the
restriction of MSOL() which allows search for sets
of vertices which are optimal with respect to some linear
evaluation functions. The problems Vertex Cover, Maximum Weight
Stable Set, Maximum Weight Clique, Steiner Tree and Domination are
examples of LinEMSOL() definable problems. Furthermore,
from the results of \cite{Oum2008} and \cite{KobRot2003}, it
follows that the chromatic number of any class of graphs with
bounded clique-width can be computed in polynomial time.

In \cite{MakRot1999}, it is shown that split graphs have unbounded
clique-width. It follows that even-hole-free graphs have unbounded
clique-width. However, it might be possible that the clique-width
of an even-hole-free graph is bounded by a function of its clique
number. To conclude our paper, we pose this as an open problem.
\begin{problem}\label{pro:clique-width}
Is the clique-width of an even-hole-free graph bounded by a
function of its clique number?
\end{problem}

\bibliographystyle{plain}
\begin{thebibliography}{99}

\bibitem{actv}
P. Aboulker, P. Charbit, N. Trotignon and K. Vu\v{s}kovi\'c.
Vertex elimination orderings for hereditary graph classes.
{\it Discrete Mathematics} 338 (2015), no. 5, pp. 825--834.

\bibitem{AddChu2008}
L. Addario-Berry, M. Chudnovsky, F. Havet, B. Reed, and P.
Seymour. Bisimplicial vertices in even-hole-free graphs. {\it
Journal of Combinatorial Theory Series B} 98 (2008), pp.
1119--1164.

\bibitem{BerChv1984}
C. Berge and V. Chv\'{a}tal (eds.).
Topics on Perfect Graphs.
North-Holland, Amsterdam, 1984.

\bibitem{BraLoz2010}
A. Brandstat, V. Lozin,  and R. Mosca. Independent sets of
maximum weight in pan-free graphs. {\it SIAM Journal on Discrete
Mathematics} 34 (2010),  pp. 239--254.

\bibitem{Buneman1972}
P. Buneman. A characterization of rigid circuit graphs. {\it
Discrete Mathematics} 9 (1974), pp. 205--212.


\bibitem{CamEsc2007}
K. Cameron, E. Eschen, C. T. Ho\`ang, and R. Sritharan.
Recognition of perfect circular-arc graphs.
{\it In: Graph Theory in Paris},
A. Bondy, J. Fonlupt, J.-L. Fouquet, J.-C. Fournier, and J. L. Ramirez Alfonsin (eds.),
pp. 97--108. Birkhauser, Basel, 2007.

\bibitem{ChaLu2013}
H.-C. Chang, and H.I. Lu.
A faster algorithm to recognize even-hole-free graphs.
{\it J. Combin. Theory Ser. B} 113
(2015), pp. 141--161.

\bibitem{ChuKaw2005}
M. Chudnovsky, K. Kawarabayashi, and  P. Seymour. Detecting even
holes. {\it J. Graph Theory} 48 (2005), pp. 85--111.

\bibitem{ConCor2002}
M. Conforti, G. Cornuejols, A. Kapoor, and   K. Vu\v{s}kovi\'c.
Even-hole-free graphs part II: Recognition algorithm. {\it J.
Graph Theory} 40 (2002), pp. 238--266.

\bibitem{CorRot2005}
D. G. Corneil and U. Rotics. On the relationship between
clique-width and treewidth. {\it SIAM J. Comput.} 34 (2005), pp.
825--847.

\bibitem{CouMak2000}
B. Courcelle, J.A. Makowsky, and U. Rotics. Linear time solvable
optimization problems on graphs of bounded clique width. {\it
Theory of Computing Systems} 33 (2000), pp. 125--150.

\bibitem{CouOla2000}
B. Courcelle and S. Olariu. Upper bounds to the clique-width of a
graph. {\it Discrete Applied Math.} 101 (2000), pp. 77--114.

\bibitem{daSV}
M.V.G. da Silva and K. Vu\v{s}kovi\'c.
Triangulated neighborhoods in even-hiole-free graphs.
{\it Discrete Mathematics} 307 (2007), pp. 1065--1073.

\bibitem{Dir1961}
G.~A.~Dirac.  On rigid circuit graphs. {\it Abh. Math. Sem. Univ.
Hamburg} 25 (1961), pp. 71--76.

\bibitem{Des1993}
C. De Simone. On the vertex packing problem. {\it Graphs Combin.}
9 (1993), no. 1, pp. 19--30.

\bibitem{Gavril1974}
F. Gavril. The intersection graphs of subtrees of trees are
exactly the chordal graphs. {\it J. Combin. Theory
Ser. B} 16 (1974), pp. 47--56.

\bibitem{Gavril1972}
F. Gavril. Algorithms for minimum coloring, maximum clique,
minimum covering by cliques, and maximum independent set of a
chordal graph. {\it SIAM Journal on Computing} 1 (1972), pp.
180--187.

\bibitem{Gol1980}
M. C. Golumbic.
Algorithmic Graph Theory and Perfect Graphs,
Academic Press, New York, 1980.



\bibitem{Hol1981}
I. Holyer, The NP-completeness of edge-coloring. {\it SIAM J.
Computing} 10 (1981), pp. 718--720.



\bibitem{KKM2000}
T. Kloks, D. Kratsch, and H. M\"uller.
Finding and counting small induced subgraphs efficiently.
{\it Information Processing Letters}
74:3 (2000), pp. 115--121.

\bibitem{KloMul2009}
T. Kloks, H. M\"uller, and  K. Vu\v{s}kovi\'c.
Even-hole-free graphs that do not contain diamonds: a structure
theorem and its consequences. {\it J. Combin. Theory Ser. B} 99
(2009), no. 5, pp. 733--800.

\bibitem{KobRot2003}
D. Kobler and U. Rotics, Edge dominating set and colorings on
graphs with fixed clique-width.  {\it Discrete Applied
Mathematics} 126 (2003), pp. 197--221.

\bibitem{KraKra2001}
J. Kratochvil, D. Kral, Zs. Tuza, and  G.J. Woeginger, Complexity
of coloring graphs without forbidden induced subgraphs. WG 2001,
{\it Lecture Notes in Computer Science}, Vol. 2204, Springer,
Berlin, 2001, pp. 254--262.



\bibitem{LS2008}
M. C. Lin and J. L. Szwarcfiter. Unit circular-arc representations and feasible circulations.
{\it SIAM J. Discrete Math.} 22 (2008), no.1,  pp. 409--423.

\bibitem{MakRot1999}
J.A. Makowsky and U. Rotics. On the clique-width of graphs with
few . {\it Internat. J. Found. Comput. Sci.} 10 (1999), pp.
329--348.

\bibitem{MarGas1996}
S. E. Markossian, G. S. Gasparian, and B. A. Reed. -perfect
graphs. {\it J. Combin. Theory Ser. B} 67 (1996), no. 1, pp.
1--11.

\bibitem{Ola1989}
S. Olariu. The Strong Perfect Graph Conjecture for pan-free
graphs. {\it J. Combin. Theory B} 47 (1989), pp. 187--191.

\bibitem{OrlBon1991}
J. B. Orlin, M. A. Bonuccelli, and D. P. Bovet. An 
algorithm for coloring proper circular-arc graphs. {\sl SIAM. J.
on Algebraic and Discrete Methods} 2 (1991), pp. 88--93.

\bibitem{Oum2008}
S.-I. Oum. Approximating rank-width and clique-width quickly. {\it
ACM Transactions on Algorithms} 5 (2009), no. 1, Article 10, 20 pp.

\bibitem{Reed}
B. A. Reed. Algorithmic aspects of tree width. {\textit Recent advances in algorithms and combinatorics},
pp. 85--107, CMS Books Math./Ouvrages Math. SMC, 11, Springer, New York, 2003.

\bibitem{ShihHsu}
W. K. Shih and W. L. Hsu. An  algorithm to color proper circular arc graphs.
{\it Discrete Appl. Math.} 24 (1989), no.3, pp. 321--323.

\bibitem{TarYan1984}
R. E. Tarjan and M. Yannakakis. Simple linear-time algorithms to
test chordality of graphs,  test acyclicity of hypergraphs, and
selectively reduce acyclic hypergraphs.
 {\it SIAM J. Computing}
13:3 (1984), pp. 566--579.

\bibitem{Tar1985}
R. E. Tarjan.
Decomposition by clique separators.
{\it Discrete Math}
55 (1985), pp. 221--232.

\bibitem{Vus2010}
K. Vu\v{s}kovi\'c.
Even-hole-free graphs: A survey.
{\sl Appl. Anal. Discrete Math.}
4 (2010), pp. 219--240.

\bibitem{Whi1984}
S. H. Whitesides. A method for solving certain graph recognition
and  optimization problems, with applications to perfect graphs,
in \cite{BerChv1984}.


\bibitem{Walter1978}
J.R. Walter. Representations of chordal graphs as subtrees of a
tree. {\it Journal of Graph Theory} 2 (1978), pp. 265--267.

\end{thebibliography}

\end{document}
