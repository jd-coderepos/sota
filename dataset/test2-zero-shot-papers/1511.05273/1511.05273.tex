\subsection{Compositionality of similar metric transition systems}
\label{sec:interconnectingMTS}

\newcommand \qot {q_{12}}
\newcommand{\qtf}{q_{34}}
\newcommand{\vot}{V_{13}}
\newcommand{\vtf}{V_{24}}
\newcommand{\bVot}{\underbar{V}_{13}}
\newcommand{\bVtf}{\underbar{V}_{24}}
\newcommand{\bV}{\underbar{V}}
\newcommand{\lblSetc}[1]{{\tlabelSet}_{#1}}
\newcommand{\Bttf}{B_\tau^{34}(\slabel_{12})}

Consider OMTS  with label sets  and .
\begin{figure}
	\centering
		\includegraphics[width=6.5cm]{figures/interconnected.pdf}
	\caption{Interconnections of similar MTS}
	\label{fig:interconnected}
\end{figure}
Systems  and  are feedback interconnected to yield , 
with state space , and label set .
Similarly, systems  and  are feedback interconnected to yield ,
with state space , and label set .
See Fig.~\ref{fig:interconnected}.
We seek conditions under which  simulates ;
based on Prop.\ref{prop:sim2closenessMTS}, this would imply that under the same conditions,  for some .
To do so, we use the functional characterization of STAS.
\begin{definition}\cite[Def. 3.2]{JuliusP_ApxSynchronizationMTS06}
\label{def:stas fnt omts}
Given two OMTS  and  with common output set  and label set ,
and non-negative real ,
a function  is a -\textbf{simulation function of}  \textbf{by}  if 
for all ,
\begin{enumerate}
	\item[A0)] 
	\item[A1)] 	
\end{enumerate}
\end{definition}
 A -simulation function defines a -STAS relation via its level sets. 
Namely, as shown in \cite[Thm. 3.4]{JuliusP_ApxSynchronizationMTS06}, the -sublevel set of 

is a -STAS relation of  by  for all . 

To keep the equations readable, in what follows, we define the following:
given ,


( is defined analogously to  in \eqref{eq:SigmaEmbedding}).
The ball  contains all labels in  whose `chronological component'  is no more than -away from .
Note that by definition for any ,  (and analogously ) so the above definition effectively bounds the distance between both chronological components of the label.

Consider the OMTS , with  in a feedback loop with , and  with .
Let  be a -STAS function of  by  (Def.~\ref{def:stas fnt omts}),
and  be a -STAS function of  by .
All systems share the same label set .
We introduce the following functions to keep the equations manageable:
given , define



Consider : if we think of  as trying to match  transitions by minimizing  over the label ball , then  measures how well it does it.
Similarly for .

Because STAS functions certify STAS relations via \eqref{eq:levelset}, the following theorem provides a way to build STAS functions for interconnections of systems, from the STAS functions of the individual connected systems.

\begin{theorem}
\label{thm:sgc}
Consider the OMTS  with common label set  interconnected as described above.
Let  be a -STAS function of  by ,
and  be a -STAS function of  by .
Set .

Define 
to be  where  is continuous and non-decreasing in both arguments.

Recall the definition of lifted label sets   in \eqref{eq:lifting}.
Let  be a non-decreasing function 
s.t.  and 
for all , ,
 satisfies


Also, let  be continuous non-increasing functions s.t. , ,
 and 
for all ,
for all , 
and all 


If the following conditions hold:
\begin{enumerate}[(a)]
\item \label{ass:Vcontinuous}
	 is continuous in the product topology of .
\item \label{ass:hhtilde}
For all ,

\item \label{ass:g dist h}
Function  distributes over , that is

\item \label{ass:sgc} [Small Gain Condition]
For all ,

\end{enumerate}
then 
 is a -STAS function of  by .
\exmend
\end{theorem}

Before proving the theorem, a few words are in order about its hypotheses.
A function  satisfying \eqref{eq:g} always exists: by observing that , we see that  can be taken to be the identity. 
A non-identity function quantifies how restrictive is the interconnection .
It does so by quantifying the difference between the full label set  available to the individual systems operating without interconnection (on the LHS of inequality \eqref{eq:g}), and the restricted label set  available to them as part of the interconnection (on the RHS).

Similarly, functions  satisfying \eqref{eq:gamma} always exist: we can take  to be identically zero.
These choices, however, are unlikely to be useful: we need  to quantify how restrictive is the interconnection .
They do so by quantifying the difference between the full label ball  available to the individual systems operating without interconnection, and the restricted label ball  available to them as part of the interconnection.
See Fig.\ref{fig:labelSets} for an illustration of the label sets.
\begin{figure}
\centering
\includegraphics[scale=0.5]{figures/labelSets}
\caption{Label sets constrained by interconnection.  is the set of label pairs compatible with the interconnection as given in Def.\ref{def:feedbackOMTS}. }
\label{fig:labelSets}
\end{figure}

These two aspects are similar to the conditions, in more classical Lyapunov-based small gain theorems, placing a minimum on the rate of decrease of the Lyapunov functions of the individual systems, and that bound is related to the growth of the other system's Lyapunov function. (For example results on input-to-state stability \cite{JiangMW_LyapunovISS},\cite{SanFelice_IOSS14}, and for bisimulation functions in non-hybrid systems~\cite{Girard_CompositionBisim07}).
Now the more restrictive  is, the bigger  can be.
The more restrictive  is, the smaller  need to be.
The Small Gain Condition (SGC) says that the restrictiveness of  must be balanced by that of :
if  is too restrictive () relative to  (), 
then  can play a label  that can't be matched, and thus we lose similarity of the systems.
Thus similar to the classical results (e.g.,~\cite{Girard_CompositionBisim07}), the SGC balances the gains of the feedback loops.




	\begin{proof} (Thm.~\ref{thm:sgc})
		
		We seek a STAS function  which would certify that  simulates , and we seek the corresponding precision .
		
		For notational convenience, introduce
		
		
		
			By definition,  must satisfy for all ,
			\begin{enumerate}
				\item[A0)] 
				\label{item:V0}
				\item[A1)] 	
				\label{item:V1}
			\end{enumerate}
		
		Condition A0 is the same as \eqref{eq:hhtilde}, and so is true by hypothesis.	
		Now for A1. 
		First we restate it using :
		
		For all ,
		
		where we used property A1 for  and  and the fact that  is non-decreasing to obtain the first inequality, and the non-decreasing nature of  to obtain the second inequality.
		(The second inequality becomes equality if  and  achieve their suprema over  and  respectively.)
		Using \eqref{eq:g}, it comes
		
		
		Applying \eqref{eq:gamma1},\eqref{eq:gamma} to the RHS of this last inequality, 
		
		where we are using  as an abbreviation for 
		
We now establish two inequalities.
		First, note that 
		
		Indeed, let 
		
		be the set over which the infimization is happening.
		We have that  is finite since  is lower bounded by 0. 
		Now since  for all ,
		and  is non-increasing, it follows that  for all . Taking the infimum on the RHS, the inequality \eqref{eq:gamma1decreasing} follows. 
		An inequality analogous to \eqref{eq:gamma1decreasing} holds for  by a similar argument.
		
		Second, note that because  and  are continuous, and  is compact, then the set  is compact as well. 
		Since  is continuous as well, it achieves its infimum over compact sets and therefore 
		
		
		We can proceed as
		
		To obtain the second inequality, we used \eqref{eq:gamma1decreasing} and the fact that  and  are non-decreasing.
		To obtain the equalities, we used \eqref{eq:h achieves inf} and the fact that  is non-decreasing.
		
		By distributivity of  over  and the SGC
		
		thus concluding that  satifies A1, and so is a -STAS function.
	\end{proof}


\textbf{About the other conditions} The distributivity assumption in (\ref{ass:g dist h}) holds, for example, if  is the max operator, i.e. .

Thm.~\ref{thm:sgc} assures us that feedback interconnection respects similarity relation, and therefore also respects conformance relations.

However, the conditions defining  and  (equations \eqref{eq:g} and \eqref{eq:gamma},\eqref{eq:gamma1}) are technical conditions that are are hard to check. 
Turning them into a computational tool for particular classes of systems is the subject of current research.
A simpler, and more conservative, criterion is given in the following theorem:
\begin{theorem}
\label{thm:suff conf for gamma}
If 

then  satisfies \eqref{eq:gamma1}.
Similarly, if 

then  satisfies \eqref{eq:gamma}.
\exmend
\end{theorem}
\begin{proof}
	We give the proof for , that for  is similar.
	Define  
	and .
	Since , .
	Thus for any 
	
\end{proof}
The challenge with the choice of  and  in Thm. \ref{thm:suff conf for gamma} is that  is now required to always `compensate' for the worst-case behavior to satisfy the SGC. 
I.e. we need  for all .
This may lead to a violation of \eqref{eq:g}. 

The next result follows from Thm.\ref{thm:sgc}, the fact that  is increasing, and \cite[Thm. 3.6]{JuliusP_ApxSynchronizationMTS06}.
\begin{theorem}
Let 

and ,
so that  -simulates ,
and  -simulates .
Then  -simulates  with 
.
\end{theorem}



