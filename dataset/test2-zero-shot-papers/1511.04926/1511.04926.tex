

The deadlock detection framework we present in this paper relies on 
abstract descriptions, 
called \emph{contracts}, that are extracted from programs by an inference system.
The syntax of these descriptions, which is defined in Figure~\ref{fig:synCon}, 
uses \emph{record names} , , , , and \emph{future names} , ,
.
\begin{figure*}[t]
  \centering
\mysyntax{\mathmode}{ 
\simpleentry \frr [] \unit \bnfor \X \bnfor
    \mbox{} \bnfor \mbox{} [future record]
\\
\\
\simpleentry \cntc [] \pinull \bnfor \pinull.(\obj,\objA') 
    \bnfor \mbox{}.(\obj,\objA')^\aa
    \bnfor \C.\m \; \frr(\bar{\frr})\rightarrow \frr'
    \bnfor \C!\m \; \frr(\bar{\frr})\rightarrow\frr'
    \bnfor \mbox{}   
    \qquad \qquad~ [contract]
    \oris  \mbox{} 
     \bnfor \cntc\fatsemi \cntc 
     \bnfor  \cntc + \cntc 
\bnfor \cntc \rfloor \cntc [{}]
\\
\\
\simpleentry \frx []  \frr\bnfor f [extended future record]
\\
\\
\simpleentry \frz []
    (\frr, \cntc) \bnfor (\frr,\pinull)^\checkmark
    [future reference values]
}
  \caption{Syntax of future records and contracts.}
\label{fig:synCon}
\end{figure*}
Future records , which
encode the values of expressions in contracts, may be one of the
following: 
\begin{itemize}
\item[--]
a dummy value  that models 
primitive types,
\item[--]
a record name  that 
 represents a place-holder for a value and can be
instantiated by substitutions, 
\item[--]

that defines an object with its cog name  and the values for fields and parameters of the object, 
\item[--]
and  
, which specifies that accessing
 requires control of the cog 
(and that the control is to be released once the method has been evaluated).
The future record  is associated with method
invocations:  is the cog of the object on which the method is invoked.
The name  in  and  
will be called \emph{root} of the future record. \end{itemize}

Contracts  collect the method 
invocations and the dependencies inside statements. 
In addition to , , and  that
respectively represent the empty behaviour, the dependencies due to a
\Abs{get} and an \Abs{await} operation, we have basic contracts that deal
with method invocations.
There are several possibilities:
\begin{itemize}
\item 

(resp.~) 
specifies that 
the method {\m} of class {\C} is going to be invoked \emph{synchronously}
(resp.~\emph{asynchronously})
on an object , with arguments
, and an object  will be returned;
\item
 indicates that 
the current method execution requires the termination of method 
 running on an object of cog  in order to release the object of the cog . 
This is the contract of an asynchronous method invocation followed by a \Abs{get} operation on 
the same future name.
\item
, 
indicating that the current method execution requires
the termination of method  running on an object of cog  in order to 
progress. This is the contract of an asynchronous method invocation followed by
an \Abs{await} operation, and, possibly but not necessarily, by a \Abs{get}
operation.
In fact, a \Abs{get} operation on the same future name does not add any
dependency, since it is guaranteed to succeed because of the preceding
\Abs{await}.
\end{itemize}
The composite contracts  and 
define the abstract behaviour of sequential compositions and
conditionals, respectively.
The contract  require a 
separate discussion because it models parallelism, which is not explicit
in {\coreABS} syntax. We will discuss this issue later on.




\begin{example}
As an example of contracts, let us discuss the terms:
\begin{enumerate}
\item[(a)] \\
 ;
 
 \medskip
 
\item[(b)] 
. 
\end{enumerate}
The contract (a) defines a sequence of two synchronous invocations of method {\m} of class {\C}. We notice that the cog names  and  are free: this indicates
that {\C.\m} returns an object of a new cog.
As we will see below, a {\coreABS} expression with  
this contract is {\tt x.m() ; y.m() ;}.


The contract (b) defines an asynchronous invocation of {\C.\m} followed by a \Abs{get} 
statement and an asynchronous one followed by an \Abs{await}.
The cog  is the one of the caller.
A {\coreABS} expression retaining 
this contract is {\tt u = x!m() ; w = u.get ; v = y!m() ; await v? ;}.
\end{example}

\medskip

\ifIFM
\else

\medskip

The inference of contracts uses two additional syntactic categories:  of future
record values and  of typing values. The former one extends future records with
\emph{future names}, which are used to carry out the \emph{alias analysis}. 
In particular, every
local variable  of methods and every object field and parameter of future type is 
associated to a future name. Assignments between these terms, such as , amounts 
to copying 
future names instead of the corresponding values ( and  become aliases). 
The category  collects the typing values of future names, which are 
either , for \emph{unsynchronised futures}, or ,
for \emph{synchronised ones}
(see the comments below).

\smallskip

The abstract behaviour of methods is defined by \emph{me\-thod contracts} 
,
where 
 is the future record of the receiver of the method,  are the future records of the arguments,  is the abstract behaviour
 of the body, where  is called 
\emph{synchronised contract} and   is called \emph{unsynchronised contract}, 
 and 
 is the future record of the returned object. 

Let us explain why method contracts  use pairs of contracts. In {\coreABS}, 
invocations in method bodies are of two types: (\emph{i})  \emph{synchronised}, that is
the asynchronous invocation has a subsequent \Abs{await} or \Abs{get} operation in the
method body and  (\emph{ii}) \emph{unsynchronised}, the asynchronous invocation 
has no corresponding  \Abs{await} or \Abs{get} in the same method body. (Synchronous
invocations can be regarded as asynchronous invocations followed by a \Abs{get}.)
For example, let
\begin{absexamplen}
x = u!m() ;
await x? ;
y = v!m() ;
\end{absexamplen}
be the main function of a program (the declarations are omitted).
In this statement, the invocation {\tt u!m()} is synchronised before the execution
of {\tt v!m()}, which is unsynchronised. {\coreABS} semantics  
tells us that the body
of {\tt u!m()} is \emph{performed before} the body of {\tt v!m()}. However, while
this ordering holds for the synchronised part of {\tt m}, it may not hold
for the unsynchronised part. In particular, the unsynchronised part of {\tt u!m()}
may run \emph{in parallel} with the body of {\tt v!m()}. For this reason, in this case,
our inference system returns the pair

where ,  and  being the 
cog of the caller, of {\tt u} and {\tt v}, respectively. Letting 
 and
, one
has (see Sections~\ref{sec.contractanalysis} and~\ref{sec.mutanalysis})

that adds the dependency  to the synchronised contract of {\tt u!m()}
and makes the parallel (the  operator) of  the unsynchronised contract of 
{\tt u!m()} and the contract of {\tt v!m()}.
Of course, in alternative to separating \emph{synchronised} and \emph{unsynchronised contracts},
one might collect all the dependencies in a unique contract.
This will imply that the dependencies in different configurations will be
gathered in the same set, thus significantly reducing the precision of the analyses
in Sections~\ref{sec.contractanalysis} and~\ref{sec.mutanalysis}.

\iffalse
For example, let \begin{update} consider the following method\vspace{1em}
\begin{absexamplen}
Unit met() {
  ...
  x = u!m() ;
  await x? ;
  y = v!m() ;
}
\end{absexamplen}
In the method body, the invocation of \Abs{u!m()} is synchronised (because of the \Abs{await} statement) while  \Abs{v!m()} is unsynchronised.
Because it is synchronised, we know that the body of \Abs{u!m()} is \emph{performed before} the end of the method \Abs{met}.
And because it is {\em not} synchronised, the body of \Abs{v!m()} may be performed {\em even after} the method \Abs{met} completed.
And this consideration is recursive: \Abs{u!m()} may also call unsynchronised methods, whose body may be executed in parallel with \Abs{v!m()}
 and after the end of the method \Abs{met}, in parallel with any method that will be executed in the future.
Hence, the kind of interleaving a synchronised method call has is very different from the one an unsynchronised method call has
 and such difference should be represented in our contracts:
 indeed, in a contract pair  typing a method \Abs{m},  is the contract of the synchronised part of the method \Abs{m}, while  is the contract of its unsynchronised part.
In fact, for our previous example, our inference system returns the pair\end{update}


where ,  and  being the 
cog of the caller, of {\tt u} and {\tt v}, respectively. Letting 
 and
, one
has (see Sections~\ref{sec.contractanalysis} and~\ref{sec.mutanalysis})

that adds the dependency  to the synchronised contract of {\tt u!m()}
and makes the parallel (the  operator) of  the unsynchronised contract of 
{\tt u!m()} and the contract of {\tt v!m()}.
Of course, in alternative to separating \emph{synchronised} and \emph{unsynchronised contracts},
one might collect all the dependencies in a unique contract.
This will imply that the dependencies in different configurations will be
gathered in the same set, thus significantly reducing the precision of the analyses
in Sections~\ref{sec.contractanalysis} and~\ref{sec.mutanalysis}.
\fi

The above discussion also highlights the need of contracts 
. In particular, this operator models \emph{ parallel
behaviours}, which is not a first class operator in {\coreABS}, while it is 
central in the semantics (the objects in the configurations). 
We illustrate the point with a statement similar to the above one, where we have
swapped the second and third instruction
\begin{absexamplen}
x = u!m() ;
y = v!m() ;
await x? ;
\end{absexamplen}
According to {\coreABS} semantics, it is possible that the bodies of
{\tt u!m()} and of {\tt v!m()} run in parallel by interleaving their executions. 
In fact, in this case, our inference system returns the pair of contracts (we keep
the same notations as before) 

which turns out to be equivalent to 

(see Sections~\ref{sec.contractanalysis}
and~\ref{sec.mutanalysis}).



The subterm  of the method contract 
is called 
\emph{header};
 is called \emph{returned future record}. We assume that cog and record
names in the header occur linearly. Cog and record names in the header \emph{bind} the 
cog and record names in  and in .
The header and the returned
future record, written , are called \emph{contract signature}. In a method contract , cog and record names occurring in  or  or 
 may be \emph{not bound} by header. These \emph{free names} correspond to 
\Abs{new cog} instructions and will be replaced by fresh cog names during the analysis.



\fi

\medskip

\begin{figure*}[t]
expressions and addresses
\myrules{\mathmode\compactmode}{
\entry[{\scriptsize (T-Var)}]{\Gamma(x) = \frx}{\inferpe{\Gamma}{\obj}{x}{\frx}} \and
\entry[{\scriptsize (T-Fut)}]{\Gamma(f) = \frz}{\inferpe{\Gamma}{\obj}{f}{\frz}} 
\and
\entry[{\scriptsize (T-Field)}]{x\not\in\dom(\Gamma)\\\Gamma(\ethis.x)=\frr}{\inferpe{\Gamma}{\obj}{\x}{\frr}} \and
\entry[{\scriptsize (T-Value)}]{\inferpe{\Gamma}{\obj}{e}{f}\\\inferpe{\Gamma}{\obj}{f}{(\frr,\cntc)^{[\checkmark]}}}
  {\inferpe{\Gamma}{\obj}{e}{\frr}} 
\and
\entry[{\scriptsize (T-Val)}]{e \quad  \textit{primitive value or arithmetic-and-bool-exp}}{\inferpe{\Gamma}{\obj}{e}{\unit}}
\and
\entry[{\scriptsize (T-Pure)}]{\inferpe{\Gamma}{\obj}{e}{\frr}}{\inferee{\Gamma}{\obj}{e}{\frr}{\pinull}{\etrue}{\Gamma}}
}	
expressions with side effects
\myrules{\mathmode\compactmode}{
\entry[{\scriptsize (T-Get)}]{\inferpe{\Gamma}{\obj}{x}{f}\\
\inferpe{\Gamma}{\obj}{f}{(\frr, \cntc)} \\\X,\objb \fresh
\\
\Gamma'=\Gamma[f \mapsto (\frr, \pinull)^\checkmark]}
  {\inferee{\Gamma}{\obj}{x.\nget}{\X}{\cntc\seqpoint(\obj,\objb) \rfloor\contract{\Gamma'}}{\frr=\fRec{\objb}{\X}}{\Gamma'}} 
 \and
\entry[{\scriptsize (T-Get-tick)}]{\inferpe{\Gamma}{\obj}{x}{f}\\
\inferpe{\Gamma}{\obj}{f}{(\frr, \cntc)^\checkmark}\\\X,\objb \fresh}
  {\inferee{\Gamma}{\obj}{x.\nget}{\X}{\pinull}{\frr=\fRec{\objb}{\X}}{\Gamma}} \and
\entry[{\scriptsize (T-NewCog)}]{\inferpe{\Gamma}{\obj}{\bar{e}}{\bar{\frr}}\\\\
    \fields{\C} = \bar{T\ x} \\ \parameters{\C}= \bar{T'\ x'} \\  \bar{\X}, \obj' \fresh}
  {\inferee{\Gamma}{\obj}{\nnew\ \ncog\ \C(\bar{e})}{[\cog {:} \obj', \bar{x{:}\X}, \bar{x'{:}\frr}]}{\pinull}{\etrue}{\Gamma}} \and
\entry[{\scriptsize (T-New)}]{\inferpe{\Gamma}{\obj}{\bar{e}}{\bar{\frr}}\\\\
    \fields{\C} = \bar{T\ x} \\ \parameters{\C}= \bar{T'\ x'}\\ \bar{\X} \fresh}
  {\inferee{\Gamma}{\obj}{\nnew\ \C(\bar{e})}{[\cog {:} \obj, \bar{x{:}\X}, \bar{x'{:}\frr}]}{\pinull}{\etrue}{\Gamma}} \and
\entry[{\scriptsize (T-AInvk)}]{\inferpe{\Gamma}{\obj}{e}{\frr}\\\inferpe{\Gamma}{\obj}{\bar{e}}{\bar{\frs}}\\\\
    \fclass{\types{e}}=\C \\\fields{\C}\cup \parameters{\C} = \bar{T\ x} \\ \X, \overline{\X}, \objb, f \fresh}
  {\inferee{\Gamma}{\obj}{e!\m(\bar{e})}{f}{\pinull}{[\cog {:} \objb , \bar{x{:} \X}] = \frr\land\C.\m \preceq\frr(\bar{\frs}) \rightarrow X}
    {\Gamma[f \mapsto (\fRec{\objb}{\X},\,\C!\m~\frr(\bar{\frs}) \rightarrow \X)]}}
\and
\entry[{\scriptsize (T-SInvk)}]{\inferpe{\Gamma}{\obj}{e}{\frr}\\\inferpe{\Gamma}{\obj}{\bar{e}}{\bar{\frs}}\\\\
     \fclass{\types{e}}=\C\\ \fields{\C} \cup\parameters{\C} = \bar{T\ x}\\  \X,\bar{\X} \fresh}
  {\inferee{\Gamma}{\obj}{e.\m(\bar{e})}{\X}{\C.\m~\frr(\bar{\frs}) \rightarrow X \rfloor\contract{\Gamma}}
    {[\cog {:} \objb , \bar{x {:} X}] = \frr \land\C.\m \preceq\frr (\bar{\frs}) \rightarrow X}{\Gamma}}
}
\caption{Contract inference for expressions and expressions with side effects.}
\label{fig:inf:exp}
\end{figure*}

\begin{figure*}[t]
{statements
\myrules{\mathmode\compactmode}{
\simpleentry[T-Skip]{\inferst{\Gamma}{\obj}{\eskip}{\pinull}{\etrue}{\Gamma}} \and
\entry[{\scriptsize (T-Field-Record)}]{x\not\in\dom(\Gamma)\\\Gamma(\ethis.x) = \frr\\\\\inferee{\Gamma}{\obj}{z}{\frr'}{\cntc}{\mathcal U}{\Gamma'}}
  {\inferst{\Gamma}{\obj}{x=z}{\cntc}{\mathcal U\land\frr=\frr'}{\Gamma'}}
\and
\entry[{\scriptsize (T-Var-Record)}]{\Gamma(x) = \frr\\\\\inferee{\Gamma}{\obj}{z}{\frr'}{\cntc}{\mathcal U}{\Gamma'}}
  {\inferst{\Gamma}{\obj}{x=z}{\cntc}{\mathcal U}{\Gamma'[x\mapsto\frr']}}
\and
\entry[{\scriptsize (T-Var-Future)}]{\Gamma(x)= f\\\inferee{\Gamma}{\obj}{z}{f'}{\cntc}{\mathcal U}{\Gamma'}}
  {\inferst{\Gamma}{\obj}{x=z}{\cntc}{\mathcal U}{\Gamma'[x \mapsto f']}} 
\and
\entry[{\scriptsize (T-Var-FutRecord)}]{\Gamma(x)= f\\\inferee{\Gamma}{\obj}{z}{\frr}{\cntc}{\mathcal U}{\Gamma'}}
  {\inferst{\Gamma}{\obj}{x=z}{\cntc}{\mathcal U}{\Gamma'[f \mapsto (\frr,\pinull)]}} 
  \and
\entry[{\scriptsize (T-Await)}]{\inferpe{\Gamma}{\obj}{e}{f} \\
\inferpe{\Gamma}{\obj}{f}{(\frr, \cntc)}
\\\X,\objb\fresh\\\Gamma'=\Gamma [f \mapsto (\frr,\pinull)^\checkmark]}
  {\inferst{\Gamma}{\obj}{\nwait\ e?}{\cntc \seqpoint (\obj,\objb)^{\aa} \rfloor\contract{\Gamma'}}{\frr=\fRec{\objb}{\X}}{\Gamma'}}
\and
\entry[{\scriptsize (T-Await-Tick)}]{\inferpe{\Gamma}{\obj}{e}{f}\\
\inferpe{\Gamma}{\obj}{f}{(\frr, \cntc)^\checkmark}
\\\X,\objb\fresh}
  {\inferst{\Gamma}{\obj}{\nwait\ e?}{\pinull}{\frr=\fRec{\objb}{X}}{\Gamma }}
\and
\entry[{\scriptsize (T-If)}]{\inferpe{\Gamma}{\obj}{e}{\Bool}\\
    \inferst{\Gamma}{\obj}{s_1}{\cntc_1}{\mathcal U_1}{\Gamma_1}\\\inferst{\Gamma}{\obj}{s_2}{\cntc_2}{\mathcal U_2}{\Gamma_2}\\
        \mathcal{U}=\Bigl(\bigland_{x\in\dom(\Gamma)}\hspace*{-1em}\Gamma_1(x)=\Gamma_2(x) \Bigr) \land
                    \Bigl(\bigland_{x\in{\tt Fut}(\Gamma)}\hspace*{-1em}\Gamma_1(\Gamma_1(x)) = \Gamma_2(\Gamma_2(x)) \Bigr)\\
        \Gamma' = \Gamma_1 + \Gamma_2 |_{ \{ f \; | \; f \notin \Gamma_2({\tt Fut}(\Gamma) )\}} 
        }
  {\inferst{\Gamma}{\obj}{\eif{e}{s_1}{s_2}}{\cntc_1+\cntc_2}{\mathcal U_1\land \mathcal U_2 \land \mathcal U}{\Gamma'}} \and
\entry[{\scriptsize (T-Seq)}]{\inferst{\Gamma}{\obj}{s_1}{\cntc_1}{\mathcal U_1}{\Gamma_1}\\\inferst{\Gamma_1}{\obj}{s_2}{\cntc_2}{\mathcal U_2}{\Gamma_2}}
  {\inferst{\Gamma}{\obj}{s_1;s_2}{\cntc_1\fatsemi\cntc_2}{\mathcal U_1\land \mathcal U_2}{\Gamma_2}}\and
\entry[{\scriptsize (T-Return)}]{\inferpe{\Gamma}{\obj}{e}{\frr}\\\Gamma(\destiny)=\frr'}
  {\inferst{\Gamma}{\obj}{\nreturn\ e}{\pinull}{\frr=\frr'}{\Gamma}}
}}
\caption{Contract inference for statements.}
\label{fig:inf:stmt}
\end{figure*}


\subsection{\em Inference of contracts}

Contracts are extracted from {\coreABS} programs by means of an inference algorithm.
Figures~\ref{fig:inf:exp}
and~\ref{fig.meth-class} illustrate the set of  rules. 
The following auxiliary operators are used:
\begin{itemize}
\item  and  respectively
return the sequence of fields and parameters and their types of a class .
Sometime we write {\small} to separate fields with a non-future type by those with 
future types. Similarly for parameters;

\item  returns the type of an expression , which is either an interface (when  is an object) or a data type;
\item  returns the unique (see the restriction \emph{Interfaces} in 
Section~\ref{sec.restrictions}) class implementing ; and 
\item  returns the sequence of method names in the sequence 
 of method declarations.
\end{itemize}

The inference algorithm  uses constraints , which 
are defined by the following syntax
{\mysyntax{\mathmode}{
\simpleentry \mathcal U [] \etrue 
\bnfor c =c' \bnfor \frr=\frr'
\bnfor \frr(\bar{\frr})\rightarrow\frs \preceq \frr'(\bar{\frr}')\rightarrow\frs' 
[ ]
\oris
 \mathcal U \land \mathcal U
[ ]}}
where  is the constraint that is always true;
  is a classic unification constraint between terms;
  is a {\em semi-unification} constraint; the constraint   is the conjunction of 
 and .
We use {\em semi-unification} constraints~\cite{Henglein:1993} to deal with method invocations: basically, in ,
 the left hand side of the constraint corresponds to the method's formal parameter,  being the record of ,  being the records of the parameters and  being the record of the returned value,
 while the right hand side corresponds to the actual parameters of the call, and the actual returned value.
The meaning of this constraint is that the actual parameters and returned value must match the specification given by the formal parameters, like in a standard unification:
 the necessity of semi-unification appears when we call several times the same method.
Indeed, there, unification would require that the actual parameters of the different calls must all have the same records, while with semi-unification all method calls are managed independently.

The judgments of the inference algorithm have a typing context  
mapping variables to extended future records, future names to future name
values and methods to 
their signatures. 
They have the following form:
\begin{itemize}
\item[--] 
for pure expressions  and  for
 future names , where 
 is the cog name of the object executing the expression and  and 
 are their inferred values.
\item[--]
 
for expressions with side effects , where , and  are as for pure expressions ,  is the contract for  created by the inference rules,
 is the generated constraint, and  is the environment  \emph{with updates} of variables and future names. We use the same 
judgment for pure expressions; in this case ,  and .
\item[--]
for statements :
 where
,  and  are as before, and
 is the environment obtained after the execution of the statement.
The environment may change because of 
variable updates.
\end{itemize}

Since  is a function, we use the standard predicates  or 
. Moreover, given a function , we define 
to be the following function
We also let  be the function

Moreover, provided that ,
the environment
  be defined as follows 


Finally, we write  whenever  and we let 

where .


\medskip

The inference rules for expressions and future names are reported in 
Figure~\ref{fig:inf:exp}. They are straightforward, except for \rulename{T-Value} that performs the dereference of variables and return 
the future record stored in the future name of the variable. \rulename{T-Pure}
lifts the judgment of a pure expression to a judgment similar to those for
expressions with side-effects. This expedient allows us to simplify rules for
statements.

\smallskip

Figure~\ref{fig:inf:exp} also reports inference rules for expressions with 
side effects. Rule \rulename{T-Get} 
deals with the \Abs{get} synchronisation primitive and returns the contract
 , where  is 
 stored in the 
future name of  and  represents a dependency between the cog of the  object executing 
the expression and the root of the expression. The constraint  
is used to extract the root  of . The contract  may have two shapes: either 
(i)  or (ii) .
The subterm  lets us collect 
all the contracts in  
\emph{that are 
stored in future names that are not check-marked}. In fact, these contracts
correspond to previous asynchronous invocations without any corresponding 
synchronisation 
(\Abs{get} or \Abs{await} operation) in the body. The evaluations of these invocations
may interleave with the evaluation of the expression . For this reason,
the intended meaning of  is that the
dependencies generated by the invocations must be collected \emph{together} with
those generated by .
We also observe that the rule updates the 
environment by check-marking the value of the future name of  and by replacing the
contract with  (because the synchronisation has been already performed). 
This allows subsequent
\Abs{get} (and \Abs{await}) operations on the same future name not to modify the contract
(in fact, in this case they are operationally equivalent to the \Abs{skip} statement)
-- see \rulename{T-Get-Tick}. 

Rule \rulename{T-NewCog} returns a record with a new cog name. This is 
in contrast with \rulename{T-New}, where the cog of the returned record is the
\emph{same} of the  object executing 
the expression~\footnote{\label{footnote.new}It is worth to recall that, in {\coreABS}, 
the creation of an object, either with a 
\Abs{new} or with a \Abs{new cog}, amounts to executing the method \name{init} of 
the corresponding 
class, whenever defined (the \Abs{new} performs a synchronous invocation, 
the \Abs{new cog} performs an asynchronous one). 
In turn, the termination of
 \name{init} triggers the execution of the method
\name{run}, if 
present. The method \name{run} is asynchronously invoked when \name{init}
is absent.
Since \name{init} may be regarded as a method in {\coreABS}, the inference
system in our tool explicitly introduces a synchronous invocation to \name{init}
in case of \Abs{new} and an asynchronous one in case of \Abs{new cog}. However,
for simplicity, we overlook this (simple) issue in the rules \rulename{T-New} and 
\rulename{T-NewCog},
acting as if \name{init} and \name{run} are always absent.}.

Rule \rulename{T-AInvk} derives contracts for asynchronous invocations. Since the
dependencies created by these invocations influence the 
dependencies of the synchronised contract only if a subsequent 
\Abs{get} or \Abs{await} operation is performed, the rule stores the invocation into
a fresh future name of the environment and returns the contract . 
This models {\coreABS} semantics that lets asynchronous invocations be synchronised 
by explicitly getting or awaiting on the corresponding future variable,
see rules \rulename{T-Get} and \rulename{T-Await}. The future name storing the invocation
is returned by the judgment. 
On the contrary, in rule \rulename{T-SInvk}, which deals with synchronous invocations,
the judgement returns a contract that is the invocation (because the corresponding 
dependencies must be added to the current ones) in parallel with the unsynchronised
asynchronous invocations stored in . 

\smallskip

The inference rules for statements are collected in Figure~\ref{fig:inf:stmt}.
The first three rules define the inference of contracts for assignment. There are
two types of assignments: those updating fields and parameters of the \Abs{this}
object and the other ones. For every type, we need to address the cases of updates with values
that are expressions (with side effects) (rules \rulename{T-Field-Record} and 
\rulename{T-Var-Record}),
or future names (rule \rulename{T-Var-Future}). 
Rules for fields and parameters updates enforce that their future records are unchanging, as
discussed in Section~\ref{sec.restrictions}. 
Rule \rulename{T-Var-Future}, 
 define the management 
of aliases: future variables are always updated with future names and never with 
future names' values.

Rule \rulename{T-Await} and \rulename{T-AwaitTick} deal with the \Abs{await}
synchronisation when applied to a simple future lookup {\tt ?}. They are
similar to the rules  \rulename{T-Get} and \rulename{T-Get-Tick}. 


Rule \rulename{T-If} defines contracts for conditionals. In this case we collect the
contracts  and  of the two bran\-ches, with the intended meaning that 
the dependencies defined by  and  are always kept separated.
As regards the environments, the rule constraints the two environments  and
 produced by typing of the two branches to \emph{be the same} on variables in
 \emph{and} on the values of future names bound to variables
in . However, the two branches may have different unsynchronised
invocations that are not bound to any variable. The environment

allows us to collect all them.

Rule \rulename{T-Seq} defines the sequential composition of contracts. 
Rule \rulename{Return} constrains the record of {\tt destiny}, which is an identifier
introduced by \rulename{T-Method}, shown in Figure~\ref{fig.meth-class}, for 
storing the return record.





\smallskip

The rules for method and class declarations are defined in Figure~\ref{fig.meth-class}.
Rule \rulename{T-Method}  derives the method contract of 
 by typing  in an environment 
extended with \name{this}, \Abs{destiny} (that will be set by \Abs{return} statements,
see \rulename{T-Return}),
the arguments , and the local variables . In order to deal with alias 
analysis of future variables, we separate fields, parameters, arguments and local
variables with future types from the other ones. In particular, we associate 
future names to the former ones and bind future names to record variables. As discussed
above, the abstract behaviour of the method body is a pair of contracts, which is
 for \rulename{T-Method}. 
This
term  collects all the contracts in  
\emph{that are 
stored in future names that are not check-marked}. In fact, these contracts
correspond to asynchronous invocations without any synchronisation 
(\Abs{get} or \Abs{await} operation) in the body. These invocations will be 
evaluated \emph{after} the termination of the body -- they are the \emph{unsynchronised  contract}.

The rule \rulename{T-Class} yields an \emph{abstract class table} that associates a 
method contract with every method name. It is this abstract class table that is 
used by our analysers in Sections~\ref{sec.contractanalysis} and~\ref{sec.mutanalysis}. 
The rule \rulename{T-Program} derives the contract of a {\coreABS} program by 
typing the main function in the same way as it was a body of a method.

\medskip

\begin{figure*}[t]
\myrules{\mathmode\compactmode}{
\entry[{\scriptsize (T-Method)}]{
  \fields{\C}\cup\parameters{\C} = \bar{T_f\ x}\ \bar{\name{Fut}\arglang{T_f'}\ x'}\\
  \obj, \bar{X}, \bar{X'}, \bar{Y}, \bar{Y'}, \bar{W}, \bar{W'}, \bar{f}, \bar{f'}, \bar{f''}, Z \ {\it fresh}\\\\
  \Gamma'=\bar{y {:} Y} + \bar{y' {:} f'} + \bar{w {:} W} + \bar{w' {:} f''}
    + \bar{f {:} (\bar{X'} , \pinull)} + \bar{f' {:} (\bar{Y'} , \pinull)} + \bar{f'' {:} (\bar{W'} , \pinull)}\\
  \inferst{\Gamma + \this: [\cog {:}\obj , \; \bar{x{:}X}, \bar{x{:}f}] + \Gamma' + \destiny : Z}{\obj}{s}{\cntc}{\mathcal U}{\Gamma''}\\
  \text{ are not future types}}
{\C,\Gamma \vdash {\T\;\m\;(\bar{T\ y}, \bar{\name{Fut} \arglang{T'}\ y'})\{\bar{T_l\ w}; \, \bar{\name{Fut} \arglang{T_l'}\ w'}; \; s\} } \;:\quad 
  \inferrule{}{[\cog : \obj , \bar{x{:}X}, \, \bar{x'{:}X'}](\bar{Y}, \bar{Y'})\{ \pairl{\cntc}{ \contract{\Gamma''}} \} \;  Z\\\\
       \rhd\; {\mathcal U} \land[\cog : \obj , \bar{x{:}X}, \, \bar{x'{:}X'}](\bar{Y}, \bar{Y'}) \rightarrow Z = \C.\m }
}\and
\entry[{\scriptsize (T-Class)}]{\inferpp{\C,\Gamma}{\bar{M}}{\bar{\mcntc}}{\bar{\mathcal{U}}}}
  {\inferpp{\Gamma}{\class\; \C (\bar{T\ x}) \; \{\bar{T'\ x'};\quad\bar{M}\}}{\bar{\C.{\it mname}(M)\mapsto\mcntc}}{\bar{\mathcal{U}}}}\and
\entry[{\scriptsize (T-Program)}]{\inferpp{\Gamma}{\bar C}{\bar S}{\bar{\mathcal U}}\\\bar{X}, \bar{X'}, \bar{f} \ {\it fresh}\\
    \inferst{\Gamma  + \bar{x{:}X} + \bar{x'{:}f} + \bar{f{:}(X', \pinull)}}{\rm start}{s}{\cntc}{\mathcal U}{\Gamma'}\\
    \text{ are not future types}}
  {\inferpp{\Gamma}{\bar I\ \bar C\ \{\bar{T\ x};\;\bar{\name{Fut}\arglang{T'}\ x'} ;\; s\}}{\bar S,\pairl{\cntc}{\contract{\Gamma'}}}{\mathcal{U} \land \bar{\mathcal U}}}
}
 \caption{Contract rules of method and class declarations and programs.}\label{fig.meth-class}
\end{figure*}


The contract class tables of the classes in a program derived by the rule 
\rulename{Class}, will be
noted . We will address the contract of {\tt m} of class {\tt C} by .
 In the following, we assume that every {\coreABS} program is a triple
, where  is the class table, 
 is the main function, and  is its contract class 
table. By rule \rulename{Program}, analysing (the deadlock freedom of) a program, 
amounts to verifying the contract of the main function with a record for 
which associates a special cog name called  to the  field 
( is intended to be the cog name of the object ).

\begin{example}
The methods of \Abs{Math} in Figure~\ref{fig.Math}
  have the following contracts, once the constraints are solved (we
  always simplify  into ):
  \begin{itemize}
  \item \name{fact\_g} has contract 
  {\footnotesize}
The name  in the header refers to the cog name associated
    with \this{} in the code, and binds the occurrences of  in
    the body.  The contract body has a recursive invocation to
    \name{fact\_g}, which is performed on an object in the same cog
     and followed by a \Abs{get} operation.  This operation
    introduces a dependency  . We observe that, if we
    replace the statement \Abs{Fut<Int> x = this\!fact\_g(n-1)} in
    \name{fact\_g} with \Abs{Math z = new Math() ;} \Abs{Fut<Int> x =
      z\!fact\_g(n-1)}, we obtain the same contract as above because
    the new object is in the same cog as \Abs{this}.

\medskip

  \item \name{fact\_ag} has contract
    {\footnotesize} In this case, the presence of an
    \Abs{await} statement in the method body produces a dependency
   . The subsequent \Abs{get} operation does
    not introduce any dependency  because the future name has a 
    check-marked value in the environment. In fact, in
    this case, the success of \Abs{get} is guaranteed, provided the
    success of the \Abs{await} synchronisation.
    
\medskip

  \item \name{fact\_nc} has contract
    {\footnotesize} This method contract differs from the
    previous ones in that the receiver of the recursive invocation is
    a free name (i.e., it is not bound by  in the header).  This
    because the recursive invocation is performed on an object of a
    new cog (which is therefore different from ). As a
    consequence, the dependency  added by the \Abs{get} relates
    the cog  of \Abs{this} with the new cog .
  \end{itemize}
\end{example}


\begin{example}Figure~\ref{fig.cpxsched} displays the contracts of the methods of class 
{\tt CpxSched} in Figure~\ref{fig:exampleC}.
\begin{figure*}
  \begin{itemize}
  \item method \Abs{m1} has contract
    {\footnotesize}
  \item method \Abs{m2} has contract
    {\footnotesize}
  \item method \Abs{m3} has contract
        {\footnotesize}
  \end{itemize}
\caption{\label{fig.cpxsched} Contracts of {\tt CpxSched}.}
\end{figure*}

According to the contract of the main function, the two invocations of \Abs{m2} are
second arguments of  operators. This will give rise, in the analysis 
of contracts, to the union of the corresponding cog relations.  
\end{example}

We notice that the inference system of contracts discussed in this section is
modular because, when programs are organised in different modules, it partially supports the separate contract inference of modules with a 
well-founded ordering relation (for example, if there are two modules, classes in the 
second module use definitions or methods in the first one, but not conversely).
In this case, if a module {\tt B} includes a module
{\tt A} then a patch to a class of {\tt B} amounts to inferring contracts for {\tt B} only. 
On the contrary, a patch to a class of {\tt A} may also require a new contract inference
of {\tt B}.

\subsection{Correctness results}

In our system, the ill-typed programs are those manifesting a failure of
the semiunification process, which does not address misbehaviours. In particular,
a program may be well-typed and still manifest a deadlock. In fact, in systems
with \emph{behavioural types}, one usually demonstrates that
\begin{enumerate}
\item
in a well-typed program, 
every configuration  has a behavioural type, let us call it ;
\item
if   then there is a relationship between
 and ;
\item
the relationship in 2 preserves a given property (in our case, deadlock-freedom).
\end{enumerate}

Item 1, in the context of the inference system of this section, means that the 
program has a contract class table. Its proof needs a contract system for configurations, which we have defined in Appendix~\ref{sec:subjred}. The theorem
corresponding to this item is Theorem~\ref{th:wt_init}.

Item 2 requires the definition of a relation between contracts, called \emph{later stage
relation} in Appendix~\ref{sec:subjred}.
This later stage relation is a syntactic relationship between contracts
whose basic law is that a method invocation is larger than the 
instantiation of its method contract (the other laws, except  and , are congruence laws). 


The statement
that relates the later stage relationship to {\coreABS} reduction is Theorem~\ref{th_subjred1}. It is worth to observe that all the theoretical development up-to this point are useless if the 
later stage relation conveyed no relevant property. 
This is the purpose of item 3, which requires the definition of \emph{contract models} and the proof that deadlock-freedom is preserved by the models of contracts in 
later stage relation. The reader can find the proofs of these statements in the
Appendices~\ref{sec:analysis-proof} and~\ref{sec:mutanalysis-proof} (they correspond
to the two analysis techniques that we study).





