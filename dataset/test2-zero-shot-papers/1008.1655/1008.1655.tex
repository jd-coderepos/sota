\documentclass[copyright]{eptcs}


\usepackage[utf8x]{inputenc}
\usepackage{latexsym,amssymb,amsmath,amsthm}


\newcounter{theorem}
\newtheorem{proposition}[theorem]{Proposition}
\newtheorem{result}[theorem]{Result}
\theoremstyle{definition}
\newtheorem{example}[theorem]{Example}

\title{Descriptional Complexity of the Languages :
Automata, Monoids and Varieties\thanks{Both authors were supported by the Ministry of Education of the
Czech Republic under the project MSM 0021622409 and by
the Grant 201/09/1313 of the Grant Agency of the Czech Republic.}}
     
\author
{Ond\v rej Kl\'{i}ma \qquad\qquad \qquad Libor Pol\'ak
\institute{Department of Mathematics and Statistics\\ Masaryk University\\
Brno, Czech Republic}
\email{\quad klima@math.muni.cz\quad\qquad polak@math.muni.cz}
}
\def\titlerunning{Descriptional Complexity of the Languages }
\def\authorrunning{O. Kl\'{i}ma  \&  L. Pol\'ak}
\begin{document}
\maketitle

\begin{abstract}
The first step when forming the polynomial hierarchies of languages
is to consider
languages of the form  where  and  are over a finite alphabet 
and from a given variety 
of languages,   being a letter. All such 's generate the  variety
of languages .

We estimate the numerical parameters of the language 
in terms of their values for  and . These parameters include 
the state complexity
of the minimal complete DFA and  the size of the syntactic monoids.
We also estimate 
the cardinality of the image of 
in the Sch\"utzenberger product of the syntactic monoids
of  and . In these three cases we obtain the optimal bounds.


Finally, we also consider estimates for 
the cardinalities of free monoids
in the variety of monoids corresponding to 
in terms of sizes of the free monoids in the variety
of monoids corresponding to~.
\end{abstract}

\section{Introduction}

The polynomial operator assigns to each variety of languages 
the class of all Boolean combinations
of the languages of the form

where  is a finite alphabet,
 (i.e. they are
over ).
Such operators on classes of languages lead to several concatenation hierarchies.
Well known cases are the
Straubing-Th\'erien and the group hierarchies. Concatenation hierarchies
has been intensively studied by many authors -- see Section 8 of the
Pin's Chapter \cite{pi-kap}.
In the restricted case we fix a natural number  and we allow only
 in  -- see \cite{kp-cai} and papers quoted there.
The resulting variety of languages is denoted by
. Using the Eilenberg correspondence, 
 operates also on pseudovarieties of monoids.
 We consider in this paper only the case .

State complexity problems are a fundamental part of automata theory. 
Recent papers of a survey nature with numerous references are \cite{b}
by Brzozowski and \cite{y} by Yu.
First we estimate the state complexity of DFA automata for the language
 in terms of the state complexities of  and . This is the content
of Section~2.

Secondly, for languages  and , we also estimate 
the cardinality of the image of  under the natural homomorphism 
into the Sch\"utzenberger product of the syntactic monoids  and 
of the languages
 and . This monoid  recognizes the language , too.
The syntactic monoid of 
is a homomorphic image of the monoid .
The third question concerns its cardinality.

In all three problems we get estimates which can be reached by concrete
examples (for the first one in Section 2 and for the two remaining ones 
in Section~3).
In general: the size of the Sch\"utzenberger product equals at least
to the size of the monoid  which is at least the size
of the syntactic monoid of .
In Section~3 we further consider natural examples showing that those
three numbers could differ drastically. The first example
is the language .
The next proposition roughly estimates  for -trivial
monoids using their structure. 


 
In the last section we consider
a variety of languages  such that the corresponding
pseudovariety of monoids consists of all finite members of a locally finite
variety of monoids . Then the free monoid 
in 
over a finite set  is the smallest one recognizing
all languages in .
We  embed the free monoid in the variety of monoids corresponding to
the class
 over  
into the product of
 copies of the Sch\"utzenberger product of 
which leads to a rough
estimate for the cardinality of this free monoid.

\section{Recognizing by Automata}

Let  be a finite alphabet and let  be a regular language.
The following construction of the minimal complete DFA is due to Brzozowski.
We put:
 --
the set of all {\it left derivatives} of 
(here ).
One assigns to  its ``canonical'' {\it minimal automaton}

using left derivatives; namely:
\begin{itemize}
\item  is the (finite) set of states,
\item  acts on  by
,
\item  is the initial state and  is a final state
(i.e., element of  ) if and only if .
\end{itemize}

\begin{proposition}
Let  and  be languages over a finite alphabet  whose minimal
complete DFA have  resp.  states and let .
Then the minimal complete
DFA for the language  has at most  states.
\end{proposition}

\begin{proof}\label{prop-aut}
Notice that an arbitrary left derivative of  is of the form

We have  possible values for  and 
, ,
has at most  values.
The statement follows.
\end{proof}

The example in the next proposition is a slight modification of the construction 
in Theorem 2.1
in \cite{yzs}. It was suggested to the authors by J. Brzozowski. It shows
that the bound from Proposition~1 is tight. 

\begin{proposition}
For arbitrary natural numbers  there exist languages  resp. 
whose
minimal complete DFA have  resp.  states such that each complete
DFA recognizing the language  has at least  states.
\end{proposition}

\begin{proof}
 Let  and
let  
where

Note that  accepts the language

Similarly, we define 

where

Clearly, for , we have

Both automata  and  are minimal.
\smallskip

We define, for all , the set

and the numbers

and

Let  be such that . 
Let  (the case  can be treated similarly).
Then  but . Then in each
complete DFA recognizing  with the initial state  we have
that


Now let  be such that  and
. Then, for ,
we have
 and .
Again, in each
complete DFA recognizing  with the initial state  we have
that 
For an arbitrary subset  of ,
where , and  there exists a word

such that  and .

Therefore each
complete DFA recognizing  has at least  states.
\end{proof}

\section{Recognizing by Monoids} 

Let  and  be languages over a finite alphabet  and let
 and  be their syntactic monoids.
In this section we will compare 
\begin{itemize}
\item 
the size of the Sch\"utzenberger product 
of monoids  and ,
\item
the cardinality of the image of  in the homomorphism 
from   
into   recognizing the language ,
\item
the size of the syntactic monoid of the language .
\end{itemize}

Let  and  be finite monoids. Their {\it Sch\"utzenberger product}
 is the set of all 22 matrices 
 where
 and

equipped with the multiplication

It is well known that this operation is associative.
This product was introduced by Sch\"utzenberger and by Straubing for
an arbitrary finite family of monoids. Basic results are also due to
Reutenauer and Pin -- see \cite{pi-kniha} Theorems 1.4 and 1.5 in Chapter 5.
Clearly, if  and , then .
\medskip

Recall that the {\it syntactic congruence} of the language 
is a relation  on  defined by:
 
The {\it syntactic monoid} of  is the quotient monoid
. It is the smallest monoid recognizing the language .

Let  be a finite alphabet and let  be homomorphisms. Let  and let , i.e.
the language  is {\it recognized} by  using  and
, and similarly for the language . One can take the mappings
 and  surjective.

For , we define a mapping  by
 

It is easy to see that it is a homomorphism and that the language
 is recognized by  using
 and 
\smallskip

Of course, the language  is also recognized by
 which can be much smaller than the whole  .
Moreover the syntactic monoid of the language  is a homomorphic
image of the monoid . Its size can be much smaller than 
the cardinality of the monoid .
\smallskip

First we present, for arbitrary  and , an example where 
the mapping  is onto. Thus the bound  for 
is sharp.

\begin{proposition}\label{prop-image-sharp} 
For arbitrary  and , there exist languages
  and  with syntactic monoids
 and  and homomorphisms
, , 
such that the mapping 
is surjective.
\end{proposition}


\begin{proof} Let ,
let  and  be natural numbers and let

The syntactic monoids of  and  are the additive groups
 and   and the syntactic homomorphisms are
given by

Let  and
 be arbitrary.
We will find  such that


Let



where

We put




where  if , for .
Finally, putting

we see that this word has all desired properties.
\end{proof}

We used GAP to calculate the sizes of syntactic monoids from the last
proof for  and . The numbers were
61, 379 and 2041. They are of the form . This led us
to the following two results.

\begin{proposition}\label{prop-groups+}
Let  and  be languages over a finite alphabet  
with syntactic monoids  and ,
let  and let .
Then the size of the syntactic monoid of 
is at most .
\end{proposition}
 
\begin{proof}
(i) Suppose first that both  and  are groups.
Let  be such that .
Then also, for each , it is the case that  
.
Therefore, each pair  with

is in the syntactic congruence of the language .
\smallskip

(ii) Suppose that the monoid  is not a group (the case  not being
a group could be treated in a similar way).
Let  be without an inverse element.  Then there is no  
with . Indeed, such  would imply that , 
is one-to-one
and due to the finiteness of  we have that . 
Thus there would
be  such that  and  -- a~contradiction.

Let  and let . Thus there exist
 such that  and .
Consequently
.
There are  matrices in  not having the element
 in the set at position , and 
 matrices in  having the element
 in the set at position  and not having  at position .

Consequently, the size of the syntactic monoid of  is less
or equal the cardinality of  which is at most
. The gap between  and the last number
is at least the needed value .
\end{proof}

Next we show that the estimate from Proposition~\ref{prop-groups+} is exact.

\begin{proposition}\label{prop-synt-monoids-sharp}
For arbitrary  and , there exist languages  and  with
syntactic monoids  and , , such that
the size of the syntactic monoid of 
is exactly .
\end{proposition}
 
\begin{proof}
We again consider the languages  and  from the proof of 
Proposition~\ref{prop-image-sharp}.

(i) Let .

Let 
Then  and .

\medskip
(ii) Let .
Let .
(The case   could be treated
analogously).

Let .\\
Then  and .
\end{proof}

The following example shows that the cardinalities of ,
the cardinality of  and the size of the syntactic monoid 
can be three quite different numbers.
 
\begin{example}
Let , let  and consider the language .
Syntactic monoids of both  and  are isomorphic to the
two element monoid  having a neutral element 1 and
a zero element 0. Moreover, for ,
 if and only if , and
 if and only if .
Finally 

Clearly, the cardinality of  is 
.

Let  and . 
One can calculate that .
Finally, it is well known and easy to see that
the syntactic monoid of  is isomorphic to the 8-element
monoid of Boolean uppertriangular matrices of order 2.
\end{example}

\medskip
We will try to estimate the number  using the structures
of monoids  and . The first little step concerns very
special monoids and certain chains of their elements. 

Green's relations are a basic tool in semigroup theory:
define on an arbitrary monoid  the quasiorders ,
 and  as follows:

A monoid  is {\it -trivial} if 
 implies that .
For each , we define  (the {\it content} of )
as the set of all letters
of .


\begin{proposition}\label{prop-chains}
Let  and  be finite -trivial monoids having cardinalities
 and .
Let the number of elements in
a longest strict -chain in  is  and the
number of elements in
a longest strict -chain in  is . Let ,
 be homomorphisms.
Then the number of elements of each set of ,
is less or equal to  (which is ).
In particular,

\end{proposition}

\begin{proof}
Let  where . Then


and the statement follows.
\end{proof}

The following example shows that the bound for  from
Proposition~\ref{prop-chains} is sharp.

\begin{example}
For  we write 

Notice first that, for ,
the syntactic monoid of  is isomorphic to
the monoid  with a zero 0 adjoined.  
The syntactic homomorphism  maps  onto 
and , otherwise, and we have .
All the relations  coincide with the reverse inclusion
.
Similarly for .

Consider first the language  for
.
Then  and  and


\medskip
We can modify this example for arbitrary  as follows:

Then



\end{example}

\section{Level of Varieties}

Let  be a variety of languages. A well known fact is that
the pseudovariety of monoids corresponding to the class

is generated by all Sch\"utzenberger products

where  are syntactic monoids of  languages from   
-- see~(\cite{pi-kniha}, Theorems 5.1.4. and 5.1.5.).
Of course being interested in , one takes
.

Here we are looking for a single finite monoid recognizing all languages
in ,  fixed.
We can succeed under certain circumstances as follows.
Let  be a locally finite variety of monoids, i.e. the 
finitely generated monoids in  are finite. Let  be the 
corresponding
fully invariant congruence on , , i.e.
the set of all identities which hold in  .
Notice that  is the free monoid in  over the set
. The
finite members 
of  form a
(the so-called equational) pseudovariety of finite monoids.
We denote the corresponding variety of languages by 
, i.e. 
if and only if the syntactic monoid of  is a member of 
. Then the free monoid in  over the set  is the smallest monoid recognizing
all languages from . Thus we consider somehow the descriptional
complexity for the whole varieties of languages.

One of the
main results of \cite{kp-cai} was an effective description of the fully
invariant congruence  
 for the variety
.
Here we treat only the case of .


\begin{result}[(\cite{kp-cai}, Theorem 1)]\label{res-kp}
For , we have


\end{result}

\begin{proposition} Let  and
let 
be given by

Then  is isomorphic to , i.e. to the
free monoid in  over the alphabet . 

In particular, if the cardinality of  is , then
the size of  is bounded by the number .

\end{proposition}

\begin{proof}
The first part follows immediately from Result~\ref{res-kp}.
To get the estimate,
 realize that all the diagonal entries in the matrices
, for a given ,
are the same.
\end{proof}

Let us consider the simplest non-trivial example.
It shows, among others, that the estimate from the last proposition can be  
far from being optimal.

\begin{example}
Let  -- the class of all semilattices.
Then  if and only if .
The free semilattice (in the signature of monoids)
over a set  is isomorphic to
. In particular, this variety is locally finite.
For the corresponding variety of languages 
and a finite alphabet , the set  consists of unions
of 's, .

Let .
We are going to improve
the bound  from the last proposition. 
Clearly, the cardinality of  is .
We will calculate the image of  first.

\def\c{\mathsf c}
\def\0{\emptyset}
\def\a{\{a\}} \def\b{\{b\}}
\def\ch{\mathsf{char}}
\def\h{\mathsf h} \def\t{\mathsf t}
We write also, for ,
 and
.
Let  where . 
The {\it characteristic sequence}  of  is

with removed repetitions. We get  when considering
it as a set. Note that
 for each .
We divide the elements of  into several classes:

(i) For  we have  (the sequence of length 0).

(ii) For , we have .

(iii) For , we have 
and , ,
and  if .

All remaining words have .

(iv) If , then
 is one of the following sequences
.
 
All remaining words have .

(v) If  not being a subword of , i.e.
. Then
either  for  or
 for .

(vi) The case   is left-right dual to (v).

(vii) If ,\  being a subword of , then
 is a subsequence of

containing the first and the last item. The following words
witness that all 4 possibilities can happen:
.

(viii)  The case left-right dual to (vii).

(ix) If , then  is a subsequence
of

containing the first and the last item. The following words
witness that all 8 possibilities can happen:
.

(x) If , then 
or   .
Appropriate words are  and .
\smallskip

Altogether we have 30 elements in . 
In fact our consideration until now could be presented in Section 3.
Returning to the free monoid in the variety corresponding to the class
 over , we can state at present only that it
has at most  elements. 
When considering the mapping
, not all possible  combinations can happen and we can
further decrease the estimate for . Using more advanced techniques
we can even get 100 as an upper bound.
\end{example}
\medskip

\noindent
{\bf Acknowledgement.} The authors would like to express their gratitude
to Janusz Brzozowski who suggested them to use the construction from \cite{yzs}
in the proof of Proposition~2.


\begin{thebibliography}{00}\label{biography}

\bibitem{b}
J. Brzozowski, Quotient complexity of regular languages,
in {\it Proc. 11th International Workshop
on Descriptional Complexity of Formal Systems (DCFS 2009)},
arXiv:0907.4547v1
       
\bibitem{kp-cai}
O. Klíma and  L. Polák,
Polynomial operators on classes of regular languages,
in {\it Proc.
International Conference on Algebraic Informatics 2009,
Thessaloniki},
Springer LNCS 5725, pp. 260--277

\bibitem{pi-kniha}
J.-E. Pin,
{\it Varieties of Formal Languages},
North Oxford Academic, Plenum, 1986

\bibitem{pi-kap}
J.-E. Pin,
Syntactic semigroups, Chapter 10 in {\it Handbook of Formal Languages},
G. Rozenberg and A. Salomaa eds, Springer, 1997

\bibitem{y}
S. Yu,  State complexity of regular languages. 
{\it J. Autom., Lang. and Comb.} {\bf 6} (2001), pp. 221--234.

  
\bibitem{yzs}
S. Yu, Q. Zhuang and K. Salomaa, The state complexities of some basic
operations on regular languages, 
{\it Theoretical Computer Science} {\bf 125} (1994), pp.
315­--328

\end{thebibliography}

\end{document}
