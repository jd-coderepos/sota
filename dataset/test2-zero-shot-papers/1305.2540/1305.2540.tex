\documentclass{article}

\usepackage[english]{babel}
\usepackage{amsfonts}
\usepackage{amsmath}
\usepackage{amsthm}
\usepackage{amssymb}
\usepackage{algpseudocode}


\theoremstyle{plain}
\newtheorem{theorem}{Theorem}[section]
\newtheorem{lemma}{Lemma}[section]
\newtheorem{corollary}{Corollary}[section]

\theoremstyle{definition}
\newtheorem{example}{Example}[section]


\begin{document}

\title{Finding Distinct Subpalindromes Online}

\author{
  Dmitry Kosolobov \\ dkosolobov@mail.ru \and 
  Mikhail Rubinchik \\ mikhail.rubinchik@gmail.com \and 
  Arseny M. Shur \\ arseny.shur@usu.ru}

\maketitle
\begin{center}
{Institute of Mathematics and Computer Science, Ural Federal University,\\
Ekaterinburg, Russia}
\end{center}

\begin{abstract}
We exhibit an online algorithm finding all distinct palindromes inside a given string in time  over an ordered alphabet and in time  over an unordered alphabet. Using a reduction from a dictionary-like data structure, we prove the optimality of this algorithm in the comparison-based computation model.

\textbf{Keywords: }{stringology, counting palindromes, subpalindromes, palindromic closure, online algorithm}

\end{abstract}


\algtext*{EndWhile}\algtext*{EndIf}\algtext*{EndFunction}\algtext*{EndProcedure}\algtext*{EndFor}

\section{Introduction}

A \emph{palindrome} is a string that is equal to its reversal. Palindromes are among the most interesting text regularities. During the last few decades, many algorithmic problems concerning palindromes were considered. In this paper we solve one problem that remained open.

There is a well known online algorithm by Manacher \cite{Man} that finds all maximal subpalindromes of a string in linear time and linear space (by a ``subpalindrome'' we mean a substring that is a palindrome). It is known \cite{DJP} that every string of length  contains at most  distinct subpalindromes, including the empty string. The following question arises naturally: \emph{can one find all distinct subpalindromes of a string in linear time and space?} In~\cite{GPR}, this question was answered in the affirmative, but with an offline algorithm. The authors stated the existence of the corresponding online algorithm as an open problem. Our main contribution is the following result.

\begin{theorem} \label{main}
Let  be a finite unordered (resp., ordered) alphabet.
There exists an online algorithm which finds all distinct subpalindromes in a string over  in  (resp., ) time and linear space. This algorithm is optimal in the comparison based computation model.
\end{theorem}

As a by-product, we get an online linear time and space algorithm that finds, for all prefixes of a string, the lengths of their maximal suffix-palindromes and of their palindromic closures.


\section{Notation and Definitions}

An alphabet  is a finite set of letters. A \emph{string}  over  is a finite sequence of letters. It is convenient to consider a string as a function . A \emph{period} of  is any period of this function. The number  is the \emph{length} of , denoted by . We write  for the -th letter of  and abbreviate  by .  A \emph{substring} of  is any string  such that  for some  and . Each occurrence of the substring  in  is determined by its \emph{position} . If  (resp. ), then  is a \emph{prefix} (resp. \emph{suffix}) of . A prefix (resp. suffix) of a string  is called \emph{proper} if it is not equal to . The string  is the \emph{reversal} of , denoted by . A string is a \emph{palindrome} if it coincides with its reversal. A palindrome of even (resp. odd) length is referred to as an \emph{even} (resp. \emph{odd}) palindrome. If a substring, a prefix or a suffix of a string is a palindrome,  we call it a \emph{subpalindrome}, a \emph{prefix-palindrome}, or a \emph{suffix-palindrome}, respectively. The \emph{palindromic closure} of a string  is the shortest palindrome  such that  is a prefix of .

Let  be a subpalindrome of . The number  is the \emph{center} of , and the number  is the \emph{radius} of . Thus, a single letter and the empty string are palindromes of radius 0. Note that the center of the empty subpalindrome is the previous position of the string.

By an \emph{online algorithm} for an algorithmic problem concerning strings we mean an algorithm that processes the input string  sequentially from left to right, and answers the problem for each prefix  of  after processing the letter .


\section{Distinct subpalindromes}
\subsection{Suffix-Palindromes and Palindromic Closure} \label{ssec:closure}

The problem of finding the lengths of palindromic closures for all prefixes of a string is closely related to the problem of finding all distinct subpalindromes of this string. It was conjectured in \cite{GPR} that there exists an online linear time algorithm for the former problem.

Let  be the maximal suffix-palindrome of . It is easy to see that the palindromic closure of  equals to the string . An offline algorithm for finding the maximal suffix-palindromes for each prefix of the string can be found, e. g., in \cite[Ch. 8]{CrRy}. Our online algorithm is a modification of Manacher's algorithm (see \cite{Man}).

We construct a data structure based on Manacher's algorithm. Let  be a boolean flag (needed to distinguish between odd and even palindromes). This data structure  contains a string  and supports the procedure  adding a letter to the end of . The function  returns the length of maximal odd/even (according to ) suffix-palindrome of .

Our data structure uses the following internal variables:\\
, which is the length of ;\\
, which is the center of the maximal odd/even (according to ) suffix-palindrome of ;\\
, which is an array of integers such that for any  the value  is equal to the radius of the maximal odd/even (according to ) subpalindrome with the center . The main property of  is expressed in the following lemma (see \cite[Lemma 8.1]{CrRy}).

\begin{lemma} \label{ManacherLemma}
Let   be an integer, .\\
(1) If  then ;\\
(2) if  then .
\end{lemma}

At the beginning,  is filled with zeros, , , "\ (the while loop will be skipped and the correct values  for the next iteration will be obtained).}\end{footnote}.

\begin{algorithmic}[1]
	\Procedure{man.AddLetter}{}
		\State \Comment{position of the max suf-pal of } \label{lst:line:calcs}
        \State 
		\While{}\label{lst:line:forbeg}
			\State \Comment{this is  in }\label{lst:line:min}
			\If{}\label{lst:line:breadth}
				\State \Comment{extending the max suf-pal}
				\State      \Comment{max suf-pal of  found}
			\EndIf
            \State  \Comment{next candidate for the center of max suf-pal}
		\EndWhile\label{lst:line:forend}
		\State 
	\EndProcedure
	\Function{man.MaxPal}{}
		\State \Return \label{lst:line:maxpal}
	\EndFunction
\end{algorithmic}

\begin{theorem} \label{PalindromeClosure}
There exists an online linear time and space algorithm that finds the lengths of the maximal suffix-palindromes of all prefixes of a string.
\end{theorem}

\begin{proof}
From the correctness of Manacher's algorithm (see \cite{Man}) and Lemma~\ref{ManacherLemma} it follows that the function  correctly returns the length of the maximal odd/even suffix palindrome of the processed string. For a string of length , we call the procedure \  times with the parameter  and  times with . If one call of the procedure uses  iterations of the loop in the lines \ref{lst:line:forbeg}--\ref{lst:line:forend}, then the value of  increases by . Hence, the loop is used at most  times in total. Apart from this loop,  performs a constant number of operations. This gives us the required  time bound.
\end{proof}

\begin{corollary}
There exists an online linear time and space algorithm that finds the lengths of palindromic closured of all prefixes of a string.
\end{corollary}

\begin{example}
Let  and consider the state of the data structure  after the sequence of calls , .

The calls to  after each call to  return consequently the values  for the case  and  for the case .
\end{example}

\subsection{Distinct subpalindromes} \label{ssec:distinct}

We make use of the following

\begin{lemma}[\cite{GPR}]
Each subpalindrome of a string is the maximal suffix-palindrome of some prefix of this string.
\end{lemma}

This lemma implies that the online algorithm designed in Sect.~\ref{ssec:closure} finds all subpalindromes of a string. To find all distinct subpalindromes, we have to verify whether the maximal suffix-palindrome of a string has another occurrence in this string. Note that the direct comparison of substrings for this purpose leads to at least quadratic overall time. Instead, we will use a version of suffix tree known as \emph{Ukkonen's tree}. To introduce it, we need some definitions.

A \emph{trie} is a rooted labelled tree in which every edge is labelled with a letter such that all edges leading from a vertex to its children have different labels. Each vertex of the trie is associated with the string labelling the path from the root to this vertex. A trie can be ``compressed'' as follows: any non-branching descending path is replaced by a single edge labelled by the string equal to the label of this path. The result of this procedure is called a \emph{compressed trie}. For a set  of strings, the \emph{compressed trie of}  is defined by the following two properties: (i) for each string of , there is a vertex associated it and (ii) the trie has the minimal number of vertices among all compressed tries with property (i).

A (compressed) \emph{suffix tree} is the compressed trie of the set of all suffixes of a string. \emph{Ukkonen's tree} is the data structure  containing a string and the suffix tree of this string (labels are stored as pairs of positions in the string). Ukkonen's tree allows one to add a letter to the end of the string (procedure ), updating the suffix tree. We also need the following parameter: the length of the minimal suffix of the processed string such that this suffix occurs in this string only once (function ). Let us recall some implementation details of Ukkonen's tree for the efficient implementation of .

The update of Ukkonen's tree is based on the system of suffix links. Such a link connects a vertex associated with a word  to the vertex associated with the longest proper suffix of . These links are also defined for ``implicit'' vertices (the vertices that are not in the compressed trie, but present in the corresponding trie). In particular, Ukkonen's tree supports the triple  such that

\tabcolsep=3pt
\begin{tabular}{rl}
(1)& is a vertex (associated with some string ) of the current suffix tree,\\
(2)& is an edge (labelled by some string ) between  and its child,\\
(3)& is an integer between  and ,
\end{tabular}

\noindent with the property that  is the longest suffix of the processed string that occurs in this string at least twice. This triple is crucial for  fast update of Ukkonen's tree (for further details, see \cite{Ukk}).

\begin{lemma}[\cite{Ukk}] \label{lemukk}
The procedure  performs  calls using  space and  (resp., ) time in the case of ordered (resp., unordered) alphabet.
\end{lemma}

We modify Ukkonen's tree, associating with each vertex  an additional field  to store the length of the string associated with . Maintaining this field requires a constant number of operations at the moment when  is created. Thus, this update adds  time and  space to the total cost of maintaining Ukkonen's tree. Thus, Lemma~\ref{lemukk} holds for the modified Ukkonen's tree as well. It remains to note that .

\begin{proof}[Theorem~\ref{main}: existence]
The following algorithm solves the problem and has the required complexity. The algorithm uses data structures  and , processing the same input string . The next (say, th) symbol of  is added to both structures through the procedures  and . After this, we call  to get the length of the maximal palindromic suffix of  and  to get the length of the shortest suffix of  that never occurred in  before. The inequality  means the detection of a new palindrome; we get its first and last positions from the structure  and output them. In the case of the opposite inequality, there is no new palindrome, and we output ``---''.

The required time and space bounds follow from Theorem~\ref{PalindromeClosure} and Lemma~\ref{lemukk}.
\end{proof}

\begin{example}
Consider the string  again. We get the following results for :

\end{example}

\subsection{Lower bounds}

Recall that a \emph{dictionary} is a data structure  containing some set of elements and designed for the fast implementation of basic operations like checking the membership of an element in the set, deleting an existing element, or adding a new element. Below we consider an \emph{insert-only dictionary over a set }. In each moment, such a dictionary  contains a subset of  and supports only the operation . This operation checks whether the element  is already in the dictionary; if no, it adds  to the dictionary.

\begin{lemma} \label{DictionaryLowerBound}
Suppose that the alphabet  consists of indivisible elements, , and the insert-only dictionary  over  is initially empty. Then the sequence of  calls of  requires, in the worst case,  time if  is ordered and  if  is unordered.
\end{lemma}

\begin{proof}
Let  be an ordered alphabet. Assume that on some stage all letters with even numbers are in the dictionary, while all elements with odd numbers are not. Consider the next operation. In the comparison-based computation model, a query ``?'' is answered by some decision tree; each node of this tree is marked by the condition ``'' for some . To distinguish between  and , the tree should contain the nodes for both  and . Now note that for any , exactly one of the letters  and  belongs to . So, to answer correctly all possible queries ``?'' the decision tree should have nodes for all letters. Then the depth of this tree is . Therefore, for some element  the number of comparisons needed to prove that  is . After processing , the content of the dictionary remains unchanged. The decision tree can change, but it does not matter: we again choose the next letter to be the one having an even number and requiring  comparisons to prove its membership in . Thus, our ``bad'' sequence of calls is as follows: it starts with , and continues with the ``worst'' letter, described above, on each next step. Even if the first  calls can be performed in  time each, the overall time is , as required.

In the case of unordered alphabet all conditions in the decision tree have the form ``''. It is clear that if the dictionary contains  elements, the maximal number of comparisons equals  as well. Choosing the bad sequence of calls in the same way as for the ordered alphabet, we arrive at the required bound .
\end{proof}

Before finishing the proof of Theorem~\ref{main} we mention the following lemma. Its proof is obvious.

\begin{lemma} \label{abx}
Suppose that  are two different letters and  is a string such that each  is a letter different from  and . Then all nonempty subpalindromes of  are single letters.
\end{lemma}

\begin{proof}[Proof of Theorem~\ref{main}: lower bounds]
We prove the required lower bounds reducing the problem of maintaining an insert-only dictionary to counting distinct palindromes in a string. Assume that we have a black box algorithm that processes an input string letter by letter and outputs, after each step, the number of distinct palindromes in the string read so far. The time complexity of this algorithm depends on the length  of the string at least linearly, and a linear in  algorithm does exist, as we have proved in the Sect.~\ref{ssec:distinct}. Thus, we can assume that the considered black box algorithm works in time , where  is the size of the alphabet of the processed string and the function  is non-decreasing.

The insert-only dictionary over a set  of size  can be maintained as follows. We pick up two letters  and mark their presence in the dictionary using two boolean variables,  and . All other letters are processed with the aid of the mentioned black box. Let us describe how to process a sequence of  calls  starting from the empty dictionary.

For each call, we first compare the current letter  to  and . If , then  is the answer to the query ``?''; after answering the query we set . The case  is managed in the same way.

If , we feed the black box with , , and  (in this order). Then we get the output of the black box and check whether the number of distinct subpalindromes in its input string increased. By Lemma~\ref{abx}, the increase happens if and only if  appears in the input string of the black box for the first time. Thus, we can immediately answer the query ``?'', and, moreover,  is now in the dictionary.

The described algorithm performs the sequence of calls  in time  plus the time used by the blackbox to process a string of length  over . Hence, the overall time bound is . In view of Lemma~\ref{DictionaryLowerBound} we obtain  (resp., ) in the case of ordered (resp., unordered) alphabet . The required lower bounds are proved.
\end{proof}

\bibliographystyle{amsplain}          \bibliography{unique_palindromes_eng2}            
\end{document}
