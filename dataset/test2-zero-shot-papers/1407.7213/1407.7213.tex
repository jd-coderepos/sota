

\documentclass[preprint,12pt]{article}


\usepackage{graphicx}

\usepackage{amssymb}
\usepackage{amsthm}
\usepackage{amsmath}


\begin{document}







\title{An extension of the Georgiou-Smith example: Boundedness and attractivity  in the presence of unmodelled dynamics via nonlinear PI control}



\author{Haris E. Psillakis\\
National Technical University of Athens (NTUA)\\
H. Polytechniou 9,
15780 Zografou, Athens, Greece\\
\texttt{hpsilakis@central.ntua.gr}}
\date{}
\maketitle
\begin{abstract}
In this paper, a nonlinear extension of the Georgiou-Smith system is considered and robustness results are proved for a class of nonlinear PI controllers with respect to fast parasitic first-order dynamics. More specifically, for a perturbed nonlinear system with sector bounded nonlinearity and unknown control direction, sufficient conditions for global boundedness and attractivity have been derived. It is shown that  the closed loop system is globally bounded and attractive if (i) the unmodelled dynamics are sufficiently fast and (ii) the PI  control gain has the Nussbaum function property. For the case of nominally unstable systems, the Nussbaum  property of the control gain appears to be crucial. A simulation study confirms the theoretical results.
\end{abstract}

\section{Introduction}
\label{intro}
\newtheorem{theorem}{Theorem}
\newtheorem{lemma}{Lemma}
\newtheorem{remark}{Remark}
\newtheorem{definition}{Definition}
\newtheorem{assumption}{Assumption}
\newtheorem{corollary}{Corollary}
The unknown control direction problem has attracted significant research interest over the last three decades. Nussbaum gains \cite{Nussbaum_paper}, \cite{Ye_Jiang98} originally introduced in \cite{Nussbaum_paper} have become the main theoretical tool for controller design for systems with unknown control directions.
Nussbaum functions (NFs) are continuous functions  with the property

Examples of NFs are ,  among many others.

For the simple integrator case  with  a nonzero constant of \emph{unknown sign}, standard analysis \cite{Nussbaum_paper} shows that the Nussbaum control law

ensures convergence of the output  to the origin and boundedness of the Nussbaum parameter . However, Georgiou and Smith demonstrated in \cite{GS} that the proposed controller is nonrobust to fast parasitic unmodelled dynamics. Particularly, they considered the  system

and showed divergence for  when the controller \eqref{nussbaum controller} is used.

An alternative nonlinear PI methodology was proposed by Ortega, Astolfi and Barabanov  in \cite{Ortega_paper} to address the unknown control direction problem. For the simple integrator case, their controller takes the form

with  the PI square error defined by . The main difference between the two controllers \eqref{nussbaum controller}, \eqref{nonlinear PI}  is the existence of the proportional term in the control gain of \eqref{nonlinear PI} (see also p. 166 of \cite{AKO_book}). It was hinted  in \cite{Ortega_paper},\cite{AKO_book}(no complete proof was given) that such a controller is robust to fast parasitic first-order perturbations and therefore can stabilize the Georgiou-Smith example system if . Their argument, however, was based on the fact that the related transfer function is  positive real and cannot be carried over to the case of an unstable unforced linear system or even a nonlinear system. In fact, the introduction of a simple destabilizing pole in the system 
() may result in instability of the closed-loop system with the controller \eqref{nonlinear PI} even if  (see Section \ref{simulation}).

It remains therefore an open problem to design a nonlinear PI controller   robust to fast parasitic dynamics when the plant to be controlled is originally unstable and nonlinear. To this end, we consider an extension of the Georgiou-Smith  system. Particularly, we examine the overall dynamic behavior  of  the  nonlinear system  with first-order unmodelled dynamics given by

when a nonlinear PI control law  designed for the unperturbed system

is applied.
\subsection{Nonlinear PI for the unperturbed system}
For system \eqref{unperturbed}, we assume that 
is a sector-bounded nonlinearity, i.e. ,  and there exist some constants  such that  .
A controllability assumption is also imposed, that is .
Let now a nonlinear PI controller of the form

with PI gain  where  is a class  function and .
For the  derivative we have for the unperturbed system \eqref{unperturbed} that

Note that  whenever  with  we have

Thus,  whenever  for every 

( denotes the largest integer not exceeding ) which in turn implies that  is bounded by  where . The fact that  implies  and  from \eqref{z} and \eqref{PI_o} respectively. Also, from \eqref{unperturbed} we have . Barbalat's lemma can now be invoked to prove that . This is a standard analysis in the spirit of  \cite{AKO_book}.

Assume now the existence of parasitic first order unmodelled dynamics in the form of \eqref{G-SE}.  Sufficient conditions are given in the next section for \emph{global boundedness and attractivity} for the closed-loop system comprised from \eqref{G-SE} and the nonlinear PI controller  \eqref{PI_o} and \eqref{z}. A key property is that the nonlinear PI gain function  should be a function of Nussbaum type. 

\section{Extended Georgiou-Smith system with sector nonlinearity}
\label{}
In this section we consider system \eqref{G-SE} with a  sector-bounded nonlinearity

for some constants . Note that  can also take positive values rendering the unforced system unstable. To simplify notation let us define the constant .
We have established the following theorem.
\begin{theorem}\label{main_theorem}
Let the closed-loop system described by \eqref{G-SE}, \eqref{PI_o}, \eqref{z} with sector-bounded nonlinearity given by \eqref{sector}, \eqref{sector bounds}. If
\begin{description}
  \item[(i)] , 
  \item[(ii)] 
  \item[(iii)]  has the  Nussbaum property \eqref{nussbaum propertyp},\eqref{nussbaum propertym}
\end{description}
then, all closed-loop signals are bounded and 
\end{theorem}
\begin{proof}
From the definition of the PI error  in \eqref{z} and \eqref{G-SE} we have that

Let now the function

with  to be defined. Replacing from \eqref{G-SE}, \eqref{PI_o}, \eqref{z}, \eqref{z_dynamics} and canceling terms we have for its time derivative that

Eq. \eqref{dotS} can be written in matrix notation as

where   denotes a symmetric w.r.t. the main diagonal element of . We claim that there is some constant  such that  is positive definite for all . Equivalently, we can prove that, for some , the two principal minors of  given by

are positive . From assumption (i)  of Theorem \ref{main_theorem}, it is obvious that  . For  we have that

with , , .  is therefore a quadratic polynomial with respect to  that is positive definite  if
\begin{description}
  \item[(a)] 
  \item[(b)] there exists some constant  for all  where ,  are the two roots of  given by

\end{description}
If we carry out the calculations we have that

and therefore the positivity  condition for  is satisfied if  and . For condition (b) to be true, as  varies in , there must be some  such that  for all . This holds true if . Function  is obviously decreasing with respect to  with minimum value . Function  on the other hand is decreasing up to some point  and then increasing up to  with value . Thus, the second condition holds true if  which is exactly assumption (ii) of the theorem. Thus, selecting   for any   we have .
Integrating now  we have that  or equivalently

The above inequality and the Nussbaum property of  ensure the boundedness of . To prove this, let us assume the contrary. From the Nussbaum property (iii) of  there exists a strictly increasing sequence  such that  and

Due to continuity of , the PI square error  will eventually pass from an infinite number of elements of . From \eqref{Sbound}, we have for the times  at which 

If we divide all terms in \eqref{Sbound_tk} with  and take into account  the limiting property \eqref{nussbaum_zk} the left hand side (l.h.s.) of \eqref{Sbound_tk} should take negative values for all  for some . This yields the desired contradiction since the l.h.s. of \eqref{Sbound_tk} is a sum of squares which is always nonnegative. Thus,  and therefore ,  and from \eqref{Sbound} . Then, the system equations \eqref{G-SE} yield . Invoking now Barbalat lemma we obtain the desired property .
\end{proof}
\begin{remark}
In the case of a linear system  condition (ii) of Theorem \ref{main_theorem} is no longer needed and  (i)  reduces to  for  (nominally stable system) and  for  (nominally unstable). Note that in the latter case the necessary condition for stabilization by simple output feedback (with known sign of ) is .
 \end{remark}
 \begin{remark}
 If  then the constant  in the definition of  can be nonnegative. This means that in \eqref{Sbound}   and the Nussbaum condition (iii) for  in Theorem \ref{main_theorem} can be relaxed to

After calculations one can show that condition  holds true iff .
Thus, if the unforced linear system is stable () and  then, the controller \eqref{nonlinear PI} results in bounded and attractive closed-loop behavior. This also provides a strict proof for the integrator example of \cite{Ortega_paper}, \cite{AKO_book}.
\end{remark}
\begin{remark}\label{remark_sector}
Note that the r.h.s. of condition (ii) tends to  in the limit . Thus,  for some sector bounded nonlinearity \eqref{sector bounds}, if we select  then there exists some  such that for all  the closed-loop system \eqref{G-SE}, \eqref{PI_o}, \eqref{z} is globally bounded and attractive.
\end{remark}
\section{Simulation results}
 \label{simulation}
A simulation study was performed for the perturbed integrator  (P-INT) and the perturbed linear system  (P-LS) described by  \eqref{G-S}, \eqref{G-SL} respectively with parameters , , ,  and initial conditions . For the specific parameters, condition (i) of Theorem \ref{main_theorem} holds true. We tested the case of a Nussbaum gain based (NG) controller \eqref{nussbaum controller} and a nonlinear PI controller \eqref{nonlinear PI} with gains   (not a Nussbaum function) denoted as nPI and  (Nussbaum function) denoted as nPI-N.
\begin{figure}[!t]
\centering
\includegraphics[width=4.1in]{comparisons}
\caption{Linear system: Time responses of   for the cases of a perturbed integrator (P-INT) and a perturbed linear system (P-LS) for the three controllers NG, nPI, nPI-N.}
\label{comparisons}
\end{figure}
The output response  shown in Fig. \ref{comparisons} verifies our theoretical analysis. Particularly, for the P-INT system with the NG controller,  is divergent as shown in \cite{GS}. If the nPI controller is used then  remains bounded and converges to zero \cite{Ortega_paper}, \cite{AKO_book}. However, the nPI control fails to regulate the P-LS system. Convergent solutions are obtained for the P-LS system only when the nPI-N is employed.

Let now the perturbed nonlinear system \eqref{G-SE} with  where ,  and . Selecting , we have that both conditions (i) and (ii) are satisfied for every  (see Remark \ref{remark_sector}).  For the  control law \eqref{PI_o}, \eqref{z},  simulation results are shown in Fig. \ref{sector_example} with   and initial conditions .
\begin{figure}[!t]
\centering
\includegraphics[width=4.1in]{sector_example}
\caption{Nonlinear system: Time responses of system states   and control input .}
\label{sector_example}
\end{figure}
As expected, all  are bounded and converge to the origin as time passes.
 \section{Conclusions}
Sufficient conditions are derived for global boundedness and attractivity of a perturbed nonlinear system with sector-bounded nonlinearity under a nonlinear PI control action.
The results further demonstrate the superiority of the nonlinear PI controls compared to simple Nussbaum gain based schemes with respect to robustness to unmodelled dynamics.







\begin{thebibliography}{00}
\bibitem{Nussbaum_paper}
R. D. Nussbaum, Some remarks on a conjecture in parameter adaptive
control, Syst. Control Lett. 3(1983) 243-246.
\bibitem{Ye_Jiang98}
X. Ye, J. Jiang, Adaptive nonlinear design without a priori knowledge
of control directions, IEEE Trans. Autom. Contr. 43 (1998) 1617-1621.
\bibitem{GS}
T.T. Georgiou, M.C. Smith, Robustness analysis of nonlinear feedback systems: an input-output approach, IEEE Trans. Automatic Contr. 42 (1997) 1200-1221.
\bibitem{Ortega_paper}
R. Ortega, A. Astolfi, N.E. Barabanov, Nonlinear PI control of uncertain systems: an alternative to
parameter adaptation, Systems \& Control Letters 47 (2002) 259-278.
\bibitem{AKO_book}
A. Astolfi, D. Karagiannis, R. Ortega, Nonlinear and Adaptive Control with Applications, Springer-Verlag, 2008.
\end{thebibliography}
\end{document}
