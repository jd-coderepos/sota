

\documentclass[3p]{elsarticle}







\usepackage{amssymb}

\usepackage{graphicx}
\usepackage{balance}
\usepackage{color}
\usepackage{xcolor}
\usepackage{multirow}
\usepackage{caption}


\usepackage{subfigure,epsfig,amsfonts,amsmath,cases,amssymb,graphics, latexsym, times, algorithmic, algorithm, bbm, color, soul,wrapfig,comment}




\let\today\relax
\journal{Smart Health}

\begin{document}

\begin{frontmatter}






\title{Early hospital mortality prediction using vital signals}




\author[wm]{Reza Sadeghi\corref{cor1}}
\ead{sadeghi.2@wright.edu}

\author[wm]{Tanvi Banerjee}
\ead{tanvi.banerjee@wright.edu}

\author[mines]{William Romine}
\ead{william.romine@wright.edu }

\address[wm]{Department of Computer Science and Engineering, Kno.e.sis Research Center, Wright State University, Dayton, OH, USA}

\address[mines]{Department of Biological Sciences, Wright State University, Dayton, OH, USA}

\cortext[cor1]{Corresponding author}



\begin{abstract}
Early hospital mortality prediction is critical as intensivists strive to make efficient medical decisions about the severely ill patients staying in intensive care units (ICUs). As a result, various methods have been developed to address this problem based on clinical records. However, some of the laboratory test results are time-consuming and need to be processed. In this paper, we propose a novel method to predict mortality using features extracted from the heart signals of patients within the first hour of ICU admission. In order to predict the risk, quantitative features have been computed based on the heart rate signals of ICU patients suffering cardiovascular diseases. Each signal is described in terms of  statistical and signal-based features. The extracted features are fed into eight classifiers: decision tree, linear discriminant, logistic regression, support vector machine (SVM), random forest, boosted trees, Gaussian SVM, and K-nearest neighborhood (K-NN). To derive insight into the performance of the proposed method, several experiments have been conducted using the well-known clinical dataset named Medical Information Mart for Intensive Care III (MIMIC-III). The experimental results demonstrate the capability of the proposed method in terms of precision, recall, F1-score, and area under the receiver operating characteristic curve (AUC). The decision tree classifier satisfies both accuracy and interpretability better than the other classifiers, producing an F1-score and AUC equal to  and , respectively. It indicates that heart rate signals can be used for predicting mortality in patients in the care units especially coronary care units (CCUs), achieving a comparable performance with existing predictions that rely on high dimensional features from clinical records which need to be processed and may contain missing information.

\end{abstract}

\begin{keyword}
intensive care, mortality prediction, statistical and signal-based features
\end{keyword}

\end{frontmatter}

\section{Introduction}
  \label{sec:intro}

Intensive care unit (ICU) is a ward in hospital, where seriously ill patients are cared for by specially trained staff.  Quick and accurate decisions for the patients are needed. As a result, a wide range of decision support systems have been deployed to aid intensivists for prioritizing the patients who have a high risk of mortality.

Most mortality prediction systems are considered as score-based models~\cite{calvert_using_2016}\cite{simpson_new_2016}\cite{le_gall_new_1993}\cite{knaus_apache_1985} which appraise disease severity to predict an outcome. These models utilize patient demographics and physiological variables such as age, temperature, and heart rate collected within the initial  to  hours after ICU admission with the aim of assessing ICU performance. The score-based models employ certain features that sometimes are not available at ICU admission. Also, they make decisions according to a collection of data after at least first  hours of ICU admission. To enhance the proficiency, the customized models refine the score-based models for usage within specific conditions. For instance,~\cite{dervishi_fuzzy_2017} introduces a model to predict the risk of mortality due to cardiorespiratory arrest. Although these models provide adequate results, the ICU patients are varied and subjected to multiple diseases. Therefore, selecting the right model for a special patient who is immediately admitted to ICU is difficult. On the other hand, various studies~\cite{awad_early_2017}\cite{wojtusiak_c-lace:_2017}\cite{ribas_severe_2011}\cite{kim_comparison_2011}\cite{purushotham_benchmark_2017} express the superiority of data mining techniques over traditional score-based models. The data mining models have exerted different techniques such as  random forest~\cite{awad_early_2017}\cite{wojtusiak_c-lace:_2017}, support vector machine~\cite{ribas_severe_2011}, decision tree~\cite{kim_comparison_2011}, and deep learning~\cite{purushotham_benchmark_2017}\cite{avati_improving_2017}\cite{beaulieu-jones_mapping_2017}\cite{song_attend_2017}. Furthermore, some of the methods like \cite{venugopalan_combination_2017} engage a pipeline of data mining techniques to predict the risk of mortality. These methods are organized based on certain clinical records which are collected in initial hours after ICU admission. However, laboratory test results need to be processed and many clinical records contain missing values~\cite{yadav_mining_2018}. While vital signals can provide numerous information which has been proven to possess strong relation with the mortality~\cite{zhang_resting_2015}. Therefore, vital signal fluctuations can provide high capability to predict the mortality risk more accurately and faster than clinical-based methods.

The main goal of this paper is to provide an early mortality prediction of patients based on their first hour after ICU admission according to their heart rate signals. Our study relies on the Medical Information Mart for Intensive Care III, MIMIC-III Waveform Database records~\cite{johnson_mimic-iii_2016}. We propose a method to extract both statistical and signal-based features from the heart signals and employ well-known classifiers such as logistic regression and decision tree to predict hospital mortality, i.e. death inside the hospital.

The rest of the paper is organized as follows: Section~\ref{sec:related_work} presents a literature review on the related studies. Section~\ref{sec:methodology} describes the proposed method in four subsections of data description, signal preprocessing, feature extraction, and classification. To evaluate the performance of the proposed method, Section~\ref{sec:experiments} is allocated to the experiments and discussions. Finally, Section~\ref{sec:conclusion} summarizes the conclusion and future work.

\section{Related Work}
  \label{sec:related_work}

There is an increasing interest in addressing early hospital mortality prediction. The proposed systems can be categorized into three classes of score-based, customized, and data mining models.

Various score-based approaches such as acute physiology and chronic health evaluation (APACHE)~\cite{knaus_apache_1985}, simplified acute physiology score (SAPS)~\cite{le_gall_new_1993}, and quick sepsis-related organ failure assessment score (qSOFA)~\cite{simpson_new_2016} have been proposed. APACHE score is the best-known and widely used in intensive cares~\cite{vincent_critical_2010}. The original APACHE score~\cite{knaus_apache-acute_1981} employed 34 physiological measures from initial 24 hours after ICU admission to determine the chronic health status of the patients.~\cite{knaus_apache_1985} introduced the APACHE II scoring model including a reduction in the number of variables to  routine physiological measurements, along with the age of patients. Extending that, the APACHE III improved the effectiveness of mortality prediction by adding new variables such as race, length of stay in ICU, and prior place before ICU. APACHE IV also endeavored to enhance the over prediction problem of the APACHE III by adding new variables and using the weights utilized in APACHE III~\cite{zimmerman_acute_2006}. The traditional severity of illness score-based models commonly attempted to predict based on either specific age ranges, or information recorded within the first  hours of ICU admission~\cite{johnson_reproducibility_2017}. Furthermore, they utilized features which are not always available at the time of ICU admission. For instance, the APACHE IV applied its analysis on over  variables like chronic health variables of AIDS, cirrhosis, hepatic failure, immunosuppression which may not be recorded at the time of admission.

The customized models make a decision according to the characteristics of either specific health problems such as cardiorespiratory arrest~\cite{dervishi_fuzzy_2017} and early severe sepsis~\cite{le_gall_customized_1995}, or specific geographical areas such as France~\cite{le_gall_mortality_2005} or Australia~\cite{metnitz_austrian_2009}. For instance, Le Gall and coworkers~\cite{le_gall_mortality_2005} customized the SAPS II model based on the French patients' characteristics. They used the logit of the original SAPS II model and computed the coefficients according to the data. Furthermore, they tried to expand the second version of SAPS by adding six variables (age, sex, length of hospital stay before ICU admission, and the patient's location before ICU) that are potentially associated with mortality. Although these models provide adequate results, most ICU patients are elderly people over  years~\cite{banerjee_validating_2017} who are faced with multiple ailments. Also, selecting the right model is challenging due to the variety of patients who are immediately admitted to ICU. Moreover, the models for specific geographical areas are not extendable for other cases. 

The third class of methods employ data mining techniques to forecast mortality. For instance,~\cite{awad_early_2017} devised a method based on random forest and the synthetic minority over-sampling technique. In another method, Venugopalan et. al~\cite{venugopalan_combination_2017} used a pipeline of logistic regression, neural network, and conditional random forest. The three categories of demographic, lab, and chart data such as gender, age, height, sodium, creatinine, and heart rate have been fed to logistic regression, neural network, and conditional random forest, respectively. These methods focus on using clinical records instead of waveform data while in practice, many clinical records such as laboratory test results need to be processed which could delay the clinical decision support process. 

To address these issues, we propose a method for early mortality prediction of patients based on the first hour after ICU admission using heart rate signals. To the best of our knowledge, this paper is the first work which utilizes only heart signals for early mortality prediction using the MIMIC-III dataset. We describe each signal in terms of  statistical and signal-based features which are fed into multiple transparent and non-transparent classifiers.

\section{Methodology}
  \label{sec:methodology}

This section presents a novel method which utilizes statistical and signal-based features with the purpose of fast and accurate early hospital mortality prediction. Subsection~\ref{sub:data_description} provides a review on the MIMIC-III clinical dataset while subsections~\ref{sub:signal} and~\ref{sub:feature} describe signal preprocessing and feature extraction, respectively. Ultimately, subsection~\ref{sub:classification} presents an overview on the descriptive classifiers employed to predict whether a patient survives or passes away based on the characteristics of their ECG signal.

\subsection{Data Description}
  \label{sub:data_description}

\begin{wrapfigure}{R}{0.46\textwidth}
\centering
\includegraphics[width=0.46\textwidth]{figures/Age_distribution_MIMIC.jpg}
\caption{The age distribution over the Whole MIMIC-III (without infants) and the Matched Subset}
\label{fig:Age_distribution}
\end{wrapfigure}

This study is conducted over the well-known MIMIC-III database comprising the records of  patients who stayed in critical care units. Due to the de-identification process, there are only  patients whose the clinical data in the MIMIC-III are associated with the related vital signals in the Matched Subset. As shown in the Figure~\ref{fig:Age_distribution}, the age distributions of the whole MIMIC-III (without infants) and the Matched Subset are similar. Hence, the outcomes of the Matched Subset can be extended to the whole database. It is worth mentioning that due to the de-identification process, all the patients greater than or equal to  years of age are assigned to one group.

Also, the hospital wards for patients throughout their hospital stay have been reported via the transfers table in the clinical dataset. Indeed, it specifies which of the care units described in Table~\ref{table:care_units} have been allocated to each patient in a certain time. Since nearly  percent of patients in the Matched Subset suffer from cardiovascular diseases, we have focused on predicting the risk of mortality among patients who stayed in coronary care unit (CCU) in this study. CCU is an ICU that takes patients with cardiac conditions required continuous monitoring and treatment.

\begin{table}[H]
\caption{Care Units in MIMIC-III}
\centering
\begin{tabular}{cccccc}
\hline
   \textbf{Care unit}  & \textbf{Description}\\
   \hline
   \vspace{5pt} 
   CCU  & Coronary care unit\\
   \vspace{5pt} 
   CSRU & Cardiac surgery recovery unit\\
   \vspace{5pt}    
   MICU & Medical intensive care unit\\
   \vspace{5pt} 
   NICU & Neonatal intensive care unit\\
   \vspace{5pt} 
   NWARD & Neonatal ward\\
   \vspace{5pt} 
   SICU   & Surgical intensive care unit\\    
   \multirow{1}{*}{TSICU}   &  Trauma/surgical intensive care unit\\
   \hline
\end{tabular}
\label{table:care_units}
\end{table}


\subsection{Signal Preprocessing}
  \label{sub:signal}

The recorded physiological signals are always accompanied with noise due to different recording systems. The MIMIC-III database is extracted from the CareVue and MetaVision clinical information systems provided by Philips and iMDSoft, respectively~\cite{johnson_mimic-iii_2016}. After extracting the data, we truncated the tails which contain only zeros or undefined values. Following this, we replaced the missing values with the previous known ones. Finally, the smoothed version of heart rate signal, , was computed according to the moving average filter with one-hour windows size  in the form of Equation~\ref{equ:smoothing}.  

\sum_{t=1}^{T} S(t) & \rho>=T>=1 \\
  \vspace{5pt} 
  \dfrac{1}{\rho}\sum_{t=L-\rho+1}^{L} S(t) & L-\rho+1>=T>=L \\
\end{cases}

\label{equ:average_power}
\bar{P} = \dfrac{E}{n_2 - n_1 +1} = \dfrac{1}{n_2 - n_1 +1} \sum_{n_1}^{n_2} S[n]^2

\label{equ:psd}
\bar{P} = \dfrac{\Delta T}{N} \mid \sum_{n=0}^{N-1} S[n]e^{-i2\pi\rho} \mid

\label{equ:gdi}
GDI = 1-\sum_{i} (p(i))^{2}

\label{equ:pre}
Precision = \dfrac {TP}{TP + FP}

\label{equ:rec}
Recall = \dfrac {TP}{TP + FN}

\label{equ:f1}
F1-score = \dfrac {2\times (Precision\times Recall)}{Precision + Recall}

\label{equ:risk}
Risk(x) = GDI(x) \times Probability(x)


The estimate of predictor importance for a certain feature is directly associated with the  gap between the node corresponding to that feature and its children. This estimation assigns higher importance to features which lead to the largest number of pure children (i.e. terminal nodes). This estimation allots greater importance to the features which have influence on a larger portion of the records. As a result, the feature comprising the root node (in this case the Averaged Power from Figure~\ref{fig:DT}) has higher probability than other features that define rules at lower levels. It allows the feature of the root node to be considered as one of the most important features.

\begin{wrapfigure}{R}{0.46\textwidth}
\centering
\includegraphics[width=0.46\textwidth]{figures/Predictor_importance.jpg}
\caption{Feature importance in the proposed model for mortality prediction based on heart rate signal}
\label{fig:Predictor_importance}
\end{wrapfigure}

The energy spectral density, averaged power, and range are found to be the most important features in the mortality prediction based on the heart rate signal (Figure~\ref{fig:Predictor_importance}). As described above, the averaged power is one of the most important features since it is placed as the root of the decision tree. However, the energy spectral density gained the highest score of importance in comparison to the other features. Hence, the nodes corresponding to the energy spectral density feature have higher amount of  compared to their children. As a matter of fact, this is a sign of high  gap between these nodes and their children.

The energy spectral density provides basic information about the power variation in frequency components comprising the original signal within a finite interval. Since the power spectral density employs Fourier transform to decompose original signals into a spectrum of frequencies, it can reflect the parasympathetic and sympathetic activities which are highly correlated to the fluctuation of frequency components of heart signals. It has been reported~\cite{hasegawa_assessment_2015} that the high-frequency component reflects parasympathetic nervous activity, while the ratio of low-frequency over the high-frequency components reflects sympathetic nervous activity. Hence, a combination of frequency-domain (e.g. energy spectral density) and time-domain signal analysis (such as skewness) enables us to separate CCU patients who survive or pass away.

\section{Conclusion and Future Work}
  \label{sec:conclusion}
Early hospital risk of mortality prediction in CCU units is critical due to the need for quick and accurate medical decisions. This paper proposes a new signal-based model for early mortality prediction, leveraging the benefits of statistical and signal-based features. Our method is a clinical decision support system which focuses on using only the heart rate signal instead of other health variables such physical state or presence of chronic diseases. Since such variables require laboratory test results which could delay the decision-making time or may not be available at the time of admission, our proposed method may give faster feedback to healthcare professionals working in CCUs. We demonstrate the capability of using statistical and signal-based features, especially the energy-based features of heart rate signals, to distinguish between patients who survive or pass away in the CCU. Among the interpretable classifiers, the decision tree achieved the highest accuracy, allowing for both accurate and explainable outcomes.
 
In our future work, we plan to apply our proposed method over other intensive care units, incorporating multiple vital signals along with the heart rate signal as a means to better understand the cause of mortality. The study also can be extended to develop a framework using sensors, laboratory data, and information cached from intensivists and nurses' reports using  knowledge graph~\cite{shekarpour_rquery:_2017} and text mining~\cite{allahyari_brief_2017}. Another direction is to explore the effect of computing features from vital signals with different length of windows and using dynamic feature selections~\cite{zabihimayvan_soft_2017}\cite{hamidzadeh_detection_2018}. Finally, we plan on creating a real-time mortality prediction system based on the variability of physiological signals~\cite{kaffashi_variability_2007} that can predict patient outcomes for early intervention.

\section*{Supplementary Material}
  \label{sec:supplementary}

 The source code is available at: https://github.com/RezaSadeghiWSU/Early-Hospital-Mortality-Prediction-using-Vital-Signals

\section*{Acknowledgments}
  \label{sec:Acknowledgments}
This paper is based on work supported by the National Institutes of Health (NIH) under Grant no. . Any opinions, findings, and conclusions or recommendations expressed in this material are those of the author(s) and do not necessarily reflect the views of the NIH. 





\bibliographystyle{elsarticle-num}

\bibliography{reference}











\end{document}
