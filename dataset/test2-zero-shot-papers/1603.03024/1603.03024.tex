

\documentclass[11pt]{article}


\usepackage{ifthen}

\ifthenelse{\isundefined{\GenerateShortVersion}}
{
\def\LongVersion{}
\def\LongVersionEnd{}
\long\def\ShortVersion#1\ShortVersionEnd{}
}
{
\def\ShortVersion{}
\def\ShortVersionEnd{}
\long\def\LongVersion#1\LongVersionEnd{}
}

\newcommand{\Hide}[1]{\ignorespaces}

\usepackage{amsmath, amssymb}

\usepackage{amsthm}

\usepackage{ifpdf}







\ifpdf
\usepackage[pdftex]{graphicx}
\usepackage[pdftex,ocgcolorlinks,linkcolor=blue,filecolor=blue,citecolor=blue,urlcolor=blue]{hyperref}
\else
\usepackage[dvips]{graphicx}
\fi

\ifthenelse{\isundefined{\NoGenerateLetterSize}}
{
\ifpdf
\setlength{\pdfpagewidth}{8.5in}
\setlength{\pdfpageheight}{11in}
\else
\special{papersize=8.5in,11in}
\fi
}
{}

\LongVersion \setlength{\textheight}{8.8in}
\setlength{\textwidth}{6.5in}
\setlength{\evensidemargin}{-0.18in}
\setlength{\oddsidemargin}{-0.18in}
\setlength{\headheight}{10pt}
\setlength{\headsep}{10pt}
\setlength{\topsep}{0in}
\setlength{\topmargin}{0.0in}
\setlength{\itemsep}{0in}
\renewcommand{\baselinestretch}{1.2}
\parskip=0.08in
\LongVersionEnd 

\ShortVersion \setlength{\textheight}{8.9in}
\setlength{\textwidth}{6.6in}
\setlength{\evensidemargin}{-0.2in}
\setlength{\oddsidemargin}{-0.2in}
\setlength{\headsep}{10pt}
\setlength{\topmargin}{-0.3in}
\setlength{\columnsep}{0.375in}
\setlength{\itemsep}{0pt}
\parskip=0.03in
\ShortVersionEnd 

\ShortVersion \usepackage{times}
\ShortVersionEnd 

\ShortVersion \renewcommand{\paragraph}[1]{\par\noindent\textbf{#1}}
\ShortVersionEnd 

\usepackage{algpseudocode}
\usepackage{algorithm}

\makeatletter
\renewcommand*{\ALG@name}{Pseudocode}
\makeatother

\usepackage{tikz}
\usetikzlibrary{arrows, positioning, calc}

\newtheorem{theorem}{Theorem}[section]
\newtheorem{lemma}[theorem]{Lemma}
\newtheorem{observation}[theorem]{Observation}
\newtheorem{corollary}[theorem]{Corollary}
\newtheorem{proposition}[theorem]{Proposition}

\theoremstyle{definition}
\newtheorem{property}[theorem]{Property}
\theoremstyle{plain}

\newtheorem*{observation*}{Observation}


\Hide{
\newtheorem{theorem}{Theorem}
\newtheorem{lemma}[theorem]{Lemma}
\newtheorem{observation}[theorem]{Observation}
\newtheorem{corollary}[theorem]{Corollary}
\newtheorem{proposition}[theorem]{Proposition}

\theoremstyle{definition}
\newtheorem{property}[theorem]{Property}
\theoremstyle{plain}
}

\Hide{
\newtheorem{theorem}{Theorem}[section]
\newtheorem{lemma}{Lemma}[section]
\newtheorem{observation}{Observation}[section]
\newtheorem{corollary}{Corollary}[section]
\newtheorem{proposition}{Proposition}[section]

\theoremstyle{definition}
\newtheorem{property}{Property}[section]
\theoremstyle{plain}
}

\newenvironment{AvoidOverfullParagraph}[0]
{\sloppy\ignorespaces}
{\par\fussy\ignorespacesafterend}

\newenvironment{DenseItemize}[0]
{\begin{list}{\labelitemi}{\usecounter{enumi}
\itemsep 0pt \parsep 0pt \leftmargin 5mm \labelwidth 5mm \parskip 0pt \topsep
0pt}}
{\end{list}}

\newenvironment{DenseEnumerate}[1][\theenumi.]
{\begin{list}{#1}{\usecounter{enumi}
\itemsep 0pt \parsep 0pt \leftmargin 5mm \labelwidth 5mm \parskip 0pt \topsep
0pt}}
{\end{list}}

\LongVersion \newenvironment{MathMaybe}[0]
{\ignorespacesafterend}
\LongVersionEnd \ShortVersion \newenvironment{MathMaybe}[0]
{}
\ShortVersionEnd 

\newenvironment{IntuitionSpotlight}[0]
{\par \setlength{\leftskip}{0.5\parindent}\setlength{\rightskip}{0.5\parindent}\noindent\itshape\textbf{Intuition spotlight:}}
{\par\ignorespacesafterend}

\newcommand{\NotationLabel}[1]{\label{notationTable:#1}\ignorespaces}
\newcommand{\NotationPageRef}[1]{\pageref{notationTable:#1}}

\newcommand{\Sect}{Sec.}
\newcommand{\Thm}{Thm.}
\newcommand{\Lem}{Lem.}
\newcommand{\Cor}{Cor.}
\newcommand{\Obs}{Obs.}
\newcommand{\Fig}{Fig.}
\newcommand{\Tabl}{Tab.}
\newcommand{\Appendix}{Apx.}

\newcommand{\Location}{\ell}
\newcommand{\aTime}{\mathit{t}}
\newcommand{\Reals}{\mathbb{R}}
\newcommand{\Integers}{\mathbb{Z}}
\newcommand{\Expect}{\mathbb{E}}
\newcommand{\Prob}{\mathbb{P}}
\newcommand{\sCost}{\mathrm{cost}^{\mathit{s}}}
\newcommand{\tCost}{\mathrm{cost}^{\mathit{t}}}
\newcommand{\Cost}{\mathrm{cost}}
\newcommand{\Leaves}{\mathcal{L}}
\newcommand{\LCA}{\mathrm{lca}}
\newcommand{\Depth}{\mathrm{depth}}
\newcommand{\Height}{\mathrm{height}}
\newcommand{\Ancestors}{\mathit{anc}}
\newcommand{\Indicator}{\mathbf{1}}
\newcommand{\ExpDist}{\mathrm{Exp}}
\newcommand{\PoisDist}{\mathrm{Pois}}
\newcommand{\EndTime}{\mathit{t}_{\mathrm{end}}}
\newcommand{\sEndCost}{\mathit{c}_{\mathrm{end}}^{\mathit{s}}}
\newcommand{\LateTime}{\mathit{t}_{\mathrm{late}}}
\newcommand{\Bank}{\mathit{B}}

\newcommand{\Active}{\mathit{C}}
\newcommand{\Odd}{\mathit{D}}
\newcommand{\Effective}{\mathit{F}}
\newcommand{\Stilts}{\mathcal{S}}
\newcommand{\Heads}{\mathit{H}}

\newcommand{\A}{\mathcal{A}}
\newcommand{\adv}[1]{{#1}^{*}}
\newcommand*{\xor}{\mathbin{\oplus}}
\begin{document}

\LongVersion \title{Online Matching: Haste makes Waste! \\ (Full Version)\footnote{An extended abstract will appear in Proceedings of ACM STOC 2016.
}}
\LongVersionEnd \ShortVersion \title{Online Matching: Haste makes Waste! \\ (Extended Abstract)}
\ShortVersionEnd 

\date{}

\author{Yuval Emek\thanks{Technion, Israel.
Email: \texttt{yemek@ie.technion.ac.il}.
Partially supported by the Technion-Microsoft Electronic Commerce Research
Center.}
\and
Shay Kutten\thanks{Technion, Israel.
Email: \texttt{kutten@ie.technion.ac.il}.
Partially supported by the Technion-Microsoft Electronic Commerce Research
Center and by the France Israel cooperation grant from the Israeli Ministry of
Science.}
\and
Roger Wattenhofer\thanks{ETH Zurich, Switzerland.
Email: \texttt{wattenhofer@ethz.ch}.}
}

\begin{titlepage}

\maketitle

\begin{abstract}
This paper studies a new online problem, referred to as \emph{min-cost
perfect matching with delays (MPMD)}, defined over a finite metric
space (i.e., a complete graph with positive edge weights obeying
the triangle inequality)  that is known to the algorithm in
advance.
Requests arrive in a continuous time online fashion at the points of
 and should be served by matching them to each other.
The algorithm is allowed to delay its request matching commitments, but this
does not come for free:
the total cost of the algorithm is the sum of metric distances between matched
requests \emph{plus} the sum of times each request waited since it arrived
until it was matched.
A randomized online MPMD algorithm is presented whose competitive
ratio is
,
where  is the number of points in  and  is its aspect
ratio.
The analysis is based on a machinery developed in the context of a new
stochastic process that can be viewed as two interleaved Poisson processes;
surprisingly, this new process captures precisely the behavior of our
algorithm.
A related problem in which the algorithm is allowed to clear any unmatched
request at a fixed penalty is also addressed.
It is suggested that the MPMD problem is merely the tip of the iceberg for a
general framework of online problems with delayed service that captures
many more natural problems.
\end{abstract}

\renewcommand{\thepage}{}
\end{titlepage}

\pagenumbering{arabic}

\section{Introduction}
\label{section:introduction}
Consider an online gaming platform supporting two-player games such as Chess,
Scrabble, or Street Fighter 4.\footnote{Leading gaming platforms include XBOX Live and the Playstation Network for
consoles, Steam for PCs, and web-based platforms such as Geewa, Pogo and Yahoo
Games.}
The platform tries to find a suitable opponent for each player connecting to
it;
matching two players initiates a new game between them.
The platform should minimize two criteria:
(i) the difference between the matched players' rating (a positive integer
that represents the player's skill), so that the game is challenging for both
players; and
(ii) the waiting time until a player is matched and can start playing since
waiting is boring.
(In reality, the -dimensional player rating space is often generalized to a
more complex metric space by taking into account additional parameters such as
the network distance between the matched players.)
It turns out, though, that these two minimization criteria are often
conflicting:
What if the pool of players waiting for a suitable opponent does not contain
anyone whose rating is close to that of a new player?
Should the system match the new player to an opponent whose rating differs
significantly from hers?

The naive approach that matches players immediately does
a terrible job:
Murphy's Law may strike, and right after matching a player, a perfect opponent
will emerge:
Haste makes waste, unbounded waste in fact.
To cope with this challenge, we must allow the platform to delay its
service in a \emph{rent-or-buy} manner.

\paragraph{Model.}
Let

be a finite metric space.
Consider a set  of \emph{requests}, where each request 
is characterized by its \emph{location}

\NotationLabel{model:location}
(also referred to as the point that \emph{hosts} ) and \emph{arrival
time}
.\footnote{
For ease of reference, \Tabl{}~\ref{table:notation} provides an index for the
notation used throughout this paper.
}
\NotationLabel{model:arrival-time}
Assume for the time being that  is even.\footnote{The problem presented here is not well defined if  is odd however, later
on we discuss a variant of this problem that is well defined for any finite
.
}
Notice that  can have multiple requests with the same location (in
particular,  is unbounded with respect to );
for simplicity, we assume that each request has a unique arrival
time.\footnote{This assumption is without loss of generality as the arrival times can be
slightly perturbed.
}

The input to an online algorithm for the \emph{min-cost perfect matching with
delays (MPMD)} problem is a finite metric space ,
provided to the algorithm before the execution commences, and a request set
 over  such that each request  is presented to
the algorithm in an online fashion at its arrival time
.
The goal of the algorithm is to construct a (perfect) \emph{matching} of the
request set --- namely, a partition of  into  unordered request
pairs --- in an online fashion with no preemption.

The algorithm is allowed to delay the matching of any request in  at a
cost.
More formally, the requests  and  can be matched at any time
;
if algorithm  matches requests  and  at time ,
then it incurs a \emph{time cost} of

\NotationLabel{model:time-cost-request}
and a \emph{space cost} of

\NotationLabel{model:space-cost-request}
for serving ,
.
The \emph{space cost} and \emph{time cost} of  for the whole request set
are

\NotationLabel{model:space-cost-instance}
and
,
\NotationLabel{model:time-cost-instance}
respectively.
The objective is to minimize
.
\NotationLabel{model:total-cost}
When  is clear from the context, we may drop the subscript.

Following the common practice in online computation (cf.\ \cite{BorodinE1998}),
the quality of an online MPMD algorithm is measured in terms of its
\emph{competitive ratio}.
Online MPMD algorithm  is said to be \emph{-competitive} if for
every finite metric space , there exists some

such that for every (even size) request set  over , it
is guaranteed that
,
where the expectation is taken over the coin tosses of the algorithm
(if any) and  is an optimal offline algorithm.
It is assumed that  and  are generated by an \emph{oblivious
adversary} that knows , but not the realization of its coin tosses.

\paragraph{Related work.}
The \emph{rent-or-buy} feature is fundamental to
\LongVersion many online applications and thus, prominent in the theoretical study of
\LongVersionEnd online computation.
Classic online
\LongVersion problems in which the rent-or-buy feature constitutes the
sole source of difficulty
\LongVersionEnd \ShortVersion rent-or-buy problems
\ShortVersionEnd include
ski-rental
\cite{KarlinMRS1986,KarlinMMLO1990,KarlinKR2001} and
TCP acknowledgment
\cite{DoolyGS1998,DoolyGS2001,KarlinKR2001}.
\LongVersion In other problems, the rent-or-buy feature is
\LongVersionEnd \ShortVersion This feature is sometimes
\ShortVersionEnd combined with a complex combinatorial structure, enhancing an already
challenging online problem, e.g., the extension of online job scheduling
\cite{AwerbuchKP1992,AwerbuchALR2002} studied in
\cite{AverbakhX2007,AverbakhB2013,AzarEJV2016}.

The \emph{matching} problem is a combinatorial optimization celebrity ever
since the seminal work of Edmonds~\cite{Edmonds1965a,Edmonds1965b}.
The realm of \emph{online algorithms} also features an extensive literature on
matching and some generalizations thereof.
Online problems that have been studied in this regard include
maximum cardinality matching
\cite{KarpVV1990,BirnbaumM2008,GoelM2008,DevanurJK2013,Miyazaki2014,NaorW2015},
maximum vertex-weighted matching
\cite{AggarwalGKM2011,DevanurJK2013,NaorW2015},
maximum capacitated assignment (a.k.a.\ the AdWords problem)
\cite{MehtaSVV2005,BuchbinderJN2007,GoelM2008,AggarwalGKM2011,NaorW2015},
metric maximum weight matching
\cite{KalyanasundaramP1993,KhullerMV1994},
metric minimum cost perfect matching
\cite{KalyanasundaramP1993,MeyersonNP2006,BansalBGN2014}, and
metric minimum capacitated assignment (a.k.a.\ the transportation
problem) \cite{KalyanasundaramP2000};
see \cite{Mehta2013} for a comprehensive survey.
All these online problems are \emph{bipartite} matching versions, where the
nodes in one side of the graph are static and the nodes in the other side are
revealed in an online fashion together with their incident edges.

\paragraph{Discussion and results.}
To the best of our knowledge,
\LongVersion the MPMD problem
\LongVersionEnd \ShortVersion MPMD
\ShortVersionEnd is the first online
\emph{all-pairs} matching version.
Moreover, in contrast to the previously studied online matching versions, in
MPMD the graph (or metric space) is known a-priori and the algorithmic
challenge stems from the unknown locations and arrival times of the
\LongVersion requests (whose number is unbounded);
\LongVersionEnd \ShortVersion requests;
\ShortVersionEnd this is more in the spirit of online problems such as the classic -server
problem.

\ShortVersion \sloppy \ShortVersionEnd
The main
\LongVersion technical \LongVersionEnd
result of this paper is a randomized online MPMD algorithm
whose competitive ratio is
,
where  is the number of points in the metric space  and
 is its aspect ratio.
This algorithm, presented in \Sect{}~\ref{section:algorithm}, is based on
\emph{exponential timers} that determine how long should we wait before
committing to a certain match.
The analysis, presented in \Sect{}~\ref{section:analysis}, relies heavily on
machinery we develop in the context of a new stochastic process named
\emph{alternating Poisson process}.
\ShortVersion \par\fussy \ShortVersionEnd

We also consider a variant of the online MPMD problem, referred to as
\emph{MPMDfp}, in which the algorithm can clear any unmatched request at a
fixed penalty.
This problem variant is motivated by noticing that clearing an unmatched
request may correspond to matching a player with a computer opponent in the
context of the aforementioned gaming platforms.
The MPMDfp problem is discussed further in
\LongVersion \Sect{}~\ref{section:fixed-penalty},
\LongVersionEnd \ShortVersion the full version,
\ShortVersionEnd where we show that our online algorithm can be adjusted to cope with this
variant as well.

It is not difficult to develop constant lower bounds on the competitive ratio
of online MPMD algorithms already for the special case of a -point metric
\LongVersion space (note that this special case generalizes the ski rental problem).
\LongVersionEnd \ShortVersion space.
\ShortVersionEnd While a -point metric space admits an -competitive online MPMD
algorithm, we conjecture that in the general case, the competitive ratio must
grow as a function of .
In particular, we believe that this conjecture holds for the -dimensional
metric spaces constructed in Appendix~C of \cite{EmekLW2015} (a variant of the
construction in Fig.~1 of \cite{ReingoldT1981}).
We also establish an algorithm-specific lower bound:
To demonstrate the role of randomness in the online algorithm presented in
\Sect{}~\ref{section:algorithm}, we show in
\LongVersion \Sect{}~\ref{section:specific-lower-bound}
\LongVersionEnd \ShortVersion the full version
\ShortVersionEnd that the competitive ratio of its natural deterministic counterpart is .

\paragraph{Online problems with delayed service.}
The online MPMD problem is obtained by augmenting the (offline) min-cost
perfect matching problem with the time axis over which service can be delayed
in a rent-or-buy manner.
This viewpoint seems to open a gate to a general framework of online problems
with delayed service since the approach of combining the rent-or-buy feature
with a combinatorial optimization offline problem can be applied to a class of
minimization problems much larger than just min-cost perfect matching.

To be more precise, consider a minimization problem  defined with
respect to some underlying combinatorial structure  with a ground
set

of \emph{input entities} and a ground set

of \emph{output entities}.
The input and output instances of  are multisets over

and
,
respectively.
For each input instance , problem 
determines a collection  of \emph{feasible}
output instances;
input instance  is said to be \emph{admissible} if
.
We restrict our attention to problems  satisfying
the property that for every two input instances
,
if  and  are admissible, then so is
.\footnote{We follow the standard multiset convention that for two multisets  over
a ground set  with multiplicity functions

and
,
the relation

holds if

for every
; and

is the multiset whose multiplicity function

satisfies

for every
.
}

Minimization problem  can be transformed into an online
problem with delayed service  by applying to it the
\emph{delayed service operator}:
Each request in  is characterized by its \emph{location} ---
an entity in  ---  and by its arrival time.
The algorithm can serve a collection  of yet unserved requests by buying a
feasible (under ) output instance  for their location multiset
at any time  after the arrival of all requests in .
The payment for this service is the cost of  plus the total waiting times
of the requests in  up to time .
Notice that this act of buying  does not serve requests other than those in
 including any request arriving at the locations of  after time .

\LongVersion The online MPMD problem
\LongVersionEnd \ShortVersion MPMD
\ShortVersionEnd is obtained by applying this delayed service operator to the
metric min-cost perfect matching problem, where
 is a finite metric space,
 is its points, and
 is the set of
\LongVersion unordered
\LongVersionEnd point pairs (a point multiset is an admissible input instance if its
cardinality is even).\footnote{In the offline version of the metric min-cost perfect matching problem it
suffices to consider only sets (rather than multisets) for the input and
output instances.
The generalization to multisets is necessary for the transition to the online
version of the problem.
}
This operator can also be applied to
the vertex cover problem
( is a graph,  is the edge set, and
 is the vertex set),
the dominating set problem
( is a graph and  and
 are the vertex set),
and many more combinatorial optimization problems.

\section{Preliminaries}
\label{section:preliminaries}


\paragraph{Tree notation and terminology.}
Consider a tree  \emph{rooted} at some vertex  with a \emph{leaf} set
.
The notions \emph{parent}, \emph{ancestor}, \emph{child}, and \emph{sibling}
are used in their usual sense.
A \emph{binary} tree is called \emph{full} if every internal vertex has
exactly two children.

Let  be some vertex in .
The parent of  in  (assuming that ) is denoted by
.
\NotationLabel{tree:parent}
We denote the \emph{subtree} of  rooted at  by 
\NotationLabel{tree:subtree}
and the leaf set of  by .
\NotationLabel{tree:subtree-leaves}
The set of ancestors of  (excluding  itself) is denoted by
.
\NotationLabel{tree:ancestors}
The \emph{depth} of  in  --- i.e., the distance (in hops) from  to
 --- is denoted by

\NotationLabel{tree:depth}
and the \emph{height} of  is denoted by
.
\NotationLabel{tree:height}

A \emph{stilt} in  is an oriented path connecting some vertex ,
referred to as the \emph{head} of the stilt, with a leaf in ,
referred to as the \emph{foot} of the stilt.
Given two leaves , their \emph{least common ancestor (LCA)}
in  is denoted by
.
\NotationLabel{tree-lca}

\paragraph{Probabilistic embedding in tree metric spaces.}
Let

be a weight function on the vertices of  that satisfies
(i)  for every leaf ; and
(ii) 
for every vertex
.
The pair

introduces a finite metric (in fact, an ultrametric) space over the leaf set
 with distance function  defined by setting

for every .
A metric space that can be realized by such a  pair is referred to as
a \emph{tree metric space}.
We subsequently identify a tree metric space with the pair  that
realizes it.

Consider some real .
A \emph{hierarchically well separated tree} with parameter 
(cf.\ \cite{Bartal1996}), or \emph{-HST} in short, is a tree metric
space  that, in addition to the aforementioned
requirements, satisfies

for every vertex
.
We refer to an -HST realized by a full binary tree  (cf.\
\cite{CoteMP2008}) as an \emph{-HSBT}.

The following theorem is established by combining a celebrated construction of
Fakcharoenphol et al.~\cite{FakcharoenpholRT2004} (improving previous
constructions of Bartal \cite{Bartal1996,Bartal1998}) with a tree
transformation technique \cite{PattShamir2015}
\LongVersion (details are deferred to \Appendix{}~\ref{appendix:embedding-in-HSBT}).
\LongVersionEnd \ShortVersion (details are deferred to the full version).
\ShortVersionEnd 

\begin{theorem} \label{theorem:HSBT}
Consider some -point metric space

of aspect ratio

and let  be the set of all -HSBTs

over  with

and with
distance functions  that dominate  in the sense that

for every .
There exists a probability distribution  over  such
that

for every .
Moreover, the probability distribution  can be sampled
efficiently.
\end{theorem}

\paragraph{Matching algorithm notation and terminology.}
Consider the operation of an MPMD algorithm on some HSBT .
\LongVersion Recall that the input to the algorithm consists of a finite set  of
requests, where each request

is characterized by its location

and arrival time
.
\LongVersionEnd Suppose that the algorithm matches requests  and  with

and
,
.
Let  be some vertex in the unique path connecting  and  in .
If
,
then we refer to this matching operation as matching \emph{across} ;
otherwise, we refer to it as matching \emph{on top of} .
Notice that matching across  corresponds to matching a request located in
 with a request located in , where  and
 are the children of  in , whereas matching on top of 
corresponds to matching a request located in  with a request
located in .

If the algorithm matches request
\LongVersion 
\LongVersionEnd \ShortVersion 
\ShortVersionEnd at time , then  is said to be \emph{active} at all times
.
Given some vertex , we denote the set of active requests in
 at time  by

\NotationLabel{alg:active}
and write
.
\NotationLabel{alg:active-root}
Vertex  is said to be \emph{odd} at time  if
;
let

\NotationLabel{alg:odd}
be the set of odd vertices at time .

A key observation is that the forest induced on  by the vertex subset
 is a collection --- denoted hereafter by  ---
\NotationLabel{alg:stilts}
of vertex disjoint stilts.
Moreover, if  is the head of a stilt in  then either
(1)  is the root of  (which implies that  is odd); or
(2) the sibling of  is also the head of a stilt in .
Let

\NotationLabel{alg:heads}
be the set of heads of stilts in .

Internal vertex  is said to be \emph{effective} at time 
if its two children are odd (which, in particular, means that  is not odd);
let

\NotationLabel{alg:effective}
be the set of effective vertices at time .
Notice that  is effective if and only if its two children are in
 and let

be their corresponding stilts.
We refer to the feet of  and  as the
\emph{supporting leaves} of  at time .

We shall apply the aforementioned matching algorithm definitions to both our
online MPMD algorithm, denoted by , and to the benchmark offline MPMD
algorithm, denoted by .
To distinguish between the two, we reserve the
aforementioned notation system for the former and add a superscript asterisk
for the latter;
in particular, the set of vertices odd under  at time  is denoted
by

\NotationLabel{alg:adv-odd}
(whereas the set of vertices odd under  at time  is denoted by
).

\section{An online MPMD algorithm}
\label{section:algorithm}
In this section, we present our online MPMD algorithm, referred to as the
\emph{stilt-walker algorithm} and denoted hereafter by ;
its competitive ratio is analyzed in \Sect{}~\ref{section:analysis}.
The algorithm works in two stages:
a preprocessing stage, in which we employ \Thm{}~\ref{theorem:HSBT} to embed
the input metric space in a random -HSBT ,
and the actual online execution, in which  processes the requests arriving
at the leaves of  and constructs the desired matching.
The remainder of this section is dedicated to describing the latter.

\paragraph{The matching policy.}
Although  operates in continuous time, it will be convenient to describe
it as if it progresses in discrete \emph{time steps}, taking the difference  between two consecutive time steps to be infinitesimally small so that at
most one request arrives in each time step.

Fix some time step .
If request  arrives at this time step and  already
hosts another active (under ) request , then the algorithm matches
 and  immediately.
Assume hereafter that each leaf in  hosts at most one active
request.

Consider some effective vertex  and let  be its supporting leaves (the feet of the corresponding stilts in
).
By definition,  hosts an odd number of active requests at time  for
 and since it cannot host more than one active request, it
follows that there exists a unique active request  at time 
with
;
we refer to  and  as the
\emph{supporting requests} of .
The algorithm tosses an independent biased coin and matches its supporting
requests (i.e., matching across ) with probability
.
In what follows, we attribute this coin toss to  so that we can distinguish
between coin tosses of different (internal) vertices.
A pseudocode description of the stilt-walker algorithm is provided in
Pseudocode~\ref{algorithm:stilt-walker}.

\def\PseudocodeStiltWalker{
\begin{algorithm}
\caption{\label{algorithm:stilt-walker}The operation of  at time step .
}
\begin{algorithmic}[1]
\If{ with } \Comment{there can be at most one such request pair}
  \State{match  and }
\EndIf
\ForAll{}
  \State{ supporting leaves of }
  \State{ unique active request with
 for }
  \State{ outcome of an independent Bernoulli trial
with parameter }
  \If{}
    \State{match  and } \Comment{matching across }
  \EndIf
\EndFor
\end{algorithmic}
\end{algorithm}
}

\LongVersion \PseudocodeStiltWalker{}
\LongVersionEnd 

An analogous ``continuous'' description of the stilt-walker algorithm's policy
regarding the effective vertices is as follows.
Consider some internal vertex  and suppose that the last
time  matched across  was at time  (take

if  still has not matched across ).
Then, the next time the algorithm matches across  is the minimum 
that satisfies
\begin{MathMaybe}
\int_{t_{0}}^{t_{1}} \Indicator(v \in \Effective(t)) \, d t
=
Z \, ,
\end{MathMaybe}
where

denotes the indicator operator and

is an (independent) random variable that obeys an exponential distribution
with rate .

\begin{IntuitionSpotlight}
The reader may wonder about the role of the exponential timers maintained at
the internal vertices.
At first, we tried to analyze the deterministic version of the algorithm,
where the -rate exponential timer maintained at vertex  is replaced by a deterministic -timer.
This seemed to make sense because it allows the algorithm to wait for  time before it pays  in space cost (the usual approach to
rent-or-buy problems).
However, as demonstrated in
\LongVersion \Sect{}~\ref{section:specific-lower-bound},
\LongVersionEnd \ShortVersion the full version,
\ShortVersionEnd this is hopeless.
Switching to the randomized version resolves this obstacle because the
memoryless exponential timers allow us to analyze each vertex independently
and partition the time into periods so that each period can be analyzed
independently --- see \Sect{}~\ref{section:analysis-heart}.
\end{IntuitionSpotlight}

Notice that our algorithm is guaranteed to eventually match all requests with
probability .
Indeed, if there are at least two active requests at time , then there is
at least one effective vertex  at time  and  matches across it (thus
matching its supporting requests) at time  with probability .

\section{Analyzing the stilt-walker algorithm}
\label{section:analysis}
Our main goal in this section is to establish the following Theorem.

\begin{theorem} \label{theorem:main}
Fix some  and consider an -HSBT 
realized by a full binary tree of height .
Let  be a request set over  and let  be
some benchmark offline MPMD algorithm for , .
The stilt-walker algorithm  guarantees that

where

depends only on  and is independent of .
\end{theorem}

\ShortVersion Showing that combining \Thm{}\ \ref{theorem:HSBT} and \ref{theorem:main}
yields the desired upper bound on the competitive ratio of  is based on
relatively standard arguments and is deferred to the full version.
\ShortVersionEnd \LongVersion We will soon turn our attention to the proof of \Thm{}~\ref{theorem:main},
but first, let us show that it yields the desired upper bound on the
competitive ratio of .
To that end, fix some -point metric space

of aspect ratio  and a request set  over  and let
 be an optimal (offline) algorithm for  (over
).
Let  be the probability distribution promised by
\Thm{}~\ref{theorem:HSBT} when applied to .
Denoting the coin tosses of  by  and taking  to be some
HSBT in the support of , we can employ \Thm{}~\ref{theorem:main}
to conclude that

where  is the projection of  on 
(that is, same requests are matched at the same time, incurring possibly
different space costs).
Therefore,

where
,
the first transition holds since the distance functions in the
support of  dominate ,
the third transition holds since the time costs of
 in  are the same as those of  in
, and
the fourth transition holds by \Thm{}~\ref{theorem:HSBT}.
\par
\LongVersionEnd The remainder of this section is dedicated to the proof of
\Thm{}~\ref{theorem:main} and is organized as follows:
First, in \Sect{}~\ref{section:alternating-Poisson-process}, we introduce a
new stochastic process, called \emph{alternating Poisson process (APP)},
together with some related machinery.
APPs play a major role in \Sect{}~\ref{section:analysis-heart} that forms the
heart of the analysis:
we prove \Thm{}~\ref{theorem:main} assuming that online algorithm 
receives a special \emph{end-of-input} signal upon receiving the last request
in  and responds to it by immediately matching all remaining active
requests.
\LongVersion Finally, in \Sect{}~\ref{section:lift-end-of-input-assumption}, we lift the
assumption of receiving the end-of-input signal, showing that it does not
affect the (multiplicative) competitive ratio.
\LongVersionEnd \ShortVersion Lifting the end-of-input signal assumption is relatively straightforward and
is deferred to the full version.
\ShortVersionEnd 

\subsection{Alternating Poisson processes}
\label{section:alternating-Poisson-process}
A major component of the analysis presented in
\Sect{}~\ref{section:analysis-heart} is a stochastic process (more
specifically, a point process) that we refer to as an \emph{alternating
Poisson process (APP)}.
This process is parametrized by its
\emph{start time} ,
\emph{length} ,
\emph{rate} , and
a right-continuous \emph{coloring function}

with finitely many discontinuity points.\footnote{The color  is redundant for the analysis of the APPs carried out in the
present section.
We introduce it because it makes things simpler in
\Sect{}~\ref{section:analysis-heart} when we employ the APP framework in the
analysis of our online algorithm.
}
For simplicity, in the remainder of this section, we assume that
the APP starts at time
\LongVersion ;
this assumption can be lifted by translating any time

to
.
\LongVersionEnd \ShortVersion .
\ShortVersionEnd 

Given some , we define the \emph{-volume} and
\emph{-volume} of the interval  as
\begin{MathMaybe}
V_{1}(t, t') = \int_{t}^{t'} \Indicator(c(x) = 1) \, d x
\end{MathMaybe}
and
\begin{MathMaybe}
V_{2}(t, t') = \int_{t}^{t'} \Indicator(c(x) = 2) \, d x \, ,
\end{MathMaybe}
respectively.
The APP is realized by independent and identically 
distributed random variables

These determine the -valued random variables
,
referred to as \emph{alternation times}, defined inductively by fixing

and setting

for 
Put differently, the alternation times divide the process into
\emph{iterations} so that iteration  lasts from time  to time
.
In odd (resp., even) iterations, the process \emph{digests} the s (resp.,
s), ignoring the s and the s (resp., s).
If the iteration did not end by time

and

(resp.,
),
then it ends at time  with probability
;
the iteration ends at time  if it did not end beforehand (an
illustration is provided in \Fig{}~\ref{figure:app}).

\def\FigureApp{
\LongVersion
\begin{figure}
\LongVersionEnd
\ShortVersion
\begin{figure}[h]
\ShortVersionEnd
\begin{center}
\includegraphics[width=\textwidth]{app}
\end{center}
\caption{ \label{figure:app}
A realization of an alternating Poisson process with time progressing from
left to right.
The dark gray, light gray, and white intervals represent the colors , ,
and , respectively.
The vertical arrows represent the meaningful alternation times and the
horizontal two-sided arrows depict the time intervals that contribute to the
digestion of the corresponding iterations.
}
\end{figure}
}
\LongVersion \FigureApp{}
\LongVersionEnd 

The definition of the alternation times implies, in particular, that if
, then ;
we say that the th alternation time is \emph{meaningful} if
.
Observe that if  is meaningful and  is odd (resp., even),
then  must be  (resp., ).
Let
\begin{MathMaybe}
N
=
\max \{ j \in \Integers_{\geq 0} \mid T_{j} < \gamma \}
\end{MathMaybe}
be the random variable counting the number of meaningful alternation times.
\LongVersion \par
\LongVersionEnd Define the -valued random variables  by setting

and let
.
We refer to  as the \emph{digestion} of the th iteration and to 
as the \emph{total digestion}.

\begin{lemma} \label{lemma:APP-low-bound-last-digestion}
For every , we have
,
where

if  is odd;
and

if  is even.\footnote{Recall that for every
,
an odd (resp., even)  implies that

(resp.,
).
}
\end{lemma}
\LongVersion \begin{proof}
Assume without loss of generality that  is odd and  (the case that
 is even and  is proved following the same line of arguments).
The design of the APP implies that conditioned on
,
the random variable  satisfies
,
that is, it is distributed identically to an exponential random variable with
rate , truncated at .
Fixing
,
the assertion follows by observing that

where the second transition is derived using integration by parts with

and
.
\end{proof}
\LongVersionEnd 

\begin{lemma} \label{lemma:APP-digestion-equals-number-meaningful}
.
\end{lemma}
\LongVersion \begin{proof}
Let , , be an indicator random variable for the event

and notice that

Recalling that

it suffices to prove that

for 
To that end, we show that

which establishes the assertion by the law of total expectation.

The random variable

maps the event

to

where

if  is odd; and

if  is even.
The proof is completed by \Lem{}~\ref{lemma:APP-low-bound-last-digestion} as
the random variable

maps the event

to
.
\end{proof}
\LongVersionEnd 

\ShortVersion \sloppy \ShortVersionEnd
\begin{lemma} \label{lemma:APP-bound-number-meaningful}
The random variable  is stochastically dominated by
,
where
\LongVersion 
\LongVersionEnd \ShortVersion 
\ShortVersionEnd is a Poisson random variable with parameter
.
Moreover, if  denotes the number of discontinuity points of the coloring
function  in , then

(with probability ).
\end{lemma}
\ShortVersion \par\fussy \ShortVersionEnd
\LongVersion \begin{proof}
Fix

and

and define the random variables

The definition of the APP ensures the following four properties:
\begin{DenseEnumerate}

\item[(P1)]
;

\item[(P2)]
;

\item[(P3)]
, , is stochastically dominated by ; and

\item[(P4)]
, , is bounded from above by the number of (set-wise)
maximal intervals

satisfying

for all .

\end{DenseEnumerate}
The second part of the assertion follows directly from properties (P1) and
(P4).
For the first part, we employ (P1) and (P2) to conclude that

for .
Then, by (P3), it follows that  is stochastically dominated by

for ,
thus it is stochastically dominated by
.
\end{proof}
\LongVersionEnd 

\LongVersion It will be convenient to also consider a generalization of the APP, referred to
as a \emph{rate-varying APP}, in which the fixed rate parameter  is
replaced by a \emph{rate function}

that may vary in time.
This affects the aforementioned iteration termination probability  so
that an odd (resp., even) iteration  that did not end by time
,

(resp.,
),
will now end at time  with probability
.
Given some (fixed)
,
it is straightforward to verify that if the rate function  is
bounded from above by , i.e.,

for all
,
then \Lem{} \ref{lemma:APP-low-bound-last-digestion} and
\ref{lemma:APP-bound-number-meaningful} hold also for rate-varying APPs,
only that in the former, we should replace the equality in

with a  inequality.
\LongVersionEnd 

\begin{IntuitionSpotlight}
APPs are utilized in the analysis conducted in
\Sect{}~\ref{section:analysis-heart} as they capture the behavior of the
stilt-walker algorithm in what can be informally described as ``toggling
situations''.
Such situations turn out to appear in multiple parts of the analysis (see
\Lem{} \ref{lemma:phase-app}, \ref{lemma:bound-time-cost-0-subphase}, and
\ref{lemma:partition-time-line-into-phases}).
\end{IntuitionSpotlight}

\subsection{Analysis under the end-of-input signal assumption}
\label{section:analysis-heart}
Let  be an -point -HSBT of aspect ratio 
and let  and

be the full binary tree and weight function that realize .
Assume without loss of generality that the minimum positive distance in
 is scaled to  so that  is the diameter of
.

Our goal in this section is to establish \Thm{}~\ref{theorem:main} under the
end-of-input signal assumption.\footnote{For the convenience of the reader, \Fig{}~\ref{figure:claim-chart} provides a
schematic overview of the analysis presented in this section.
}
More formally, assume that the online algorithm is signaled at time

\NotationLabel{analysis:end-time}
(the arrival time of the last request in );
upon receiving this signal, the algorithm clears the remaining active
requests by immediately matching across  for every effective vertex

(this is guaranteed as the number of active requests at time  must
be even).
Let

\NotationLabel{analysis:end-cost}
be the space cost of these matching operations and observe that
.
\LongVersion (Although it does not affect our analysis, it is interesting to point out that
 is, in fact, the cost of an optimal matching of the remaining
requests.)
\LongVersionEnd We start the analysis with the following ``warmup''
\LongVersion observation regarding the operation of the stilt-walker algorithm.
\LongVersionEnd \ShortVersion observation.
\ShortVersionEnd 

\begin{observation*}
Consider an internal vertex  with children .
The design of  ensures that:
\begin{DenseEnumerate}

\item \label{item:independence}
the random variable

is independent of the coin tosses of all vertices  (including
);

\item \label{item:match-on-top}
 can match on top of  only when  is odd; and

\item \label{item:match-across}
if  matched across or on top of  at time , then
\LongVersion , , and  are not odd immediately following time
, i.e.,
\LongVersionEnd 
for infinitesimally small
.

\end{DenseEnumerate}
\end{observation*}
\begin{proof}
To establish property \ref{item:independence}, notice that the coin tosses of
vertex  determine the decisions of  to match across .
Matching across  decreases  by , hence it does not
affect its parity.
\LongVersion \par \LongVersionEnd
Property \ref{item:match-on-top} is proved by recalling that matching on top
of  at time  is realized by matching a request located in some leaf

to a request located in some leaf
.
Since , it must belong to the stilt in 
whose foot is  which establishes the assertion by the definition of
.
\LongVersion \par \LongVersionEnd
Finally, observe that property \ref{item:match-across} holds trivially if 
matched across  at time  because this means that

and thus,
.
Otherwise, if  matched on top of  at time , then

which means that

and

for some
.
This also means that  matched on top of  at time , therefore
.
\end{proof}

\begin{IntuitionSpotlight}
A key ingredient in the analysis of 's competitive ratio is an alternative
method for measuring its time and space cost on a \emph{per-vertex basis}.
This is facilitated by the definitions of time and space potentials for each
internal vertex .
\end{IntuitionSpotlight}

\paragraph{Time and space potentials.}
Consider some internal vertex  with children  and some
.
The \emph{time potentials} of , denoted 
and ,
capture the contributions of  to  and
, respectively, in a certain time interval.
They are defined by setting

\NotationLabel{analysis:tau}
\NotationLabel{analysis:adv-tau}
in other words, a  amount is deposited into  whenever  and into  whenever 
for .

\ShortVersion \sloppy \ShortVersionEnd
The \emph{space potentials} of , denoted 
\NotationLabel{analysis:sigma}
and ,
\NotationLabel{analysis:adv-sigma}
capture the contributions of  to  and
, respectively, in a certain time interval.
An amount of  is deposited into  whenever  matches
across ;
an amount of  is deposited into  whenever
 matches across or on top of .
In other words, given two requests  with

and
,
if  matches requests  and ,
then we deposit an amount of  into  for
;
if  matches requests  and ,
then we deposit an amount of  into  for every internal
vertex  along the unique path connecting  and  in .
Let  and  be the
total amount deposited into  and , respectively,
during the time interval .
\ShortVersion \par\fussy \ShortVersionEnd

For clarity of the exposition, we often write ,
, , and
 instead of the aforementioned notations.
We also extend the definition of these four notations from intervals to
collections of disjoint intervals in the natural manner.
\Thm{}~\ref{theorem:main} is established by proving the following three
lemmas.

\begin{IntuitionSpotlight}
\Lem{}~\ref{lemma:expressing-costs-by-potentials} allows us to
express the time and space costs by means of the per-vertex potentials.
\Lem{}~\ref{lemma:bounding-time-potential} then means that we can
bound the time potential of  under  by the time and space potentials of
 under , charging the extra  on the additive term of the
competitive ratio, whereas \Lem{}~\ref{lemma:bounding-space-potential} means
that we can bound the space potential of  under  by its time
potential.
\end{IntuitionSpotlight}

\begin{lemma} \label{lemma:expressing-costs-by-potentials}
There exists some

such that the time potentials satisfy

(recall that  denotes the height of ).
The space potentials satisfy

(recall that the parameter  is set in \Thm{}~\ref{theorem:main}).
\end{lemma}

\begin{lemma} \label{lemma:bounding-time-potential}
For every , it holds that
.
\end{lemma}

\begin{lemma} \label{lemma:bounding-space-potential}
For every , it holds that
.
\end{lemma}

\begin{proof}[Proof of \Lem{}~\ref{lemma:expressing-costs-by-potentials}]
We first note that
\begin{MathMaybe}
\tCost_{\A}(R, \mathcal{T})
=
\sum_{v \in T} \int_{0}^{\EndTime} \Indicator(v \in \Heads(t)) \, d t \, .
\end{MathMaybe}
Indeed, as each leaf contains at most one active request, an active request
 is accounted for in exactly one term of the sum in the
RHS of the equation, that is, the term corresponding to the head
of the stilt whose foot is .
Since an internal vertex is effective at time  if and only if its two
children are in , the last equation can be rewritten as

On the other hand, the inequality
\begin{MathMaybe}
\tCost_{\adv{\A}}(R, \mathcal{T})
\geq
\frac{1}{h} \cdot \sum_{v \in T} \int_{0}^{\EndTime} \Indicator(v \in
\adv{\Odd}(t)) \, d t
\end{MathMaybe}
holds since each active request under  is accounted for in at
most  terms of the sum in the RHS of the inequality, therefore

The first part of the assertion is established by observing that

if and only if
,
hence we can fix
\begin{MathMaybe}
\zeta
=
\int_{0}^{\EndTime} \Indicator(r \in \Odd(t)) \, d t
=
\int_{0}^{\EndTime} \Indicator(r \in \adv{\Odd}(t)) \, d t \, .
\end{MathMaybe}

The contribution to  of matching requests 
and  by  is ;
this is also its contribution to the space potentials , hence

The contribution to

of matching requests  and  by  is ,
whereas since

is an -HSBT (recall that
),
its contribution to the space potentials  is bounded
from above by
,
hence,

which completes the proof.
\end{proof}

\paragraph{Convenient notation.}
The remainder of this section is dedicated to the proofs of \Lem{}
\ref{lemma:bounding-time-potential} and \ref{lemma:bounding-space-potential}.
To this end, we fix some internal vertex  with children
 and  which facilitates switching to a shorter and simpler
notation:
Denote
,
,
, and
.
Given some time , we write for short

\NotationLabel{analysis:variable-X-i}
\NotationLabel{analysis:variable-adv-X-i}
for  and

\NotationLabel{analysis:variable-X}
\NotationLabel{analysis:variable-adv-X}
\LongVersion Notice that
\LongVersionEnd \ShortVersion As
\ShortVersionEnd 
and
,
\LongVersion thus
\LongVersionEnd \ShortVersion we get
\ShortVersionEnd 

It will be convenient to also define
\begin{MathMaybe}
Y_{i}(t)
=
|\{ \rho \in R \mid \Location(\rho) \in \Leaves(u_{i}) \land
\aTime(\rho) \leq t \}| \pmod{2}
\end{MathMaybe}
\NotationLabel{analysis:variable-Y-i}
for  and
\begin{MathMaybe}
Y(t) = Y_{1}(t) \xor Y_{2}(t) \, ,
\end{MathMaybe}
\NotationLabel{analysis:variable-Y}
observing that the parity of the number of times 
matched on top of  (resp., ) up to time  equals

(resp.,
).

\paragraph{Phases and subphases.}
We partition the time line

into \emph{phases} (defined with respect to ),
where each phase is a time interval that starts when the previous phase
ends (or at time  if this is the first phase) and ends when  matches on
top of  (or at time  if this is the last phase).
A crucial observation is that this partition is fully determined by the coin
tosses of  (namely, the ancestors of ) independently of the
coin tosses of .

We further partition every phase

of  into \emph{subphases}, where each subphase is
a time interval that starts when the previous subphase ends (or at time
 if this is the first subphase of ) and ends when
 matches across \emph{or} on top of  (or at time  if this
is the last subphase of ).
Notice that matching operations across  performed by  (fully
determined by the coin tosses of ) can occur at the midst of a subphase.

\begin{lemma} \label{lemma:phase-app}
For every phase

of , it holds that
.
\end{lemma}
\LongVersion \begin{proof}
We investigate the dynamics of

and

that take values in

(an illustration is provided in \Fig{}~\ref{figure:phase-app}).
Observe that a new request arriving in , ,
flips  and  without affecting  and .
While  is affected only by new request arrivals, the dynamic
of  is tied to the actions of  too.
Specifically,  can match across  (recall that  does not match on
top of  in the midst of phase ) only when

and if

throughout the infinitesimally small time interval ,
then  matches across  at time  with probability

(depending solely on the coin tosses of ),
in which case

flips to
.
Moreover, we know that
.

Let
.
We color the times in  using the coloring function

by setting

The key observation now is that the times at which  matches across 
can be viewed as the meaningful alternation times of an APP 
defined over the time interval  with coloring function  and
rate .
(Notice that the role of  in the validity of this observation
is simply to adjust the dynamic of , starting with
,
to the APP framework in which the first digested color is defined to be .)
Taking  to be the random variable counting the number of meaningful
alternation times in  and  to be its total digestion, we
conclude that

and
.
The assertion follows by
\Lem{}~\ref{lemma:APP-digestion-equals-number-meaningful}.
\end{proof}
\LongVersionEnd 

\def\FigurePhaseApp{
\LongVersion
\begin{figure}
\LongVersionEnd
\ShortVersion
\begin{figure}[h]
\ShortVersionEnd
\begin{center}
\includegraphics[width=\textwidth]{phase-app}
\end{center}
\caption{ \label{figure:phase-app}
Phase  with time progressing from left to right, assuming that .
Bottom rows:
the dark gray and light gray intervals represent the times  at which

and
,
respectively.
Top row:
the dark gray, light gray, and white intervals represent the times  at
which
,
,
and
,
respectively.
The vertical arrows represent the times at which  matches across  and
the horizontal two-sided arrows depict the time intervals that contribute to
, i.e., when
.
Notice that towards 's end, we must have

unless  is the last phase.
}
\end{figure}
}
\LongVersion \FigurePhaseApp{}
\LongVersionEnd 

Fixing the coin tosses in  and thus, fixing the partition of
 into phases, we can apply \Lem{}~\ref{lemma:phase-app}
to the each individual phase, thus establishing
\Lem{}~\ref{lemma:bounding-space-potential} by the linearity of expectation.
The remainder of this section is dedicated to proving
\Lem{}~\ref{lemma:bounding-time-potential}.
The first step towards achieving this goal is to bound the time potential of
 per subphase based on the following subphase classification.

\paragraph{- and -subphases.}
Fix some subphase  of .
Notice that matching across  (by ) does not affect , , nor does it change
.
Thus, there exists some

such that

for all
;
in what follows, we distinguish between two types of subphases:
\emph{-subphases}, for which , and \emph{-subphases}, for which
.

\begin{observation} \label{observation:bound-time-cost-1-subphase}
If  is a -subphase, then
.
\end{observation}
\begin{proof}
Recall that

and
.
The assertion follows by the definition of a -subphase ensuring that for
every
,
if
,
then
.
\end{proof}

\begin{lemma} \label{lemma:bound-time-cost-0-subphase}
If  is a -subphase, then
.
\end{lemma}
\ShortVersion \sloppy \ShortVersionEnd
\begin{proof}
We investigate the dynamics of

and

that take values in

(an illustration is provided in \Fig{}~\ref{figure:subphase-app}).
By the definition of a -subphase, at any time
,
either

or
;
we refer to the former (resp., latter) as an \emph{agreement} (resp.,
\emph{disagreement}) state of  and .

Observe that a new request arriving in , ,
flips  and  without affecting  and .
While  is affected only by new request arrivals
(recall that  does not match across or on top of  in the midst of
subphase ), the dynamic of  is tied to the actions of
 too.
Specifically,  can match across  (recall that  does not match on
top of  in the midst of subphase ) only when

and if 

throughout the infinitesimally small time interval
,
then  matches across  at time  with probability

(depending solely on the coin tosses of ), in which case 
flips to
,
thus toggling the agreement/disagreement state.

Define the functions

and

as follows:

We color the times in  using the coloring function

by setting

if the subphase starts in an agreement
\LongVersion state; and
\LongVersionEnd \ShortVersion state; and
\ShortVersionEnd 
\LongVersion if the subphase starts in a disagreement state.
\LongVersionEnd \ShortVersion otherwise.
\ShortVersionEnd The key observation now is that the times at which  matches across 
can be viewed as the meaningful alternation times of an APP 
defined over the time interval  with coloring function  and
rate .
(Notice that the role of the  vs.\ 
distinction in the validity of this observation is simply to adjust the
dynamic of , starting in an agreement/disagreement state, to
the APP framework in which the first digested color is defined to be .)

Taking  to be the total digestion of , we notice that
.
Moreover, the construction of the coloring function  ensures that
,
where  and  are the total - and -volumes of ,
respectively.
The assertion follows by \Lem{}\
\ref{lemma:APP-digestion-equals-number-meaningful} and
\ref{lemma:APP-bound-number-meaningful}.
\end{proof}
\ShortVersion \par\fussy \ShortVersionEnd

\def\FigureSubphaseApp{
\LongVersion
\begin{figure}
\LongVersionEnd
\ShortVersion
\begin{figure}[h]
\ShortVersionEnd
\begin{center}
\includegraphics[width=\textwidth]{subphase-app}
\end{center}
\caption{ \label{figure:subphase-app}
Subphase  with time progressing from left to right, assuming that the
subphase starts in an agreement state.
Bottom rows:
the dark gray and light gray intervals represent the times  at which

and
,
respectively.
Top row:
the dark gray, light gray, and white intervals represent the times  at
which
,
,
and
,
respectively.
The vertical arrows represent the times at which  matches across  and
the horizontal two-sided arrows depict the time intervals that contribute to
, i.e., when
.
Notice that by the definition of , times  at which

(marked as white intervals in the top row)
also contribute to , but this contribution is ignored by our
analysis.
}
\end{figure}
}
\LongVersion \FigureSubphaseApp{}
\LongVersionEnd 

\paragraph{- and -phases.}
Phase  of  is said to be a \emph{-phase} (resp., a
\emph{-phase}) if it starts with a -subphase (resp., a -subphase).
Let

\NotationLabel{analysis:P-0}
(resp.,
)
\NotationLabel{analysis:P-1}
be the set of -phases (resp., -phases) of
.
Using \Obs{}~\ref{observation:bound-time-cost-1-subphase} and
\Lem{}~\ref{lemma:bound-time-cost-0-subphase}, we establish
\Lem{}~\ref{lemma:bounding-time-potential} (our goal in the remainder of this
section) by proving the following inequalities:

\Lem{}~\ref{lemma:0-and-1-phases} (a combination of
\Obs{}~\ref{observation:bound-time-cost-1-subphase} and
\Lem{}~\ref{lemma:bound-time-cost-0-subphase} essentially) plays an important
role in the desired proofs.

\begin{lemma} \label{lemma:0-and-1-phases}
If  is a -phase, then
\begin{MathMaybe}
\Expect_{v} \left[ \tau(\phi) \right]
\leq
\adv{\tau}(\phi) + 2 \adv{\sigma}(\phi) + w(v) \, ;
\end{MathMaybe}
if  is a -phase, then
\begin{MathMaybe}
\Expect_{v} \left[ \tau(\phi) \right]
\leq
\adv{\tau}(\phi) + 2 \adv{\sigma}(\phi) \, .
\end{MathMaybe}
\end{lemma}
\LongVersion \begin{proof}
Let

be the set of -subphases of  for .
If

and
,
then we can employ \Obs{}~\ref{observation:bound-time-cost-1-subphase} to
conclude that
.
If

and
,
then we can employ \Lem{}~\ref{lemma:bound-time-cost-0-subphase} to conclude
that
.

Since all but the last subphases of  end when  matches across
or on top of , it follows by the definition of  that
.
Therefore, if
,
then we can employ \Obs{}~\ref{observation:bound-time-cost-1-subphase} and
\Lem{}~\ref{lemma:bound-time-cost-0-subphase} to conclude that

where the last transition holds since

implies that
.
The assertion follows.
\end{proof}
\LongVersionEnd 

Fixing the coin tosses in  (and thus, fixing the partition of
 into phases), we can apply \Lem{}~\ref{lemma:0-and-1-phases}
to each individual -phase, hence obtaining
(\ref{equation:target-bound-1-phases}) by the linearity of expectation.

\begin{IntuitionSpotlight}
It remains to establish (\ref{equation:target-bound-0-phases}) which turns out
to be more demanding:
for -phases , the upper bound on  promised by
\Lem{}~\ref{lemma:0-and-1-phases} includes an additive  term and we have
to make sure that it does not dominate the  and
 terms too often.
This is done via a classification of the phases with respect to
their starting time.
\end{IntuitionSpotlight}

\paragraph{Early and late phases.}
Recall the definition of

and let  be the smallest

such that
.
\NotationLabel{analysis:late-time}
Phase  with starting time  is
\LongVersion said to be an \emph{early}
phase if

and a \emph{late} phase if
.
\LongVersionEnd \ShortVersion called \emph{early} if

and \emph{late} if
.
\ShortVersionEnd (Intuitively, this means that when an early phase starts, we still have more
than  time units of

and more than  time units of
.)
Let

\NotationLabel{analysis:P-early}
and

\NotationLabel{analysis:P-late}
be the sets of early and late phases, respectively.
Let

\NotationLabel{analysis:K}
be the number of discontinuity points of  in the interval
.

We would like to take a closer look at the partition of  into
phases.
\LongVersion To that end, consider
\LongVersionEnd \ShortVersion Consider
\ShortVersionEnd some phase  with starting time  and end
time .
Fixing

for some
,
the end time  is a random variable fully determined by the coin tosses
in  after time .
\LongVersion An important property of this random variable is cast in the following lemma
(together with
two other important properties of the partition of  into
phases).
\LongVersionEnd 

\begin{lemma} \label{lemma:partition-time-line-into-phases}
The partition of  into phases satisfies the following three
properties: \\
(P1)
if
,
then
; \\
(P2)
; and \\
(P3)

with an exponentially vanishing upper tail.
\end{lemma}
\LongVersion \begin{proof}
We investigate the dynamics of

and

(an illustration is provided in \Fig{}~\ref{figure:time-line-app}).
A new request arriving in  flips  and .
While  is affected only by new request arrivals, the dynamic of  is tied
to the actions of  too.
Specifically, the design of the stilt-walker algorithm ensures that  can
match on top of  only when  (recall that matching across  does
not affect the partition of  to phases).
Suppose that  throughout the infinitesimally small time
interval
;
let  be the stilt in  to which  belongs for all 
and let

be the head of .
Then  matches across  and on top of  at time  with probability

(depending solely on the coin tosses of ), in which case  flips to
.
Since  is an ancestor of , we know that
.

We color the time line using the coloring function

by setting

(note that , whose preimage under  is empty, is included in the range
of  for the sake of compatibility with the APP framework).
The key observation now is that the times at which  matches on top of 
can be viewed as the meaningful alternation times of a rate-varying APP
 defined over the time interval  with
coloring function  and rate function bounded from above by  (recall that a rate-varying APP is a generalization of an APP defined in
the end of \Sect{}~\ref{section:alternating-Poisson-process} of the full
version).


\sloppy
Taking  to be the digestion of the iteration in
 that starts at time
,
we notice that
;
recalling that the definition of  guarantees that

for every
,
we obtain property (P1) by applying (the rate-varying version of)
\Lem{}~\ref{lemma:APP-low-bound-last-digestion} to .
Property (P2) holds simply by applying
\Lem{}~\ref{lemma:APP-bound-number-meaningful} to the -restriction of .
To obtain property (P3), we consider the -restriction
of , denote its number of meaningful alternation times by
, and observe that  is stochastically dominated by ;
the property then follows by \Lem{}~\ref{lemma:APP-bound-number-meaningful}
since
.
\end{proof}
\par\fussy
\LongVersionEnd 

\def\FigureTimeLineApp{
\LongVersion
\begin{figure}
\LongVersionEnd
\ShortVersion
\begin{figure}[h]
\ShortVersionEnd
\begin{center}
\includegraphics[width=\textwidth]{time-line-app}
\end{center}
\caption{ \label{figure:time-line-app}
Interval  with time progressing from left to right.
The dark gray and light gray intervals represent the times  at which

and
,
respectively.
The solid vertical arrows represent the times at which  matches on top of
.
The dashed vertical arrow represent time .
}
\end{figure}
}
\LongVersion \FigureTimeLineApp{}
\LongVersionEnd 

\begin{corollary} \label{corollary:low-bound-early-phase-opt}
If
 is a -phase that starts at time
,
then
.
\end{corollary}
\begin{proof}
As  is a random variable fully determined by the coin tosses in
 after time ,  and 
are also random variables fully determined by the coin tosses in
 after time .
Since

and since  starts with a -subphase  during which

implies
,
the assertion follows from
\Lem{}~\ref{lemma:partition-time-line-into-phases}(P1), recalling that if
 contains any subphase other than  (in particular, a
-subphase during which

does not imply
),
then
.
\end{proof}

We are now ready to establish (\ref{equation:target-bound-0-phases}).
This is done by defining

and

to be the sets of early and late -phases, respectively, and proving the
following two lemmas.

\begin{lemma} \label{lemma:bound-early-phases}
.
\end{lemma}
\LongVersion \begin{proof}
\Lem{}~\ref{lemma:partition-time-line-into-phases}(P2) ensures
that
.
Let

be the sequence of early -phases, where, for the sake of the analysis, we
introduce a suffix of empty \emph{dummy} phases so that each dummy phase
, ,
starts and ends at some arbitrary dummy time
,
thus ensuring that
.

Fix some

and let  and  be the random variables that capture the starting
time and end time of .
We argue that

for any  in the support of .
This clearly holds if  is an empty dummy phase (which means that
),
so assume that
.
Consider the random variable

that maps the event

(defined over the coin tosses in )
to
.
By \Lem{}~\ref{lemma:0-and-1-phases}, the latter satisfies
.
Therefore,

where the last transition follows from
\Cor{}~\ref{corollary:low-bound-early-phase-opt},
thus establishing (\ref{equation:bound-single-phase-conditional}) by the law
of total expectation.

Consider the random variable

that maps the event

(defined over the coin tosses in )
to
.
Using the bound provided for the latter by
(\ref{equation:bound-single-phase-conditional}) and applying the law of total
expectation, we conclude that

Therefore, by the linearity of expectation, we derive

which establishes the assertion.
\end{proof}
\LongVersionEnd \ShortVersion \sloppy
\begin{proof}[Proof (sketch).]
Consider an early -phase  and let  and  be the random
variables that capture its starting time and end time, respectively.
By \Lem{}~\ref{lemma:0-and-1-phases}, we know that
.
Employing \Cor{}~\ref{corollary:low-bound-early-phase-opt}, we can
show that
,
hence, applying the law of total expectation, we prove that
.
Then, we employ \Lem{}~\ref{lemma:partition-time-line-into-phases}(P2) to
bound the number of early -phases, so we can use the linearity of
expectation to establish the assertion.
\end{proof}
\par\fussy
\ShortVersionEnd 

\begin{lemma} \label{lemma:bound-late-phases}
.
\end{lemma}
\LongVersion \begin{proof}
Conditioned on ,
\Lem{}~\ref{lemma:0-and-1-phases} guarantees that

By \Lem{}~\ref{lemma:partition-time-line-into-phases}(P3),

with an exponentially vanishing upper tail, thus

which establishes the assertion.
\end{proof}
\LongVersionEnd \ShortVersion \begin{proof}[Proof (sketch).]
\Lem{}~\ref{lemma:0-and-1-phases} guarantees that 

for each phase .
The assertion follows by
\Lem{}~\ref{lemma:partition-time-line-into-phases}(P3) that practically
provides a constant bound on
.
\end{proof}
\ShortVersionEnd 

\LongVersion \subsection{Lifting the end-of-input signal assumption}
\label{section:lift-end-of-input-assumption}
We now turn to lift the end-of-input signal assumption, showing that
\Thm{}~\ref{theorem:main} holds also without it.
Recall that

denotes the arrival time of the last request in  and let

and

be the set of remaining active requests and the set of effective vertices at
time , respectively.
The analysis presented in \Sect{}~\ref{section:analysis-heart} relies on the
assumption that upon receiving the end-of-input signal at time , the
algorithm immediately clears all the requests in  by matching across every
vertex in  which contributes

to the space cost of  (this contribution to the space cost of  is
taken into account in \Sect{}~\ref{section:analysis-heart}).

An examination of the matching policy of the stilt-walker algorithm reveals
that in reality, the requests in  are indeed cleared by matching across the
vertices in , only that these matching operations are not performed
immediately at time , but rather at slightly later (random) times,
thus introducing an additional contribution to the time cost of .
Specifically, taking

to be the supporting requests of some effective vertex
,
notice that on expectation,  matches across  at time

which accounts for an additional contribution of a  term to the
algorithm's expected time cost.
Summing over all vertices in , we conclude that by adding

to the  term in \Thm{}~\ref{theorem:main}, we can lift the end-of-input
assumption as promised.
\LongVersionEnd 

\LongVersion \section{A fixed penalty for clearing requests}
\label{section:fixed-penalty}
In this section, we consider the online \emph{MPMDfp} problem:
a variant of MPMD in which the algorithm is allowed to clear any request

at time

without matching it to another request, incurring a fixed penalty

(a parameter of the problem), on top of the time cost

of ,
that adds to its total cost.
Notice that in contrast to MPMD, the MPMDfp problem is well defined also for
odd values of .

\begin{theorem}
There exists a randomized online MPMDfp algorithm for  whose
competitive ratio is
,
where  is the number of points in the underlying metric space and 
is its aspect ratio.
\end{theorem}
\begin{proof}
Consider the underlying -point metric space

and let

and

be the minimum and maximum distances between any two distinct points in
, respectively, so that the aspect ratio of  is
.
Assume for the time being that the penalty  satisfies
.

Let

be the metric space defined by setting

for every

and
.
The assumption that

implies that the aspect ratio of  is proportional to
.
Let
,
where , , is defined by setting

and
.

We construct an online MPMDfp algorithm  with the desired
competitive ratio from the stilt-walker algorithm  as follows.
Algorithm  simulates  on

and handles the requests in  according to the actions of  on the
requests in .
Specifically, for every
,
if  matches  to some request , located in
,
at time , then  matches  to  at time ;
if  matches  to some request , located in
, at time , then  clears  without
matching it (paying the fixed -penalty) at time .

The design of  and the fact that

for every 
guarantee that

Moreover, the construction of  and 
ensures that if  is an optimal offline MPMD algorithm and
 is an optimal offline MPMDfp algorithm, then

since  can project the actions of  on each
side of , matching  to  whenever
 clears  without matching it.
The assertion follows since the stilt-walker algorithm  is -competitive for the MPMD problem.

Now, if
,
then an MPMDfp (online or offline) algorithm is always better off clearing the
requests by paying the fixed penalty than by matching them.
Therefore, in this case, the MPMDfp problem over  can be
decomposed into  independent instances of the MPMDfp over a -point
metric space.
Each such instance (essentially a repeated version of the ski rental problem)
admits an -competitive online algorithm, thus so does the whole
problem.

It remains to consider the case where
.
In this case, we construct the metric space  slightly
differently:
first employ \Thm{}~\ref{theorem:HSBT} to probabilistically embed
 in a -HSBT ;
then, take two copies of , call them  and , and connect them
so that their roots become the children of a new root , extending
the weight function  by setting
.
Notice that the resulting metric space is also a -HSBT whose point set can be renamed

so that  is a leaf of  for every

and
.
The rest of the construction of  is unchanged.

Although the aspect ratio of the metric space  in this
case may be large (as large as
),
notice that the height of the underlying -HSBT is
still ,
where  is the aspect ratio of .
This establishes the assertion by recalling that the  term in the
competitive ratio of the stilt-walker algorithm comes from an upper bound on
the height of its HSBT.
\end{proof}
\LongVersionEnd 

\LongVersion \section{The deterministic version of the stilt-walker algorithm}
\label{section:specific-lower-bound}
In this section, we consider the deterministic version of the stilt-walker
algorithm, denoted , obtained by replacing the -rate
exponential timer maintained at each internal vertex  with
a deterministic -timer.
In other words, the matching policy of  is similar to that of 
with one difference:
If the last time  matched across  was at time 
(take  if  still has not matched across ), then the next
time it matches across  is the minimum  that satisfies


\begin{theorem} \label{theorem:specific-lower-bound}
The competitive ratio of  on -point
-HSBTs
is .
\end{theorem}
\begin{proof}
Let  be some large power of  and let  be an -leaf perfect binary
tree (with all leaves at depth ).
Let

be the weight function defined by setting

for every leaf ; and

for every internal vertex  of depth .
Consider the HSBT

and notice that the distance between any two distinct points in 
is .
We name some of the internal vertices and subtrees of  according to the
labels in \Fig{}~\ref{figure:specific-lower-bound}.

Let  denote the benchmark offline algorithm and take 
to be a small positive real.
For every subtree , , fix some arbitrary leaf 
and consider the following scenario  (refer to
\Fig{}~\ref{figure:specific-lower-bound} for an illustration):

\begin{DenseItemize}

\item
 requests arrive at time  at leaves  and  (one each).
 immediately matches these requests.
Following that, the sole effective vertex of  is  with supporting
leaves  and .

\item
 requests arrive at time

at leaves
, , , and  (one each).
Following that, the effective vertices of  are: \\
 with supporting leaves  and ; \\
 with supporting leaves  and ; and \\
 with supporting leaves  and .

\item
The timer of  expires at time  and  matches (across
) the active requests hosted at leaves  and .

\item
 requests arrive at time

at leaves
, , , and  (one each).
Both  and  immediately match the request pairs hosted at
 and ;
 also immediately matches the request pairs hosted at 
and .
Following that, the effective vertices of  are: \\
 with supporting leaves  and ; and \\
 with supporting leaves  and .

\end{DenseItemize}

Consider the subscenario  induced on  by the time interval
.
The key observation is that at the beginning of ,  had
 active requests located at leaves whose LCA is the the root of 
(), whereas at its end,  has  active requests located at leaves
whose LCAs are the two depth  vertices ( and ).
On the other hand,  started and ended subscenario  with
no active requests, paying a total cost of  during that time
period.

Subscenarios analogous to  are now applied in a recursive manner to
the subtrees rooted at the depth  vertices of , for
.
This results in  having active requests at exactly  distinct
leaves of ;
to clear all of them,  will have to pay  in space cost.
On the other hand, the total cost payed by  during all these
applications is  which can be made arbitrarily small.
Adding the  space cost payed by  at time  for matching
the first two requests (across ), we conclude that the competitive
ratio of  is , as promised.
\end{proof}

\def\FigureSpecificLowerBound{
\begin{figure}
\begin{center}
\includegraphics[width=\textwidth]{specific-lower-bound}
\end{center}
\caption{ \label{figure:specific-lower-bound}
The perfect binary tree  and scenario  at times of interest
featured on the left.
For every  and for every time ,
a diamond shape depicts a request arriving at leaf  at time ;
a vertical segment depicts an active request under  at leaf  at
time ; and
a horizontal segment depicts an active request under  at leaf
 at time .
}
\end{figure}
}
\FigureSpecificLowerBound{}

\LongVersionEnd 

\clearpage
\renewcommand{\thepage}{}

\bibliographystyle{abbrv}
\bibliography{references}

\LongVersion \clearpage
\pagenumbering{roman}
\appendix

\renewcommand{\theequation}{A-\arabic{equation}}
\setcounter{equation}{0}

\begin{center}
\textbf{\large{APPENDIX}}
\end{center}

\section{Probabilistic embedding of arbitrary metric spaces in HSBTs}
\label{appendix:embedding-in-HSBT}
Our goal in this section is to prove \Thm{}~\ref{theorem:HSBT}.
The main ingredient in this proof is the following celebrated theorem of
Fakcharoenphol et al.~\cite{FakcharoenpholRT2004}.

\begin{theorem}[\cite{FakcharoenpholRT2004}] \label{theorem:FRT}
Consider some -point metric space

and let  be the set of all -HSTs over  with distance
functions  that dominate  in the sense that

for every .
There exists a probability distribution  over  such
that

for every .
Moreover, the probability distribution  can be sampled
efficiently.
\end{theorem}

Observe that by the definition of HSTs, the rooted trees realizing the
-HSTs promised by \Thm{}~\ref{theorem:FRT} are of height
,
where

is the aspect ratio of the metric space
.
These rooted trees have arbitrary degrees, whereas \Thm{}~\ref{theorem:HSBT}
requires rooted trees with degrees at most .
We resolve this obstacle with the help of the following lemma, proved by
Patt-Shamir \cite{PattShamir2015}.

\begin{lemma}[\cite{PattShamir2015}] \label{lemma:Boaz}
Consider some -leaf rooted tree .
There exist
a (rooted) full binary tree  and an injection

such that \\
(1)
 is an ancestor of  in  if and only if  is an ancestor of
 in ; \\
(2)
, where
 denotes the depth operator in tree ; and \\
(3)
.
\end{lemma}

Consider some -point tree metric space  in the support of the
probability distribution promised by \Thm{}~\ref{theorem:FRT} and let  and

be the full binary tree and injection obtained by applying
\Lem{}~\ref{lemma:Boaz} to .
We construct a weight function

on the vertices of  by first setting

for every
,
and then fixing

for every
.
\Lem{}~\ref{lemma:Boaz} guarantees that
 is a -HSBT.
Taking  and  to be the distance functions of  and
, respectively, we observe that

for every two points  in the metric space(s), thus establishing
\Thm{}~\ref{theorem:HSBT}.

\LongVersionEnd 

\clearpage
\renewcommand{\thepage}{}

\ShortVersion 

\begin{center}
\textbf{\large{FIGURES}}
\end{center}

\PseudocodeStiltWalker{}

\FigureApp{}



\FigureSubphaseApp{}





\ShortVersionEnd 

\begin{figure}[h]
\begin{center}
\begin{tikzpicture}[box/.style={draw, rectangle, rounded corners, minimum
width=2cm, minimum height=1cm, text centered, draw=black, fill=red!30},
arrow/.style={thick,->,>=stealth}]

\node (chart-theorem-main)
[box]
{\Thm{}~\ref{theorem:main}};

\node (chart-lemma-expressing-costs-by-potentials)
[box, left=of chart-theorem-main]
{\Lem{}~\ref{lemma:expressing-costs-by-potentials}};

\node (chart-lemma-bounding-time-potential)
[box, below=of chart-theorem-main]
{\Lem{}~\ref{lemma:bounding-time-potential}};

\node (chart-lemma-bounding-space-potential)
[box, right=of chart-theorem-main]
{\Lem{}~\ref{lemma:bounding-space-potential}};

\node (chart-lemma-phase-app)
[box, below=of chart-lemma-bounding-space-potential]
{\Lem{}~\ref{lemma:phase-app}};

\node (chart-equation-target-bound-0-phases)
[box, below=of chart-lemma-bounding-time-potential]
{Eq.~\ref{equation:target-bound-0-phases}};

\node (chart-equation-target-bound-1-phases)
[box, right=of chart-equation-target-bound-0-phases]
{Eq.~\ref{equation:target-bound-1-phases}};

\node (chart-lemma-bound-late-phases)
[box, below=of chart-equation-target-bound-0-phases]
{\Lem{}~\ref{lemma:bound-late-phases}};

\node (chart-lemma-bound-early-phases)
[box, left=of chart-lemma-bound-late-phases]
{\Lem{}~\ref{lemma:bound-early-phases}};

\node (chart-lemma-partition-time-line-into-phases)
[box, below=of chart-lemma-bound-late-phases]
{\Lem{}~\ref{lemma:partition-time-line-into-phases}};

\node (chart-equation-low-bound-early-phase-opt)
[box, left=of chart-lemma-partition-time-line-into-phases]
{\Cor{}~\ref{corollary:low-bound-early-phase-opt}};

\node (chart-lemma-0-and-1-phases)
[box, right=of chart-lemma-partition-time-line-into-phases]
{\Lem{}~\ref{lemma:0-and-1-phases}};

\node (chart-lemma-bound-time-cost-0-subphase)
[box, below=of chart-lemma-0-and-1-phases]
{\Lem{}~\ref{lemma:bound-time-cost-0-subphase}};

\node (chart-observation-bound-time-cost-1-subphase)
[box, left=of chart-lemma-bound-time-cost-0-subphase]
{\Obs{}~\ref{observation:bound-time-cost-1-subphase}};

\draw[arrow]
(chart-lemma-expressing-costs-by-potentials.east) --
(chart-theorem-main.west);

\draw[arrow]
(chart-lemma-bounding-time-potential.north) --
(chart-theorem-main.south);

\draw[arrow]
(chart-lemma-bounding-space-potential.west) --
(chart-theorem-main.east);

\draw[arrow]
(chart-lemma-phase-app.north) --
(chart-lemma-bounding-space-potential.south);

\draw[arrow]
(chart-equation-target-bound-0-phases.north) --
(chart-lemma-bounding-time-potential.south);

\draw[arrow]
(chart-equation-target-bound-1-phases.north) --
(chart-lemma-bounding-time-potential.south);

\draw[arrow]
(chart-lemma-bound-early-phases.north) --
(chart-equation-target-bound-0-phases.south);

\draw[arrow]
(chart-lemma-0-and-1-phases.north) --
(chart-lemma-bound-early-phases.south);

\draw[arrow]
(chart-lemma-bound-late-phases.north) --
(chart-equation-target-bound-0-phases.south);

\draw[arrow]
(chart-equation-low-bound-early-phase-opt.north) --
(chart-lemma-bound-early-phases.south);

\draw[arrow]
(chart-lemma-partition-time-line-into-phases.north) --
(chart-lemma-bound-early-phases.south);

\draw[arrow]
(chart-lemma-partition-time-line-into-phases.west) --
(chart-equation-low-bound-early-phase-opt.east);

\draw[arrow]
(chart-lemma-0-and-1-phases.north) --
(chart-equation-target-bound-1-phases.south);

\draw[arrow]
(chart-observation-bound-time-cost-1-subphase.north) --
(chart-lemma-0-and-1-phases.south);

\draw[arrow]
(chart-lemma-bound-time-cost-0-subphase.north) --
(chart-lemma-0-and-1-phases.south);


\draw[arrow]
(chart-lemma-0-and-1-phases.north) --
(chart-lemma-bound-late-phases.south);

\draw[arrow]
(chart-lemma-partition-time-line-into-phases.north) --
(chart-lemma-bound-late-phases.south);

\end{tikzpicture}
\end{center}
\caption{\label{figure:claim-chart}
A schematic overview of the analysis carried out in
\Sect{}~\ref{section:analysis-heart}, depicting the interdependencies between
its components.
An arrow pointing from A to B indicates that the proof corresponding
to B depends on the statement corresponding to A.
The proofs of \Lem{}\ \ref{lemma:phase-app},
\ref{lemma:bound-time-cost-0-subphase}, and
\ref{lemma:partition-time-line-into-phases} are based on the APP machinery
developed in \Sect{}~\ref{section:alternating-Poisson-process}.
}
\end{figure}

\begin{table}[h]
\begin{center}
{\small
\begin{tabular}{|l|l|l|}
\hline
\textsf{Notation} & \textsf{Definition} & \textsf{Defined on page} \\
\hline

 &
location of  &
\NotationPageRef{model:location} \\

 &
arrival time of  &
\NotationPageRef{model:arrival-time} \\

 &
space cost of  &
\NotationPageRef{model:space-cost-request} \\

 &
time cost of  &
\NotationPageRef{model:time-cost-request} \\

 &
space cost of  &
\NotationPageRef{model:space-cost-instance} \\

 &
time cost of  &
\NotationPageRef{model:time-cost-instance} \\

 &
total cost &
\NotationPageRef{model:total-cost} \\

 &
parent of  &
\NotationPageRef{tree:parent} \\

 &
subtree rooted at  &
\NotationPageRef{tree:subtree} \\

 &
leaves of  &
\NotationPageRef{tree:subtree-leaves} \\

 &
ancestors of  &
\NotationPageRef{tree:ancestors} \\

 &
depth of  &
\NotationPageRef{tree:depth} \\

 &
height of  &
\NotationPageRef{tree:height} \\

 &
least common ancestor of  and  &
\NotationPageRef{tree-lca} \\

 &
set of requests with locations in  active at time  &
\NotationPageRef{alg:active} \\

 &
 for  &
\NotationPageRef{alg:active-root} \\

 &
set of odd vertices at time  &
\NotationPageRef{alg:odd} \\

 &
set of vertices odd under  at time  &
\NotationPageRef{alg:adv-odd} \\

 &
set of stilts induced by the odd vertices at time  &
\NotationPageRef{alg:stilts} \\

 &
set of heads of the stilts in  &
\NotationPageRef{alg:heads} \\

 &
set of effective vertices at time  &
\NotationPageRef{alg:effective} \\

 &
arrival time of the last request &
\NotationPageRef{analysis:end-time} \\

 &
space cost of matching the active requests at time  &
\NotationPageRef{analysis:end-cost} \\

 &
time potential  accumulates during  under  &
\NotationPageRef{analysis:tau} \\

 &
time potential  accumulates during  under  &
\NotationPageRef{analysis:adv-tau} \\

 &
space potential  accumulates during  under  &
\NotationPageRef{analysis:sigma} \\

 &
space potential  accumulates during  under  &
\NotationPageRef{analysis:adv-sigma} \\

 &
 &
\NotationPageRef{analysis:variable-X-i} \\

 &
 &
\NotationPageRef{analysis:variable-adv-X-i} \\

 &
 &
\NotationPageRef{analysis:variable-X} \\

 &
 &
\NotationPageRef{analysis:variable-adv-X} \\

 &
 &
\NotationPageRef{analysis:variable-Y-i} \\

 &
 &
\NotationPageRef{analysis:variable-Y} \\

 &
set of -phases &
\NotationPageRef{analysis:P-0} \\

 &
set of -phases &
\NotationPageRef{analysis:P-1} \\

 &
smallest  s.t.\ 
 &
\NotationPageRef{analysis:late-time} \\

 &
set of early phases &
\NotationPageRef{analysis:P-early} \\

 &
set of late phases &
\NotationPageRef{analysis:P-late} \\

 &
number of discontinuity points of  in  &
\NotationPageRef{analysis:K} \\

\hline
\end{tabular}
} \end{center}
\caption{\label{table:notation}
A table of notations.
}
\end{table}

\end{document}
