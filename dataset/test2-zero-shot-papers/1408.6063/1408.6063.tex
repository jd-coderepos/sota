\documentclass[12pt]{article}
\usepackage{amsmath,amssymb,amsthm,graphicx,mathrsfs,color,fancyhdr,fancybox,setspace}
\usepackage{placeins}
\usepackage{amsfonts}

\renewcommand{\labelenumi}{\theenumi)}
\newtheorem{theorem}{Theorem}[section]
\newtheorem{example}{Example}
\newtheorem{problem}[theorem]{Problem}
\newtheorem{conjecture}[theorem]{Conjecture}
\newtheorem{definition}[theorem]{Definition}
  \newtheorem{observation}[theorem]{Observation}
  \newtheorem{fact}{Fact}[theorem]
  \newtheorem{proposition}[theorem]{Proposition}
\newtheorem{defin}[theorem]{Definition}
\newtheorem{lemma}[theorem]{Lemma}
\newtheorem{corollary}[theorem]{Corollary}
\newtheorem{nt}{Note}
\renewcommand{\thent}{}
\newenvironment{pf}{\medskip\noindent{\bf Proof.  \hspace*{-.4cm}}
       \enspace}{\hfill \qed \newline \medskip}
\newenvironment{defn}{\begin{defin} \em}{\end{defin}}
\newenvironment{pbm}{\begin{problem} \em}{\end{problem}}
\newenvironment{cjc}{\begin{conjecture} \em}{\end{conjecture}}
\newenvironment{note}{\begin{nt} \em}{\end{nt}}
\newenvironment{exa}{\begin{example}
      \em}{\end{example}}
\setlength{\unitlength}{12pt}
\newcommand{\comb}[2]{\mbox{}}
\renewcommand{\labelenumi}{(\theenumi)}

\newcommand{\prooff}[2]{\textbf{Proof of #1 : ~}}

\newenvironment{discussao}{
  \begin{quotation}
    \noindent
    \hrulefill
    \bf
    \begin{list}{}{}
    }{
    \end{list}
    \hrulefill
  \end{quotation}
}



\bibliographystyle{abbrv}

\begin{document}
\parindent 0cm
\parskip .3cm
\thispagestyle{empty}
\begin{center}
\Large
 {\bf Almost every graph is divergent under the biclique operator}
\vfill

\large
Marina Groshaus \footnote{Partially
  supported by UBACyT grant  20020100100754, PICT ANPCyT grant 2010-1970,
  CONICET PIP grant 11220100100310}\footnote{Partially
  supported by Math-Amsud project 14 Math 06}\\
 \normalsize
Universidad de Buenos Aires \\
Departamento de Computaci\'on \\
groshaus@dc.uba.ar \\
\vfill

\large
Andr\'{e} L.P. Guedes\footnotemark[2]\\
 \normalsize
Universidad de Buenos Aires / Universidade Federal do Paran\'{a} \\
Departamento de Computaci\'on / Departamento de Inform\'{a}tica \\
andre@inf.ufpr.br \\
\vfill

\large
Leandro Montero \\
 \normalsize
Universidad de Buenos Aires / Universit\'e Paris-Sud \\
Departamento de Computaci\'on / Laboratoire de Recherche en Informatique \\
lmontero@\{dc.uba.ar/lri.fr\} \\



\normalsize 
\vspace{.5cm}

ABSTRACT
\end{center}

\small A biclique of a graph  is a maximal induced complete bipartite subgraph of . The biclique graph of  denoted by , is the intersection 
graph of all the bicliques of . The biclique graph can be thought as an operator between the class of all graphs. The iterated biclique graph of  
denoted by , is the graph obtained by applying the biclique operator  successive times to . 
The associated problem is deciding whether an input graph converges, diverges or is periodic under the biclique operator when  grows to infinity. 
All possible behaviors were characterized recently and an  algorithm for deciding the behavior of any graph under the biclique operator was also 
given. In this work we prove new structural results of biclique graphs. In particular, we prove that every false-twin-free graph with at least 
 vertices is divergent. These results lead to a linear time algorithm to solve the same problem.

\normalsize

{\bf Keywords:} Bicliques; Biclique graphs; False-twin-free graphs; Iterated graph operators; Graph dynamics
\newpage


\section{Introduction}
Intersection graphs of certain special subgraphs of a general graph have been studied 
extensively. For example, line graphs (intersection graphs of the edges of a graph), 
interval graphs (intersection of intervals of the real line), clique graphs (intersection of cliques of a graph), etc 
\cite{BoothLuekerJCSS1976,BrandstadtLeSpinrad1999,EscalanteAMSUH1973,FulkersonGrossPJM1965, GavrilJCTSB1974, LehotJA1974,McKeeMcMorris1999}.

The \emph{clique graph} of  denoted by , is the intersection graph of the family of all maximal cliques of . Clique graphs were introduced by Hamelink in \cite{HamelinkJCT1968} and characterized by Roberts and Spencer in  \cite{RobertsSpencerJCTSB1971}. 
The computational complexity of the recognition problem of clique graphs had been open for more that 40 years. In  \cite{Alc'onFariaFigueiredoGutierrez2006} they proved that clique graph recognition problem is NP-complete.

The clique graph can be thought as an operator between the class of all graphs. The \emph{iterated clique graph}  is the graph obtained by applying the clique operator  successive times (). Then  is called \emph{clique operator} and it was introduced by Hedetniemi and Slater in \cite{HedetniemiSlater1972}. 
Much work has been done on the scope of the clique operator looking at the different possible behaviors. 
The associated problem is deciding whether an input graph converges, diverges or is periodic under the clique operator when  grows to infinity. 
In general it is not clear that the problem is decidable.  
However, partial characterizations have been given for convergent, divergent and periodic graphs restricted to some classes of graphs. Some of these lead to polynomial time recognition algorithms.
For the clique-Helly graph class, graphs which converge to the trivial graph have been characterized in~\cite{BandeltPrisnerJCTSB1991}.
Cographs, -tidy graphs, and circular-arc graphs  are examples of classes where the different behaviors are 
characterized~\cite{MelloMorganaLiveraniDAM2006,LarrionMelloMorganaNeumann-LaraPizanaDM2004}. Divergent graphs were also considered. For example in 
\cite{Neumann1981}, families of divergent graphs are shown. Periodic graphs were studied in
\cite{EscalanteAMSUH1973,LarrionNeumann-LaraPizanaDM2002}. In particular it is proved that for every integer , there exist periodic graphs with period  and also convergent graphs which converge in  steps. More results about iterated clique graph can be found in 
\cite{LarrionPizanaVillarroel-FloresDM2008,Frias-ArmentaNeumann-LaraPizanaDM2004,LarrionNeumann-LaraGC1997,LarrionNeumann-LaraDM1999,
LarrionNeumann-LaraDM2000,PizanaDM2003}.

A biclique is a maximal bipartite complete induced subgraph. Bicliques have applications in various fields, for example biology: protein-protein interaction networks~\cite{Bu01052003}, 
social networks: web community discovery~\cite{Kumar}, genetics~\cite{Atluri}, medicine~\cite{Niranjan}, information theory~\cite{Haemers200156}, etc. 
More applications (including some of these) can be found in~\cite{blablamec}.

The \emph{biclique graph} of a graph  denoted by , is the intersection graph of the family of all maximal bicliques of . 
It was defined and characterized in~\cite{GroshausSzwarcfiterJGT2010}. However no polynomial time algorithm is known for recognizing biclique graphs. 
As for clique graphs, the biclique graph construction can be viewed as an operator between the class of graphs. 

The \emph{iterated biclique graph}  is the graph obtained by applying to  the biclique operator   times iteratively. It was introduced in~\cite{marinayo} and all possible behaviors were characterized.
It was proven that a graph  is either divergent or convergent
but it is never periodic (with period bigger than ). In addition, general characterizations for 
convergent and divergent graphs were given. These results were based on the fact that if a graph  contains a clique of size at least , 
then  or  contains a clique of larger size. Therefore, in that case  diverges. Similarly if  contains the  or the  
graphs as an induced subgraph, then  
contains a clique of size , and again  diverges. Otherwise it was shown that after removing false-twin vertices of , the resulting graph is a clique on at most  vertices, in which case  converges. Moreover, it was proved that if a graph  converges, it converges to the graphs 
 or , and it does so in at most  steps. 
These characterizations leaded to an  time algorithm for recognizing convergent or divergent graphs under the biclique operator. 

In this work we show new results that lead to a linear time algorithm to
solve the same problem. We study conditions for a graph to contain a , a , 
a , a  or a  (see Figure~\ref{grafitos}) as induced subgraphs so we can decide
divergence (since  ). 
First we prove that if  has at least  bicliques then it diverges. 
Then, we show that every false-twin-free graph with at least  vertices has at least  bicliques, and therefore diverges.
Since adding false-twins to a graph does not change its  behavior, then the linear algorithm is based on the deletion of false-twin vertices of 
the graph and looking at the size of the remaining graph.  

\begin{figure}[ht!]
	\centering
	\includegraphics[scale=.2]{grafitos}
	\caption{Graphs , , ,  and , respectively.}
	\label{grafitos}
\end{figure}

It is worth to mention that these results are indeed very different from the ones known for the clique operator, for which it is still an open problem to know the computational complexity of deciding the behavior of a graph under the clique operator. 

This work is the full version of a previous extended abstract  published in~\cite{marinayoENDM}. It is organized as follows. In Section  the notation is given. Section  contains some preliminary results that we will use later. In Section  we prove that any graph with at least  bicliques diverges, and that every graph with at least  vertices with no false-twins vertices  contains at least  bicliques. This leads to a linear time algorithm to decide convergence or divergence under the biclique operator.

\section{Notation and terminology}

Along the paper we restrict to undirected simple graphs. Let  be a graph with vertex set  and edge set , and 
let  and . A \emph{subgraph}  of  is a graph  where  and 
. A subgraph  of  is \emph{induced} when for every pair of vertices ,  
if and only if . A graph  is \emph{-free} if it does not contain  as an induced subgraph.
A graph  is \emph{bipartite} when ,  and . Say that  is a 
\emph{complete graph} when every possible edge belongs to . A complete graph of  vertices is denoted . A \emph{clique} of  is a maximal complete induced subgraph while a \emph{biclique} is a maximal bipartite complete induced subgraph of . The \emph{open neighborhood} of a 
vertex  denoted , is the set of vertices adjacent to  while the \emph{closed neighborhood} of  denoted by 
, is . Two vertices ,  are \emph{false-twins} if . A vertex  is \emph{universal} if it is 
adjacent to all of the other vertices in . A \emph{path} (\emph{cycle}) of  vertices, denoted by  (), is a set of vertices 
 such that  for all  and  is adjacent to  
for all  (and  is adjacent to ). A graph is \emph{connected} if there exists a path between each pair of vertices. 
We assume that all the graphs of this paper are connected.

A  is a complete graph with  vertices and a vertex adjacent to two of them. 
A  is the graph obtained by joining two copies of the  with a common vertex.

Given a family of sets , the \emph{intersection graph} of  is a graph that has the members of 
 as vertices and there is an edge between two sets  when  and  have non-empty intersection.

A graph  is an \emph{intersection graph} if there exists a family of sets  such that  is the intersection 
graph of . We remark that any graph is an intersection graph \cite{Szpilrajn-MarczewskiFM1945}. 

A family of sets  is \emph{mutually intersecting} if every pair of sets  have non-empty intersection.

Let  be any graph operator. Given a graph , the \emph{iterated graph} under the operator  is defined iteratively as 
follows:  and for , . We say that a graph  \emph{diverges} under the operator  whenever 
. We say that a graph  \emph{converges} under the operator  whenever 
 for some , that is,  for every  and some . We say that a graph  is \emph{periodic} under 
the operator  whenever  for some , . 

The \emph{iterated biclique graph}  is the graph obtained by applying iteratively the biclique operator  times to . 

In the paper we will use the terms convergent or divergent meaning convergent or divergent under the biclique operator .

By convention we arbitrarily say that the trivial graph  is convergent under the biclique operator (observe that this remark is needed 
since the graph  does not contain bicliques).


\section{Preliminary results}\label{resultsantes}

We start with this easy observation.

\begin{observation}[\cite{marinayo}]\label{l3}
If  is an induced subgraph of , then  is a subgraph (not necessarily induced) of .
\end{observation}

The following proposition is central in the characterization of convergent and divergent graphs under the biclique operator. Basically, it shows that if a graph contains a big complete subgraph, it is going to grow in one or two steps of .

\begin{proposition}[\cite{marinayo}]
Let  be a graph that contains  as a subgraph, for some . Then,  or 
.
\end{proposition}

Next theorem characterizes the behavior of a graph under the biclique operator.

\begin{theorem}[\cite{marinayo}] \label{divergencia}
If  contains either  or the  or the  as an induced subgraph, then  is divergent. 
Otherwise,   converges to  or  in at most 3 steps.
\end{theorem}

Notice that differently than the clique operator, a graph is never periodic under the biclique operator (with period bigger than 1). 
We remark the importance of the graph  to decide the behavior of a graph under the biclique operator since we have that 
 and .

Observe that as proved in~\cite{marinayo}, the biclique graph does not change by the deletion or addition of false-twin vertices since each pair of 
false-twins belongs to exactly the same set of bicliques. That is, for any graph ,  for any false-twin vertex . It follows that the behavior of a graph under  does not change either. Therefore we focus our study on false-twins-free graphs. For that we need the following definition used in~\cite{marinayo}.

Consider all maximal sets of false-twin vertices  and let  be the set of \emph{representative vertices} such that . The graph obtained by the deletion of all vertices of  for , is denoted . Observe that  has no false-twin vertices.

Using , as a corollary of Theorem~\ref{divergencia}, the next useful result was obtained.

\begin{corollary}[\cite{marinayo}]\label{contraccion}
A graph  is convergent if and only if  has at most four vertices. Moreover,  for .
\end{corollary}

We recall that the number of bicliques of a graph may be exponential in the number of its vertices~\cite{PrisnerC2000}. However, if some vertex of a 
graph  lies in five bicliques, then  contains a  thus  diverges. If every vertex of  belong to at most four bicliques, then  has at 
most  bicliques. Therefore, since each biclique can be generated in ~\cite{DiasFigueiredoSzwarcfiterTCS2005,DiasFigueiredoSzwarcfiterDAM2007}, 
constructing  takes . Building  can be done in  time using the modular decomposition~\cite{modulardecomp}. From 
Corollary~\ref{contraccion}, if  has at most four vertices, then  is convergent, otherwise  is divergent. Hence the overall algorithm 
runs in  time.

\section{Linear time algorithm}\label{algolinear}

In this section we give a linear time algorithm for deciding whether a given
graph is divergent or convergent under the biclique operator.

Motivated by Theorem~\ref{divergencia} and Corollary~\ref{contraccion}, we study structural properties of a graph  to guarantee that 
its biclique graph  contains  and therefore  diverges.

The following two lemmas answer that question.

\begin{lemma}\label{2gemelos}
Let  for some graph . Let ,  be false-twin vertices of  and
,  their associated bicliques in . Suppose that there are no edges
between vertices of  and vertices of . Then there exists a vertex  such that  is adjacent to every vertex of  and .
Furthermore,  contains a  as induced subgraph.
\end{lemma}

\begin{proof} Let ,  be false-twin vertices of  and ,  their associated bicliques in , such that there are no edges between vertices of  and vertices of . Since  is connected, take the shortest path from some vertex of  to . Let  be the first vertex in the path such that . Clearly, . Let  be a vertex adjacent to . 

First, suppose that there exists a vertex  such that  is not adjacent to . Consider the following alternatives:

\textbf{Case 1}: . Then  is contained in some biclique ,  and , such that it does not intersect  since there is no edge between  and . This is a contradiction since  and  are false-twin vertices. It follows that every vertex in  not adjacent to  is not adjacent to .

\textbf{Case 2}: . Then there exists a vertex  adjacent to  and . By case 1,  must be adjacent to . This is the same situation as previous case but considering  instead of  and the biclique containing  instead of . A contradiction.

We conclude that for all ,  is adjacent to .

Now, the edge  is contained in a biclique  that must intersect  as  are false-twin vertices of . Since there are no edges between  and  there exists a vertex 
 such that  is adjacent to . The same argument used for  and  also holds for  and . That is, for all 
,  is adjacent to .

\begin{figure}[ht!]
	\centering
	\includegraphics[trim=0 100 100 140, clip, scale=.4]{lemma1}
	\caption{Bicliques  and  with  new bicliques containing .}
	\label{Figlemma1}
\end{figure}

Finally, let  be adjacent vertices in  and let  be adjacent vertices in . Since  are adjacent to , then , ,  and  are contained in four different bicliques , ,  and  such that , for . As , for  (Fig.~\ref{Figlemma1}),  is an induced subgraph of .
\end{proof}

\begin{lemma}\label{3gemelos}
Let  for some graph . Let  be false-twin vertices of 
and let  be their associated bicliques in . Suppose that for
any pair of bicliques , , there is an edge
between some vertex of  and some vertex of . Then,  is an induced
subgraph of .
\end{lemma}

\begin{proof}
Let  be the false-twin vertices of  and  their associated
bicliques in  such that for any pair of bicliques , , there is an edge between some vertex of  and some vertex of
. We will show that  contains either a , a , a  or a , or 
four mutually intersecting bicliques also intersecting with ,  and .
In any case we obtain a  in .
We have the following cases:

\textbf{Case 1}: There is a  with one vertex in each biclique. Let , ,  be the . Now ,  and  are contained
in  different bicliques of . It is easy to see that none of ,  or  are bicliques isomorphic to , otherwise 
they would not intersect the biclique containing the opposite edge of the  (e.g.  with ) contradicting that 
are false-twin vertices.

\textbf{Case 1.1}: One of the bicliques, say , is isomorphic to  where the vertex  is in the partition of size one.
As the biclique containing  must intersect , there exists a vertex  adjacent to
 and not adjacent to . Now, as , there exists a vertex , such that  is adjacent to .
Therefore  induces a  or a  depending on the
edge . See Figure~\ref{case1-1}.

\FloatBarrier
\begin{figure}[h]
	\centering
	\includegraphics[scale=.4]{case1-1}
	\caption{Case 1.1}
	\label{case1-1}
\end{figure}
\FloatBarrier

\textbf{Case 1.2}: None of the bicliques ,  and  are isomorphic to  where the vertex of the  is in the partition of 
size one. As the biclique containing  has to intersect , 
call  a vertex in that intersection and w.l.g. assume 
adjacent to  and not to .

\textbf{Case 1.2.1}: Suppose  is adjacent to . Now, as  is not isomorphic to , we have the following cases. 

If there exists a vertex  adjacent to  and not adjacent to
.  Depending on the edge  ,    induces  a    or
,  ,   and   are  contained in
four mutually intersecting bicliques.
See Figure~\ref{case1-2-1-1}.

\FloatBarrier
\begin{figure}[h]
	\centering
	\includegraphics[scale=.4]{case1-2-1-1}
	\caption{Case 1.2.1 with  adjacent to  and not to }
	\label{case1-2-1-1}
\end{figure}
\FloatBarrier

Otherwise, assuming that  every  adjacent  to  is adjacent
to ,  and considering that , there exists  
adjacent to   and .  In this case   induces a 
or a  depending on the edge .
See Figure~\ref{case1-2-1-2}.

\FloatBarrier
\begin{figure}[h]
	\centering
	\includegraphics[scale=.4]{case1-2-1-2}
	\caption{Case 1.2.1 with  adjacent to  and }
	\label{case1-2-1-2}
\end{figure}
\FloatBarrier

\textbf{Case 1.2.2}: There exists  not adjacent to  and , and adjacent to .
Let  be any vertex adjacent to  (and consecuently to ). Clearly, if  is adjacent to , it must be adjacent to , otherwise we
would be in the case above. So, if  is adjacent to both,  induces a .
Therefore, we can assume that for every  adjacent to  and ,  is not adjacent to  and .
Moreover, this must be also true for every vertex in  adjacent to  and every vertex in  adjacent to , that is,
every vertex in  adjacent to  is not adjacent to  and , and every vertex in  adjacent to  is not 
adjacent to  and . Suppose that there exists  adjacent to  and not adjacent to , then ,
,  and  are contained in four mutually intersecting bicliques.
Then, we can assume  is adjacent to . Indeed, assume that every vertex in  adjacent to  is adjacent to 
every vertex in  adjacent to  and to every vertex in  adjacent to . Also 
every vertex in  adjacent to  is adjacent to every vertex in  adjacent to .
Otherwise, we would obtain four mutually intersecting bicliques. Let  adjacent to .
Observe that if  is adjacent to  then  is also adjacent to ,
otherwise we are in case  1.2.1 considering the . Then,
depending on the edge ,    induces  a ,  or
, ,  and  
are contained in four mutually intersecting bicliques. 
See Figure~\ref{case1-2-2}.

\FloatBarrier
\begin{figure}[h]
	\centering
	\includegraphics[scale=.4]{case1-2-2}
	\caption{Case 1.2.2}
	\label{case1-2-2}
\end{figure}
\FloatBarrier
We covered all the cases when a  is in .

\textbf{Case 2}:  There is an induced  in  such
that ,  and , that is, .  Now  as ,  there exists  either 
adjacent to  and , or  adjacent to  and not adjacent
to . We have the following cases:

\textbf{Case 2.1}:  is adjacent to  and  (the case where  is
adjacent to  and  is analogous).  Observe that  is not adjacent
to  as  we would obtain a  triangle with one vertex  in each biclique
(case 1).  Let   be  a vertex  adjacent to  . If   is
adjacent to   then   induces a   (otherwise
case 1, considering  and , or  and , adjacent vertices).  Then  assume every vertex  adjacent to   is not
adjacent to . Furthermore, if any vertex  adjacent to , 
is also adjacent to , then  induces a , a  or a  depending
on the edges , . Therefore we can assume that every vertex  adjacent to   is not adjacent to .
See Figure~\ref{case2-1}.

\FloatBarrier
\begin{figure}[h]
	\centering
	\includegraphics[scale=.4]{case2-1}
	\caption{Case 2.1}
	\label{case2-1}
\end{figure}
\FloatBarrier

\textbf{Case 2.1.1}: There  is some  not adjacent to  .  Now as , there  exists  adjacent to   and not adjacent
to  .  If    is adjacent  to   then   induces  a
. We can assume  is not adjacent to .  

If  is adjacent to  then , ,  and
one of  or  depending on the edge , are contained
in four different mutually intersecting bicliques. So we can assume 
is not adjacent to .

As ,  either  is not adjacent to some vertex of
 that is adjacent to , or  forms a triangle with two vertices of
. 

Suppose first that  is not adjacent to   such that  is
adjacent do .  Note that ,  and 
are contained  in three  different mutually intersecting  bicliques. See
Figure~\ref{case2-1-1a}.

\FloatBarrier
\begin{figure}[h]
	\centering
	\includegraphics[scale=.4]{case2-1-1a}
	\caption{Case 2.1.1 with  not adjacent to }
	\label{case2-1-1a}
\end{figure}
\FloatBarrier

If  is  not adjacent  to   then   is contained  in the
fourth  biclique  (and  we  got  four  different  mutually  intersecting
bicliques).  So suppose   is adjacent to . If   is not adjacent
to , the fourth biclique  contains .  Finally, if  is adjacent
to  then   is contained  in the
fourth biclique.

Suppose next that  forms a triangle with two vertices of . That is, there are two adjacent vertices   such that  is adjacent to  and , and  is adjacent
to  (see Figure~\ref{case2-1-1b}).  If  is adjacent to , 
then depending on the  edge  ,   induces a  or a 
. Assume therefore that   is  not  adjacent  to  . 
Then, ,  and depending on the edge , either  and , 
or   and    are  contained  in  four  different
mutually  intersecting  bicliques.  


\FloatBarrier
\begin{figure}[h]
	\centering
	\includegraphics[scale=.4]{case2-1-1b}
	\caption{Case 2.1.1  form a  triangle with 2 vertices of }
	\label{case2-1-1b}
\end{figure}
\FloatBarrier

\textbf{Case  2.1.2}: Every vertex    adjacent to   is
adjacent  to .   Now as  , there  exists 
adjacent  to   and  .  Note  that   is  not  adjacent to  ,
otherwise case 1.  Then  induces a ,  or
 depending on the edges  and . See Figure~\ref{case2-1-2}.


\begin{figure}[ht!]
	\centering
	\includegraphics[scale=.4]{case2-1-2}
	\caption{Case 2.1.2}
	\label{case2-1-2}
\end{figure}


\textbf{Case 2.2}:   is adjacent to   and not adjacent  to . By
symmetry there  exists   adjacent to  and  not adjacent to
.  Assume that  is not adjacent  to  and  is not adjacent to
 (otherwise case 2.1).

Suppose first  that there  exists  adjacent  to   and .
Observe  that   is  not  adjacent  to    (case 1  considering  the
) and  is  not adjacent to  and to   at the same
time (case 2.1 considering the ). Depending on the edge
,  one  of    or   along  with  ,
,   are contained  in four  different mutually
intersecting bicliques.



\begin{figure}[ht!]    
	\centering
	\includegraphics[scale=.4]{case2-2a}
	\caption{Case 2.2, with edge  and without edge }
	\label{case2-2a}
\end{figure}

Suppose therefore that every  adjacent to  is not adjacent
to .  If  is not adjacent to  or  is  adjacent to , then
,  ,    and      are
contained  in  four  different  mutually  intersecting  bicliques.   See
Figure~\ref{case2-2a}. Otherwise,  is adjacent  to  and not  adjacent to .
Consider the , where the edge  is contained in
, vertex  and .  Now, following the same
arguments as above, considering vertex  as , vertex  as , and
vertex  as , since the vertex  (that is adjacent to
 and not adjacent to  and ) has the same ``role'' as the vertex , we arrive exactly to the previous case (when  is not adjacent to 
 or  is adjacent to , Figure~\ref{case2-2a}).
Therefore , ,  and  are contained in four different mutually intersecting bicliques, 
where  is adjacent to   and not adjacent to  nor to , and  is
adjacent   to    and  not   adjacent   to     nor  to   . See Figure~\ref{case2-2b}.

\begin{figure}[ht!]    
	\centering
	\includegraphics[scale=.4]{case2-2b}
	\caption{Case 2.2, with }
	\label{case2-2b}
\end{figure}


We covered all the cases when a  is in  with all of the vertices
in the bicliques ,  and .

\textbf{Case 3}: There is an induced ,  in  with at least one vertex from each
biclique ,  and . For the case  there is nothing to do.
Finally, for , it is easy to see that, as each biclique containing two consecutive edges of the  has
to intersect ,  and , then we would obtain a smaller cycle and therefore this case cannot occur.

Since we covered all cases the proof is done.

\end{proof}

Next, we present the main theorem of this section. This theorem shows that almost every graph is divergent under the biclique operator.
We remark that the linear time algorithm for recognizing convergent or divergent graphs given later in this section is based on this theorem.

\begin{theorem}
Let  be a graph. If  has at least  bicliques, then  diverges under the biclique operator.
\end{theorem}

\begin{proof} By way of contradiction, suppose that  has at least 
bicliques and  converges under the biclique operator. By
Corollary~\ref{contraccion},  for .
Consider the following cases.

\textbf{Case }. Then  is a contradiction since  has at least 
bicliques.

\textbf{Case }. In~\cite{GroshausSzwarcfiterJGT2010} it was proved that no bipartite graph with more than two vertices is a biclique graph. Then  what means that  has only  bicliques and therefore a contradiction.  

\textbf{Case }. Since  has at least  bicliques it follows that in 
there exists a set of false-twin vertices of size at least three.  Consider the
bicliques  of  associated to the three false-twin vertices. If there
is a pair of bicliques  such that there is no edge between any vertex
of  and any vertex of , by Lemma~\ref{2gemelos} it follows that 
is an induced subgraph of . Otherwise, for every two pair of bicliques
 there is an edge between some vertex of  and some vertex of
 and by Lemma~\ref{3gemelos}  contains  as an induced
subgraph. In any case, by Theorem~\ref{divergencia}  diverges under the
biclique operator, a contradiction. 

\textbf{Case }. There are two alternatives. Suppose that  has a set of false-twin
vertices of size at least three. Then following the proof of the case  we
arrive to a contradiction. Otherwise, there are only two possible graphs
isomorphic to  ( has  or  vertices and it has no set of
three false-twin vertices, see Fig.~\ref{Fig_7_8_vertices}). By inspection, using the characterization of biclique graphs given in~\cite{GroshausSzwarcfiterJGT2010},  
we prove that these two graphs are not biclique graphs. We conclude that this case cannot occur.



\begin{figure}[ht!]
	\centering
	\includegraphics[trim=0 100 0 100, clip, scale=.3]{7_8_vertices}
	\caption{Unique two possible graphs for case .}
	\label{Fig_7_8_vertices}
\end{figure}


Since we covered all cases,  diverges under the biclique operator and the
proof is finished.
\end{proof}

The next step is to study graphs without false-twin vertices with at least  bicliques.  This will complete the idea of  
the linear time algorithm for  recognizing divergent and convergent graphs under the biclique operator.

\begin{theorem}\label{13}
Let  be a false-twin-free graph. If  has at least  vertices then  has at least  bicliques.
\end{theorem}
\begin{proof}
We prove the result by induction on . For , by inspection of all graphs without false-twin vertices the result holds.
Suppose now that . 
Theorem  in~\cite{Sumner} states that if a graph  has no false-twin vertices, then there exists a vertex  such that  is also 
false-twin free. Consider such a vertex  and let . If  is connected, since it has at least  vertices, by the inductive hypothesis it has at least  bicliques.
Now as  is an induced subgraph of  we conclude that  also has at least  bicliques. Suppose now that  is not connected.
Let  be the connected components of  on  vertices respectively. Since  has no 
false-twin vertices, it can be at most one  such that . If there is one component with at least  vertices, then by the inductive 
hypothesis this component has at least  bicliques and so does . Therefore every component has at most  vertices. Now, by inspection we can 
verify that every component  (but maybe one with just  vertex) has at least  bicliques. Also, since  is 
disconnected,  along with at least one vertex of each of the  components is a biclique in  isomorphic to  that is lost in . 
Summing up and assuming the worst case, that is, there exists one  (suppose ) we obtain that the number of bicliques of  is at least 

as we wanted to prove. Now the proof is complete.
\end{proof}

Theorem~\ref{13} implies that the number of convergent graphs without false-twin vertices is finite since convergent graphs without false-twin 
vertices have at most  vertices. This fact leads to the following linear time algorithm. 

\textbf{Algorithm}: Given a graph , build . If  has at least  vertices, answer `` diverges'' and STOP.
Otherwise, build . If  has at most  vertices answer `` converges'' and STOP. Otherwise, answer `` diverges'' and STOP.

The algorithm has  time complexity. For this observe that  can be built in  time using the modular decomposition~\cite{modulardecomp}. 
Finally, if  has at most  vertices any further operation takes  time complexity.

\section{Conclusions}
In~\cite{marinayo} it is given an  time algorithm to recognize convergent and divergent graphs under the biclique operator. 
In this paper we prove that graphs without false-twin vertices with at least  vertices diverge. This shows that ``almost every'' graph is 
divergent and as a direct consequence, we obtain a linear time algorithm for recognizing the behavior of a graph under the biclique operator. 
We remark that in contrast as the iterated clique operator, no polynomial time algorithm is known for recognizing any of its possible behaviors. 

\bibliography{biblio2}

\end{document}
