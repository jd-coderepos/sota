
\documentclass[submission]{fundam}


\newenvironment{prOOf}[1]
  {\trivlist\PRstyle\item[]{\bfseries Proof {#1}:}\newline}{\QED\endtrivlist}
\def\squareforqed{\hbox{\rlap{}}}
\def\QED{\ifmmode\squareforqed\else{\unskip\nobreak\hfil
\penalty50\hskip1em\null\nobreak\hfil\squareforqed
\parfillskip=0pt\finalhyphendemerits=0\endgraf}\fi}



\usepackage{graphicx}
\usepackage{amssymb,amsmath,wasysym}\usepackage[mathscr]{euscript}
\usepackage[utf8]{inputenc}
\DeclareGraphicsRule{.tif}{png}{.png}{`convert #1 `dirname #1`/`basename #1 .tif`.png}
\usepackage{hyperref}
\usepackage{multicol}
\usepackage{color,booktabs}
\pagestyle{plain}
\sloppy
\title{Stochastic Cellular Automata:\\ Correlations, Decidability and Simulations}
\runninghead{P. Arrighi, N. Schabanel, G. Theyssier}{Stochastic Cellular Automata: Correlations, Decidability and Simulations}

\author{	Pablo Arrighi\thanks{This work has been partially funded by the ANR-10-JCJC-0208 CausaQ grant.}\\
			{Université de Grenoble (LIG, UMR 5217), France}\-1mm]
        		{Université de Lyon (IXXI), France}
 	\and 	Guillaume Theyssier\\
			{CNRS, Université de Savoie (LAMA, UMR 5127), France}
      }



\newcommand\GT{{\color{red}\bf\Large {GT}}}

\newcommand\TODO[1]{\framebox{\bf {#1}}}
\newcommand\DELETE[1]{}

\newcommand{\Qi}{\makebox{}}
\newcommand{\Qo}{\rightarrow}
\newcommand{\Qf}{\checked}
\newcommand{\Qe}{\bot}
\newcommand{\PPT}{\ensuremath{\textsc{PPT}}}



\newcommand{\PF}{\ensuremath{\operatorname{\mathscr{PF}}}}
\newcommand{\ZZ}{\ensuremath{\mathbb Z}}
\newcommand\Z\ZZ
\newcommand{\NN}{\ensuremath{\mathbb N}}
\newcommand\N\NN
\newcommand{\QQ}{\ensuremath{\mathbb Q}}
\newcommand{\Restriction}[3]{\ensuremath{#1_{#2..#3}}}
\newcommand{\nn}[1]{\Restriction{#1}{-n}{n}}
\newcommand{\rr}[1]{\Restriction{#1}{-n-\rho}{n+\rho}}
\newcommand{\AUTO}[1]{{\ensuremath{\mathcal{#1}}}}

\newcommand\CAA{\AUTO A}
\newcommand\CAB{\AUTO B}

\newcommand{\COUP}{\ensuremath{\operatorname{\mathfrak{C}}}}
\newcommand{\COUPL}{\ensuremath{\operatorname{\mathfrak{Coupled}}}}
\newcommand{\NOISE}[1]{\ensuremath{\operatorname{\text{}-\mathfrak{Noise}}}}

\newcommand{\PR}[1]{\ensuremath{\Pr\bigl\{{#1}\bigr\}}}

\newcommand{\SUBA}{\sqsubseteq}


\newcommand\Mes[1]{\ensuremath{\mathcal{M}({#1})}}

\newcommand\Pof[1]{\ensuremath{\mathcal{P}({#1})}}
\newcommand\DET[1]{\ensuremath{\mathcal{D}_{#1}}}
\newcommand\NDET[1]{\ensuremath{\mathcal{N}_{#1}}}
\newcommand\STOC[1]{\ensuremath{\mathcal{S}_{#1}}}
\newcommand\cyl[2]{\ensuremath{[{#1}]_{#2}}}

\newcommand{\SUB}{\sqsubseteq}
\newcommand{\PROJ}{\unlhd}
\newcommand{\MIX}{{\PROJ{}\hskip-5pt\SUB}}
\newcommand{\NSUB}{\overset{N}{\SUB}}
\newcommand{\NPROJ}{\overset{N}{\PROJ}}
\newcommand{\NMIX}{\overset{N}{\MIX}}
\newcommand{\SSUB}{\overset{S}{\SUB}}
\newcommand{\SPROJ}{\overset{S}{\PROJ}}
\newcommand{\SMIX}{\overset{S}{\MIX}}
\newcommand{\DSUB}{\overset{D}{\SUB}}
\newcommand{\DPROJ}{\overset{D}{\PROJ}}
\newcommand{\DMIX}{\overset{D}{\MIX}}
\newcommand\rest[2]{{}_{#1}{#2}}
\newcommand\proj[2]{{}^{#1\!}{#2}}
\newcommand\mix[3]{{{}^{#1}_{#2}}{#3}}

\newcommand\shift[1]{\mathfrak{\sigma}_{#1}}
\newcommand\grp[2]{{#1}^{\langle#2\rangle}}
\newcommand\bloc[1]{b_{#1}}
\newcommand\debloc[1]{b^{-1}_{#1}}

\newcommand\simu{\preccurlyeq}
\newcommand\ssimui{{\simu_i^S}}
\newcommand\ssimus{{\simu_\pi^S}}
\newcommand\ssimum{{\simu_m^S}}
\newcommand\nsimui{{\simu_i^N}}
\newcommand\nsimus{{\simu_\pi^N}}
\newcommand\nsimum{{\simu_m^N}}
\newcommand\dsimui{{\simu_i^D}}
\newcommand\dsimus{{\simu_\pi^D}}
\newcommand\dsimum{{\simu_m^D}}

\newcommand\somerel\leq


\newcommand\CFCA{\textsf{CFCA}}

 \renewcommand{\leq}{\leqslant}
 \renewcommand{\geq}{\geqslant}
 \renewcommand{\emptyset}{\varnothing}
\newcommand\probfa[1]{\mathbb{P}_{\mathcal{A}}\left(#1\right)}
 
 \newcommand{\anevent}{\operatorname{\mathcal E}}


\begin{document}
\maketitle

\begin{abstract}
This paper introduces a simple formalism for dealing with deterministic, non-deterministic and stochastic cellular automata in an unified and composable manner. This formalism allows for local probabilistic correlations, a feature which is not present in usual definitions. We show that this feature allows for strictly more behaviors (for instance, number conserving stochastic cellular automata require these local probabilistic correlations). We also show that several problems which are deceptively simple in the usual definitions, become undecidable when we allow for local probabilistic correlations, even in dimension one. Armed with this formalism, we extend the notion of intrinsic simulation between deterministic cellular automata, to the non-deterministic and stochastic settings. Although the intrinsic simulation relation is shown to become undecidable in dimension two and higher, we provide explicit tools to prove or disprove the existence of such a simulation between any two given stochastic cellular automata. Those tools rely upon a characterization of equality of stochastic global maps, shown to be equivalent to the existence of a stochastic coupling between the random sources. We apply them to prove that there is no universal stochastic cellular automaton. Yet we provide stochastic cellular automata achieving optimal partial universality, as well as a universal non-deterministic cellular automaton.
\end{abstract}


\section{Introduction} 

\paragraph{A motivation: stochastic simulation.} Cellular Automata (CA) are a key tool in simulating natural phenomena. This is because they constitute a privileged mathematical framework in which to cast the simulated phenomena, and they describe a massively parallel architecture in which to implement the simulator. 
Often however, the system that needs to be simulated is a noisy system. More embarrassingly even, it may happen that the system that is used as a simulator is again a noisy system. The latter is uncommon if one thinks of a classical computer as the simulator, but quite common for instance if one thinks of using a small scale model of a system as a simulator for that system.\\
Fortunately, when both the simulated system and the simulating system are noisy, it may happen that both effects cancel out, i.e. that the noise of the simulator is made to coincide with that of the simulated. In such a situation a model of noise is used to simulate another, and the simulation may even turn out to be\ldots exact. This paper begins to give a formal answer to the question: When can it be said that a noisy system is able to exactly simulate another?\\
This precise question has become crucial in the field of quantum simulation. Indeed, there are many quantum phenomena which we need to simulate, and these in general are quite noisy. Moreover, only quantum computers are able to simulate them efficiently, but in the current state of experimental physics these are also quite noisy. Could it be that noisy quantum computers may serve to simulate a noisy quantum systems? The same remark applies to Natural Computing in general. Still, the question is challenging enough in the classical setting.

\paragraph{A challenge: the need for local probabilistic correlations.} The first problem that one comes across is that stochastic CA have only received little attention from the theoretical community. When they have been considered, they were usually defined as the application of a probabilistic function uniformly across space \cite{Toom,Gacs,Fates,Mairesse,RST,FatesRST06}. In this paper we will refer to this model as local Correlation-Free CA (CFCA). Indeed, this particular class of stochastic CA has the unique property that, starting from a determined configuration, the cell's distributions remain uncorrelated after one step. This was pointed out in \cite{DBLP:conf/cie/ArrighiFNT11}, which provides an example (cf.  stochastic CA which we will use later) which cannot be realized as CFCA, in spite of the fact that they require only local probabilistic correlations and hence fit naturally in the CA framework. Moreover, \cite{DBLP:conf/cie/ArrighiFNT11} points out that the composition of two CFCA is not always a CFCA. The lack of composablity of a model is an obstacle for defining intrinsic simulation, because the notion must be defined up to grouping in space and in time. In \cite{DBLP:conf/cie/ArrighiFNT11} a composable model is suggested, but it lacks formalization.\\ 
In this paper we propose a simple formalism to deal with general stochastic CA. The formalism relies on considering a CA  fed, besides the current configuration , with a new fresh independent uniform random configuration  at every time step. This allows any kind of local probabilistic correlations and includes in particular all the examples of \cite{DBLP:conf/cie/ArrighiFNT11}. As it turns out, the definition also captures deterministic and non-deterministic CA (non-deterministic CA are obtained by ignoring the probability distribution over the random configuration). 

\paragraph{Results on stochastic simulation.} This formalism allows us to extend the notions of simulation developed for the deterministic setting \cite{bulking1,bulking2}, to the non-deterministic and stochastic settings. The choice of making explicit the random source in the formalism has turned out to be crucial to tackle the second problem, as it allows a precise analysis of the influence of randomness, in terms of simulation power. \\
The second problem that one comes across is that the question of whether two such stochastic CA are equal in terms of probability distributions is highly non-trivial. In particular, we show that testing if two stochastic CA define the same random map becomes undecidable in dimension  and higher (Theorem~\ref{thm:deciding}). Still, we provide an explicit tool (the coupling of the random sources of two stochastic CA) that allows to prove (or disprove) the equality of their probability distributions. More precisely, we show that the existence of such a coupling is strictly equivalent to the equality of the distribution of the random maps of two stochastic CA (Theorem~\ref{thm:coupling}).\\
The choice of making explicit the random source allows us to show some no-go results. Any stochastic CA may only simulate stochastic CA with a compatible random source (where compatibility is expressed as a simple arithmetic equation, Theorem~\ref{thm:primefactors}). It follows that there is no universal stochastic CA (Corollary~\ref{cor:no:stoc:universal}). Still, we show that there is a universal CA for the non-deterministic dynamics (Theorem~\ref{thm:nondet:univ}), and we are able to provide a universal stochastic CA for every class of compatible random source (Theorem~\ref{thm:stoc:univ}).  

\paragraph{Results on questions of computability versus allowing for local probabilistic correlations.} 
The fact that testing if two stochastic CA define the same random map is undecidable in dimension  and higher, which is not the case in the particular case of CFCA, suggested that many problems appear deceptively simple in the CFCA formalism. And indeed, their difficulty comes back as soon as one iterates the CFCA: for instance, we show that in dimension  and higher, it is undecidable whether the squares of two CFCA define the same random map (Corollary \ref{cor:undeciteratedequalityforCFCA}). Worse even, it is undecidable whether the square of a CFCA is noisy (Theorem \ref{thm:undecnoisecf}). \\
In dimension one, these problems, and many others, are shown to be decidable for general stochastic CA (Corollary~\ref{cor:dec1Dequality} and \ref{cor:buchi}). Thus, they cannot serve to point out a separation with CFCA. Yet, we show that the Pattern-Probability-Threshold problem (i.e. the question whether some pattern can appear with a probability higher that some threshold) is again undecidable for stochastic CA (Theorem \ref{thm:pptundeci}), whereas it is decidable for CFCA (Theorem \ref{th:pptdeccfca}). We also show that some behaviors are out-of-reach of CFCA. Indeed, CFCA cannot be number-conserving unless they are deterministic (Lemma \ref{lem:CFCA:num:det}). Moreover, even the iterates of a CFCA cannot be surjective number-conserving unless they are deterministic (Theorem \ref{thm:CFCA:num:det}). Iterates of two-state CFCA cannot reproduce behaviors alike the  example either (Theorem \ref{th:reven}). 


\paragraph{Plan.} Section~\ref{sec:basic} recalls the vital minimum about probability theory. Section~\ref{sec:SCA} states our formalism. Section~\ref{sec:localglobal} explains that some global properties of the SCA (such as equality of stochastic global functions, the probability of appearance of certain patterns, injectivity and surjectivity) cannot be decided from the local rules, but may be approached through some other proof techniques. These problems tend to simplify for the particular case of local Correlation-Free SCA, which corresponds to the more usual definition, and for the one-dimensional case, as explained in Sections~\ref{sec:correlationfree} and \ref{sec:dimensionone}. This shows that local Correlation-Free SCA are fundamentally simpler: some behaviors cannot be reached. Section~\ref{sec:simul} extends the notion of  intrinsic simulation to the non-deterministic and stochastic settings.  Section~\ref{sec:univ} provides the no-go results in the stochastic setting, the universality constructions. Section~\ref{sec:open} concludes this article with a list of open questions. 


\section{Standard Definitions}
\label{sec:basic}

\textit{Even if this article focuses mainly on one-dimensional CA for the sake of simplicity, it extends naturally to higher dimensions. Each time a result is sensitive to dimension, it will be explicited in the statement.}
\newcommand\extendstoalldimensions{\textit{The proof is written for 1D CA to simplify notations but it extends to any dimension in a straighforward way.}}
 

For any finite set  we consider the symbolic space . For any  and  we denote by  the value of  at point .  is endowed with the Cantor topology (infinite product of the discrete topology on each copy of ) which is compact and metric (see \cite{kurkabook} for details). A basis of this topology is given by cylinders which are actually clopen sets: given some finite word  and some position , the cylinder  is the set


We denote by  the set of Borel probability measures on . By Carath\'eodory extension theorem, Borel probability measures are characterized by their value on cylinders. Concretely, a measure is given by a function  from cylinders to the real interval  such that  and


We denote by  the uniform measure over  (s.t. ). We shall denote it as  when the underlying alphabet~ is clear from the context. 

We endow the set  with the compact topology given by the following distance:
.
See \cite{Pivato09} for a review of works on cellular automata from the measure-theoretic point of view.

\section{Stochastic Cellular Automata}
\label{sec:SCA}

Non-deterministic and stochastic cellular automata are captured by the same syntactical object given in the following definition. They differ only by the way we look at the associated global behavior. Moreover, deterministic CA are a particular case of stochastic CA and can also be defined in the same formalism.

\subsection{The Syntactical Object}

\begin{definition}
\label{def:syntax}
A \emph{stochastic cellular automaton}  consists in:
\begin{itemize}
\item a finite set of \emph{states} 
\item a finite set  called the \emph{random symbols}
\item two finite subsets of :  and , called the \emph{neighborhoods};  and  are the sizes of the neighborhoods and  is the \emph{radius} of the neighborhoods.
\item a \emph{local transition function} 
\end{itemize}
A function  is called a \emph{configuration};  is called the \emph{state} of the \emph{cell}~ in configuration~. A function  is called an \emph{-configuration}.

In the particular case where  (i.e., where each cell uses its own random symbol only), we say that \CAA{} is a \emph{Correlation-Free Cellular Automaton} (\CFCA{} for short).
\end{definition}



\begin{definition}[Explicit Global Function]
 To this local description, we associate the \emph{explicit global function}  defined for any configuration  and -configuration  by:

Given a sequence  of -configurations and an initial configuration , we define the associated \emph{space-time diagram} as the bi-infinite matrix  where  is defined by  and . We also define for any  the  iterate of the explicit global function  by  for all configuration~ and 

so that .\DELETE{ We denote by  the stochatic CA whose explicit global function is .}
\end{definition}

In this paper, we adopt the convention that local functions are denoted by a lowercase letter (typically~) and explicit global functions by the corresponding capital letter (typically ). Moreover, we will often define CA through their explicit global function since details about neighborhoods often do not matter in this paper.

The explicit global function captures all possible actions of the automaton on configurations. This function allows to derive three kinds of dynamics: deterministic, non-deterministic and stochastic. 


\subsection{Deterministic and Non-Deterministic Dynamics}

\paragraph{Deterministic.}  The \emph{deterministic global function}  of  is  defined by  where  is a distinguished element of . \CAA{}  is said to be \emph{deterministic} if its local transition function~ does not depend on its second argument (the random symbols).

\paragraph{Non-Deterministic.} The \emph{non-deterministic global function}  of~ is  defined for any configuration  by 
.

\paragraph{Dynamics.} The deterministic dynamics of \CAA{} is given by the sequence of iterates . Similarly the non-deterministic dynamics of  is given by the iterates  defined by  and . 


\subsection{Stochastic Dynamics}
\label{sec:stochdyn}

 The stochastic point of view consists in taking the -component as a source of randomness. More precisely, the explicit global function  is fed at each time step with a
random uniform and independent -configuration. This defines a stochastic process for which we are then interested in the distribution of states across space and time. By Carath\'eodory extension theorem, this distribution is fully determined by the probabilities of the events of the form  ``starting from , the word  occurs at position  after  steps of the process''.
Formally, for , this event is the set:

In order to evaluate the probability of this event, we use the locality of the explicit global function . The event ``'' only depends of the cells of  from position  to position . Therefore, if , then  and hence  is a measurable set of probability:  (recall that  is the uniform measure over ).



More generally to  any CA \CAA{} we associate its \emph{stochastic global function}  defined for any configuration  by: 
 


\paragraph{Example.} For instance, consider the stochastic function  that maps every configuration over the alphabet  to a random configuration in which every -word delimited by two consecutive   is replaced by a random independent uniform word of length with even parity. This cannot be realized by a \CFCA{}. Still, one can realize the stochastic function  as a stochastic CA with ,  and local rule  given by: for all   and ,
2mm]
\multicolumn{3}{c}{
	f(c_{-1}c_0\#,s_{-1}s_0) = s_{-1}
\quad\quad	f(c_{-1}c_0c_1,s_{-1}s_0) = s_{-1}+s_0}
\end{array}
\anevent^t_{c,\cyl{u}{z}}=\bigl\{(s^1,\ldots,s^t)\in\bigl(R^\ZZ\bigr)^t : F^t(c,s^1,\ldots,s^t)\in\cyl{u}{z}\bigr\}\bigl(\STOC{F}^t(c)\bigr)(\cyl{u}{z}) = \nu_{R^t}(\anevent^t_{c,\cyl{u}{z}})
\text{ =  the probability of event }

\Pr\{ F^t(u) = v\} =  \bigl(\STOC{F}^t(c)\bigr)(\cyl{v}{kt}), \text{\quad for any .} 

 \anevent^{t+1}_{c,\cyl{u}{z}} = \bigcup_{v\in Q^{b-a}} \left(\anevent^t_{c,\cyl{v}{a}}\times \anevent_{c_v,\cyl{u}{z}}\right).
 
  \bigl(\STOC{F}^{t+1}(c)\bigr)(\cyl{u}{z}) =
  \nu_{R^{t+1}}(\anevent^{t+1}_{c,\cyl{u}{z}}) &=
  \sum_{v\in Q^{b-a}} \nu_{R^{t+1}}\bigl(\anevent^t_{c,\cyl{v}{a}}\times
  \anevent_{c_v,\cyl{u}{z}}\bigr)\\ &=
  \sum_{v\in Q^{b-a}}\nu_{R^t}(\anevent^t_{c,\cyl{v}{a}})\cdot\nu_R(\anevent_{c_v,\cyl{u}{z}})\\
  &= \sum_{v\in Q^{b-a}} \bigl(\STOC{F}^{t}(c)\bigr)(\cyl{v}{a})\cdot\bigl(\STOC{F}(c_v)\bigr)(\cyl{u}{z})
\NDET{F}(c)\cap\cyl{u}{z}\neq\emptyset\Leftrightarrow \anevent_{c,\cyl{u}{z}}\neq\emptyset\Leftrightarrow\bigl(\STOC{F}(c)\bigr)(\cyl{u}{z})>0\mu_F(\cyl{u}{z}) = \nu\bigl(F^{-1}(\cyl{u}{z})\bigr)
    \anevent_1 &= \{s\in R_1^\ZZ : F_1(c,s)\in\cyl{u}{z}\}\\
    \anevent_2 &= \{s\in R_2^\ZZ : F_2(c,s)\in\cyl{u}{z}\}\\
    X &= \{(s_1,s_2) \in R_1^\ZZ\times R_2^\ZZ : F_1(c,s_1)=F_2(c,s_2)\}
  \bigl(\STOC{F_1}(c)\bigr)(\cyl{u}{z}) = \nu_1(\anevent_1) = \gamma_c(\anevent_1\times R_2^\ZZ)\bigl(\STOC{F_2}(c)\bigr)(\cyl{u}{z}) = \gamma_c\bigl((R_1^\ZZ\times \anevent_2) \cap X\bigr).
  S_u^1 &= \{s\in R_1^\ZZ : F_1(c,s)\in\ccyl{u}\}\\
  S_u^2 &= \{s\in R_2^\ZZ : F_2(c,s)\in\ccyl{u}\}
S_u^i = \bigcup_{v\in P_u^i}\ccyl{v}I_u^i(v) = \left[\frac{\ord{v}}{\# P_u^i};\frac{\ord{v}+1}{\# P_u^i}\right[|I_u^i(v)| = \frac{1}{\# P_u^i} = \frac{\nu_i(\ccyl{v})}{\mu(\ccyl{u})}\gamma^n(\ccyl{v^1},\ccyl{v^2}) =
\begin{cases}
  |I_u^1(v^1)\cap I_u^2(v^2)|\cdot\mu_Q(\ccyl{u}) &\text{ if  s.t.  for both }\\
  0 &\text{ otherwise.}
\end{cases}\gamma^n(\ccyl{v^1},\ccyl{v^2}) =
\begin{cases}
  \gamma^n(\ccyl{w^1},\ccyl{w^2}) &\text{ if  for }\\
  0 &\text{ else.}
\end{cases}
\gamma^n(\ccyl{v^1},R_2^\ZZ) = |I_u^1(v^1)|\cdot\mu_Q(\ccyl{u})X_n = \bigcup_{u\in Q^{2n+1}} S_u^1\times S_u^2\bigcap_n X_n = X = \{(s_1,s_2) \in R_1^\ZZ\times R_2^\ZZ : F_1(c,s_1)=F_2(c,s_2)\}\prod_{1\leq i\leq n}M_{u_i}(q_{i-1},q_i).\probfa{u} = \sum_{
 \begin{matrix}
   q\in Q^{n+1}\\
   q_0= I,\, q_n\in\fsta
 \end{matrix}
}\prod_{1\leq i\leq n}M_{u_i}(q_{i-1},q_i). f(c_{-1}c_0,\underbrace{(q_{-1},i_{-1})}_{=s_{-1}}\underbrace{(q_0,i_0)}_{=s_0}) = 
 \begin{cases}
 \Qi & \text{if },\\
  \Qo & \text{if  and  with c_{-1} = \Qi}\\
   \Qf &\text{if  and },\\
   \Qe &\text{in every other case.}
 \end{cases}

\bigl(\STOC{F}(c)\bigr)&(\cyl{\Qi\cdot \Qo^n\cdot \Qf}{0}) 
	 = \frac{
		\#\left\{ ((q_0,i_0),\ldots,(q_n,i_n)) \in R^{n+1}  \left| \begin{array}{l}  q_1 = \tau(u_1,I,i_0) \text{ and }  q_n\in\fsta \text{, and}\
by definition of , which concludes the proof.
\end{proof}
 
Using classical undecidability results concerning acceptance threshold in probabilistic finite automata we can now prove the Theorem stated earlier.
 
\begin{prOOf}{of the undecidability result in Theorem \ref{thm:pptundeci}}
 In \cite{GimbertO10}, the following problem is shown undecidable:
 \begin{description}
 \item[input:] a probabilistic finite automaton with all transition probability in 
 \item[question:] is there a word accepted with probability at least ?
 \end{description}
 Using Proposition~\ref{prop:encodePFA} it is straightforward to check that this problem reduces to problem \PPT{}.
\end{prOOf}
 
Let us now show that if the threshold is superexponential in , then one can decide \PPT{} in dimension~. Assume that the threshold function  is non-increasing and superexponential (i.e.  for all ). We will show the following structural lemma. Let us say that a word  is \emph{-looping} if  for some  and  (recall that  where  is radius of the considered SCA). We say that a word  is \emph{non-deterministic} for a SCA  if there are two random words  such that , and \emph{deterministic} otherwise. By extension, we say a word  is deterministic for  if all the subwords of length  it contains are deterministic for .

\newcommand\probdonne[2]{\Pr\{#1\overset{F}{\rightarrow} #2\}}
To simplify notations, we denote by  the probability  when  and  have appropriate lengths ().
 
\begin{lemma} \label{lem:ppt:superexp}
If there are  and a word  such that , then there are  and a word  such that  with either
\begin{itemize}
\item , or
\item  where , , , ,  and the word  is deterministic for . 
\end{itemize}
where  is a constant that can be algorithmically computed from , ,  and .
\end{lemma}
 
\begin{proof}

First note that by the pigeonhole principle, any word in  of length at least  contains a -looping subword  with  and . Let us isolate the centerpart of  by writing  where . Let us mark in  all the letters which are at the center of non-deterministic neighbourhoods. The marked letters split  into subwords where each letter belongs to a deterministic neighbourhood. 
 
	Let us first assume that all of this deterministic subwords have length  at most , then there are at least  distinct letters at the center of non-deterministic neighbourhoods in . Since neighbourhoods at distance at least  from each other evolve independently, and since every non-deterministic neighbourhood produces an error in the pattern with probability at least ,  it follows that the image of  by  will be  with probability at most  where . As , it follows that  but since  is superexponential and non-increasing, there is a constant  that can be algorithmically computed from  and  (assuming an appropriate oracle for the superexponentiallity of )
such that for all , . It follows that  and case 1 is verified for~.
	
	Let us now assume that one of the deterministic (unmarked) subwords of  has length at least , it follows that it contains a -looping subword of length at most . Note that one can strip or duplicate the loop  in any -looping subword  in : this will only affect the image (by adding or deleting some s in the image) but will not change the probability of obtaining the pattern . We then strip in  all the loops in the -looping deterministic subwords but one and duplicate the remaining one as many times as necessary to obtain a word  at least as long as , i.e. so that  has length  with . Since  is non-increasing and since the probability to obtain  from  is identical to the probability to obtain  from , we have . Now,  has the form  where  is deterministic and all the deterministic subwords in  and  have length at most . Using the same argument as before, the sum of the lengths of the two words  and  is bounded by the constant  and therefore  has the desired properties for case 2.
\end{proof}
 
\begin{prOOf}{of the decidability part of Theorem~\ref{thm:pptundeci}}
Let .
Consider a SCA  with state set , random symbol set , radius  and neighborhood width . Let  and compute  such that  for all . According to Lemma~\ref{lem:ppt:superexp},  there are an  and a word  such that  if and only if there is such a pair with  or there is a deterministic -looping word  of length at most  and two words  of total length at most  and a  such that  verifies the condition. One can easily check these conditions by enumerating all the subwords of length at most  and loops of length , the only difficulty consists in computing the right value for  in the second case. 
This is achieved as follows: first compute the word  with  and  such that  is deterministic for  and for which  is maximum. If , then we conclude that the second possibility is not possible because pumping in the loop  does not change the probability to obtain the pattern  since  is deterministic. 
If , then using the monotonicity of , there must exist some  such that  where  (again because pumping in the deterministic loop  does not change probabilities). Then the second case is verified.
\end{prOOf}

\subsection{About the Garden of Eden Theorem}

One of the most celebrated theorems in the setting of deterministic CA states that surjectivity is equivalent to pre-injectivity (Garden-of-Eden Theorem, \cite{Coornaert10}).
In the case of stochastic CA the situation is different as we will show in this section.

\begin{definition}
  Let  be a stochastic CA.
  \begin{itemize}
  \item  is \emph{surjective} if for any  there is some  and some  such that .
  \item  is \emph{injective} if for any  and  we have
    
  \item  is \emph{pre-injective} if for any  with finitely many differences and  we have
    
  \end{itemize}
\end{definition}

Some remarks:
\begin{enumerate}
\item the definitions above agree with the classical deterministic setting when the stochastic CA considered turns out to be deterministic.
\item in general a stochastic system can be injective yet non-deterministic, for instance consider the following stochastic map  s.t.
  
\item it is easy to define some stochastic CA which is surjective but not injective (and not pre-injective), for instance  s.t.
  
\end{enumerate}

Contrary to the deterministic setting, there is no Garden-of-Eden Theorem for stochastic CA. However, there are still strong relationships between the notion defined above.

\begin{theorem}\label{thm:injec}
  Let  be a stochastic CA, then we have:
  \begin{itemize}
  \item if  is pre-injective then  is surjective;
  \item if  is injective then  is deterministic.
  \end{itemize}
\end{theorem}
\begin{proof}
\extendstoalldimensions{}  For any  and any word  we define the \textbf{deterministic} CA  as the grouping by groups of size  of the map:
  
  where  is the periodic configuration of period  (with  on cell ).

  First, since  is pre-injective, we have that, for any ,  is pre-injective (straightforward). Hence, by choosing some , we deduce that  is surjective (by the Garden-of-Eden theorem). Therefore  is also surjective, which proves the first assertion of the theorem.

  Now suppose in addition that  is injective. Then for any  and any  we have . Indeed, if it was not the case we would have some  such that . But since  is surjective (shown above) there would exist some  such that . Since  must be different from  (because ) this contradicts the injectivity of  (grouping is a bijective operation and does not affect injectivity).

To conclude the proof it is sufficient to take  large enough (larger than  where  is the radius of ): in this case  for any  means that  does not depend on its -component, hence it is deterministic.
\end{proof}



\section{Correlation-free local rules are simpler}
\label{sec:correlationfree}

\newcommand\cfl{\mathbb{P}_f}

A stochastic CA  is \textit{correltation-free} if its neighborhood associated to the random component is trivial:  (see Definition~\ref{def:syntax}). Letting , its local function  is then of the form  and it can be seen has a map  from  to probability distributions over  (maps from  to  summing to ) as follows:

Note that most of the literature concerning stochastic CA is restricted to local Correlation-Free CA and use map  to define them \cite{Toom,Gacs,Fates,Mairesse,RST,FatesRST06}.

As an immediate consequence of this form of local function, one can compute probabilities involved in the global function as a product of 'local probabilities' as shown by the following lemma. To simplify notations, it is stated in dimension 1 but extends without difficulty to higher dimensions.

\begin{lemma}
  \label{lem:cf}
  If  is a local Correlation-Free stochastic CA of local function  and radius , we have for all configuration  and all finite words :
  
\end{lemma}
\begin{proof}
  It is sufficient to check that the set  (see section~\ref{sec:stochdyn}) is a union of cylinders which is in one-to-one correspondence with the set
  
  The lemma follows by application of the uniform measure on both sets.
\end{proof}


\subsection{Impossible behaviors}

We now present behaviors than can be realized by general stochastic CA but not by CFCA.

\subsubsection{Number-conserving CA}

Number-conserving CA are regularly used to model interacting particles (see \cite{DurandFR03,FormentiG03,FormentiKT08} for the case of deterministic CA). A classic example of interacting particles model is the usual random walk, which is number-conserving because the number of walkers is conserved. Again, we restrict to dimension 1 to simplify notations but the extension to higher dimension is straightforward.

\begin{definition}
A SCA  is number-conserving if  for some  and for any finite configuration  (i.e. a configuration with finitely many cells in a state other than ), we have 

where the infinite sums are well-defined because only finite configurations are considered.
\end{definition}

Note in particular that the definition implies  and more generally, when  is a finite configuration and  is reachable from  then  (because there are only finitely many configurations reachable from ).

Remark that our definition requires  the number to be conserved almost surely and is thus more restrictive than the definition in \cite{Fuks2004}, which requires only the number to be conserved in expectation. The conclusion of \cite{Fuks2004}, leaves open the question of strictly conservative particle system in \CFCA{}. We settle this question by showing that there is \emph{no} \CFCA{} (\emph{nor powers} of \CFCA{}) that can simulate (surjective) conservative particle system.

First, remark that it is easy to design a SCA that simulates a conservative particle system. For instance, consider the following SCA  with states , random symbols , and radius~.  The s represent the particles and the s the empty cells. The random symbol represents the movement each particle is trying to make: stay for ; move right for  to be performed if the right cell is  and if this move does not induce any conflict with another particle;  move left for  to be performed if the left cell is  and if this move does not induce any conflict with another particle.  Here is its local rule (we only give the neighbourhoods whose image are , the others have image ):

\medskip

\newcommand{\RS}[1]{\makebox[1em][c]{}}
\newcommand{\RR}{{\RS{\scriptsize\rightarrow}}}
\newcommand{\RNR}{{\RS{\scriptsize\not\rightarrow}}}
\newcommand{\RL}{{\RS{\scriptsize\leftarrow}}}
\newcommand{\RNL}{{\RS{\scriptsize\,\not\!\leftarrow}}}
\newcommand{\RE}{{\RS*}}
\newcommand{\RD}{{\RS\cdot}}
\newcommand{\RA}{\RS{*}}
\newcommand{\RO}{\RS{0}}
\newcommand{\RI}{\RS{1}}
\newcommand{\FF}[2]{f\!\left(\!\!\begin{array}{cc}c:#1\\s:#2\end{array}\!\!\!\!\right)}

\centerline{\begin{minipage}{.8\textwidth}
\begin{multicols}{3}
\noindent \1mm]
\1mm]
\1mm]
\1mm]

\end{multicols}
The images of all other patterns are  ( stands for an arbitrary symbol).
\end{minipage}}
\medskip

This SCA is clearly non-deterministic, number-conserving and surjective (each cell remains unchanged when its random symbol is ).

Note that most interacting particle systems are not only number-conserving but also surjective. It turns out that no CFCA nor iterates of CFCA can express such systems.

\begin{lemma} \label{lem:CFCA:num:det}
If a CFCA is number conserving, then it is a deterministic map.
\end{lemma}

\begin{proof}
Assume by contradiction that  is a number conserving CFCA which is not deterministic. Let  be a finite configuration such that there are  such that  and . Let  be such that . Then, since the updates are independent in CFCA: with positive probability,  is mapped to  and with positive probability,  is mapped to  where  and . This is a contradiction since the total weight of  and  differ by .
\end{proof}



\begin{theorem}\label{thm:CFCA:num:det}

Let  be a CFCA. If  is number-conserving and surjective for some , then  is deterministic.

\end{theorem}

\begin{proof}
Assume by contradiction that  is non-deterministic. By Lemma~\ref{lem:CFCA:num:det},  is not number conserving. Therefore there are finite   such that , , and the weight of  and  differ. As  is surjective, so is  and let  be a finite configuration such that . It follows that  and  which contradicts the fact that  is number conserving.     
\end{proof}

Whereas the surjectivity constraint was not needed in the lemma, it is required for the theorem to hold. Indeed, the square of the CFCA illustrated in Fig.~\ref{fig:sq:number} is non-deterministic and number-conserving. This CFCA~ has neighbourhood  (a neighbourhood large enough to prevent unstable patterns from propagating). The only patterns yielding to a state change are:
1mm] 
f({*}0000\mathbf011000{*}) = 0 \text{ or } 1  \text{ with probability } \frac12 
&	f({*}{*}000\mathbf1001000) = 0 
\end{array}
A = \bigcup_{0\leq j\leq \ell-1}Q^{j}\times R^{j}\bigl((\overline{a},\overline{b}),q,(\overline{a}q,\overline{b}r)\bigr)\in\delta\bigl((q_1\cdots q_{\ell-1},r_1\cdots r_{\ell-1}), \alpha, (q_2\cdots q_{\ell},r_2\cdots r_{\ell})\bigr)\in\deltaw'\bigl(\overline{a},q,(\overline{a}q)\bigr) = 1;w'\bigl((q_1,\ldots,q_{\ell-1}),(q_\ell,q),(q_2,\ldots,q_\ell)\bigr) = \bigl(\cfl(q_1,\ldots,q_\ell)\bigr)(q)\phi(c) = n\mapsto \bigl(c(n),c(-n)\bigr).\bigl\{\bigl(\phi(c_1),\ldots,\phi(c_{k_i})\bigr) : (c_1,\ldots,c_{k_i})\in R_i\bigr\}.\rest{i}{\CAA}=(Q',R,V,V',\rest{i}{f})\proj{\pi}{\CAA}=(Q',R,V,V',\proj{\pi}{f})\grp{F}{m,t,z}(c,s) = \bloc{m}\circ \shift{z}\circ F^t(\debloc{m}(c),\debloc{m}(s^1),\ldots,\debloc{m}(s^t))\CAA_1\simu\CAA_2 \Leftrightarrow \exists m_1,m_2,t_1,t_2,z_1,z_2, \grp{\CAA_1}{m_1,t_1,z_1}\,\somerel\,\grp{\CAA_2}{m_2,t_2,z_2}\CAA_1\somerel\CAA_2\Rightarrow \grp{\CAA_1}{m,t,z}\somerel\grp{\CAA_2}{m,t,z}\grp{\grp{\CAA_1}{m,t,k}}{m',t',z'} \somerel \grp{\grp{\CAA_1}{m',t',z'}}{m,t,z}\bigl(\STOC{F_1}(c)\bigr)(\cyl{u}{z}) = \nu_1(\anevent^1_{c,\cyl{u}{z}}) = \frac{p}{q}<1\bigl(\STOC{F_2}(c)\bigr)(\cyl{u}{z}) = \frac{p}{q}
and by a similar argument as above we deduce that . The lemma follows since  (because ).
\end{proof}

From Lemma \ref{lem:primefactors} it follows, surprisingly perhaps, that the random symbols of a stochastic CA limit its simulation power to stochastic CA that have compatible random symbols.
\begin{theorem}\label{thm:primefactors}
  Let  be any stochastic simulation pre-order, and  and  two stochastic CA which are not deterministic. If  then .
\end{theorem}
\begin{proof} Trimming operations (restrictions and projections) do not modify the set of random symbols. Rescaling transformations modify the set of random symbols in the following way:  for some integer . Therefore such transformations preserve the set of prime factors  of the considered CA \CAA. Moreover, rescaling transformations do not affect determinism: the rescaled version of a CA which is not deterministic cannot be deterministic. Hence, the relation  implies an equality of stochastic global functions of two CA which have the same prime factors as  and , one of which is not deterministic. Therefore none of them is deterministic and the theorem follows from lemma~\ref{lem:primefactors}.
\end{proof}
\DELETE{\begin{proof}(Sketch) First  is invariant under any rescaling transformation. Then, if the distribution matches, there must be coupling of the random strings of some  finite lenght by Theorem~\ref{thm:coupling}. The existence of such a coupling implies that  and  must have a common prime factor. 
\end{proof}}
\DELETE{Given any stochastic pre-order , a -universal CA cannot be deterministic because determinism is preserved by simulation (Fact~\ref{fact:ideals}). Moreover, for any prime~ there is some  with . Hence Theorem~\ref{thm:primefactors},}

The consequence in terms of universality is immediate and breaks our hopes for a stochastic universality construction. 
\begin{corollary} \label{cor:no:stoc:universal}
  Let  be any stochastic simulation pre-order. There is no \mbox{-universal} stochastic CA.
\end{corollary}

\subsection{Positive results}


Still, the negative result of Corollary \ref{cor:no:stoc:universal} leaves open the possibility of partial universality constructions. We will now describe how to construct a stochastic CA which is -universal (hence also -universal; however note that the existence of a  - or even of a -universal is still open), and then draw the consequences.\\
Since we are not concerned with size optimization, we will use simple construction techniques using parallel Turing heads and table lookup as described for classical deterministic CA in \cite{OllingerUnivhistory}. More precisely, we construct a stochastic CA  able to -simulate any stochastic CA  with no rescaling transformation on  and no shift in the rescaling of . Therefore each cell of  will be simulated by a block of  cells of  and each step of  will be simulated by  steps of  ( and  depend on \CAA{} and are to be determined later).\\
The blocks of  cells have the following structure (the restriction in the pre-order handles the trimming of any invalid block):
\begin{center}\scriptsize\sf
    \begin{tabular}{|c|c|c|c|c|c|}
      \hline
      SYNC & transition table & -state & -symbol & -states of neighbors & -symbols of neighbors \\
      \hline
    \end{tabular}
\end{center}
where each part uses a fixed alphabet (independent of  and ) and only the width of each part may depend on \CAA{}\DELETE{ (note that the transition table encodes in particular the size of )}. To each such block is attached a Turing head which will repeat cyclically a sequence of  steps (sub-routines) described below. On a complete configuration made of such blocks there will be infinitely many such heads (one per block) executing these steps in parallel. Execution is synchronized at the end of each step (\textsf{SYNC} part) and such that two Turing heads never collide. More precisely, for some steps (2 and 4) the moves of all heads are rigorously identical (hence synchronous and without head collision). For some other steps (1 and 3), the sequence of moves of each head depends on the content of their corresponding block but these steps are always such that the head does not go outside the block (hence no risk of head collision) and they are synchronized at the end by the \textsf{SYNC} part which implements a small time countdown initialized to the maximum time needed to complete the step in the worst case. The parts holding -symbols are initially empty (uniformly equal to some symbol) for each block. The  steps are as follows:
\begin{enumerate}
\item generate \DELETE{surjectively }a string representing a random -symbol in the \textsf{-symbol} part using (possibly several) random -symbols present in that part of the block;
\item copy the \textsf{-symbol} part to the appropriate position in the \textsf{-symbols of neighbors} part of each neighboring block. Do the same for \textsf{-state};
\item using information about -states and -symbols in the block, find the corresponding entry in the transition table and update the \textsf{-states} part of the block accordingly;
\item clean \textsf{-symbol} and \textsf{-symbols of neighbors} parts (i.e. write some uniform symbol everywhere).
\end{enumerate}

This construction scheme is very similar to the one used for classical deterministic CA but two points are important in our context:
\begin{itemize}
\item step  is here to ensure that each configuration of \CAA{} has a canonical corresponding configuration of  made of blocks where the parts holding -symbols is clean (i.e., step~4 is required for the existence of the injection~);
\item depending on the way we generate strings representing an -symbols from strings of -symbols in step , we will obtain or not a uniform distribution over  (recall Theorem~\ref{thm:primefactors}).
\end{itemize}

In the general case, we can always fix (by the means of the injection~) a  width large enough for the \textsf{-symbols} parts containing  so that all -symbols can be obtained (but with possibly different probabilities). We therefore obtain a universality result for non-deterministic simulations.

\begin{theorem}\label{thm:nondet:univ}
  Let  be either  or . There exists a -universal CA.
\end{theorem}

Note that this  -universal CA is a \CFCA, and we obtain thus a stronger version of the simulation mentioned in Section~\ref{sec:pCFCAsimul} page~\pageref{sec:pCFCAsimul}.

Now, if we are in a case where  then it is possible to choose a generation process in step  such that each -symbol is generated with the same probability. We therefore obtain an optimal partial universality construction for stochastic simulations.

\begin{theorem}\label{thm:stoc:univ}
  Let  be either  or . For any finite set  of prime numbers, there is a stochastic CA  such that for any stochastic CA \CAA: 
Moreover,   is a \CFCA.
\end{theorem}


\section{Recap of Some Decidability Results}

\newcommand\Ude{\textit{\textcolor{red}{Undecidable}}}
\newcommand\Dec{\textit{\textcolor{black}{Decidable}}}

\begin{center}
\begin{tabular}{ccccc}
\bfseries Problem  & \bfseries General & \bfseries Correlation-Free & \bfseries 1D & \bfseries Correlation-free 1D\\ \toprule
 deterministic & \Dec &  \Dec &  \Dec &  \Dec \\ 
 \midrule
 & \Ude & \Dec & \Dec & \Dec \\
 \midrule
  noisy & \Ude & \Dec & \Dec & \Dec \\
 \midrule
  & \Ude & \Ude & \Dec & \Dec \\
 \midrule
  noisy & \Ude & \Ude & \Dec & \Dec \\
 \midrule
\PPT{}
& \Ude & ? &  \Ude & \Dec \\
\bottomrule
\end{tabular}
\end{center}

\section{Open Problems}
\label{sec:open}


Intrinsic simulation has been proven to be a powerful tool to hierarchize behaviors in the deterministic world. In particular, the notion of universal CA allows to formalize the concept of ``most complex'' CA as the ones concentrating ``all the possible behaviors'' within a given class \cite{bulking1,bulking2}.
The formalism and the notion of intrinsic simulation developed here for stochastic CA, enable us to export this classification tool to the stochastic world. In particular, it would be interesting to see whether our partial universality construction relates to experimentally observed classes, as in \cite{RST}.\\
It would also be interesting to extend these notions of intrinsic simulation between stochastic CA to noisy Quantum Cellular Automata, as this could be of use for quantum simulation.\\
At the theoretical level, and amongst all the concrete questions raised by this article, the following ones are of particular interest:
\begin{itemize}
\item 
Is there for any stochastic \CAA{}, a \CFCA{}  which is  -equivalent to ?
\item 
Is there for any stochastic \CAA{}, a \CFCA{}  such that ?
\item 
Are there -universal cellular automata?
\item  
Are universal CA the same for pre-order  and ?
\item 
Is \PPT{} undecidable for CFCA in dimension  and higher?
\end{itemize}

Our setting can also be generalized by taking any Bernouilli measure on the -component (instead of the uniform measure). We believe that positive and negative results about universality essentially still hold but under a different form.

More generally, as far as we know, there is no characterization of the probability distributions over the configurations that correspond to images of cellular automata: deterministic automata starting from a random initial configuration, nor stochastic cellular automata starting from fixed or random distribution. In particular, we failed in our attempts to obtain an ``Hedlund-like'' characterization \cite{Hedlund} of the stochastic maps corresponding to stochastic cellular automata  (recall that there are \emph{constant} stochastic maps which do not correspond to any stochastic cellular automaton). One possible direction might be to explore extensions of our framework  allowing arbitrary shift-invariant distributions for the -configuration; a characterization of this extension would however be still unsatisfying since many  shift-invariant distributions are highly non-local and are thus only remotely related to cellular automata.

\bibliographystyle{fundam}
\bibliography{uspca}


\end{document}  
