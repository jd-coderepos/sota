\documentclass[technote, a4paper, onecolumn]{IEEEtran}  \newcommand{\avtor}{Aleksandar Simevski}
\newcommand{\naslov}{Theoretical Aspects of a Design Method for Programmable NMR Voters}
\usepackage[pdftex]{graphicx}
\usepackage[caption=false]{subfig}
\usepackage{amssymb}
\usepackage{amsthm} \graphicspath{{H:/thesis/pictures/}}
\DeclareGraphicsExtensions{.jpg}
\usepackage[cmex10]{amsmath}
\hyphenation{op-tical net-works semi-conduc-tor}

\newtheorem{property}{Property}
\newtheorem{assertion}{Assertion}
\newtheorem{theorem}{Theorem}
\newtheorem{corollary}{Corollary}
\newtheorem{lemma}{Lemma}
\newtheorem{ex}{Example}



\begin{document}
\title{\naslov} 

\author{\IEEEauthorblockN{Elena Hadzieva}
\IEEEauthorblockA{\\\textit{University of Information Science and Technology\\"St. Paul the Apostle"
\\Partizanska bb, 6000 Ohrid, Macedonia\\elena.hadzieva@uist.edu.mk}}

\and 
\vspace{0.05in}
\IEEEauthorblockN{\avtor}
\IEEEauthorblockA{\\\textit{IHP GmbH\\Im Technologiepark 25, D-15236 Frankfurt (Oder), Germany\\simevski@ihp-microelectronics.com}}



}




\maketitle

\begin{abstract} 

Almost all dependable systems use some form of redundancy in order to increase fault-tolerance. Very popular are the -Modular Redundant (NMR) systems in which a majority voter chooses the voting output. However, elaborate systems require fault-tolerant voters which further give additional information besides the voting output, e.g., how many module outputs agree. Dynamically defining which set of inputs should be considered for voting is also crucial. Earlier we showed a practical implementation of programmable NMR voters that self-report the voting outcome and do self-checks. Our voter design method uses a binary matrix with specific properties that enable easy scaling of the design regarding the number of voter inputs N. Thus, an automated construction of NMR systems is possible, given the basic module and arbitrary redundancy . In this paper we present the mathematical aspects of the method, i.e., we analyze the properties of the matrix that characterizes the method. We give the characteristic polynomials of the properly and erroneously built matrices in their explicit forms. We further give their eigenvalues and corresponding eigenvectors, which reveal a lot of useful information about the system. At the end, we give relations between the voter outputs and eigenpairs.

\end{abstract} \IEEEpeerreviewmaketitle

\section{Introduction}\label{sec_introduction} 

Widely used scheme for increasing system dependability is -Modular Redundancy (NMR). Fig.~\ref{fig_nmr_sys} presents an NMR system. The  identical modules  fed with the same input  are expected to produce equal outputs . However, in a real system the modules are subject to faults that lead to differences in these outputs. Therefore, a decision maker D selects the final output of the system . One of the most frequently used decision makers is the majority voter, where at least  outputs of the  modules have to be equal.

\begin{figure}[!ht]
    \centerline{\includegraphics[width=0.5\linewidth]{049_NMR_system}}
    \caption{NMR system}
    \label{fig_nmr_sys}
\end{figure}

Dependable systems employ some form of redundancy (time, space, information) which affect other system properties such as performance, power consumption or complexity (cost). A trade-off is therefore necessary. However, intelligent mechanisms may enable a dynamic trade-off, i.e., increase dependability and performance or lower power consumption on demand. Consider the dependable 4MR system depicted in Fig.~\ref{fig_dep_sys} as an example. The system acquires information by four identical sensors measuring the same physical quantity. This information is further processed by four processors that output the results  and .

\begin{figure}[!ht]
    \centerline{\includegraphics[width=0.5\linewidth]{044_scenario}}
    \caption{Dependable 4MR system}
    \label{fig_dep_sys}
\end{figure}

By observing the responses of the processors over a period of time, the system could differentiate between permanent and transient faults in the processor-sensor pairs. Thus, if the system detects a permanent fault in one of the processor-sensor pairs, it may decide to switch them off in order to save power. Furthermore, consider the following NMR on demand (NMROD) adaptive behavior. Normally, only two processor-sensor pairs are operating in a dual-modular redundant (DMR) fashion. The power supply is switched off for all other pairs. As long as the two results are equal, the output of voting is equal to the results and the operation is considered error-free. A single disagreement between the two operating pairs is a signal for the system to power up a third pair and restart the operation in a triple-modular redundant (TMR) fashion. The fourth pair could be included only in critical situation when faults frequently occur, otherwise the system may opt  switching back to DMR. Besides the output of voting , these example systems have to know exactly which processor-sensor pairs disagree, as well as the total number of pairs that disagree. Furthermore, they require dynamically building 1MR to NMR systems with any possible combination of processor-sensor pairs. In this discussion we have assumed that the voter itself is not subject to faults. However, system operation is compromised if a fault occurs in the voter. Therefore, it is preferable to have some dependability mechanisms which detect and report incorrect voter operation, or if possible, mask the errors.

So far we have illustrated our motivation for a special type of decision maker -- a programmable NMR voter with self-report and self-checking capabilities that is suitable for all the scenarios discussed previously. These voters describe the situation at their inputs, e.g., which modules disagree. Moreover, they could be dynamically programmed in order to form different NMR systems on the fly. In \cite{Simevski2012a} we show an intuitive method for designing such type of voters as well as the results from their actual implementation. Furthermore, we use these voters in order to investigate a dynamic scheme of core-level NMR in multiprocessors~\cite{Simevski2014a}. In this paper, we present the theoretical aspects of the method and we formally prove our assumptions. This is important since the method enables automated construction of elaborate NMR systems. That is, given the basic module (e.g., a single processor-sensor pair) and arbitrary redundancy , the whole system could be built automatically. We present technical details of a register-transfer level NMR system generator in~\cite{Simevski2013a}.

The rest of the paper is organized as follows. Section~\ref{sec_related} presents related work. In Section~\ref{sec_definitions} we give a complete formal specification of our voters as well as some basic definitions that we use in the following Sections. We describe the method in Section~\ref{sec_design} and give its formal description and proofs of  properties in Section~\ref{sec_results}. The conclusion is in Section~\ref{sec_conclusion}.

\section{Related work}\label{sec_related}

A totally self-checking TMR system with concurrent error location capability is presented in~\cite{Jiang1999}. The system determines whether an error occurred during voting as well as its location. The error coverage is 100\%, i.e., the error can be detected in the redundant modules, the voter, or the error-checking circuit. The work is compared to a similar scheme proposed in~\cite{Gaitanis1988}. Yet another technique for increasing the reliability of NMR voters based on error correction by Alternate-Data Retry is introduced in~\cite{Takaesu2004}.

While the focus in~\cite{Jiang1999}, \cite{Gaitanis1988} and~\cite{Takaesu2004} is locating the error by using special circuits that observe the outputs of the redundant modules and the voter, our primary target is establishing a design method for programmable NMR voters which besides self-checks, output additional information for the state of their inputs. In particular, here we pay special attention to the mathematical analysis of this method in order to confirm its validity and importance, and enhance its capabilities.

Design of a reconfigurable NMR system is introduced in~\cite{Lo1990}. The design method enables scalability regarding the number of redundant modules  and adaptability. Moreover, the authors in~\cite{Ruiz2008} present a strategy for automated generation of redundant modules and a corresponding majority voter. On the other side, the method that we present here enables not only simple but also elaborate NMR system generation (such as dynamic NMROD), using special NMR voters.

Dependability and performance analyses of NMR systems are given in~\cite{Srihari1982, Koren1979, Beaudry1978}, while dependability modeling of NMROD systems is found in~\cite{Al-Hashimi2001, Lombardi2001}.

\section{Basic definitions and voter specification}\label{sec_definitions}

An NMR system is practically determined by the properties and characteristics of the decision maker. As said, the most freqently used decision makers are various types of voters. We first give some basic definitions and make a short voter classification in order to set the frame for the following Sections. Then, we specify our type of voter.

Let the set of inputs of an NMR voter be . The absolute difference between the input values  and  wil be denoted by , i.e. . Exact voting algorithms consider  and  equal only if , while inexact voting algorithms allow defining a  parameter and consider  and  equal if . At last, approved voting algorithms define a set or range of approved input values. The voter considers  and  equal if they belong to the defined set/range. A complete voter classification with in-depth analyses is given in~\cite{Parhami1994}. Generally, the voters are marked by an -out-of- label denoting that voting is successful if there are at least M equal inputs of the  inputs in total. If , ambiguous situations may occur since more than one input values could be legitimate candidates for the voting output.

Although in this paper we mainly assume an exact 1-out-of- voter, the design method is general and could be applied for almost any voter type. Fig.~\ref{fig_voter} depicts our voter which reports its state and checks its own operation.

\begin{figure}[!h]
    \centerline{\includegraphics[width=0.5\linewidth]{045_programmable_NMR_voter}}
    \caption{Programmable NMR voter with self-report and self-checks}
    \label{fig_voter}
\end{figure}

The voting output  is equal to , where  is in the largest group of equal inputs. The  output gives the total number of inputs which differ from . Equivalently, the voter could use an  output, which gives the total number of inputs that are equal to the output of voting . Outputs  specify exactly which input  equals to , i.e. , if , and  if . The  output signals ambiguous input situations where  could be equal to any of the legitimate candidates for the voting outputs. The  output signals an unsuccessful self-check. Actually, outputs  (or ), , and  describe what is happening at the voter inputs. We refer to these outputs as the Input State Descriptor (ISD).

Furthermore, the voter is programmable. Each of the  inputs could be dynamically programmed to be an \textit{active input}, through a special signal . Each active input is included in the voting process, while all inactive inputs are excluded. This enables dynamically forming NMR systems (varying ) with any possible input combination. For example, defining  and  in a 4MR system, transforms the system to a 2MR system taking into account only modules 1 and 2; modules 0 and 3 do not participate in the voting process; later, however, module 3 may be included -- making a 3MR system. The programming signals imply two special configurations. Firstly, if no inputs are active, a 0MR system is formed, which is illegal. In this situation, the voter outputs are undefined. Secondly, if only one input is defined as active (1MR system),  is always equal to the active input . Thus, at least one input should be active for proper operation.

At last, but not least important is that the method enables scaling. That is, the complete voter with interface as in Fig.~\ref{fig_voter} could be generated solely by specifying the  parameter. Furthermore, the design method is general in the sense that the specific implementation could be done either in hardware, software or any other technology. In~\cite{Simevski2012a} and~\cite{Simevski2013b} we show hardware and software realization, respectively.

\section{Voter design method}\label{sec_design}

Our method is based on a binary matrix that reflects the equal inputs of the voter. The matrix enables determining the voting output and the ISD, and performing self-checks. 

\subsection{Matrix construction}\label{subsec_matrix}

Inherently, the set of voter inputs  might contain repeatable elements. Let  contain  different elements. If some element  is repeated  times in , then we say that \textit{the frequency of  in  is } (or simply, \textit{the frequency of  is }). Let  are all possible frequencies of the elements of ; then 

We construct the matrix , corresponding to the set , as follows:



(In the further text, we use the shorter notation  to express an integer index range from  to  with step 1. E.g.,  instead of .)

By  its definition, the matrix  is symmetric, with ones on its main diagonal.  If all voter inputs are different one from each other than  equals the identity matrix. In the opposite case, if all voter inputs are the same, than all matrix elements are ones. Additionally, the matrix  represents the relation ``='', defined on the set . As such, the matrix  represents equivalence  relation (that is, reflexive, symmetric and transitive).

\medskip
\begin{ex}\label{ex_matrix_build}
Let , , , , , and all inputs are  active (). The set  has three different elements (); their frequencies are . The corresponding matrix would be: 
\end{ex}

\medskip

The voter has additional  input signals that define which of its inputs  are active, thus dynamically setting the voter as 2MR, 3MR, \dots, NMR voter. The question is, how is this reflected into the  matrix. We consider each inactive input  to be different from all other inputs . Thus,  and .

\subsection{Construction of ISD}\label{subsec_isd}

Taking into consideration that the matrix  is symmetric, with main diagonal of ones, all information about the input state can be obtained from the elements above (or below) its main diagonal  i.e., the elements . The elements of the matrix above the main diagonal from Example~\ref{ex_matrix_build} are: 

By simply filling the missing  places with zeros (in the present case there are 6 such places, since ), we get a reduced  matrix:  where we preserve the enumeration of rows and columns ( and ) as in the  matrix for the elements above the main diagonal.

Now the ISD (signals ,  and ) could be simply determined as follows.

where the notation  is used for the cardinality of a given set . Actually, to find , we search all (incomplete) rows,  of the  matrix, to find a row , with the smallest number of zeros; then  and . We assign ,  for  and , for . Passing through columns  of row  we determine , for , with . The ambiguous signal   is set to 1 if more than one (incomplete) row with the same, smallest number of zeros are encountered, otherwise .

For instance, the smallest number of zeros in Example~\ref{ex_matrix_build} is in row . Thus, ,  (two zeros in row ), , . Passing through row 0 we determine  for , , , i.e., we distinguish which inputs are equal to the output of voting . The ambiguous signal is zero () since we have a single row in the matrix with the smallest number of zeros.

Probing many examples (with many variations of input sets) we observed that  is exactly equal to the largest eigenvalue of the  matrix and that all other eigenvalues are integers. The question is, does this property hold in general, and, is it possible to prove it?

\subsection{Construction of self-checks}\label{subsec_self_checks}

In our previous paper \cite{Simevski2012a} self-check construction was based on violations of the transitivity property of the equivalence relation represented by the matrix . More precisely, it was based on the misrepresentation in the matrix  (consequently in the matrix  of the following obvious property of the relation ``='' defined on the set of voter inputs.


\begin{ex}\label{ex_self_checks}
For ,  

\noindent Suppose that  is set to 0 instead of 1.  That is , respectively. The matrices would become

\noindent Here, it is obvious that in the matrix representation of the method, the transitivity is violated. The  matrix simultaneously states that , and , but also that .
\end{ex}

\medskip
When the transitivity is violated as described in the example \ref{eq_syllogism}, we say that the matrix is \textit{erroneously built}. Erroneously built matrices indicate one or more such errors in voting. The voter could use these matrices to do self-checks. For instance, the voter could check if transitivity~(\ref{eq_syllogism}) is satisfied for each  of the  matrix, and each  and  where  and . If the self-check passes, then  else . Nevertheless, this simple check of transitivity violation does not mean that the voter is 100\% operating correctly. It only tells if an error is present in the matrix information or not. In other words, errors in the voter parts that later use the matrix information may not be caught without additional checks.

As we did for properly built matrices, here too, we examined the eigenvalues of erroneously built matrices. In this case, all experiments indicated that they always have a non-integer eigenvalue. Thus, a challenge to deeply analyze the basic matrix upon which our method is built, was posed. 



\section{Theoretical aspects}\label{sec_results}

\newcommand{\be}{}
\newcommand{\ds}{\displaystyle}



In this Section we give proofs of assertions that were stated intuitively in~\cite{Simevski2012a} and state and prove new assertions related to the matrix.The Section is divided into three Subsections which treat the properties of a properly and an erroneously built matrix, as well as the relations of the characteristics of these matrices with the voter outputs.

\subsection{Characteristics of a properly built matrix}\label{properly_matrix}
\medskip













Some of the obvious properties of the properly built  matrix   were stated right after its definition. Another straightforward property is that the matrix has only real eigenvalues, since it is real symmetric and therefore  Hermitian. We found that the eigenvalues and eigenvectors of the  matrix give a lot of information about the NMR voter.





\medskip
Recall that  are all possible frequencies of the elements of  . Generality is preserved if  The following property serves as a basis for deriving the next few properties.

\medskip
\begin{property}\label{prop_similar}
The  matrix is similar to the block -- matrix
, where  is
 matrix whose elements are all ones.
\end{property}

\begin{proof}\label{proof_similar}
The  matrix is similar to the 
matrix, by the similarity  transformation , where
 is a product of a finite number of permutation matrices.
\end{proof}

Note that the matrix  represents the same set of voter inputs, but with reordered elements. The elements are listed such that the set starts with the same elements with highest frequency, followed by the same elements with non-increasing  frequencies.  

\medskip
\begin{ex}\label{ex_similar}
For the matrix from Example~\ref{ex_matrix_build}, we  have:

where

i.e.,

Here, the permutation matrix  is obtained from
 identity matrix, by interchanging 1-st and 2-nd row.
Multiplication  interchanges
1-st and 2-nd rows of  and multiplication by
 interchanges 1-st and 2-nd column of
. In this way, we get the
block -- matrix , with blocks of
ones on the main diagonal, with sizes ,  and
. The sizes of the blocks are exactly equal to the frequencies of the elements of
.
\end{ex}

\medskip
It is easy to see (according to the Sylvester criterion) that  is positive semi-definite. So is the  matrix~\cite{Meyer2000}, which implies that their eigenvalues are non-negative. So far, we know that the eigenvalues of  are non-negative reals. The following properties reveal, step-by step, the whole spectrum of .



\medskip
Let  and  denote the spectrum and spectral radius of , respectively.

\medskip
\begin{property}\label{prop_nula} .\end{property}

\begin{proof}\label{proof_nula}
Two similar matrices have the same determinant, thus

The relation  implies that  is an eigenvalue
of .
\end{proof}

\medskip
\begin{property}\label{theo_max}
. Moreover, , i.e., the largest frequency of the elements of  is an eigenvalue of .
\end{property}

\begin{proof}\label{proof_max}
(This proof is different than the one given in~\cite{Simevski2012a}.) As concluded before, all of the eigenvalues of , ,  are non-negative, so its spectral radius \be\label{ro}\rho(\mathbf{A})=\displaystyle\max_{i=1,2,\ldots,N}|\lambda_i|=\max_{i=1,2,\ldots, N}\lambda_i.\ee
The inequality \be\rho(\mathbf{A})\leq ||\mathbf{A}||\ee holds for every norm of  (\cite{Meyer2000}, p.~497). If we choose  -- norm, then we obtain \be\label{eq_pomalo}\rho(\mathbf{A})\leq ||\mathbf{A}||_1=\max_i\sum_j|A_{ij}|=f_{1},\ee since the largest absolute row sum in  is .

On the other hand,  is an eigenvalue of , since for the  -- dimensional  vector  the equality  is satisfied. (We use the fact that the similar matrices  and  have  the same eigenvalues.)

Taking into account the fact that  is an eigenvalue of , and the relations (\ref{ro}) and (\ref{eq_pomalo}), we conclude  that the maximal eigenvalue of  is .
\end{proof}



\medskip
\begin{property}\label{theo_fr}
All frequencies of the elements of  are eigenvalues of the  matrix.
\end{property}

\begin{proof}\label{proof_fr}
It is enough to show that  are eigenvalues of the  matrix.

For , we define -dimensional vectors  with 

It is easy to check that 
\end{proof}

The proofs of Properties~\ref{theo_max} and~\ref{theo_fr} contain
explicit  formulas of the eigenvectors of ,
that correspond to the eigenvalues of ,
i.e., the eigenpairs of  are . What about the
eigenpairs of ? Of course, the eigenvalues are the
same. We denote the eigenvectors of  corresponding to the
eigenvalues , by
 and give their description in the following property. 

\medskip
\begin{property}\label{theo_pos}
The eigenvectors of , , are 0-1 -dimensional vectors with ones at
the positions that coincide with the positions of the elements
with frequency  of the set .
\end{property}

\begin{proof}\label{proof_pos}
It can be easily checked that
 i.e., both
vectors,  and 
have -s at the positions of the elements with frequency .
All other components are zeros.
\end{proof}

\medskip
For the sake of clarity, we give the eigenvector  in its explicit form. Since  is a frequency of some element  of
, there exist ,  such that  The components of the vector
 are  and
the components of the vector  are  In
general, the eigenvector  gives information
about the ordinal numbers of the  elements (inputs) that have frequency
. At those positions  has ones. Other
elements of  are zeros.

{\it Remark:} Note that the rows (columns) of  are its eigenvectors.



\medskip
\begin{ex}
For the matrix from the Example~\ref{ex_matrix_build}, 
, ,   The eigenpairs
of  for the frequencies 
are: 
 and
 The eigenpairs of
 for the frequencies  are:

 and
 from which we read
the information that  is an element with frequency
,  and  are elements with frequencies
.
\end{ex}



\medskip
So far we showed that all frequencies of the elements of  and zero are eigenvalues  of . The question is, whether  has some other eigenvalues?  Before answering this question (the answer is actually the property \ref{theo_spektar}) we will compute the  determinants   and  defined for  and  by \be D_n(s)=\left|\begin{array}{rrrrrr}s&-1&-1&-1&\ldots&-1\\
-1&s&-1&-1&\dots&-1\\-1&-1&s&-1&\ldots&-1\\& & & &\ddots&
\\-1&-1&-1&-1&\ldots&s
\end{array}\right|\qquad \hbox{\rm and} \ee \be F_n(s)=\left|\begin{array}{rrrrrr}-1&-1&-1&-1&\ldots&-1\\
-1&s&-1&-1&\dots&-1\\-1&-1&s&-1&\ldots&-1\\& & & &\ddots&
\\-1&-1&-1&-1&\ldots&s
\end{array}\right|.\ee
We claim that
\be\label{eq_matind} D_n(s)=(s+1)^{n-1}(s-n+1),\ee
\be\label{eq_matind1} F_n(s)=-(s+1)^{n-1}.\ee

By cofactor expansion of both determinants along their first row, we get:
\be\label{dn} D_n(s)=sD_{n-1}(s)+(n-1)F_{n-1}(s),\ee\be\label{fn} F_n(s)=-D_{n-1}(s)+(n-1)F_{n-1}(s). \ee 
For ,  and , which corresponds to the formulas. Assuming that (\ref{eq_matind}) and (\ref{eq_matind1}) hold for  and taking into account (\ref{dn}) and (\ref{fn}), we obtain:

which, by means of mathematical induction, proves the formulas  (\ref{eq_matind}) and (\ref{eq_matind1}).

\medskip
\begin{property}\label{theo_spektar}, i.e., the spectrum of  matrix consists of  and
all frequencies of the elements of .\end{property}

\begin{proof}\label{proof_spektar}
We use the notation  to indicate   matrix
\be\label{eq_blok} (\lambda \mathbf{I}-
\mathbf{1})_{f_i}=\left[\begin{array}{rrrrrr}\lambda -1&-1&-1&-1&\ldots&-1\\
-1&\lambda-1&-1&-1&\dots&-1\\-1&-1&\lambda -1&-1&\ldots&-1\\& & &
&\ddots&
\\-1&-1&-1&-1&\ldots&\lambda -1
\end{array}\right].\ee

The  matrix is a block-diagonal matrix,  consisting of  blocks of sizes  (), of type~(\ref{eq_blok}). We give the characteristic polynomial of the , i.e.  matrix,  in its explicit form, using  the formulas  (\ref{eq_matind}) and (\ref{eq_matind1}) (substituting , ):

\end{proof}
The explicit form of the characteristic polynomial of the properly built matrix does not only give information about the spectrum of the  matrix, but also for the algebraic multiplicity of each eigenvalue. The importance of this fact will be elaborated later.

\subsection{Characteristics of an erroneously built matrix}\label{erroneously_matrix}
\medskip


In order to find the characteristic polynomial of the erroneously built matrix, we first define the  determinant   (where  and ) by:
\be Q_n(s)=\left|\begin{array}{rrrrrr}s&0&-1&-1&\ldots&-1\\
0&s&-1&-1&\dots&-1\\-1&-1&s&-1&\ldots&-1\\& & & &\ddots&
\\-1&-1&-1&-1&\ldots&s
\end{array}\right|.\ee Its value can be easily obtained by cofactor expansion along its first row, 

\be\label{eq_q} Q_n(s)=s(s+1)^{n-3}(s^2+(-n+3)s-n+4).\ee



\medskip
\begin{property}\label{theo_nontransitivity}
Let there exist three equal elements  If the following holds for the entries of the  matrix:
\be\label{eq_nontr}
A_{ij}=A_{ji}=1=A_{ik}=A_{ki}\,\wedge\, A_{jk}=0=A_{kj},\ee then it has a characteristic polynomial of the type

\end{property}

\begin{proof}\label{proof_nontransitivity}
Note that, since there should exist at least three equal voter inputs to consider transitivity at all, the frequency  should be greater or equal to three. Let the elements  have the frequency . Let the   matrix be erroneously
built, as described by (\ref{eq_nontr}). Then, there
exists a  matrix such that
 where  is a product of a finite number of permutation matrices,  and  is the  matrix
  Then the matrix  is a block-diagonal matrix
consisting  of blocks of type (\ref{eq_blok}) for all , and the  block . Thus, the
characteristic polynomial of  is:

Using (\ref{eq_matind}) and (\ref{eq_q}) (with substitutions , and  ) we obtain:

\end{proof}

\medskip
\begin{corollary}\label{cor_notnat} If the  matrix is erroneously built, then it has two non-integer eigenvalues.\end{corollary}

\begin{proof}
The roots of the characteristic polynomial are  1, all frequencies except ,  then 0 (if ), and the scalars  The last eigenvalues are non-integers, since  is non-integer for . We certify this with the inequality  that holds for  and  for 
\end{proof}

\medskip
Corollary~\ref{cor_notnat}  shows another way to  the voter how to do self-checks. Another useful fact in this direction is that the zero eigenvalue has algebraic multiplicity  for an erroneously built matrix, opposed to the algebraic multiplicity  for a properly built matrix. Similarly, for a properly built matrix, the eigenvalue 1 has algebraic multiplicity equal to the number of inputs with frequency 1 (including inactive inputs). For an erroneously built matrix, the algebraic multiplicity of 1 is bigger than this number for 1.



\medskip
\begin{ex}\label{pr5}
If  (, ) and the corresponding matrix  is erroneously built,  (the matrix implies that ), then

and 
The eigenvalues are  (since ),   (since  is always an eigenvalue of an erroneously built matrix, see Property~\ref{theo_nontransitivity}) and two non-integer eigenvalues 
\end{ex}

\medskip
\begin{ex}
If  (, ) and the corresponding matrix  is erroneously built,
 (the matrix implies that ), then
 and 
The eigenvalues are , with algebraic multiplicity 1; , with algebraic multiplicity  and two simple non-integer eigenvalues 
\end{ex}


\subsection{Matrix -- voter outputs relationship}\label{relations}
\medskip
At the end, we give the relations between the voter outputs (if the input set is ) and the scalar characteristic of a properly built matrix  corresponding to the set . The output  is actually the element  (recall the comment after Property ~\ref{theo_pos});  -- the largest eigenvalue of ; ;  are the components of the vector ; the ambiguous signal , if  is simple eigenvalue and  if   if there is a non-integer eigenvalue of , and  if all eigenvalues are non-negative integers.

In other words, the eigenvalues of  answer the questions like ``What are the frequencies of the inputs?'', ``What is the output of the voter?'' or ``Is the matrix erroneously built?''.  The corresponding eigenvectors answer the question ``What are the positions of the equal inputs (with the corresponding frequency)?''. The multiplicity of the largest eigenvalue answers the question ``Is there an ambiguity between the inputs?". Examples~\ref{ex7} and \ref{ex8} illustrate these issues.

\medskip
\begin{ex} \label{ex7} For the  matrix from Example~\ref{pr5} (because of its non-integer eigenvalues), the value of the error signal is ,  If it was properly built,

then 
and the eigenvectors of  matrix corresponding to  and , are  and 
We obtain: 
\end{ex}

\medskip
\begin{ex} \label{ex8} If , , then the corresponding matrix  is
 
Its characteristic polynomial is
 which means that its eigenvalues are 
and . The eigenvectors corresponding to the
frequency  are  and .
We obtain: 
\end{ex}

\section{Conclusion} \label{sec_conclusion}

Outlining our motivation in Section~\ref{sec_introduction} we gave several examples of sophisticated dependable NMR systems. They actually led us to a design method for programmable NMR voters that self-report their state and self-check their operation. The method is based on a binary matrix, which enables simplicity and scalability of the voter design. We got experimental results that foreshadowed interesting matrix properties, which in this paper were shown to be true by rigorous mathematical proofs.  We characterized the design method through the most important matrix characteristics -- the eigenvalues and eigenvectors. All exposed, theoretically-proven characteristics of the method enhance its possibilities in different applications. Although in hardware-realized NMR systems  is usually in the range from two to eight, in this paper we showed that the method is general and can be used to construct NMR systems for any natural number . A software realization for large  is given in~\cite{Simevski2013b}.

\begin{thebibliography}{10}

\bibitem{Simevski2012a}
A.~Simevski, E.~Hadzieva, R.~Kraemer, and M.~Krstic.
\newblock Scalable design of a programmable nmr voter with inputs' state
  descriptor and self-checking capability.
\newblock In {\em Adaptive Hardware and Systems (AHS), 2012 NASA/ESA Conference
  on}, pages 182--189, June 2012.

\bibitem{Simevski2014a}
A.~Simevski, R.~Kraemer, and M.~Krstic.
\newblock {Investigating core-level N-modular redundancy in multiprocessors}.
\newblock In {\em International Symposium on Embedded Multicore/Many-core
  Systems-on-Chip (MCSoC-14) , 2014 IEEE 8th International Symposium on},
  September 2014.

\bibitem{Simevski2013a}
A.~Simevski, R.~Kraemer, and M.~Krstic.
\newblock Register-transfer level nmr system generator.
\newblock In {\em Zuverlässigkeit und Entwurf - 7. ITG/GI/GMM-Fachtagung}. VDE
  Verlag GmbH - Berlin - Offenbach, September 2013.

\bibitem{Jiang1999}
Jianhui Jiang, Hongbao Shi, and Xiaodong Zhao.
\newblock A novel nmr structure with concurrent output error location
  capability.
\newblock In {\em Dependable Computing, 1999. Proceedings. 1999 Pacific Rim
  International Symposium on}, pages 32 --39, 1999.

\bibitem{Gaitanis1988}
N.~Gaitanis.
\newblock The design of totally self-checking tmr fault-tolerant systems.
\newblock {\em Computers, IEEE Transactions on}, 37(11):1450 --1454, nov 1988.

\bibitem{Takaesu2004}
Kohtaro Takaesu and Takeo Yoshida.
\newblock Construction of a fault-tolerant voter for n-modular redundancy.
\newblock {\em Electronics and Communications in Japan (Part II: Electronics)},
  87:62–71, December 2004.

\bibitem{Lo1990}
H.-Y. Lo, L.-P. Ju, and C.-C. Su.
\newblock General version of reconfiguration n modular redundancy system.
\newblock {\em Circuits, Devices and Systems, IEE Proceedings G}, 137(1):1 --4,
  feb 1990.

\bibitem{Ruiz2008}
J.-C. Ruiz, D.~de~Andres, S.~Blanc, and P.~Gil.
\newblock Generic design and automatic deployment of nmr strategies on hw
  cores.
\newblock In {\em Dependable Computing, 2008. PRDC '08. 14th IEEE Pacific Rim
  International Symposium on}, pages 265 --272, dec. 2008.

\bibitem{Srihari1982}
Sargur~N. Srihari.
\newblock Reliability analysis of biased majority-vote systems.
\newblock {\em Reliability, IEEE Transactions on}, R-31(1):117 --118, april
  1982.

\bibitem{Koren1979}
I.~Koren and S.Y.H. Su.
\newblock Reliability analysis of n-modular redundancy systems with
  intermittent and permanent faults.
\newblock {\em Computers, IEEE Transactions on}, C-28(7):514 --520, july 1979.

\bibitem{Beaudry1978}
M.D. Beaudry.
\newblock Performance-related reliability measures for computing systems.
\newblock {\em Computers, IEEE Transactions on}, C-27(6):540 --547, june 1978.

\bibitem{Al-Hashimi2001}
M.~Al-Hashimi, H.H. Pu, N.~Park, and F.~Lombardi.
\newblock Dependability under malicious agreement in n-modular
  redundancy-on-demand systems.
\newblock In {\em Network Computing and Applications, 2001. NCA 2001. IEEE
  International Symposium on}, pages 80 --91, 2001.

\bibitem{Lombardi2001}
F.~Lombardi, N.~Park, M.~Al-Hashimi, and H.H. Pu.
\newblock Modeling the dependability of n-modular redundancy on demand under
  malicious agreement.
\newblock In {\em Dependable Computing, 2001. Proceedings. 2001 Pacific Rim
  International Symposium on}, pages 68 --75, 2001.

\bibitem{Parhami1994}
B.~Parhami.
\newblock Voting algorithms.
\newblock {\em Reliability, IEEE Transactions on}, 43(4):617 --629, dec 1994.

\bibitem{Simevski2013b}
A.~Simevski and E.~Hadzieva.
\newblock Software implementation of programmable nmr voters.
\newblock In {\em Electronics, Telecommunications, Automatics and Informatics
  (ETAI), 2013 XI international conference on}, September 2013.

\bibitem{Meyer2000}
Carl~D. Meyer.
\newblock {\em {Matrix analysis and applied linear algebra}}.
\newblock SIAM, April 2000.

\end{thebibliography}


\end{document}
