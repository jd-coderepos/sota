

\documentclass{llncs}
\usepackage{times}
   \usepackage{mathptm}
\usepackage{epsfig}
\usepackage{color}
   \usepackage{latexsym}
   \usepackage{pslatex}






\newcommand{\procname}[1]{\textsc{#1} }
\newcommand{\ALGORITHM}{\textbf{algorithm} }
\newcommand{\AND}{\textbf{and} }
\newcommand{\ARRAY}{\textbf{array} }
\newcommand{\BEGIN}{\textbf{begin} }
\newcommand{\CASE}{\textbf{case} }
\newcommand{\ELSE}{\textbf{else} }
\newcommand{\ESAC}{\textbf{esac} }
\newcommand{\END}{\textbf{end} }
\newcommand{\DO}{\textbf{do} }
\newcommand{\FI}{\textbf{fi} }
\newcommand{\FOR}{\textbf{for} }
\newcommand{\FOREACH}{\textbf{for each} }
\newcommand{\IF}{\textbf{if} }
\newcommand{\IN}{\textbf{in} }
\newcommand{\OD}{\textbf{od} }
\newcommand{\OF}{\textbf{of} }
\newcommand{\OR}{\textbf{or} }
\newcommand{\OUT}{\textbf{out} }
\newcommand{\PRINT}{\textbf{print} }
\newcommand{\QUEUE}{\textbf{queue} }
\newcommand{\RECORD}{\textbf{record} }
\newcommand{\RETURN}{\textbf{return} }
\newcommand{\SET}{\textbf{set} }
\newcommand{\THEN}{\textbf{then} }
\newcommand{\TO}{\textbf{to} }
\newcommand{\TYPE}{\textbf{type} }
\newcommand{\UNTIL}{\textbf{until} }
\newcommand{\VAR}{\textbf{var} }
\newcommand{\WHILE}{\textbf{while} }
\newcommand{\CONTINUE}{\textbf{continue} }
\newcommand{\BREAK}{\textbf{break} }
\newcommand{\ENDWHILE}{\textbf{end while} }
\newcommand{\true}{\mathsf{true}}
\newcommand{\false}{\mathsf{false}}
\newcommand{\implies}{\longrightarrow}
\newcommand{\biimplies}{\Leftrightarrow}
\newcommand{\props}{\Pi}
\newcommand{\scope}{\mathbin{.}}

\begin{document}

\title{An Equivalence Preserving Transformation from \\ the Fibonacci to the Galois NLFSRs}
\author{Elena Dubrova}
\institute{Royal Institute of Technology (KTH), Electrum 229, 164 46 Kista, Sweden\\
\email{dubrova@kth.se}}

\maketitle

\begin{abstract}
Conventional Non-Linear Feedback Shift Registers (NLFSRs) use the
Fibonacci configuration in which the value of the first bit is updated
according to some non-linear feedback function of previous values of
other bits, and each remaining bit repeats the value of its previous
bit.  We show how to transform the feedback function of a Fibonacci
NLFSR into several smaller feedback functions of individual bits. Such
a transformation reduces the propagation time, thus increasing the
speed of pseudo-random sequence generation. The practical significance
of the presented technique is that is makes possible increasing the
keystream generation speed of any Fibonacci NLFSR-based stream cipher
with no penalty in area.
\end{abstract}

\noindent {\bf Keywords:} Fibonacci NLFSR, Galois NLFSR, pseudo-random sequence, 
keystream, stream cipher.

\section{Introduction}

Non-Linear Feedback Shift Registers (NLFSRs) have been proposed as an
alternative to Linear Feedback Shift Registers (LFSRs) for generating
pseudo-random sequences for stream ciphers. NLFSR-based stream ciphers
include Achterbahn~\cite{GaGK07}, Dragon~\cite{CHM05}, Grain~\cite{hell-grain},
Trivium~\cite{canniere-trivium}, VEST~\cite{cryptoeprint:2005:415},
and~\cite{GaGK06}. NLFSRs have been shown to be more resistant to
cryptanalytic attacks than LFSRs~\cite{1241371,Ca05}.  However,
construction of large NLFSRs with guaranteed long periods remains an
open problem. A systematic algorithm for NLFSR synthesis has not been
discovered so far. Only some special cases have been
considered~\cite{Golomb_book,Mykkeltveit77,MykkeltveitST79,Ro84,Ja89,Ro92,603256,838191,JaS04}.

In general, there are two ways to implement an NLFSR: in the Fibonacci
configuration, or in the Galois configuration.  The {\em Fibonacci}
configuration, shown in Figure~\ref{nlfsr_f}, is conceptually more
simple. The Fibonacci type of NLFSRs consists of a number of bits
numbered from left to right as  with feedback
from each bit to the th bit. At each clocking instance, the value
of the bit  is moved to the bit . The value of the bit 0
becomes the output of the register. The new value of the bit  is
computed as some non-linear function of the previous values of other bits.

\begin{figure}[t!]
\begin{center}
\resizebox{0.75\columnwidth}{!} {\begin{picture}(0,0)\includegraphics{fsr_date.xfig.pstex}\end{picture}\setlength{\unitlength}{3947sp}\begingroup\makeatletter\ifx\SetFigFont\undefined \gdef\SetFigFont#1#2#3#4#5{\reset@font\fontsize{#1}{#2pt}\fontfamily{#3}\fontseries{#4}\fontshape{#5}\selectfont}\fi\endgroup \begin{picture}(5049,1374)(3364,-2173)
\put(7426,-1936){\makebox(0,0)[lb]{\smash{\SetFigFont{12}{14.4}{\rmdefault}{\mddefault}{\updefault}{\color[rgb]{0,0,0}}}}}
\put(3751,-1936){\makebox(0,0)[lb]{\smash{\SetFigFont{12}{14.4}{\rmdefault}{\mddefault}{\updefault}{\color[rgb]{0,0,0}}}}}
\put(4726,-1936){\makebox(0,0)[lb]{\smash{\SetFigFont{12}{14.4}{\rmdefault}{\mddefault}{\updefault}{\color[rgb]{0,0,0}}}}}
\put(5701,-1936){\makebox(0,0)[lb]{\smash{\SetFigFont{12}{14.4}{\rmdefault}{\mddefault}{\updefault}{\color[rgb]{0,0,0}}}}}
\put(8010,-2092){\makebox(0,0)[lb]{\smash{\SetFigFont{12}{14.4}{\rmdefault}{\mddefault}{\updefault}{\color[rgb]{0,0,0}output}}}}
\put(5476,-1036){\makebox(0,0)[lb]{\smash{\SetFigFont{12}{14.4}{\rmdefault}{\mddefault}{\updefault}{\color[rgb]{0,0,0}feedback function}}}}
\end{picture}
 }
\caption{An Fibonacci type of NLFSR.}\label{nlfsr_f}
\end{center}
\end{figure}

In the {\em Galois} type of NLFSR, shown in Figure~\ref{nlfsr_g}, each
bit  is updated according to its own feedback function. Thus, in
contrast to the Fibonacci NLFSRs in which feedback is applied to the
th bit only, in the Galois NLFSRs feedback is potentially applied
to every bit. Since the next state functions of individual bits of a
Galois NLFSR are computed in parallel, the propagation time is reduced
to that of smaller functions of individual bits. This makes Galois
NLFSRs particularly attractive for stream ciphers application in which
high keystream generation speed is important.

However, Galois NLFSRs also have the following two drawbacks:
\begin{enumerate}
\item An -bit Galois NLFSR with the period of  does not 
necessarily satisfy the 1st and the 2nd postulates of
Golomb~\cite{DuTT08}. An -bit Fibonacci NLFSR with the period of
 always satisfy both postulates~\cite{Golomb_book}.
\item The period of the output sequence of a Galois NLFSR is not 
necessarily equal to the length of the longest cyclic sequence of its
consecutive states~\cite{DuTT08}. The period of a Fibonacci NLFSR always
equals to the longest cyclic sequence of its
consecutive states~\cite{Golomb_book}.
\end{enumerate}

These drawbacks do not create any problems in the linear
case because, for LFSRs, there exist a one-to-one mapping between 
the Fibonacci and Galois configurations. A Galois LFSR generating the
same output sequence as a given Fibonacci LFSR (and therefore
possessing none of the above mentioned drawbacks) can be obtained by
reversing the order of the feedback taps and adjusting the initial
state. For example, Figure~\ref{lfsr} shows the
Fibonacci and Galois configurations for the generator polynomial . If
the Fibonacci LFSR is initialized to the state  and the Galois
one is initialized to the state , then they generate the same
periodic sequence .

In the non-linear case, however, no mapping between the Fibonacci and
the Galois configurations has been known until now. The problem of
finding such a mapping is addressed in this paper.
We show that, for each Fibonacci NLFSR, there exist a class of
equivalent Galois NLFSRs which produce the same output sequence. 
We show how to transform a given Fibonacci NLFSR into an equivalent
Galois NLFSR. 




The most significant contribution of the paper is a sufficient
condition for equivalence of two NLFSRs before and after
the transformation. It is formulated and proved for the general case which
covers not only the equivalence between a Fibonacci and a Galois
NLFSRs, but only the equivalence between two Galois NLFSRs. 




The paper is organized as follows. Section~\ref{prel} describes main
notions and definitions used in the sequel.  Section~\ref{rec_cond}
formulates a sufficient condition for existence of a non-linear
recurrence describing the output sequence of an NLFSR.
Section~\ref{suf_cond} presents a sufficient condition for the
equivalence of two NLFSRs. In Section~\ref{most_dec}, we define a
 Galois NLFSR which is unique for a given Fibonacci
NLFSR and show how to compute it. 
Section~\ref{conc} concludes the
paper and discusses open problems.

\section{Preliminaries} \label{prel}

In this section, we describe basic definitions and notation used in
the sequel.

The {\em algebraic normal form (ANF)} of a Boolean function  is a polynomial in  of type

where  and  is the binary
expansion of  with  being the least significant bit.

The {\em dependence set} (or {\em support set}) of a Boolean function
 is defined by

where 
for .

Let  () be the smallest (largest)
index of variables in .


\begin{figure}[t!]
\begin{center}
\resizebox{0.99\columnwidth}{!} {\begin{picture}(0,0)\includegraphics{nlfsr_date.xfig.pstex}\end{picture}\setlength{\unitlength}{3947sp}\begingroup\makeatletter\ifx\SetFigFont\undefined \gdef\SetFigFont#1#2#3#4#5{\reset@font\fontsize{#1}{#2pt}\fontfamily{#3}\fontseries{#4}\fontshape{#5}\selectfont}\fi\endgroup \begin{picture}(6999,1470)(1864,-2548)
\put(3001,-2311){\makebox(0,0)[lb]{\smash{\SetFigFont{12}{14.4}{\rmdefault}{\mddefault}{\updefault}{\color[rgb]{0,0,0}}}}}
\put(5101,-2311){\makebox(0,0)[lb]{\smash{\SetFigFont{12}{14.4}{\rmdefault}{\mddefault}{\updefault}{\color[rgb]{0,0,0}}}}}
\put(4201,-2311){\makebox(0,0)[lb]{\smash{\SetFigFont{12}{14.4}{\rmdefault}{\mddefault}{\updefault}{\color[rgb]{0,0,0}}}}}
\put(2101,-2311){\makebox(0,0)[lb]{\smash{\SetFigFont{12}{14.4}{\rmdefault}{\mddefault}{\updefault}{\color[rgb]{0,0,0}}}}}
\put(7876,-2311){\makebox(0,0)[lb]{\smash{\SetFigFont{12}{14.4}{\rmdefault}{\mddefault}{\updefault}{\color[rgb]{0,0,0}}}}}
\put(6901,-2311){\makebox(0,0)[lb]{\smash{\SetFigFont{12}{14.4}{\rmdefault}{\mddefault}{\updefault}{\color[rgb]{0,0,0}}}}}
\put(8536,-2480){\makebox(0,0)[lb]{\smash{\SetFigFont{12}{14.4}{\rmdefault}{\mddefault}{\updefault}{\color[rgb]{0,0,0}output}}}}
\end{picture}
 }
\caption{A Galois type of NLFSR.}\label{nlfsr_g}
\end{center}
\end{figure}








Let  be a feedback function of the
bit , , of an NLFSR.  All results in this
paper as derived for NLFSRs whose feedback functions are {\em
singular} functions of type

where , , and the sign ``'' is modulo . Singularity guarantees
that the state transition graph of an NLFSR is ``branchless'',
i.e. that each state belongs to one of the state
cycles~\cite{Golomb_book}. 

Let  denote the value of the bit  at time .  The
sequence of states an -bit NLFSR with the singular feedback
functions can be described by a system of  non-linear equations of
type:
1mm]
s_{n-2}(t) = s_{n-1}(t-1) \oplus g_{n-2}(s_0(t-1), s_1(t-1), \ldots, s_{n-2}(t-1))\\
\ldots \\
s_0(t) = s_1(t-1) \oplus g_0(s_0(t-1), s_2(t-1), \ldots, s_{n-1}(t-1)).
\end{array}
\right.
 \label{e_rec}
s_i(t) = 
\sum_{j=0}^{2^n-1}  (a_j \cdot \prod_{k=0}^{n-1} s^{j_k}(t-n+k)),

s^{j_k}(t-n+k) = 
\left\{
\begin{array}{ll}
s(t-n+k), & \mbox{for} \ i = 1, \\
1,	 & \mbox{for} \ i = 0. \\
\end{array}
\right.

\left\{
\begin{array}{l}
s_3(t) = s_0(t-1) \oplus s_1(t-1) \oplus s_2(t-1) \oplus s_1(t-1)s_3(t-1), \\s_2(t) = s_3(t-1), \\
s_1(t) = s_2(t-1), \\
s_0(t) = s_1(t-1).
\end{array}
\right.

111011000101001 \ldots

s_3(t) = s_1(t-2) \oplus s_1(t-1) \oplus s_2(t-1) \oplus s_1(t-1)s_3(t-1).

s_3(t) = s_2(t-3) \oplus s_2(t-2) \oplus s_2(t-1) \oplus s_2(t-2)s_3(t-1).

s_3(t) = s_3(t-4) \oplus s_3(t-3) \oplus s_3(t-2) \oplus s_3(t-3)s_3(t-1).

\begin{array}{l}
f_3 = x_0 \oplus x_1 x_3, \\
f_2 = x_3, \\
f_1 = x_2, \\
f_0 = x_1 \oplus x_2 \oplus x_3.
\end{array}

s_3(t) = s_3(t-4) \oplus s_3(t-3) \oplus s_3(t-2) \oplus s_3(t-3)s_3(t-1).

p_{-1} = x_3 x_0 x_2, ~ p_{-2} = x_2 x_3 x_1, ~ p_{-3} = x_1 x_2 x_0. 

\begin{array}{l}
f_3 = x_0 \oplus x_1 x_3 \\
f_2 = x_3
\end{array}

\begin{array}{l}
f_3 = x_0 \\
f_2 = x_3 \oplus x_0 x_2.
\end{array}
 \label{e1}
\alpha_{max}(g_i) \leq \tau,

\begin{array}{l}
f_3 = x_0, \\
f_2 = x_3, \\
f_1 = x_2 \oplus x_0 \oplus x_3, \\
f_0 = x_1 \oplus x_0 x_2.
\end{array}
 \label{e3}
b \geq a - \alpha_{min}(p).
 \label{d_tau}
\begin{array}{lccl}
\tau & = & max & (\alpha_{max}(p)-\alpha_{min}(p)), \\
     & & \forall p \in P_{g_{n-1}} & \\
     & & \mbox{with} |p| > 1 & 
\end{array}

g_{n-1} \stackrel{p}{\longrightarrow} g_{n-1-\alpha_{min}(p)}.

g_{n-1} \stackrel{p}{\longrightarrow} g_{\tau}.

\begin{array}{ll}
f_{31} & = \ x_0 \oplus x_2 \oplus x_6 \oplus x_7 \oplus x_{12} \oplus x_{17} \oplus x_{20} \oplus x_{27} \oplus x_{30} \oplus x_3 x_9 \oplus \ x_{12} x_{15} \oplus x_4 x_5 x_{16}
\end{array}

\begin{array}{l}
f_{31} = x_0 \\
f_{29} = x_{30} \oplus x_0 \\
f_{28} = x_{29} \oplus  x_0 x_6  \\
f_{27} = x_{28} \oplus  x_0 x_1 x_{12} \\
f_{25} = x_{26} \oplus x_0 \\
f_{24} = x_{25} \oplus x_0 \\
f_{19} = x_{20} \oplus x_0 \oplus x_0 x_3 \\
f_{14} = x_{15} \oplus x_0 \\
f_{12} = x_{13} \oplus  x_1 \oplus  x_8 \oplus x_{11} \\
\end{array}

\begin{array}{l}
f_{31} = x_0 \\
f_{29} = x_{30} \oplus x_0 \\
f_{28} = x_{29} \oplus  x_0 x_6  \\
f_{27} = x_{28} \oplus  x_0 x_1 x_{12} \\
f_{25} = x_{26} \oplus x_0 \\
f_{24} = x_{25} \oplus x_0 \\
f_{20} = x_{21} \oplus x_1 x_4 \\
f_{19} = x_{20} \oplus x_0  \\
f_{16} = x_{17} \oplus  x_{12} \\
f_{14} = x_{15} \oplus x_0 \\
f_{13} = x_{14} \oplus  x_{12} \\
f_{12} = x_{13} \oplus  x_1  \\
\end{array}

\left\{
\begin{array}{l}
s_{n-1}(t) = s_0(t-1) \oplus g_{n-1}(s_0(t-1),s_1(t-1),\ldots,s_b(t-1)) \\
\ldots \\
s_a(t) = s_{a+1}(t-1) \oplus g^*_a(s_0(t-1),s_1(t-1),\ldots,s_b(t-1)) \oplus s_k(t-1) s_a(t-1)\
Since  for , each
 does not depends of . However, we keep this
redundant term in the equations in order to be able to later introduce
the same abbreviations for all .

Note, that each of  depends on variables
with indexes smaller or equal than  only since, by assumption, the
condition~(\ref{e1}) holds after the shifting.

A substitution  is equivalent to replacing the
variable  in the equation of each successor of  by
. After the sequence of  substitutions
, each  gets replaced
by , so the above equations reduce to:
1mm]
	& \oplus \ g^*_a(s_a(t-a-1),s_a(t-a),\ldots,s_a(t-1-(a-b))) \

To shorten the expressions, let us introduce an abbreviation
 and let
the notation  mean that each element 
 of is replaced by . For example,
.
Then, the above equations can be re-written us:



After a sequence of  substitutions ,
we get a non-linear recurrence describing the sequence of values of the bit :
1mm]
         & \oplus \ \ldots \oplus g^*_a(\tilde{s}_a)+s_a(t-1-a+i) s_a(t-1)
\end{array}
 \label{rec_a}
\begin{array}{ll}
s_a(t) \ & = \ s_a(t-n) \1mm]
 	 & \oplus \ g_{n-2}(s_a(t-n-1),s_a(t-n),\ldots,s_a(t-n+b-1)) \\
	 & \ldots \\
         & \oplus \ g^*_a(s_a(t-a-1),s_a(t-a),\ldots,s_a(t-1-a+b)) \


On the other hand, the NLFSR after the shifting can be represented by the
following system of equations:
1mm]
s_{a-1}(t) = s_a(t-1) \\
\ldots \\
s_b(t) = s_{b+1}(t-1) \oplus s_{i-(a-b)}(t-1) s_b(t-1) \\
\ldots \\
s_0(t) = s_1(t-1) \\
\end{array}
\right.

\left\{
\begin{array}{l}
s_{n-1}(t) = s_b(t-b-1) \oplus g_{n-1}(s_b(t-b-1),s_1(t-b),\ldots,s_b(t-1)) \\
\ldots \\
\ldots \\
s_a(t) = s_{a+1}(t-1) \oplus g^*_a(s_b(t-b-1),s_1(t-b),\ldots,s_b(t-1)) \

Introducing an abbreviation 
we can re-write the above equations us:
1mm]
s_{a-1}(t) = s_a(t-1) \\
\ldots \\
s_b(t) = s_{b+1}(t-1) \oplus s_b(t-1+i-a) s_b(t-1)
\end{array}
\right.

\begin{array}{ll}
s_b(t) \ & = \ s_b(t-n) \oplus g_{n-1}(\tilde{s}_b(-n+b+1)) \oplus g_{n-2}(\tilde{s}_b(-n+b)) \

After expanding the abbreviation , the above recurrence becomes:
1mm]
         & \oplus \ g_{n-1}(s_b(t-n),s_b(t-n+1),\ldots,s_b(t-n+b)) \1mm]
         & \oplus \ s_b(t-1-a+i) s_b(t-1)
\end{array}


The non-linear recurrences~(\ref{rec_a}) and (\ref{rec_b}) are the same,
so two NLFSRs are equivalent.
\begin{flushright}

\end{flushright}

\end{document}
