\documentclass[]{article}
\pdfoutput=1 
\providecommand{\keywords}[1]{\textbf{\textit{Key words---}} #1}
\bibliographystyle{elsarticle-num}
\usepackage{lineno,hyperref}
\usepackage{amsmath}
\usepackage{amsfonts}
\usepackage{amssymb}
\usepackage{mathrsfs}
\usepackage[pdftex]{graphicx}
\usepackage{indentfirst}

\usepackage{lipsum} 

\usepackage[sc]{mathpazo} \usepackage[T1]{fontenc} \usepackage{lmodern}
\linespread{1.05} \usepackage{microtype} 

\usepackage[hmarginratio=1:1,top=32mm,columnsep=30pt]{geometry} \usepackage[hang, small,labelfont=bf,up,textfont=it,up]{caption} \usepackage{booktabs} \usepackage{float} \usepackage{hyperref} 

\usepackage{lettrine} \usepackage{paralist} 

\usepackage{abstract} 

\usepackage{titlesec} \renewcommand\thesection{\Roman{section}} \renewcommand\thesubsection{\Alph{subsection}} \titleformat{\section}[block]{\large\scshape\centering}{\thesection.}{1em}{} \titleformat{\subsection}[block]{\large}{\thesubsection.}{1em}{} 





\title{\vspace{-15mm}\fontsize{24pt}{10pt}\selectfont\textbf{Integral Control on Lie Groups}} 

\author{
\large
{Zhifei Zhang\thanks{Corresponding author,~E-mail:~zhifei.zhang@ugent.be}~~~Alain Sarlette~~~Zhihao Ling}\\xi^l = [\omega^l]^\wedge = \begin{bmatrix}
    0 & -\omega_3 & \omega_2 \\
    \omega_3 & 0 & -\omega_1\\
    -\omega_2 & \omega_1 & 0\\
\end{bmatrix}
\Leftrightarrow [\xi^l]^{\vee} = \omega^l = \begin{bmatrix} \omega_1 \\ \omega_2 \\ \omega_3 \end{bmatrix}
g=\begin{bmatrix}
    R & p \\
    0_{1\times 3} & 1 \\
\end{bmatrix}\xi^l = g^{-1} \tfrac{d}{dt}g = \begin{bmatrix}
    [\omega^l]^{\wedge} & v^l \\
    0_{1\times 3} & 0 \\
\end{bmatrix}\xi^l_p = -k_P \text{grad}^l\phiL_{g^{-1}}\tfrac{d}{dt}g = \xi^l \;\; , \quad \tfrac{d}{dt}\xi^l = F^lF^l_p = -k_P \text{grad}^l\phiF^l_d = -k_D \xi^l u_I(t) = \int_0^t \, T_{(x(\tau),x(t))}[u_{PD}(\tau)] \, d\tau \, , \tfrac{d}{dt} u_I(t) = u_{PD}(t) + DT_{dx/dt}[u_I(t)]
\label{eqmy1}
L_{g^{-1}}\frac{d}{dt} g &=& -k_p\, \text{grad}^l \phi + k_i\xi^l_i + \xi^l_B\,,\label{eq:1-st integral control}\\
\label{eqmy2}
\frac{d}{dt}\xi^l_i &=& -k_p\, \text{grad}^l \phi \;.\label{eq:1-st integral term}
\label{eq:1st Lyapunov function}
V\,=\,\alpha\phi + \frac{1}{2}\beta\parallel k_i \xi_i^l + \xi^l_B\parallel^2\;,

L_{g^{-1}}\frac{d}{dt}g & = & \xi^l \label{eq:2-nd integral control velocity}\\
\frac{d}{dt}\xi^l & = & -k_p \text{grad}^l \phi - k_d\xi^l + k_iF^l_i + F^l_B \label{eq:2-nd integral control force}\\
\frac{d}{dt}F^l_i & = & -k_p \text{grad}^l \phi - k_d\xi^l \;.\label{eq:2-nd integral term}
\label{eq:2-nd Lyapunov function}
V\,=\,\alpha\phi + \frac{1}{2}\beta\parallel\xi^l\parallel^2 + \frac{1}{2}\gamma\parallel k_i (F^l_i-\xi^l) + F^l_B\parallel^2\;.
\nonumber
\begin{split}
\dot{V} = & -\left(-\frac{1}{4\gamma k_i}\beta^2 + \left(k_d - \frac{1}{2}k_i\right)\beta-\frac{1}{4}\gamma k_i^3\right)\Vert\xi^l\Vert^2 \\
		 &-\left\Vert\frac{1}{2}\frac{\beta+\gamma k_i^2}{\sqrt{\gamma k_i}}\xi^l - \sqrt{\gamma k_i}(k_iF^l_i + F^l_B)\right\Vert^2\;.
\end{split}
\label{eq:BetaIdentifier}
\frac{\left(k_d-k_i/2\right)-\sqrt{\Delta}}{1/(2\gamma k_i)} < \beta < \frac{\left(k_d-k_i/2\right)+\sqrt{\Delta}}{1/(2\gamma k_i)}

\label{eq:Rg} R_{g^{-1}}\frac{d}{dt} g & = & -k_p \text{grad}^r \phi + k_i\xi^r_i + \xi^r_B\\
\nonumber \frac{d}{dt}\xi^r_i & = & -k_p \text{grad}^r \phi \;.

\label{eq:Rghop} 
\begin{split}
R_{g^{-1}}\frac{d}{dt} g \;& =\;  Ad_{g(t)}(-k_p \text{grad}^{l*} \phi + k_i\xi^l_i) + \xi^r_B \\
\frac{d}{dt} \xi^l_i \;& =\;   -k_p \text{grad}^{l*} \phi - [\xi^l, \xi^l_i] \; ,
\end{split}

\label{eq:Rg2hop}
\begin{split}
R_{g^{-1}}\frac{d}{dt}g \;& = \; \xi^r \\
\frac{d}{dt}\xi^r \;& = \; Ad_{g(t)}(-k_p \text{grad}^{l*} \phi - k_d\xi^l + k_iF^l_i) + F^r_B\\
\frac{d}{dt}F^l_i \; & = \; -k_p \text{grad}^{l*} \phi - k_d\xi^l - [\xi^l, F^l_i] \;.
\end{split} 

\label{eq:xibar} 
\begin{split}
\frac{d}{dt} \xi^l_i \; & = \; -K_p \text{grad}^{l*} \phi - [\bar{\xi}^l, \xi^l_i] \; \text{ with}\\
\bar{\xi}^l \; & = \; -K_p \text{grad}^{l*} \phi \; .
\end{split}

\text{grad}^l\phi \;=\; [\text{skew}(Q)]^\vee = [\tfrac{1}{2}(Q-Q^T)]^\vee

\begin{split}
Q^{T}\frac{d}{dt} Q \;& = \; \left[ -k_p\text{skew}(Q)^\vee + k_i\omega^l_i + \omega^l_B \right]^\wedge
\,,\\
\frac{d}{dt}\omega^l_i \;& = \; -k_p\, [\text{skew}(Q)]^\vee \;.
\end{split}

\begin{split}
Q^T\frac{d}{dt}Q \;& = \; [\omega^l]^\wedge \\
\frac{d}{dt}\omega^l \;& = \; -k_p [\text{skew}(Q)]^\vee - k_d\omega^l + k_iF^l_i + F^l_B\\
\frac{d}{dt}F^l_i \;& = \; -k_p [\text{skew}(Q)]^\vee - k_d\omega^l \;.
\end{split}

\phi_1(Q,p)=\frac{1}{2} \text{tr}(I_{3\times 3}-Q)+\frac{1}{2}\parallel p\parallel^2\;.

g^{-1}\frac{d}{dt} g & = & -k_p\begin{bmatrix}
\text{skew}(Q) & Q^T p \\ 0 & 0
\end{bmatrix} \\
& & + k_i\begin{bmatrix}
[\omega^l_i]^\wedge & v^l_i \\ 0 & 0
\end{bmatrix} + \begin{bmatrix}
[\omega^l_B]^\wedge & v^l_B \\ 0 & 0
\end{bmatrix}\,,\\
\frac{d}{dt}\begin{bmatrix}
\omega^l_i\\
v^l_i
\end{bmatrix} & = & -k_p\begin{bmatrix}
[\text{skew}(Q)]^\vee\\
Q^T p
\end{bmatrix} \;.

Simulation results are shown in Fig.~\ref*{SE(3)_1st_Trajectory}, representing only the position part of . In addition to the parameters already used for rotation, we take ,  and . We have plotted the ideal trajectory of P control \emph{without bias} as a reference. In presence of bias, under \emph{P control} the position moves in a wrong direction, converges to a stable point  and stays there with a steady-state error whose gradient pull compensates the bias. Under \emph{PI control}, the bias still starts the system in the wrong direction, but once the integral term takes over it converges back to the desired equilibrium \, and stays there, while the bias is countered by an integral term  (see Fig.~\ref*{SE(3)_1st_IntegralLyapunov}). 

\begin{figure}[thpb]
 	\centering
 	\makebox{\parbox{3.2in}{\includegraphics[scale=0.54]{1SE3Trajectory}}}
 	\caption{Trajectories of different control strategies for first-order system on  --- only the position part of  is represented}
 	\label{SE(3)_1st_Trajectory}
\end{figure}

\begin{figure}[thpb]
 	\centering
 	\makebox{\parbox{3.2in}{\includegraphics[scale=0.54]{1SE3IntegralLyapunov}}}
 	\caption{Integral term and Lyapunov function for first-order system on }
 	\label{SE(3)_1st_IntegralLyapunov}
\end{figure}

The second-order case follows exactly the same principles. Simulations can be easily established with the corresponding parameters taken over from previous cases. In accordance with Proposition 2, the system shall converge to the equilibrium where  and , while the bias in actuators is countered by  and \;.

 
Besides calibration errors or actuator leakage, a bias on the underwater vehicle could be caused by slow (errors in cancellation of) internal dynamics. Also a constant bias in \emph{inertial} frame would make sense, e.g.~caused by ocean flow (see extensions Section \ref{ssec:extensions}). The second-order system is then covered by Corollary 3, but for the first-order model  does not satisfy the requirement of unitary adjoint representation for Proposition 4. Realistic settings also include the nonholonomic ``steering control'' case, which is worth future interest. 




\section{CONCLUSIONS}\label{conclusion} We propose a general integral control method for systems on nonlinear manifolds by explicitly defining the integral term as the integration of the control commands in the corresponding transported tangent spaces. In particular, for Lie groups, the transport maps associated to left and right group actions are a natural choice. Under this rigorous definition, we can easily extend PID control from Euclidean space to Lie groups. Both first order integrators with bias in velocity and second order integrators with bias in controlled torque are shown to be well corrected by applying our integral control. Stability is proved with Lyapunov functions. We also take typical applications in robotics as examples to illustrate the physical meanings of the setting and developments, and to confirm the stability in simulation results. As for linear systems, the potential advantage of PID control over the observer-based approach to bias rejection on Lie groups is that PID controllers do not have to simulate and hence know the full dynamical model of the system. For instance, it can be expected that the stability proven here on simple examples remains valid more or less verbatim if actuator dynamics are added to the system. Future research should investigate to which underactuated contexts the approach could be adapted, especially when left-invariant control (i.e.~inputs constrained in body frame) is combined with right-invariant constant biases (i.e.~forces/torques/flows attached to inertial frame). Another opened research direction is more explicit integral control for Riemannian manifolds, i.e.~investigating meaningful transport maps both for applications and regarding convergence properties. In this regard, we already note that the equivalent of a ``constant bias'' cannot be defined on all manifolds, as e.g.~the even-dimensional spheres cannot support non-vanishing smooth vector fields~\cite[Th.2.2.2]{burns2005differential}. The implications of our integral controller for robust coordinated motion should also be investigated.

\section*{Acknowledgment}

This paper presents research results of the Belgian Network DYSCO  
(Dynamical Systems, Control, and Optimization), funded by the  
Interuniversity Attraction Poles Programme, initiated by the Belgian  
State,  Science Policy Office. The first author's visit to the SYSTeMS research group is supported by CSC-grant (No.201306740021) and the associated BOF-cofunding, initiated by the China Scholarship Council and Ghent University respectively.


\bibliography{icol}


\end{document}
