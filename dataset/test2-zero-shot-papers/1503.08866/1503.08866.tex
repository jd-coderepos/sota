\documentclass{llncs}
\usepackage{makeidx}  \usepackage{subfigure}
\usepackage{multirow}
\usepackage{epsfig} \usepackage{color}
\usepackage{booktabs}
\usepackage{amsmath} \usepackage{amssymb}  \usepackage{epstopdf}
\begin{document}

\title{Towards Data-Driven Hierarchical Surgical Skill Analysis}
\author{Bin Li \inst{1} \and B\`{e}r\`{e}nice Mettler \inst{1} \and Timothy M. Kowalewski \inst{2}}
\institute{ (1) Department of Aerospace Engineering and Mechanics and (2) Department of\\Mechanical Engineering, University of Minnesota, Minneapolis, MN, 55455, USA\\
{\small \{lixx1778, mettler, timk\}.umn.edu} }

\maketitle

\begin{abstract}
This paper evaluates methods of hierarchical skill analysis developed in aerospace to the problem of surgical skill assessment and modeling. The analysis employs tool motion data of Fundamental of Laparoscopic Skills (FLS) tasks collected from clinicians of various skill levels at three different clinical teaching hospitals in the United States.
Outcomes are evaluated based on their ability to provide relevant information about the underlying processes across the entire system hierarchy including control, guidance and planning.
\end{abstract}
\section{Introduction}

Over 32,000 deaths and \g_kx_\text{ref}NN66[.59, .73]615^{th}D(p||q) = [D_{KL}(p||q) + D_{KL}(q||p)]/2e_{m,n}{D(P_{grp_m \backslash subj_i }||P_{grp_n \backslash subj_j })}, \forall i \ne j\Delta t$ is the sampling time~\cite{Li2013}. Each interaction pattern is identified as a model of different set of parameters.
The PWARX results are shown in Figures~\ref{Fig:pwa_novice} and~\ref{Fig:pwa_expert}, where three clusters are identified for both novice and expert groups. To make the analysis more intuitive, the PWARX parameters are transformed into speed, normal and tangential acceleration in the ellipsoid plot.

\begin{figure}[th]
  \centering
  \begin{minipage}{.495\textwidth}
\centering
    \subfigure[Trajectory]{
      \includegraphics[width=0.45\columnwidth]{PWA_Traj_Novice_Left.png}
      \label{Fig:pwa_traj_novice}
    }
    \subfigure[Ratio]{
      \includegraphics[width=0.45\columnwidth]{PWA_Ratio_Novice_Left.png}
      \label{Fig:pwa_ratio_novice}
    }
    \vspace{-0.6em}
    \caption{Dynamic clustering of novices}
    \label{Fig:pwa_novice}
  \end{minipage}
  \begin{minipage}{.495\textwidth}
    \centering
    \subfigure[Trajectory]{
      \includegraphics[width=0.45\columnwidth]{PWA_Traj_Expert_Left.png}
      \label{Fig:pwa_traj_expert}
    }
    \subfigure[Ratio]{
      \includegraphics[width=0.45\columnwidth]{PWA_Ratio_Expert_Left.png}
      \label{Fig:pwa_ratio_expert}
    }
    \vspace{-0.6em}
    \caption{Dynamic clustering of experts}
    \label{Fig:pwa_expert}
  \end{minipage}
\end{figure}

\section{Hierarchical Skill Assessment Results}

\subsection{Subgoal closure}
As suggested in the introduction, dynamics-based clustering reveals spatial organization abilities of expert surgeons in performing Peg Transfer task, and makes it possible to delineate the different phases of the task and therein analyze specific performance characteristics. In Figure~\ref{Fig:pwa_traj_expert}, the spatial organization of the behavior of expert surgeons is closely correlated with the three phases in Peg Transfer task:
\begin{enumerate}
  \item Starting phase (cluster mode 2) coincides with the surgeons picking up the blocks. The movement during this phase follows a medium velocity range.
  \item Maneuvering phase (cluster mode 1) coincides with the surgeons moving the gripped blocks to the central area of the board. There is no restriction on the movement during the maneuvering phase, and the objective of the phase is to be as fast as possible. Therefore the surgeons adopt high velocity and the accelerations span a large range.
  \item Interception phase (cluster mode 3) coincides with the blocks being transferred in the air between two hands of laparoscopic tool. This phase is critical in that it
requires a large amount of coordination effort for both hands.
\end{enumerate}

For each phase of the task, expert surgeons adopt very consistent strategy. In contrast, the maneuvers of novices are less consistent. During the starting phase, novices sometimes drive the control to high velocity, which penalizes the accuracy. The lack of consistency in the strategy also demands more attention load to plan for new trials and to handle the range of conditions. More attempts are required for novices to successfully pick up the blocks. In the maneuvering phase, the laparoscopic tool frequently slows to a lower velocity, penalizing the completion speed of the task.

\subsection{High-level planning}

To facilitate the accomplishment of complex task, humans divide the task into subtasks. In~\cite{KongMettler13agentenvironmentinteractions}, Kong and Mettler have shown that subtasks exploit invariants in the dynamic environment interactions. The invariants in the human behavior emerge through extensive practice ostensibly as a result of the assimilation of coordinated movement and perceptual processes in procedural memory. These interaction pattern can then be used as a unit of behavior for the larger organization. High-level planning can therefore be assessed from the organization of interaction patterns. Effective planning allows using interaction patterns that take advantage of the dynamic interaction between human\rq{}s motor skills and task elements that also reduce the attention load. For this reason, spatial organization of the interaction pattern is an important measure of the surgeons' planning skill. To quantify the spatial organization, the Cartesian coordinates of trajectory points are classified using a Fisher classifier based on the tags obtained in PWA clustering.
The misclassification ratio is then used as the measure of spatial organization, as shown in Table~\ref{Tab:spatial_org}.

\begin{table}
  \centering
  \caption{Skill metric in high-level planning}\label{Tab:spatial_org}
  \begin{tabular}{c|ccc}
    \hline
Spatial organization [\%] & Expert & Intermediate & Novice \\ \hline
      Complete Groups & 17.9 & 29.6 & 38.6\\
      Leave-One-Out Mean(std dev) & 13.3  (4.4)  & 27.1 (5.6)  &  35.0 (6.5) \\
\end{tabular}
\end{table}

\section{Conclusion} \label{sec:conclusion}
The results underscore the limitations of simple outcome measures such as those obtained from kinematic characteristics (see Section \ref{sec:TaskStats} and \ref{sec:KinematicClassification}) and on the other hand demonstrates the discriminative power of dynamical characteristics obtained here using a PWARX model. The latter provides a more detailed segmentation and insights into the dynamic make-up of the behavior and their spatial organization. This more detailed information provides correlation with important procedural movement stages, yet it requires no prior, high-level knowledge about the task to be implemented. Finally, the spatial characteristics of the segmented performance data provides a measure of the ability to organize the different stages of behavior in a manner which is consistent with the spatial and dynamic constraints of the task and operator skills. These results demonstrate that dynamical segmentation techniques can access attributes across the entire process hierarchy and provide the foundation to a more comprehensive skill analysis and modeling framework. Future work will immediately extend this approach to more FLS tasks and ultimately to different tasks in laparoscopic, robotic, and open surgery and incorporate gaze characteristics.

\bibliographystyle{IEEEtran}
\bibliography{M2CAI_Surgery_Skill_Analysis}

\end{document}
