This section describes the new packet marking scheme, probabilistic
congestion notification (PCN).  This scheme
uses only a single bit, does not require per-flow state at the router
and produces an estimate
for the congestion level at each queue the traffic encounters on its
outward
journey.

\subsection{Protocol at router and end hosts}
\label{sec:protocol}

The PCN scheme marks a single bit in the packet header, this is used
with statistical methods to estimate the congestion of any of several
possible intermediate routers.  PCN relies on two fields in
the IP header: the IP identifier (IPid) and the time to live (TTL)
fields.
These are initialised by the original source of the packet
and the TTL field is reduced by one at every hop on the path.

The marking scheme at the router is both stateless and extremely
simple.
Assume each intermediate router has a current load factor
(\ref{eqn:pcnlf}) which the source wishes to estimate. The load factor
() is in the range  [0,100]
(values above 100 are rounded down).  This range is not a necessary
assumption of
the algorithm as discussed later.

Assume that there are at most  intervening routers 
-- initially this is fixed at 32 for all connections.
It is not a problem for the algorithm if there are fewer intervening
routers (but it does cause loss of efficiency).  If there are more
then some information will be split between routers
(see later
discussion).
The strategy
is to allow each outgoing packet to have its ECN bit set by at most
one of
the intervening routers.  This ECN bit is set in a probabilistic
manner
governed by the load factor in such a way that the ratio of marked
packets
to total packets is equal to the load factor (divided by 100).
Whether the ECN bit is marked or not marked
is communicated back to the source on the acknowledgement (ACK) for
the outgoing packet
(so the receiver simply has to copy the state
of the ECN bit onto the ACK).  The IPid and TTL are used to determine
which router can mark the packet.

Consider the condition

By
definition the IPid remains constant and
TTL decreases by one at each hop.  Therefore,
if there are at most  intervening routers then for only one such
router will this condition be true.  Only the router for which
(\ref{eqn:ttlcond}) is true may set the ECN bit on a packet.
Define a packet as {\em markable\/} by router  if
(\ref{eqn:ttlcond}) is true for that packet for router .
The source knows the IPid and TTL for every packet and
can calculate for which router a given packet was markable.

The router for which condition (\ref{eqn:ttlcond}) is met marks
the packet with a probability equal to the load factor.
The expectation value of the proportion of markable packets with
the ECN bit set by router  is equal to its load factor (while
the load factor remains constant -- see section \ref{sec:prediction}
for more discussion of this issue).
Therefore, the proportion of ACKs from packets
markable by router  which have their ECN
bit set is an unbiased estimate of the load factor.

The full protocol for PCN can be simply given as follows.
\begin{itemize}
\item The source sets the ECN bit to zero, TTL sets to , and
increments the IPid by one
for each packet.
\item Intermediate routers, if condition (\ref{eqn:ttlcond}) is met,
set
the ECN bit with a probability equal to their load.
\item The receiver copies the ECN bit from a packet onto that packet's
ACK (there must be one ACK for every packet).
\item The source tracks the ECN bits on ACKs to estimate congestion
on intermediate routers.
\end{itemize}

This algorithm replaced ECN at routers and cannot coexist with it
(routers performing standard ECN marking will confuse PCN estimates).
It is this final part which enables the source to estimate congestion
and which will be a main focus of the results in this paper.  Section
\ref{sec:prediction} shows how time series modelling techniques can
be used to improve the estimate of congestion level and section
\ref{sec:results} shows ns-2 modelling results which prove the scheme
practical for realistic estimation scenarios.

The value  was chosen since \cite{hopCount}
shows that a hop count of more than 30 is extremely rare in the
real Internet.  However, if there are more than  routers the
protocol's failure mode is not a major issue although some packets may
be marked by more than one router.

If there are less than  intervening routers some packets
are not markable by any and an
opportunity to get data is lost.
A possible improvement is to pre-signal the actual
number of intervening routers (by communicating the TTL of the SYN
packet).  The PCN source can ensure that only packets
which have condition (\ref{eqn:ttlcond}) met for some intervening
router are sent.  This will increase the number of samples at each
router.
This improvement is also tested in section \ref{sec:results}.

Estimates of congestion other than that given by (\ref{eqn:pcnlf}) can
easily by used by PCN.  If the new load equation is
not in the range  then a simple linear transform, ,
will map
it into this range.  If some regions of the range are more of interest
than others then nonlinear transforms could be used.

\subsection{Comparison with other packet marking schemes}
\label{sec:comparison}

It is important to recognise how PCN differs from other packet marking
schemes recently suggested in the literature.
Like XCP, PCN estimates the congestion level at each intervening
queue.  However,
XCP requires a new 128 bit header to record the information whereas
PCN requires
only a single ECN bit. DQM (and its variants) use a similar scheme
involving
both TTL and IPid.
However, they require
a lookup table to be stored in each router in advance, it
is a deterministic rather than a probabilistic scheme and requires two
bits in the IP
header. RAM like PCN is a probabilistic packet marking scheme. It also
uses a single ECN bit to mark packets, but it attempts to calculate
the load on the whole path rather than on each queue on the path
separately. It uses the IP TTL field to estimate the number of routers
in the network.


\subsection{Congestion estimation algorithm}
\label{sec:prediction}

Let  be the load factor in time period
 with
.  Each time period is of length 
so   is the estimate for time .
The source estimates congestion every period .
Consider the source attempting to estimate the load at the th
intermediate router.
Let  be the ratio of ECN marked ACKs to total
ACKs (markable by a given intermediate router ) in the th such
period
.  If this time period is
short then insufficient packets will be received to get a good
estimate.
If it is too long, the estimate will not capture the
dynamic nature of the traffic.

Let  be the mean value of the load factor  in the time
period
.  Note that the load factor may have changed over
this period, particularly if  (the time scale over which load
factor is estimated by the source) is significantly larger than

the time scale over which the load factor is held constant.  This
problem is to some extent unavoidable without time synchronisation
between source and router.
For the real traffic traces investigated in this work the congestion
level remained similar for
much longer time scales than  and .

One possible estimate for  is .
However, it is clear that if  does not change too much between
given
time periods then
 can provide additional
information to help predict . The approach taken here is to
use the well-known
Autoregressive Integrated Moving Average (ARIMA) models
to provide an improved predictor \cite{GBox}.  If
ARIMA models can be fitted to the time series  this can be used
to get a good prediction of 
(since  is an unbiased estimate of  as previously proved).
The selection of an appropriate ARIMA model is discussed
in section \ref{sec:arimatune}.

