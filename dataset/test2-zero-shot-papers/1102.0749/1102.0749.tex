\subsection{Strong Normalisation}\label{sec:SN}

In this section, we prove the strong normalisation property for
\CA. That is, we show that all possible reductions for well-typed
terms are finite.  We use the standard notion of \emph{reducibility
  candidates}~\cite[Chapter 14]{Girard89}, extended to account for
linear combinations of terms. Confluence follows as a
corollary. Notice that we cannot reuse the proofs of previous typed
versions of \llin\ (\eg \cite{ArrighiDiazcaroQPL09,DiazcaroPetit10})
since in \cite{ArrighiDiazcaroQPL09} only terms of the same type can
be added together, and in \cite{DiazcaroPetit10} the calculus under
consideration is a fragment of \CA. Therefore, none of them have the
same set of terms as \CA.

A closed term in \CA\ is a \emph{value} if it is an abstraction, a sum
of values or a scalar multiplied by a value, \ie values are closed
terms that conform to the following grammar:

\begin{grammar}
  [(colon){::}]
  [(semicolon){}]
  [(period){.}]
  [(nonterminal){\bf}{}]
<v> :  ;  ; <v>  <v> ; <v>
\end{grammar}

If a closed term is not a value, it is said to be \emph{neutral}. A term
\ve{t} is \emph{normal} if it has no reducts, \ie there is no term
\ve{s} such that . A \emph{normal form} for a term
\ve{t} is a normal term \ve{t'} such that .
We define  as the set of reducts of 
reachable in one step. 

A term \ve{t} is \emph{strongly normalising} if there are no infinite
reduction sequences starting from \ve{t}. We write  for
the set of strongly normalising closed terms of \CA.

\begin{definition}[Reducibility candidates]
  A set of terms  is a \emph{reducibility candidate} if
  it satisfies the following conditions:

  \begin{description}
  \item[(CR)] \emph{Strong normalisation}: 
  \item[(CR)] \emph{Stability under reduction}: If
     and , then
    .
  \item[(CR)] \emph{Stability under neutral expansion}: If \ve{t}
    is neutral and , then
    .
  \end{description}

\noindent  In the sequel, ,  stand for reducibility
  candidates, and  stands for the set of all reducibility
  candidates.
\end{definition}

The idea of the strong normalisation proof is to interpret types
by reducibility candidates and then show that whenever a term has a
type, it is in a reducibility candidate.

\begin{remark}
  Note that  is a reducibility candidate. In addition, the term \ve{0} is a neutral and normal term, so it is in every reducibility candidate. This ensures that every reducibility candidate is inhabited, and since every typable term can be closed by typing rule , it is enough to consider only closed terms.
\end{remark}

The following lemma ensures that the strong normalisation property is
preserved by linear combination.

\begin{lemma}\label{lemmSNcloLC}
  If  and  are strongly normalising, then
   is strongly normalising.
\end{lemma}

\begin{proof} Induction on a positive algebraic measure defined on
  terms of ~\cite[Proposition 10]{ArrighiDowekRTA08}, showing that every
  algebraic reduction makes this number strictly decrease. \qed
\end{proof}

The following operators 
ensure that all types of \CA\ are interpreted by a reducibility
candidate.

\begin{definition}[Operators in ]\label{def:operatorsRC}
  Let ,  be reducibility candidates. We
  define operators , ,  over  and  such that

  \begin{itemize}
  \item  is the closure of
     under
    (CR),
  \item  is the closure of  under (CR) and (CR),
  \item  is the set 
  \item  is the closure of  under (CR).
  \end{itemize}
\end{definition}

\begin{remark}
Notice that  is neutral and it is in normal form. Therefore the closure of  under (CR) is not empty, it includes, at least, the term .
\end{remark}


\begin{lemma}\label{lemm:operatorsRC}
  Let  and  be reducibility candidates. Then
  , , ,  and 
  are all reducibility candidates.
\end{lemma}

\begin{proof} We show the proof for  and
  . The rest of the cases are similar.

\begin{itemize}
 \item The three conditions hold trivially for .
 \item Let . We must check that the
     three conditions hold.
     \begin{description}
     \item[(CR)] Induction on the construction of
       . If
       , the result is trivial by
       condition (CR) on  and  and
       lemma~\ref{lemmSNcloLC}. If  with
       , then \ve{t} is strongly
       normalising by induction hypothesis; therefore, so is
       \ve{t'}. If \ve{t} is neutral and
       , then \ve{t}
       is strongly normalising since by induction hypothesis all
       elements of  are strongly normalising.
      
      \item[(CR) and (CR)] Trivial by construction of
        . \qed
      \end{description}
\end{itemize}
  
\end{proof}

\newcommand{\interp}[1]{\ensuremath{\llbracket #1 \rrbracket}}

\noindent We can now introduce the interpretation function for the types of
\CA. The definition relies on the operators for reducibility
candidates defined above.

A \emph{valuation}  is a partial function from type variables to
reducibility candidates, written as a sequence of comma-separated
mappings of the form , with 
denoting the empty valuation.

\begin{definition}[Reducibility model] Let  be a type and  a
  valuation. We define the \emph{interpretation}  as follows:

\end{definition}

Note that lemma \ref{lemm:operatorsRC} ensures that every type is
interpreted by a reducibility candidate.

A \emph{substitution}  is a partial function from term
variables to basis terms, written as a sequence of semicolon-separated
mappings of the form , with  denoting the
empty substitution. The action of substitutions on terms is given by


A \emph{type substitution}  is a partial function from type
variables to unit types, written as a sequence of semicolon-separated
mappings of the form , with  denoting the
empty substitution. The action of type substitutions on types is given by

They are extended to act on terms in the natural way.

Let  be a typing context, then we say that a substitution pair
 \emph{satisfies}  for a valuation  (written
) if  implies
.

A typing judgement  is said to be
\emph{valid} (written ) if for every
valuation , for every type substitution  and every
substitution  such that , we have
.
The following lemma proves that every derivable typing judgement is valid.

\begin{lemma}[Adequacy Lemma]\label{adequacyLemma}
  Let , then .
\end{lemma}

\begin{proof}
  We proceed by induction on the derivation of . The base cases (rules \textsc{Ax} and \textsc{Ax})
  are trivial. We show the cases for rules  and \textsc{sI} for
  illustration purposes.  

\begin{itemize}
\item Case : 

  By induction hypothesis, we have . We
  will prove that for all  and
  , . Suppose that . Let
   (note that there is at least one basis
  term, , in ), and let . So , hence
  . This means both \ve{b} and
   are strongly normalising, so we shall first prove
  that all reducts of  are in
  .

    \begin{itemize}
    \item  or , with
       or .  The result
      follows by induction on the reductions of  and
      , respectively: by induction hypothesis we
      have , so both , .

    \item .
    \end{itemize}

    Therefore,  is a neutral
    term with all of its reducts in , so  . Hence, by
    definition of , we conclude .

\item Case \textsc{sI}:
  

  By induction hypothesis, we have .  Let
   be a valuation and  a
  substitution pair satisfying  in . So
  , hence
   by construction. \qed
\end{itemize}
\end{proof}

\noindent Since this proves that every well-typed term is in a reducibility
candidate, we can easily show that such terms are strongly normalising.

\begin{theorem}[Strong Normalisation for \CA]\label{thm:SN}
  All typable terms of \CA\ are strongly normalising.
\end{theorem}

\begin{proof}
  Let \ve{t} be a term of \CA\ of type . If  is an open term, the open variables are in the context, so we can always close it and the term will be closed and typable. Then we can consider  to be closed. Then, by the Adequacy
  Lemma (lemma~\ref{adequacyLemma}), we know that . Furthermore, by lemma~\ref{lemm:operatorsRC},
  we know  is a reducibility candidate, and
  therefore . Hence,
  \ve{t} is strongly normalising.\qed 
\end{proof}

\subsubsection{Confluence}\label{sec:conf}

Now confluence follows as a corollary of the strong normalisation theorem.

\begin{corollary}[Confluence]\label{cor:confluenceADD}
 The typed language \CA\ is confluent: for any term , if  and , then there exists a term  such that  and .
\end{corollary}
\begin{proof}
  The proof of the local confluence of the system, \ie the property
  saying that  and  imply that there
  exists a term  such that  and , is an extension of the one presented for the untyped
  calculus in \cite{ArrighiDiazcaroValironDCM11}, where the set of
  algebraic rules (\ie all rules but the beta reductions) have
  been proved to be locally confluent using the proof assistant
  Coq. Then, a straightforward induction entails the (local) commutation
  between the algebraic rules and the -reductions. Finally,
  the confluence of the -reductions is a trivial extension of
  the proof for -calculus.
Local confluence plus strong normalisation (\cf Theorem~\ref{thm:SN}) implies confluence~\cite{Terese03}. \qed
\end{proof}
