In this section, we show users' privacy attitudes, expectations and awareness. We discuss four themes: 1) Perceptions about personal information on mobile phones and OSN, 2)Trust in service operators and fear of surveillance, 3) Social sharing of private information, and 4) Privacy attitudes that can influence technology use.

\subsection{Personal information on mobile phones \& OSN}
Most ICT projects such as mobile ~finance and ~health ~~~projects require collection of users' personal information. Users' unwillingness to share this information can be an obstacle for such schemes. We now present: 1) Participants disproportionately consider passwords or financial detail as personal information that they will not like to share when compared with health details or physical details and 2) Participants heavily rely on mobile phone to store PII and share it on OSN.

Participants marked their response to the following question, \emph{ which of the following information is personal to you that you would NOT like to share?} Participants chose from a range of options such as annual household income, marital status, bank account details, mobile number, passwords, e-mail address, and physical details -- height, weight, eye color (See Table~\ref{tab:PIIwhat}). We found that disproportionately large number of participants perceived information such as password (88.39\%), and financial details such as credit card number (68.18\%), bank account details (64.63\%) as PII  that they will not like to share with others. Participants considered medical health records (27.17\%), and physical details (8.47\%) could be  shared with others. 

\begin{table}[!htbp]
\caption{\small{Participants' response to information that they consider as PII. All values are in percentage. N shows the number of participants who answered this question.}}
\small
\centering
\setlength{\extrarowheight}{2pt}
\begin{tabular}{p{6cm} r p{1.4cm} r}
\midrule
\textbf{N=10,377}&\textbf{(\%)}\\
\midrule
 \rowcolor {gray!16 }
\raggedright Annual house hold income &53.64\\
\raggedright Bank account details & 64.63\\
 \rowcolor {gray!16 }
\raggedright Credit card number & 68.18\\
\raggedright Health and medical history &27.17\\
 \rowcolor {gray!16 }
\raggedright Passport number& 64.45\\
\raggedright Passwords &88.39\\
 \rowcolor {gray!16 }
\raggedright Personal income &62.77\\
\raggedright Physical details - height, weight, eye colour & 8.47\\
\midrule
\hline
\end{tabular}
\label{tab:PIIwhat}
\end{table} 

Next, we analyzed how comfortable users felt sharing personal information on OSN. We asked participants what personal information they have shared on OSN. Participants said that they shared information such as photos (48.95\%) and videos (45.41\%) on OSN with friends (see Table ~\ref{tab:PIIOSN}). A significant number of participants marked sharing information such as location (45.95\%) and religious preference (40.20\%) with everyone on OSN. Furthermore, most participants reported use of mobile phone to store personal information such as videos and photographs (64.57\%). Few participants also agreed to use mobile phone to store personal information such as passwords (25\%), credit card number(s) / ATM card number(s) / PIN number(s) (26.2\%), date of birth and ID number (30.51\%). These percentages show participants used technology such as mobile phones and OSN to store personal information. However, they did not feel comfortable storing personal details such as passwords on mobile phones. Participants who did not save information on mobile phones said that they were concerned about somebody accessing the phone at work, or outdoors without permission (38.51\%) and somebody accessing the phone at home without permission (24.86\%). 


We used Chi - squared test with holm corrections to examine if statistical difference exists between two genders on PII understanding and sharing of personal information. Our results show statistically significant difference between men and women on both a) Perception of \emph{what was PII} (Chi -sq test with holm corrections, p<0.01) and b) \emph{if they shared PII on OSN or stored PII on phone} (Chi -sq test with holm corrections, p<0.01). Figure~\ref{fig:FM2} shows gender preference for storing PII on mobile phones. Women used mobile phones more than men to store personal information. We found statistically significant difference between men  and women for saving information (shown in Figure~\ref{fig:FM2}) such as credit card number(s) / ATM card number(s) / PIN number(s) on mobile phone ( = 55.723, df = 1, p-value < 0.0001). Similarly, we found statistical difference on gender's preference for storing passwords on mobile phones ( = 40.4386, df = 1, p-value< 0.0001). 

\begin{figure}[!htbp]
\begin{center}
\vspace{-5mm}
\includegraphics*[viewport= 2 2 800 300, scale=0.40]{PIIFM.pdf}
\vspace{-3mm}
\caption{\small{Gender preference for storing PII on mobile phones. Females used mobile phones more than males to store personal information.}}
\label{fig:FM2}
\end{center}
\vspace{-5mm}
\end{figure}
	
		






\subsection{Trust in service operators and ~fear ~of ~~surveillance}
We analyzed participants' trust in the service operators for privacy, they provided. ~We ~found ~that ~participants ~showed: 1) Concern for personal data collected by service providers and governments and 2) High concern for improper access, i.e., disclosure of information by service providers to unauthorized parties~\cite{bellmaninternationaldifferences:2004}. 

\begin{table}[!htbp]
\vspace{-4mm}
\centering
\caption{\small{Participants' response for information that they have shared on OSN. All values are in percentage.}}
\small
\setlength{\extrarowheight}{2pt}
\begin{tabular}{ p{2.5cm} p{1.5cm}  p{1.0cm}  p{1.2cm} }
\midrule
\midrule          
& Not Shared & Friends & Everyone\\
\hline
      \rowcolor {gray!16 }
   Age   & 13.24 & 40.64  & 33.14  \\
Date of Birth & 7.36  & 44.59 & 31.89  \\
        \rowcolor {gray!16 }
    E mail ID & 6.49  & 44.17 & 30.77  \\
Gender & 2.97  & 32.49 & 45.95  \\
\rowcolor {gray!16 }
    Location & 5.77  & 33.13  & 39.83\\
Marital Status & 8.13  & 32.59 & 39.08  \\
\rowcolor {gray!16 }

\raggedright Other Profile Information e.g. education and work details & 5.95  & 40.34  & 30.37  \\


    Pictures / Photos & 7.18  & 48.95 & 21.02  \\
\rowcolor {gray!16 }
    Religion & 8.96  & 32.18 & 40.20  \\
Videos & 11.01 & 45.41 & 21.20 \\
    \midrule
    \hline
    \end{tabular}\vspace{-4mm}
    \label{tab:PIIOSN}
\end{table}


\subsubsection{Trust for data collection}
Participants marked how comfortable they felt sharing PII with websites and government. We found that participants felt more comfortable sharing financial information such as annual household income, bank account details, and credit card numbers with the government than with websites (See Table \ref{tab:PIIshared}). However, they felt more comfortable sharing information such as medical history, marital status, and mobile number with websites than the government.
We also asked participants if they would share personal information with mobile service providers if not mandatory (See Table \ref{table:ans4}). We found that participants felt comfortable sharing contact information such as address, ID proof / passport number, and contact number with mobile service providers, followed by websites. We also analyzed participants' confidence in websites and mobile service providers for maintaining privacy of the data they collected. About 21\% participants strongly agreed with the statement that websites can hinder privacy by collecting personal information. Percentage of participants (63.21\%) trusting mobile service providers was much higher than websites (See Table~\ref{tab:pi2013:govcop}).
\vspace{-4mm}
\begin{table}[!htbp]
\small
\centering
  \caption{\small{Participants shared various PII with websites and government. All values are in percentage.}} \vspace{.2mm}
\begin{tabular}{p{4.0cm} p{1.3cm} p{1.6cm} }
 \hline
\midrule
	\raggedright  & \bf{Websites} &\bf{Government} \\
& (\%) &(\%)\\
\midrule
 \rowcolor {gray!16 }
\raggedright Annual household income &7.83&21.40\\
\raggedright Bank account details &3.02&14.14\\
\rowcolor {gray!16 }
\raggedright Credit card number &3.08&5.05\\
\raggedright Date of birth & 32.96&12.85\\
\rowcolor {gray!16 }
\raggedright Email address &33.38&9.98 \\
\raggedright Family details &20.98&11.57\\
\rowcolor {gray!16 }
\raggedright Full name&42.33&10.09\\
\raggedright Health and medical history &16.64&5.97\\
\rowcolor {gray!16 }
\raggedright Landline number &22.40&8.74\\
\raggedright Marital status&34.26&10.02\\
\rowcolor {gray!16 }
\raggedright Mobile number &24.50&12.54\\
\raggedright Passport number&4.76&11.22\\
\rowcolor {gray!15 }
\raggedright Passwords &2.79&1.01\\
\raggedright Personal income &4.20&6.16\\
\rowcolor {gray!15 }
\raggedright Pictures and videos featuring self &21.67&3.47\\
\raggedright Physical details - height, weight, eye color & 26.15&6.14\\
\rowcolor {gray!15 }
\raggedright Postal mailing address&30.73 &15.46\\
\raggedright Religion &39.87 &8.94\\
\midrule
\hline
\end{tabular}
\vspace{-3mm}
\label{tab:PIIshared}
\end{table}

\begin{table*}[!htbp]
 \centering
 \small
\caption{\small{Participants' response to data collection and handling by businesses on a Likert scale of 5. High trust on mobile service provider for data collection, whereas participants' showed lack of trust in websites. SA = Strongly Agree and A = Agree.  represents mean rating on a likert scale of 5 and  represents the standard deviation. }}
\setlength{\extrarowheight}{2pt}
\begin{tabular}{p{4.8cm} p{1.8cm} p{1.4cm} p{2cm} p{1.5cm} p{1.2cm} p{1cm} p{1cm}}
\midrule
\midrule
\centering
\raggedright & \bf{Strongly agree/Agree} & \bf{Neutral} & \bf{Strongly Disagree/Disagree} & \bf{ }&\bf{} & \bf{Male* (SA+A)} & \bf{Female* (SA+A)}\\
& &  & && & & \\
\midrule
\rowcolor {gray!15 } 
\raggedright Websites hinder privacy by collecting personal information (N=10,415) &71.22 &20.60 &7.25&2.16&0.85 &75.41&70.18\\
\raggedright Mobile service providers give reasonable protection to collected information (N=10,379) & 63.21 &20.86 &15.93 &2.42 & 0.97&58.98 & 65.62\\
\midrule
\midrule
\end{tabular}
\begin{tablenotes}[para,flushleft]
 \item{Note: All values are in percentages except mean and standard deviation. Both male and female participants show higher confidence in mobile service provider than websites. * shows p 0.05 (Mann Whitney Test)}
  \end{tablenotes}
  \vspace{-4mm}
\label{tab:pi2013:govcop}
\end{table*}


 


\begin{table}[!hb]
\vspace{-8mm}
 \centering
 \small
\caption{\small{Information shared with mobile service provider if not mandatory.}}
\setlength{\extrarowheight}{2pt}
\begin{tabular}{p{4cm}p{1.5cm}p{1cm}}
\midrule
\midrule
\bf{N=10,093}& \bf{Female} & \bf{Male}\\
&(\%)&(\%)\\
\midrule
 \rowcolor {gray!15 }
\raggedright Alternative address proof** & 39.8 & 43.99\\
\raggedright Another contact number & 33.82& 33.9\\
\rowcolor {gray!15 }
\raggedright Educational qualification** & 31.11& 21.54\\
\raggedright ID proof** & 60.23& 70.44\\
\rowcolor {gray!15 }
\raggedright Permanent address proof* & 32.24& 29.19\\
\raggedright Photograph(s)**& 54.69& 67.86\\
\rowcolor {gray!15 }
\raggedright Proof of place of work & 17.85& 18.12\\
\raggedright Parents' details**& 10.55& 7.08\\
\rowcolor {gray!15 }
\raggedright All of the above**& 12.09& 6.41\\
\raggedright None of the above*& 4.72& 3.7\\
\rowcolor {gray!15 }
\raggedright Others** & 0.41&0.21\\
\midrule
\midrule
\end{tabular}
\begin{tablenotes}[para,flushleft]
 \item{Note: All values are in percentage. Chi - square test with Holm correction shows statistically significant difference (df=1, p  0.05) between male and female for all pieces of information except proof of place of work. * shows  p - value  0.01(df=1), ** shows p value   0.0001}
  \end{tablenotes}
\vspace{-4mm}
\label{table:ans4}
\end{table}


We checked if different genders showed different concern about data collected and handled by service providers. Chi - square test with Holm correction showed that men and women had a different preference for sharing information with mobile service provider (see Table~\ref{table:ans4}). We found statistically significant difference (df=1, p  0.05) between men and women for sharing all kind of information except proof of place of work and contact number. Although different genders showed different preferences for sharing personal information with service providers, we found that both genders consistently showed higher confidence in mobile service providers than websites (see Table~\ref{tab:pi2013:govcop}) for data collected (Mann Whitney test,  p>0.05). 



\subsubsection{Improper access to personal information}
We asked participants if they agreed that \emph{consumers have lost all control over how personal information about them is circulated and used by the companies}. About 53\% participants agreed and 23.66\% strongly agreed with this statement. Participants felt that mobile service providers could allow improper access of personal information to third parties and government (see Table~\ref{tab:pi2013:govcop2}). When asked if mobile service providers can share their private information with government organizations without informing customers, 67.41\% participants agreed or strongly agreed with the statement. Opinion was no different for the government. About 55\% participants agreed or strongly agreed that government agencies could misuse information e.g. banking, phone records, property records, insurance, and income tax shared with them through UID project. This high percentage indicates that participants were worried about improper access by both government and mobile service providers.

\begin{table*}[!htbp]
\centering
\small
\caption{\small{Unauthorized / improper access of personal information to businesses, mobile service provider and Govt. There is an increased unrest among the citizens about information access than collection.  represents mean rating on a likert scale of 5 and  represents the standard deviation. Rest of the values are in percentage. SA = Strongly Agree and A = Agree.}}
\setlength{\extrarowheight}{2pt}
\begin{tabular}{p{7cm} p{1.5cm} p{1.2cm} p{1.6cm} p{1cm}r p{1cm} p{1cm} p{.05cm}}
\midrule
\midrule
\centering
\small
\raggedright & \bf{Strongly agree/ Agree} & \bf{Neutral} & \bf{Strongly Disagree/ Disagree} & & & \bf{Male (SA+A)} & \bf{Female (SA+A)}\\
\midrule
\rowcolor {gray!15 } 
\raggedright Personal information and biometric data could be accessible to other private corporate through UID with whom you would NOT like to share otherwise (N=7,202) & 56.34&25.78 &17.87 &2.57&1.00&58.19&51.74\\
\raggedright Mobile service providers can share consumer private information with third parties (N=
10,305)  &47.31 &22.04 &30.66&2.81&1.23&47.12&47.40\\
\rowcolor {gray!15 } 
\raggedright Government agencies could have access to details e.g. banking, land records, and income tax records which can be misused by government agencies (through UID) (N=7,193) &55.92 &26.76 &17.95&2.53&0.98&55.52&54.70\\
\raggedright Phone conversations can be tapped by mobile service providers in national interest (N=10,363) &68.17&21.19 &10.62 &2.25&0.96&65.71&73.50\\
\rowcolor {gray!15 } 
\raggedright Mobile service providers can share the customer's information with government organization without informing the customers (N=10,366) &67.41 &19.90 &12.09&2.28&0.99&67.21&68.17\\
\midrule
\hline 
\end{tabular}
\vspace{-4mm}
\label{tab:pi2013:govcop2}
\end{table*}

We did not find statistically significant difference between the genders for concern on improper access to their personal information by governments and mobile service providers. On a likert scale of strongly agree to strongly disagree (5 levels), most participants -- female (48.05\%) and male (47.32\%) agreed that  phone conversations can be tapped by mobile service providers in national interest ( = 2.8585, df = 4, p-value = 0.5818). Similarly we found no statistical difference between men and women on \emph{Government agencies accessing details e.g. banking, land records, and income tax records which can be misused by government agencies through UID} ( = 2.4378, df = 4, p-value = 0.6558)

We found statistically significant different between participants' opinion on 1) Concerns for personal data collected by service providers such mobile service providers and government and 2) Improper access i.e disclosure of information by service providers to unauthorized parties (Wilcoxon Signed, p<0.001). Contrary to high trust in mobile service provider for protecting the private data they collected (see Table~\ref{tab:pi2013:govcop}), participants were concerned that mobile service providers may allow unauthorized access to personal information (see Table~\ref{tab:pi2013:govcop2}). Showing concern, 67.41\% participants agreed and strongly agreed that mobile service providers can share customer's information with government without informing them. However, showing confidence in the protection provided to data collected, 63.21\% participants agreed or strongly agreed that mobile service providers give reasonable protection to data collected. 

\subsection{Privacy and social sharing}
We asked participants \emph{with whom (friends, family, relatives, and society) would you share the following information} [a list]. Most participants were willing to share personal information with family members and were not willing to share any PII with society (see Table~ \ref{tab:pi2013:familyprivacy}).
Few participants (13.38\%) were willing to share information such as passwords (important PII) with family. Table~ \ref{tab:pi2013:familyprivacy} shows the difference between four groups --  friends, family, relatives, and society on sharing PII. Sharing preferences were statistically significant for each piece of PII (Cochran Test, p<0.0001). We further performed Post hoc Mc-nemar test and found statistically significant difference between various groups for sharing PII (except for landline numbers). There was no statistically significant difference for sharing landline numbers with friends or relatives. 

Hofstede found that a collectivistic society such as India shows high cohesion within groups and such societies share information with a larger group beyond family members and close friends~\cite{hofstedecluturalbook:1991}. However, in our survey, we found that participants shared personal information mostly with family members and few participants felt comfortable sharing personal information with friends and relatives (See Table~\ref{tab:pi2013:familyprivacy}). For example, 47.82\%  participants said they would share bank account details with family but 7.53\% and 1.10\% said that they would share bank account details with friends and society respectively.
Further, we found that participants' decision to share PII with family or society was dependent on how important they considered a piece of information. For financial information and passwords, we found a strong correlation (r-0.95) between how important that PII was for a participant, and participant's willingness to share that PII with family. We found statistically significant difference between men and women for sharing information with friends, family, and society (Chi - squared test with Holm correction, df = 1, p<0.001). These percentages and difference in opinion shows that participants may share information with family but felt uncomfortable sharing information with friends and society at large.
 




\begin{table}[!htbp]
\small
\vspace{-4mm}
\caption{\small{PII sharing behavior with family, friends, relatives and society. All values are in percentages. }} \setlength{\extrarowheight}{2pt}
\begin{tabular}{p{2.1cm}p{1cm}p{1 cm}p{1.2cm}p{0.9cm}}
\midrule
\midrule
          & \bf{Friends} & \bf{Family} & \bf{Relatives} & \bf{Society}\\
  \midrule
   \raggedright Annual house hold income & 17.30  & 58.32 & 15.42 & 2.17\\
\rowcolor {gray!15 } 
   \raggedright Bank account details** & 7.53  & 47.82 & 6.83  & 1.10\\
\raggedright Credit card number & 5.02  & 38.65 & 3.75  & 0.91\\
\rowcolor {gray!15 } 
  \raggedright  Date of birth & 31.38 & 40.03 & 27.91 & 7.59\\
\raggedright Email address & 32.36 & 39.69 & 26.74 & 6.44\\
\rowcolor {gray!15 } 
   \raggedright Family details & 35.47 & 50.46 & 39.22 & 6.47\\
\raggedright Full name & 20.74 & 26.45 & 17.93 & 5.92\\
\rowcolor {gray!15 } 
    \raggedright Health and medical history*** & 26.79 & 59.66 & 27.96 & 2.85\\
\raggedright  Landline number* & 36.45 & 44.41 & 36.57 & 5.63\\
\rowcolor {gray!15 } 
  \raggedright  Marital status & 27.06 & 34.05 & 25.14 & 7.84\\
\raggedright  Mobile number & 39.54 & 43.80  & 34.90  & 5.57\\
\rowcolor {gray!15 } 
  \raggedright  Passport number & 9.00 & 34.49 & 6.85  & 1.35\\
\raggedright  Passwords & 2.10   & 13.38 & 1.30   & 0.52\\
\rowcolor {gray!15 } 
   \raggedright Personal income & 12.10  & 41.99 & 9.43  & 1.43\\
\raggedright Pictures and videos  & 44.81 & 56.53 & 34.74 & 3.02\\
\rowcolor {gray!15 } 
   \raggedright Physical details & 33.97 & 50.75 & 29.84 & 3.74\\
\raggedright Postal mailing address & 32.05 & 41.29 & 29.46 & 6.00\\
\rowcolor {gray!15 } 
  \raggedright  Religion & 18.12 & 25.84 & 19.16 & 6.40\\
\midrule
   \midrule
    \end{tabular}\begin{tablenotes}[para,flushleft]
 \item{Note: Statistically significant difference exists between all four groups (Cochran's Q test, p0.0001). Post hoc McNemar's test (after Bonferroni's correction). *, **, ***, show p=0.76, p=0.003, p  0.001 respectively when comparing friends and relatives. For all other combinations McNemar's showed significant difference among groups (McNemar's, p  0.0001).}
  \end{tablenotes}
    \vspace{-4mm}
  \label{tab:pi2013:familyprivacy}\end{table}

\vspace{-1mm}
\subsection{ Privacy attitudes influence technology use}
In this section, 1) we show privacy concerns can influence participants' willingness to use various services through mobile phone and OSN and 2) we find that users understand little about their actions that can result in privacy threats and thus show low preparedness to protect themselves.
\vspace{-1.5mm}
\subsubsection{Willingness to use technology}
We asked participants if they would use phone banking service to check their balance or to transfer money from their account. Among 10,349 \footnote{Some participants chose not to answer this question, therefore, the total number of responses were not 10,427.} participants, only 15.73\% said \emph{phone banking is safe to use to check balance}. However, 21.11\% expressed concern about ~information~being ~leaked through phone tapping and  33.93\% participants refused to use phone banking as they were not sure who was on the other side of the call. Participants expressed higher anxiety for transferring money using phone banking than checking balance. Only 12.77\% (N= 10,291) found phone banking to be safe for transferring money whereas 22.71\% mentioned concerns about fear of information being leaked through phone tapping. Moreover, 37.34\% mentioned concerns about not being sure of who was on the other side (nearly 3\% higher to the percentage of participants who mentioned same concerns for checking balance). Figure~\ref{fig:FPB} illustrates participants' concern and fear for using phone banking facilities. These observations show high anxiety among participants that may result from privacy concerns while using mobile phones.      


We found a significant difference between men and women on using phone banking services to check the balance in their bank account ( = 596.46, df = 4, p-value < 0.001). Higher number of women (41.73\%) than men (30.40\%) said that will not like to use phone banking for checking account balance as they were not sure of who was on the other side of the phone call. Table 9 shows male and female participants' concern for not using phone banking services to check account balance. Similar to perceptions for checking account balance, there was statistically significant difference ( =  589.34, df = 4, p-value < 0.001) between men and women for using phone banking services to transfer money from their account (See Table 9). Higher number of women (46.65\%) were not sure of who was on the other side of the phone call while using phone banking services to transfer money than men (30.40\%). Aforementioned behavior indicates that participants were concerned about giving details to an unknown person, which could lead to a privacy breach. Thus, participants refrained from using technology that did not offer adequate privacy.



\begin{table}[!htbp]
\vspace{-3mm}
{\caption{ \small{Gender preference on the use of phone banking services to check the balance and to transfer money from one account to another. }}}
\small

\begin{tabular}{p{5cm}p{1cm}p{1cm}}
\midrule
\midrule
\bf{N=10,349}& \bf{Female}& \bf{Male}\\
\midrule
\multicolumn{3} {c} {\bf{Phone banking to check account balance}}\\
\midrule
 \rowcolor {gray!15 } 
Yes, it is safe to use & 13.01& 17.06\\
Yes, I don't have a choice & 11.34& 7.13\\
\rowcolor {gray!15 } 
No, I fear information may be leaked through phone tapping & 25.99& 18.6\\
No, not sure of who is on the other side & 41.73& 30.4\\
\rowcolor {gray!15 } 
Others & 7.92& 26.81\\
\midrule				
\multicolumn{3} {c} {\bf{Phone banking services to transfer money}}\\
\midrule
 \rowcolor {gray!15 } 				
Yes, it is safe to use & 9.76& 14.13\\
Yes, I don't have a choice & 9.43& 5.63\\
\rowcolor {gray!15 } 
No, fear information may be leaked through phone tapping & 27& 20.56\\
No, not sure of who is on the other side & 46.65& 33.15\\
\rowcolor {gray!15 } 
Others & 7.16& 26.53\\
\midrule
\hline
\end{tabular}
\label{tab:bankingcheck1}
    \begin{tablenotes}[para,flushleft]
 \item{Note: Statistical difference exists between genders to check balance ( = 596.46, df = 4, p-value < 0.001) and transfer money (=  589.34, df = 4, p-value < 0.001).}
  \end{tablenotes}
\end{table} 
 
\subsubsection{Low Preparedness to protect privacy}
Next, we studied how aware were participants of the actions that may result in a privacy breach. For this, we analyzed participants' preference for accepting friend requests on their most used OSN. This knowledge of participants' preference helped in judging their capabilities to make friends and also to keep themselves protected. Table~\ref{tab:acceptfriendrequest} shows participants' decision to accept friendship request from different people. About 28\% participants said that they would accept friendship request from unknown people who were of the same hometown, and 19.51\% would accept friendship requests from a person of the opposite gender. Moreover, 8.31\% would accept friendship requests from somebody whom they did not recognize but have mutual friends with. These preferences show that a considerable number of participants may accept friend request from random and unknown people. Such an attitude may interfere with information protection on OSN and indicate low preparedness level to handle privacy breach. These numbers show women were more conscious of the actions that may result in a privacy breach than men.


\begin{table}[!htbp]
\caption{ \small{Participants' response for accepting friendship request on their most frequently used OSN. All values are in percentage.}}
\centering
\small
\setlength{\extrarowheight}{2pt}
\begin{tabular}{p{5cm}p{2cm}}
\hline
\midrule
& N=6,929\\
\hline
Colleagues &61.97\\
\rowcolor {gray!15 } 
Family Members&71.21\\
Friends&79.03\\
\rowcolor {gray!15 } 
People from my hometown&27.39\\
Person of opposite gender&19.51\\
\rowcolor {gray!15 } 
Person with nice profile picture&10.12\\
Strangers (people you do not know) &4.99\\
\rowcolor {gray!15 } 
Somebody, whom you do not know or recognize but have mutual / common friends with &8.31\\
Anyone&2.99\\
\rowcolor {gray!15 } 
Others&0.74\\
\midrule
\hline
\end{tabular}
\vspace{-1mm}
\label{tab:acceptfriendrequest}
\end{table}


Marginal difference exists in both the genders for accepting friend requests on OSN. We asked participants \emph{if they receive a friendship request on their most frequently used network, whom would they add as friends.} Participants chose their answer from the following options --  colleagues, family members, friends, people from hometown, opposite gender, nice profile picture, strangers, mutual friends, anyone or others. Figure~\ref{fig:FM3} illustrates male and female participants' preference for accepting friendship request from different individuals. In comparison to women, more male participants said that they felt comfortable while accepting friendship requests from colleagues, people from hometown, opposite gender, mutual friends and others (Chi squared test, p-value  0.001). In comparison to 14.10\% women, 22.20\% men (nearly 8\% more) said that they would accept friend request from a person of the opposite gender. About double the number of men than women said that they would accept friendship requests from people with nice profile picture. However, men were less likely to make friends with family members than women on OSN. 

 \begin{figure}[!h]
 \vspace{-6mm}
\begin{center}
\includegraphics*[viewport= 2 2 470 300 , scale=0.45 ]{banking.pdf}
\vspace{-3mm}
\caption{\small{Participants expressed concerns that information can be leaked through phone tapping; 33.93\% and refused to use phone banking as they were not sure who was on the other side of the call.}}
\label{fig:FPB}
\end{center}
\vspace{-6mm}
\end{figure}



\begin{figure}[!h]
\begin{center}
\includegraphics*[viewport= 2 30 420 260, scale=0.50]{friendrequest.pdf}
\vspace{-4mm}
\caption{\small{Gender difference in accepting friend requests on OSN. Females were more conscious than males for accepting friend requests.}}
\label{fig:FM3}
\end{center}
\vspace{-6mm}
\end{figure}






















