








\documentclass[conference]{IEEEtran}
\IEEEoverridecommandlockouts




\usepackage{cite}


















\ifCLASSINFOpdf
\else
\fi








































\usepackage[cmex10]{amsmath}
\usepackage{mathtools}
\usepackage{amssymb}
\usepackage{array}
\usepackage[pdftex]{graphicx}
\usepackage{multirow}
\usepackage{cases}
\usepackage[tight,footnotesize]{subfigure}
\usepackage{algorithm} \usepackage{algorithmic} \usepackage[cmex10]{amsmath}
\usepackage{amssymb}
\usepackage{booktabs}
\usepackage{url}
\usepackage{graphicx}
\usepackage{subfigure}
\newtheorem{assumption}{Assumption}
\newtheorem{lemma}{Lemma}
\newtheorem{theorem}{Theorem}
\newtheorem{proof}{Proof}
\usepackage{bbm}
\newcolumntype{C}[1]{>{\vspace{0.5em}\begin{minipage}{#1}\centering\let\newline\\
\arraybackslash\hspace{0pt}}m{#1}<{\end{minipage}\vspace{0.5em}}}




\hyphenation{op-tical net-works semi-conduc-tor}


\begin{document}


\title{Incentivizing Users of Data Centers \\Participate in The Demand Response Programs\\
via Time-Varying Monetary Rewards}


\author{\IEEEauthorblockN{Yong Zhan\IEEEauthorrefmark{1}, Du Xu\IEEEauthorrefmark{1}, Hongfang Yu\IEEEauthorrefmark{1}, Shui Yu\IEEEauthorrefmark{4}}

\IEEEauthorblockA{\IEEEauthorrefmark{1}Key Laboratory of Optical Fiber Sensing and Communications, Ministry of Education, \\University of Electronic Science and Technology of China, Chengdu, P. R. China}
\IEEEauthorblockA{\IEEEauthorrefmark{4}School of Information Technology, Deakin University, Victoria, 3125, Australia}}




\maketitle

\begin{abstract}
Demand response is widely employed by today's data centers to reduce energy consumption in response to the increasing of electricity cost. To incentivize users of data centers participate in the demand response programs, i.e., breaking the ``split incentive'' hurdle, some prior researches propose market-based mechanisms such as dynamic pricing and static monetary rewards. However, these mechanisms are either intrusive or unfair. In this paper, we use time-varying rewards to incentivize users, who have flexible deadlines and are willing to trading performance degradation for monetary rewards, grant time-shifting of their requests. With a game-theoretic framework, we model the game  between a single data center and its users.  Further, we extend our design via integrating it with two other emerging practical demand response strategies: server shutdown and local renewable energy generation. With real-world data traces, we show that a DC with our design can effectively shed its peak electricity load and overall electricity cost without reducing its profit, when comparing it with the current practice where no incentive mechanism is established.
\end{abstract}







\IEEEpeerreviewmaketitle


\vspace{0.2cm}

\section{Introduction}\label{sec:introduction}
Several Demand Response (DR) strategies for Data Centers (DCs) emerged recently, such as time-shifting of workloads  \cite{Ghatikar2014}. They can effectively reduce/defer a DC's electricity load at specified time in response to the increasing of electricity cost \cite{Wierman2014}. For example, most of today's DCs need to pay both energy and demand charge, where the demand charge is the cost of the DC's peak electricity load during a billing cycle. In this case, the DCs can shed their demand charges via deferring part of the electricity load  from peak hour to off-peak hour \cite{Mathew2014}.

Inevitably, most DR strategies, especially the ones who need workload management, will lead to performance degradation \cite{Liu2014}. For example, time-shifting of workloads can greatly increase the latency of service \cite{Wang2014}. Naturally, users of DCs are not tend to participate in these DR programs if they are not well incentivized \cite{Zhan2015}. However, today's DCs usually charge their users via flat-rate based pricing or usage based pricing. Namely, the cost of a user is decided based on the total usage of DC resources, regardless of when the DC resources are utilized \cite{Li2010}. In this case, users do not share the same incentive as the DCs to participate in specified DR programs, which is referred to as ``split incentive'' \cite{Zhan2016}.



In this paper, we aim at designing a market-based mechanism for a single DC to break the ``split incentive'' hurdle, which holds the following characteristics:

\begin{itemize}
\item \emph{Monetary incentive}: It provides users of DC with incentives to participate in specified DR programs.

\item \emph{Non-intrusiveness}: The DR programs are fully voluntary and the costs and performances of users, who decide not to participate in the DR programs, are not effected.

\item \emph{Deadline awareness}: All users' requests are processed before their deadlines.

\item \emph{Fair reward system}: Only if one participates in the DR programs can gain monetary reward and the reward is proportional to the user's contribution to the DR programs.

\item \emph{Simplicity and max-cost guarantee}: The pricing policy, which is used to charge users of DC, should be easy to understand. Meanwhile, it can guarantee that a user's cost for the DC services will not exceed the \emph{baseline cost}. 

\end{itemize}
Note, we assume that every user was previously charged via a commonly-applied pricing policy, e.g., usage based pricing \cite{Li2010}, and the corresponding cost is referred to as \emph{baseline cost}.


We use time-varying rewards, on top of the commonly applied pricing policy such as flat-rate based pricing or usage based pricing, to incentivize users of DCs grant time-shifting of their requests for DR. With our design, users, who have tight deadlines and/or are not interested in trading performance degradation for monetary rewards, will be charged by the original pricing policy, and their requests will never be deferred. Namely, it is non-intrusive and deadline aware. Meanwhile, we only reward users, whose requests are deferred, and the reward is proportional to the amounts of requests being deferred. Thus, our reward system is fair. Moreover, our pricing policy and reward system can be easily understand and users' maximum costs with our design will never exceed the \emph{baseline costs}.



With our design, the DC needs to make a suitable trade-off between workload management flexibility and monetary reward. Namely, via giving users higher rewards, the  users are more likely to allow the DC to time-shift their requests. On the contrary, if the rewards are relatively low, the DC has less flexibility to response to the electricity price signals.  
The key contributions of this paper are listed as follows:
\begin{itemize}
\item We propose a non-intrusive and fair reward system to incentivize users  with flexible deadlines grant time-shifting of their requests. In this case, the DC has more workload management flexibility to manage its electricity load and thus reduce its electricity cost. 

\item With a game-theoretic framework, we model the game between the DC and its users. We first study users' dominant strategies for deciding whether to join in the DR programs. Then, DC's optimal choices of the time-varying rewards are deduced. Specifically, the DC aims at minimizing its electricity cost without reducing of its profit and the problem of electricity cost minimization is formulated via a convex program.

\item We extend our design via integrating it with each one of the two emerging practical DR strategies: server shutdown \cite{Zhan2016,Paul2015,Li2015} and local renewable energy generation \cite{Zhan2015,Deng2014,Ghamkhari2013}. Therefore, our design can apply to different combinations of DR strategies.
\end{itemize}


In the rest of the paper, we first outline several prior work about using market-based mechanisms to incentivize users of DCs participate in the DR programs in section \ref{sec:relatedworks}. In section \ref{sec:problemformulation}, user's surplus and DC's profit are formulated. We model the game between users and DC in section \ref{sec:interaction} and extend our design in section \ref{sec:extension}. We evaluate our design with real-world data traces and conclude this paper in section \ref{sec:casestudy} and \ref{sec:conclusion}, respectively.


\section{Related Works}\label{sec:relatedworks}


With specified DR programs, DC can effectively reduce its electricity load \cite{Liu2014}, as well as the demand charge and the overall electricity cost \cite{Mathew2014,Xu2014}. Considering that most DR strategies involve management of users' workloads, e.g., time-shifting of requests, several market-based mechanisms have been proposed to incentivize users grant workload management\cite{Zhan2016,Nasiriani2015,Wang2015,Zhan2015,Wang2014a}. 

\cite{Zhan2015} propagated the overall electricity cost onto the users' costs via charging users by Time-of-Use (ToU) pricing. Namely, at each period of time, the price of an instance of DC, i.e., a combination of DC resources such as CPU, memory and storage, is set to be proportional to the prediction of the DC's electricity load. With this pricing policy, users, who can tolerate performance degradation and are sensitive to price, may reduce their purchases of instances at peak hour in response to the increasing of cost. In this case, the DC's peak electricity load and its demand charge can be reduced. 

\cite{Nasiriani2015} fairly distributed a DC's overall electricity cost among its users. With \cite{Nasiriani2015}'s design, if the DC's electricity cost is mainly decided by its peak electricity load, users of the DC can gain considerable cost reduction via reducing its demand when an overall peak occurs. Thus, users are well incentivized to participate in the DR programs to shed the DC's peak electricity load.

However, both \cite{Zhan2015} and \cite{Nasiriani2015} are intrusive since the costs of all users are effected regardless of whether they are interested in participating in the DR programs. Meanwhile, they do not guarantee that the cost of a user with their designs will never exceed the \emph{baseline cost}. 

\cite{Zhan2016} proposed Usage-based Pricing with Monetary Reward (UPMR). \cite{Zhan2016}'s mechanism is non-intrusive and can provide max-cost guarantee. At the beginning of each billing cycle, DC releases the reward function to its users. Then, each user can make the decision of whether to join the DR programs in the upcoming billing cycle based on the reward function and its sensitivity to the performance degradation. The key difference between \cite{Zhan2016}'s work and our design is that: the monetary reward offered by \cite{Zhan2016} is static, i.e., it will remain unchanged during a billing cycle, and the reward is given to all users, who are willing to participate in the DR programs, regardless of whether their workloads are modified for DR. Thus, \cite{Zhan2016}'s reward system is not fair. On the contrary, In this paper, we provide users with time-varying monetary rewards and users can change their decisions of whether to join the DR programs during a billing cycle based on the real-time reward. In this case, the amount of users, who have joined in the DR programs, is time-varying, and thus the DC embraces more load flexibility. Moreover, the reward given to each user with our design is proportional to the user's contribution to the DR program, i.e., the amount of requests being deferred, which is more fair.

\section{Problem Formulation}\label{sec:problemformulation}
A time-slotted system is used in this paper. Namely, we divide a billing cycle into  time slots, with equally length of  (hour). In this section, we formulate a user's surplus, as well as a DC's profit.

\subsection{User's Surplus}\label{subsec:userssurplus}
Let  denote the net utility can be gained by user  after processing one of its request generated at time slot  without participating in any DR programs, where  denotes the index for users and  denotes the index for time slots. Here,  denotes the number of active users. Let  (it\kappa_{i}[t]\gamma[t]ititS_{i}[t]\mathbbm{1}_{i}[t]\mathbbm{1}_{i}[t]=1i\kappa_{i}[t]\gamma[t]\mathbbm{1}_{i}[t]=0\lambda_{i}[t]it\alpha[t]/KWh) denote the price of energy charge at time slot  and  (\jj \in \{1,\cdots,J\}JA_{j}jP[t]ttE_{pue}[t]tn \in \{1,\cdots,N\}Ne_{0}e_{1}u_{n}[t]ntnt\hat{\lambda}[t]t\nuDtt+D\eta_{d}[t]tt+dd \in \{0,\cdots,D\}\eta_{0}[t]t\sum_{d=1}^{D} \eta_{d}[t]tt\gamma[t]\gamma[t]\kappa_{i}[t]\forall i,t\gamma[t]>\kappa_{i}[t]\forall i,t\gamma[t] \leq \kappa_{i}[t]\gamma[t]\eta_{d}[t]\gamma[t]\eta_{d}[t]Lb[t]Ub[t]\kappa_{i}[t]\kappa_{i}[t]\forall \gamma[t] \in [0,Lb[t]]\forall \gamma[t] \in [Ub[t],\infty)Lb[t]Ub[t]\lambda[t]t\forall t\kappa_{i}[t]Lb[t]Ub[t]\forall t\forall i^{1},i^{2}\in \{1,\cdots,I\}\lambda_{i^{1}}[t]=\lambda_{i^{2}}[t]\forall t\pi[t] \lambda[t]\frac{\gamma[t]-Lb[t]}{Ub[t]-Lb[t]}\pi[t]t\eta_{d}[t]\gamma^{*}[t]\gamma[t]\gamma[t]=\gamma^{*}[t]\forall t85\%m_{on}[t]m_{off}[t]\text{Cost}_{S}P_{s}[t]P_{o}[t]tP_{s}[t]P_{o}[t]m[0]NG[t]t_{R}[x]^{+}\max\{x,0\}\forall t\alpha[t]=\ per KWh,  and 15.59T=1\tau=720e_{0}=0.1e_{1}=0.1\nu=20E_{pue}=1.2\forall t\pi[t]=0.5Lb[t]=10e^{-4}. Next, in this section, we first evaluate our work by comparing its performance with the one without any incentive mechanisms, which represents the current practice of DC and is referred to as \textbf{Baseline}. Then, we study the impact of integrating our design with another DR strategy on the reduction of peak electricity load and overall electricity cost.

\subsection{Performance Evaluation}\label{subsec:performanceevaluation}

\begin{figure}[!t]
\centering
\subfigure[]
{ \label{fig:performance_peak}
\includegraphics[width=0.465\columnwidth]{fig/performance_peak}
}
\subfigure[] 
{ \label{fig:performance_cost}
\includegraphics[width=0.465\columnwidth]{fig/performance_cost}
}
\centering\caption{Normalized results of a DC with our design: (a) Peak electricity load, (b) Electricity cost.}\label{fig:performance}
\end{figure}

We study the DC's peak electricity load and electricity cost with our design under different maximum lengths of deferrable periods of time, i.e., . Fig. \ref{fig:performance} shows the normalized results with respect to the results of Baseline.

Three information can be found from Fig. \ref{fig:performance}: 1) With the increasing of the maximum length of deferrable periods of time, our design can shed the DC's peak electricity load and reduce its overall electricity cost more effectively. 2) When the maximum length of deferrable periods is relatively large, e.g., larger than 10 time slots as shown in Fig. \ref{fig:performance}, the effectiveness of our design cannot be greatly increased via further increasing the maximum length of deferrable periods of time. 3) DC with our design always outperforms the one without any DR  programs and incentive mechanisms. For example, when , the DC with data traces of Youtube U.S. can shed its peak electricity load by  and reduce its overall electricity cost by .

\subsection{Impact of Server Shutdown and Local Renewable Energy Generation}\label{subsec:impact}
We integrate our design with another emerging practical DR strategy: server shutdown or local renewable energy generation, and evaluate their performances. Fig. \ref{fig:impact_turn} and Fig.
\ref{fig:impact_green} show the normalized results of a DC with server shutdown and local renewable energy generation, respectively. Here, the base for both normalizations is the results of Baseline. Specifically, the DC as shown in Fig.
\ref{fig:impact_green} is equipped with several wind turbines and the wind speed data is gathered from January 1, 2012 to January 30, 2012 and is available in \cite{winddata}. By analyzing Fig. \ref{fig:impact_turn} and
\ref{fig:impact_green}, the three information found in section \ref{subsec:performanceevaluation} still hold.  Moreover, By comparing Fig. \ref{fig:impact_turn} and 
\ref{fig:impact_green} against Fig. \ref{fig:performance}, we find that the peak electricity load and the overall electricity cost of a DC with our design can be further reduced via integrating with other DR strategies such as server shutdown and local renewable energy generation.

\begin{figure}[!t]
\centering
\subfigure[]
{ \label{fig:impact_turn_peak}
\includegraphics[width=0.465\columnwidth]{fig/impact_turn_peak}
}
\subfigure[] 
{ \label{fig:impact_turn_cost}
\includegraphics[width=0.465\columnwidth]{fig/impact_turn_cost}
}
\centering\caption{Normalized results of a DC with our design and server shutdown: (a) Peak electricity load, (b) Electricity cost.}\label{fig:impact_turn}
\end{figure}

\begin{figure}[!t]
\centering
\subfigure[]
{ \label{fig:impact_green_peak}
\includegraphics[width=0.465\columnwidth]{fig/impact_green_peak}
}
\subfigure[] 
{ \label{fig:impact_green_cost}
\includegraphics[width=0.465\columnwidth]{fig/impact_green_cost}
}
\centering\caption{Normalized results of a DC with our design and local renewable energy generation: (a) Peak electricity load, (b) Electricity cost.}\label{fig:impact_green}
\end{figure}


\section{Conclusion}\label{sec:conclusion}
In this paper, we proposed time-varying rewards to incentivize users of DCs grant time-shifting of their requests. Our design is non-intrusive, deadline aware and easy to understand. Moreover, our reward system is fair and we can provide users with max-cost guarantee. With a game-theoretic framework, we modeled the game between users and a single DC and formulated/solved the  problem of minimization of the DC's electricity cost without reducing its profit. With real-world data traces, we showed that our design can greatly reduce the DC's peak electricity load and overall electricity cost. We also extended our design to another emerging practical scenario where the DC has employed workload time-shifting and server shutdown or local renewable energy generation simultaneously. Accordingly, our design can apply to a DC with different combinations of DR strategies.


\appendices

\section{Proof of Theorem \ref{the:dominant}}\label{proo:dominant}
Apparently, , from (\ref{equ:usersurplus}), if a user does not grant time-shifting of its requests, its surplus at time time slot , which is referred to as , equals 

On the contrary, if the user decides to participate in the DR programs, its surplus at time slot , which is referred to as , is

where  denotes amount of deferred requests generated by user  at time slot . Note, since DC makes the decision of whether to defer part or all of the users requests if the user grant time-shifting of its requests, the user cannot perfectly predict how many requests will be deferred but can ensure that .

\noindent
Next, we prove that Theorem \ref{the:dominant} is correct in two complementary cases: First, , if , . Namely, in this case, the dominant strategy for the user is to join in the DR programs. Second, , if  , , and thus the optimal choice for the user is not to participate in the DR program in this case. 


\section{Proof of Theorem \ref{the:optimalreward}}\label{proo:optimal}
First, we can easily obtain that , . Second, from (\ref{con:alldeferredrequests}), we have ,

Meanwhile,  satisfy constraint (\ref{con:alldeferredrequests}) of problem (\ref{problem:dc}). Thus,   is a feasible solution of  of problem (\ref{problem:dc}). Next, we prove Theorem \ref{the:optimalreward} by contradiction. Assume that  is not an optimal solution of  of problem (\ref{problem:dc}) and let  denote the true optimal solution of  of problem (\ref{problem:dc}). From (\ref{equ:cost}), , which contradicts the assumption that  is not an optimal solution of  of problem (\ref{problem:dc}).


\begin{thebibliography}{10}

\bibitem{Ghatikar2014}
G.~Ghatikar, ``Demand response opportunities and enabling technologies for data
  centers: Findings from field studies,'' tech. rep., PG\&E/SDG\&E/CEC/LBNL,
  2014.

\bibitem{Wierman2014}
A.~Wierman, Z.~Liu, I.~Liu, and H.~Mohsenian-Rad, ``Opportunities and
  challenges for data center demand response,'' in {\em Proc. of IEEE IGSC},
 Dallas, TX, Nov 2014.

\bibitem{Mathew2014}
V.~Mathew, R.~Sitaraman, and P.~Shenoy, ``Reducing energy costs in
  internet-scale distributed systems using load shifting,'' in {\em Proc. of
  IEEE COMSNETS}, Bangalore, India, Jan 2014.

\bibitem{Liu2014}
Z.~Liu, I.~Liu, S.~Low, and A.~Wierman, ``Pricing data center demand
  response,'' {\em SIGMETRICS Perform. Eval. Rev.}, vol.~42, pp.~111--123, Jun
  2014.

\bibitem{Wang2014}
C.~Wang, B.~Urgaonkar, Q.~Wang, and G.~Kesidis, ``A hierarchical demand
  response framework for data center power cost optimization under real-world
  electricity pricing,'' in {\em Proc. of IEEE MASCOTS},  Paris,
  France, Sep 2014.

\bibitem{Zhan2015}
Y.~Zhan, M.~Ghamkhari, D.~Xu, and H.~Mohsenian-Rad, ``Propagating electricity
  bill onto cloud tenants: Using a novel pricing mechanism,'' in {\em Proc. of
  IEEE Globecom},  San Diego, CA, Dec 2015.

\bibitem{Li2010}
A.~Li, X.~Yang, S.~Kandula, and M.~Zhang, ``Cloudcmp: comparing public cloud
  providers,'' in {\em Proc. of ACM IMC}, Melbourne, Australia, Nov
  2010.

\bibitem{Zhan2016}
Y.~Zhan, M.~Ghamkhari, D.~Xu, S.~Ren, and H.~Mohsenian-Rad, ``Extending demand
  response to tenants in cloud data centers via non-intrusive workload
  flexibility pricing,'' {\em arXiv:1603.05746}, Mar 2016.

\bibitem{Paul2015}
D.~Paul, W.-D. Zhong, and S.~K. Bose, ``Demand response in data centers through
  energy-efficient scheduling and simple incentivization,'' {\em IEEE Systems
  Journal}, vol.~PP, pp.~1 -- 12, Sep 2015.

\bibitem{Li2015}
J.~Li, Z.~Bao, and Z.~Li, ``Modeling demand response capability by internet
  data centers processing batch computing jobs,'' {\em IEEE Trans. on Smart
  Grid}, vol.~6, pp.~737--747, Mar 2015.

\bibitem{Deng2014}
W.~Deng, F.~Liu, H.~Jin, B.~Li, and D.~Li, ``Harnessing renewable energy in
  cloud datacenters: opportunities and challenges,'' {\em IEEE Network},
  vol.~28, pp.~48--55, Jan 2014.

\bibitem{Ghamkhari2013}
M.~Ghamkhari and H.~Mohsenian-Rad, ``Energy and performance management of green
  data centers: A profit maximization approach,'' {\em IEEE Transactions on
  Smart Grid}, vol.~4, pp.~1017--1025, Jun 2013.

\bibitem{Xu2014}
H.~Xu and B.~Li, ``Reducing electricity demand charge for data centers with
  partial execution,'' in {\em Proc. of ACM e-Energy}, Cambridge,
  UK, Jun 2014.

\bibitem{Nasiriani2015}
N.~Nasiriani, C.~Wang, G.~Kesidis, B.~Urgaonkar, L.~Y. Chen, and R.~Birke, ``On
  fair attribution of costs under peak-based pricing to cloud tenants,'' in
  {\em Proc. of IEEE MASCOTS}, Atlanta, GA, Sep 2015.

\bibitem{Wang2015}
C.~Wang, N.~Nasiriani, G.~Kesidis, B.~Urgaonkar, Q.~Wang, L.~Y. Chen, A.~Gupta,
  and R.~Birke, ``Recouping energy costs from cloud tenants,'' in {\em Proc. of
  ACM e-Energy}, Bangalore, India, Jul 2015.

\bibitem{Wang2014a}
C.~Wang, B.~Urgaonkar, G.~Kesidis, U.~V. Shanbhag, and Q.~Wang, ``A case for
  virtualizing the electric utility in cloud data centers,'' in {\em Proc. of
  USENIX HotCloud}, Philadelphia, PA, Jun 2014.

\bibitem{Islam2015}
M.~A. Islam, H.~Mahmud, S.~Ren, and X.~Wang, ``Paying to save: Reducing cost of
  colocation data center via rewards,'' in {\em Proc. of IEEE HPCA},  Burlingame, CA, Feb 2015.

\bibitem{Lin2013}
M.~Lin, A.~Wierman, L.~L. Andrew, and E.~Thereska, ``Dynamic right-sizing for
  power-proportional data centers,'' {\em IEEE/ACM Trans. on Networking},
  vol.~21, pp.~1378--1391, Oct 2013.

\bibitem{Gibbens2000}
R.~Gibbens, R.~Mason, and R.~Steinberg, ``Internet service classes under
  competition,'' {\em IEEE Journal on Selected Areas in Communications},
  vol.~18, pp.~2490--2498, Dec 2000.

\bibitem{CVX}
``Cvx: Matlab software for disciplined convex programming.''
  \url{http://cvxr.com/cvx/}.

\bibitem{Qureshi2010}
A.~Qureshi, {\em Power-demand routing in massive geo-distributed systems}.
\newblock PhD thesis, Massachusetts Institute of Technology, 2010.

\bibitem{Googledata}
``Browse real-time traffic to google products and services.''
  \url{http://www.google.com/transparencyreport/traffic/explorer}.

\bibitem{SCEG}
``{SCEG} electric rates.''
  \url{https://www.sceg.com/for-my-business/manage-my-service/rates}.

\bibitem{winddata}
``Wind energy.'' \url{http://www.windenergy.org}.

\end{thebibliography}




\end{document}
