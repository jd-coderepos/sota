
\documentclass[11pt]{article}
\usepackage{graphicx,amssymb,amsmath}
\usepackage{epsf}
\usepackage{amsfonts}
\usepackage{latexsym}
\usepackage{amssymb}
\usepackage{amsthm}
\usepackage{graphics}
\usepackage{psfrag}
\usepackage{float}
\usepackage{bbm}
\usepackage[small]{caption}

\usepackage{paralist}\usepackage{xspace}\usepackage{xcolor}\usepackage{euscript}
\setlength{\oddsidemargin}{0in}
\setlength{\evensidemargin}{0in}
\setlength{\topmargin}{0in}
\setlength{\headheight}{0in}
\setlength{\headsep}{0in}
\setlength{\textwidth}{6.5in}
\setlength{\textheight}{9in}

\newtheorem{theorem}{Theorem}
\newtheorem{lemma}{Lemma}
\newtheorem{corollary}{Corollary}
\newtheorem{proposition}{Proposition}
\newtheorem{observation}{Observation}
\newtheorem{conjecture}{Conjecture}
\newtheorem{question}{Question}
\newenvironment{Proof}{\vspace{1ex}\noindent{\bf Proof.}\hspace{0.5em}}
        {\hfill\qed\vspace{1ex}}

\newcommand{\old}[1]{{}}
\newcommand{\later}[1]{{}}

\def\etal{{et~al.}}
\def\eg{{e.g.}}
\def\ie{{i.e.}}

\newcommand{\alg}{{\rm ALG}}
\newcommand{\opt}{{\rm OPT}}

\newcommand{\eps}{\varepsilon}
\newcommand{\NN}{\mathbb{N}}
\newcommand{\ZZ}{\mathbb{Z}}
\newcommand{\RR}{\mathbb{R}}

\def\L{\mathcal L}
\newcommand{\conv}{{\rm conv}}
\newcommand{\area}{{\rm Area}}

\newcommand{\etalchar}[1]{}
\newcommand{\PntSet}{S}\newcommand{\PntSetA}{Q}\newcommand{\Body}{C}\newcommand{\Copt}{C_{\mathrm{opt}}}\newcommand{\Ell}{\mathcal{E}}\newcommand{\F}{\mathcal{F}}\newcommand{\VolX}[1]{\mathrm{vol}\pth{#1}}\newcommand{\VC}{\ensuremath{VC}\xspace}
\newcommand{\pnt}{p}\newcommand{\atgen}{\symbol{'100}}\newcommand{\SarielThanks}[1]{\thanks{Department of Computer Science;
      University of Illinois; 201 N. Goodwin Avenue; Urbana, IL,
      61801, USA; {\tt sariel\atgen{}uiuc.edu}; {\tt
         \url{http://www.uiuc.edu/\string~sariel/}.} #1}}
\newcommand{\pth}[2][\!]{#1\left({#2}\right)}
\newcommand{\ceil}[1]{\left\lceil {#1} \right\rceil}
\newcommand{\floor}[1]{\left\lfloor {#1} \right\rfloor}


\newcommand{\lemlab}[1]{\label{lemma:#1}}
\newcommand{\lemref}[1]{Lemma~\ref{lemma:#1}}

\newcommand{\seclab}[1]{\label{sec:#1}}
\newcommand{\secref}[1]{Section~\ref{sec:#1}}

\newcommand{\myqedsymbol}{\rule{2mm}{2mm}}

\newcommand{\Grid}{\mathcal{G}}
\newcommand{\GridX}[1]{\Grid\pth{#1}}
\newcommand{\brc}[1]{\left\{ {#1} \right\}}
\newcommand{\sep}[1]{\,\left|\, {#1} \MakeBig\right.}
\newcommand{\MakeBig}{\rule[-.2cm]{0cm}{0.4cm}}
\newcommand{\CHX}[1]{{\mathcal{CH}}\pth{#1}}
\newcommand{\emphind}[1]{\emph{#1}\index{#1}}
\definecolor{blue25}{rgb}{0,0,0.95}
\newcommand{\emphic}[2]{\textcolor{blue25}{\textbf{\emph{#1}}}\index{#2}}

\newcommand{\emphi}[1]{\emphic{#1}{#1}}\renewcommand{\th}{th\xspace}\newcommand{\bd}{{\partial}}\newcommand{\cardin}[1]{\left|{#1}\right|}
\newcommand{\RSample}{\EuScript{R}}\newcommand{\aftermathA}{\par\vspace{-\baselineskip}}
\newcommand{\thmlab}[1]{{\label{theo:#1}}}
\newcommand{\thmref}[1]{Theorem~\ref{theo:#1}}
\newcommand{\vOptX}[1]{\mathrm{vol}_{\mathrm{opt}}\pth{#1}}

\newcommand{\pslg}{{\sc pslg}}

\newcommand{\figlab}[1]{\label{fig:#1}}
\newcommand{\figref}[1]{Figure~\ref{fig:#1}}


\title{Minimum Convex Partitions and Maximum Empty Polytopes\footnote{A
preliminary version of this paper appeared in the {\em Proceedings of
the 13th Scandinavian Symposium and Workshops on Algorithm Theory,
Helsinki, Finland, July, 2012.}}}

\author{Adrian Dumitrescu\thanks{Department of Computer Science, University of
      Wisconsin--Milwaukee, WI 53201-0784, USA\@.
      Email:~\texttt{dumitres@uwm.edu}.  Supported in part by NSF
      grant DMS-1001667.}\and Sariel Har-Peled\thanks{Department of Computer Science, University of Illinois at
      Urbana--Champaign, Urbana, IL 61801-2302, USA\@.
      Email:~\texttt{sariel@cs.uiuc.edu}. Work on this paper was
      partially supported by a NSF AF award CCF-0915984.}\and Csaba D. T\'oth\thanks{Department of Mathematics,
      California State University, Northridge, Los Angeles, CA; and
      Department of Computer Science, Tufts University, Medford, MA, USA\@.
      Email:~\texttt{cdtoth@acm.org}.  Work on this paper was
      supported in part by NSERC grant RGPIN 35586.}}


\begin{document}
\maketitle

\begin{abstract}
    Let  be a set of  points in . A Steiner convex
    partition is a tiling of  with empty convex bodies.  For
    every integer , we show that  admits a Steiner convex
    partition with at most  tiles. This bound is
    the best possible for points in general position in the plane, and
    it is best possible apart from constant factors in every fixed
    dimension . We also give the first constant-factor
    approximation algorithm for computing a minimum Steiner convex
    partition of a planar point set in general position.

    Establishing a tight lower bound for the maximum volume of a tile
    in a Steiner convex partition of any  points in the unit cube is
    equivalent to a famous problem of Danzer and Rogers.  It is
    conjectured that the volume of the largest tile is .
    Here we give a -approximation
    algorithm for computing the maximum volume of an empty convex body
    amidst  given points in the -dimensional unit box .

    \medskip
    \noindent\textbf{\small Keywords}:
Steiner convex partition,
Horton set,
epsilon-net,
lattice polytope,
approximation algorithm.

\end{abstract}


\section{Introduction} \seclab{sec:intro}

Let  be a set of  points in , .  A
convex body  is \emph{empty} if its interior is disjoint from .
A \emph{convex partition} of  is a partition of the convex hull
 into empty convex bodies (called \emph{tiles}) such that
the vertices of the tiles are in . In a \emph{Steiner convex
   partition} of  the vertices of the tiles are arbitrary: they can
be points in  or \emph{Steiner points}.  For instance, any
triangulation of  is a convex partitions of , where the convex
bodies are simplices, and so  can be always partitioned into
 empty convex tiles~\cite{DRS10}.

In this paper, we study the minimum number of tiles that a Steiner
convex partition of every  points in  admits, and the
maximum volume of a single tile for a given point set.  The research
is motivated by a longstanding open problem by Danzer and
Rogers~\cite{ABFK92, BW71, BC87, FP94, PT12}:
What is the maximum volume of an empty convex body 
that can be found amidst any set  of  points in a
unit cube?  The current best bounds are  and
, respectively (for a fixed ).  The lower bound,
for instance, can be deduced by decomposing the unit cube by
~parallel hyperplanes, each containing at least one point, into at
most  empty convex bodies.
The upper bound is tight apart from constant factors for 
randomly and uniformly distributed points in the unit cube.
It is suspected that the
largest volume is  in any dimension , \ie, the
ratio between this volume and  tends to .

For a convex body  in , denote by  the Lebesgue
measure of , \ie, its area when , or its volume when .

\paragraph{Minimum number of tiles in a convex partition.}
A {\em minimum convex partition} of  is a convex
partition of  with a minimum number of tiles. Denote this number by
.  Further define (by slightly abusing notation)

Similarly define a {\em minimum Steiner convex partition} of  as
one with a minimum number of tiles and let  denote this
number.  We also define


There has been substantial work on estimating , and computing
 for a given set  in the plane.  It has been shown
successively that  by
Neumann-Lara~\etal~\cite{NRU04},  by
Knauer and Spillner~\cite{KS06}, and  for
 by Sakai and Urrutia~\cite{SU09}.  From the other direction,
Garc\'ia-L\'opez and Nicol\'as~\cite{GN06} proved that , for , thereby improving an earlier lower
bound  by Aichholzer and Krasser~\cite{AK01}.  Knauer
and Spillner~\cite{KS06} have also obtained a -factor
approximation algorithm for computing a minimum convex partition for a
given set , no three of which are collinear. There are
also a few exact algorithms, including three fixed-parameter
algorithms~\cite{FMR01,GL05,Sp08}.

The state of affairs is much different in regard to Steiner convex
partitions.  As pointed out in~\cite{DT11}, no corresponding results
are known for the variant with Steiner points. Here we take the first
steps in this direction, and obtain the following results.

\begin{theorem}\thmlab{T1}For , we have .  For , this bound is
    the best possible, that is, ; and for
    every fixed , we have .
\end{theorem}

We say that a set of points in  is in \emph{general position}
if every -dimensional affine subspace contains at most  points
for .  We show that in the plane every Steiner convex
partition for  points in general position,  of which lie in the
interior of the convex hull, has  tiles. This
leads to a simple constant-factor approximation algorithm.

\begin{theorem} \thmlab{T2}
    Given a set  of  points in general position in the plane, a
    ratio  approximation of a minimum Steiner convex partition of
     can be computed in  time.
\end{theorem}

The \emph{average} volume of a tile in a Steiner convex partition of 
points in the unit cube  is an obvious lower bound for the
maximum possible volume of a tile, and for the maximum volume of any
empty convex body . The lower bound 
in \thmref{T1} shows that the average volume of a tile is
 in some instances, where the constant of proportionality
depends only on the dimension.  This implies that a simple
``averaging'' argument is not a viable avenue for finding a solution
to the problem of Danzer and Rogers.

\paragraph{Maximum empty polytope among  points in a
   unit cube.}  In the second part of the paper, we consider the
following problem: Given a set of  points in a rectangular box  in
, find a maximum-volume empty convex body . Since
the ratio between volumes is invariant under affine transformations,
we may assume without loss of generality that .  We
therefore have the problem of computing a maximum volume empty convex
body  for a set of  points in .
It can be argued that the maximum volume empty convex body is a polytope,
however, the number and location of its vertices is unknown and this
represents the main difficulty.
For  there is a polynomial-time exact algorithm (see~\secref{sec:conclusion})
while for  we are not aware of any exact algorithm.
Thus the problem of finding faster approximations naturally suggests itself.

There exist exact algorithms for some related problems.
Eppstein~\etal~\cite{EORW92} find the maximum area empty convex
-gon with vertices among  points in  time, if it exists.
As a byproduct, a maximum area empty convex polygon with vertices among 
given points can be computed exactly in  time with their dynamic
programming algorithm. The running time was subsequently improved to 
by Fischer~\cite{Fis97} and then to   by Bautista-Santiago~\etal~\cite{BDL+11}.

By John's ellipsoid theorem~\cite{Ma02}, the maximum volume empty
ellipsoid in  gives a -appro\-xi\-ma\-tion.
Here we present a -approximation for a maximum volume empty convex
body  by first guessing a good approximation of the bounding
hyperrectangle of  of minimum volume,
and then finding a sufficiently close approximation of  inside it.
We obtain the following two approximation algorithms.
The planar algorithm runs in quadratic time in ,
however, the running time degrades with the dimension.

\begin{theorem}\thmlab{algorithm:2d}
    Given a set  of  points in  and
    parameter , one can compute an empty convex body
    , such that . The running time of the algorithm is
    .
\end{theorem}

\begin{theorem}\thmlab{empty}
    Given a set  of  points in , , and
    a parameter , one can compute an empty convex body
    , such that . The running time of the algorithm is
   
\end{theorem}

As far as the problem of Danzer and Rogers is concerned,
one need not consider convex sets---it suffices to consider
simplices---and for simplices the problems considered are much
simpler. Specifically, every convex body  in ,
, contains a simplex  of volume
~\cite{Las11}.
That is, for fixed , the largest empty simplex amidst  points in
the unit cube  yields a constant-factor approximation
of the largest volume convex body (polytope) amidst the same  points.
Consequently, the asymptotic dependencies on  of the volumes of the
largest empty simplex and convex body are the same.
For  there is a polynomial-time exact algorithm for computing the
largest empty triangle amidst  points in  (see~\secref{sec:conclusion})
while for  we are not aware of any exact algorithm for computing the
largest empty simplex amidst  points in .


\paragraph{Related work.}
Decomposing polygonal domains into convex sub-polygons has been also
studied extensively. We refer to the article by Keil~\cite{K00} for a
survey of results up to the year 2000. For instance, when the polygon
may contain holes, obtaining a minimum convex partition is NP-hard,
regardless of whether Steiner points are allowed.  For polygons
without holes, Chazelle and Dobkin~\cite{CD79} obtained an 
time algorithm for the problem of decomposing a polygon with 
vertices,  of which are reflex, into convex parts, with Steiner
points permitted. Keil~\cite{K00} notes that although there are
an infinite number of possible locations for the Steiner points,
a dynamic programming approach is amenable to obtain an exact
(optimal) solution; see also~\cite{KS02,Sh92}.

Fevens~\etal~\cite{FMR01} designed a polynomial time algorithm for
computing a minimum convex partition for a given set of  points in
the plane if the points are arranged on a constant number of convex
layers.
The problem of minimizing the total Euclidean length of the edges of a
convex partition has been also considered.  Grantson and
Levcopoulos~\cite{GN06}, and Spillner~\cite{Sp08} proved that the
shortest convex partition and Steiner convex partition problems are
fixed parameter tractable, where the parameter is the number of points
of  lying in the interior of .
Dumitrescu~and~T\'oth~\cite{DT11} proved that every set of  points in 
admits a Steiner convex partition which is at most  times longer than the minimum spanning tree, and this bound cannot
be improved. Without Steiner points, the best upper bound for the
ratio of the minimum length of a convex partition and the length of a
minimum spanning tree (MST) is ~\cite{Ki80}.

A largest area convex polygon contained in a given (non-convex) polygon
with  vertices can be found by the  algorithm of Chang and
Yap~\cite{CY86} in  time. The problem is known as the
\emph{potato-peeling problem}.
On the other hand, a largest area triangle contained in a simple polygon
with  vertices, can be found by the  algorithm of Melissaratos and
Souvaine~\cite{ms-sphsg-92} in  time.
Hall-Holt~\etal~\cite{hkkms-flspp-06} compute a constant approximation
in time . The same authors show how to compute a
-approximation of the largest \emph{fat} triangle inside a
simple polygon (if it exists) in time . Given a triangulated
polygon (with possible holes) with  vertices, Aronov~\etal~\cite{AKLS11}
compute the largest area  convex polygon respecting the triangulation edges
in  time.

For finding a maximum volume empty axis-parallel box amidst  points
in , Backer and Keil~\cite{BK10} reported an algorithm with
worst-case running time of . An empty
axis-aligned box whose volume is at least  of the maximum
can be computed in

time by the algorithm of Dumitrescu and Jiang~\cite{DJ12}.

Lawrence and Morris~\cite{LM09} studied the minimum integer 
such that the complement  of any -element set
, not all in a hyperplane, can be \emph{covered} by
 convex sets.  They prove .
It is known that covering the complement of  uniformly distributed points
in  requires  convex sets, which
follows from the upper bound in the problem of Danzer and Rogers.


\section{Combinatorial bounds} \seclab{sec:bounds}

In this section we prove \thmref{T1}. We start with the upper bound.
The following simple algorithm returns a Steiner convex partition with
at most  tiles for any  points in .

\medskip
\noindent Algorithm {\bf A1}:
\begin{enumerate} \itemsep 1pt
    \item[{\sc Step 1.}] Compute the convex hull  of
    .  Let  be the set of hull vertices, and let  denote the remaining points.
    \item[{\sc Step 2.}] Compute , and let  be the
    supporting hyperplane of an arbitrary -dimensional face of
    .  Denote by  the halfspace that contains , and
    .  The hyperplane  contains  points
    of , and it decomposes  into two convex bodies: 
    is empty and  contains all points in . Update  and .
    \item[{\sc Step 3.}] Repeat {\sc Step 2} with the new values of
     and  until  is the empty set.  (If , then any
    supporting hyperplane of  completes the partition.)
\end{enumerate}

\begin{figure}[htb]
    \centerline{\epsfxsize=6.1in \epsffile{f1.eps}}
    \caption{Steiner convex partitions with Steiner points drawn as
       hollow circles. Left: A Steiner convex partition of a set of 13
       points.  Middle: A Steiner partition of a set of 12 points into
       three tiles.  Right: A Steiner partition of the same set of 12
       points into 4 tiles, generated by Algorithm {\bf A1} (the
       labels reflect the order of execution).}
    \figlab{f1}
\end{figure}

It is obvious that the algorithm generates a Steiner convex partition
of . An illustration of Algorithm {\bf A1} on a small planar
example appears in \figref{f1}~(right).  Let  and  denote the
number of hull and interior points of , respectively, so that
. Each hyperplane used by the algorithm removes  interior
points of  (with the possible exception of the last round if  is
not a multiple of ). Hence the number of convex tiles is , and we have , as required.

\paragraph{Lower bound in the plane.}  A matching lower
bound in the plane is given by the following construction.  For , let , where  is a set of 3 non-collinear points in
the plane, and  is a set of  points that form a regular
-gon in the interior of , so that 
is a triangle. If , then  is an empty triangle, and
. If ,  is not in
convex position, and so .  Suppose
now that .

Consider an arbitrary convex partition of . Let  be a point in
the interior of  such that the lines , , do not
contain any edges of the tiles. Refer to \figref{lower-bound}.
For each point , choose a \emph{reference point}  on the ray  in  sufficiently close to point , and lying in the interior
of a tile.  Note that the convex tile containing  cannot contain
any reference points. We claim that any tile contains at most 2
reference points.  This immediately implies .

Suppose, to the contrary, that a tile  contains 3 reference
points , corresponding to the points .
Refer to \figref{lower-bound}.
\begin{figure} [htb]
\centerline{\epsfxsize=2.1in \epsffile{constructions1.eps}}
    \caption{Lower bound construction in .}
    \figlab{lower-bound}
\end{figure}
Note that  cannot be in the interior of , otherwise 
would contain all points  in its interior.
Hence  is a quadrilateral, and
 is also a quadrilateral, since the
reference points are sufficiently close to the corresponding points in
.  We may assume w.l.o.g. that
vertices of  are , , ,  in
counterclockwise order. Then  lies in the interior of
. Hence the tile containing , , and ,
must contain point  in its interior, a contradiction. We conclude that
every tile  contains at most 2 reference points, as required.

\paragraph{Lower bounds for .} A similar
construction works in for any , but the lower bound no longer
matches the upper bound  for .

Recall that a \emph{Horton set}~\cite{H83} is a set  of  points
in the plane such that the convex hull of any 7 points is non-empty.
Valtr~\cite{Va92} generalized Horton sets to . For every
, there exists a minimal integer  with the property
that for every  there is a set  of  points in general
position in  such that the convex hull of any  points
in  is non-empty. It is known that , and Valtr proved that
, and in general that , where
 is the product of the first  primes.

We construct a set  of  points in  as follows.
Let , where  is a set of  points in general
position in , and  is a generalized Horton set of 
points in the interior of , such that the interior of any
 points from  contains some point in .

Consider an arbitrary Steiner convex partition of . Every point
 is in the interior of , and so it lies on the
boundary of at least 2 convex tiles. For each , place two
\emph{reference points} in the interiors of 2 distinct tiles incident
to . Every tile contains at most  reference points. Indeed,
if a tile contains  reference points, then it is incident to
 points in , and some point of  lies in the interior of
the convex hull of these points, a contradiction.

There are  reference points, and every tile contains at most
 of them. So the number of tiles is at least . Consequently, for every fixed , we
have .


\section{Approximating the minimum Steiner convex partition in }
\seclab{sec:steiner}

In this section we prove~\thmref{T2} by showing that our simple-minded
algorithm {\bf A1} from \secref{sec:bounds} achieves a constant-factor
approximation in the plane if the points in  are in general
position.

\paragraph{Approximation ratio.}  Recall that algorithm
{\bf A1} computes a Steiner convex partition of  into at most
 parts, where  stands for the number of
interior points of .

If , the algorithm computes an optimal partition, \ie,
. Assume now that .  Consider an optimal
Steiner convex partition  of  with  tiles.  We construct a
planar multigraph  as follows. The \emph{faces} of  are
the convex tiles and the exterior of  (the outer face).  The
\emph{vertices}  are the points in the plane incident to at least 3
faces (counting the outer face as well). Since ,  is
non-empty and we have . Each \emph{edge} in  is a
Jordan arc on the common boundary of two faces. An edge between two
bounded faces is a straight line segment, and so it contains at most
two interior points of . An edge between the outer face and a
bounded face is a convex arc, containing hull points from . Double
edges are possible if two vertices of the outer face are connected by
a straight line edge and a curve edge along the boundary---in this
case these two parallel edges bound a convex face. No loops are
possible in . Since  is a convex partition,  is connected.

Let , , and , respectively, denote the number of vertices,
edges, and bounded (convex) faces of ; in particular, .  By
Euler's formula for planar multigraphs, we have , that is,
.  By construction, each vertex of  is incident to at
least 3 edges, and every edge is incident to two vertices. Therefore,
, or .  Consequently, .  Since  is in general position, each straight-line
edge of  contains at most 2 interior points from . Curve edges
along the boundary do not contain interior points. Hence each edge in
 is incident to at most two interior points in , thus .  Substituting this into the previous inequality on  yields
.  Comparing this lower bound with the
upper bound , we conclude that

and the approximation ratio of 3 follows.

\paragraph{Tightness of the approximation ratio.}  We
first show that the ratio  established above is tight for Algorithm {\bf A1}.
We construct a planar point set  as follows. Refer to  \figref{fig:tightness}~(left).
Consider a large (say, hexagonal) section of a hexagonal lattice.  Place Steiner
vertices at the lattice points, and place two points in  on each
lattice edge. Slightly perturb the lattice, and add a few more points
in  near the boundary, and a few more Steiner points, so as to
obtain a Steiner convex partition of  with no three points
collinear. Denote by , , and , the elements of the planar
multigraph  as before. Since we consider a large lattice section,
we have . We write , whenever .
As before, we have , and since each non-boundary edge is
shared by two convex faces, we have . By construction,
, hence . Therefore the convex
partition constructed above has , while Algorithm {\bf A1}
constructs one with about  faces. Letting , then , and the ratio  approaches  in the limit:
.
\medskip

\begin{figure} [htb]
\centerline{\epsfxsize=5in \epsffile{constructions2.eps}}
    \caption{Left: two points on each edge of a section of a perturbed
      hexagonal lattice in , and four extra vertices of a bounding box.
     Right: Points in general position on a saddle surface in .}
    \figlab{fig:tightness}
\end{figure}

\paragraph{Time analysis.} Algorithm {\bf A1} can be implemented to
run in  time for a set  of  points in the plane.
We employ the semi-dynamic (delete only) convex
hull data structure of Hershberger and Suri~\cite{HS92}. This data
structure supports point deletion in  time, and uses 
space and  preprocessing time. We maintain the boundary of
a convex polygon  in a binary search tree, a set  of
points lying in the interior of , and the convex hull 
with the above semi-dynamic data structure~\cite{HS92}.  Initially,
, which can be computed in  time; and
 is the set of interior points.  In each round of the
algorithm, consider the supporting line  of an arbitrary edge 
of  such that  lies in the halfplane .  The two
intersection points of  with the boundary of  can be computed in
 time. At the end of the round, we can update  and  in  time, where  is the
number of points removed from ; and we can update  in  time. Every point is removed from  exactly
once, and the number of rounds is at most , so
the total update time is  throughout the algorithm.

\paragraph{Remark.}
Interestingly enough, in dimensions 3 and higher, Algorithm {\bf A1}
does not give a constant-factor approximation.  For every integer ,
one can construct a set  of  points in general position in
 such that  of them lie in the interior of ,
but the minimum Steiner convex partition has only  tiles.
In contrast, Algorithm {\bf A1} computes a Steiner partition with
 convex tiles.

We first construct the convex tiles, and then describe the point set
. Specifically,  consists of 4 points of a large tetrahedron,
and 3 points in general position on the common boundary of certain
pairs of adjacent tiles.

Let . Place  Steiner points
 on the saddle surface  for pairs of
integers , .  The four points  form a parallelogram for every ,
.  Refer to
\figref{fig:tightness}~(right).  These parallelograms form a terrain over
the region .  Note that no two parallelograms are coplanar.
Subdivide the space \emph{below} this terrain by vertical planes
, .  Similarly,
subdivide the space \emph{above} this terrain by planes ,
. We obtain 
interior-disjoint convex regions,  above and  below the terrain,
such that the common boundary of a region above and a region below is
a parallelogram of the terrain.  The points in  that do not lie
above or below the terrain can be covered by 4 convex wedges.

Enclose the terrain in a sufficiently large tetrahedron . Clip the
 convex regions and the 4 wedges into the interior of . These
 convex bodies tile .  Choose 3 noncollinear points of  in
each of the  parallelograms, such that no 4 points are coplanar
and no 2 are collinear with vertices of . Let the point set  be
the set of 4 vertices of the large tetrahedron  and the 
points selected from the parallelograms.



\section{Approximating the maximum empty convex body}
\seclab{sec:body}

Let  be a set of points in the unit cube . Our task is to approximate the largest convex body  that contains no points of  in its
interior. Let  denote this body.

\subsection{Approximation by the discrete hull}

In the following, assume that  is some integer, and consider the
grid point set

Let  be a point set, and let  be
the corresponding largest empty convex body in .
Given a grid , we call 
the \emph{discrete hull} of ~\cite{hp-osafd-98a}.
We need the following easy lemma.

\begin{lemma}\lemlab{discrete:hull}\label{discrete:hull}
    Let  be a convex body and .  Then we have , where
    the constant of proportionality depends only on .
\end{lemma}
\begin{proof}
    Consider a point , and the cube 
    centered at  with side length . If this cube is contained
    in , then all grid points of the grid cell of  containing
     are in , and  lies in . Therefore, for every
    point , the cube  is not contained
    in . By convexity, at least one of the vertices of the cube
     lies outside of . Therefore, the distance
    from  to the boundary of  is at most the distance from  to
    a corner of this cube, which is .

    It follows that all the points in the corridor  are
    at distance at most  from the boundary of . The volume
    of the boundary of  is bounded from above by the volume of the boundary of
    the unit cube, namely . As such, the volume of this corridor
    is .
    For a fixed , this is , as claimed.
\end{proof}

\lemref{discrete:hull} implies that if ,
in order to obtain a )-approximation,
we can concentrate our search on convex polytopes that have their
vertices at grid points in , where .
If  is a constant, then the maximum volume empty lattice
polytope in  with  is an -approximation
for . However, for arbitrary ,
a much finer grid would be necessary to achieve this approximation.


\subsection{An initial brute force approach}

In this section we present approximation algorithms (for all ) relying on
\lemref{discrete:hull} alone, approximating the maximum volume empty
polytope by a lattice polytope in a sufficiently fine lattice (grid).
We shall refine our technique in Subsections~\ref{ssec:refine}
and~\ref{ssec:refine:d}.

For the plane, we take advantage of the existence of an efficient
solution for a related search problem. Refining a natural dynamic programming
approach by Eppstein~\etal~\cite{EORW92} and Fischer~\cite{Fis97},
Bautista-Santiago~\etal~\cite{BDL+11} obtained the following result.

\begin{lemma}[\cite{BDL+11}]\lemlab{2:d:compute:empty}\label{2:d:compute:empty}
    Given a set  of  points and a set  of 
    points in the plane, one can compute a convex polygon with the
    largest area with vertices in  that does not contain any
    point of  in its interior in  time.
\end{lemma}

\noindent{\bf Remark.}
The algorithm has the same running time if  is a set of
 forbidden rectangles.

\smallskip
The combination of Lemmas~\ref{discrete:hull} and \ref{2:d:compute:empty}
readily yields an approximation algorithm for the plane, whose running
time depends on .

\begin{lemma}\lemlab{brute:force} \label{brute:force}
    Given a set  of  points, such that
    , and a parameter , one can
    compute an empty convex body  such that
    .  The running time of the
    algorithm is .
\end{lemma}
\begin{proof}
    Consider the grid  with .
    By \lemref{discrete:hull} we can restrict our search to a grid polygon.
    Going a step further, we mark all the grid cells containing points of
     as forbidden. Arguing as in \lemref{discrete:hull},
    one can show that the area of the largest convex grid polygon avoiding
    the forbidden cells is at least , where  is
    a constant.

    We now restrict our attention to the task of finding a largest
    polygon. We have a set  of  grid points that
    might be used as vertices of the grid polygon, and a set of
     grid cells that cannot intersect the interior of the
    computed polygon. By \lemref{2:d:compute:empty}, a largest
    empty polygon can be found in  time. Setting
    , we get an algorithm with overall
    running time .
\end{proof}

For dimensions , we are not aware of any analogue of the
dynamic programming algorithm in \lemref{2:d:compute:empty}.
Instead, we use a brute force approach that enumerates all feasible
subsets of a sufficiently fine grid.

\begin{lemma}\lemlab{brute:force:h:dim} \label{brute:force:h:dim}
    Given a set  of  points such that
    , and a parameter , one can
    compute an empty convex body , such that
    . The running time of the
    algorithm is ,
    where  and  is fixed.
\end{lemma}
\begin{proof}
    Consider the grid  with .
    Let  be the set of vertices of all grid cells of 
    that contain some point from  (\ie,  vertices per cell).
    Note that .
    Andrews~\cite{And63} proved that a convex lattice polytope of volume
     has  vertices. Hence a convex lattice
    polytope in  has  vertices.
     By the well-known inequality ,
    the number of subsets of size  from  is
    
    For each such candidate subset  of size ,
    test whether  is empty of points from .
    For each point in , the containment test reduces to a linear
    program that can be solved in time polynomial in .
    Returning the subset with the largest hull volume found yields
    the desired approximation. The runtime of the algorithm is
    
\end{proof}

B\'arany and Vershik~\cite{BV92} proved that there are
 convex lattice polytopes in
. If the polytopes can also be enumerated in this time
(as in the planar case~\cite{BP92}), then the runtime
in \lemref{brute:force:h:dim} reduces accordingly.


\subsection{A faster approximation in the plane}
\label{ssec:refine}

If  is long and skinny (e.g.,  is close to ), then
the uniform grid  we used in Lemmas~\ref{brute:force} and
\ref{brute:force:h:dim} is unsuitable for finding
a -approximation efficiently. Instead, we employ a rotated and
stretched grid (an affine copy of ) that has similar
orientation and aspect ratio as . This overcomes one of the
main difficulties in obtaining a good approximation. Since we do not
know the shape and orientation of , we guess these
parameters via the minimum area rectangle containing .

\begin{lemma}\lemlab{algorithm:2d}Given a set  of  points such that
    , and a parameter ,
    one can compute an empty convex body  such that
    .
   The running time of the algorithm is
.
\end{lemma}
\begin{proof}
   The idea is to first guess a rectangle  that contains
    such that  is at least a constant
   fraction of the area of , and then to apply
   \lemref{brute:force} to the rectangle  (as the unit square) to
   get the desired approximation.

   Let  be the minimum area rectangle (of arbitrary orientation)
   that contains ; see \figref{boundingbox}~(left). We guess an
   approximate copy of . In particular, we guess the lengths of
   the two sides of  (up to a factor of ) and the orientation of
    (up to an angle of ), and then try to position a
   scaled copy of the guessed rectangle so that that it fully contains .

\begin{figure} [htb]
\centerline{\epsfxsize=4.9in \epsffile{b123.eps}}
    \caption{Left: , a minimum area rectangle ,
      ,
    and a minimum area rectangle , , with
    canonical side lengths and the same orientation as .
    Right: Rectangle , a rotated copy  with the closest canonical
    orientation, and a minimum area scaled copy  such that
    .}
    \figlab{boundingbox}
\end{figure}

    Assume for convenience that .
    We now show that , using \thmref{T1}.
    Augment the point set  with the four corners of the unit square 
    into a set of  points. By \thmref{T1}, the augmented point set has a Steiner
    convex partition into at most  tiles.
    The area of the largest tile is at least that of the average tile
    in this partition, that is,
    , for .
    Therefore, we may assume that .

    Denote by  and  the lengths of the two sides of , where .
    It is clear that , the diameter of the unit square.
    We also have , hence the aspect ratio of  is .

    Assume now that 
    for some .
    If we want to guess the aspect ratio of  up to a factor of two,
    we need to consider only  possibilities.
    Indeed, we consider the \emph{canonical aspect ratios}  for
    ,
    and \emph{canonical side lengths}  and
    .
    Let  be a minimum area rectangle with canonical side lengths
    and the same orientation as , so that .

    The \emph{orientation} of a rectangle is given by the angle between
    one side and the -axis. We approximate the orientation of 
    by \emph{canonical orientations} , for
    . Let  be a congruent copy of 
    rotated clockwise to the nearest canonical orientation about the
    center of . We show that , \ie, a scaled
    copy of  contains . Let  be the minimum scale
    factor such that . Refer to \figref{boundingbox}~(right).
    Denote by  the common center of  and , let  be a vertex
    of  on the boundary of , and let  be the
    corresponding vertex of . Clearly,  since we rotate by at most .
    The aspect ratio of the rectangle  is .
    Since , we have
    .
    The law of sines yields ;
    and we have  by the triangle inequality.
    If follows that , and so  suffices.
    Summing over all possible areas, canonical aspect ratios,
    and orientations, the number of possibilities is
    

    So far we have guessed the canonical side lengths and orientation
    of , however, we do not know its location in the plane. If a
    translated copy  of  intersects , then
     contains it, since .
    Consider an arbitrary tiling of the plane with translates of .
    By a packing argument, only  translates
    intersect the unit square . One of these translates,
    say , intersects , and hence the rectangle  contains .

    We can apply \lemref{brute:force} to the rectangle  (as the
    unit square) to get the desired approximation. Specifically, let
     be an affine transformation
    that maps  into the unit square , and apply \lemref{brute:force}
    for the point set  and . The grid 
    clipped in  corresponds to a stretched and rotated grid
    in ; each grid cell of  is stretched to a rectangle with the
    same aspect ratio as . The convex polygon  occupies a constant
    fraction of the area of , and so the resulting running time is
    , where  is the number of points in .
    Note that the algorithm of \lemref{brute:force} partitions 
    into a grid with  cells. The approximation algorithm only cares
    about which cells are empty and which are not.

    Since the algorithm of \lemref{brute:force} is repeated for all
    possible positions of ,
    the overall running time is
    ,
    where the first factor of  counts possible areas, canonical aspect ratios,
    and orientations, and the second factor of  inside the
    parenthesis counts possible positions of the rectangle .
\end{proof}

\noindent{\bf Remark.}
If  the running time of this planar algorithm is linear in .

\medskip
Since , the running time of the algorithm in \lemref{algorithm:2d}
is bounded by .
We summarize our result for the plane in the following.

\medskip
\noindent
{\bf \thmref{algorithm:2d}.} {\em
    Given a set  of  points in  and a
    parameter , one can compute an empty convex body
    , such that . The running time of the algorithm is
    .
}

\subsection{A faster approximation in higher dimensions}
\label{ssec:refine:d}

Given a set  of  points
and a parameter , we compute an empty
convex body  such that
.
Similarly to the algorithm in Subsection~\ref{ssec:refine},
we guess a hyperrectangle  that contains 
such that  is at least a constant fraction
of ; and then apply \lemref{brute:force:h:dim} to 
(as the hypercube) to obtain the desired approximation.

Consider a hyperrectangle  of minimum volume (and arbitrary
orientation) that contains . The  edges incident to a vertex of a
hyperrectangle  are pairwise orthogonal. We call these  directions the
\emph{axes} of ; and the \emph{orientation} of  is the set of its axes.

We next enumerate all possible discretized hyperrectangles of volume ,
guessing the lengths of  their axes, their orientations, and their locations
as follows:

Guess the length of every axis up to a factor of 2. Since the minimum length
of an axis in our case is  and the maximum is ,
the number of possible lengths to be considered is .
Let  be a hyperrectangle of minimum volume with canonical side lengths
and the same orientation as  such that .

We can discretize the orientation of a hyperrectangle as follows.
We spread a dense set of points on the sphere of directions,
with angular distance  between any point on the
sphere and its closest point in the chosen set. 
points suffice for this purpose. We try each point as the
direction of the first axis of the hyperrectangle, and then
generate the directions of the remaining axes analogously
in the orthogonal hyperplane for the chosen direction.
Overall, this generates 
possibilities.

Successively replace each axis of  by an approximate axis
that makes an angle at most  with its
corresponding axis, where  is a constant depending on .
Let  be a congruent copy of  obtained in this way.
If  is sufficiently small, then .

Consider a tiling of  with translates of . Note that
only  translates intersect the unit cube
. One of these translates  intersects ,
and then the hyperrectangle  contains .
Since  takes a constant fraction of the volume of ,
we can deploy \lemref{brute:force:h:dim} in this case, and get the desired
-approximation in
 time.
Putting everything together, we obtain the following.

\medskip \noindent {\bf \thmref{empty}} {\em
    Given a set  of  points in , , and
    a parameter , one can compute an empty convex body
    , such that . The running time of the algorithm is
   
}

\paragraph{Remark.}  Consider a set  of  points in
.  The approximation algorithm we have presented can be
modified to approximate the largest empty tile, \ie, the largest empty
convex body contained in , rather than .  The
running time is slightly worse, since we need to take the boundary of
 into account. We omit the details.


\section{Conclusions} \seclab{sec:conclusion}

In this section we briefly outline two exact algorithms for finding
the largest area empty convex polygon and the largest area empty
triangle amidst  points in the unit square. At the end we list a
few open problems.


\paragraph{Largest area convex polygon.}
Let , where . Let  be the set of
four vertices of . Observe that the boundary of an optimal convex
body, , contains at least two points from .
By convexity, the midpoint of one of these  segments
lies in . For each such midpoint ,
create a weakly simple polygon  by connecting each
point  to the boundary of the square along the
ray . The polygon  has  vertices and is empty of points
from  in its interior. Then apply the algorithm of Chang and
Yap~\cite{CY86} for the potato-peeling problem (mentioned
in~\secref{sec:intro}) in these  weakly simple
polygons. The algorithm computes a largest area empty convex
polygon contained in a given (non-convex) polygon with  vertices in
 time. Finally, return the largest convex polygon obtained
in this way. The overall running time is .

The running time can be reduced to  as follows.
Instead of considering the  midpoints, compute a set  of
 points so that every convex set of area at least
 contains at least one of these points. In particular, 
contains a point from . The set  can be computed by starting with
a  grid, and then computing an -net for it,
where , using discrepancy~\cite{Ma02}. The running
time of this deterministic procedure is roughly , and the
running time of the overall algorithm improves to
.


\paragraph{Largest area empty triangle.}
The same reduction can be used for finding largest area empty triangle
contained in , resulting in  weakly simple polygons .
Then the algorithm of Melissaratos and Souvaine~\cite{ms-sphsg-92} for
finding a largest area triangle contained in a polygon is applied to
each of these  polygons.
The algorithm finds such a triangle in  time, given a
polygon with  vertices. Finally, return the largest triangle
obtained in this way. The overall running time is . Via the
-net approach (from the previous paragraph) the running time of
the algorithm improves to .


\paragraph{Open questions.}
Interesting questions remain open regarding the structure of optimal
Steiner convex partitions and the computational complexity of
computing such partitions. Other questions relate to the problem of
finding the largest empty convex body in the presence of points.

\begin{itemize}

    \item [(1)] Is there a polynomial-time algorithm for computing a
    minimum Steiner convex partition of a given set of  points in
    ? Is there one for points in the plane?

    \item [(2)] Is there a constant-factor approximation algorithm for
    the minimum Steiner convex partition of an arbitrary point set in
     (without the general position restriction)? Is there one
    for points in the plane?

    \item [(3)] For , the running time of our approximation
    algorithm for the maximum empty polytope has a factor of the form
    .  It seems natural to conjecture that this term can
    be reduced to . Another issue of interest is extending
    Lemma~\ref{2:d:compute:empty} to higher dimensions for a faster
    overall algorithm.

    \item [(4)] Given  points in , the problem of finding
    the largest convex body in  that contains up to 
    (outlier) points naturally suggests itself and appears to be also
    quite challenging.

\end{itemize}

\paragraph{Acknowledgement.}
The authors thank Joe Mitchell for helpful discussions regarding the
exact algorithms in~\secref{sec:conclusion}, in particular for
suggesting the reduction of the maximum-area-empty-convex-body problem
to the potato-peeling problem. Many thanks also go to Sergio Cabello
and Maria Saumell for pointing us to the recent results of
Bautista-Santiago~\etal~\cite{BDL+11} and for suggesting logarithmic factor
improvements in the running time of the approximation algorithm in
Section~\ref{ssec:refine}.


\begin{thebibliography}{99}

\bibitem {AK01}
O. Aichholzer and H. Krasser,
The point set order type data base: A collection of applications and results,
in \emph{Proc. 13th Canadian Conf. on Comput. Geom.},
Waterloo, ON, Canada, 2001, pp.~17--20.

\bibitem{ABFK92}
N. Alon, I. B\'ar\'any, Z. F\"uredi, and D. Kleitman,
Point selections and weak -nets for convex hulls,
\emph{Combinatorics, Probability \& Computing} \textbf{1} (1992), 189--200.

\bibitem{And63}
G.~E.~Andrews,
A lower bound for the volumes of strictly convex bodies with many boundary points,
\emph{Transactions of the AMS} \textbf{106} (1963), 270--279.

\bibitem{AKLS11}
B.~Aronov, M.~van~Kreveld, M.~L\"offler, and R.~I.~Silveira,
Peeling meshed potatoes, \emph{Algorithmica} \textbf{60(2)} (2011), 349--367.

\bibitem{BK10}
J. Backer and M. Keil,
The mono- and bichromatic empty rectangle and square problems in all dimensions,
in \emph{Proc. 9th Latin American Sympos. on Theoretical Informatics},
vol. 6034 of LNCS, Springer, 2010, pp.~14--25.

\bibitem{BW71}
R. P. Bambah and A. C. Woods,
On a problem of Danzer,
\emph{Pacific J. Math.} \textbf{37(2)} (1971), 295--301.

\bibitem{BP92}
I. B\'ar\'any and J. Pach,
On the number of convex lattice polygons,
\emph{Combinatorics, Probability \& Computing} \textbf{1} (1992), 295--302.

\bibitem{BV92}
I.~B\'ar\'any and A.~M.~Vershik,
On the number of convex lattice polytopes,
\emph{Geometric and Functional Analysis} \textbf{2} (1992), 381--393.

\bibitem{BDL+11}
C.~Bautista-Santiago, J.~M.~D\'{\i}az-B\'a\~{n}ez, D.~Lara,
P.~P\'erez-Lantero, J.~Urrutia, and I.~Ventura,
Computing optimal islands,
\emph{Oper. Res. Lett.} {\bf 39} (4) (2011), 246--251.

\bibitem{BC87}
J. Beck and W. Chen,
\emph{Irregularities of Distributions},
vol.~89 of Cambridge Tracts in Mathematics,
Cambridge University Press, Cambridge, 1987.

\bibitem{CY86}
J.-S. Chang and C.-K. Yap,
A polynomial solution for the potato-peeling problem,
\emph{Discrete Comput. Geom.} \textbf{1} (1986), 155--182.

\bibitem{CD79}
B.~Chazelle and D.~P.~Dobkin,
Decomposing a polygon into its convex parts,
in \emph{Proc. 11th Sympos. Theory of Computing}, ACM Press, 1979, pp.~38--48.

\bibitem{DRS10}
J.~A.~De~Loera, J.~Rambau, and F.~Santos,
Triangulations: Structures for Algorithms and Applications,
vol. 25 of \emph{Algorithms and Computation in Mathematics},
Springer, 2010.

\bibitem{DJ12}
A. Dumitrescu and M. Jiang,
On the largest empty axis-parallel box amidst  points,
\emph{Algorithmica} \textbf{66(2)} (2013), 225--248.

\bibitem{DT11}
A. Dumitrescu and Cs. D. T\'oth,
Minimum weight convex Steiner partitions,
\emph{Algorithmica} \textbf{60(3)} (2011), 627--652.

\bibitem{EORW92}
D.~Eppstein, M.~Overmars, G.~Rote, and G.~Woeginger,
Finding minimum area -gons,
\emph{Discrete Comput. Geom.} \textbf{7(1)} (1992), 45--58.

\bibitem{FMR01}
T.~Fevens, H.~Meijer, and D.~Rappaport,
Minimum convex partition of a constrained point set,
\emph{Discrete Appl. Math.} \textbf{109(1-2)} (2001), 95--107.

\bibitem{Fis97}
P.~Fischer,
Sequential and parallel algorithms for finding a maximum convex polygon,
\emph{Comput. Geom. Theory Appl.} {\bf 7} (1997), 187--200.

\bibitem {FP94}
Z. F\"uredi and J. Pach,
Traces of finite sets: extremal problems and geometric applications,
in \emph{Extremal Problems for Finite Sets
(P. Frankl, Z. F\"uredi, G. Katona, D. Mikl\'os, editors)},
vol.~3 of Bolyai Society Mathematical Studies, Budapest, 1994,
pp.~251--282.

\bibitem{GN06}
J. Garc\'ia-L\'opez and C. Nicol\'as,
Planar point sets with large minimum convex partitions,
in \emph{Proc. 22nd European Workshop on Comput. Geom.},
Delphi, Greece, 2006, pp.~51--54.

\bibitem{GL05}
M. Grantson and C. Levcopoulos,
A fixed-parameter algorithm for the minimum number convex partition problem,
in \emph{Proc. Japanese Conf. on Discrete Comput. Geom.}, vol.~3742 of
LNCS, Springer, 2005, pp.~83--94.

\bibitem{hkkms-flspp-06}
O.~A. Hall-Holt, M.~J. Katz, P.~Kumar, J.~S.~B. Mitchell, and A.~Sityon,
Finding large sticks and potatoes in polygons,
in \emph{Proc. 17th ACM-SIAM Sympos. Discrete Algorithms},
2006, pp.~474--483.

\bibitem{hp-osafd-98a}
S.~Har-Peled,
An output sensitive algorithm for discrete convex hulls,
\emph{Comput. Geom. Theory Appl.} \textbf{10} (1998), 125--138.

\bibitem{HS92}
J.~Hershberger and S.~Suri,
Applications of a semi-dynamic convex hull algorithm,
\emph{BIT} \textbf{32(2)} (1992), 249--267.

\bibitem {H83}
J. Horton,
Sets with no empty convex 7-gons,
\emph{Canadian Math. Bulletin} \textbf{26} (1983), 482--484.

\bibitem {K00}
J.~M.~Keil,
Polygon partition, in
\emph{Handbook of Computational Geometry (J.-R. Sack, J.~Urrutia, editors)},
Elsevier, 2000, pp.~491--518.

\bibitem{KS02}
J.~M.~Keil and J.~Snoeyink, On the time bound for convex partition
of simple polygons,
\emph{Internat. J. Comput. Geom. Appl.} \textbf{12} (2002), 181--192.

\bibitem{Ki80}
D. G. Kirkpatrick, A note on Delaunay and optimal triangulations,
\emph{Inform. Proc. Lett.} \textbf{10(3)} (1980), 127--128.

\bibitem{KS06}
C.~Knauer and A.~Spillner,
Approximation algorithms for the minimum convex partition problem,
in \emph{Proc.~SWAT}, vol.~4059 of LNCS, Springer, 2006, pp.~232--241.

\bibitem{Las11}
M.~Lassak,
Approximation of convex bodies by inscribed simplices of maximum volume,
\emph{Contributions to Algebra and Geometry} \textbf{52(2)} (2011), 389--394.

\bibitem{LM09}
J.~Lawrence and W.~Morris,
Finite sets as complements of finite unions of convex sets,
\emph{Discrete Comput. Geom.} \textbf{42(2)} (2009), 206--218.


\bibitem{Ma02}
J. Matou\v{s}ek,
\emph{Lectures on Discrete Geometry}, Springer, 2002.

\bibitem{ms-sphsg-92}
E.~A. Melissaratos and D.~L. Souvaine,
Shortest paths help solve geometric optimization problems in planar regions,
\emph{SIAM J. Comput.} \textbf{21(4)} (1992), 601--638.

\bibitem{NRU04}
V.~Neumann-Lara, E.~Rivera-Campo, and J.~Urrutia,
A note on convex partitions of a set of points in the plane,
\emph{Graphs and Combinatorics}  \textbf{20(2)} (2004), 223--231.

\bibitem{PT12}
J.~Pach and G.~Tardos,
Piercing quasi-rectangles---on a problem of Danzer and Rogers,
\emph{J. of Combinatorial Theory, Series A} \textbf{119} (2012),
1391--1397.

\bibitem{SU09}
T.~Sakai and J.~Urrutia,
Convex partitions of point sets in the plane,
in \emph{Proc. 7th Japan Conf. on Comput. Geom. and Graphs
(Kanazawa, 2009)}, JAIST.

\bibitem{Sh92}
T. Shermer,
Recent results in art galleries,
\emph{Proceedings of the IEEE} \textbf{80(9)} (1992), 1384--1399.

\bibitem{Sp08}
A. Spillner,
A fixed parameter algorithm for optimal convex partitions,
\emph{J. Discrete Algorithms} \textbf{6(4)}  (2008), 561--569.

\bibitem{Va92}
P. Valtr,
Sets in  with no large empty convex subsets,
\emph{Discrete Mathematics} \textbf{108(1-3)} (1992), 115--124.

\end{thebibliography}

\end{document}
