\documentclass{llncs}
\usepackage{amssymb,amsmath}
  \usepackage{enumerate}
  \usepackage{graphicx}
  \usepackage{verbatim}





  \newcommand{\Alp}{\textsf{Alph}}
  \newcommand{\proot}{\textsf{root}}
  \newcommand{\per}{\textsf{per}}
  \newcommand{\card}{\textsf{card}}
  \newcommand{\runs}{\rho}
  \newcommand{\cubicruns}{\rho_{cubic}}
  \newcommand{\rexp}{\sigma}
  \newcommand{\cubicrexp}{\sigma_{cubic}}

  \newcommand{\nth}{\mbox{}}
  \newcommand{\nd}{\mbox{}}
  \newcommand{\rd}{\mbox{}}

  \def\rdots{\mathinner{\ldotp\ldotp}}
  \newcommand{\eqdef}{\ensuremath{\stackrel{\text{\tiny def}}{=}}}

  \date{}
  \author{\bf
    Maxime Crochemore\inst{1}\fnmsep\inst{3}
    \and
    Marcin Kubica\inst{2}
    \and
    Jakub Radoszewski\thanks{Some parts of this paper were written during the author's Erasmus exchange
      at King's College London
    }
    \inst{2}
    \and \\
    Wojciech Rytter\inst{2}\fnmsep\inst{5}
    \and
    Tomasz Wale\'n\inst{2}
  }


  \institute{
    King's College London, London WC2R 2LS, UK \\
    \email{maxime.crochemore@kcl.ac.uk}
    \and
    Dept.~of Mathematics, Computer Science and Mechanics, \\
    University of Warsaw, Warsaw, Poland\\
    \email{[kubica,jrad,rytter,walen]@mimuw.edu.pl}
    \and
    Universit\'e Paris-Est, France
    \and
    Dept. of Math. and Informatics,\\
    Copernicus University, Toru\'n, Poland
  }






  \title{
    On the Maximal Sum of Exponents\\ of Runs in a String
  }


\begin{document}
  \maketitle
  \begin{abstract}
    A run is an inclusion maximal occurrence in a string (as a subinterval)
    of a repetition  with a period  such that .
    The exponent of a run is defined as  and is .
    We show new bounds on the maximal sum of exponents of runs in
    a string of length .
    Our upper bound of  is better than the best previously known
    proven bound of  by Crochemore \& Ilie (2008).
    The lower bound of , obtained using a family of binary words,
    contradicts the conjecture of Kolpakov \& Kucherov (1999)
    that the maximal sum of exponents of runs in a string of length 
    is smaller than .
  \end{abstract}


  \section{Introduction}
    Repetitions and periodicities in strings are one of the fundamental topics in
    combinatorics on words \cite{Karhumaki,Lothaire}.
    They are also important in other areas: lossless compression, word representation, computational biology, etc.
    In this paper we consider bounds on the sum of exponents of repetitions that a string
    of a given length may contain.
    In general, repetitions are studied also from other points of view, like:
    the classification of words (both finite and infinite) not containing repetitions of a given exponent,
    efficient identification of factors being repetitions of different types
    and computing the bounds on the number of various types of
    repetitions occurring in a string.
    The known results in the topic and a deeper description of the
    motivation can be found in a survey by Crochemore et al.~\cite{Survey}.

    The concept of runs (also called maximal repetitions) has been introduced to
    represent all repetitions in a string in a succinct manner.
    The crucial property of runs is that their maximal number in a string of
    length  (denoted as ) is , see Kolpakov \& Kucherov \cite{KolpakovKucherov}.
    This fact is the cornerstone of any algorithm computing all repetitions in
    strings of length  in  time.
    Due to the work of many people, much better bounds on  have been obtained.
    The lower bound  was first proved by Franek \& Yang \cite{Franek08}.
    Afterwards, it was improved by Kusano et al.~\cite{Matsubara} to  employing computer experiments, 
    and very recently by Simpson~\cite{Simpson10} to .
    On the other hand, the first explicit upper bound  was settled by Rytter~\cite{Rytter06}, 
    afterwards it was systematically improved to  by Puglisi et al.~\cite{Puglisi08},
     by Rytter~\cite{Rytter07},
     by Crochemore \& Ilie~\cite{CrochemoreIlie,Crochemore08} and  by Giraud~\cite{Giraud08}.
    The best known result  is due to Crochemore et 
    al.~\cite{DBLP:conf/cpm/CrochemoreIT08}, but it is conjectured
    \cite{KolpakovKucherov} that .
    Some results are known also for repetitions of exponent higher than 2.
    For instance, the maximal number of cubic runs (maximal repetitions with exponent at least 3)
    in a string of length  (denoted ) is known to be between  and ,
    see Crochemore et al.~\cite{Lata10}.

    A stronger property of runs is that the maximal sum of their exponents in a string
    of length  (notation: ) is linear in terms of , see Kolpakov \& Kucherov~\cite{KolpakovKucherovLORIA}.
    It has applications to the analysis of various algorithms, such as
    computing branching tandem repeats: the linearity of the sum of exponents
    solves a conjecture of \cite{Gusfield98} concerning the linearity of the number of maximal
    tandem repeats and implies that all can be found in linear time.
    For other applications, we refer to \cite{KolpakovKucherovLORIA}.
    The proof that  in Kolpakov and Kucherov's paper \cite{KolpakovKucherovLORIA} is very complex
    and does not provide any particular value for the constant .
    A bound can be derived from the proof of Rytter \cite{Rytter06} but he mentioned only
    that the bound that he obtains is ``unsatisfactory'' (it seems to be ).
    The first explicit bound  for  was provided by Crochemore and Ilie \cite{Crochemore08},
    who claim that it could be improved to  employing computer experiments.
    As for the lower bound on , no exact values were previously known and
    it was conjectured \cite{Kolpakov99,KolpakovKucherovLORIA} that .

    In this paper we provide an upper bound of  on the maximal sum of exponents
    of runs in a string of length  and also a stronger upper bound of  for the
    maximal sum of exponents of cubic runs in a string of length .
    As for the lower bound, we bring down the conjecture  by
    providing an infinite family of binary strings for which the sum of exponents of runs is greater than
    .


  \section{Preliminaries}
    We consider \emph{words} (\emph{strings})  over a finite alphabet , ;
    the empty word is denoted by ;
    the positions in  are numbered from  to .
For , let us denote by  a \textit{factor}
    of  equal to  (in particular ).
Words  are called prefixes of , and words  suffixes of .

    We say that an integer  is the (shortest) \emph{period} of a word
     (notation: ) if  is the smallest positive integer such
    that  holds for all .
We say that words  and  are cyclically equivalent (or that one of them is a cyclic
    rotation of the other) if  and  for some .


    A \emph{run} (also called a maximal repetition) in a string  is an interval
     such that:
    \begin{itemize}
      \item
        the period  of the associated factor  satisfies
        , 
      \item 
        the interval cannot be extended to the right nor to the left, without violating the above property, 
        that is,  and .
    \end{itemize}
    A \emph{cubic run} is a run  for which the
    shortest period  satisfies .
    For simplicity, in the rest of the text we sometimes refer to runs and cubic runs as
    to occurrences of the corresponding factors of .
    The (fractional) \emph{exponent} of a run is defined as .

    For a given word , we introduce the following notation:
    \begin{itemize}
      \item  and  are the numbers of runs and cubic
      runs in  resp.
      \item  and  are the sums of exponents of runs and cubic
      runs in  resp.
    \end{itemize}
    For a non-negative integer , we use the same notations , ,
     and  to denote the maximal value of the respective function
    for a word of length .


  \section{Lower bound for }
    Tables \ref{fig:franek} and \ref{fig:padovan}
    list the sums of exponents of runs for several words of two known families
    that contain very large number of runs: the words  defined by
    Franek and Yang \cite{Franek08} (giving the lower bound ,
    conjectured for some time to be optimal)
    and the modified Padovan words  defined by Simpson \cite{Simpson10}
    (giving the best known lower bound ).
    These values have been computed experimentally.
    They suggest that for the families of words  and  the maximal sum of
    exponents could be less than .

    \begin{table}
      \begin{center}
      \begin{tabular*}{0.6\textwidth}{@{\extracolsep{\fill}}|r|r|r|r|r|}
      \hline
 &  &  &  &  \\\hline
 &  &  &  &  \\\hline
 &  &  &  &  \\\hline
 &  &  &  &  \\\hline
 &  &  &  &  \\\hline
 &  &  &  &  \\\hline
 &  &  &  &  \\\hline
 &  &  &  &  \\\hline
 &  &  &  &  \\\hline
 &  &  &  &  \\\hline
      \end{tabular*}
      \end{center}
      \caption{\label{fig:franek}
        Number of runs and sum of exponents of runs in Franek \& Yang's \cite{Franek08} words .
      }
    \end{table}

    \begin{table}
      \begin{center}
      \begin{tabular*}{0.6\textwidth}{@{\extracolsep{\fill}}|r|r|r|r|r|}
      \hline
 &  &  &  &  \\\hline
 &  &  &  &  \\\hline
 &  &  &  &  \\\hline
 &  &  &  &  \\\hline
 &  &  &  &  \\\hline
 &  &  &  &  \\\hline
 &  &  &  &  \\\hline
 &  &  &  &  \\\hline
 &  &  &  &  \\\hline
 &  &  &  &  \\\hline
 &  &  &  &  \\\hline
      \end{tabular*}
      \end{center}
      \caption{\label{fig:padovan}
        Number of runs and sum of exponents of runs in Simpson's \cite{Simpson10} modified Padovan words .
      }
    \end{table}


  We show, however, a lower bound for  that is greater than .

  \begin{theorem}
    There are infinitely many binary strings  such that
    
  \end{theorem}

  \begin{proof}
    Let us define two morphisms  and  as follows:
    
    
    We define .
    Table~\ref{fig:lower} shows the sums of exponents of runs in words , computed experimentally.

    Clearly, for any word , , we have
    


    \begin{table}
      \begin{center}
      \begin{tabular*}{0.5\textwidth}{@{\extracolsep{\fill}}|r|r|r|r|}
      \hline
 &  &  &  \\\hline
 &  &  &  \\\hline
 &  &  &  \\\hline
 &  &  &  \\\hline
 &  &  &  \\\hline
 &  &  &  \\\hline
 &  &  &  \\\hline
 &  &  &  \\\hline
 &  &  &  \\\hline
 &  &  &  \\\hline
 &  &  &  \\\hline
      \end{tabular*}
      \end{center}
      \caption{\label{fig:lower}
        Sums of exponents of runs in words .
      }
    \end{table}


\begin{comment}
    \bigskip
  \noindent
  \emph{TODO: Analysis of structure of runs in .
  There are also some alternative families of words that could be analyzed instead:}
\begin{verbatim}
a -> aaca
b -> aac 
c -> b
a -> 0101101011010
b -> 01011010
c -> 11010
The limit is at least: 2.03482

a -> abababbaba
b -> babbaba
a -> 10100101
b -> 10101
The limit is at least: 2.03341
a -> abababbaba
b -> babbaba
a -> 10100101
b -> 00101
The limit is at least: 2.03343
a -> ababb
b -> abbabbbabb
a -> 101
b -> 00101
The limit is at least: 2.03349
a -> ababb
b -> abbbabbabb
a -> 101
b -> 00101
The limit is at least: 2.03350
\end{verbatim}
\end{comment}
  \qed
  \end{proof}

  \section{Upper bounds for  and }
    In this section we utilize the concept of \emph{handles} of runs as
    defined in \cite{Lata10}.
    The original definition refers only to cubic runs, but here we extend it also to ordinary runs.

    Let  be a word of length .
    Let us denote by  the set of inter-positions in  
    that are located \emph{between} pairs of consecutive letters of .
    We define a function  assigning to each run  in  
    a set of some inter-positions within  (called later on \emph{handles}) --- 
     is a mapping from the set of runs occurring in  to the set  of subsets of .
    Let  be a run with period  and let  be the prefix of  of length .
    Let  and  be the minimal and maximal words (in lexicographical order)
    cyclically equivalent to .
     is defined as follows:
    \begin{enumerate}[a)]
      \item
      if  then  contains all inter-positions
      within ,
      \item
      if  then  contains inter-positions between
      consecutive occurrences of  in  and between consecutive
      occurrences of  in .
    \end{enumerate}
    Note that  can be empty for a non-cubic-run .

    \begin{figure}[th]
    \begin{center}
      \includegraphics[width=7cm]{rys3}
      \caption{\label{f:handles_ex}
          An example of a word with two highlighted runs  and .
          For  we have  and for 
          the corresponding words are equal to  (a one-letter word).
          The inter-positions belonging to the sets  and 
          are pointed by arrows
      }
    \end{center}
    \end{figure}

    Proofs of the following properties of handles of runs can be found in \cite{Lata10}:
    \begin{enumerate}
      \item Case (a) in the definition of  implies that .
      \item  for any two distinct runs  and  in .
    \end{enumerate}

    To prove the upper bound for , we need to state an additional property of
    handles of runs.
    Let  be the set of all runs in a word , and let
     and  be the sets of runs with period
    1 and at least 2 respectively.

    \begin{lemma}
      \label{lem:propH}~

    \noindent
      If  then
      .\\
      If  then
      .
    \end{lemma}

    \begin{proof}
      For the case of , the proof is straightforward from the definition of handles.
      In the opposite case, it is sufficient to note that both words
       and  for  are factors of , and thus
      
      \qed
    \end{proof}

    Now we are ready to prove the upper bound for .
    In the proof we use the bound  on the number of runs from \cite{DBLP:conf/cpm/CrochemoreIT08}.

    \begin{theorem}\label{thm:upper_bound}
      The sum of the exponents of runs in a string of length  is less than .
    \end{theorem}

    \begin{proof}
      Let  be a word of length .
      Using Lemma \ref{lem:propH}, we obtain:
      
      where  and .
      Due to the disjointness of handles of runs (the second property of handles), , and
      thus, .
      Combining this with \eqref{eq:Ru}, we obtain:
      
      \qed
    \end{proof}


    A similar approach for cubic runs, this time using the bound of  for  from \cite{Lata10},
    enables us to immediately provide a stronger upper bound for the function .

    \begin{theorem}
      The sum of the exponents of cubic runs in a string of length  is less than .
    \end{theorem}

    \begin{proof}
      Let  be a word of length .
      Using same inequalities as in the proof of Theorem \ref{thm:upper_bound}, we obtain:
      
      where  denotes the set of all cubic runs of .
      \qed
    \end{proof}


  \bibliographystyle{abbrv}
  \bibliography{exprun}

\end{document}
