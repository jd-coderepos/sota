In the previous section we outlined the \clientcash~mechanism using the randomized algorithm \selpred~in a black-box manner. We now examine the 
exact formulation of this algorithm.

\subsection{Security Requirements}
The probability that  yields a particular outcome  may depend on the input password . Indeed, our goal is to introduce a cost asymmetry so that, in expectation, the stopping time  for the correct password is less than stopping time  for any incorrect passwords . One natural way to achieve this asymmetry might be to define a family of predicates 
for each password  and . Then  might select a stopping time  at random from some distribution and set  where  for each . This solution could provide an extreme cost asymmetry. In particular, we would have  for the correct password and with high probability we would have  for all other passwords. Indeed, this solution is simply a reformulation of Boyen's~\cite{boyen2007halting} halting puzzles.  

The downside to this solution is that it would enable an adversary to mount an offline attack using {\em only} state from the client (e.g., without breaching the authentication server) because the output  of  is stored on the client and this output would include the hash value . 

In contrast to Boyen~\cite{boyen2007halting}, we will require the  function to satisfy a very stringent information theoretic security property. In particular, for every  and for all  with , we will require that 
Here,  is the security parameter which upper bounds the amount of information about the user's password that is leaked by the outcome .  This requirement is exactly equivalent to the powerful notion of -differential privacy~\cite{dwork2006calibrating}. Intuitively, the probability that we produce a particular outcome  cannot depend too much on particular password  that the user chose. 

{\bf Discussion.} Differential privacy is an information theoretic guarantee, which holds even if the adversary has background knowledge about the user. The security guarantees are quite strong. In particular, let  denote {\em any} attack that an adversary with background knowledge  might use after observing the outcome  from  and let  denote the set of outcomes that our user Alice would consider harmful. It is easy (e.g., see \cite{dwork2013algorithmic}) to show that 

Intuitively, the numerator represents the adversary's actual success rate and the denominator represents the adversary's success rate if the output of  were not even correlated with Alice's real password. Thus, the constraint implies that the probability the adversary's attack succeeds cannot greatly depend on the output of the  function. 

\subsection{Cost Requirements}
Performing client-side key stretching requires computational resources on the client side. We will require that the amortized cost of hashing a password does not exceed the maximum amortized cost the user is willing to bear. Let  denote the user's maximum amortized cost and let  denote the cost of executing  one time. Finally, for a fixed password  let  denote the probability that  returns an outcome  which yields stopping time   for the password . Given a password  the expected cost of executing  in \clientcash~ is 
 where the expectation is taken over the randomness of the algorithm .  Thus, we require that 
 

\subsection{Symmetric Predicate Sets}
In this section we show how to simplify the security and cost constraints from the previous sections by designing  with certain symmetric properties. In particular, we need to design the  algorithm in such a way that the cost constraint (eq. \ref{eq:CostConstraint}) is satisfied for 
{\em all passwords} , and we need to ensure that our security constraint (eq. \ref{eq:Security}) holds for {\em all pairs} of passwords . Working with  different security constraints is unwieldy. Thus, we need a way to simplify these requirements. 

 One natural way to simplify these requirements is to construct a \textit{symmetric} ; that is, for all  and for all , . We stress that this restriction does not imply that the sets  and  are the same, simply that their sizes are the same. We will also require that 
  as there is no clear reason to favor one outcome over the other since both outcomes  yield the same halting time for their respective passwords  and  (i.e, ) . 

\noindent{\bf Notation for Symmetric .} Given a symmetric , we define . We define  for an arbitrary password  and output  since, by symmetry, the choice of  and  do not matter. Intuitively,  denotes the probability that  outputs  with stopping time . We will also write  instead of  because the value will be the same for all passwords .


\cut{We can similarly write the rest of the system properties in terms of the 's and 's which gives us a more convenient way of representing the mechanisms independent of the specific password used. A full list of properties and descriptions using both the traditional and equivalence class notations are shown in Table 2. In the \selpred algorithm, the probability distribution  over the elements of  is thus dependent on  (which is accessible to the adversary) as well as the partition which can only be computed when  is known. \\}

Now the  algorithm can be completely specified by the values  for . It remains to construct a symmetric . The construction of a symmetric  used in this work is simple we will focus on stopping predicates with the following form: 



Theorem \ref{thm:symmetric_output} says that we can use these predicates to construct a symmetric set . 

\begin{theorem}
\label{thm:symmetric_output}
Let integers  be given and suppose that    ,, . Then   is symmetric, i.e. for all  and for all  we have . 
\end{theorem}

One of the primary advantages of using a symmetric  is that we can greatly reduce the number of security and cost constraints. Table \ref{tab:Constraints} compares the security and cost constraints with and without symmetry. Without symmetry we had to satisfy separate cost constraints for all passwords . With symmetry we only have to satisfy one cost constraint.  Similarly, with symmetry we only need to satisfy one security constraint for each  instead of multiple constraints for each pair of passwords  that the user might select. While the space of passwords may be very large, , the number of rounds of hashing, will typically be quite small (e.g., in this paper ).  \fullversion{See Theorems \ref{thm:diff_privacy} and \ref{thm:cost_constraints} in the appendix for formal statements of these results and their proofs.} \confversion{See the full version of this paper for proofs of the statements in Table \ref{tab:Constraints}.}

In our analysis of \clientcash~we will focus on two simple symmetric constructions of :
\begin{itemize}
\item ``Two Round" Case: set  in Theorem \ref{thm:symmetric_output} so .
\item ``Three Round" Case: set  in Theorem \ref{thm:symmetric_output} so . 
\end{itemize}
We do not rule out the possibility that other cases would yield even stronger results. However, we are already able to obtain significant reductions in the adversary's success rate with these simple cases.

\cut{
\newcommand{\thmdiffprivacy}{
Suppose that  is constructed as specified in Theorem \ref{thm:symmetric_output}, equation \ref{eq:symmetry} holds and . Then . 
}

\begin{theorem}
\label{thm:diff_privacy}
\thmdiffprivacy
\end{theorem}}


\cut{
\newcommand{\thmcostconstraints}{
Suppose that  and the 's are constructed as specified in Theorems \ref{thm:symmetric_output} and \ref{thm:diff_privacy}, and that . Then for all  we have . 
}


\begin{theorem}
\label{thm:cost_constraints}
\thmcostconstraints
\end{theorem}}


\cut{
From a security perspective, this method of finding the distribution used to select  depends on the user's choice of password . Without this password, the adversary can only compute 's and not the probabilities of selecting individual . }

\begin{table}[t]
\centering
\begin{tabular}{|c|c|c|}
\hline
No Symmetry & Constraint & With Symmetry \\
\hline
\cut{ & probability constraint & \\
 & &\\
& & \\
 & probability constraint & \\
&&\\}
 & Security  &   \\
 &  \fullversion{(Thm  \ref{thm:diff_privacy})}\confversion{\cite[Thm 4]{fullVersion}}   & \\
&  &\\
\hline
\cut{ & Stopping time probabilities & \\
& &\\}
   & & \\
  & Cost & \\
 & \fullversion{(Thm \ref{thm:cost_constraints})}\confversion{\cite[Thm 5]{fullVersion}} & \\
 \hline
\end{tabular}
\vspace{-0.2cm}
\caption{Security and cost constraints for  with symmetric predicate set .}
\label{tab:Constraints}

\end{table}

\cut{
\begin{algorithm}[tbh]
\begin{algorithmic}
\State {\bf Input:} 
State {\bf Parameters: } 
\State 
\State 
\State 
\State {\bf Return} 
\end{algorithmic}
\caption{SelectPredicates (Client Side) \jnote{Where is the  notation defined? Possibly move to appendix}}
\label{alg:SelectPredicates}
\end{algorithm} 
}

