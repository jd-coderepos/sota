\documentclass[envcountsame]{fsttcs-ps}

\usepackage{gastex}
\usepackage[utf8x]{inputenc}
\usepackage[T1]{fontenc}
\usepackage{amsmath,amssymb}
\usepackage{xspace}
\usepackage{subfigure}
\usepackage{wasysym}
\usepackage{array}
\usepackage{multicol}



\newenvironment{example}{\theoremlike{Example}}{\par\medskip}

\newcommand{\until}{{Until}\xspace}
\newcommand{\since}{{Since}\xspace}
\newcommand{\tomorrow}{{Next}\xspace}
\newcommand{\eventually}{{Eventually}\xspace}
\newcommand{\always}{{Always}\xspace}

\newcommand{\A}{\mathcal A}
\newcommand{\B}{\mathcal B}
\newcommand{\firstautomaton}{\A_1}
\newcommand{\secondautomaton}{\A_2}
\newcommand{\AP}{\mathrm{AP}}
\newcommand{\J}{J}
\newcommand{\K}{K}
\newcommand{\Until}{{\mathrel{\mathcal U}}}
\newcommand{\Since}{{\mathrel{\mathcal S}}}
\newcommand{\ie}{\textit{i.e.}\xspace}
\newcommand{\eg}{e.g.\xspace}
\newcommand{\truth}{\ensuremath{v}}
\renewcommand{\phi}{\varphi}
\newcommand{\reals}{\mathbb{R}}
\newcommand{\neXt}{\mathop{\mathcal X}}
\newcommand{\Always}{\mathop{\mathcal G}}
\newcommand{\Eventually}{\mathop{\mathcal F}}
\newcommand{\FO}{\mathrm{FO}\xspace}
\newcommand{\pspace}{\textsc{PSpace}\xspace}
\newcommand{\expspace}{\textsc{ExpSpace}\xspace}
\newcommand{\exptime}{\textsc{ExpTime}\xspace}

\newcommand{\qzero}{q_0}
\newcommand{\qone}{q_1}
\newcommand{\qtwo}{q_2}
\newcommand{\qhole}{q_3}
\newcommand{\qfour}{q_4}
\newcommand{\qfive}{q_5}
\newcommand{\qsix}{q_6}
\newcommand{\qseven}{q_7}
\newcommand{\qeight}{q_8}
\newcommand{\qnine}{q_9}

\newcommand{\shuffle}{\mathop{sh}}

\makeatletter
\newenvironment{tablehere}
  {\def\@captype{table}}
  {}

\newenvironment{figurehere}
  {\def\@captype{figure}}
  {}
\makeatother

\setlength{\abovecaptionskip}{0pt}
\setlength{\belowcaptionskip}{-10pt}
\renewcommand{\subfigcapskip}{-5pt}



\begin{document}

\title{Automata and temporal logic over arbitrary linear time}
\author{Julien Cristau}
\affiliation{
LIAFA --- CNRS \& Université Paris 7
}

\runningtitle{Automata and temporal logic over arbitrary linear time}
\runningauthors{Julien Cristau}

\begin{abstract}
Linear temporal logic was introduced in order to reason about reactive
systems.  It is often considered with respect to infinite words, to specify
the behaviour of long-running systems.  One can consider more general models
for linear time, using words indexed by arbitrary linear orderings.  We
investigate the connections between temporal logic and automata on linear
orderings, as introduced by Bruyère and Carton.  We provide a doubly
exponential procedure to compute from any LTL formula with \until, \since, and
the Stavi connectives an automaton that decides whether that formula holds on
the input word.  In particular, since the emptiness problem for these automata
is decidable, this transformation gives a decision procedure for the
satisfiability of the logic.
\end{abstract}

\section{Introduction}

Temporal logic, in particular LTL, was proposed by Pnueli to specify the
behaviour of reactive systems~\cite{DBLP:conf/focs/Pnueli77}.  The model of
time usually considered is the ordered set of natural numbers, and executions
of the system are seen as infinite words on some set of atomic propositions.
This logic was shown to have the same expressive power as the first order
logic of order~\cite{Kamp}, but it provides a more convenient formalism to express
verification properties.  It is also more tractable: while the satisfiability
problem of  is non-elementary~\cite{stockmeyer74}, it was shown
in~\cite{DBLP:journals/jacm/SistlaC85} that the decision problem of LTL with
\until and \since on -words is \pspace-complete.  This logic has also
strong ties with automata, with important work to provide efficient
translations to Büchi automata, e.g.~\cite{DBLP:conf/cav/GastinO01}.

Within this time model, a number of extensions of the logic and the automata
model have been studied.  But one can also consider more general models of
time: general linear time could be useful in different settings, including
concurrency, asynchronous communication, and others, where using the set of
integers can be too simplistic.  Possible choices include ordinals, the reals,
or even arbitrary linear orderings.  In terms of expressivity, while LTL with
\until and \since is expressively complete (\ie equivalent to ) on
Dedekind-complete orderings (which includes the ordering of the reals as
well as all ordinals), this does not hold in the general case.  Two more
connectives, the future and past Stavi operators, are necessary to handle
gaps~\cite{DBLP:conf/popl/GabbayPSS80} when considering arbitrary linear
orderings.

Over ordinals, LTL with \until and \since has been shown to have a
\pspace-complete satisfiability problem~\cite{DBLP:conf/lpar/DemriR07}.  Over
the ordering of the real numbers, satisfiability of LTL with until and since
is \pspace-complete, but satisfiability of MSO is undecidable.  Over
general linear time, first order logic has been shown to be decidable, as well
as universal monadic second order logic.  Reynolds shows
in~\cite{DBLP:journals/jcss/Reynolds03} that the satisfiability problem of
temporal logic with only the \until connective is also \pspace-complete,
and conjectures that this might stay true when adding the \since
connective.  The upper bound in~\cite{DBLP:conf/lpar/DemriR07} is obtained by
reducing the satisfiability of LTL formulae to the accessibility problem in an
appropriate automata model, accepting words indexed by ordinals.
In this paper, we focus on the general case of arbitrary linear orderings,
using the full logic with \until, \since and both Stavi connectives.
Our aim is to investigate the connections between LTL and automata in this
setting.

Automata on linear orderings were introduced by Bruyère and
Carton~\cite{DBLP:conf/mfcs/BruyereC01}.  This model extends traditional finite
automata using ``limit'' transitions to handle positions with no successor
or predecessor, furthering Büchi's model of automata on words of ordinal
length~\cite{buchiordinals}.  Carton showed in~\cite{DBLP:conf/mfcs/Carton02}
that accessibility over scattered ordering is decidable in polynomial time,
and in~\cite{DBLP:journals/ijfcs/RispalC05} it was shown that these automata
can be complemented over countable scattered linear orderings.  The
accessibility result can be extended to arbitrary
orderings~\cite{cartonprcomm}.

From any formula in this logic, we define an automaton which determines
whether the formula holds on its input word.  Satisfiability of the formula is
reduced to accessibility in this automaton, and that way we get decidability
of the satisfiability problem of LTL with \until, \since and the Stavi
operators for any rational subclass.

Section~\ref{s:defs} presents some definitions about linear orderings, linear
temporal logic, and the model of automata used.
Section~\ref{s:trans} introduces our main result, an algorithm to translate
any LTL formula into a corresponding automaton.
Section~\ref{s:discuss} discusses the expressivity of the logic and automata
considered, and looks at some natural fragments.

\section{Definitions}
\label{s:defs}
\subsection{Linear orderings}

We first recall some basic definitions about orderings, and introduce some
notations.  For a complete introduction to linear orderings, the reader is
referred to~\cite{Rosenstein82}.
A \emph{linear ordering}  is a totally ordered set  (considered
modulo isomorphism).
The sets of integers (), of rational numbers (), and of real
numbers with the usual orderings are all linear orderings.


Let  and  be two linear orderings.  One defines the reversed ordering
 as the ordering obtained by reversing the relation  in , and
the ordering  as the disjoint union  extended with  for
any  and .
For example,  is the ordering of negative integers.  
is the usual ordering of , also denoted by .


A non-empty subset  of an ordering  is an \emph{interval} if for any
 in , if  and  then .
In order to define the runs of an automaton, we use the notion of cut.
A \emph{cut} of an ordering  is a partition  of  such that for
any  and , .
We denote by  the set of cuts of .  This set is equipped with the
order defined by  if .  This ordering
can be extended to  in a natural way ( iff ).
Notice that  always has a smallest and a biggest element, respectively
 and .
For example, the set of cuts of the finite ordering  is the
ordering , and the set of cuts of  is .

For any element  of , there are two successive cuts  and
, respectively  and
.    A \emph{gap} in an ordering
 is a cut  which is not an extremity ( or ), and has
neither a successor nor a predecessor.

Given an alphabet , a \emph{word} of length  is a sequence
 of elements of  indexed by .
For example, 
 is a word of length ;
the sequence  is a word of length ,
and
 is a word of length .


\subsection{Temporal logic}

We use words over linear orderings to model the behaviour of systems
over linear time.  To express properties of these systems, we consider
linear temporal logic.
The set of LTL formulae is defined by the following grammar, where  ranges
over a set  of atomic propositions:


Besides the usual boolean operators, we have four temporal connectives.  The
 connective is called ``\until'', and  is called ``\since''.
 and  are respectively the future and past Stavi
connectives.  Other usual connectives such as ``\tomorrow'' (),
``\eventually'' (), ``\always'' () can be defined using
these, as we see below.





These formulae are interpreted on words over the alphabet .
A letter in those words is the set of atomic propositions that hold at the
corresponding position.
Let  a word of length .  A formula  is
evaluated at a particular position  in ; we say that  holds at
position  in , and we write , using the following
semantics:


Note that we use a ``strict'' semantic for the \until operator, contrary to a
common definition, which would be:


\noindent In the strict version, the current position  is not considered
for either the  or the  part of the definition.  Using the
strict or non-strict version makes no difference when considering
-words, but in the case of arbitrary orderings, the strict \until is
more powerful, as noted by Reynolds in~\cite{DBLP:journals/jcss/Reynolds03}.

The formula ``\tomorrow '', or , is equivalent to
.  ``Eventually '', noted , is
, and ``always '', noted
, can be expressed as .



Given a word  of length , the \emph{truth word} of  on  is
the word  of length  over the alphabet 
where the position  is labelled by  iff .  A formula
is \emph{valid} if its truth word on any input only has ones.  A formula is
\emph{satisfiable} if there exists an input word such that the truth word
contains a one.

Consider the formula , with
.  If  (where  stands for ),
then  (at every position, either  is true or
 is true in the successor).  On the other hand, if , then : at positions 0,  and at the last position,  is true
so the formula doesn't hold; at all other positions,  is false, and there
is no position in the input word where  holds.


The \emph{satisfiability problem} for a formula  consists in deciding
whether there exists a word  and a position  in  such that
.
As  is decidable, and every LTL formula can be expressed using first
order, satisfiability of LTL is decidable.  Note however that in terms of
complexity  is already non-elementary on finite
words~\cite{stockmeyer74}, which is not true of LTL.

\subsection{Automata}

On infinite words, Büchi automata can be used to decide satisfiability of LTL
formulae.  In the case of words over linear orderings, a model of automata has
been introduced in~\cite{DBLP:conf/mfcs/BruyereC01}.  Instead of accepting or
rejecting each input word, as in the case of -words, we use these
automata to compute the truth words corresponding to an LTL formula.  Our
model of automata thus has an output letter on each transition, so they are
actually letter-to-letter transducers, which make composition easier
(see Section~\ref{s:composition}).

An \emph{automaton} is a tuple  where
 is a finite set of states,
 is a finite input alphabet,
 is a finite output alphabet,
 and  are subsets of , respectively the set of initial and
final states,
and  is the set of transitions.  We note:
\begin{itemize}
\item  if  (\emph{successor}
transition)
\item  if  (\emph{left limit} transition)
\item  if  (\emph{right limit} transition).
\end{itemize}


Consider a word  over .
We define the left and right limit sets of  at position  as the
sets of labels that appear arbitrarily close to  (respectively to its left
and to its right).  Formally:
\begin{center}
\\

\end{center}
Note that  is non-empty if and only if the transition to  is a
left limit, and similarly for  if the transition from  is a
right limit.  These sets help define the possible limit transitions in a run.

Given an automaton , an accepting run of  on a word  is a word  of length  over  such that:
\begin{itemize}
\item  and ;
\item for each , there exists  such that
;
\item if  has no predecessor,
, and if  has no successor, .
\end{itemize}



\begin{figure}
\begin{center}
\begin{picture}(80,32)(0,-32)
\node[Nmarks=if](A)(10,-7){}
\node(B)(40,0){}
\node[Nmarks=f,fangle=220](C)(40,-15){}
\drawedge(A,B){}
\drawedge[ELside=r](A,C){}
\drawloop[loopangle=0](B){}
\drawedge[curvedepth=5](B,C){}
\drawedge[curvedepth=5](C,B){}
\drawloop[loopangle=0](C){}

\put(-5,-25){Limits:}
\put(10,-25){ \qquad }
\put(10,-30){for any }
\end{picture}
\begin{picture}(60,32)(0,-32)

\node[Nmarks=if](A)(30,0){}
\node(B)(15,-15){}
\node(C)(45,-15){}
\drawedge(B,C){}

\put(-5,-25){Limits:}
\put(10,-25){}
\put(10,-30){}

\end{picture}
\end{center}
\caption{Example automata}
\label{fig:automata}
\end{figure}





\begin{example}
The first automaton in Figure~\ref{fig:automata} outputs  at each position
immediately followed by a  in the input word, and  at other positions.

The second automaton accepts input words whose length is a linear ordering
without first or last element, and without two consecutive elements (\ie
\emph{dense} orderings).  The notation  means that there
is a transition  and a transition .
\end{example}

In~\cite{DBLP:conf/mfcs/Carton02}, Carton proves that the accessibility
problem on these automata can be solved in polynomial time, when only
considering scattered orderings.  This result can be extended to
arbitrary orderings~\cite{cartonprcomm} as it is done for rational expressions
in~\cite{DBLP:journals/ijfcs/BesC06}.  The idea is to build an automaton
over finite words which simulates the paths in the initial automaton and
remembers their contents.  In order to handle the general case (as opposed to
only scattered orderings), the added operation is called ``shuffle'':
 where  is a dense and complete
ordering without a first or last element, partitioned in dense suborderings
, such that  if .  Looking at automata,
this means that if there are paths from  to  with content ,
\dots, from  to  with content , and transitions from
 to each , transitions from each  to
, a transition from  to  and a
transition from  to , then there is a path from 
to~.

\section{Translation between formulae and automata}
\label{s:trans}

Over -words, problems on temporal logics are commonly solved
using tableau methods~\cite{Wolper85thetableau}, or automata-based
techniques~\cite{DBLP:conf/lics/VardiW86}.  In this work we extend the
correspondence between LTL and automata to words over linear orderings.  Our
main result is Theorem~\ref{th:main}.

\begin{theorem}
\label{th:main}
For every LTL formula , there is an automaton  which
given any input word  outputs the truth word .
\end{theorem}

Moreover, this automaton  can be effectively computed, and has a
number of states exponential in the size of .  Because we can compute
the product of  with any given automaton and check for its
emptiness, we get Corollary~\ref{cor:sat}, which states that given a temporal
formula and a rational property (\ie an automaton on words over linear
orderings), we can check whether there exists a model of the formula which is
accepted by the automaton.

\begin{corollary}
\label{cor:sat}
The satisfiability problem for any rational subclass is decidable.
\end{corollary}



The idea is to build  by induction on the formula.  We construct
an elementary automaton for each logical connective. We use composition and
product operations to build inductively the automaton of any LTL formula from
elementary automata.  All automata used in this proof have the particular
property that there exists exactly one accepting run for each possible input
word, \ie they are non-deterministic, but also non-ambiguous.  This property
is preserved by composition and product.

The structure of the proof is the following: we define the composition and
product operators on automata, then we present the elementary automata that are
needed to encode logical connectives. Finally, we give the inductive method to
build the automaton corresponding to a formula from elementary ones.





\subsection{Product, composition and elementary automata}
\label{s:composition}

Let  and
 be two automata.
The product consists in running both automata with the same input alphabet in
parallel, and outputting the combination of their outputs.  If
's output alphabet and 's input alphabet
are the same, the composition consists in running  over
's output.  We use the notation  and
 for the first and second projections.

\begin{definition}
Suppose that  and  have the same input
alphabet, \ie .
The \emph{product} of  and  is the
automaton , where
 contains the following transitions:
\begin{itemize}
\item  if 
and ,
\item  if  and ,
\item  if  and
.
\end{itemize}
\end{definition}

\begin{definition}
Suppose now that the output alphabet of  is the input
alphabet of , \ie .
The \emph{composition}
of  and  is the automaton
.  The transitions in  are:
\begin{itemize}
\item  if 
and ,
\item  if  and
,
\item  if  and
.
\end{itemize}
\end{definition}





Recall that LTL formulae are given by .  For each atomic proposition  we construct an automaton
 which, given a word , outputs .  For each logical
connective of arity , we construct an automaton with input alphabet
, and output alphabet .  The input word is the tuple of
truth words of the connective's variables, the output is the truth word of the
complete formula.  For temporal connectives, we only describe the automata
corresponding to  and .  For the ``past'' connectives, the
automata are the same with all transitions (successor and limits) reversed,
and initial and final states swapped.

For any , the automaton  is
 where .  This automaton simply
outputs  at positions where  is true, and  everywhere else.  Note
that the run is uniquely determined by the input word; such a transducer is
called non-ambiguous.

Figures~\ref{fig:not} and~\ref{fig:or} show the automata corresponding to the
negation () and disjunction () connectives.  Their limit
transitions are  and .  Again, these
automata admit exactly one run for each input word.


\begin{figure}[h]
\begin{center}
\subfigure[Automaton for negation]{
{
\begin{picture}(50,18)(0,-18)
\node[Nmarks=if,iangle=135,fangle=-45](n1)(25,-9.0){}
\drawloop[loopangle=-180](n1){}
\drawloop[loopangle=0](n1){}
\end{picture}
\label{fig:not}
}}
\subfigure[Automaton for disjunction]{
\begin{picture}(50,18)(0,-18)\nullfont
\node[Nmarks=if,iangle=135,fangle=-45](n1)(25,-9.0){}
\drawloop[loopangle=180](n1){}
\drawloop[loopangle=0](n1){  }
\end{picture}
\label{fig:or}
}
\end{center}
\end{figure}





\subsection{Automaton for }

The difficulty starts with the ``\until'' connective ().  We recall
that  holds at position  in a word  if there exists
 such that  holds at , and such that  holds at every
position  such that .

We build an automaton  with input alphabet 
(corresponding to the truth value of  and  at each position), and
output alphabet .  On an input word of the form
 for some word , we want the output
to be .
Let , and .  We have five different situations.  For
each of these cases the figure shows an example, with ``''
representing the cut , and each  representing a position in the input
word.

\begin{enumerate}
\setcounter{enumi}{-1}
\item  is followed by a position where  and  are true.\hfill
\item , and  is such that  is false and  is
true.
\hfill


\item other cases where  is true at .
\hfill


\item  is followed by a position where both  and  are false.
\hfill


\item other cases where  is false at .  If 
then the input at position  is .


\end{enumerate}

The structure of the automaton  and the limit transitions are
given by Figure~\ref{fig:until}.  This automaton has five states  to
 corresponding to the situations described above.  Given any two states
 and  there exists a transition  except from 
to  or  and from  to ,  or .  The input label of
successor transitions is determined by the origin node:  for ,
 for ,  for , and  for  and .  The
output label is  on transitions leading to ,  or , and 
on transitions leading to  or .  All states are initial, while  is the
only final state.




\begin{figure}
\begin{tabular}{m{9cm}m{5cm}}{
\setlength{\unitlength}{.3ex}
\begin{picture}(170,120)(0,-150)\nullfont
\node[NLangle=0.0,Nmarks=i](A)(40.0,-52.0){}
\node[NLangle=0.0,Nmarks=i,iangle=0](B)(116.0,-52.0){}
\node[NLangle=0.0,Nmarks=i,iangle=0](C)(148.0,-96.0){}
\node[NLangle=0.0,Nmarks=i,iangle=200](D)(80.0,-122.0){}
\node[NLangle=0.0,Nmarks=if,fangle=180,iangle=135](E)(20.0,-96.0){}
\drawloop(A){}
\drawloop(B){}
\drawloop[loopangle=270](C){}
\drawloop[loopangle=270](D){}
\drawloop[loopangle=270](E){}
\drawedge[curvedepth=6.0](A,B){}
\drawedge[curvedepth=6.0](A,C){}\drawedge[curvedepth=6.0](A,D){}\drawedge[ELside=r](A,E){}
\drawedge(B,E){}\drawedge[ELside=r](D,C){}
\drawedge[curvedepth=2.0](B,A){}\drawedge[curvedepth=6.0](B,C){}
\drawedge[curvedepth=6.0](B,D){}\drawedge[curvedepth=6.0](C,A){}\drawedge[curvedepth=4.0](C,B){}\drawedge[curvedepth=6.0](D,A){}\drawedge[curvedepth=6.0](D,B){}\drawedge[curvedepth=6.0](D,E){}
\drawedge[curvedepth=4.0](E,D){}\end{picture}
}&
\begin{minipage}{4cm}
{

 }
\end{minipage}
\end{tabular}
\caption{Automaton for }
\label{fig:until}
\end{figure}
 
\begin{lemma}
Let  and  two formulae.  Let  and  be the truth words of
 and  on a word  of length .  The output of 
on  is the truth word of  on .
\end{lemma}

\begin{proof}
Let  be the word of length  on  defined by
\begin{itemize}
\item if , then ;
\item if  and  then ;
\item if  then ;
\item otherwise, if there exists  such that  and for all
 such that , , then ;
\item otherwise, .
\end{itemize}
We show that  is a run of , that it is unique, and
that its output is indeed the truth word of  on .


By definition,  ends in , which is the final state of
.  Let .  If  is ,  
or , then  for some  and the successor transition from 
to the next cut is allowed by the automaton.  If , and  for some , then  and , and  is ,
 or .  If , then similarly  and , and  can be  or .  Every successor transition in
 is thus allowed by .

We now need to show the same for limit transitions.
If a left limit transition leads to a cut , then either  is true
arbitrarily close to the left of  (in which case the corresponding limit
set contains  or ), or it is always false (and the limit set is
 or a subset of ).  If the limit set contains ,
 or , any state for  is allowed.  If it is , the cut 
can't be labelled by  or  without violating the definition of
.  Conversely, if the limit set is ,  is necessarily
 or .

Let's now consider a right limit transition starting at a cut .  The label
of this cut can only be  or .  In the first case,  must
be true everywhere in the limit set, which is thus a subset of .
In the second case, either  is false infinitely often in the limit,
or  is always false.  This means that the limit set contains  or
, or is restricted to .

We now show that a run on  is uniquely determined by the input
word.
Let  a run of  on .  Because of the constraints on
the successor transitions, a cut  is labelled by ,  or  in
 if and only if it is labelled by the same state in .

Let's suppose that a cut  is labelled by  in .  Since  is
not final, there exists  labelled by some other state.  If there is a
first such cut, its label is necessarily  or  (by a successor
transition from  or a limit transition from ).  Otherwise, there
is a transition of the form  or .  In both cases,  satisfies the condition for cuts labelled by
 in the definition of .
A similar argument shows that a cut labelled by  in  has the same
label in .  The run of  on a given input word is thus
unique.

The last step is to show that the output word is really the truth word of
.
Let  an element of .
First, suppose that .  If  has a successor ,
and  is true at , then , and  outputs  at
position .  Otherwise, there exists  such that  (i.e.
), and  whenever .  Thus,  is labelled
with , and  once again outputs  at position .


Similarly, if , there are two cases.
In the first case,  has a successor , and .  This means
that  is labelled by , so the output at position  is .
In the last case,  is labelled by , and once again 
outputs .
\qed\end{proof}

\subsection{Automaton for the future Stavi connective ()}

Let's recall that  holds at position  if there exists a
gap  such that  holds at every position , the property
 holds at every position in some interval starting at , and
 holds at positions arbitrarily close to  to the right.

The central point in this definition is the gap , which corresponds to
state  in the automaton.  States ,  and  follow
the positions, before , where the formula holds.  States ,
, , ,  follow the positions where the formula
doesn't hold.
If a run reaches ,  or , it has to leave this region
through , and all successor transitions until then have input label
 or .  The structure of this automaton is depicted in
Figure~\ref{fig:stavi}.
All states except  and  are initial;  and 
are final.  Transitions from  and  have input label ,
transitions from  and  have input label , transitions from
 have input label , and transitions from  have input
label .  The output is  for transitions to ,  and
, and  for transitions to , , ,  and
.

\begin{figure}
\begin{tabular}{ll}{


\setlength{\unitlength}{.3ex}

\begin{picture}(150,70)(10,-100)\nullfont
\node[NLangle=0.0](q2)(40.0,-32.0){}

\node[NLangle=0.0](q1)(88.0,-32.0){}

\node[NLangle=0.0](q0)(64.0,-52.0){}

\node[NLangle=0.0](qhole)(144.0,-32.0){}

\node[NLangle=0.0](q7)(44.0,-156.0){}

\node[NLangle=0.0](q6)(108.0,-156.0){}

\node[NLangle=0.0](q5)(44.0,-100.0){}

\node[NLangle=0.0](q4)(108.0,-100.0){}

\node[NLangle=0.0,Nmarks=f](q8)(144.0,-104.0){}

\node[NLangle=0.0,Nmarks=f](q9)(144.0,-64.0){}

\drawedge[curvedepth=6.0](q5,q7){}

\drawedge[curvedepth=6.0](q7,q5){}

\drawedge[curvedepth=4.0](q4,q5){}

\drawedge[curvedepth=4.0](q5,q4){}

\drawedge[curvedepth=4.0](q4,q6){}

\drawedge[curvedepth=4.0](q6,q4){}

\drawedge[curvedepth=4.0](q6,q7){}

\drawedge[curvedepth=4.0](q7,q6){}

\drawedge[curvedepth=4.0](q5,q6){}

\drawedge[curvedepth=4.0](q6,q5){}

\drawedge[curvedepth=8.0](q7,q4){}

\drawedge[curvedepth=8.0](q4,q7){}

\drawedge[curvedepth=8.0](q2,q1){}

\drawedge[curvedepth=8.0](q1,q2){}

\drawedge[curvedepth=-4.0](q2,q0){}

\drawedge[curvedepth=8.0](q1,q0){}

\drawloop[loopangle=-180.0,ELpos=70](q7){1,1/0}

\drawloop[loopangle=180.0,ELpos=30](q5){0,1/0}

\drawloop[loopangle=-0.0](q6){1,0/0}

\drawloop[loopangle=45,ELpos=60](q4){0,0/0}

\drawloop[loopangle=0.0](q1){1,0/1}

\drawloop[loopangle=180.0,ELpos=30](q2){1,1/1}

\drawedge[curvedepth=8.0](q4,q0){}

\drawedge[curvedepth=8.0](q5,q0){}

\drawedge[curvedepth=-4.0](q6,q8){}

\drawedge[curvedepth=-4.0](q4,q8){}

\drawedge[curvedepth=-12.0](q5,q8){}

\drawedge[curvedepth=-8.0](q7,q8){}

\drawedge[curvedepth=8.0](q5,q2){}

\drawedge[curvedepth=-12.0](q5,q1){}

\drawedge[curvedepth=-8.0](q4,q1){}

\drawedge[curvedepth=15.0](q4,q2){}









\end{picture}


 }&
\begin{minipage}{6cm}{
\begin{itemize}
\item  if  or 
\item  if 
\item  if 
\item  if 
\item  if  and

\item  if 
\item  if  and 
\item  if 
\item  if  and either  or  intersects

\end{itemize}
}\end{minipage}
\end{tabular}
\caption{Automaton for the future Stavi operator}
\label{fig:stavi}
\end{figure}

We define a labelling  of the cuts of a word  on  using the
states of the automaton in the following way:
\begin{itemize}
\item  has no successor,  is true
\item  has an outgoing transition labelled ,  is
true
\item  has an outgoing transition labelled ,  is
true
\item  is a gap,  is true before it and false
afterwards
\item  has an outgoing transition labelled ,  is
false
\item  has an outgoing transition labelled ,  is
false
\item  has an outgoing transition labelled ,  is
false
\item  has an outgoing transition labelled , 
is false
\item  has no successor,  doesn't hold in the left limit if it
has no predecessor, and  is false
\item  is a gap or is the last cut,  is false, and
 is true in some interval to the left
\end{itemize}



\begin{lemma}
 defines the unique run of the automaton on its input word.  If the
input is  for some word , then the
output of this run is .
\end{lemma}
\begin{proof}
We first show that  is a run.  Successor transitions correspond almost
directly to the definitions of the labelling , so let's look at limit
transitions.  For left limits, the following cases need to be considered:
\begin{itemize}
\item if a transition  is taken at a cut , then either
 is true in the limit, and so  is too, and , or it's not, and either  or  appear
in the limit
\item the same reasoning applies for  and 
\item if  is labelled  then the incoming transition has to come
from a subset of  since  is true in
the limit.
\item if a transition  is used, then  is
not true in the limit (otherwise it would still be true), and so
; the same applies for ,
, ,  and 
\item if  is a left limit and is labelled  then the incoming
transition comes from a set  intersecting  because
 is repeated
\item if  is labelled  then  and  can't appear in
the left limit set ( is true)
\end{itemize}
If  is a right limit cut, it can only be labelled 
, ,  or .
Here are the possible right-limit transitions:
\begin{itemize}
\item if a right-limit cut  is labelled , the
limit transition has to go to a subset of  since
 holds in the limit
\item if  is labelled with , the limit transition to its right
leads necessarily to a set  not including ,  and 
since  is always true, and including  because  is
repeated
\item if  is labelled  or , the right limit set can't be a
subset of  otherwise  would have been labelled

\item if  is labelled  we have the additional condition that
either  holds in the limit (and neither  nor  appears)
or  doesn't (and one of ,  and  is in the limit)
\end{itemize}

The labelling of cuts defined above is thus a path of the automaton, and we
only need to show that it's the only one, using the same method as for the
.  Moreover, the definition of  means that the output
is  whenever  holds, and  at all other positions.\qed\end{proof}

\subsection{Construction of }

Now that we have the basic blocks for our construction, we can build an
automaton for any formula .  If  is an atomic proposition
, then we have seen how to build  in the previous section.  If
, then .
If , then .  If , then
.
The same can be done for  and for the past connectives.

The number of states of the resulting automaton is the product of the number
of states of all the elementary automata, and is thus exponential in the size
of the formula.  The actual size of the automaton includes limit transitions,
so can be doubly exponential in the size of the formula, if those transitions
are represented explicitly.

To check whether the formula  is satisfiable by a model which is
recognized by an automaton , we can compute the product of the automaton
 with , and check whether a transition where 
outputs  is accessible and co-accessible.  This ensures that there exists a
successful run of the product automaton going through that transition, meaning
that the corresponding input word is accepted by  and there is a position
where  holds.  This concludes the proof of Corollary~\ref{cor:sat}.



\section{Discussion}
\label{s:discuss}

\noindent \textbf{Logical characterization of automata.} We have shown that
any LTL, and thus , formula can be represented as a non-ambiguous
automaton with output.  But one can also build such an automaton where the
output is the truth word of a property which can't be expressed as a
first-order formula.  The automaton shown on Figure~\ref{fig:gap} outputs 1
whenever ``there is a gap somewhere in the future'' is true; that formula
can't be expressed in .  It would be interesting to find a logical
characterisation of the properties that can be expressed using such automata.

\begin{figure}
\subfigure{
\setlength{\unitlength}{.5ex}
\begin{picture}(80,65)(-10,-65)
\node[Nmarks=if,iangle=165,fangle=195](i)(0,-30){}
\node[Nmarks=f](f)(80,-30){}
\node(a)(15,-15){}
\node(b)(40,-15){}
\node(c)(65,-15){}
\node(h)(65,0){}
\node(a')(15,-45){}
\node(b')(40,-45){}
\node(c')(65,-45){}

\drawloop[loopdiam=6](b){1}
\drawloop[loopdiam=6,loopangle=270](b'){0}
\drawedge[curvedepth=15](i,b){1}
\drawbpedge(i,90,50,c,135,30){1}
\drawedge(a,b){1}
\drawedge[curvedepth=-10](a,c){1}
\drawedge(b,c){1}
\drawedge[curvedepth=-15,ELside=r](i,b'){0}
\drawbpedge(i,270,50,c',225,30){0}
\drawedge(a',b'){0}
\drawedge[curvedepth=10](a',c'){0}
\drawedge(b',c'){0}

\drawedge[curvedepth=10,ELpos=80](a',f){0}
\drawedge[curvedepth=5](b',f){0}
\end{picture}
}
\subfigure{
\begin{picture}(40,50)(0,-50)
\put(20,0){ Limit transitions:}
\put(20,-10){ if }
\put(20,-20){ if }
\put(20,-30){ for any }
\put(20,-40){ if }

\end{picture}
}
\caption{Automaton checking whether a gap exists in the future}
\label{fig:gap}
\end{figure}

\noindent \textbf{Computational complexity.}  The exact complexity of the
satisfiability problem for LTL on arbitrary orderings remains open.  We give a
2\expspace procedure to compute an automaton from a formula, whose emptiness
can then be checked efficiently.  A classical optimization in similar problems
is to compute the automaton on the fly, which saves a lot of complexity, so an
algorithm using this technique for LTL on arbitrary orderings would be
interesting.

\noindent \textbf{Expressive power.}  On finite and -words, LTL
restricted to the unary operators (, , and their past
counterparts) is equivalent to first-order logic restricted to two variables,
~\cite{DBLP:journals/iandc/EtessamiVW02}.  Restricting even
further to  and its reverse, we get a logic expressively
equivalent to .  In the case of finite words,  corresponds
to ``partially ordered'' two-way automata~\cite{SchThVo2002}.  The proof of
equivalence between unary temporal logic and  can be easily extended to
the case of arbitrary linear orderings.  It would be interesting to find such
a correspondence for arbitrary orderings as well, and to see if these
restrictions provide lower complexity results.

\noindent \textbf{Mosaics technique.}  In his work on LTL(), Reynolds
uses ``mosaics'' to keep track of the subformulas that need to be satisfied in
particular intervals, and to find a decomposition that shows the
satisfiability of the initial formula.  Unfortunately it is not clear if and
how this can be extended to handle a larger fragment of the logic.

\section{Conclusion}

We investigate linear temporal order with \until, \since, and the Stavi
connectives over general linear time, and its relationship with automata over
linear orderings.  We provide a translation from LTL to a class of
non-ambiguous automata with output, giving a 2\expspace procedure to decide
satisfiability of a formula in any rational subclass.

This leaves a number of immediate questions, starting with the actual
complexity for the satisfiability problem for LTL, but also for some of its
fragments, where some operators are excluded.  While the full class of
automata over linear orderings is not closed under
complementation~\cite{DBLP:conf/csr/BedonBCR08}, it might still be possible to
find a logical characterization for some interesting subclasses.

\bibliographystyle{plain}
{\small\bibliography{bib}}
\end{document}
