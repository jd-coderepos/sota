\documentclass{article}


\usepackage{hyperref}
\usepackage{amsmath,amssymb,amsfonts,amsthm}
\usepackage{tablefootnote}
\usepackage{xspace}
\usepackage{lettrine,color}
\usepackage{graphicx}
\usepackage{subfigure}
\usepackage{latexsym}
\usepackage{paralist}
\newcommand{\prob}[1]{\mathbf{Pr}[#1]}
\newcommand{\Prob}[1]{\mathbf{Pr}\bigg[#1\bigg]}
\usepackage{algorithm}
\usepackage{algorithmicx}
\usepackage{algpseudocode}
\usepackage[T1]{fontenc}
\usepackage{lmodern}
\usepackage{thmtools}
\renewcommand{\O }{\ensuremath{{\cal O}}}

\newtheorem{theorem}{Theorem}
\newtheorem{corollary}[theorem]{Corollary}
\newtheorem{lemma}[theorem]{Lemma}
\newtheorem{proposition}[theorem]{Proposition}
\newtheorem{definition}[theorem]{Definition}
\newtheorem{claim}[theorem]{Claim}

\newcommand{\NATURAL}{\ensuremath{\mathbb{N}}}  \newcommand{\nat}{\ensuremath{\mathbb{N}}} 
\newcommand{\net}{\ensuremath{\mathfrak{N}}\xspace}

\newcommand{\Exp}[1]{\mathbf{E}\left[#1\right]}
\newcommand{\maxdeg}{\ensuremath{\Delta}} 
\newcommand{\ecc}{\ensuremath{D}\xspace}
\newcommand{\comment}[1]{\text{\phantom{(#1)}} \tag{#1}
}
\newcommand{\cj}{\ensuremath{c}\xspace}
\newcommand{\ci}{\ensuremath{c_1}\xspace}
\newcommand{\cii}{\ensuremath{c_2}\xspace}
\newcommand{\ciii}{\ensuremath{c_3}\xspace}
\newcommand{\civ}{\ensuremath{c_4}\xspace}
\newcommand{\cv}{\ensuremath{c_5}\xspace}

\newcommand{\Proc}{Proceedings of the\xspace}
\newcommand{\STOC}{Annual ACM Symposium on Theory of Computing (STOC)}
\newcommand{\FOCS}{IEEE Symposium on Foundations of Computer Science (FOCS)}
\newcommand{\SODA}{Annual ACM-SIAM Symposium on Discrete Algorithms (SODA)}

\newcommand{\COCOON}{Annual International Computing Combinatorics Conference (COCOON)}
\newcommand{\DISC}{International Symposium on Distributed Computing (DISC)}
\newcommand{\ESA}{Annual European Symposium on Algorithms (ESA)}
\newcommand{\ICALP}{Annual International Colloquium on Automata, Languages and Programming (ICALP)}
\newcommand{\IPL}{Information Processing Letters}
\newcommand{\JACM}{Journal of the ACM}
\newcommand{\JALGORITHMS}{Journal of Algorithms}
\newcommand{\JCSS}{Journal of Computer and System Sciences}
\newcommand{\PODC}{Annual ACM Symposium on Principles of Distributed Computing (PODC)}
\newcommand{\SICOMP}{SIAM Journal on Computing}
\newcommand{\STACS}{Annual Symposium on Theoretical Aspects of Computer Science (STACS)}
\newcommand{\TALG}{ACM Transactions on Algorithms}
\newcommand{\TCS}{Theoretical Computer Science}


\author{\textbf{Artur Czumaj} \hspace{4mm} \textbf{Peter Davies} \
    \sum_{i=1}^{\ecc} h'(\frac{n}{\ecc})
        =
    \sum_{i=1}^{\ecc} \mathcal{B}\frac{\frac{n}{\ecc} +2r}{r}
        =
    3\mathcal{B}\ecc
        =
    3c'n\log \ecc \log\log\frac{\ecc\maxdeg}{n}

    \sum_{i=1}^{\min(\ecc,\frac n\maxdeg)} g(\maxdeg)
        =
    \sum_{i=1}^{\min(\ecc,\frac n\maxdeg)} \frac{c\maxdeg \log \maxdeg \log n}{\log\log \maxdeg}
        =
    \frac{c\min(n,\ecc \maxdeg) \log n \log \maxdeg}{\log\log \maxdeg}

    \{(v,\psi(v)): \omega(v)< j'\}

	\frac{c\log n}{6(j+c\log n)} \cdot |C(j)|
		\ge
	\frac{c\log n}{6(j+c\log n)}
		>
	\frac{1}{12}
		\ge
	\frac{\log\log{|C|}}{12\log{|C|}}

	j + \omega(X)
		< g(|C|) +\omega(X)
	= g(|X_{j'}|) +\omega(X)
		\le
    j'

	j
		=
	(j+\omega(X))-\omega(X)
		<
	g(|X_{j+\omega(X)}|)
		= g(|C(j)|)

    f_C(j)
        \ge
    \frac{c\log n}{6(j+c\log n)} \cdot |C(j)|
        >
    \frac{j\log\log |C(j)|}{6(j+c\log n)\log |C(j)|}
        \ge
    \frac{\log\log{|C|}}{12\log{|C|}}

    \sum_{j <g(|C|)} f_C(j)
        &=
    \sum_{j <g(|C|)} \sum_{(v,\psi(v))\in C(j)} \frac{c\log n}{6(j - \psi(v)+c\log n)}
        \displaybreak[2]\\
        &=
    \sum_{(v,\psi(v)) \in C} \sum_{j <g(|C|)} \frac{c\log n}{6(j - \psi(v)+c\log n)}
        \\
        &\le
    \sum_{(v,\psi(v)) \in C}\int_{\psi(v)-1}^{g(|C|)-1}\frac{c\log n}{6(j - \psi(v)+c\log n)} dj
    	   \comment{\footnotesize\sf by standard integral bound}
        \\
        &=
    \frac{c\log n}{6}\sum_{(v,\psi(v)) \in C}\ln\left(\frac{g(|C|) - 1 - \psi(v)+c\log n}{c\log n-1}\right)\comment{\footnotesize\sf evaluating integral}
        \displaybreak[2]\\
        &\le
    \frac{c \log n \cdot |C|}{6} \cdot \ln\left(\frac{g(|C|)+c\log n-1}{c\log n-1}\right)
        \\
        &=
    \frac{c |C| \log n}{6} \cdot \ln\left(\frac{\frac {c |C| \log n \log |C|}{\log\log |C|}+c\log n-1}{c\log n-1}\right)\comment{\footnotesize\sf substituting 's definition}
		\displaybreak[2]\\
        &\le
    \frac{c |C| \log n}{6} \cdot \ln\left(\frac{\frac {c |C| \log n \log |C|}{\log\log |C|}}{\frac12 c\log n} + 1\right)
        \displaybreak[2]\\
       	&\le
    \frac{c |C| \log n}{6} \cdot \ln(4|C|^{1.1})
        \displaybreak[2]\\
		&=\frac{1.1\ln 2 \log |C|+ \ln 4}{6} c |C| \log n
		\\
        &\le
    0.45 c |C| \log n \log |C|

    |\mathcal{F}_C|
        &\ge
    g(|C|) -\frac{0.9c|C| \log n \log  |C|}{\log\log |C|}
        = 
    \frac {c |C| \log n \log |C|}{10\log\log |C|}

	\Pr{\sum_{i\in [n]} x_i = 1} &= \sum_{j\in [n]} \Pr{x_j = 1 \land x_i = 0 \forall i \neq j}\\
	&\geq \sum_{j\in [n]} \Pr{x_j = 1}\cdot \Pr{x_i = 0 \forall i}\\
	&\geq f\cdot \Pr{x_i = 0 \forall i}\\
	&= f\cdot \prod_{i\in [n]} (1- \Pr{x_i = 1})\\
	&\geq f\cdot \prod_{i\in [n]} 4^{-\Pr{x_i = 1}}\\
	&= f\cdot 4^{-\sum_{i\in [n]} \Pr{x_i = 1}}\\
	&=f4^{-f}
	
&\Pr{\text{no }\text{column hits}}
        \le
    \prod_{j < g(|C|)} (1-f_C(j) \cdot 4^{-f_C(j)})
        \\
        &\hspace{1cm}\le
    \prod_{j \in \mathcal{F}_C} (1-f_C(j) \cdot 4^{-f_C(j)})
        \displaybreak[2]\\
        &\hspace{1cm}=
    \prod_{\substack{j\in\mathcal{F}_C,\\\mu < f_C(j) \le \frac 12\log\log |C|}} (1-f_C(j) 4^{-f_C(j)})
    \prod_{\substack{j\in\mathcal{F}_C,\\\frac{\log\log{|C|}}{12\log{|C|}} < f_C(j) \le \mu}} (1-f_C(j) \cdot 4^{-f_C(j)})\hspace{1in}
        \\
        &\hspace{1cm}\le
    \prod_{\substack{j\in\mathcal{F}_C,\\\mu < f_C(j) \le \frac 12\log\log |C|}} \left(1-\frac{\log\log |C|}{2\log |C|}\right)
    \prod_{\substack{j\in\mathcal{F}_C,\\\frac{\log\log{|C|}}{12\log{|C|}} < f_C(j) \le \mu}} \left( 1-\frac{\log\log{|C|}}{14\log{|C|}}\right)
    	\comment{\footnotesize\sf since products are maximised by setting  and , respectively}
        \displaybreak[2]\\
        &\hspace{1cm}\le
    \left(1-\frac{\log\log |C|}{2\log |C|}\right)^h \cdot \left( 1-\frac{\log\log{|C|}}{14\log{|C|}}\right)^{|\mathcal{F}_C|-h}
        \displaybreak[2]\\
        &\hspace{1cm}\le
    \left( 1-\frac{\log\log{|C|}}{14\log{|C|}}\right)^{|\mathcal{F}_C|}
        \\
        &\hspace{1cm}\le
    \left(1-\frac{\log\log |C|}{14\log |C|}\right)^{\frac {c |C| \log n \log |C|}{10\log\log |C|}}
        \comment{\footnotesize\sf by Lemma \ref{lemma:bound-for-FC}}
        \\
        &\hspace{1cm}\le
    e^{\frac{-c |C| \log n}{140}}
    		\comment{\footnotesize\sf using  for }
        \\
        &\hspace{1cm}=
    n^{\frac{-c |C|}{140 \ln 2}}

    &\Pr{\mathcal{S}\text{ is an -universal synchronizer}}
        \le
    \sum_{q=1}^{n}\sum_{C \in C_q} \Pr{\text{C is not hit}}\\
        &\hspace{1cm}\le
    \sum_{q=1}^{n}\sum_{C \in C_q} n^{\frac{- c |C|}{140 \ln 2}}
        \le
    \sum_{q=1}^{n} n^{3q} \cdot n^{\frac{- c q}{140 \ln 2}}
        =
    \sum_{q=1}^{n} n^{(3-\frac{c}{140 \ln 2}) q}\\
        &\hspace{1cm}\le
    \sum_{q=1}^{n} n^{-2q}
        <
    1\enspace.

    \{(v,\phi(v)): v \in X, s(v)< j'\}

    f_\mathcal{C}(j)
        \ge
    |\mathcal{C}(j)| \cdot \frac{c\log D \log\log \frac{D\maxdeg}{n}}{2(j + B)}
        >
    \frac{cj}{2c(j+B)}
        \ge
    \frac16
\allowdisplaybreaks
    \sum_{\substack{j < \frac{B \cdot |\mathcal{C}|}{r}\\ \rho(j)=0}} f_\mathcal{C}(j)&
        =
    \sum_{\substack{j <\frac {B \cdot |\mathcal{C}|}{r}\\ \rho(j)=0}} \sum_{(v,\phi(v))\in \mathcal{C}(j)}\frac{c\log D \log\log \frac{D\maxdeg}{n}}{2(j-B\phi(v) + B)}
        \displaybreak[2]\\
        &=
    \sum_{(v,\phi(v))\in \mathcal{C}}\sum_{\substack{B\phi(v)\le j <\frac {B \cdot |\mathcal{C}|}{r}\\ \rho(j) = 0}}\frac{c\log D \log\log \frac{D\maxdeg}{n}}{2(j-B\phi(v) + B)}
        \displaybreak[2]\\
		&=
    \sum_{(v,\phi(v))\in \mathcal{C}}\sum\limits_{i=\frac{B\phi(v)}{2\log\log\frac{D\maxdeg}{n}}}^{\frac {B \cdot |\mathcal{C}|}{2r\log\log\frac{D\maxdeg}{n}}-1}\frac{c\log D \log\log \frac{D\maxdeg}{n}}{2(2i\log\log \frac{D\maxdeg}{n}-B\phi(v) + B)}
	    \comment{\footnotesize\sf substitution of sum index variable}
        \displaybreak[2]\\
        &\le
    \sum_{(v,\phi(v))\in \mathcal{C}}\int_{\frac{B\phi(v)}{2\log\log\frac{D\maxdeg}{n}}-1}^{\frac {B \cdot |\mathcal{C}|}{2r\log\log\frac{D\maxdeg}{n}}-1}\frac{c\log D \log\log \frac{D\maxdeg}{n}}{2(2i\log\log \frac{D\maxdeg}{n}-B\phi(v) + B)} di
    		\comment{\footnotesize\sf using standard integral bound}
        \displaybreak[2]\\
        &=
    \frac{c\log D}{4} \sum_{(v,\phi(v))\in \mathcal{C}}\ln \left(\frac{\frac {B \cdot |\mathcal{C}|}{r}-2\log\log\frac{D\maxdeg}{n}-B\phi(v) + B}{B-2\log\log\frac{D\maxdeg}{n}}\right)
    		\comment{\footnotesize\sf evaluating integral}
        \displaybreak[2]\\
        &\le
    \frac{c|\mathcal{C}|\log D}{4} \ln \left(\frac{\frac {B \cdot |\mathcal{C}|}{r}-2\log\log\frac{D\maxdeg}{n} + B}{B-2\log\log\frac{D\maxdeg}{n}}\right)
        \displaybreak[2]\\
        &=
    \frac{c|\mathcal{C}|\log D}{4}
    \ln \left(
        \frac{|\mathcal{C}| c \log D \log\log\frac{D\maxdeg}{n} - 2\log\log\frac{D\maxdeg}{n} + B}
        {B-2\log\log\frac{D\maxdeg}{n}}
    \right)
        \displaybreak[2]\\
        &\le
    \frac{c|\mathcal{C}|\log D}{4} \ln \left(\frac{2(c|\mathcal{C}|\log D \log\log \frac{D\maxdeg}{n} +B)}{B}\right)
        \displaybreak[2]\\
        &=
    \frac{c|\mathcal{C}|\log D}{4} \ln \left(\frac{2(|\mathcal{C}| +\frac nD)}{\frac nD}\right)
        \\
        &\le
    \frac14 c|\mathcal{C}|\log D \ln \frac{4|\mathcal{C}|D}{n}
            \comment{\footnotesize\sf using the assumption }
        \displaybreak[2]\\
        &\le
    \frac14 c|\mathcal{C}| \log D \log \frac{|\mathcal{C}|D}{n}

    |\mathcal{F_C}|
        &\ge
    \frac{B \cdot |\mathcal{C}|}{2 r \log\log\frac{D\maxdeg}{n}}
        -
    \frac{B}{4\log\log\frac{D\maxdeg}{n}} - \frac{1}{12} c |\mathcal{C}| \log D\\
        &=
    \frac{c}{2} \log D
            \left(|\mathcal{C}|-\frac{n}{2D}-\frac{|\mathcal{C}|}{6}\right)
        \ge
    \frac{c}{6} |\mathcal{C}| \log D

    j'
        <
    j + \log (3\log\frac{|\mathcal{C}|D}{n}) - 1
        <
    j+2\log\log\frac{D \maxdeg}{n}

    f_\mathcal{C}(j')
        &=
    \sum_{(v,\phi(v))\in \mathcal{C}(j')}\frac{c\log D \log\log \frac{D\maxdeg}{n}}{(j'-B\phi(v) + B)2^{\rho(j')+1}}
        \\
        &=
    \sum_{(v,\phi(v))\in \mathcal{C}(j)}\frac{c\log D \log\log \frac{D\maxdeg}{n}}{(j'-B\phi(v) + B)2^{\rho(j)+\log f_\mathcal{C}(j)}}
        \\
        &=
    \sum_{(v,\phi(v))\in \mathcal{C}(j)}\frac{c\log D \log\log \frac{D\maxdeg}{n}}{(j'-B\phi(v) + B)f_\mathcal{C}(j)}
        \\
        &=
    \frac{2}{f_\mathcal{C}(j)}\sum_{(v,\phi(v))\in \mathcal{C}(j)}\frac{c\log D \log\log \frac{D\maxdeg}{n}}{2(j-B\phi(v) + B)}\cdot \frac{(j-B\phi(v) + B)}{(j'-B\phi(v) + B)}

    f_\mathcal{C}(j')
        \le
    \frac{2}{f_\mathcal{C}(j)}\sum_{(v,\phi(v))\in \mathcal{C}(j)}\frac{c\log D \log\log \frac{D\maxdeg}{n}}{2(j-B\phi(v) + B)}\cdot 1
        = 2

   	f_\mathcal{C}(j')
		>
	\frac{2}{f_\mathcal{C}(j)}\sum_{(v,\phi(v))\in \mathcal{C}(j)}\frac{c\log D \log\log \frac{D\maxdeg}{n}}{2(j-B\phi(v) + B)}\cdot \frac 13
		=
    \frac23

    \Pr{\text{no column hits}}
        &\le
    \prod_{j<\frac {B \cdot |\mathcal{C}|}{r}} (1-f_\mathcal{C}(j) \cdot 4^{-f_\mathcal{C}(j)})
        \le
    \prod_{j \in \mathbb{F}_\mathcal{C}} (1-f_\mathcal{C}(j) \cdot 4^{-f_\mathcal{C}(j)})
        \\
        &\le
    \prod_{j \in \mathbb{F}_\mathcal{C}} \frac 78
        \le
    \left(\frac 78\right)^{\frac {c}{12} |\mathcal{C}|\log D }
        =
    D^{-\frac{c}{12} |\mathcal{C}|\log \frac 78 }
        \le
    D^{-\frac{c \cdot |\mathcal{C}|}{63}}

    &\Pr{S \text{ is not an upper block synchronizer}}
        \le
    \sum_{q=\frac nD}^{\maxdeg}\sum_{C \in C_q} \Pr{\text{C is not hit}}\\
        &\hspace{1cm}\le
    \sum_{q=\frac nD}^{\maxdeg}\sum_{C \in C_q} D^{-cq/63}
        \le
    \sum_{q=\frac nD}^{\maxdeg} D^{2q} D^{-cq/63}
        =
    \sum_{q=\frac nD}^{\maxdeg} D^{2q} D^{-3q}\\
        &\hspace{1cm}=
    \sum_{q=\frac nD}^{\maxdeg} D^{- q}
        <
    \frac2D
        <
    1

    \mathcal{S}
        =
    \{\mathcal{S}^v\}_{v \in [n]} \text{ is defined by } \mathcal{S}^v_j
        =
    \begin{cases}
        R^v_{j \bmod \mathcal{B}} &\text{if } (j \bmod \mathcal{B}) < |R|,\\
        S^v_{j - \lceil\frac{j}{\mathcal{B}}\rceil R} &\text {otherwise.}
    \end{cases}


Setting , we show that  satisfies the conditions of an -block synchronizer.

Let  be a core of size at most .

\begin{description}
\item[Case 1: .]

 we have , since the core ends before column  by Definition \ref{def:bcore}, and so  will be hit by the -selective family . It will therefore be hit by  on some column . Note that this case is the reason we require the ceiling function in the definition of a block synchronizer, but not in an upper block synchronizer.

\item[Case 2: .]
If , then it will be hit by a column  in the upper block synchronizer , which corresponds to the column  in . Since , this satisfies the block synchronizer property.
\end{description}

So,  hits all cores  with  within  columns, and therefore hits all sets  within  under any activation schedule, fulfilling the criteria of an -block synchronizer.
\end{proof}


\section{Conclusions}

The task of broadcasting in radio networks is a longstanding, fundamental problem in communication networks. Our result for deterministic broadcasting in directed networks combines elements from several of the previous works with some new techniques, and, in doing so, makes a significant improvement to the fastest known running time. Our algorithm for wake-up also improves over the previous best running time, in both directed and undirected networks, and relies on a proof of smaller universal synchronizers, a combinatorial object first defined in \cite{-CK04}.

Neither of these algorithms are known to be optimal. The best known lower bound for both broadcasting and wake-up is  \cite{-CMS03}; our broadcasting algorithm therefore comes within a log-logarithmic factor, but our wake-up algorithm remains a logarithmic factor away.

As well as the obvious problems of closing these gaps, there are several other open questions regarding deterministic broadcasting in radio networks. Firstly, the lower bound for undirected networks is weaker than that for directed networks \cite{-KP03b}, and so one avenue of research would be to find an  lower bound in undirected networks, matching the broadcasting time of \cite{-K05}. Secondly, the algorithms given here, along with almost all previous work, are non-explicit, and therefore it remains an important challenge to develop explicit algorithms that can come close to the existential upper bound. The best constructive algorithm known to date is by \cite{-I02}, but it is a long way from optimality.

Some variants of the model also merit interest, in particular the model with collision detection. It is unknown whether the capacity for collision detection improves deterministic broadcast time, as it does for randomized algorithms \cite{-GHK13}. Collision detection does remove the requirement of spontaneous transmissions for the use of the  algorithm of \cite{-CGGPR00}, but a synchronized global clock would still be required. It should be noted that collision detection renders the wake-up problem trivial, since if every active node transmits in every time-step, collisions will wake up the entire network within  time-steps.

\bibliographystyle{plain}
\bibliography{BC}
\end{document}