\documentclass{article}
\usepackage{hyperref}

\usepackage{graphicx,amssymb,amsmath}
\usepackage{amsfonts,amsthm}
\usepackage{color}
\usepackage{subfigure}
\usepackage{bm}
\usepackage{a4wide}




\graphicspath{{figs/}}

\newcommand{\Poly}{\ensuremath{\mathcal{P}}}
\newcommand{\marrow}{\marginpar[\hfill]{}}
\newcommand{\niceremark}[3]{\textcolor{red}{\textsc{#1 #2:} \marrow\textsf{#3}}}
\newcommand{\mati}[2][says]{\niceremark{Mati}{#1}{#2}}
\newcommand{\alex}[2][says]{\niceremark{Alex}{#1}{#2}}
\newcommand{\birgit}[2][says]{\niceremark{Birgit}{#1}{#2}}
\newcommand{\wolfgang}[2][says]{\niceremark{Wolfgang}{#1}{#2}}
\newcommand{\todo}[1]{\textcolor{red}{{#1}}}

\newcommand{\MinDT}{\ensuremath{\mathrm{minDT}}}
\newcommand{\MaxDT}{\ensuremath{\mathrm{maxDT}}}
\newcommand{\triang}{\ensuremath{T}}



\newtheorem{theorem}{Theorem}[section]
\newtheorem{prop}[theorem]{Proposition}
\newtheorem{cor}[theorem]{Corollary}
\newtheorem{obs}[theorem]{Observation}
\newtheorem{lemma}[theorem]{Lemma}
\newtheorem{conjecture}[theorem]{Conjecture}




\title{The Dual Diameter of Triangulations} 

\author{
Matias~Korman\thanks{National
Institute of Informatics (NII), Tokyo, Japan. {\tt korman@nii.ac.jp}}
\thanks{JST, ERATO, Kawarabayashi Large Graph Project.}
\and Stefan~Langerman\thanks{Universit\'e Libre de
Bruxelles (ULB), Brussels, Belgium.
{\tt stefan.langerman@ulb.ac.be}. Directeur de Recherches du FRS-FNRS.}
\and Wolfgang~Mulzer\thanks{Institut f\"ur Informatik, Freie
	Universit\"at Berlin,
 Germany. {\tt mulzer@inf.fu-berlin.de}}
\and Alexander~Pilz\thanks{Institute for Software Technology,
Graz University of Technology, Austria, {\tt [apilz|bvogt]@ist.tugraz.at}}
\and Maria~Saumell\thanks{Department of Mathematics and European Centre of Excellence NTIS (New Technologies for the Information Society), University of West Bohemia, Czech Republic, {\tt saumell@kma.zcu.cz}}
\and Birgit~Vogtenhuber\footnotemark[6]
}




\begin{document}
\maketitle

\begin{abstract}
Let  be a simple polygon with  vertices.
The \emph{dual graph}  of a triangulation~ of~
is the graph whose vertices correspond to the bounded faces of
 and whose edges connect those faces of~ that share an edge.
We consider triangulations of~ that minimize or maximize the 
diameter of their dual graph. 
We show that both triangulations can be constructed in  time
using dynamic programming.
If  is convex, we show that any minimizing triangulation has dual 
diameter exactly  or 
, depending on~. Trivially, in this
case any maximizing triangulation has dual diameter .
Furthermore, we investigate the relationship between the dual diameter 
and the number of \emph{ears} (triangles with exactly two edges incident 
to the boundary of ) in a triangulation. 
For convex , we show that there is always a triangulation that
simultaneously minimizes the dual diameter and maximizes the number of 
ears.
In contrast, we give examples of general simple polygons where every 
triangulation that maximizes the number of ears has dual diameter that 
is quadratic in the minimum possible value.
We also consider the case of point sets in general position in the plane. 
We show that for any such set of  points there are triangulations with 
dual diameter in~ and in~.
\end{abstract}







\section*{Foreword}
Research on this topic was initiated at the \emph{Brussels Spring 
Workshop on Discrete and Computational Geometry}, which took place 
May 20--24, 2013.  The authors would like to thank all the participants in general and  Ferran Hurtado in particular. Ferran participated in the early stages of the discussion, but modestly decided not to be an author of this paper. To us he has been a teacher, supervisor, advisor, mentor, colleague, coauthor, and above all: a friend. We are very grateful that he was a part of our lives.

\section{Introduction}
Let  be a simple polygon with  vertices. 
We regard  as a closed two-dimensional subset of the plane, 
containing its boundary. A \emph{triangulation}  of~ is a 
maximal crossing-free geometric (i.e., straight-line) graph whose 
vertices are the vertices of~ and whose edges lie inside~. 
Hence,  is an outerplanar graph. Similarly, for a set  of 
 points in the plane, a \emph{triangulation}  of~ is a maximal 
crossing-free geometric graph whose vertices are exactly the points of~.
It is well known that in both cases all bounded faces of  are triangles.
The \emph{dual graph}  of  is the graph 
with a vertex for each bounded face of  and an edge between 
two vertices if and only if the corresponding triangles share an edge in 
. If all vertices of  are incident to the unbounded face,
then  is a tree. An \emph{ear} in a 
triangulation of a simple polygon is a triangle whose vertex in the dual graph is a leaf (equivalently, two out of its three edges are edges of ).
We call the diameter of the dual graph  the \emph{dual diameter 
(of the triangulation )}.
In the following, we will study combinatorial and algorithmic properties
of
\emph{minimum} and \emph{maximum dual diameter triangulations} 
for simple polygons and for planar point sets
(\MinDT{}s and \MaxDT{}s for short).
Note that both triangulations need not to be unique.

\paragraph{Previous Work}
Shermer~\cite{s-cbtt-91} considers \emph{thin} and \emph{bushy} 
triangulations of simple polygons, i.e., triangulations that minimize or 
maximize the number of ears.
He presents algorithms for computing a thin triangulation in time
 and a bushy triangulation in time .
Shermer also claims that bushy triangulations are useful for finding paths 
in the dual graph, as is needed, e.g., in geodesic algorithms.
In that setting, however, the running time is not actually determined 
by the number of ears, but by the dual diameter of the triangulation.
Thus, bushy triangulations are only useful for geodesic problems if 
there is a connection between maximizing the number of ears and minimizing
the dual diameter.
While this holds for convex polygons, we show that, in general, there 
exist polygons for which no  maximizes the number of ears.
Moreover, we give examples where forcing a single ear into a 
triangulation may almost double the dual diameter, and the dual diameter 
of any bushy triangulation may be quadratic in the dual diameter of a 
.

The dual diameter also plays a role in the study of edge flips: 
given a triangulation , an \emph{edge flip} is the operation of replacing
a single edge of  with another one so that the resulting graph
is again a valid triangulation.
In the case of convex polygons, edge flips correspond to rotations in the
dual binary tree~\cite{sleator_tarjan_thurston}.
For this case, Hurtado, Noy, and 
Urrutia~\cite{hurtado_noy_urrutia, urrutia_flip_talk} show that a 
triangulation with dual diameter  can be transformed into a fan 
triangulation by a sequence of most  parallel flips (i.e., two 
edges not incident to a common  triangle may be flipped simultaneously).
They also obtain a triangulation with logarithmic dual diameter 
by recursively cutting off a linear number of ears.

While we focus on the dual graph of a triangulation, distance
problems in the primal graph have also been considered.
For example, Kozma~\cite{kozma} addresses the problem of finding a 
triangulation 
that minimizes the total link distance over all vertex pairs.
For simple polygons, he gives a sophisticated  time 
dynamic programming algorithm.
Moreover, he shows that the problem is strongly NP-complete for 
general point sets when arbitrary edge weights are allowed
and the length of a path is measured as the sum of the
weights of its edges.

\paragraph{Our Results}
In Section~\ref{sec_ears}, we present several properties of the dual 
diameter for triangulations of simple polygons.
Among other results, we calculate the exact dual diameter of s and 
s of convex polygons, which
can be obtained by maximizing and minimizing the number of ears of the 
triangulation, respectively.
On the other hand, we show that there exist simple polygons where 
the dual diameter of any  is , while that 
of any triangulation that maximizes the number of ears is in .
Likewise, there exist simple polygons where the dual diameter of any 
triangulation that minimizes the number of ears is in , 
while the maximum dual diameter is linear.
In Section~\ref{sec_poly}, we present efficient algorithms to construct a 
 and a  for any given simple polygon.

Finally, in Section~\ref{sec_points} we consider the case of planar
point sets, showing that for any point set in the plane in general position there
are triangulations with dual diameter in
 and in~, respectively.

\section{The Number of Ears and the Diameter}\label{sec_ears}

The dual graph of any triangulation  has maximum degree . 
In this case,
the so-called \emph{Moore bound} implies that the dual diameter of 
 is at least , where  is the number of 
triangles in  (see, e.g.,~\cite{moore_survey}).
For convex polygons, we can compute the minimum dual diameter exactly.

\begin{prop}\label{prop_convex}
Let  be a convex polygon with  vertices, and let
 such that
. 
Then any 
\MinDT\ of   has dual diameter
   if
  ,
  and
   if
  , for
  some .
\end{prop}
\begin{proof} 
The dual graph of any triangulation of~ is a tree with  
vertices and maximum degree ; see \figurename~\ref{fig_convex_tree} for 
an example.
Furthermore, every tree with  vertices and maximum degree  
is dual to some triangulation of~.

For the upper bound, suppose first that , for some 
. We define a triangulation  as follows.
It has a central triangle that splits  into three 
sub-polygons, each with  edges on the boundary. 
For each sub-polygon, the dual tree for  
is a full binary tree of height 
 with  leaves; see 
\figurename~\ref{fig_convex_full}.
The leaves of~ correspond to the ears of~.
The shortest path between any two ears in two different 
sub-polygons has length exactly
. The shortest path between any two ears in 
the same sub-polygon has length at most .
Thus, the dual diameter of~ is .

Now let  and consider the triangulation  of~ 
obtained by cutting off  
ears that are consecutive in the convex hull from . Then  is a subtree
of~. Since  has  
ears,  has at least  ears in common with .
Two of them lie in different sub-polygons, so the dual diameter remains 
.

Finally, for 
, 
if we remove   ears from 
such that all ears in two of the sub-polygons are removed, we get
a triangulation with dual 
diameter ; see 
\figurename~\ref{fig_convex_sparse}.
	
\begin{figure}[htb]
\centering
\subfigure[]{\label{fig_convex_tree}\includegraphics[scale=0.55, page=3]{fig_convex}} \hspace{0.3cm}
\subfigure[]{\label{fig_convex_full}\includegraphics[scale=0.55, page=1]{fig_convex}} \hspace{0.3cm}
\subfigure[]{\label{fig_convex_sparse}\includegraphics[scale=0.55, page=2]{fig_convex}}
\caption{The convex case. 
	(a) A triangulation and its dual tree. The ears 
	are gray. 
	(b) 
	The triangulation  for 
	. The 
	central triangle creates sub-polygons with  edges of  each.
	Any path between ears in different sub-polygons has length .
	Other paths are shorter.
	(c) 
	The triangulation  for  vertices 
	(). The central 
	triangle creates three sub-polygons, one with
	 edges of  and two with   edges of .
}
\label{fig_convex}
\end{figure}

For the lower bound, assume there is a tree  with 
 vertices, maximum degree , and diameter  strictly smaller 
than in the proposition. 
Consider a longest path  in  and a vertex  on 
 for which 
the distances to the endpoints of  differ by at most one. 
By adding vertices, we can turn   into a tree with 
 vertices, diameter , and the same structure as 
 or  for a convex polygon with 
vertices
(with  as central vertex).  
Since the upper 
bound on the dual diameter grows monotonically, this means that
the triangulation  or  for a convex polygon with 
vertices has diameter , a contradiction.
\end{proof}

As the dual graph of a triangulation of \emph{any} simple polygon has 
maximum degree , the proof of Proposition~\ref{prop_convex}
yields the following corollary.
\begin{cor}\label{cor_lb}
Let  be a simple polygon with  vertices, and let
 such that
. 
The dual diameter of 
any triangulation of  is at least 
 if
,
and 
 if 
. 
\end{cor}

Proposition~\ref{prop_convex} also shows that if  is convex, 
there exists a  with a maximum number of ears. 
Next, we show that this does not hold for general simple polygons.
Hence, any approach that tries to construct s by maximizing the 
number of ears is doomed to fail.

\begin{prop}\label{prop_leaves}
For arbitrarily large , there
exist simple polygons with  vertices in which any  minimizes the 
number of ears. 
\end{prop}
\begin{proof}
Let  and consider the polygon  with  vertices 
sketched in \figurename~\ref{fig_ears1}.
Any triangulation of  has either  or  ears.
The triangulation in \figurename~\ref{fig_ears1_5} is the only 
triangulation with 5 ears, and it has dual diameter .
However, as depicted in \figurename~\ref{fig_ears1_4good}, omitting the 
large ear at the bottom allows a triangulation with  ears and dual 
diameter .
Thus, forcing even one additional ear may nearly double the dual diameter. 
\end{proof}

Figure~\ref{fig_ears1_4bad} shows a triangulation
of  with  ears and almost twice the 
diameter as in \figurename~\ref{fig_ears1_4good}. Thus, neither for 
minimizing the diameter nor for maximizing the number of ears this 
triangulation is desirable. 
However, it has the nice property that the 
two top ears are connected by a dual path with four interior vertices.
Below, this property will be useful when making a larger construction.

\begin{figure}[htb]
\centering
\subfigure[5 ears, dual diameter .]{\label{fig_ears1_5}\includegraphics[scale=0.5, page=1]{fig_ears_new}}\hspace{0.4cm}
\subfigure[4 ears, dual diameter ]{\label{fig_ears1_4good}\includegraphics[scale=0.5, page=2]{fig_ears_new}}\hspace{0.4cm}
\subfigure[4 ears, dual diameter ]{\label{fig_ears1_4bad}\includegraphics[scale=0.5, page=3]{fig_ears_new}}
\caption{Three triangulations of a polygon with  vertices 
   and paths that define their dual diameters.
  The ears are shaded.}
\label{fig_ears1}
\end{figure}

\begin{theorem}\label{thm_leaves}
For arbitrarily large , there 
is a simple polygon with  vertices that has minimum dual diameter
 while any triangulation
that maximizes the number of 
ears has dual diameter . 
\end{theorem}
\begin{proof}
Let  be a parameter to be determined later, and
let  be the polygon constructed in Proposition~\ref{prop_leaves}. 
We construct a polygon  by concatenating  copies of  
as in \figurename~\ref{fig_ears2}.
 has  vertices.
Using the triangulation from \figurename~\ref{fig_ears1_5} for each copy,
we obtain a triangulation with the maximum number  of ears and dual 
diameter  (the curved line in \figurename~\ref{fig_ears2} 
indicates a longest path).
On the other hand, using the triangulation from 
\figurename~\ref{fig_ears1_4good} for the leftmost and rightmost part of 
the polygon and the one from \figurename~\ref{fig_ears1_4bad} for all 
intermediate parts yields a triangulation with dual diameter  
that has only  ears.
For , we obtain .
Thus, the dual diameter for the triangulation with maximum number of ears 
is , while the optimal dual diameter is .
\end{proof}

\begin{figure}[htb]
\centering
\subfigure[ ears, dual diameter .] {\includegraphics[width=.45\columnwidth, page=4]{fig_ears_new}} \hspace{0.5cm} \subfigure[ ears, dual diameter .] {\includegraphics[width=.45\columnwidth, page=5]{fig_ears_new}}
\caption{Two triangulations of a polygon with  vertices  
and corresponding longest paths. 
The ears are shaded.}
\label{fig_ears2}
\end{figure}

Similarly, for maximizing the dual diameter, we can give examples 
where the dual diameter is suboptimal when the number of ears is minimized.

\begin{theorem}
For arbitrarily large , there 
is a simple polygon with  vertices that has maximum dual diameter
 while any triangulation
that minimizes the number of 
ears has dual diameter . 
\end{theorem}
\begin{proof}
\figurename~\ref{fig:fig_ears_max} shows a triangulation of a part 
of a simple polygon.
We suppose that the indicated dual path~ is the only one of maximum 
length.
In addition to the ears at the endpoints of~, there are two ears at 
the vertices  and .
If we want to have at most one ear in this part of the polygon,
at least one of  and  must be connected to a 
non-neighboring vertex by a triangulation edge.
For this, the only possibilities are  and .
But then there cannot be any edge between the bottommost vertex~ 
and the  vertices between  and .
In particular, that part must be triangulated as shown to the right 
of \figurename~\ref{fig:fig_ears_max}.
Here, there is only one ear, but the dual diameter is reduced 
by~ (assuming the remainder of the polygon is large enough).
As in the proof of Theorem~\ref{thm_leaves}, we concatenate 
 copies of this construction and choose 
.
The parts are independent in the sense that they are separated by 
\emph{unavoidable} edges (i.e., edges that are present in any 
triangulation of the resulting polygon).\footnote{Unavoidable edges are defined by segments between two vertices s.t.\ no other edge crosses them.
Hence, they have to be present in every triangulation.
Unavoidable edges of point sets have been investigated by Karoly and Welzl~\cite{karolyi_welzl} (as ``crossing-free segments''), and Xu~\cite{xu} (as ``stable segments'').}
Hence, while the dual diameter of a \MaxDT{} is linear in~, it is 
in~ for any triangulation that minimizes the number of ears.
\end{proof}

\begin{figure}
\centering
\includegraphics[scale=0.8]{fig_ears_max}
\caption{Two triangulations of a part of a polygon where the dual diameter is locally decreased by~ when minimizing the number of ears.}
\label{fig:fig_ears_max}
\end{figure}


It is easy to construct polygons for which the dual graph of any 
triangulation is a path, forcing minimum dual diameter .
The other direction is slightly less obvious.

\begin{prop}
For any , there exists a simple polygon~ with  vertices such that the dual diameter of any \MaxDT{} of  is in~.
\end{prop}
\begin{proof}
We incrementally construct  by starting with an arbitrary 
triangle~.
See \figurename~\ref{fig:fig_max_log} for an accompanying illustration.
We replace every corner of~ by four new vertices so that two of them 
 can see only these four new vertices.
This means that the edge between the other two newly added vertices is 
unavoidable.
We repeat this construction recursively in a balanced way.
If necessary, we add dummy vertices to 
obtain exactly~ vertices.
The unavoidable edges partition  into convex regions, either 
hexagons or quadrilaterals. The dual tree of this partition is balanced 
with diameter . 
Since every triangulation of  contains all unavoidable edges,
the maximum possible dual diameter is .
\end{proof}

\begin{figure}
\centering
\includegraphics[scale=0.8]{fig_max_log}
\caption{The convex vertices of a polygon are incrementally replaced by four new vertices, resulting in unavoidable edges (dotted).}
\label{fig:fig_max_log}
\end{figure}

\section{Optimally Triangulating a Simple Polygon}\label{sec_poly}
We now consider the algorithmic question of constructing a  
and a  of a simple polygon  with  vertices. 
Let  be the vertices of~ in counterclockwise order.
The segment  is a \emph{diagonal} of  if it lies 
completely in~ but is not part of
the boundary of~.
For a diagonal , , we define  as the 
polygon with vertices ; 
see \figurename~\ref{fig_pij}.
Observe that  is a simple polygon contained in~.
If  is not a diagonal, we set .

\begin{figure}[tb]
\centering
\includegraphics[width=.5\columnwidth]{fig_pij}
\caption{Any triangulation of  
(gray) has exactly one triangle adjacent to 
 (dark gray).}
\label{fig_pij}
\end{figure}


\begin{theorem}\label{thm_min_alg}
For any simple polygon  with  vertices, we can compute a
 in  time.
\end{theorem}
\begin{proof}
We use the classic dynamic programming approach~\cite{klincsek}, with 
an additional twist to account for the non-local nature of the 
objective function.
Let  be a diagonal.
Any triangulation  of  has exactly one 
triangle  incident to ; see \figurename~\ref{fig_pij}. 
Let  be the maximum length of a path in 
 that has  as an endpoint.

For  and , with , let 
 be the set of all triangulations of  
with dual diameter at most  (we set 
 if  is not a diagonal of ).
We define , 
if , or , otherwise.
Intuitively, we aim for a triangulation that minimizes the distance 
from  to all other triangles of  while keeping 
the dual diameter below  (the value of  is the smallest 
possible distance that can be obtained). Let  be 
all vertices  of  such that the 
triangle  lies inside .
We claim that  obeys the following recursion:


where


We minimize over all possible triangles  in  incident to .
For each , the longest path to  is the longer of the paths to the other edges of  plus  itself.
If  joins two longest paths of total length more than , there is no valid solution with .
Thus, we can decide in  time whether there is a triangulation with dual diameter at most , i.e., if .
Since the dual diameter is at most , a binary search gives an  time algorithm.
\end{proof}

We can use a very similar approach to obtain some \MaxDT. 

\begin{theorem}
For any simple polygon  with  vertices, we can compute a
 in  time.
\end{theorem}
\begin{proof}
The proof is similar to the one of Theorem~\ref{thm_min_alg}.
This time, we are looking for triangulations that have dual diameter at 
least~.
Let  be defined as before, and let  be the set of 
all triangulations of  (this time, we do not need the 
third parameter).
We define  in the following way.
If , then .
If  contains a triangulation with diameter at 
least~, .
Otherwise, let .
Clearly, there is a triangulation with diameter at 
least~ if and only if .
With  defined as before, the recursion for 
 is


where



For the given diagonal , we maximize over all possible triangles.
If at some point the triangle  at  closes a path of length at 
least~, we are basically done, as any triangulation of the remainder 
of the polygon results in a triangulation with dual diameter at least~.
If the triangulation of  does not contain such a long path, 
we store the longer one to , as before.
Again, we can find the optimal dual diameter via a binary search, 
giving an  time algorithm.
\end{proof}


\section{Bounds for Point Sets}\label{sec_points}
We are now given a set  of  points
in the plane in general position, and we need to find a triangulation
of  whose dual graph optimizes the diameter.
Since the dual graph has maximum degree , it is easy to see that 
the  lower bound for simple polygons extends 
for this case.
It turns out that this bound can always be achieved, as we show in 
Section~\ref{sec_min_point_set}.
In Section~\ref{sec_max_point_set}, we find a triangulation of~ that 
has dual diameter in~.

\subsection{Minimizing the Dual Diameter}\label{sec_min_point_set}

\begin{theorem}\label{theo_pointset}
Given a set  of  points in the plane in general position, we can
compute in  time a triangulation
of  with dual diameter .
\end{theorem}

\begin{proof}
Let  be a convex polygon with  vertices and
 a triangulation of  with dual diameter
 (e.g., the triangulation from
Proposition~\ref{prop_convex}).
The triangulation  is an outerplanar graph.
Any outerplanar graph of  vertices has a plane
straight-line embedding on any given -point set~\cite{pgmp-eptvsp-91}.
Furthermore, such an embedding can be found in 
time and  space, where  is the dual diameter of the
graph~\cite{b-oeopgps-02}.

Let  be the embedding of  on~.
In general,  does not triangulate ; see \figurename~\ref{fig_pointset}.
\begin{figure}[tb]
\centering
\includegraphics[width=0.7\columnwidth]{fig_pointset2}
\caption{When computing a  of a point set , we first view it as if it were in convex position and construct a  (left image).
Then, we embed  into the actual point set (solid edges in the right image).
(The correspondence is marked by the central triangle and the thick boundary edge.)
All remaining untriangulated pockets (highlighted region in the figure) are triangulated arbitrarily (dashed edges). }
\label{fig_pointset}
\end{figure}
Consider the convex hull of  (which equals the convex hull of~).
The untriangulated \emph{pockets} are simple polygons.
We triangulate each pocket arbitrarily to obtain
a triangulation  of .
We claim that the dual diameter of  is .
\begin{lemma}\label{lem_logpoly}
The dual distance from any
triangle in a pocket to any triangle in  
is .
\end{lemma}
\begin{proof}
Let  be a pocket, and 
a triangulation of .
Since  is a simple polygon, the dual  is a 
tree with maximum degree .
A triangle  of  
not incident to the boundary of  either has
degree  in , or it is the unique triangle in  
that shares an edge with the convex hull of .
We perform a breadth-first-search in  starting from ,
and let  be the maximum number of consecutive layers from the root
of the BFS-tree that
do not contain a triangle incident to the boundary. 
By the above observation, all vertices in the first  levels have degree three in . Thus, each vertex of level  or lower has two children. In particular, at each level the number of vertices must double (except at the topmost level where the number of vertices is tripled), hence . 
\end{proof}

Given Lemma~\ref{lem_logpoly} and the fact that
 has dual diameter , 
Theorem~\ref{theo_pointset} is now immediate.
\end{proof}

\subsection{Obtaining a Large Dual Diameter}\label{sec_max_point_set}

We now focus our attention on the problem of triangulating  so that the 
dual diameter is maximized. 


\begin{theorem}\label{theo_pointslargeDiam}
Given a set  of  points in the plane in general position, we can
compute in  time a triangulation
of  with dual diameter at least .
\end{theorem}
\begin{proof}
Naturally, the triangulation  must contain the edges of the convex hull of .
Let  be the vertices of the convex hull of  
in clockwise order.
We connect  to the vertices ; see 
\figurename~\ref{fig:fig_points_max} (left).
In order to complete this set of edges to a triangulation, it suffices to consider 
the triangular regions  (for ) 
with at least one point of  in their interior.

\begin{figure}[bt]
\centering
\includegraphics{fig_points_max}
\caption{Left:  is connected to all remaining vertices in the convex hull.
Right: additional edges added inside . We connect the points of an increasing subsequence of  to both ,  as well as the predecessor and successor in the subsequence.}
\label{fig:fig_points_max}
\end{figure}

Let  be such a triangular region,  
the points in the interior of , and .
Let  denote the points in  sorted in 
clockwise order with respect to , and  denote the same points sorted in counterclockwise order 
with respect to .
By the Erd\H{o}s-Szekeres theorem~\cite{es_subsequence}, the index sequence  
contains an increasing or decreasing subsequence  of length 
at least .

If  is increasing, we connect all points of  to 
both  and . In addition, we connect each point of  
to its predecessor and successor in ; see 
\figurename~\ref{fig:fig_points_max} (right). Since the 
clockwise order with respect to  coincides with 
the counterclockwise order with respect to , the 
new edges do not create any crossing.

If  is decreasing, we claim that the corresponding point 
sequence is in counterclockwise order around .
Indeed, let  and  be two vertices of  whose indices appear 
consecutively in  (with  before ). By definition, the 
segment  crosses the segment . Moreover, points  and 
 are on the same side of the line through  and~; see \figurename~\ref{fig:fig_points_subsequence} (left). Since 
 and  are contained in , we conclude that  
must lie on the opposite side of the line. Thus,  form a 
counterclockwise sequence around , and
we can connect each point of  to , , and its 
predecessor and successor in  without crossings.
Finally, we add arbitrary edges to complete the resulting graph inside 
 into a triangulation .
\begin{figure}[tb]
\centering
\includegraphics{fig_points_subsequence}
\caption{Left: if  generates a decreasing sequence, the same sequence must be increasing when we view the angles with  instead.
Right: any path between  and  in the dual graph must visit all triangles  and at least  additional triangles between the crossing of segments  and .}
\label{fig:fig_points_subsequence}
\end{figure}

 We claim that, regardless of how we complete the triangulation, there 
 are two triangles whose distance in the dual graph is at least 
 . 
Let  and  be the triangles of  incident to edges  
and , respectively (since both segments are on the convex hull, 
 and  are uniquely defined). Let  be the shortest path from
 to . Clearly,  must cross each segment , for 
, exactly once and in increasing order.
This gives  steps (one step for each triangle 
incident in clockwise order around  on an edge 
, ).
In addition, 
at least  additional triangles must be traversed 
between the 
segments  and  (for all ): 
indeed, for each vertex , the edges 
 and either  or  (depending on whether  
was increasing or decreasing) disconnect  and .,
Hence at least one 
of the two must be crossed by , and the triangles following these
edges are pairwise distinct and distinct from the triangles following
the segments .
Summing over , we get

In the second inequality we used the fact that 
. (since a point is 
either on the convex hull or in its interior), the third inequality follows from  (and the fact that the expression is minimized when  is as small as possible). 

Finding a longest increasing (or decreasing) subsequence of  numbers 
takes  time, which is optimal in the comparison 
model~\cite{fredman}.
Hence, all subsequences, as well as the whole triangulation can 
be computed in  time,
where the last part also uses the fact that point location in a triangulation 
on  vertices can be done in  time after 
 preprocessing.
\end{proof}

\begin{prop}\label{prp:convex_subset}
Any set of  points in the plane in general position with  points in convex position has a triangulation with dual diameter in .
\end{prop}
\begin{proof}
See \figurename~\ref{fig:fig_convex_k_set} for an accompanying illustration.
Let  be such a point set with  being the convex subset of size~.
First, triangulate only~ by a zig-zag chain of edges: for the convex hull of~ being defined by the sequence , add the edges  and  for , as well as the boundary of the convex hull of~.
Then, add the extreme points of  and triangulate the convex hull of  without  such that each added edge is incident to a point of  (this is not necessary to obtain the result, but it makes our arguments simpler).
The resulting triangulation has dual diameter ,
as is witnessed by the triangles at  and  inside : if we label each triangle with the index of the incident point in  that is closest to , then this index can change by at most~1 along a step in any dual path
between the triangles inside  incident to  and
.
The dual diameter does not decrease when adding the remaining points of~ and completing the triangulation arbitrarily.
\end{proof}

\begin{figure}
\centering
\includegraphics{fig_convex_k_set}
\caption{Left: zig-zag triangulation of a convex subset of size~.
Right: adding the edges to the extreme points implies a labeling of the triangles that relate to their distance to .}
\label{fig:fig_convex_k_set}
\end{figure}


\section{Conclusions}\label{sec_conclusion}

The proof of Corollary~\ref{cor_lb} (lower bound for simple polygons) 
is essentially based on fundamental properties of graphs (i.e., bounded 
degree) rather than geometric properties.
Since the bound is tight even for the convex case, it cannot be 
tightened in general.
However, we wonder if, by using geometric tools, one can construct a 
bound that depends on the number of reflex vertices of the polygon 
(or interior points for the case of sets of points).
Another natural open problem is to extend our dynamic programming 
approach for simple polygons to general polygonal domains 
(or even sets of points).

It is open whether Theorem~\ref{theo_pointslargeDiam} is tight. That is:
does there exist a point set~ such that the diameter of the dual 
graph of any triangulation of  is in ? From Proposition~\ref{prp:convex_subset}, 
we see that any such point set can contain at most  points
in convex position. 
Thus, the point set must have  convex hull 
layers, each with  points. We suspect that some smart 
perturbation of the grid may be an example, but we have been unable to 
prove so.  


\section*{Acknowledgements}
Research for this work was supported by the ESF EUROCORES
programme EuroGIGA--ComPoSe, Austrian
Science Fund (FWF): I 648-N18 and grant
EUI-EURC-2011-4306. 
M.~K. was supported in part by the ELC project (MEXT KAKENHI No. 24106008).
S.L. is Directeur de Recherches du FRS-FNRS.
W.M.\@ is supported in part by DFG grants
MU 3501/1 and MU 3501/2.
Part of this work has been done while A.P.\ was recipient of a 
DOC-fellowship
of the Austrian Academy of Sciences at
the Institute for Software Technology,
Graz University of Technology, Austria. 
M.S.\@ was supported by the project LO1506 of the Czech Ministry of Education, Youth and Sports, and by the project NEXLIZ CZ.1.07/2.3.00/30.0038, 
which was co-financed by the European Social Fund and the state budget of the Czech Republic.



\bibliographystyle{abbrv}
\bibliography{bibliography}

\end{document}
