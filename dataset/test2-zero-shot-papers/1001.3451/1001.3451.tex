\documentclass{article-cfip}

\usepackage{amsmath}
\usepackage{subfig}
\usepackage{graphicx}
\usepackage{array}
\usepackage[utf8]{inputenc}
\usepackage[pdftex,unicode]{hyperref}

\begin{document}

\title[ICMN et Dimensionnement de Messages]{Dimensionnement des messages dans un réseau \\ mobile opportuniste}

\author{John Whitbeck\fup{*,**} \andauthor Vania Conan\fup{*} \andauthor Marcelo Dias de Amorim\fup{**}}
\address{
  \fup{*}Thalès Communications \
r = \frac{1}{1-q_c} \quad ; \quad \lambda = \frac{1-q_c}{1-q_i} .

E(T_c) = r \tau \quad ; \quad E(T_i) = \lambda r \tau .
\label{contact_inter_contact_times}

\pi_\uparrow = \frac{1}{1+\lambda} \quad ; \quad \pi_\downarrow = \frac{\lambda}{1+\lambda}.

P_{cont}(m,p,|U|,|W|) = \textrm{pdf}_{\mathcal{B}}\left(m,1-(1-p)^{|U|},|W|\right)

P_{succ}(i,j) = 1 - \pi_\downarrow^j q_i^i.
\label{p_succ}

  \lefteqn{\left( 1-P_{succ}(i,j) \right) \sum_{m=0}^{j'} \big( P_{cont}(m,\pi_\uparrow,j,N-1-i-j)} \nonumber \\
  & & \times P_{cont}(j'-m,1-q_i,i,N-1-i-j-m) \big).
\label{p_trans}

P_{deliv}( d, \alpha = 1 ) = \mathbf{i} \mathbf{T}^d \mathbf{s}.

\left( 1-P_{succ}^{static}(i,j) \right) \cdot P_{cont}(j',\pi_\uparrow,j,N-1-i-j).

P_{deliv}(d,\alpha < 1) = \mathbf{i} \left( \mathbf{T} \mathbf{R}^{\lfloor \frac{1}{\alpha} \rfloor -1} \right)^d \mathbf{s}.

\mathbf{i} \mathbf{T_{inf}}^{\frac{d}{\lceil \alpha \rceil}} \mathbf{s} \le P_{deliv}(d,\alpha > 1) \le \mathbf{i} \mathbf{T_{sup}}^{\frac{d}{\lceil \alpha \rceil}} \mathbf{s}.


\subsection{Influence de la taille du paquet}
\begin{figure}[t]
  \centering
  \includegraphics{sizes}
  \caption{Influence de la taille du paquet sur le taux de livraison
    (, , ) pour différentes valeurs du
    délai maximal toléré (). Il y a deux lignes pour chaque valeur
    de  qui correspondent aux bornes inférieure et supérieure.}
  \label{param_size}
\end{figure}

Les paquets de taille supérieure à la capacité du lien voient leur
taux de livraison se dégrader sérieusement, bien que ce soit à nuancer
pour de grands délais tolérés (fig.~\ref{param_size}). A l'inverse,
les paquets plus petits que la capacité du lien sont capables
d'effectuer plusieurs sauts par pas de temps. Ceci est
particulièrement avantageux lorsque les contraintes de délai sont
fortes ( sur la fig.~\ref{param_size}), mais l'est moins lorsque
cette contrainte est relâchée. L'influence de la mobilité des n{\oe}uds
apparaît ici. En effet, puisque la taille des paquets est
proportionnelle au pas de temps 
(cf. section~\ref{subsec:hypotheses}), une mobilité importante
(c.-à-d. un  petit) rend plus petite la capacité réelle des
liens (également proportionelle à ) et donc place davantage de
contraintes sur les tailles de paquet envisageables.

\subsection{Influence des autres paramètres}
\label{params}

\begin{figure}[t]
  \centering
  \subfloat[Nombre de n{\oe}uds \label{param_nodes}]{\includegraphics{params_num_nodes}} \qquad
  \subfloat[Durée de vie moyenne des liens \label{param_life}]{\includegraphics{params_r}} \\
  \subfloat[Degré moyen des n{\oe}uds \label{param_degree}]{\includegraphics{params_l}} \qquad
  \subfloat[Délai maximum \label{param_delay}]{\includegraphics{params_delay}}
  \caption{Influence des paramètres du modèle sur le taux de
    livraison. Sauf mention contraire, , ,
     et .}
  \label{params_influence}
\end{figure}

\noindent\textbf{Nombre de n{\oe}uds.}  (fig.~\ref{param_nodes}) Le taux
de livraison tend vers  lorsque  augmente. En effet, pour une
paire source/destination donnée, chaque nouveau n{\oe}ud est un nouveau
relais potentiel lors de la dissémination épidémique, ce qui ne peut
qu'aider le taux de livraison.

\noindent\textbf{Durée de vie moyenne des liens.}
(fig.~\ref{param_life}) Une plus courte durée de vie moyenne des liens
correspond à une topologie plus dynamique du réseau. En effet, de
petites valeurs de  se traduisent par des temps de contact et
d'inter-contact plus courts (Eq. \ref{contact_inter_contact_times}) et
donc augmentent les opportunités de contact. De petits paquets
() peuvent en profiter et leur taux de livraison
augmente lorsque  décroît. A l'inverse, une trop grande instabilité
des liens fait tendre le taux de livraison de paquets plus grands
() vers  car il existe alors moins de liens qui durent
plus d'un pas de temps.

\noindent\textbf{Degré moyen des n{\oe}uds.}  (fig.~\ref{param_degree})
Une plus grande connectivité accroît le taux de livraison. La forte
pente de la courbe lorsque  rappelle le phénomène de
percolation dans des graphes aléatoires réguliers quand le degré moyen
atteint .

\noindent\textbf{Délai maximum.} (fig.~\ref{param_delay}) Toutes
choses égales par ailleurs, il y a clairement une valeur au-delà de
laquelle presque tous les paquets sont livrés. On peut relier ceci au
diamètre spatio-temporel de la topologie
sous-jacente~\cite{chaintreau_diam}.


\section{Résultats expérimentaux}
\label{sec:experimental}

Les résultats théoriques de la section précédente guident notre
intuition sur des scénarios réels. Bien que le modèle ne soit pas
quantitativement comparable à des résultats obtenus à partir de traces
de connectivité réelles à cause de propriétés ``petit-monde''
indésirables, il prédit précisément les relations entre le taux de
livraison, le délai maximum et la taille des paquets, comme nous le
verrons dans cette section.

\subsection{Jeux de données}
\label{datasets}

\begin{table}[t]
  \centering
\caption{Quelques traces Bluetooth.}
\begin{tabular}{lccc}
& \textbf{MIT} & \textbf{Infocom} & \textbf{Rollernet} \\
  \hline
  \textbf{Durée (jours)}    & 365     & 3              & 0.125 \\
  \textbf{Période d'échantillonage (s)}& 600     & 120     & 15 \\
  \textbf{Nombre de capteurs}            & 100    & 41              & 62 \\
  \hline
\end{tabular}
  \label{blue}
\end{table}

La collecte de traces de connectivité sans-fil entre appareils
portables s'effectue typiquement à partir de sondages Bluetooth
périodiques. Dans ce papier, nous avons choisi d'étudier la trace
Rollernet~\cite{tournoux08_rollernet}, collectée lors d'une randonnée
roller, pour sa très courte période d'échantillonage. En effet, plus
la période d'échantillonage est longue,
plus il devient difficile de supposer qu'un contact équivaut à
l'existence d'un lien qui dure à peu près aussi longtemps que la
période d'échantillonage, ce qui est l'une de nos hypothèses de
base. C'est pourquoi, afin de pouvoir comparer théorie et pratique,
nous avons besoin de traces avec de très courtes périodes
d'échantillonage.

D'autres traces furent considérées, comme l'expérience ``Reality
Mining'' du MIT, dans laquelle chaque participant faisait tourner une
application sur son téléphone mobile qui relevait périodiquement sa
proximité avec les 100 autres participants pendant une année
complète~\cite{mit}. Le projet Haggle utilisa des Intel iMotes pour
mesurer les contacts entre participants de la conférence Infocom
2005~\cite{chaintreau_mobility}. Une rapide comparaison des ces traces
se trouve sur le tableau~\ref{blue}. La trace Rollernet est la seule à
proposer une période d'échantillonage suffisamment courte. D'une
certaine façon, on peut dire que les traces MIT et Infocom capturent
un sous-ensemble des opportunités de contact, tandis que la trace
Rollernet s'approche de l'évolution du graphe de connectivité.

\subsection{Méthodologie}

Nous étudions le taux de livraison réalisé par du routage épidémique
pour différentes tailles de paquets sur la trace de connectivité
sans-fil Rollernet~\cite{tournoux08_rollernet}.
Comme dans le modèle Markovien décrit à la section~\ref{sec:theory},
nous supposons que tous les liens ont la même capacité. La période
d'échantillonage dans Rollernet est de 15 secondes. La trace dure 3000
secondes. Toutes les 15 secondes pendant les premières 2000 secondes,
nous tirons au sort 60 couples source/destination pour une simulation
de routage épidémique.

\subsection{Des paquets plus petits augmentent le taux de livraison}

\begin{figure}[t]
  \centering
  \subfloat[Taux de livraison en fonction de la taille des paquets.\label{exp_results}]{\includegraphics{rollernet}} \qquad
  \subfloat[Taille de paquet maximale capable de respecter un taux de livraison fixé en fonction du délai.\label{size_delay}]{\includegraphics{rollernet_delay}}
  \caption{Résultats sur la trace Rollernet.}
  \label{exp}
\end{figure}

Sur la fig.~\ref{exp_results}, le taux de livraison est constant et
proche de  avant de chuter rapidement au-delà d'une certaine taille
de paquet qui dépend du délai fixé. Due a la mobilité très
élevée, la durée de vie moyenne d'un lien dans Rollernet est 
secondes et plus de la moitié des liens durent moins de 
secondes. Ainsi, des paquets de taille plus grande que  perdent de
nombreuses opportunités de contact mais ceci peut être compensé par
une plus grande tolérance sur les délais. Ces observations
correspondent exactement aux résultats théoriques sur la taille, le
délai et la mobilité de la fig.~\ref{param_size}.


\subsection{Un gain borné pour des paquets plus petits}

Sur la fig.~\ref{exp_results}, lorsque le délai maximum est d'une
minute, le taux de livraison maximal est d'environ . Prendre des
paquets plus petits n'y changera rien. Cette borne sur le gain obtenu
par de petits paquets apparaît car ils atteignent les limites de
performance du routage épidémique. En effet, la plus rapide diffusion
épidémique possible infectera, à chaque pas de temps, la totalité
d'une composante connexe dès lors que l'un de ses n{\oe}uds était
contaminé. Un paquet suffisamment petit pourrait produire cet effet,
mais des paquets encore plus petits n'auraient aucun avantage en
termes de taux livraison. Cette même limite de gain pour les petits
paquets est visible sur la courbe  de la fig.~\ref{param_size}.

\subsection{Des délais courts nécessitent de petits paquets}

Afin de mieux comprendre la relation entre délai maximum et taille de
paquet convenable, la fig.~\ref{size_delay} trace la plus grande
taille de paquet qui est capable de respecter un taux de livraison
fixé, en fonction du délai maximum. Une contrainte forte sur le délai,
moins de quelques minutes par exemple, impose l'usage de petits
paquets afin d'obtenir un taux de livraison satisfaisant. \`A l'inverse,
le relâchement de cette contrainte de délai amène davantage de
flexibilité concernant la taille des paquets.

\section{Conclusion}

Dans ce papier, nous avons proposé un nouveau modèle de graphes
temporels aléatoires qui, pour la première fois, capture la
corrélation entre graphes de connectivité successifs. Les résultats
théoriques concernant l'interaction entre la mobilité des n{\oe}uds, le
délai maximum toléré et la taille des paquets sont confirmés
expérimentalement. En particulier, nous avons montré que, pour un
délai maximum fixé et une certaine mobilité des n{\oe}uds, la taille des
paquets a un impact majeur sur le taux de livraison. Ce résultat
devrait être pris en considération lors de la conception et
l'implémentation de nouveaux services mobiles.

\acknowledgements{Ce travail a été partiellement soutenu par le projet
  ANR Crowd ANR-08-VERS-006}

\bibliography{whitbeckcfip09}

\end{document}
