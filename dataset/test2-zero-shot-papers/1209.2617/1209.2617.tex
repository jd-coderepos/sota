



\subsection{Lemmas}

The following lemmas are used in the proofs for the results in the article. Most of them are straightforwardly proved by induction, so we only detail the proof in the interesting cases.




\begin{lemma}\label{shellPatterns}
.
\end{lemma}



\begin{lemma}\label{lemma:crwlPatterns}
.
\end{lemma}



\begin{lemma}\label{LLetOrd2}
 Given , , if  then .
\end{lemma}


\begin{lemma}\label{auxMon}
Given , , if  then .
\end{lemma}



\begin{lemma}\label{LLetOrd3}
For every , , if  then .
\end{lemma}
\begin{proof}\label{DEMO_LLetOrd3}
We proceed by induction on the structure of . The base case is straightforward because of the hypothesis. For the Inductive Step we have:
\begin{itemize}
         \item . Directly by IH.
         \item , so . Then:

 By IH we have , therefore . Finally, by Lemma \ref{LLetOrd2}, .
         \item . Similar to the previous case but using Lemma \ref{auxMon} to obtain   from the IH .
\end{itemize}
\end{proof}

\begin{lemma}\label{lemma:context}
If  then 
\end{lemma}
\begin{proof}\label{DEMO_lemma:context}
Since  is a partial order, we know by reflexivity that  and . Then by Lemma \ref{LLetOrd3} we have  and . Finally, by antisymmetry of the partial order  we have that .
\end{proof}

\begin{lemma}\label{LCasc1}
For all , 
\end{lemma}
\begin{proof}\label{DEMO_LCasc1}
By induction on the structure of . The most interesting case is when .
By the variable convention  and , so:

(*) Using Lemma \ref{auxBind} with the matching .
\end{proof}

\begin{lemma}\label{lemaAproxTheta}
Given , , if  then .
\end{lemma}


\begin{lemma}\label{T28}\label{lemSintaxtConSubst}
For every ,  and  such that  we have that . \end{lemma}
\begin{proof}
By induction on the structure of . The most interesting cases are those concerning let-expressions:
\begin{itemize}
        \item : therefore . Then

: by definition , so .\\
: we can apply the last step because by hypothesis we can assure that we do not need any renaming to apply .
        \item : therefore . Then

: we can apply the last step because by hypothesis we can assure that we do not need any renaming to apply .
\end{itemize}
\end{proof}


\begin{lemma}\label{lDios}
For any ,  and program , if
 then there is a derivation for  in which every
free variable used belongs to .\end{lemma}
\begin{proof}\label{DEMO_lDios}A simple extension of the proof in \cite{DiosLopez07}.
\end{proof}



\begin{lemma}\label{LAlfCrwlP}
For every  derivation  there exists  which is syntactically equivalent to  module
-conversion, and a  derivation for  such
that if  is the set of bound variables used in  and
 is the set of free variables used in the instantiation of extra
variables in  then .
\end{lemma}
\begin{proof}\label{DEMO_LAlfCrwlP}By Lemma \ref{lDios}, if  is the set of free variables used in , then , in fact , as  and  are used in the top derivation of the
derivation tree for . As by definition , if we prove  then
 is a trivial consequence. To
prove that we will prove that for every  used in the
derivation for  we have . We can build
 using -conversion to ensure that . This can be easily maintained as an invariant during the derivation,
as the new let-bindings that appear during the derivation are those
introduced in the instances of the rule used during the \textbf{OR} steps, and
be can ensure by -conversion that  for
these instances too, as -conversion leaves the hypersemantics
untouched.
\end{proof}


\subsection{Proofs for Section \ref{sectPrelimCRWL}}

\teoremi{Theorem \ref{thCompoCrwl} (Compositionality of \crwl)}
For any , 

As a consequence: 

\begin{proof}\label{DEMO_thCompoCrwl}
We prove that  such that  and .
~\\~\\
) Induction on the size of the proof for .

{\bf Base case} The base case only allows the proofs  using \clrule{B},  using \clrule{RR} and  with  using \clrule{DC}, that are clear. When  the proof is trivial with  and using Lemma \ref{lemma:crwlPatterns}.

{\bf Inductive step} Direct application of the IH.

~\\~\\
) By induction on the size of the proof for 

{\bf Base case} The base case only allows the proofs ,  and  with , that are clear. When  we have that  such that  and . Since  by Lemma \ref{lemmashells} we have , and using Proposition \ref{propCrwlletPolar}  ---as  because  is a partial order.

{\bf Inductive step} Direct application of the IH.



\end{proof}

\subsection{Proofs for Section \ref{discussion}}

\teoremi{Theorem \ref{brcrwl}}
Let  be a \crwl-program,  and . Then:

\begin{proof}
It is easy to see that  coincides with the  relation  defined by the   {\it BRC}-proof calculus of \cite{GHLR99}, that is,
.
But in that paper it is proved that {\it BRC}-derivability
and \crwl-derivability (called there {\it GORC}-derivability) are equivalent.
\end{proof}

\subsection{Proofs for Section \ref{let-rewriting}}

\teoremi{Lemma \ref{auxBind} (Substitution lemma for let-expressions)}
Let ,  and  such that .
Then:


\begin{proof}\label{DEMO_auxBind}
By induction over the structure of . The most interesting cases are the base cases:\begin{itemize}
\item : Then

\item : Then

\end{itemize}
\end{proof}

\subsection{Proofs for Section \ref{sect:letRwRelation}}

\teoremi{Lemma \ref{LRwCerr} (Closedness under  of let-rewriting)}
For any ,  we have that  implies .
\begin{proof}\label{DEMO_LRwCerr}
We prove that  implies  by a case distinction over the rule of the let-rewriting calculus applied:
 \begin{description}
        \item[\lrrule{Fapp}] Assume , using  and  such that  and . But since  and  then we can perform a \lrrule{Fapp} step .



        \item[\lrrule{LetIn}] Easily since  because  is fresh.

        \item[\lrrule{Bind}] Assume  and some . Then  by the conditions of \lrrule{Bind}, hence  too and we can perform a \lrrule{Bind} step . Besides  by the variable convention, and so  by Lemma \ref{auxBind}, so are done.

        \item[\lrrule{Elim}] Easily as  because  by the variable convention.

        \item[\lrrule{Flat}] Similar to the previous case since .

        \item[\lrrule{Contx}] Assume  because  by one of the previous rules, and some . Then we have already proved that . Besides by the variable convention we have , hence by Lemma \ref{T28} . Furthermore, if  was a \lrrule{Fapp} step using  to build the instance of the program rule , then  by the conditions of \lrrule{Contx}, and therefore . But as  is the substitution used in the \lrrule{Fapp} step , then  by \lrrule{Contx}. 
        On the other hand, if  was not a \lrrule{Fapp} step then  too, and finally we can apply Lemma \ref{T28} again to get .
\end{description}

The proof for  proceeds straightforwardly by induction on the length  of the derivation.
\end{proof}

\teoremi{Proposition \ref{termlr} (Termination of )}
Under any program we have that  is terminating.
\begin{proof}\label{DEMO_termlr}
We define for any  the size , where
\0.5ex]
}

\noindent
Sizes are lexicographically ordered. We prove now that application of \emph{\lrrule{LetIn}, \lrrule{Bind}, \lrrule{Elim}, \lrrule{Flat}} in any context (hence, also the application of \lrrule{Contxt}) decreases the size, what proves termination of . The effect of each rule in the size is summarized as follows (in each case, we stop
at the decreasing component):\begin{center}
\begin{tabular}{ll}
\emph{(LetIn):} & 
\\
\emph{(Bind):} & 
\\
\emph{(Elim):} & 
\\
\emph{(Flat):} & 
\\
\end{tabular}\end{center}
\end{proof}


\teoremi{Lemma \ref{lemma:bigPeeling} (Peeling lemma)}
For any  if  ---i.e,  is a  normal form for --- then  has the shape  such that  or  with ,  and . \\
Moreover if  with , then

under the conditions above, and verifying also that  whenever .

\begin{proof}\label{DEMO_lemma:bigPeeling}
We prove it by contraposition: if an expression  does not have that shape,  is not a  normal form. We define the set of expressions which are not cterms as:

\begin{tabular}{ll@{}}
 ::=    &  \\
            &  \\
            &  \\
\end{tabular}

We also define the set of expressions which do not have the presented shape recursively as:

\begin{tabular}{ll}
 ::=    &  \\
            &  \\
            &  \\
            &  \\
            &  \\
\end{tabular}

We prove by induction on the structure of an expression  that it is always possible to perform a  step:

{\bf Base case:}
\begin{itemize}
\item : there are various cases depending on :
    \begin{itemize}
        \item at some depth the non-cterm will contain a subexpression  where  is a function application  or a let-rooted expression . Therefore we can apply the rule \lrrule{Contx} with \lrrule{LetIn} in that position.
\item : we can apply the rule \lrrule{LetIn} and perform the step 
        \item : the same as the previous case.
    \end{itemize}
\item : we can perform a \lrrule{Contx} with \lrrule{LetIn} step in  as in the previous  case.
\item : if  are cterms , then  is a cterm and we can perform a \lrrule{Bind} step . If  contains any expression  then we can perform a \lrrule{Contx} with \lrrule{LetIn} step as in the previous  case.
\item : by the variable convention we can assume that , so we can perform a \lrrule{Flat} step .
\end{itemize}

{\bf Inductive step:}
\begin{itemize}
    \item : by IH we have that , so by the rule \lrrule{Contx} we can perform a step .
\end{itemize}


Notice that if the original expression has the shape  the arguments  which are cterms remain unchanged in the same position. The reason is that no rule can affect them: the only rule applicable at the top is \lrrule{LetIn}, and it can not place them in a let binding outside ; besides cterms do not match with the left-hand side of any rule, so they can not be rewritten by any rule.
\end{proof}





\teoremi{Lemma \ref{LCascCrec} (Growing of shells)}
Under any program  and for any 
\begin{enumerate}
    \item[i)]  implies 
    \item[ii)]  implies 
\end{enumerate}

\begin{proof}[Proof for Lemma \ref{LCascCrec}]\label{DEMO_LCascCrec}
We prove the lemma for one step ( and ) by a case distinction over the rule of the let-rewriting calculus applied:
\begin{description}
\item[(Fapp)] The step is , and .

\item[(LetIn)] The equality  follows easily by a case distinction on .


\item[(Bind)] The step is , so  by Lemma \ref{LCasc1}.


\item[(Elim)] The step is  with . Then . Since the variables in the shell of an expression is a subset of the variables in the original expression, we can conclude that if  then .

\item[(Flat)] The step is  with . By the variable convention we can assume that  ---in particular . Then:
        
        Notice that  and  because  and . Therefore we can use Lemma \ref{auxBind}:
        

\item[(Contx)] The step is  with  using any of the previous rules. Then we have , and by Lemma \ref{LLetOrd3} . If the step is  then rule (Fapp) has not been used in the reduction  and by the previous rules we have . In that case by Lemma \ref{lemma:context} we have .
\end{description}
The extension of this result to  and  is a trivial induction over the number of steps of the derivation.
\end{proof}




\subsection{Proofs for Section \ref{crwllet}}

\teoremi{Theorem \ref{thEquivCrwlCrwllet} (\crwl\ vs. \crwll)}
For any program  without lets, and any :

\begin{proof}
As any calculus rule from \crwl\ is also a rule from \crwll, then any \crwl-proof is also a \crwll-proof, therefore .
For the other inclusion, assume no let-binding is present in the program and let . Then, for any , as the rules of \crwll\ do not introduce any let-binding and the rule (Let) is only used for let-rooted expressions, the \crwll-proof  will be also a \crwl-proof for , hence  too.
\end{proof}



The following Lemma is used to prove point \emph{iii)} of Lemma \ref{lemmashells}. Notice that this Lemma uses the notions of hyperdenotation () and hyperinclusion () presented in the final part of Section \ref{crwllet}.

\begin{lemma}\label{LemEx2iii}
Under any program  and for any  we have that .
\end{lemma}
\begin{proof}
We will use the following equivalent characterization of :

note that  is precisely the set . Besides note that:

where  is implied by . To prove this last formulation first consider the case when . Then we are done with  because then  and .

For the other case we proceed by induction on the structure of . Regarding the base cases:
\begin{itemize}
    \item If  then  and we are in the previous case.
    \item If  then , and as  then  which implies  by Lemma \ref{lemmashells}. But then we can take  for which  and  ---by Lemma \ref{shellPatterns} since ---, and .
    \item If  then either  and we are in the previous case, or . But then we can take  for which , and .
    \item If  then , and so  and , so we are done.
\end{itemize}
Concerning the inductive steps:
\begin{itemize}
    \item If  for  then  and we proceed like in the case for .
    \item If  for  then either  and we are in the previous case, or  such that . But then by IH we get , so we can take  for which  and .
    \item If  then either  and we are in the previous case, or we have the following proof:


Then by IH over  we get that . Hence  so by Proposition \ref{PropMonSubstCrwlLet} we have that  implies . But then we can apply the IH over  with  to get some  such that  and , which implies:





\end{itemize}
\end{proof}

\teoremi{Lemma \ref{lemmashells}}
For any program  , :
  \begin{enumerate}
  \item  iff .
  \item .
  \item , where for a given  its upward closure is , its downward closure is , and those operators are overloaded for let-expressions as  and .
\end{enumerate}

\begin{proof}\label{DEMO_lemmashells}
\begin{enumerate}
  \item Easily by induction on the structure of .
  \item Straightforward by induction on the structure of . In the case of let expressions, the proof uses  and Proposition \ref{closednessCSubst} in order to apply the  \crwll ~rule (Let).
  \item By Lemma \ref{LemEx2iii} we have that . By definition of hyperinclusion ---Definition \ref{def:setFunOp}--- we know that , so .
\end{enumerate}
\end{proof}

\teoremi{Proposition \ref{propCrwlletPolar} (Polarity of \crwll)}
For any program  , , if  and  then  implies  with a proof of the same size or smaller---where the size of a \crwll-proof is measured as the number of rules of the calculus used in the proof.

\begin{proof}\label{DEMO_propCrwlletPolar}
By induction on the size of the \crwl-derivation. All the cases are straightforward except the \clrule{Let} rule:
\begin{description}
    \item[\clrule{Let}] We have the derivation:
     
     Since  then  with  and . As  and  ---because  is reflexive--- then by IH we have . We know that  so by Lemma \ref{lemaAproxTheta} we have  and by IH  such that . Therefore:
     
     \end{description}
\end{proof}

\teoremi{Proposition \ref{closednessCSubst} (Closedness under c-substitutions)}
For any   , , ,  implies .

\begin{proof}\label{DEMO_closednessCSubst}
By induction on the size of the \crwll-proof. All the cases are straightforward except the \clrule{Let} rule:
\begin{description}
    \item[\clrule{Let}] In this case the expression is  so we have a derivation
    
    By IH we have that  and . By the variable convention we assume that , so by Lemma \ref{auxBind}  and . Then we can construct the proof:
    
    \end{description}

\end{proof}

\teoremi{Theorem \ref{lem:weakcomp} (Weak Compositionality of \crwll)}
For any , 

As a consequence, .

\begin{proof}\label{DEMO_lem:weakcomp}
We prove that  such that  and .
~\\~\\
) By induction on the size of the proof for . The proof proceeds in a similar way to the proof for Theorem \ref{thCompoCrwl}, page \pageref{DEMO_thCompoCrwl}, so we only have to prove the \clrule{Let} case:
\begin{description}
\item[\clrule{Let}] There are two cases depending on the context  (since ):
    \begin{itemize}
    \item ) Straightforward.
    \item ) The proof is
    
    We assume that  by the variable convention, since  is bound in  and we can rename it freely. Moreover, we assume also that  because  is bound in , so we could rename the bound occurrences in . Therefore  and  by Lemma \ref{T28}. Since  by the premise and  then , so . Then by IH  such that  and . Therefore we can build:
    
    (*) Using Lemma \ref{T28} as above and the assumption that  by the variable convention, since  is bound in  and we can rename it freely.
    \end{itemize}
\end{description}
~\\~\\
) By induction on the size of the proof for . As before, the proof proceeds in a similar way to the proof for Theorem \ref{thCompoCrwl}, page \pageref{DEMO_thCompoCrwl}, so we only have to prove the \clrule{Let} case:

\begin{description}
\item[\clrule{Let}] If we use \clrule{Let} then there are two cases depending on the context  (since ):
    \begin{itemize}
    \item ) Straighforward.
\item ) then we have  and
        
        By the same reasoning as in the second case of the \clrule{Let} rule of the ) part of this theorem, . Then by IH . Again by the same reasoning we have , so we can build the proof:

\end{itemize}
\end{description}



~\\~\\This ends the proof of the main part of the theorem. With respect to the consequence  we have:


In the last step we replace  by  which is a \lrrule{Bind} step of , so by Proposition \ref{propFnfPreservHipSem} it preserves the denotation.
\end{proof}

For Proposition \ref{PropMonSubstCrwlLet}, in this Appendix we prove a generalization of the statement appearing in Section \ref{crwllet} (page \pageref{PropMonSubstCrwlLet}). However, it is easy to check that Proposition \ref{PropMonSubstCrwlLet} in Section \ref{crwllet} follows easily from points {\em 2} and {\em 3} here.

\teoremi{Proposition \ref{PropMonSubstCrwlLet} (Monotonicity for substitutions of \crwll)}
For any program , , 
\begin{enumerate}
    \item If  given  with size  we also have  with size , then  with size  implies  with size .
    \item If  then  implies  with a proof of the same size or smaller.
    \item If  then .
\end{enumerate}

\begin{proof}\label{DEMO_PropMonSubstCrwlLet}
\begin{enumerate}
\item If , assume , then  with a proof of the same size or smaller, by hypothesis. Otherwise we proceed by induction on the structure of the proof .
\begin{description}
 \item[Base cases]~
 \begin{description}
   \item[\clrule{B}] Then  and  with a proof of size  just applying rule \clrule{B}.
   \item[\clrule{RR}] Then  and we are in the previous case.
   \item[\clrule{DC}] Then , as , hence  and every proof for  is a proof for .
  \end{description}
 \item[Inductive steps]~
   \begin{description}
     \item[\clrule{DC}] Then , as , and we have:

By IH or the proof of the other cases  we have  with a proof of the same size or smaller, so we can built a proof for  using \clrule{DC}, with a size equal or smaller than the size of the starting proof.
   \item[\clrule{OR}]  Similar to the previous case.
\item[\clrule{Let}] Then , as , and we have:

    By IH we have . By the variable convention we assume that  and . Then it is easy to check that , given  with size  we also have  with size . Then by IH we have . Therefore we can construct a proof with a size equal or smaller than the starting one:
    
   \end{description}
 \end{description}

\item By induction on the size of the \crwll-proof. The cases for classical CRWL appear in \cite{vado02}, so we only have to prove the case for the \clrule{Let} rule:
    \begin{description}
    \item[\clrule{Let}] In this case the expression is  so we have a proof
    
    By IH we have that . By the variable convention we can assume that  and . With the previous properties it is easy to see that , so by IH . Therefore we can build the proof:
    

\end{description}

\item By induction on the structure of :
    \begin{description}
    \item[ -] In this case  because by the hypothesis .
    \item[ -] Applying Theorem \ref{lem:weakcomp} with  we have  because . On the other hand, by Theorem \ref{lem:weakcomp} we also know that 
    
    Since by IH we have  it is easy to check that  so . Using the same reasoning in the rest of subexpressions  we can prove:\\ \\
\\
\ldots \\
\\
Then by  transitivity of  we have:\\
 
 \\
.
    \item[ -] As Theorem \ref{lem:weakcomp} states, . By the Induction Hypothesis we have that . Due to the variable convention we assume that  and , so it is easy to check that  for any . Then by the Induction Hypothesis we know that . Therefore 

    \end{description}
\end{enumerate}
\end{proof}





\teoremi{Theorem \ref{CompHipSem} (Compositionality of hypersemantics)}
For all , 

As a consequence: .

\begin{proof}\label{DEMO_CompHipSem}
By induction over the structure of contexts. The base case is , so , as  is the identity function by definition. Regarding the inductive step:
\begin{itemize}
 \item : Then

 \item : Then

(*): by Proposition \ref{propFnfPreservHipSem}  since . \item : Then

\end{itemize}
\end{proof}




\teoremi{Proposition \ref{HipSemDecUnion}}
Consider two sets , and let  be the set of functions . Then:
\begin{enumerate}
    \item[i)]  is indeed a partial order on , and  is indeed a decomposition of , i.e., .
    \item[ii)] Monotonicity of hyperunion wrt. inclusion: for any 

    \item[iii)] Distribution of unions: for any 

    \item[iv)] Monotonicity of decomposition wrt. hyperinclusion: for any 

\end{enumerate}

\begin{proof}\label{DEMO_HipSemDecUnion}~
\begin{enumerate}
    \item[i)] The binary relation  is a partial order on  because:
    \begin{itemize}
        \item It is reflexive, as for any function  and any  we have that , and thus , therefore .
        \item It is transitive because given some functions  such that  and , then for any  we have  by definition of , hence .
        \item It is antisymmetric \wrt\ extensional function equality, because for any pair of hypersemantics  such that  and  and any  we have that  and  by definition of , hence  by antisymmetry of  and .
    \end{itemize}
In order to prove that  is indeed a decomposition of  we first perform a little massaging by using the definitions of  and .

Now we will use the fact that  is a partial order, and therefore it is antisymmetric, so mutual inclusion by  implies equality.

\begin{itemize}
\item \underline{}: Given arbitraries ,  then

\item \underline{}: Given arbitraries ,  then we have that , therefore  such that . But then  ---otherwise --- and  ---because ---, and so  implies .
\end{itemize}
        \item[ii)] Given an arbitrary  then

\item[iii)]


\item[iv)] Suppose an arbitrary  with  and  by definition. Since  then . Therefore  and .
\end{enumerate}
\end{proof}

\teoremi{Proposition \ref{HipSemDistCntx} (Distributivity under context of hypersemantics union)}

\begin{proof}\label{DEMO_HipSemDistCntx}
We proceed by induction on the structure of . Regarding the base case, then  and so:

For the inductive step we have several possibilities.
\begin{itemize}
    \item : then


    \item : then


    \item : then


\end{itemize}


\end{proof}


\subsection{Proofs for Section \ref{sectEqLetrwCrwlLet}}




\teoremi{Theorem \ref{T27} (Hyper-Soundness of let-rewriting)}
For all , if  then .

\begin{proof}\label{DEMO_T27}


We first prove the theorem for a single step of . We proceed assumming some  such that 
and then proving . The case where  holds
trivially using the rule \textbf{B}, so we will prove the rest by a case distinction
on the rule of the let-rewriting calculus applied:
\begin{description}
\item[\lrrule{Fapp}] Assume  with , , such that  and , and  such that . Then as  and  we can use the \clrule{OR} rule to build the following proof:


\item[\lrrule{LetIn}] Assume  by \lrrule{LetIn} and  such that . This proof must be of the shape of:

for some . Besides  by the variable convention\footnote{Actually, to prove this theorem properly, we cannot restrict the substitution to fulfill these restrictions, so in fact we rename the bound variables in an -conversion fashion and use the equivalence  (with  the new bound variable), to use the hypothesis. This will be done implicitly when needed during the remaining of the proof.}, hence  and so , as  is fresh by the conditions in \lrrule{LetIn} and so it does not appear in any . Now we have two possibilities:
\begin{enumerate}
\item[a)]  : Then  must proved by \clrule{DC}:

for some . Then  implies  by Lemma \ref{lemmashells}, hence  implies  by Proposition \ref{propCrwlletPolar}, and we can build the following proof: 

        \item[b)]  : Then  must be proved by \clrule{OR}:

for some , , .
Then we can prove  like in the previous case, to build the following proof:

\end{enumerate}

\item[\lrrule{Bind}] Assume  by \lrrule{Bind} and  such that . Then  by the variable convention, so we can apply Lemma \ref{auxBind} (Substitution lemma) to get . Besides  and  by hypothesis, hence  and we can build the following proof:



\item[\lrrule{Elim}] Assume  by \lrrule{Elim} and  such that . Then  by the variable convention and  by the condition of \lrrule{Elim}, hence  and we can build the following proof:



\item[\lrrule{Flat}] Assume  by \lrrule{Flat} and  such that
  . This proof must be must be of the shape of:

for some . Besides  by the variable convention and  by the condition of \lrrule{Flat}, hence   and we can build the following proof: 

 \item[\lrrule{Contx}] By the proof of the other cases, , but then  by Lemma \ref{T25}, and we are done.
\end{description}

The proof for several steps is a trivial induction on the length of the derivation .
\end{proof}

\teoremi{Proposition \ref{propFnfPreservHipSem} (The  relation preserves hyperdenotation)}
For all , if  then ---and therefore .

\begin{proof}\label{DEMO_propFnfPreservHipSem}
We first prove the lemma for one step of  by case distinction over the rule applied to reduce  to . By Theorem \ref{T27} we already have that  if  then , so all that is left is proving that  also, and finally applying the transitivity of , as it is a partial order by Lemma \ref{HipSemDecUnion}-i. We proceed assumming some  such that  and then proving . The case where  holds trivially using the rule \clrule{B}, so we will prove the other by a case distinction
on the rule of the  calculus applied:
\begin{description}

\item[\lrrule{LetIn}] Assume  by the \lrrule{LetIn} rule and  such that 

Then by the compositionality of Theorem \ref{thCompoCrwllet} we have that  such that . Besides  is fresh and  by the variable convention, hence 
and

and so we can do:
\small{

}

\item[\lrrule{Bind}] Assume  by \lrrule{Bind} and  such that . Then it must be with a proof of the following shape:

But  and  implies , and so  implies  by Lemma \ref{lemmashells}-1. Hence  and so  implies  by the monoticity of Proposition \ref{PropMonSubstCrwlLet}. Besides  by the variable convention, and so we can apply Lemma \ref{auxBind} (substitution lemma) to get , so we are done.

\item[\lrrule{Elim}] Assume  by \lrrule{Elim} and  such that . Then it must be with a proof of the following shape:

Then  by the variable convention and  by the condition of \lrrule{Elim}, hence , so we are done.

\item[\lrrule{Flat}] Straightforward since  because  by the variable convention and  by the condition of \lrrule{Flat}.


 \item[\lrrule{Contx}] By the proof of the other cases, , but then  by Lemma \ref{T25}, and we are done.
\end{description}
\end{proof}













The following lemmas ---Lemmas \ref{T35}, \ref{FVShells}, \ref{lemmaShellSubst1} and \ref{lemmaShellSubst2}--- will be used to prove Lemma \ref{T32}.

\begin{lemma}\label{T35}
Let linear  such that  for . Then  such that  and .
\end{lemma}
\begin{proof}
By induction on the structure of . For the base case () we define a function   that replaces the occurrences of  in  by the expression . We define this function recursively on the structure of :
\begin{itemize}
  \item 
  \item 
  \item 
\end{itemize}
  It is easy to check that  implies . Then we define  as:
  
  Trivially  and  because  by the premise and .

Regarding the inductive step ------ we know that 

so . Then by IH  such that  and . Then we define  as:
  
  The substitution  is well defined because  is linear. Then  and  by IH and the fact that .
\end{proof}






\begin{lemma}\label{FVShells}
For any , .
\end{lemma}
\begin{proof}
Straightforward by induction on the structure of .
\end{proof}

\begin{lemma}\label{lemmaShellSubst1}
Given , ,  where  is defined as 
\end{lemma}
\begin{proof}
By induction on the structure of . We have two base cases:
\begin{itemize}
  \item . Then .
  \item . Then .
\end{itemize}

Regarding the inductive step we have:
\begin{itemize}
  \item . Straightforward.
  \item . Then . By IH we have that  and , so . By the variable convention we can assume that , and since  and  ---using Lemma \ref{FVShells}--- we can use Lemma \ref{auxBind} and obtain . Finally, .
\end{itemize}
\end{proof}

\begin{lemma}\label{lemmaShellSubst2}
Given , , if  then .
\end{lemma}
\begin{proof}
By induction on the structure of . Notice that  cannot be a variable  or an applied constructor symbol  because in those cases . The base case  is straightforward. Regarding the inductive step we have  such that . Then . By Lemma \ref{LCasc1} , and since  by the variable convention then we can apply Lemma \ref{auxBind} and . Finally by Lemma \ref{lemmaShellSubst1} , and by Lemma \ref{LCasc1} .
\end{proof}



\teoremi{Lemma \ref{T32} (Completeness lemma for let-rewriting)}
For all  and  such that ,

for some  and   in such a way that  and  for every . As a consequence, .

\begin{proof}\label{DEMO_T32}
By induction on the size  of the {\it CRWL}-proof, that we measure as the number of  rules applied. Concerning the base cases:
\begin{description}
  \item[\clrule{B}] This contradicts the hypothesis because then , so we are done. In the rest of the proof we will assume that  because otherwise we would be in this case.

  \item[\clrule{RR}] Then we have . But then  and
  , so we are done with .

  \item[\clrule{DC}] Then we have . But then  and
      , so we are done with .
\end{description}

Now we treat the inductive step:
\begin{description}
 \item[\clrule{DC}] Then we have  and the
    {\it CRWL}-proof has the shape:
    
    In the general case some  will be equal to  and some others will be different. For the sake of simplicity we consider the case when  with  and , the proof can be easily extended to the general case.
Then  we have , so by IH over the second
    argument we get 
    with ,  for every  and
    . So:
    
    Then there are several possible cases:
\begin{enumerate}
    \item[a)] : Then
    , by
    \lrrule{LetIn}. So we are done as   for every
     by the IH,  and
 
    because
     by the IH, and  is fresh and
    so it does not appear in \item[b)] : Then we are done as 
    for every  by the IH, and ,
    because  by the IH\item[c)]  with : 
Then by Lemma \ref{T36} we have the derivation .
But then:
    
    In the last step notice that  is fresh and it cannot appear in . Then we are done as ,  for every  by the IH, and  
because  by the IH, and no variable in  appears in  by -conversion, as those are bound variables which were present in  or that appeared after applying Lemma \ref{T36} to it, and this expression was placed in a position parallel to the position of .\item[d)] : Then by Lemma \ref{T36}  where  or . Then:


Then we have two possibilities depending on :
    \begin{enumerate}
    \item[i)] : Then we can do:

Then we are done as ,  for every  by IH, and , as  by IH, and no variable in  appears in  by -conversion, like in the case {\it c)}.\item[ii)] : there are two possible cases:
        \begin{enumerate}
    \item[A)] : We are done as ,  for every  by IH, , and , as by IH ,  is fresh and so it does not appear in , and no variable in  appears in  as in the case {\it i)}.\item[B)] : Then we can do a \lrrule{Bind} step:

Then we are done as ,  for every  by IH, and 

as  by IH, and no variable in  appears in , as we saw in {\it i)}.\end{enumerate}
    \end{enumerate}

\end{enumerate}

\item[\clrule{OR}] If  has no arguments () then we have:
  
with  and . Let us define  as the substitution which is equal to  except that every
   introduced by  is replaced with some constructor symbol or
  variable. Then , so by Proposition \ref{PropMonSubstCrwlLet} we have
   with a proof of the same size. But then applying the
  IH to this proof we get  under the
  conditions of the lemma. Hence  applying (Fapp) in the first step, and we are
  done. \\

  If , we will proceed as in the case for \clrule{DC}, doing a preliminary version for  which can be easily extended for the general case. Then we have:
  
  such that , and with , , such that  and . Then applying the IH to  we get that  such that  for every
   and . Then we can do:
  
  Then applying Lemma \ref{T36} we get
  
Now as  then , so by Lemma \ref{T35} there must exist
   such that  and . Then by
  Proposition \ref{PropMonSubstCrwlLet}, as  then  with a proof of the same size. As  and  (because it is part of the program) then  and we can
  apply the IH to that proof getting that  such that  for every  and
  . Then we can do:
  
  Then  for every  by IH,  and  for every  by IH.
Besides the variables in  either belong to  or are fresh, hence none of them may appear in  (by Lemma \ref{LAlfCrwlP} over  or by freshness).
So
   implies that  such
  that  for some  then
  . But then
  .

\item[\clrule{Let}] Then  and we have a proof of the following shape:

Then we have two possibilities:
\begin{enumerate}
 \item[a)] : Then . Hence, as  and , by Proposition \ref{PropMonSubstCrwlLet} we get  with a proof of the same size or smaller, and so by IH we get , with ,  for every  and , and we can do:

Besides  by Lemma \ref{LAlfCrwlP} over , and then  implies  such that  then , and we have several possible cases:
\begin{enumerate}
  \item[i)] : Then we are donde because  by IH,  and , as  and  such that  then , as we saw above.

  \item[ii)] : But then

and we are done because  by IH, and so  by Lemma \ref{lemmaShellSubst2}. Besides, as in {\it i)},  combined with the fact that  such that  we have , implies that .

  \item[iii)]  with : Then by Lemma \ref{T36} we have  , hence

As by IH  then  by Lemma \ref{lemmaShellSubst2}. At this point we have to check that    .
The variables in  either belong to  or are fresh, hence by -conversion none of them may appear in , because in  the expression  has no access to the variables bound in  . Hence , for some .
But then, as in {\it ii)},  combined with the fact that  such that  we have , implies that .
\item[iv)] : Then by Lemma \ref{T36} we have , and so

Then either  and we are like in {\it iii)} before the final \lrrule{Bind} step, or  and  and  (by IH), and  because , as we saw in {\it iii)}. But then, as in {\it ii)},  combined with the fact that  such that  we have , implies that .
\end{enumerate}
 \item[b)] : Then by IH we get , with ,  for every  and . Hence  and so , but then  implies  with a proof of the same size or smaller, by Proposition \ref{propCrwlletPolar}. Therefore we may apply the IH to that proof to get , with ,  for every  and . But then we can do:

Then by the IH's we have  and . Besides the variables in  either belong to  or are fresh, hence none of them may appear in  (by Lemma \ref{LAlfCrwlP} over  or by freshness). So   implies that  such that  for some  then . But then .
\end{enumerate}
\end{description}
\end{proof}

