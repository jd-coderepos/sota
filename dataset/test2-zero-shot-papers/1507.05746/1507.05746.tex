\documentclass{amsart}
\usepackage{amssymb,bbm,comment}
\usepackage{graphics,url,hyperref}
\usepackage{enumerate}
\usepackage{caption}
\usepackage{subcaption}
\usepackage{tikz}
\usetikzlibrary{decorations.markings}
\usepackage{pgfplots}
\usetikzlibrary{arrows,calc}

\newcommand{\ind}[1]{{\mathbf 1}_{\{#1\}}}
\def\C{\mathbb{C}}
\def\N{\mathbb{N}}
\def\E{\mathbb{E}}
\def\P{\mathbb{P}}
\def\R{\mathbb{R}}
\def\Z{\mathbb{Z}}
\newcommand{\diff}{\mathop{}\mathopen{}\mathrm{d}}
\def\n{\mathbf{n}}
\def\cal{\mathcal}

\newtheorem{proposition}{Proposition}
\newtheorem{lemma}{Lemma}[proposition]
\newtheorem{theorem}{Theorem}
\newtheorem{definition}{Definition}
\newtheorem{corollary}{Corollary}


\def\var{\mathrm{var}}
\def\atan{\mathrm{ArcTan}}

\markboth{Fricker and Guillemin and Robert and Thompson}{Offloading in fog computing}

\title[Offloading scheme for data centers]{Analysis of an offloading scheme for data centers  in the framework of fog computing}

\author[C. Fricker]{Christine Fricker}
\author[F. Guillemin]{Fabrice Guillemin}
\address[F. Guillemin]{CNC/NCA Orange Labs2, Avenue Pierre Marzin, 22300 Lannion, France}
\email{Fabrice.Guillemin@orange.com}
\author[Ph. Robert]{Philippe Robert}
\author[G. Thompson]{Guilherme Thompson}\thanks{ G. Thompson's research was supported by Brazilian Government/CAPES grant BEX 13748-13-0}
\address[C. Fricker, Ph. Robert, G. Thompson]{INRIA Paris, 2 rue Simone Iff, CS 42112,
75589 Paris Cedex 12, France}
\email{Philippe.Robert@inria.fr}
\urladdr{http://www-rocq.inria.fr/\~{}robert}


\begin{document}

\begin{abstract}
{In the context of fog computing, we consider a simple case when data centers are installed at the edge of the network and assume that if a request arrives at an overloaded data center, then it is forwarded to a neighboring data center with some probability. Data centers are assumed to have a large number of servers and that  traffic at some of them is causing saturation. In this case the other data centers may help to cope with this saturation regime by  accepting some of the rejected requests. Our aim  is to qualitatively estimate the gain achieved via cooperation between  neighboring data centers. After proving some convergence results, related to the scaling limits of loss systems, for the process describing the number of free servers at both data centers, we show that the performance of the system can be expressed in terms of the invariant distribution of a random walk in the quarter plane. By using and developing existing results in the technical literature, explicit formulas for the blocking rates of such a system are derived. }
\end{abstract}

\maketitle

\section{Introduction}

Cloud computing has become one of the major stakes in the development of information technology by offering the possibility of reserving computing resources online. Commercial offers already exist for customers (residential or business) relying on big data centers like Amazon \nocite{AmazonEC2} or Azure \nocite{Azure} for example. This kind of technology is also relevant for network operators in the framework of network function virtualization, where network functions can be instantiated on data centers instead of dedicated hardware. In this context, there is currently a clear trend to distribute data centers. For network operators, it is possible to instantiate  at the edge of the network  functions which were so far centralized in servers (e.g., mobile core functions). Furthermore, by allocating resources closer to end users, it is expected to offer better quality of experience. Distributing cloud computing resources at the edge of the network is known as fog computing. See~\cite{Bonomi,Shenker,Wood} and~\cite{Rai}.

{Data centers involved in fog computing have  a smaller capacity than those in the case of cloud computing and therefore more subject to congestion. Hence, to reduce the probability of request blocking, fog computing data centers have to collaborate. For instance, when one request cannot be accommodated by one of them, it may be forwarded to another one.}

{A typical example of such a situation is when data centers are located on a logical ring at the edge of the network. See Figure~\ref{FogFig}. A request arriving in an overloaded data center with index ,  may be forwarded to a neighboring data center  or  with some probability.  Hence,  if the  traffic to a data center is causing saturation, the other data centers may help alleviate this saturation regime. The aim of this paper is of investigating the impact of such a cooperative scheme. In practice, the network could be backed up by a central (bigger) data center at the core of the network but at a price in terms of latency. We will not consider this additional feature here.}

\begin{figure}[ht]
\centering 
	\scalebox{0.3}{\includegraphics{fog.pdf}}
\put(-150,120){Data Centers}
\put(-175,100){}
\put(-148,90){}
\put(-120,85){}
\put(-180,20){\textcolor{red}{}}
\put(-200,35){\textcolor{red}{}}
\put(-140,15){\textcolor{red}{}}
\caption{A Fog Computing Architecture}\label{FogFig}
\end{figure}

\subsection*{Collaboration of Two Data Centers}
Our aim here is to qualitatively estimate the gain achieved by the collaboration of data centers at the edge of the network. The main part of our analysis will concern the impact of the collaboration of two data centers. It is shown in Section~\ref{ExtSec} that the analysis applies also to more general architectures of fog computing, as in Figure~\ref{FogFig}, provided they are not  congested. 

For , the external arrival process of  requests to data center/facility \#,  referred to as class  requests,  is Poisson with parameter . If one of the  servers is idle upon arrival, then the request is processed by this data center. Otherwise, if the data center is saturated, i.e., all the  servers are busy, then with probability  the request is forwarded to the other data center if it is not saturated too, otherwise with probability  the request is rejected. A request allocated at data center \# is processed at rate . 

By considering the number of requests processed at both data centers, this scheme can be clearly represented by a two dimensional Markov process on . This Markov process, related to loss networks, is not reversible in general and its invariant distribution {does not  have a product form expression.  Even if a numerical analysis of the equilibrium equations is always possible, it is very likely that it will not give precise qualitative and quantitative results concerning  the impact of  rerouting parameters  and  of the offloading scheme. Our goal is of giving {\em explicit} closed form expressions of the equilibrium probability that a request is rejected, see Theorem~\ref{TheoLoss} which is our main result in this domain. }

To overcome the difficulty of not having an explicit expression of the equilibrium, we have chosen to study a scaled version of this network. The input rates ,  and the capacities ,  are assumed to be proportional to a large parameter  which goes to infinity. {This scaling has been  introduced by Kelly in the context of loss networks, see~\cite{Kelly}. As it will be seen, there is a relation  between the parameters (see Condition~(E) below), which implies that both data centers can be saturated with positive probability. We will focus mainly on this case which is, in our view, the most interesting situation to assess the benefit of offloading mechanisms in a congested environment. Otherwise, the situation is much  simpler.  One of the data centers will be underloaded, so that the rejection rate at equilibrium will  converge to  as  gets large, in particular external arrivals to this data center and the rerouted jobs from the other data center will be accepted with probability  in the limit. See Proposition~\ref{P1} and Theorem~\ref{T1}. }

In this limiting regime we prove convergence results for the process describing the number of free servers at both data centers in the same way as in ~\cite{Hunt} for loss networks. {We show that the invariant distribution of a random walk in the quarter plane is playing a key role in the asymptotic behavior of the loss probabilities at equilibrium. The derivation of the equilibrium is based on the analysis of random walks in   by~\cite{FayolleIas}.  By taking advantage of the specific characteristics of the random walks considered, we are able to get an explicit expression of the generating function of their invariant distributions in terms of elliptic integrals instead of contour integrals in the complex plane as in~\cite{FayolleIas}. See Theorem~\ref{TheoLoss}. With these results we can then  assess quantitatively the interest of this load balancing mechanism by comparing the respective loss probabilities of the two streams of requests.}

The organization of this paper is as follows: in Section~\ref{model} the stochastic model is introduced and the limit results for the scaling regime are obtained. A family of random walks is shown to play a central role. Section~\ref{LimitRW} establishes the functional relation satisfied by the generating function of the invariant measure of one of these random walks. Section~\ref{BoundValue} gives an explicit representation of this generating function in Theorem~\ref{TheoLoss} and therefore of the performance metrics of the load balancing mechanism. Section~\ref{App} presents some numerical examples of these results. Concluding remarks are presented in Section~\ref{conclusion}.

\section{Model description}\label{model}

\subsection{Model} We consider in this paper two processing facilities in parallel. The first one is equipped with  servers and serve requests (for computing resources) arriving according to a Poisson process with rate ; each request requires an exponentially distributed service time with mean  (a request if accepted occupies a single server). Similarly, the second processing facility is equipped with  servers and serves service requests arriving according to  a Poisson process with rate ; service times are exponentially distributed with mean . 


To reduce the blocking probability, we assume that requests arriving at a service facility with no available servers are forwarded to the other one with a given probability. More precisely, if a request arrives at service facility \#1 with no available servers, the request is forwarded to the other service facility with probability . Similarly, a request arriving at facility \#2 with no available servers is forwarded to the other facility with probability . See Figure~\ref{fignet}.
\begin{figure}[h]
\centering
	\scalebox{0.35}{\includegraphics{net.pdf}}
	\put(-165,137){\#}
	\put(-165,60){}
	\put(-230,90){}
	\put(-150,35){}
	\put(-210,30){}	
	\put(-90,120){}
	\put(-162,113){}
	\put(-260,60){if \# saturated}
	\put(-92,150){\#}		
	\put(-92,4){}
	\put(10,100){}
	\put(-120,145){}
	\put(-50,155){}
	\put(-90,107){}
	\put(-162,100){}
	\put(-50,125){if \# saturated}
	\caption{Load Balancing between Two Data Centers}\label{fignet}
\end{figure}


Let  and  denote the number of occupied servers in facilities \#1 and \#2 at time , respectively. Owing to the Poisson and exponential service time assumptions,  is a Markov process with values in the set , and  
 transitions from  to  occurring at rate

and  otherwise. 

The equilibrium characteristics of this Markov process on a finite state space, like loss probabilities, do not seem to have closed form expressions in general. A scaling approach is used in the following to get some insight on the performance of such a strategy. We first introduce a random walk in . 

\subsection{A random walk in the extended positive quadrant}
We now consider the following random walk in the extended positive quadrant.

\begin{definition}\label{defim}
For fixed , one defines the random walk  on  as follows:  the transition from  to  occurs at rate

for  with the convention that  for  (see Figure~\ref{fig_transitions_ml}).
\end{definition}

\begin{figure}
\centering
\scalebox{0.8}{\begin{tikzpicture}
\draw[->]
  (0,0) -- (7,0) node[below] {};
\draw[->]
  (0,0) -- (0,7) node[left] {};

\node at (-.2,-.2) {};


\draw[->, thick]
	(0,5) -- ++	(0,1)	node[right]	{};
\draw[->, thick]
	(0,5) -- ++	(0,-1)	node[right]	{};  
\draw[->, thick]
	(0,5) -- ++	(1,0)	node[right]	{};  

\draw[->, thick]
	(5,0) -- ++	(1,0)	node[above]	{};
\draw[->, thick]
	(5,0) -- ++	(-1,0)	node[above]	{};  
\draw[->, thick]
	(5,0) -- ++	(0,1)	node[above]	{};  

\draw[->, thick]
	(0,0) -- ++	(1,0)	node[above]	{};
\draw[->, thick]
	(0,0) -- ++	(0,1)	node[right]	{};  

\draw[<->, thick]
	(3,4)	node[left]	{} --	(5,4)	node[right]	{};
\draw[<->, thick]
	(4,3)	node[below]	{} --	(4,5)	node[above]	{};
\end{tikzpicture}}
\caption{Transitions for .}
       \label{fig_transitions_ml}
\end{figure}

In particular  is an absorbing point for the process . 
The random walk  is a special case of the Markov process investigated in~\cite{FayolleIas}. 

The following result summarizes the stability properties of this random walk. Critical cases are omitted. 
\begin{proposition}\label{P1}
For ,
\begin{enumerate}[(i)]
\item If one of the conditions 

holds then the Markov process  is ergodic on . In this case the unique invariant distribution on  is denoted by . 

\item If ,
the unique invariant distribution of  on the extended state space  is the Dirac measure .
\item If , the unique invariant distribution of  on  is , where  is the geometric distribution with parameter . 
\item If , the unique invariant distribution of  on  is , where  is the geometric distribution with parameter .
\end{enumerate}
\end{proposition}

\begin{proof}
Due to~\cite{FayolleIas}, see also~\cite[Proposition~9.15]{Robert},  is ergodic if and only if one of the conditions of ()  holds. {As long as it does not hit , the first (resp. second)  coordinate of  behaves as an  queue with arrival rate  (resp. )  and service rate  (resp.  ). Under the conditions of~({\em ii}), each of these   queues is transient, in particular starting from , it has a positive probability of not returning to . This implies that after some random time, the process  stays in the interior of the quadrant  and therefore  behaves as a couple of independent transient  queues. Consequently, both coordinates of  are converging in distribution to .} Similarly, for ({\em iii}) and ({\em iv}), the process  can be coupled to two queues, the first one, an  queue which is transient and the second one, an ergodic  queue, with an invariant distribution which is geometrically distributed. 
\end{proof}

\subsection{Heavy Traffic Scaling Regime}
We investigate now the case when some of the parameters of the processing facilities are scaled up by a factor . The arrival rates are given by  and  with  and . Similarly the capacities are given by  and  for some positive constants  and . To indicate the dependence of the numbers of idle servers upon , an upper index  is added to the stochastic processes. A similar approach has been used in~\cite{Hajek} to study a load balancing scheme in an Erlang system. 

We will consider the process 

describing the number of idle servers in both processing facilities. As it will be seen, the random walks , ,  play an important role in the asymptotic behavior of  as  goes to infinity.

\begin{theorem}\label{T1}
If one of the following conditions

holds, and if the initial conditions are such that  and

then, for the convergence in distribution,

for any function  with finite support on ,  is the invariant distribution of the process  defined previously.
\end{theorem}

\begin{proof}
By using the same method as in~\cite{Hunt}, one gets an analogous result to Theorem~3 of this reference. For the convergence in distribution of processes, the relation

holds, where  satisfying the following integral equations

for ,  and, for ,  is the {\em unique } invariant distribution of .

Let us assume without loss of generality  that, under condition~(E), for example the first condition of~(E) is satisfied. It will be assumed throughout the paper. It is not difficult to construct a coupling so  holds almost surely for all , where  is the number of jobs of an  queue with arrival rate  and service rate . Since , a classical result, see Section~7 of Chapter~6 of~\cite{Robert} for example, gives the convergence in distribution  in particular,  is a constant equal to . 

If, for ,  is the Poisson process of arrivals at facility , by using the same coupling as before one gets that the number , arrivals at facility \#2 up to time , satisfies, for , , 
 where  is a family of independent Bernoulli random variables with parameter . The lower bound includes the direct arrivals  to facility  and the rejected jobs from . One gets that  where  is a  queue with the arrival process .

The ergodic theorem gives that, almost surely

by condition~(E). By using this relation, one can show that, for the convergence in distribution, the relation  holds. In particular  is constant and equal to . Therefore, almost surely,  holds, hence . Relation~\eqref{hkcv} shows that the theorem is proven.
\end{proof} 

The following proposition states that the performances of the load balancing mechanism can be expressed with the invariant distribution .

\begin{proposition}
Under Condition~(E), as  goes to infinity, the probability that at equilibrium a job of class  is rejected converges to 
where  is the invariant distribution of .
\end{proposition}

\begin{proof}
Assume that  is at equilibrium, the number of class~ jobs rejected between  and  is given by

The probability of rejecting a class~1 job is hence given by 


By using the martingales associated with Poisson processes, one gets

one concludes with the convergence of the previous theorem. 
\end{proof}
When condition~(E) is not satisfied, one can obtain an analogous result. Its (elementary) proof is skipped. 
The results on the asymptotic blocking probability of jobs are summarized in the following proposition, where ,  and  are exclusive.
\begin{proposition}\label{othercases}
Let

then, at equilibrium, the loss probability of a job of class  is converging to  as  goes to infinity, where

\end{proposition}

{
\subsection{An Extension to Multiple Data Centers}\label{ExtSec}
In this section, it is assumed that there are  data centers, for , the th data center has  servers and the external arrivals to it are Poisson with parameter  and services are exponentially distributed with parameter . If an external request at data center  finds all  servers occupied, it is re-routed to data center  (with )  or to data center  (with ) with probability , otherwise it is rejected. In particular a job is rerouted with probability  in the case of congestion.  See Figure~\ref{FogFig}.

For  , one defines the random walk  on  as follows: the transition from  to  occurs at rate

If one of the conditions 

holds, one gets that the associated Markov process is ergodic by Proposition~\ref{P1}, one denotes by  its invariant probability distribution. 
As before, one takes the following convention for the indices,  and .

We now give a version of the previous proposition in this context. 
\begin{proposition}\label{propExt}
At equilibrium, the loss probability of a job of class  is converging to  as  goes to infinity in the following cases,
\begin{enumerate}
\item No Congestion.\\
If  for all  then .
\item One saturated node.\\
If, for some ,   and if  holds for all  and 

then  if  and 
\item Two saturated neighboring nodes.\\
If, for some ,  one of the conditions~() holds and  holds for all  and

then

\end{enumerate}
\end{proposition}
The proof is similar to the proof of the previous proposition and is therefore omitted. Note that Condition~\eqref{F1} implies that the nodes with index  and  are underloaded, so that only nodes with index  and  are congested.  This result covers  partially the set of various possibilities but, as long as only two neighboring nodes are congested, it can be extended quite easily to the case where only pairs of nodes are congested. 

When there are at least three neighboring congested nodes, this method does not apply. It occurs when one of the conditions~() holds and one of the conditions of~\eqref{F1} is not satisfied. One has to consider the invariant distributions of a three dimensional random walk in  for which there are scarce results. Nevertheless this situation should be, in practice, unlikely if the fog computing architecture is conveniently designed so that a local congestion can be solved by using the neighboring resources.

This proposition shows that the evaluation of the performances of the offloading algorithm can be expressed in terms of the invariant distributions of the random walks  introduced in Definition~\ref{defim}. The rest of the paper is devoted to the analysis of these invariant distributions when they exist. In particular, we will derive an explicit expression of  the blocking probabilities  at facility \# .}

\section{Characteristics of the limiting random walk}\label{LimitRW}

\subsection{Fundamental equations}
Throughout this section, we assume that the first condition of~(E) holds. Let  and  denote the abscissa and the ordinate of the random walk  in the stationary regime. Under stability condition~(E), it is shown in \cite{FayolleIas} that the generating function of the stationary numbers  and , defined by  for complex  and  such that  and , satisfies the functional equation

with  standing for  and

It is worth noting that 


In \cite{FayolleIas,FIM}, it is shown how to compute the unknown functions by using the zeros of the kernel  and the results on Riemann-Hilbert problems. In the following we briefly describe how to achieve this goal. For the system under consideration, let us recall that the performance of the system is characterized by the blocking probabilities of the two classes of customers. For customers arriving at facility \#1, the blocking probability is given by

and that for customers arriving at the second facility by


By using the normalizing condition , we can easily show that  and 

We then deduce that 

The above relations show that the blocking probabilities  and  can be estimated as soon as the quantity  is known.

\subsection{Zero pairs of the kernel}

The kernel  has already been studied in \cite{FayolleIas} in the framework of coupled servers. For fixed , the kernel  has two roots  and . By using the usual definition of the square root such that  for , the solution which is null at the origin and denoted by , is defined and analytic in  where the reals  and  are such that . The other solution  is meromorphic in  with a pole at 0. The function  is precisely defined by
 with 
where  is the analytic continuation in  of the function  defined in the neighborhood of 0. The other solution .

When  crosses the segment ,  and  describe the circle  with center 0 and radius , since from the first of condition of~(E), we have .

Similarly, for fixed , the kernel  has two roots  and . The root , which is null at the origin, is analytic in  where the reals , ,  and  are such that with  and is given by

with

where  is the analytic continuation in   of the function  defined in the neighborhood of 0.  The other root  and is meromorphic in  with a pole at the origin. 

When  crosses the segment ,  and  describe the circle  with center 0 and radius .

\section{Boundary value problems}\label{BoundValue}

\subsection{Problem formulation}

In \cite{FIM}, it is proven that the functions  and  can be extended as meromorphic functions in  and , respectively. By using the fact that  and  are on circle  for , we easily deduce that the function , analytic in  (the disk with center 0 and radius ), is such that for  

where .


Similarly, the function  is analytic in , which is the disk with center 0 and radius , and for , we have


By using Equation~\eqref{h2h3h4}, we have

Equation~\eqref{RHPy} can then be rewritten as


Similarly, Equation~\eqref{RHPx} can be rewritten as


Problem~\eqref{RHPxb} corresponds to Problem~(7.6) in \cite{FayolleIas} for which  in the notation of that paper. The ratio  corresponds to the function  in \cite{FayolleIas}.

In the following, we focus on Riemann-Hilbert problem~\eqref{RHPyb}. The analysis of problem~\eqref{RHPxb} is completely symmetrical. Moreover, to compute the blocking probabilities  and , we only need to compute the quantity .

\subsection{Problem resolution}

The function  is analytic in the open disk . By using the reflection principle \cite{Cartan}, the function 

is analytic on the outside of the closed disk . It is then easily checked that if we define 

the function  is sectionally analytic with respect to the circle ,  the quantity   tends to  when  goes to infinity, and for 

where  (resp. ) is the interior (resp. exterior) limit of the function  at the circle , and the function  is defined on  by

with


The solutions to Riemann-Hilbert problems of form~\eqref{Hilberty} are given in \cite{Lions}. We first have to determine the index of the problem defined as

In \cite[Theorem~7.2]{FayolleIas} it is shown that the stability condition~(E) is equivalent to .

To obtain explicit expressions, let us first study the function , which can be expressed as follows. 

\begin{lemma}
The function  defined for  by Equation~\eqref{defalphayini} can be extended as a meromorphic function in  by setting

where

\end{lemma}


\begin{proof}
We have for  such that  and 

By using the fact that  and , we deduce that 

where  is defined by Equation~\eqref{defRY}, and the result follows.
\end{proof}

Since the index of the Riemann-Hilbert~\eqref{Hilberty} is null, the solution is as follows.

\begin{lemma}
\label{lemHilbert}
The solution to the Riemann-Hilbert problem~\eqref{Hilberty} exists and is unique and given for  by

where

and

\end{lemma}

\begin{proof}
Since the index of the Riemann-Hilbert~\eqref{Hilberty} is null, the solution reads \cite{Lions}

where the function  is defined by Equation~\eqref{defalphaY} and  is a polynomial. Since we know that  as , then


Let for  and  for 

By using the expression of , Equation~\eqref{defThetaY} follows. It is clear that 



Since , we have

It is easily checked from the equation  that

For , we have

By using once again , we obtain for 

and then for real 

It follows that for real 

It is easily checked that the function on the right hand side of the above equation can analytically be continued in the disk  and the result follows. \end{proof}

In view of the above lemma, we can state the main result of this section.

\begin{theorem}
The function  can be defined as a meromorphic function in  by setting 

where  is defined by Equation~\eqref{defvarphiy}.
\end{theorem}

\begin{proof}Since the solution to the Riemann-Hilbert problem~\eqref{RHPyb} is unique, the function  coincides with the function  in . We can extend this function as follows. Noting that the function  is analytic in a neighborhood of the circle , the function 

defined for  can be continued as a meromorphic function in  by considering the function defined for  by

where the last equality is obtained by using the same arguments as above (consider first real  and then extend the function by analytic continuation).
\end{proof}

For the system under consideration, let us recall that the performance of the system is characterized by the blocking probabilities of the two classes of customers. The following theorem summarizes the main results of the paper for Condition~(E). Proposition~\ref{othercases} covers the other cases. 
\begin{theorem}\label{TheoLoss}
Under Condition~(E), as  goes to infinity, the probability that at equilibrium a job of facility \#,  is rejected converges to  with 

and the quantity  is given by

where  is defined by Equation~\eqref{defvarphiy}.
\end{theorem}
\begin{proof}
In the case , the result easily follows by using Equation~\eqref{defP0y} for  and Equation~\eqref{P01}.

In the case  (and then  by Condition~(E)), we have  and then, by Relation~\eqref{defalphaY}, one gets the expression for ,

Equation~\eqref{P00} then easily follows. The formulas for the blocking probabilities are obtained by using Relations~\eqref{defbeta1} and~\eqref{defbeta2} for  and  and the expressions~\eqref{P01} and~\eqref{P10} for  and . 
\end{proof}

To conclude this section, it is worth noting that the computation of the function  in the quantity  involves elliptic integrals. In addition, a similar result holds for the function . This enables us to completely compute the generating function .

\section{Numerical results: Offloading small data centers}\label{App}

In this section, we illustrate the results obtained in the previous sections (in particular Theorem~\ref{TheoLoss}) in order to estimate the gain achieved by the offloading scheme. We assume that the service rate at  both facilities is the same and taken equal to unity (). Assume in addition that the first data center has a capacity much smaller than the second one, e.g.,  and .

We consider the case when all the requests blocked at the first data center are forwarded to the second one () and none blocked at the second data center is forwarded to the first one (). 

In Figures~\ref{ffig1} and~\ref{ffig2}, when the arrival rate  at the first data center increases, the loss rate  goes from  if  or  holds to a positive value if  or   holds. For example in Figure~\ref{ffig1}, for , we can see the transition from  to  when , and for , the transition from  to  when . We can checked that  is a continuous and not differentiable function of  at . If  or ,  holds for the range of values  considered here for .
In Figure~\ref{ffig1},  thus  holds for , as  and .

In conclusion, Figures~\ref{ffig1} and~\ref{ffig2} show that the offloading mechanism improves a lot the loss rate  of the requests of class  and does not significantly deteriorate the corresponding performances at facility \# in the case of systematic rerouting (), even when this data center is already significantly loaded as in Figure~\ref{ffig2}~(B). This means that offloading small date centers with a big back-up data center is a good strategy to reduce blocking in fog computing.



\begin{figure} [htbp]
        \begin{subfigure}[b]{0.46\textwidth}
                \begin{tikzpicture}
                 \begin{axis}[legend style={draw=none},
                                    font=\tiny,
                                       label style={font=\tiny},
      			           xlabel={},
			           xlabel style={yshift=+9pt},		
                                       ylabel= {},
                                       ylabel style={yshift=-16pt},
                                       xmin=1, xmax=5, ymin=0.0, ymax=0.8,
                                      width=\linewidth,height=\linewidth,legend pos= north west]
                \addplot[thick, smooth,red] table[x=lbd1, y=beta1_p1_0.00] {ffig1.dat};
\addlegendentry{}
\addplot[thick, smooth,green] table[x=lbd1, y=beta1_p1_0.35] {ffig1.dat};
\addlegendentry{}
\addplot[thick, smooth,blue] table[x=lbd1, y=beta1_p1_0.70] {ffig1.dat};
\addlegendentry{}
\addplot[thick, smooth,pink] table[x=lbd1, y=beta1_p1_1.00] {ffig1.dat};
\addlegendentry{}
\addplot[only marks,black,mark=x] table[x=lbd1, y=simu_beta1_p1_0.00] {ffig1_simu.dat};
\addplot[only marks,black,mark=x] table[x=lbd1, y=simu_beta1_p1_0.35] {ffig1_simu.dat};
\addplot[only marks,black,mark=x] table[x=lbd1, y=simu_beta1_p1_0.70] {ffig1_simu.dat};
\addplot[only marks,black,mark=x] table[x=lbd1, y=simu_beta1_p1_1.00] {ffig1_simu.dat};
                \end{axis}
      \end{tikzpicture}
       \hspace{0.05\linewidth}
       \caption{ Loss probability of class }
       \label{ffig1_a}
        \end{subfigure}
   \quad
\begin{subfigure}[b]{0.46\textwidth}
                \begin{tikzpicture}
                 \begin{axis}[legend style={draw=none},
                                     font=\tiny,
                                       label style={font=\tiny},
      			           xlabel={},
			           xlabel style={yshift=+9pt},		
                                       ylabel= {},
                                       ylabel style={yshift=-16pt},
                                       xmin=1, xmax=5, ymin=0., ymax=0.25,
                                      width=\linewidth,height=\linewidth,legend pos=  north west]
                \addplot[thick, smooth,red] table[x=lbd1, y=beta2_p1_0.00] {ffig1.dat};\addlegendentry{}
\addplot[thick, smooth,green] table[x=lbd1, y=beta2_p1_0.35] {ffig1.dat};
\addlegendentry{}
\addplot[thick, smooth,blue] table[x=lbd1, y=beta2_p1_0.70] {ffig1.dat};
\addlegendentry{}
\addplot[thick, smooth,pink] table[x=lbd1, y=beta2_p1_1.00] {ffig1.dat};
\addlegendentry{}
\addplot[only marks,black,mark=x] table[x=lbd1, y=simu_beta2_p1_0.00] {ffig1_simu.dat};
\addplot[only marks,black,mark=x] table[x=lbd1, y=simu_beta2_p1_0.35] {ffig1_simu.dat};
\addplot[only marks,black,mark=x] table[x=lbd1, y=simu_beta2_p1_0.70] {ffig1_simu.dat};
\addplot[only marks,black,mark=x] table[x=lbd1, y=simu_beta2_p1_1.00] {ffig1_simu.dat};
                \end{axis}
      \end{tikzpicture}
       \hspace{0.05\linewidth}
       \caption{Loss probability of class }\label{ffig1_b}
       \end{subfigure}
\caption{Loss probabilities as a function of  with , , ,  , , . The crosses represent simulation points while solid curves are plotted from analytical results.}\label{ffig1}
\end{figure}
\begin{figure} [htbp]
   \centering      
        \begin{subfigure}[b]{0.46\textwidth}
                \centering
                \begin{tikzpicture}
                 \begin{axis}[legend style={draw=none},
                                     font=\tiny,
                                       label style={font=\tiny},
      			           xlabel={},
			           xlabel style={yshift=+9pt},		
                                       ylabel= {},
                                       ylabel style={yshift=-16pt},
                                       xmin=1, xmax=5, ymin=0.0, ymax=0.8,
                                      width=\linewidth,height=\linewidth,legend pos=  north west]
                \addplot[thick, smooth,red] table[x=lbd1, y=beta1_p1_0.00] {ffig2.dat};
\addlegendentry{}
\addplot[thick, smooth,green] table[x=lbd1, y=beta1_p1_0.35] {ffig2.dat};
\addlegendentry{}
\addplot[thick, smooth,blue] table[x=lbd1, y=beta1_p1_0.70] {ffig2.dat};
\addlegendentry{}
\addplot[thick, smooth,pink] table[x=lbd1, y=beta1_p1_1.00] {ffig2.dat};
\addlegendentry{}
 \addplot[only marks,black,mark=x] table[x=lbd1, y=simu_beta1_p1_0.00] {ffig2_simu.dat};
\addplot[only marks,black,mark=x] table[x=lbd1, y=simu_beta1_p1_0.35] {ffig2_simu.dat};
\addplot[only marks,black,mark=x] table[x=lbd1, y=simu_beta1_p1_0.70] {ffig2_simu.dat};
\addplot[only marks,black,mark=x] table[x=lbd1, y=simu_beta1_p1_1.00] {ffig2_simu.dat};
\end{axis}
      \end{tikzpicture}
       \hspace{0.05\linewidth}
       \caption{ Loss probability of class }
       \label{ffig2_a}
        \end{subfigure}
   \quad
\begin{subfigure}[b]{0.46\textwidth}
                \centering
                \begin{tikzpicture}
                 \begin{axis}[legend style={draw=none},
                                       font=\tiny,
                                       label style={font=\tiny},
      			           xlabel={},
			           xlabel style={yshift=+9pt},		
                                       ylabel= {},
                                       ylabel style={yshift=-16pt},
                                       xmin=1, xmax=5, ymin=0.00, ymax=0.4,
                                      width=\linewidth,height=\linewidth,legend pos= north west]
                \addplot[thick, smooth,red] table[x=lbd1, y=beta2_p1_0.00] {ffig2.dat};\addlegendentry{}
\addplot[thick, smooth,green] table[x=lbd1, y=beta2_p1_0.35] {ffig2.dat};
\addlegendentry{}
\addplot[thick, smooth,blue] table[x=lbd1, y=beta2_p1_0.70] {ffig2.dat};
\addlegendentry{}
\addplot[thick, smooth,pink] table[x=lbd1, y=beta2_p1_1.00] {ffig2.dat};
\addlegendentry{}
\addplot[only marks,black,mark=x] table[x=lbd1, y=simu_beta2_p1_0.00] {ffig2_simu.dat};
\addplot[only marks,black,mark=x] table[x=lbd1, y=simu_beta2_p1_0.35] {ffig2_simu.dat};
\addplot[only marks,black,mark=x] table[x=lbd1, y=simu_beta2_p1_0.70] {ffig2_simu.dat};
\addplot[only marks,black,mark=x] table[x=lbd1, y=simu_beta2_p1_1.00] {ffig2_simu.dat};
\end{axis}
      \end{tikzpicture}
       \hspace{0.05\linewidth}
       \caption{Loss probability of class }\label{ffig2_b}
       \end{subfigure}
\caption{Loss probabilities as a function of  with , , ,  , , . The crosses represent simulation points while solid curves are plotted from analytical results.}\label{ffig2}
\end{figure}
\begin{figure} [htbp]
   \centering      
        \begin{subfigure}[b]{0.46\textwidth}
                \centering
                \begin{tikzpicture}
                 \begin{axis}[legend style={draw=none},
                                  font=\tiny,
                                       label style={font=\tiny},
      			           xlabel={},
			           xlabel style={yshift=+9pt},		
                                       ylabel= {},
                                       ylabel style={yshift=-16pt},
                                       xmin=0, xmax=1, ymin=0.0, ymax=0.18,
                                      width=\linewidth,height=\linewidth,legend pos= north east]
                \addplot[thick, smooth,red] table[x=p1, y=beta1_lbd2_9.90] {ffig3.dat};
\addlegendentry{}
\addplot[thick, smooth,green] table[x=p1, y=beta1_lbd2_11.00] {ffig3.dat};
\addlegendentry{}
                \end{axis}
      \end{tikzpicture}
       \hspace{0.05\linewidth}
       \caption{ Loss probability of class }
       \label{ffig3_a}
        \end{subfigure}
   \quad
\begin{subfigure}[b]{0.46\textwidth}
                \centering
                \begin{tikzpicture}
                 \begin{axis}[legend style={draw=none},
                                 font=\tiny,
                                       label style={font=\tiny},
      			           xlabel={},
			           xlabel style={yshift=+9pt},		
                                       ylabel= {},
                                       ylabel style={yshift=-16pt},
                                       xmin=0, xmax=1, ymin=0., ymax=0.18,
                                      width=\linewidth,height=\linewidth,legend pos= north west]
                \addplot[thick, smooth,green] table[x=p1, y=beta2_lbd2_9.90] {ffig3.dat};\addlegendentry{}
\addplot[thick, smooth,red] table[x=p1, y=beta2_lbd2_11.00] {ffig3.dat};
\addlegendentry{}
                \end{axis}
      \end{tikzpicture}
       \hspace{0.05\linewidth}
       \caption{Loss probability of class }\label{ffig3_b}
       \end{subfigure}
\caption{Loss probabilities as a function of  with , , ,  , , . }\label{ffig3}
\end{figure}


Figures~\ref{ffig3} illustrate the impact of the choice of  when facility \# is almost overloaded,  so that , and with a high load . As it can be seen, even when , the performances of class  requests are not really impacted by the offloading scheme, whereas the loss rate of class  is significantly changed. This confirms the benefit of the offloading strategy. 

\section{Conclusion}
\label{conclusion}

We have proposed in this paper an analytical model to study a simple offloading strategy for data centers in the framework of fog computing under heavy loads. The strategy considered consists of forwarding with a certain probability requests blocked at a small data center to a big back-up data center. The model considered could also be used to study the offload of requests blocked at the big data center onto a small data center but this case has not been considered in the numerical applications. The key finding is that the proposed strategy can significantly improves blocking at a small data center without affecting too much blocking at the big data center. 

\providecommand{\bysame}{\leavevmode\hbox to3em{\hrulefill}\thinspace}
\providecommand{\MR}{\relax\ifhmode\unskip\space\fi MR }
\providecommand{\MRhref}[2]{\href{http://www.ams.org/mathscinet-getitem?mr=#1}{#2}
}
\providecommand{\href}[2]{#2}
\begin{thebibliography}{10}

\bibitem{AmazonEC2}
\emph{Amazon {EC2}}, \url{http://aws.amazon.com/ec2/}.

\bibitem{Azure}
\emph{Microsoft {A}zure}, \url{http://www.microsoft.com/windowsazure/}.

\bibitem{Hajek}
Murat Alanyali and Bruce Hajek, \emph{Analysis of simple algorithms for dynamic
  load balancing}, Mathematics of Operations Research \textbf{22} (1997),
  no.~4, 840--871.

\bibitem{Bonomi}
Flavio Bonomi, Rodolfo Milito, Jiang Zhu, and Sateesh Addepalli, \emph{Fog
  computing and its role in the internet of things}, Proceedings of the First
  Edition of the MCC Workshop on Mobile Cloud Computing (New York, NY, USA),
  MCC '12, ACM, 2012, pp.~13--16.

\bibitem{Cartan}
H.~Cartan, \emph{Elementary theory of one or several complex variables}, Dover
  Publications, 1950.

\bibitem{Lions}
R.~Dautray and J.L. Lions, \emph{Analyse math\'ematique et calcul num\'erique
  pour les sciences et les techniques}, Masson, 1985.

\bibitem{FayolleIas}
G.~Fayolle and R.~Iasnogorodski, \emph{Two coupled processors: {T}he reduction
  to a {R}iemann-hilbert problem}, Z. Wahrscheinlichkeitstheorie verw. Gebiete
  \textbf{47} (1979), 325 -- 351.

\bibitem{FIM}
G.~Fayolle, R.~Iasnogorodski, and V.~Malyshev, \emph{Random walks in the
  quarter-plane. algebraic methods, boundary value problems and applications},
  Applications of Mathematics, vol.~40, Springer-Verlag, 1999.

\bibitem{Shenker}
A.~Ghodsi, M.~Zaharia, B.~Hindman, A.~Konwinski, S.~Shenker, and I.~Stoica,
  \emph{Dominant resource fairness: Fair allocation of multiple resources in
  datacenters}, Proceedings of the 8th USENIX Symposium on Networked Systems
  Design and Implementation (NSDI 2011), 2011, pp.~323--336.

\bibitem{Hunt}
P.J. Hunt and T.G Kurtz, \emph{Large loss networks}, Stochastic Processes and
  their Applications \textbf{53} (1994), 363--378.

\bibitem{Kelly}
F.P. Kelly, \emph{Loss networks}, Annals of Applied Probability \textbf{1}
  (1991), no.~3, 319--378.

\bibitem{Rai}
Anshul Rai, Ranjita Bhagwan, and Saikat Guha, \emph{Generalized resource
  allocation for the cloud}, Proceedings of the Third ACM Symposium on Cloud
  Computing (New York, NY, USA), SoCC '12, ACM, 2012, pp.~15:1--15:12.

\bibitem{Robert}
Philippe Robert, \emph{Stochastic networks and queues}, Stochastic Modelling
  and Applied Probability Series, vol.~52, Springer, New-York, June 2003.

\bibitem{Wood}
Timothy Wood, K.~K. Ramakrishnan, Prashant Shenoy, and Jacobus van~der Merwe,
  \emph{Cloudnet: Dynamic pooling of cloud resources by live wan migration of
  virtual machines}, SIGPLAN Not. \textbf{46} (2011), no.~7, 121--132.

\end{thebibliography}

\end{document}
