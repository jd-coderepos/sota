\documentclass[11pt]{llncs}
\usepackage{algorithm2e}
\usepackage{algorithmicx}
\usepackage{epsfig}
\usepackage{fancybox}
\usepackage{graphics}
\usepackage{amssymb}
\usepackage{geometry}



\newcounter{qcounter}
\newgeometry{tmargin=1in, bmargin=1in, lmargin=1in, rmargin=1in}

\begin{document}

\title{-skeletons for a set of line segments in 
\thanks{This research is supported by the ESF EUROCORES program EUROGIGA, CRP VORONOI.}}
\author{
        Miros{\l}aw Kowaluk 
\and
                     Gabriela Majewska 
\institute{
Institute of Informatics, University of Warsaw, Warsaw, Poland, \\
\texttt{kowaluk@mimuw.edu.pl} \texttt{gm248309@students.mimuw.edu.pl} 
}
}
\date{}


\maketitle

\begin{abstract}
-skeletons are well-known neighborhood graphs for a set of points.  
We extend this notion to sets of line segments in the Euclidean plane 
and present algorithms computing such skeletons for the entire range of  values.
The main reason of such extension is the possibility to study -skeletons for points
moving along given line segments.
We show that relations between -skeletons for , -skeleton 
(Gabriel Graph), 
and the Delaunay triangulation for sets of points hold also for sets of segments.
We present algorithms for computing circle-based and lune-based -skeletons. 
We describe an algorithm that for  computes the -skeleton for a set  
of  segments in the Euclidean plane in  time 
in the circle-based case and in  in the lune-based one,
where the construction relies on the Delaunay triangulation for ,  is a functional inverse of Ackermann function and  denotes 
the maximum possible length of a  Davenport-Schinzel sequence.
When , the -skeleton can be constructed in a  time. 
In the special case of , which is a generalization of Gabriel Graph, 
the construction can be carried out in a  time.
\end{abstract}

\section{Introduction}
 
-skeletons in  belong to the family of proximity graphs, 
geometric graphs in which an edge between two vertices (points) exists 
if and only if they satisfy particular geometric requirements. 
In this paper we use the following 
definitions of the -skeletons for sets of points in the Euclidean space (-skeletons are also defined for  but those cases
have no significant influence on our considerations) :


\begin{definition} 
\label{betaskeleton}
For a given set of points   in , a distance 
function   and a parameter  we define a graph 
\begin{itemize}
\item
 -- called a lune-based -skeleton \cite{kr85} -- as follows:  
two points  are connected with an edge if and only if no point 
from  belongs to the set 
 (neighborhood) where:

\begin{enumerate} 
\item 
for ,  is the intersection of two discs, 
each with radius  and having the segment 
as a chord,
\item 
for ,  is the intersection of two  
discs, each with radius , whose centers are in points 
 and 
in , respectively;
\end{enumerate}

\item
 -- called a circle-based -skeleton \cite{e02} -- as follows:  
two points  are connected with an edge if and only if no point 
from  belongs to the set 
 (neighborhood) where:
\begin{enumerate}
\item
for  there is , 
\item
for  the set  is a union of two discs, 
each with radius  and having the segment 
as a chord.
\end{enumerate}
\end{itemize}

\end{definition}

\begin{figure}[htbp]
\centering
\includegraphics[scale=0.3]{region-lune-f.eps}
\caption{Neighborhoods of the -skeleton: (a) for  , (b) the lune-based 
skeleton, (c) the circle-based skeleton for . The relation between 
neighborhoods: (d) .}
\label{fig:region}
\end{figure}



Points  are called generators of the neighborhood 
(, respectively).
 The neighborhood  is called a lune. 
It follows from the definition that .
 
-skeletons are both important and popular because of many practical applications 
which span a spectrum of areas from geographic information systems to wireless ad hoc 
networks and machine learning. 
For example, they allow us to reconstruct a shape of a two-dimensional object from 
a given set of sample points and they are also helpful in finding the minimum 
weight triangulation of a point set. 

Hurtado, Liotta and Meijer \cite{hlm03} presented an  algorithm for
the -skeleton when .
Matula and Sokal \cite{ms84} showed that the lune-based -skeleton (Gabriel Graph
) can be computed from the Delaunay triangulation in a linear time.
Supowit \cite{su83} described how to construct the lune-based -skeleton (Relative 
Neighborhood Graph ) of a set of  points in  time.
Jaromczyk and Kowaluk \cite{jk87} showed how to construct the  
from the Delaunay triangulation  for the  metric  
in  time.
This result was further improved to  time \cite{l94} for -skeletons 
where .
For , the  circle-based -skeletons can be 
constructed in  time from the  Delaunay triangulation  
with a simple test to filter edges of the   \cite{e02}.
On the other hand, so far the fastest algorithm for computing the lune-based 
-skeletons for  runs in  
time \cite{k12}.
 


 Let us consider the case when we compute the -skeleton for a set of  points 
 where every point  is allowed to move along a straight-line segment . 
Let . 
For each pair of segments  containing points , 
respectively, we want to find such positions of points  
and  that 
 for any . 
We will attempt to solve this problem 
by defining a -skeleton for the set of line segments  as follows.


\begin{definition}
 (, respectively) is a graph with  vertices such that there exists a bijection between the set of vertices and the set of segments ,
and for  an edge  exists
if there are points  and  such that 

(, 
respectively).
\end{definition}

Note that when segments degenerate to points, we have the standard -skeleton 
for a point set.

Geometric structures concerning a set of line segments, e.g. the Voronoi diagram 
\cite{bms84,pz13} or the straight skeleton \cite{aa96} are well-studied in the literature.

Chew and Kedem \cite{ck89} defined the Delaunay triangulation for line segments.
Their definition was generalized by Br\'{e}villiers et al. \cite{bcs07}.  

However, -skeletons for a set of line segments were completely unexplored.
This paper makes an initial effort to fill this gap.

The paper is organized as follows.
In the next section  we present some basic facts and we prove that 
the definition of -skeletons for a set of line segments preserves inclusions 
from the theorem of Kirkpatrick and Radke \cite{kr85} formulated for a set of points.
In Section 3 we show a general algorithm computing -skeletons for a set of line 
segments in Euclidean plane when . In Section 4 we present 
a similar algorithm for  in both cases of lune-based and circle-based 
-skeletons. In Section 5 we consider an algorithm for Gabriel Graph. 
The last section contains open problems and conclusions. 


\section{Preliminaries}


Let us consider a two-dimensional plane  with the Euclidean metric 
and a distance function . 


Let  be a finite set of disjoint closed line segments in the plane.
Elements of  are called sites.
A circle is tangent to a site  if  intersects the circle 
but not its interior.
We assume that the sites of  are in general position, i.e., no three segment 
endpoints are collinear and no circle is tangent to four sites.

 The Delaunay triangulation for the set of line segments  
is defined as follows.




\begin{definition} \cite{bcs07}
\label{segment-dt}
 The segment triangulation  of  is a partition of the convex hull  of  
in disjoint sites, edges and faces such that:
\begin{itemize}
\item
Every face of  is an open triangle whose vertices belong to three distinct sites of  
and whose open edges do not intersect ,
\item
No face can be added without intersecting another one,
\item
The edges of  are the (possibly two-dimensional) connected components 
of , where  is the set of faces of .
\end{itemize} 

The segment triangulation  such that the interior of the circumcircle of each triangle 
does not intersect  is called the segment Delaunay triangulation. 

\end{definition}

 In this paper we will consider a planar graph (a planar multigraph, respectively) 
 corresponding to the segment Delaunay triangulation  and its relations 
with -skeletons. This graph has a linear number of edges and is dual to the Voronoi Diagram graph for .
It is also possible to study properties of plane partitions generated by -skeletons 
for line segments. We will discuss this problem in the last section of this paper. 










 We can  consider open (closed, respectively) neighborhoods 
that lead to {\em open} ({\em closed}, respectively) {\em -skeletons}. For example, 
the {\em Gabriel Graph}  \cite{gs69} is the closed -skeleton and the
{\em Relative Neighborhood Graph}  \cite{tou80} is the open -skeleton.

Kirkpatrick and Radke \cite{kr85} showed a following important inclusions connecting 
-skeletons for a set of points  with the Delaunay triangulation 
 of  : 
, 
where . 



We show that definitions of the -skeleton and the Delaunay 
triangulation for a set of line segments  preserve those inclusions. 
We define  as a -skeleton.


\begin{theorem}
\label{luneinkluzja}
Let us assume that line segments in  are in general position and let  
(, respectively) denote the lune-based (circle-based, respectively)
-skeleton for the set .
For  following inclusions hold true: 

(, respectively). 
\end{theorem}
\begin{proof} 
First we prove that .
Let  be such a pair of points that there exists a disc  
with diameter  containing no points belonging to segments 
from  inside of it. 
We transform  under a homothety with respect to  so that its image  is 
tangent to  in the point . Then we transform  under a homothety with respect 
to  so that its image  is tangent to  (see Figure \ref{fig:gg-dt}). 
The disc  lies inside of , i.e., it does not intersect segments 
from , and it is tangent to  and  , so the center of  lies on the Voronoi Diagram  edge. Hence, if the edge  belongs to  then it also belongs 
to . 

The last inclusion is based on a fact that for  and for any two points  it is true that  (see \cite{kr85}).






The sequence of inclusions for circle-based -skeletons is a straightforward 
consequence of the fact that two different circles intersect in at most two points.

\begin{figure}[htbp]
\centering
\includegraphics[scale=0.3]{okregi-f.eps}
\caption{.}
\label{fig:gg-dt}
\end{figure}  

\end{proof}



\section{Algorithm for computing -skeletons for }

Let us consider a set  of  disjoint line segments in the Euclidean plane. 
First we show a few geometrical facts concerning -skeletons 
.



The following remark is a straightforward consequence of the inscribed angle theorem. 

\begin{remark}
\label{lematkat}
For a given parameter  if  is a point on the boundary of  
, different from  and , then an angle  
has a constant measure which depends only on .
\end{remark}





Let us consider a set of parametrized lines containing given segments. 
A line  contains a segment  and has a parametrization 
, 
where  and  are ends of the segment  
and . 

  Let respective points from segments  and  be generators of a lune and let  
an inscribed angle determining a lune for a given value of  be equal to .
The main idea of the algorithm is as follows.
For any point  we compute points  for which 
there exists a point , where ,
such that 
, i.e., 
(see Figure \ref{fig:hiper}).  
Then we analyze a union of pairs of neighborhoods generators for all . 
If this union contains all pairs of points , where  and ,
then .

For a given  and a segment 
we shoot rays from a point  towards .
Let us assume that a given ray intersects  in a point 
  for some value of . 
Let  be the vector between points  and . 
Then  where coefficients  for  
depend only on endpoints coordinates of segments  and .
The ray refracts in  from  in such a way that the angle between
directions of incidence and refraction of the ray is equal to . 
The parametrized equation of the refracted ray is 
 for  
(or  for , respectively) 
where  (, respectively) denotes a rotation matrix 
for a clockwise (counter-clockwise, respectively) angle . 
If refracted ray  intersects line  in a point  
(it is not always possible - see Figure \ref{fig:hiper}) then we compare the 
-coordinates of  and . 
As a result we obtain a function containing only parameters  and :  
, 
where coefficients  are fixed.
Since -coordinates of  and  are also equal we obtain 
, 
where  and  are (at most quadratic) polynomials of variable 
 and  are fixed (the exact description of those polynomials and variables
is much more complex than the description of the coefficients in the previous step and it is omitted here 
- see them in Appendix).
 


 Let  denote a value of the parameter  of the intersection point
of the line  and the line containing the ray that starts in  and refracts
in  creating an angle . Let , where  is 
a set of values of  such that the ray refracted in  intersects . 
The function  is a hyperbola and the function  is a part of it
(see Figure \ref{fig:hiper}).



\begin{figure}[htbp]
\centering
\includegraphics[scale=0.34]{hiper-t-2.eps}
\caption{Examples of correlation between parameters  and  (for a fixed ) 
for a presented composition of segments and (a) a refraction angle near  
(dotted lines show refracted rays that are analyzed) and (b) near . 
The value  corresponds to the intersection point of lines  and .
 Dotted curves show a case when a line containing a refracted ray intersects 
 but the ray itself does not. }
\label{fig:hiper}
\end{figure}





 Note that for a given angle  (, respectively) extreme points
of the function   (, respectively) do not have to 
belong to the set . We can find them by computing a derivative
. 


Then we can compute the corresponding values of the parameter . 
This way we obtain the pair  such that the segment 
is a chord of a circle that is tangent to the analyzed segment  in  and
 (,
respectively).  







 Let ,
i.e., this is a set of all  such that points  and  generate 
a lune intersected by the analyzed segment .
Let  be a set of pairs 
of parameters  such that the segment  intersects a lune generated by points 
 and .
The set  is an area limited by  algebraic curves 
of degree at most . 
The curves match the values of the parameter  corresponding 
to extreme points of  (, respectively). In particular 
there are hyperbolas for angles  and  correlated with the rays refracted 
in the ends of the segment  (for parameters  and ) - see 
Figure \ref{fig:polygon}.

\begin{figure}[htbp]
\centering
\includegraphics[scale=0.33]{polygon-t-2.eps}
\caption{ Examples of sets  for  near (a)  and (b)  
(the shape of  also depends on the position of the segment  with respect 
to  and ). Dotted (dashed, respectively) curves limit the area corresponding 
to rays refracted through the segment  and creating the angle  (, 
respectively).} 
\label{fig:polygon}
\end{figure}

\begin{lemma}
The edge  belongs to the -skeleton  if and only if \\
. 
\end{lemma}
\begin{proof}
If 
then there exists a pair of parameters  such that 
a lune generated by points  and  is not intersected 
by any segment , i.e., .
The opposite implication can be proved in the same way.
\end{proof}

\begin{theorem}
\label{lambda4}
For  the -skeleton  can be found 
in  time.
\end{theorem}
\begin{proof}
We analyze  pairs of line segments. For each pair of segments  
we compute . 
For each  we find a set of pairs of parameters
 such that .
The arrangement of  curves in total can be found in  time
\cite{gr04}. 
Then the difference 

can be found in  time (see Figure \ref{fig:omega4} in Appendix). 
Therefore we can verify which edges belong to  in  time. 
\end{proof}









\section{Finding -skeletons for }

Let us first consider the circle-based -skeletons. According to Theorem \ref{luneinkluzja} for  there are only 
edges which can belong to the -skeleton for a given set of line segments.
  We will use this property to compute -skeletons faster than in the previous 
section.  



\begin{lemma}
\label{components}
For  and the set  of  line segments the number of connected components 
of the set 
is O(n) for any pair .
\end{lemma}
\begin{proof}
 According to Theorem by Kirkpatrick and Radke \cite{kr85} for  
the following inclusion holds .
Therefore any neighborhood for  is included in some neighborhood for  with 
the same pair of generators.
On the other hand, for a given parameter  and a given connected component 
of the set 
there exists a sufficiently big  such that for  the component contains only one point
(we increase an arbitrary neighborhood corresponding to the connected component for 
a given ).
Hence, the number of one point components (for all values of ) estimates
the number of connected components for a given .
But in this case at least one disc forming the neighborhood is tangent to two
segments different than  and  or at least one generator of the neighborhood
is at the end of  or .
In the first case the two segments tangent to the disc and segments  are the
the closest ones to the center of the disc. Therefore the complexity
of the set of such components does not exceed the complexity of the -order
Voronoi diagram for , i.e., it is  \cite{pz13}.
In the second case there is a constant number of additional components.   

\end{proof}


\begin{lemma}
\label{two-side}
For any  and  there is at most one connected component of the set

that contains points with the same  coordinate. 
\end{lemma}
\begin{proof}
 Let the inscribed angle corresponding to  be equal to .
Let  and  (, respectively) be points 
such that the angle between  (, respectively) and  is equal to 
(for  we have ), see Figure \ref{fig:two-side}.
 Boundaries of all neighborhoods  
generated by  and a point in  contain either  or . 
There exists the leftmost (rightmost, respectively) position (might be in infinity) 
of the second neighborhood generator with respect to the direction of . Between 
those positions no neighborhood intersects segments from . 
 Hence, points corresponding to positions of such generators 
belong to the same connected component of 
. 
\end{proof}


\begin{figure}[htbp]
\centering
\includegraphics[scale=0.4]{two-side.eps}
\caption{Neighborhoods that have one common generator.}
\label{fig:two-side}
\end{figure}

The algorithm for computing circle-based -skeletons for  is almost
the same as the algorithm for . 


\begin{theorem}
For  the circle-based -skeleton  can be found 
in  time. 
\end{theorem} 
\begin{proof}
 Due to Theorem \ref{luneinkluzja} we have to analyze  edges of .
For  and for the given segments  each set 
can be divided in two sets with respect to the variable . For each  the first 
set contains part of  that is unbound from above with respect to  
and the second one contains part of  unbound from below.
The part that contains pairs  such that the set of values of  is  can be divided arbitrarily.
We use Hershberger's algorithm \cite{h89} to compute unions of sets for 
 in each group separately. 
Then, according to Lemma \ref{two-side} we find an intersection of complements of computed
unions.  
It needs  time. Hence, the total time complexity of the algorithm 
is .  
\end{proof}

\begin{figure}[htbp]
\centering
\includegraphics[scale=0.33]{circle-n.eps}
\caption{An example of (a) refracted rays and (b) correlations between variables  
and  for circle-based -skeletons, where .
(the shape of  depends on the position of the segment  with respect 
to  and )}
\label{fig:circle}
\end{figure}   


Let us consider the lune-based -skeletons now. Unfortunately, Lemma \ref{two-side}
does not hold in this case (see Figure \ref{fig:paski} in Appendix).



According to Theorem \ref{luneinkluzja}, in this case we have to consider only 
pairs of line segments in  (the pairs corresponding to edges of ).
 We will analyze pairs of points belonging to given segments 
which generate discs such that each of them is intersected by any segment 
.
We will consider -skeletons for  (a -skeleton is the same 
in the circle-based and lune-based case). 
Let  and  be generators of a lune 
 and let  be a circle creating 
a part of its boundary containing point . 

We will shoot a ray from a lune generator and we will compute a possible position
of the second generator when the refraction point belongs to the lune.
Let an angle between a shot ray and a refracted ray be equal to 
and let .
Unfortunately, the ray shot from  and refracted in 
does not intersect the segment  in .
However, we can define a segment  such that the ray shot from  refracts
in  if and only if the same ray refracted in a point of  passes through 
(see Figure \ref{fig:tales}).   







\begin{figure}[htbp]
\centering
\includegraphics[scale=0.3]{tales-n.eps}
\caption{ The auxiliary segment  and rays refracted in  and . 
}
\label{fig:tales}
\end{figure}

\begin{lemma}
 Assume that ,  and ,
where . Let a point , where , 
belong to . 
Let  be a line perpendicular to the segment , passing
through  and crossing  in a point . Then 
.
    
\end{lemma}
\begin{proof} 
 Let  be an opposite to  end of the diameter of .    
Then , where  is the center of .
From the definition of the -skeleton follows that 
.
According to Thales' theorem  (see Figure \ref{fig:tales}). 

\end{proof}






 The algorithm computing a lune-based -skeleton for  is similar 
to the previous one. Let , where
 is a homothety with respect to a point  
and a ratio .
Like in the case of circle-based -skeletons we compute pairs of parameters  
such that the ray shot from  refracts in a point of  and intersects the segment
 in , i.e., an analyzed segment  intersects a disc limited by the circle 
. 

However, in the case of lune-based -skeletons we analyze only one hyperbola 
(functions for clockwise and counterclockwise refractions are the same). 
Moreover, sets  and  are different. 
 They contain pairs of parameters  corresponding to points generating discs 
such that each of them separately is intersected by the segment .
Therefore, we have to intersect those sets to obtain a set of pairs of parameters
corresponding to points generating lunes intersected by  (see Figure \ref{fig:right}). 



\begin{figure}[htbp]
\centering
\includegraphics[scale=0.3]{right-n-2.eps}
\caption{An example of (a) a composition of three segments , (b) correlations 
between variables  and  (parametrizing  and , respectively), 
(c) the set  and (d) the intersection ,
where . }
\label{fig:right}
\end{figure}

\begin{theorem}
For  the lune-based -skeleton  can be found 
in  time.
\end{theorem} 
\begin{proof}
-skeletons for  satisfy the inclusions from Theorem \ref{luneinkluzja}.
Hence, the number of tested edges is linear. For each such pair of segments  
we compute the corresponding sets of pairs of points generating lunes that do not intersect 
segments from . 
Similarly as in Theorem \ref{lambda4} we can do it in  time.
Therefore, the total time complexity of the algorithm (after analysis of  pairs 
of segments) is .   
\end{proof}



\section{Computing Gabriel Graph for segments}



 In the previous sections we constructed sets of all pairs of points generating 
neighborhoods that do not intersect segments other than the segments containing generators. 
Now we want to find only  pairs of generators (one pair for each edge 
of a -skeleton) that define the graph.   
Let  denote a region of the -order Voronoi diagram for the set 
corresponding to  and  denote a region of the -order Voronoi 
diagram for the set  corresponding to .
If an edge , where  belongs to the Gabriel Graph then there 
exists a disc  centered in , which does not contain points from 
 and its diameter is , where , 
 and . 
The disc center  belongs to the set  
for some .

 First, for segments  we define a set of all middle points 
of segments with one endpoint on  and one on .
This set is a quadrilateral  (or a segment if ) with vertices 
in points , where  
for  are endpoints of the segments  for  (boundaries of the set 
 are determined by the images of  and  under four homotheties with respect to the ends of those segments 
and a ratio ).






 




 Let us analyze a position of a middle point of a segment  whose ends slide 
along the segments . Let the length of  be .  
We rotate the plane so that segment  lies in the negative part of -axis 
and the point of intersection of lines containing segments  and  
(if there exists) is .
 Let the segment  lie on the line parametrized by
 for . Then the middle point of  
is , where . 

Since , 
then we have , so all points 
 for a given  lie on an ellipse - see Figure \ref{fig:krzywa}.



\begin{figure}[htbp]
\centering
\includegraphics[scale=0.4]{krzywa.eps}
\caption{(a) The set of middle points of segments , 
where  and } and (b) the curve  such that the distance between a point on the curve  and the segment  is equal to the length of the radius of a corresponding disc centered in .
\label{fig:krzywa}
\end{figure}   


We want to find a point  which is a center of a segment 
, where  and , and .
 Then the disc with the center in  and the radius 
intersects only segments , i.e., there exists an edge of 
between  and . 

We need to examine two cases. 
First, we consider the situation when the closest to  point of a segment  belongs
to the interior of .  
Let  be the line that contains segment , which endpoints are  
and , and let  be 
the parametrization of . Let  be a line parallel to  with 
parametrization  such that 
the distance between  and  is equal to . We compute the intersection 
of the ellipse  and the line . 
The result is 
, so  satisfies an equation 
 where coefficients  are fixed and depend on 
 for .
This equation defines a curve  (see Figure \ref{fig:krzywa}) which intersects 
corresponding ellipses. A point  which belongs to a part of the ellipse that lies 
on the opposite side of the curve  than the segment  is a center of  a disc
which has a diameter , where , , and does not
intersect segment . 



In the second case one of the endpoints of the segment  is the nearest point to  (among the points from ). 
Let  and  be discs with diameter  and with centers in corresponding 
ends of the segment .  
We compute the intersection of  
(, respectively) and ellipse 
. We obtain 
, so 
 and  satisfies an equation 
 where coefficients  and  
for  depend on  (or on 
, respectively). 
If there exists a point  that belongs to 
the part of the ellipse between the segments , then there also exists  
a disc with center in  and a diameter , where  
and , which does not contain ends of the segment .

In both cases we obtain a curve  dependent on the parameter  
- see Figure \ref{fig:krzywa}.
We check if a set 
and the segment  are on the same side of the curve .
Otherwise, the segment  belongs to the Gabriel Graph for the set 
(i.e., there exists a point  which is the center of a segment 
, where , , and 
for all ).




\begin{theorem}
For a set of  segments  the Gabriel Graph  can be computed in  
time.
\end{theorem}
\begin{proof}
The -order Voronoi diagram and the -order Voronoi diagram can be found in  
time \cite{pz13}. The number of triples of segments we need to test is linear. 
For each such triple we can check if there exists an empty -skeleton lune
in time proportional to the complexity of the set 
.
The total complexity of those sets is . 
Hence, the complexity of the algorithm is .
\end{proof}

 
\section{Conclusions}

The running time of the presented algorithms for -skeletons for sets 
of  line segment ranges between ,  
and  and depends on the value of .  
For  the -skeleton 
is not related to the Delaunay triangulation of the underlying set of segments.
The existence of a relatively efficient algorithm for the Gabriel Graph 
suggests that it may be possible to find a faster way 
to compute -skeletons for other values of , especially 
for .

 The edges of the Delaunay triangulation for line segments can be represented 
in the form described in this paper as rectangles contained in 
square in the -coordinate system.
If for each pair of -skeleton edges the intersection of the corresponding
sets for the -skeleton and the Delaunay triangulation is not empty then
there exist a plane partition generated by some pairs of generators of -skeleton
neighborhoods. Unfortunately, it is not always possible (see Figure \ref{fig:diameter} in the Appendix).\\
The algorithms shown in this work for each pair of segments find such a position 
of generators that the corresponding lune does not intersect any other segment. 
We could consider a problem in which the number of used generators of neighborhoods is  (one generator per each edge). Then the method described in the paper can also be used. 
We analyze a -dimensional space and test if 
, where  and  also define corresponding coordinates 
in . Unfortunately, such an algorithm is expensive. 
 However, in this case a -skeleton already generates a plane partition.

 The total kinetic problem that can be solved in similar way is a construction 
-skeletons for points moving rectilinear but without limitations concerning 
intersections of neighborhoods with lines defined by the moving points. In this case the form of sets 
 changes and the solution is much more complicated.

 Are there any more effective algorithms for those problems? 

Additional interesting questions about -skeletons are related 
to their connections with -order Voronoi diagrams for line segments.  




\noindent
{\bf Acknowledgments}\\
\noindent
The authors would like to thank Jerzy W. Jaromczyk for important comments.












\begin{thebibliography}{50}
\bibitem{aa96}
O. Aichholzer, F. Aurenhammer,
\textit{Straight Skeletons for General Polygonal Figures in the Plane},
\textit{Proceedings of 2nd Computing and Combinatorics Conference (COCOON)}, 1996, 117-126



\bibitem{bcs07}
M. Br\'{e}villiers, N. Chevallier, D. Schmitt,
\textit{Triangulations of line segment sets in the plane},
\textit{Proceedings of the 27th international Conference on Foundations of Software 
Technology and Theoretical Computer Science}, 2007, 388-399



\bibitem{bms84}
Ch. Burnikel, K.Mehlhorn, S. Schirra,
\textit{How to compute the Voronoi diagram of line segments: Theoretical and experimental results},
\textit{Proceedings of 2nd Annual European Symposium On Algorithms (ESA)}, 1994, 227-239 

\bibitem{ck89}
L.P.Chew, K.Kedem,
\textit{Placing the largest similar copy of a convex polygon among polygonal obstacles}
\textit{Proceedings of the 5th Annual ACM Symposium on Computational Geometry}, 1989, 167-174





\bibitem{e02} 
D. Eppstein, 
\textit{-skeletons have unbounded dilation}, 
\textit{Computational Geometry}, vol. 23, 2002, 43-52

\bibitem{gr04} 
J. E. Goodman, J. O'Rourke, 
\textit{Handbook of Discrete and Computational Geometry}, 
Chapman \& Hall/CRC, 2004

\bibitem{gs69} 
K.R. Gabriel, R.R. Sokal, 
\textit{A new statistical approach to geographic variation analysis}, 
\textit{Systematic Zoology} 18, 1969, 259-278

\bibitem{h89}
J. Hershberger, 
\textit{Finding the Upper Envelope of n Line Segments in O(n log n) Time},
\textit{Information Processing Letters}, 33(4), 1989, 169-174

\bibitem{hlm03} 
F. Hurtado, G. Liotta, H. Meijer, 
\textit{Optimal and suboptimal robust algorithms for proximity graphs}, 
\textit{Computational Geometry: Theory and Applications} 25 (1-2), 2003, 35-49.

\bibitem{jk87} 
J.W. Jaromczyk, M. Kowaluk, 
\textit{A note on relative neighborhood graphs}, 
\textit{Proceedings of the 3rd Annual Symposium on Computational Geometry}, Canada, Waterloo, 
ACM Press, 1987, 233-241 



\bibitem{jt92} 
J.W. Jaromczyk, G.T. Toussaint, 
\textit{Relative neighborhood graphs and their relatives}, 
\textit{Proceedings of the IEEE}, volume 80, issue 9, 1992, 1502-1517

\bibitem{kr85} 
D.G. Kirkpatrick, J.D. Radke, 
\textit{A framework for computational morphology}, 
\textit{Computational Geometry}, North Holland, Amsterdam, 1985, 217-248

\bibitem{k12}
M. Kowaluk,
\textit{Planar -skeleton via point location in monotone subdivision of subset of lunes},
EuroCG 2012, Italy, Assisi, 2012, 225-227



\bibitem{l94} 
A. Lingas. 
\textit {A linear-time construction of the relative neighborhood graph 
from the Delaunay triangulation}, 
\textit{Computational Geometry} 4, 1994, 199-208



\bibitem{ms84} 
D.W. Matula, R.R. Sokal, 
\textit{Properties of Gabriel graphs relevant to geographical variation research 
and the clustering of points in plane}, 
\textit{Geographical Analysis} 12, 1984, 205-222

\bibitem{pz13} 
E. Papadopoulou, M. Zavershynskyi,
\textit{A Sweepline Algorithm for Higher Order Voronoi Diagrams}, 
\textit{Proceedings of 10th International Symposium on Voronoi Diagrams in Science and Engineering (ISVD)}, 2013, 16-22 

\bibitem{sa95}
M. Sharir, P.K. Agarwal,
\textit{Davenport-Schinzel Sequences and Their Geometric Applications},
\textit{Cambridge University Press}, 1995

\bibitem{su83} 
K.J. Supowit, 
\textit{The relative neighborhood graph, with an application to minimum spanning trees}, 
\textit{Journal of the ACM} volume 30, issue 3, 1983, 428-448

\bibitem{tou80} 
G.T. Toussaint, 
\textit{The relative neighborhood graph of a finite planar set}, 
\textit{Pattern Recognition} 12, 1980, 261-268 

\end{thebibliography}

\clearpage

\section{Appendix}

\subsection{Coefficients used in Section :} 

, 
. \\

\subsection{Polynomials computed in Section :} 

, \\
, \\
.

\begin{figure}[htbp]
\centering
\includegraphics[scale=0.3]{paski.eps}
\caption{An example of a set  (for  close to infinity) where 
the same point in  generates lunes which correspond to points in different connected components of .}
\label{fig:paski}
\end{figure}

\begin{figure}[htbp]
\centering
\includegraphics[scale=0.25]{omega4.eps}
\caption{
An example of a set  where for segments  the difference 

contains  connected components (for a very small ). 
}
\label{fig:omega4}
\end{figure}

\begin{figure}[htbp]
\centering
\includegraphics[scale=0.55]{diameter.eps}
\caption{
An example of a set  where for segments  there is no such Gabriel Graph neighborhood that the diameter of the disc defining this neighborhood is contained in the Delaunay Triangulation edge  (the light grey area) even though the edge  belongs to the Gabriel Graph for . Moreover, generators of the neighborhoods of the {\em GG}-edges  and  create intersecting diameters between corresponding line segments.
}
\label{fig:diameter}
\end{figure}

\clearpage



\end{document}