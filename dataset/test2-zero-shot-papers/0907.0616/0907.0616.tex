\documentclass{LMCS}

\usepackage[nogroup]{versions}

\includeversionnogroup{LMCS}
\excludeversion{conference}
\excludeversion{web}

\usepackage{pwlogic,FO2_on_words}

\usepackage{caption}
\usepackage[basic]{complexity}
\usepackage{enumerate}
\usepackage{hyperref}
\usepackage{picins}
\usepackage{url}

\newcommand{\qedconf}{}

\processifversion{LMCS}{
  \includeversionnogroup{full}
}

\processifversion{web}{
  \includeversionnogroup{full}
  \usepackage[nocitecount]{drftcite}
  \usepackage{scrpage2}
  \pagestyle{scrheadings}
}

\processifversion{full}{
}

\processifversion{conference}{
  \renewcommand{\qedconf}{\qed}
  \newcommand{\qedhere}{\qedconf}
}




\newcommand{\abs}[1]{ \vert #1 \vert }
\newcommand\bfhead[1]{\medskip\noindent{\bf #1 } }  
\newcommand{\commentout}[1]{}
\newcommand{\fo}{\FO}
\newcommand{\set}[1]{\{  #1 \} }




\usepackage{FO2_on_words_wordpics}

\def\doi{5 (3:3) 2009}
\lmcsheading {\doi}
{1--23}
{}
{}
{Feb.~\phantom{0}7, 2008}
{Aug.~\phantom{0}3, 2009}
{} 

\begin{document}

\author[P.~Weis]{Philipp Weis}
\author[N.~Immerman]{Neil Immerman}



\newcommand{\myaddress}{Department of Computer Science\\
  University of Massachusetts, Amherst\\
  140 Governors Drive, Amherst, MA 01003, USA
}
\newcommand{\myinfo}{
  \myaddress\\
  \url{{pweis,immerman}@cs.umass.edu}\\
  \url{http://www.cs.umass.edu/~{pweis,immerman}}\\\bigskip\bigskip
  \footnotesize{1998 ACM Subject Classification: F.4.1, F.4.3}
}
\newcommand{\mythanks}{Supported in part by NSF grant CCF-0514621.}
\newcommand{\mytitle}{Structure Theorem and Strict Alternation Hierarchy for
  \texorpdfstring{}{FO^2} on Words\rsuper*} 

\begin{conference}
  \institute{\myinfo}
  \title{\mytitle\thanks{\mythanks}}
\end{conference}
\begin{LMCS}
  \address{\myaddress}
  \email{\{pweis,immerman\}@cs.umass.edu}
  \thanks{\mythanks}
  \title{\mytitle}
  \keywords{descriptive complexity, finite model theory, alternation hierarchy,
\ef{} games}
  \subjclass{F.4.1, F.4.3}
  \titlecomment{{\lsuper*}This work is an extended version of \cite{WI07}}
\end{LMCS}
\begin{web}
  \date{\myinfo}
  \title{\mytitle\thanks{\mythanks}}
\end{web}

\begin{abstract}
It is well-known that every first-order property on words is
expressible using at most three variables. The subclass of properties
expressible with only two variables is also quite interesting and
well-studied. We prove precise structure
theorems that characterize the exact expressive power of first-order
logic with two variables on words. Our results apply to both the case with
and without a successor relation.

For both languages, our structure theorems show exactly what is expressible
using a given quantifier depth, , and using  blocks of alternating
quantifiers, for any . Using these characterizations, we prove,
among other results, that there is a strict hierarchy of alternating
quantifiers for both languages. The question whether there was such a
hierarchy had been completely open. As another consequence of our structural
results, we show that satisfiability for first-order logic with two
variables without successor, which is \NEXP-complete in general, becomes
\NP-complete once we only consider alphabets of a bounded size.
\end{abstract}

\maketitle

\section{Introduction} \label{sec:intro}

It is well-known that every first-order property on words is
expressible using at most three variables \cite{IK89,K68}. 
The subclass of properties expressible with only
two variables is also quite interesting and well-studied (Fact
\ref{fact}).  

In this paper we prove precise structure theorems that characterize
the exact expressive power of first-order logic with two variables on
words.  Our results apply to < and , the
latter of which includes the binary successor relation in addition to
the linear ordering on string positions.  

For both languages, our structure theorems show exactly what is expressible
using a given quantifier depth, , and using  blocks of alternating
quantifiers, for any . Using these characterizations, we prove that
there is a strict hierarchy of alternating quantifiers for both languages.
The question whether there was such a hierarchy had been completely open
since it was asked in \cite{EVW97,EVW02}. As another consequence of our
structural results, we show that satisfiability for , which is
\NEXP-complete in general \cite{EVW02}, becomes \NP-complete once we only
consider alphabets of a bounded size.

Our motivation for studying  on words comes from the desire to
understand the trade-off between formula size and number of
variables.  This is of great interest because, as is well-known, this
is equivalent to the trade-off between parallel time and number of
processors \cite{I99}.  Adler and Immerman \cite{AI03}
 introduced a game that can be used to determine the
minimum size of first-order formulas with a given number of variables
needed to express a given property.  These games, which are closely
related to the communication complexity games of Karchmer and
Wigderson  \cite{KW90}, were used to prove two optimal size bounds
for temporal logics \cite{AI03}.  Later Grohe and Schweikardt used
similar methods to study the size versus variable trade-off for
first-order logic on unary words \cite{GS05}.  They proved that all
first-order expressible properties of unary words are already
expressible with two variables and that the variable-size trade-off
between two versus three variables is polynomial whereas the trade-off
between three versus four variables is exponential.  They left open
the trade-off between  and  variables for .  While we
do not directly address that question here, our classification of
 on words is a step towards the general understanding of the
expressive power of  needed for progress on such trade-offs.

Our characterization of  and  on words is based
on the very natural notion of -ranker (Definition \ref{def:n-ranker}).
Informally, a ranker is the position of a certain combination of letters in
a word. For example,  and  are
1-rankers where  is the position of the first
\texttt{a} in  (from the left) and  is the position
of the first \texttt{b} in  from the right. Similarly, the 2-ranker  denotes the position of the first
\texttt{c} to the right of the first \texttt{a}, and the 3-ranker,  denotes the
position of the first \texttt{b} to the left of . If there is no such
letter then the ranker is undefined. For example,  and  is undefined.

Our first structure theorem (Theorem \ref{thm:ranker-char}) says that the
properties expressible in <, i.e. first-order logic with two
variables and quantifier depth , are exactly boolean combinations of
statements of the form, `` is defined'', and `` is to the left (right)
of '' for -rankers, , and -rankers, , with  and
. A non-quantitative version of this theorem was previously known
\cite{STV01}.\footnote{See item \ref{turtle} in Fact \ref{fact}: a ``turtle
  language'' is a language of the form `` is defined'', for some ranker,
  .} Furthermore, a quantitative version in terms of iterated block
products of the variety of semi-lattices is presented in \cite{TT07}, based
on work by Straubing and Th\'erien \cite{ST02}.

Surprisingly, Theorem \ref{thm:ranker-char} can be generalized in almost
exactly the same form to characterize  where there are at
most  blocks of alternating quantifiers, . This second structure
theorem (Theorem \ref{thm:alt-ranker-char}) uses the notion of
-ranker where there are  blocks of 's or 's, that
is, changing direction in rankers corresponds exactly to alternation of
quantifiers. Using Theorem \ref{thm:alt-ranker-char} we prove that there is
a strict alternation hierarchy for  (Theorem
\ref{thm:alt-hierarchy}) but that exactly at most 
alternations are useful, where  is the size of the alphabet
(Theorem \ref{thm:alt-alphabet}).

The language  is more expressive than  because
it allows us to talk about consecutive strings of symbols\footnote{With
  three variables we can express  using the ordering: .}. For , a
straightforward generalization of -ranker to -successor-ranker allows
us to prove exact analogs of Theorems \ref{thm:ranker-char} and
\ref{thm:alt-ranker-char}. We use the latter to prove that there is also a
strict alternation hierarchy for  (Theorem
\ref{thm:suc-alt-hierarchy}). Since in the presence of successor we can
encode an arbitrary alphabet in binary, no analog of Theorem
\ref{thm:alt-alphabet} holds for .

The expressive power of first-order logic with three or more variables on
words has been well-studied. The languages expressible are of course the
star-free regular languages \cite{MP71}. The dot-depth hierarchy is the
natural hierarchy of these languages. This hierarchy is strict \cite{BK78}
and identical to the first-order quantifier alternation hierarchy
\cite{T82,T84}.

Many beautiful results on  on words were also already known. The
main significant outstanding question was whether there was an alternation
hierarchy. The following is a summary of the main previously known
characterizations of  on words. For a detailed treatment of all
these characterizations, we refer the reader to \cite{TT01}.

\begin{fact}\label{fact}
  \textup{\cite{EVW97,EVW02,PW97,S76,TW98,STV01}}
  Let . The following conditions are equivalent:
  \setlength{\itemsep}{0pt}
  \begin{enumerate}[(1)]
  \item 
  \item  is expressible in unary temporal logic
  \item 
  \item\label{UL}   is an unambiguous regular language
  \item The syntactic semi-group of  is a member of {\bf DA}
  \item  is recognizable by a partially-ordered 2-way automaton
  \item\label{turtle}  is a boolean combination of ``turtle languages''
  \end{enumerate}
\end{fact}

The proofs of our structure theorems are self-contained applications
of \ef{} games.  All of the above characterizations follow from these
results.  Furthermore, we have now exactly connected quantifier and
alternation depth to the picture, thus adding tight bounds and further
insight to the above results.

For example, one can best understand item \ref{UL} above -- that
 on words corresponds to the unambiguous regular languages
-- via
Theorem \ref{thm:unique-rankers} which states that any 
formula with one free variable that is always true of at most one position
in any string, necessarily denotes an -ranker.

In the conclusion of \cite{STV01}, the authors define the subclasses
of rankers with one and two blocks of alternation.  They write that,
``\ldots turtle languages might turn out to be a helpful tool for
further studies in algebraic language theory.''  We feel that the
present paper fully justifies that prediction.  Turtle languages --- aka
rankers --- do provide an exceptionally clear and precise
understanding of the expressive power of  on words, with and
without successor.

In summary, our structure theorems provide a complete classification of
the expressive power of  on words in terms of both quantifier
depth and alternation.  They also tighten several previous
characterizations and lead to the alternation hierarchy results.

We begin the remainder of this paper with a brief review
of logical background including \ef{} games, our main tool. In 
Sect.~\ref{sec:FO2} we formally define rankers and present our structure
theorem for <. The structure theorem for
< is covered in Sect.~\ref{sec:FO2-alt},
including our alternation hierarchy result that follows from
it. Sect.~\ref{sec:FO2-suc} extends our structure theorems and the
alternation hierarchy result to <,\suc. Finally, we discuss
applications of our structural results to satisfiability for < in
Sect.~\ref{sec:satisfiability}.

\section{Background and Definitions} \label{sec:background}



We recall some notation concerning strings, first-order logic,
and \ef{} games.  See \cite{I99} for more details, including the
proof of Facts \ref{fact:ef1} and \ref{fact:ef2}.

 will always denote a finite alphabet and  the empty
string. For a word  and , let  be
the -th letter of ; and for  a subinterval of , let
 be the substring . Slightly abusing notation, we
identify a word  with the logical structure .
Here  are all unary relation symbols,
and  and  are the only two variables. If not specified otherwise, we
have  by default, and for all ,
.
Furthermore, we write  for the word structure  with the two
variables set to  and , respectively, and  for the word
structure  with . Thus , and  iff .

We use < to denote first-order logic with a binary linear order
predicate , and <, \suc for first-order logic with an
additional binary successor predicate.  refers to the restriction
of first-order logic to use at most two distinct variables, and quantifier
depth .  is the further restriction to formulas such that
any path in their parse tree has at most  blocks of alternating
quantifiers, and . We write
 to mean that  and  agree on all formulas from
, and  if they agree on
.

We assume that the reader is familiar with our main tool: the \ef{} game. In
each of the  moves of the game , Samson places one of the
two pebble pairs,  or  on a position in one of the two words and
Delilah then answers by placing that pebble's mate on a position of the
other word. Samson wins if after any move, the map from the chosen points in
 to those in , i.e., ,  is not an
isomorphism of the induced substructures; and Delilah wins otherwise. The
fundamental theorem of \ef{} games is the following:

\begin{fact}\label{fact:ef1}
  Let , .
  Delilah has a winning strategy for the game  iff
  .
\end{fact}

Thus, \ef{} games are a perfect tool for determining what is
expressible in first-order logic with a given quantifier-depth and
number of variables.  The game  is the restriction
of the game  in which Samson may change which word he
plays on at most  times.

\begin{fact}\label{fact:ef2}
  Let  and let  with .
  Delilah has a winning strategy for the game  iff .
\end{fact}

We end this section with a simple lemma that will be useful whenever we want
to prove that there is a formula expressing a property of strings. With this
lemma, it suffices to show that for any pair of strings, one with the
property in question and one without, there is a formula that distinguishes
between these two particular strings.

\begin{lem} \label{lem:fin-equiv}
  Let  and let  be a logic closed
  under boolean operations with only finitely many inequivalent formulas. If
  for every  and every  there is a formula
   such that  and , then there is a formula  such
  that for all , .
\end{lem}

\begin{proof}
  Let , and
  let  be a maximal subset of  containing only inequivalent
  formulas. Since  contains only finitely many inequivalent formulas,
   is finite. For every , we define the finite sets of
  formulas .
  Since all these sets are subsets of the finite set , there can
  only be finitely many of them. Thus there is a finite set 
  such that . Now
  we set
  
  We have  and for every ,  as required.
\end{proof}

It is well-known \cite{I99} that for any , the logics
 and , both with and without the successor
predicate, have only finitely many inequivalent formulas. Thus the above
lemma applies to these logics.


\section{Structure Theorem for \texorpdfstring{\FOV[]{2}}{FOV[<]2}} \label{sec:FO2}

We define boundary positions that point to the first or last occurrences of
a letter in a word, and define an -ranker as a sequence of  boundary
positions. In terms of \cite{STV01}, boundary positions are turtle
instructions and -rankers are turtle programs of length . The
following three lemmas show that basic properties about the definedness and
position of these rankers can be expressed in <, and we use these
results to prove our structure theorem.

\begin{defi} \label{def:boundary-position}
  A \emph{boundary position} denotes the first or last
  occurrence of a letter in a given word. Boundary positions are of the form
   where  and . The
  interpretation of a boundary position  on a word  is defined as follows.
  
  Here we set  and  to be undefined, thus  is
  undefined if  does not occur in . A boundary position can also be
  specified with respect to a position .
  
\end{defi}

\begin{defi} \label{def:n-ranker}
  Let  be a positive integer. An \emph{-ranker}  is a sequence of
   boundary positions. The interpretation of an -ranker  on a word  is defined as follows.
  
\end{defi}

Instead of writing -rankers as a formal sequence , we
often use the simpler notation . We denote the set of all
-rankers by , and the set of all -rankers that
are defined over a word  by . Furthermore, we set
 and .

\begin{defi}
  Let  be an -ranker. As defined above, we have  for boundary positions . The \emph{-prefix ranker} of  for
   is  .
\end{defi}

\begin{defi} \label{def:order-type}
  Let . The \emph{order type} of  and  is defined as
  
\end{defi}

\begin{lem}[distinguishing points on opposite sides of a ranker]
  \label{lem:ranker-sides}
  Let  be a positive integer, let  and let . Samson wins the game  where
  initially .
\end{lem}

\begin{proof}
  We only look at the case where  and  since all
  other cases are symmetric to this one. For  Samson has a winning
  strategy: If  is the first occurrence of a letter, then Samson places
   on  and Delilah cannot reply. If  marks the last occurrence
  of a letter in the whole word, then Samson places  on . Again,
  Delilah cannot reply with any position and thus loses.

  \piccaption{\label{fig:ranker-sides-2}The case }
  \parpic(7cm,3.3cm)[fr]{
      \begin{tikzpicture}
        \word[6cm]{\wordu}{}
        \word[6cm]{\wordv}{}
        \dwordpos{4cm}{}
        \dwordpos{2cm}{}
        \wordupoint{5cm}{}
        \wordvpoint{3cm}{}
        \wordupoint{4cm}{}
        \wordvpoint{1cm}{}
      \end{tikzpicture}
    }

  For , we look at the prefix ranker  of . One of the
  following two cases applies.
  \begin{enumerate}[(1)]
  \item
    \parpic(7cm,2.8cm)[r]{} , as shown in
    Fig.~\ref{fig:ranker-sides-2}. Samson places pebble  on , and
    Delilah has to reply with a position that is
    to the left of . She cannot
    choose a position in the interval , because this
    section does not contain the letter . Thus she has to choose a
    position left of or equal to . By induction Samson wins the
    remaining game.\medskip

  \item
    \piccaption{\label{fig:ranker-sides-3}The case }
    \parpic(7cm,3.3cm)[fr]{
      \begin{tikzpicture}
        \word[6cm]{\wordu}{}
        \word[6cm]{\wordv}{}
        \dwordpos{2cm}{}
        \dwordpos{4cm}{}
        \wordupoint{3cm}{}
        \wordvpoint{1cm}{}
        \wordvpoint{2cm}{}
        \wordupoint{5cm}{}
      \end{tikzpicture}
    }
    , as shown in Fig.
    \ref{fig:ranker-sides-3}. Samson places  on , and Delilah has to
    reply with a position to the right of  and thus to the right of .
    She cannot choose any position in
    , because this interval does not
    contain the letter , thus Delilah has to choose a position
    to the right of or equal to . By induction Samson wins the
    remaining game.\qedhere
  \end{enumerate}
\end{proof}

\pagebreak[3]


\begin{lem}[expressing the definedness of a ranker] \label{lem:ranker-def}
  Let  be a positive integer, and let . There is a
  formula < such that for all , .
\end{lem}

\begin{proof}
  Using Lemma  it suffices to consider arbitrary  with  and , and using
  Fact \ref{fact:ef1}, it suffices to show that Samson wins the game
  . If , the shortest prefix ranker of , is not
  defined over , the letter referred to by  occurs in  but does
  not occur in . Thus Samson easily wins in one move.

  \piccaption{\label{fig:rankpos-4} is undefined}
  \parpic(5cm,3.3cm)[fr]{
    \begin{tikzpicture}
      \word[4cm]{\wordu}{}
      \word[4cm]{\wordv}{}
      \wordpos[]{\wordu}{1cm}
      \dwordpos{2cm}{}
      \wordupoint{1cm}{}
      \wordvpoint{3cm}{}
    \end{tikzpicture}
  }
  Otherwise we let  be the shortest prefix ranker of
   that is undefined over . Thus  is defined over both words.
  Without loss of generality we assume that . This
  situation is illustrated in Fig.~\ref{fig:rankpos-4}. Notice that 
  does not contain any \texttt{a}'s to the left of , otherwise
   would be defined over . Samson places  in  on , and
  Delilah has to reply with a position right of or equal to .
  Now Lemma \ref{lem:ranker-sides} applies and Samson wins in  more
  moves.\qedconf
\end{proof}


\begin{lem}[position of a ranker]
  \label{lem:ranker-exp}
  Let  be a positive integer and let . There is a formula
  < such that for all  and
  for all ,   .
\end{lem}

\begin{proof}
  As in the proof of Lemma \ref{lem:ranker-def}, it suffices to show that 
  for arbitrary , Samson wins the game
   where initially 
  and . If  is
  defined over , then we can apply Lemma \ref{lem:ranker-sides}
  immediately to get the desired strategy for Samson. Otherwise we use the
  strategy from Lemma \ref{lem:ranker-def}.\qedconf
\end{proof}


\begin{thm}[structure of {\FOVD[]{2}{n}}] \label{thm:ranker-char}
  Let  and  be finite words, and let . The following two
  conditions are equivalent.
  \begin{enumerate}[\em(i)]
  \item
    \begin{enumerate}[\em(a)]
    \item , and,
    \item for all  and , 
    \end{enumerate}
  \item 
  \end{enumerate}
\end{thm}

Notice that condition (i)(a) is equivalent to . 
Instead of proving Theorem \ref{thm:ranker-char} directly, we prove the
following more general version on words with two interpreted variables.


\begin{thm} \label{thm:structure-variables}
  Let  and  be finite words, let , let , and let . The following two conditions are
  equivalent.
  \begin{enumerate}[\em(i)]
  \item
    \begin{enumerate}[\em(a)]
    \item , and,
    \item for all  and , , and,
    \item , and,
    \item for all ,  and 
    \end{enumerate}
  \item 
  \end{enumerate}
\end{thm}

\begin{proof}
  For , (i)(a), (i)(b) and (i)(d) are vacuous, and (i)(c) is equivalent
  to (ii). For , we prove the two implications individually using
  induction on .

  We first show ``''. Assuming
  that (i) holds for  but fails for , we show that  by giving a winning
  strategy for Samson in the \FOVD{2}{n+1} game on the two structures. If (i)(c)
  does not hold, then Samson wins immediately. If (i)(d) does not
  hold for , then Samson wins by Lemma \ref{lem:ranker-sides}. If
  (i)(a) or (i)(b) do not hold for , then one of the following three
  cases applies.
  \begin{enumerate}[(1)]
  \item There is an -ranker that is defined over one word but not
    over the other.
  \item There are two -rankers that do not agree on their ordering in 
    and .
  \item There is an -ranker that does not appear in the same order on
    both structures with respect to a -ranker where .
  \end{enumerate}

  We first look at case (2) where there are two rankers  that disagree on their ordering in  and .
  Without loss of generality
  we assume that  and , and present a winning
  strategy for Samson in the  game. In the first move he places
   on  in . Delilah has to reply with  in , otherwise
  she would lose the remaining -move game as shown in Lemma
  \ref{lem:ranker-sides}. Let  be the -prefix-ranker of
  . We look at two different cases depending on the ordering of
   and .

  \piccaption{\label{fig:char-order-1}Two -rankers appear in different
    order and  ends with .}
  \parpic(7cm,3.3cm)[fr]{
    \begin{tikzpicture}
      \word[6cm]{\wordu}{}
      \word[6cm]{\wordv}{}
      \dwordpos{2cm}{}
      \wordupos{3cm}{}
      \dwordpos{4cm}{}
      \wordvpos{5cm}{}
      \wordupoint{3cm}{}
      \wordvpoint{5cm}{}
      \wordvpoint{4cm}{}
      \wordupoint{1cm}{}
    \end{tikzpicture}
  } 
  
  For , the situation is illustrated in Fig.
  \ref{fig:char-order-1}. In his second move, Samson places  on .
  Delilah has to reply with a position to the left of , but she cannot
  choose any position from the interval  because it
  does not contain the letter . So she has to reply with
  a position left of or equal to , and Samson wins the
  remaining \FOVD{2}{n-1} game as shown in Lemma \ref{lem:ranker-sides}.\bigskip

  \piccaption{\label{fig:char-order-2}Two -rankers appear in different
    order and  ends with .}
  \parpic(7cm,3.3cm)[fr]{
    \begin{tikzpicture}
      \word[6cm]{\wordu}{}
      \word[6cm]{\wordv}{}
      \dwordpos{4cm}{}
      \wordupos{1cm}{}
      \dwordpos{2cm}{}
      \wordvpos{3cm}{}
      \wordupoint{1cm}{}
      \wordvpoint{3cm}{}
      \wordupoint{2cm}{}
      \wordvpoint{5cm}{}
    \end{tikzpicture}
  }

  For , the situation is illustrated in Fig.
  \ref{fig:char-order-2}. In his second move, Samson places pebble  on , and
  Delilah has to reply with a position to the right of , but she cannot
  choose anything from the interval  because this
  section does not contain the letter . Thus she has to reply with
  a position right of or equal to , and Samson wins the
  remaining \FOVD{2}{n-1} game as shown in Lemma \ref{lem:ranker-sides}.
  
  \piccaption{\label{fig:char-final}A letter \texttt{a} occurs between
    -rankers  in  but not in }
  \parpic(5.8cm,3.3cm)[fr]{
    \begin{tikzpicture}
      \word[4cm]{\wordu}{}
      \word[4cm]{\wordv}{}
      \dwordpos{1cm}{}
      \dwordpos{3cm}{}
      \wordupoint{2cm}{}
      \worduletter{2cm}{\texttt{a}}
    \end{tikzpicture}
  }

  \picskip{11}
  Now we look at cases (1) and (3), assuming that case (2) does not apply.
  We know that condition (i) from the statement of the theorem fails, but
  still all -rankers agree on their ordering. In both case (1) and case (3),
  there are two
  consecutive -rankers  with  and a
  letter  such that without loss of generality
  \texttt{a} occurs in the segment  but not in the
  segment . We describe a winning strategy for Samson in
  the game . He places  on an \texttt{a} in the
  segment  of , as shown in Fig.~\ref{fig:char-final}.
  Delilah cannot reply with anything in the interval . If she
  replies with a position left of or equal to , then  is on
  different sides of the -ranker  in the two words. Thus Lemma
  \ref{lem:ranker-sides} applies and Samson wins the remaining -move
  game. If Delilah replies with a position right of or equal to ,
  then we can apply Lemma \ref{lem:ranker-sides} to  and get a winning
  strategy for the remaining game as well. This concludes the proof of
  ``''.

  To show ``(i)  (ii)'', we assume (i) for , and present a
  winning strategy for Delilah in the  game on the two
  structures. In his first move Samson picks up one of the two pebbles, and
  places it on a new position. Without loss of generality we assume that
  Samson picks up  and places it on  in his first move. If  for any ranker , then Delilah replies with
  . This establishes (i)(c) and (i)(d) for , and thus
  Delilah has a winning strategy for the remaining \FOVD{2}{n} game by
  induction.

  If Samson does not place  on any ranker from ,
  then we look at the closest rankers from  to the left and
  right of , denoted by  and , respectively. Let
   and define the -ranker . On  we have . Because
  of (i)(a)  is defined on  as well, and because of (i)(b), we have
  . If  is not contained in the interval
  , then Delilah places  on , which
  establishes (i)(c) and (i)(d) for . Thus by induction Delilah has a
  winning strategy for the remaining \FOVD{2}{n} game.

  \piccaption{\label{fig:rankchar-1} and  are in the same section}
  \parpic(7cm,3.3cm)[fr]{
    \begin{tikzpicture}
      \word[6cm]{\wordu}{}
      \word[6cm]{\wordv}{}
      \dwordpos{1cm}{}
      \dwordpos{4cm}{}
      \dwordpos{5cm}{}
      \wordupoint{2cm}{}
      \wordvpoint{2cm}{}
      \wordupoint{3cm}{}
    \end{tikzpicture}
  }

  \picskip{10}
  If both pebbles  and  occur in the interval , then we need to be more careful. Without loss of generality we
  assume  as illustrated in Fig.~\ref{fig:rankchar-1}. Thus
  Delilah has to place  in the interval  and at
  a position with letter . We define the
  -ranker . From (i)(d) we know that 
  appears on the same side of  in both structures, thus we have . Delilah places her pebble  on , and thus
  establishes (i)(c) and (i)(d) for . By induction, Delilah has a winning
  strategy for the remaining \FOVD{2}{n} game.\qedconf
\end{proof}


A fundamental property of an -ranker is that it uniquely describes a
position in a given word. Now we show that the converse holds as well: Any
position in a word that can be uniquely described with an \FOV[]{2}
formula can also be described by a ranker (Lemma \ref{lem:unique-rankers}).
Furthermore, any \FOV[]{2} formula that describes a unique position in
any given word is equivalent to a boolean combination of rankers (Theorem
\ref{thm:unique-rankers}).

\begin{defi}[unique position formula] \label{def:unique-pos}
  A formula < with  as a free variable is a
  \emph{unique position formula} if for all  there is at
  most one  such that .
\end{defi}


\begin{lem} \label{lem:unique-rankers} 
  Let  be a positive integer and let < be a
  unique position formula. Let  and let 
  such that . Then  for some ranker .
\end{lem}

\begin{full}
\begin{proof}
  Suppose for the sake of a contradiction that there is no ranker  such that . Because the first and
  last positions in  are described by 1-rankers, we know that .
  We construct a new word  by doubling the symbol at position  in ,
  . By assumption,
  there is no -ranker that describes position  in . A brief
  argument by contradiction shows that there are also no -rankers that
  describe positions  or  in : Assuming that such a ranker
  exists, let  be the shortest such ranker. Thus none of the prefix
  rankers of  point to either positions  or  in . This means
  that all prefix rankers of  are interpreted in exactly the same way on
  both  and , and irrespective of whether  points to  or
  , we have have , a
  contradiction. Hence all -rankers are insensitive to the doubling of
  , and the two
  words  and  agree on the definedness of all -rankers and on their
  ordering. 
  By
  Theorem \ref{thm:structure-variables}, we thus have , which contradicts the fact that 
  is a unique position formula.\qedconf
\end{proof}
\end{full}


\begin{thm} \label{thm:unique-rankers}
  Let  be a positive integer and let < be a
  unique position formula. There is a , and there
  are mutually exclusive formulas < and rankers
   such that
  
  where < is the formula from Lemma
  \ref{lem:ranker-exp} that uniquely describes the ranker .
\end{thm}

\begin{full}
\begin{proof}
  Let  be the set of all \FOVD[]{2}{n} types of words over
   with one interpreted variable. Because there are only finitely
  many inequivalent formulas in \FOVD[]{2}{n},  is finite. Let
   be the set of all types that satisfy
  . We set  and let < be a description of type .  Thus .

  Now suppose that . Thus 
  for some . By Lemma \ref{lem:unique-rankers}  for some . Thus  since  and  is a complete
   formula. Thus  so
   is in the desired form.\qedconf
\end{proof}
\end{full}


\section{Alternation hierarchy for \texorpdfstring{\FOV[]{2}}{FOV[<]2}} \label{sec:FO2-alt}

We define alternation rankers and prove our structure theorem (Theorem
\ref{thm:alt-ranker-char}) for <. Surprisingly the number
of alternating blocks of  and  in the rankers corresponds
exactly to the number of alternating quantifier blocks. The main ideas from
our proof of Theorem \ref{thm:ranker-char} still apply here, but keeping
track of the number of alternations does add complications.

\begin{defi}[-alternation -ranker] \label{def:alt-ranker} Let
   with . An \emph{-alternation -ranker}, or
  -ranker, is an -ranker with exactly  blocks of boundary
  positions that alternate between  and .
\end{defi}


We use the following notation for alternation rankers.


\begin{full}
\begin{lem} \label{lem:alt-ranker-sides} Let  and  be positive
  integers with , let , and let . Samson wins the game 
  where initially .

  Furthermore, Samson can start the game with a move on  if  ends with
  ,  and , or if  ends with
  ,  and . He can start the game with
  a move on  if  ends with ,  and , or if  ends with ,  and .
\end{lem}

\begin{proof}
  If , then we can immediately apply the base case from the proof
  of Lemma \ref{lem:ranker-sides}. Samson wins in one move, placing his
  pebble on  or  as specified.

  For the remaining cases, we assume without loss of generality that 
  ends with  and that  and . Let
   be the -prefix ranker of . This situation is
  illustrated in Fig.~\ref{fig:ranker-sides-2} of Lemma
  \ref{lem:ranker-sides}. Samson places  on , and creates a
  situation where  and . If
   ends with , then by induction Samson wins the remaining
  \FOVDA{2}{n-1}{m-1} game and thus he has a winning strategy for the
  \FOVDA{2}{n}{m} game. If  ends with , then by induction
  Samson wins the remaining \FOVDA{2}{n-1}{m} game starting with a move on
  , and thus he has a winning strategy for the \FOVDA{2}{n}{m} game.\qedconf
\end{proof}

\begin{lem} \label{lem:alt-ranker-def} Let  and  be positive
  integers with  and let . There is a 
  < such that for all ,
  .
\end{lem}

\begin{proof}
  Using Lemma  it suffices to consider arbitrary  with  and , and
  using Fact \ref{fact:ef1}, it suffices to show that Samson wins the game
  . 
  Let  be the shortest prefix ranker
  of  that is undefined over , and we assume without loss of
  generality that this ranker ends with the boundary position  for some . This situation is illustrated in Fig.
  \ref{fig:rankpos-4} for Lemma \ref{lem:ranker-exp}. In his first move
  Samson places  on  and thus forces a situation where  and . If  ends with ,
  then according to Lemma \ref{lem:alt-ranker-sides}, Samson wins the
  remaining \FOVDA{2}{n-1}{m} game starting with a move on . Otherwise
   ends with , and thus by Lemma
  \ref{lem:alt-ranker-sides} Samson wins the remaining
  \FOVDA{2}{n-1}{m-1} game
  starting with a move on .\qedconf
\end{proof}

\begin{lem} \label{lem:alt-ranker-exp} Let  and  be positive
  integers with  and let . There is a formula
  < such that for all 
  and for all , .
\end{lem}

\begin{proof}
  As in the proof of Lemma \ref{lem:alt-ranker-def}, it suffices to show
  that Samson wins the game
   where initially 
   and . Depending on
  whether  is defined over , we use the strategies from Lemma
  \ref{lem:alt-ranker-sides} or Lemma \ref{lem:alt-ranker-def}.\qedconf
\end{proof}
\end{full}


\begin{thm}[structure of {\FOVDA[]{2}{n}{m}}]
  \label{thm:alt-ranker-char}
  Let  and  be finite words, and let  with . The
  following two conditions are equivalent.
  \begin{enumerate}[\em(i)]
  \item
    \begin{enumerate}[\em(a)]
    \item , and,
    \item for all  and for all , we have\\
      , and,
    \item for all  and 
      such that  and  end with different directions,
      
    \end{enumerate}
  \item 
  \end{enumerate}
\end{thm}

Just as before with Theorem \ref{thm:ranker-char}, instead of proving
Theorem \ref{thm:alt-ranker-char} directly, we prove a more general version
that applies to words with two interpreted variables. The statement of the
general version is asymmetric with respect to the roles of the two
structures  and . This is necessary because of the correspondence
between quantifier alternations (i.e. alternations between  and  in
the game) and alternations of directions in the rankers. This asymmetry
already affected the statement of Lemma \ref{lem:alt-ranker-sides}, where
Samson's winning strategy starts with a move on the specified structure. In
fact, as the proof of the following theorem shows, he does not have a
winning strategy that starts with a move on the other structure. We remark
that conditions (i)(a) through (i)(e) of the general theorem are completely
symmetric with respect to the roles of  and , and only conditions
(i)(f) and (ii) are asymmetric. Theorem \ref{thm:alt-ranker-char} follows
directly from the general theorem, since here ,
thus conditions (i)(e) and (i)(f) or trivially true, and the equivalence
holds with the roles of  and  reversed as well.
\pagebreak[4]

\begin{thm}
  Let  and  be finite words, let , let , and let  with . The following two
  conditions are equivalent.

  \begin{enumerate}[\em(i)]
  \item
    \begin{enumerate}[\em(a)]
    \item , and,
    \item for all  and for all , we have\\
      , and,
    \item for all  and 
      such that  and  end with different directions,
      
    \item , and,
    \item for all ,  and , and,
    \item for all , and ,
      \begin{enumerate}[\em(f)]
      \item if  ends on  and , then 
      \item if  ends on  and , then 
      \item if  ends on  and , then 
      \item if  ends on  and , then 
      \end{enumerate}
    \end{enumerate}
  \item Delilah wins the game < if Samson starts with a move on .
  \end{enumerate}
\end{thm}

\begin{full}
\begin{proof}


  As in the proof of Theorem \ref{thm:ranker-char}, we use induction on .
  For , condition (i)(d) just by itself is equivalent to (ii), and
  all other conditions of (i) are vacuous. For , we we first show
  `` (i)   (ii)''.

  Suppose that (i) holds for , but fails for . If (i)(d)
  does not hold then Samson wins immediately. If (i)(e) does not hold for
  , then by Lemma \ref{lem:alt-ranker-sides}, Samson wins the
  -game on , starting with a move on either  or . If
  Samson can start with a move on , we have established that (ii) is
  false. Otherwise, we reverse the roles of  and , and observe that
  condition (i)(e) still remains the same. Thus, even if Samson needs to
  start with a move on , he still has a winning strategy, and (ii) does
  not hold for . If (i)(f) does not hold for , then again
  by using Lemma \ref{lem:alt-ranker-sides}, Samson wins the -game
  on  starting with a move on .

  If one of (i)(a), (i)(b) or (i)(c) fail, then we show that Samson has a
  winning strategy for
  the game . We observe that it does not matter what
  structure Samson chooses for his first move, since all of (i)(a), (i)(b)
  and (i)(c) are completely symmetric with respect to the roles of  and
  . Thus if Samson's winning strategy starts with a move on , we can
  reverse the roles of  and  and get a winning strategy starting with
  move on . One of the following cases applies.
  \begin{enumerate}[(1)]
  \item There is a ranker  that is defined over one
    structure but not over the other.
  \end{enumerate}
  This first case applies if (a) fails for . If condition (2) fails
  for , then there are two -rankers for which it fails, or an
  -ranker and an -ranker. This leads to the following two cases.
  \begin{enumerate}[(1)]
    \setcounter{enumi}{1}
  \item There are two rankers  and 
    that disagree on their order, i.e. . 
  \item There are two rankers  and 
    that disagree on their order.
  \end{enumerate}
  In a similar fashion, we obtain the remaining two cases if condition (3)
  fails for .
  \begin{enumerate}[(1)]
    \setcounter{enumi}{3}
  \item There are rankers  that end on different
    directions and disagree on their order.
  \item There are rankers  and  that end
    on different directions and disagree on their order.
  \end{enumerate}


  We look at the
  cases (2) and (4) first, then deal with case (1) assuming that cases (2)
  and (4) do not apply, and finally look at cases (3) and (5).

  For case (2), we assume that , as
  illustrated in Fig.~\ref{fig:alt-rankchar-order}. The situation for
   is completely symmetric. Depending on the last boundary
  position of , one of the following two subcases applies.

  \begin{enumerate}[]
    \piccaption{\label{fig:alt-rankchar-order} and  appear in different
      order}
    \parpic(5cm,2.7cm)[fr]{
      \begin{tikzpicture}
        \word[4cm]{\wordu}{}
        \word[4cm]{\wordv}{}
        \wordupos{1cm}{}
        \dwordpos{2cm}{}
        \wordvpos{3cm}{}
      \end{tikzpicture}
    }
  \item
     ends with . Samson places  on  in his first
    move. If Delilah replies with a position to the left of  
    or equal to , then . Thus we
    can apply Lemma \ref{lem:alt-ranker-sides} to get a winning strategy for
    Samson in the remaining \FOVDA{2}{n}{m} game that starts with a move on .
    If Delilah replies with a position to the right of , Samson has a
    winning strategy for the remaining \FOVDA{2}{n}{m-1} game. Thus we have a
    winning strategy for Samson in the \FOVDA{2}{n+1}{m} game.
  \item
    \picskip{0}
     ends with . This is similar to the previous case, but
    now Samson places  on  in his first move. If Delilah replies
    with a position to the right of , or equal to ,
    then as above we get a winning
    strategy for Samson in the remaining \FOVDA{2}{n}{m} game that starts with a
    move on . Otherwise we get a winning strategy for Samson with only
     alternations for the remaining game. Thus again he has a winning
    strategy for the \FOVDA{2}{n+1}{m} game.
  \end{enumerate}

  \picskip{0}
  For case (4), Samson's winning strategy is very similar to the previous
  case. If  and  ends with , then Samson places
   on  in his first move. If Delilah replies with a position to the
  right of , then Samson's winning strategy is as above. Otherwise 
  is on different sides of  and Samson has a winning strategy for the
  remaining \FOVDA{2}{n}{m} game 
  that starts with a move on . All together, he
  has a winning strategy for the \FOVDA{2}{n+1}{m} game.
  The remaining three cases (ordering of  and  and ending
  direction of ) work in the same way.

  Similar to what we did in the proof of Theorem \ref{thm:ranker-char}, we
  can reduce the remaining cases to an easier situation where a certain
  segment contains a certain letter in one structure, but not in the other
  structure, and then apply Lemma \ref{lem:alt-ranker-sides} to obtain a
  winning strategy for Samson.\pagebreak[2]

  To deal with case (1), we assume that the previous two cases, (2) and (4),
  do not apply. Without loss of generality, say that the -ranker  is
  defined over  but not over . Let  be the
  letter in  at position . We define the following sets of rankers.
  
  Notice that all rankers from  appear to the left of all rankers
  from  in . From the inductive hypothesis, and from the fact that
  both cases (2) and (4) do not apply, it follows that over , all rankers
  from  appear to the left of all rankers from  as well.
  However, the rankers from  and  by themselves do
  not necessarily appear in the same order in both structures. We look at
  the ordering of these rankers in , and let  be the rightmost
  ranker from  and  be the leftmost ranker from .
  By construction, we have , so
  the segment  in  contains the letter . Let
   be the -prefix-ranker of , and observe that  is defined
  on both structures and that  is contained in either  or
  . Because  is not defined on , the letter  does not
  occur in  either to the right of  if , 
  or to the left of
   if . Thus the segment  does not contain the
  letter  in .

  \piccaption{\label{fig:alt-ranker-char-middle}A letter occurs between
    rankers ,  in  but not in }
  \parpic(5.7cm,3.3cm)[fr]{
    \begin{tikzpicture}
      \word[4cm]{\wordu}{}
      \word[4cm]{\wordv}{}
      \dwordpos{1cm}{}
      \dwordpos{3cm}{}
      \wordupoint{2cm}{}
      \worduletter{2cm}{\texttt{a}}
    \end{tikzpicture}
  }

  Now we know that  occurs in the segment  in 
  but not in , and thus we have established the situation illustrated in
  Fig.~\ref{fig:alt-ranker-char-middle}. Samson places his first pebble on
  an  within this section of , and Delilah has to reply with
  a position outside of this section. No matter what side of the segment
  she chooses, with Lemma \ref{lem:alt-ranker-sides} Samson has a winning
  strategy for the remaining game and thus wins the \FOVDA{2}{n+1}{m} game.
  
  \picskip{3}
  In cases (3) and (5), we again assume that cases (2) and (4) do not apply, and
  we look at the same sets of rankers,  and , and
  at , the -prefix-ranker of . We assume that 
  and that  ends with , all three other cases are completely
  symmetric. Notice that  is an -ranker, or an
  -ranker that ends with . Thus both structures agree on the
  ordering of  and . The relative positions of all these rankers
  are illustrated in Fig.~\ref{fig:alt-ranker-char-closest}. As above, let
   be the rightmost ranker from  and let  be the
  leftmost ranker from , with respect to the ordering of these rankers
  on . Again we know that  and therefore the
  segment  of  contains an . Notice that  and , thus . Thus the segment  does not contain the letter
   in , providing Samson with a winning strategy as argued above.

  \piccaption{\label{fig:alt-ranker-char-closest}Ranker positions, case (4)}
  \parpic(6.2cm,2.5cm)[fr]{
    \begin{tikzpicture}
      \word[5cm]{\wordu}{}
      \word[5cm]{\wordv}{}
      \wordupos{2cm}{}
      \dwordpos{3cm}{}
      \wordvpos{4cm}{}
      \dwordpos{1cm}{}
    \end{tikzpicture}
  }

  To prove ``(i)  (ii)'', we assume that the theorem holds for
  , and that (i) holds for , and we present a winning strategy
  for Delilah in the game  where Samson starts with
  a move on .

  \picskip{3}
  If Samson places  on a ranker , then Delilah
  replies by placing  on the same ranker on . Since (i)(b) holds for
  , this establishes (i)(e) and (i)(f) for . It also
  establishes (i)(e) and (i)(f) for  with reversed roles of  and
  . Thus we can apply the inductive hypothesis to get a winning strategy
  for Delilah in the remaining game.

  If  after Samson's first move, then Delilah replies with . We use the inductive hypothesis to argue that Delilah wins the
  remaining -move game, no matter what structure Samson chooses for his
  next move. If he chooses to play on , then the remaining game is an
  -game. Since in the first move Delilah set , we have
  (i)(e) and (i)(f) for , and thus the inductive hypothesis applies
  and Delilah wins the remaining game. On the other hand, if Samson chooses
  to play on  for the next move, the remaining game is an -game,
  since he started with a move on . Because Delilah set  in
  the first move, (i)(e) for  implies both (i)(e) and (i)(f) for
   with reversed roles of  and . Thus we can again use the
  inductive hypothesis to get a winning strategy for Delilah in the
  remaining game.
  
  Otherwise we assume that  after Samson's first move, the case
  for  is completely symmetric. We look at the following two sets
  of rankers.
  
  On , all rankers from  occur to the left of all rankers from
  . Since (i)(c) holds for
  , this is also true for the positions of these rankers on .
  Let  be the letter Samson places his pebble on. To establish
  both (i)(e) and (i)(f) for , Delilah needs to find an 
  in  that is to the right of all rankers from  and to the left
  of all rankers from . We define 
  
  and have Delilah place her pebble  on the rightmost ranker from
   on . This position of course is labeled with an .
  Since on  all rankers from  occur to the left of or at ,
  all of them occur strictly to the left of . Since all rankers in
   are from  or ,
  we can apply (i)(e) and (i)(f), and we see that all of these rankers
  also appear to the left of . Therefore we have , which
  makes sure that Delilah does not lose in this move, and also establishes
  (i)(d).
  
  To complete the inductive step, we need to argue that Delilah's move also
  establishes (i)(e) and (i)(f), both for , and for  with
  reversed roles of  and . Then, using the inductive hypothesis,
  Delilah has a winning strategy for the remaining game, no matter what side
  Samson chooses for his next move.

  We observe that all rankers from  appear to the right of the
  rankers from . This is true by definition on , and holds for 
  because (i)(b) and (i)(c) hold for . Since Delilah placed 
  on a ranker from , we have (i)(e), (i)(f) and (i)(f) for
   for all all rankers from . And since Delilah placed  on
  the rightmost of the rankers from , we know that all rankers from
   appear to the left of , just as they do on . Thus we have
  (i)(e), (i)(f) and (i)(f) for the rankers from  as well, and
  therefore for all rankers mentioned in those conditions.

  All rankers from  that appear at  are in
  , since we already dealt with the case where  does
  appear at a ranker from . Since Delilah chose  as
  the rightmost ranker from , all of these rankers appear to the
  left of or at , and we have established (i)(f) for . For
  condition (i)(f), we need to argue about . From (i)(b) and
  (i)(c) for , we know that all rankers from  appear to the
  right of or at the same position as the rankers from  on ,
  just as they do on . Thus (i)(f) holds as well.

  Now that we have established (i) for , we use the inductive
  hypothesis to get a winning strategy for Delilah for the remaining game if
  Samson's next move is on . For the case where his next move is on ,
  we only need to establish (i) for , but with reversed roles of
   and . Reversing the roles of the two structures only affects
  condition (i)(f), and (i)(f) for  follows immediately from (i)(e)
  for . Thus Delilah also wins the remaining game if Samson's next
  move is on .
\end{proof}
\end{full}


Using Theorem \ref{thm:alt-ranker-char}, we show that for any fixed alphabet
, at most  alternations are useful. Intuitively, each
boundary position in a ranker says that a certain letter does not occur in
some part of a word. Alternations are only useful if they visit one of these
previous parts again. Once we visited one part of a word  times,
this part cannot contain any more letters and thus is empty.

\begin{thm} \label{thm:alt-alphabet}
  Let  be a finite alphabet, let  and . If , then .
\end{thm}

\begin{full}
\begin{proof}
  Suppose for the sake of a contradiction that 
  and . Thus, using Theorem \ref{thm:alt-ranker-char},
   and  agree on the definedness of all -rankers, and
  on their order with respect to all -rankers and some
  -rankers. But since ,  and 
  need to disagree on the properties of some other ranker. Let  with  be the shortest such ranker. We know that
   has more than  blocks of alternating directions, say  is
  an -alternation ranker for some . Let  be the indices of the boundary positions at the end of each
  block, i.e. where ,  points to a different direction
  than . For the last of those indices we have .

We look at the prefix rankers of  up to the
  end of each alternating block, ,
  and the
  intervals defined by these prefix rankers. We set ,
   if  points to the right, and  if 
  points to the left. For all  let,
  
  Notice that by definition the letter mentioned in  does not occur
  in the interval .

  Suppose that for all  we have .  Then the
  letter mentioned in  has to occur in the interval  of , but
  the interval  of  cannot contain any of the 
  distinct letters. Therefore  and
  we have a contradiction.

  Otherwise there is an  such that . We will construct a ranker  that is shorter than ,
  does not have more alternations than  and occurs at exactly the same
  position as  in both  and . The main idea for this construction
  is that if , then it is not useful to enter
  this interval at all. By our assumption,  and 
  disagree on some property of the ranker , and thus on some property of
  the shorter ranker . This contradicts our assumption that  was the
  shortest such ranker.

  Now we show how to construct a shorter ranker  that occurs at the same
  position as . We assume
  without loss of generality that  points to the left. In this case
  we have 
  .
  We look at the
  relative positions of the rankers  with
  respect to the ranker . We know that , and we are interested in the right-most of the rankers
   that is still outside of the interval
  . Let  be this ranker.
  Thus we have
  
  We know that , thus by Theorem
  \ref{thm:alt-ranker-char}, these rankers occur in exactly the same order
  in . Now we set .
  Because  and  agree on the ordering of the relevant rankers, we have
   and . Therefore we have reduced the
  size of a prefix of  without increasing the number of alternations, and
  thus have a shorter ranker  that occurs at the same position as  in
  both structures.\qedconf
\end{proof}
\end{full}


In order to prove that the alternation hierarchy for  is strict,
we define example languages that can be separated by a formula of a
given alternation depth , but that cannot be separated by any formula of
lower alternation depth. As Theorem \ref{thm:alt-alphabet} shows,
we need to increase
the size of the alphabet with increasing alternation depth.
We inductively define the example words  and  and the
example languages  and  over finite alphabets . Here ,  and  are positive integers.

Notice that  and  are almost identical -- if we delete
only one  from , we get . Finally, we set  and .

\begin{defi}
  A formula  \emph{separates} two languages  if for all  we have  and for all
   we have  or vice versa.
\end{defi}


\begin{full}

\begin{lem} 
  \label{lem:alt-hierarchy-exp}
  For all , there is a formula < that
  separates  and . 
\end{lem}

\begin{proof}
  For , we can easily separate  and  with the formula .
  For all larger , we show that the two languages  and  differ on
  the ordering of two -alternation rankers. Then by Theorem
  \ref{thm:alt-ranker-char} there is an \FOVDA[]{2}{m}{m} formula that
  separates  and . We inductively define the rankers
  

  For , it is easy to see that , but
  . For , these rankers disagree on
  their order as well. To prove this, we prove the following two equalities.
  
  To prove this, we first use the definitions above and write
  
  The letter  does not occur in the word ,
  and thus  points to the first
  position in  right after the copy of . We observe
  that  starts with , and that  is defined on
  . Thus the evaluation of the remainder of  on
   never leaves the copy of , and we have
  
  For the second part of the equality, we have
  
  As above, the letter  does not occur in the word
  , and thus  points to the
  position in  right before the copy of . The ranker
   starts with , and  is defined on .
  Thus, just as above, the evaluation of the remainder of  on
   never leaves the copy of , and we have
  
  Exactly the same holds for the other rankers () and words
  ). We have
  

  Now an easy inductive argument, based on the two equalities we just
  proved, shows that the rankers disagree on their order. Therefore
  condition (i)(b) of Theorem \ref{thm:alt-ranker-char} fails for any pair
  of words, and there is a formula in \FOVDA[]{2}{m}{m} that separates
   and .\qedconf
\end{proof}\pagebreak[2]

\begin{lem} 
  \label{lem:alt-hierarchy-nexp}
  For , ,
  and all , we have .
\end{lem}

\begin{proof}
  Because we do not have constants, there are no quantifier-free sentences.
  Thus < 
  does not contain any formulas and the statement holds
  trivially for .

  For  and any , we claim that exactly the same
  -rankers are defined over  and , and that all
  -rankers appear in the same order with respect to all
  -rankers and all -rankers that end on a different
  direction. Once we established this claim, the lemma follows immediately
  from Theorem \ref{thm:alt-ranker-char}. We already observed that 
  and  are almost identical. The only difference between the two
  words is that  contains the letter  in the middle whereas
   does not. Thus we only have to consider rankers that are
  affected by this middle .

  We claim that any ranker that points to the middle  of 
  requires at least  alternations. Furthermore, we claim that any such
  ranker needs to start with  for even  and with  for
  odd . We prove this by induction on .

  For  we have . Any -ranker that starts
  with  cannot reach the first , thus we need a ranker that
  starts with .

  For odd  we have . Any
  -ranker that starts with  cannot leave the first block of  symbols of this word and thus not reach the middle .
  Therefore we need to start with , and in fact use
   at some point, because we would not be able to leave
  the last section of  otherwise. But with  we
  move past all of , and we need one alternation to turn around
  again. By induction, we need at least  alternations within
  , and thus  alternations total.

  The argument for even  is completely symmetric. Thus we showed that we
  need at least  alternation blocks to point to the middle .
  Furthermore, we showed that if we have exactly  alternation blocks,
  then the last of these blocks uses .
  Therefore we only need to consider
  -alternation rankers that end on  and pass through the
  middle . It is easy to see that all of these rankers agree on their
  ordering with respect to all other -alternation rankers, and with
  respect to all -alternation rankers that end on .

  To summarize, we showed that  and  satisfy condition (i)
  from Theorem \ref{thm:alt-ranker-char} for  alternations. Thus the
  two words agree on all formulas from \FOVDA[]{2}{n}{m-1}.\qedconf
\end{proof}
\end{full}

\begin{conference}
  Our example words are constructed such that for ,  and
   can be distinguished by the ordering of two -rankers. In
  the case  for example, we can use the two rankers  and . A
  formal argument for all  is given in \cite{WI07}. There we also argue
  that the example words  and 
  agree on the definedness of all -rankers, and that these rankers
  appear in exactly the same order with respect to shorter rankers. Thus
  the two languages  and  cannot be separated by any
  < formula. Thus we have the following theorem.
\end{conference}

\begin{thm}[alternation hierarchy for {\FOV[]{2}}]
  \label{thm:alt-hierarchy}
  For any positive integer , there is a <
  and there are two languages  such that  separates
   and , but no < separates  and
  .
\end{thm}

\begin{full}
\begin{proof}
  The theorem immediately follows from Lemma \ref{lem:alt-hierarchy-exp}
  and Lemma \ref{lem:alt-hierarchy-nexp}.\qedconf
\end{proof}
\end{full}


Theorem \ref{thm:alt-hierarchy} resolves an open question from
\cite{EVW97,EVW02}.


\section{Structure Theorem and Alternation Hierarchy for \texorpdfstring{\FOV[]{2}}{FOV[<,suc]{2}}}
\label{sec:FO2-suc}

We extend our definitions of boundary positions and rankers from 
Sect.~\ref{sec:FO2} to include the substrings of a given length that occur
immediately before and after the position of the ranker.

\begin{defi}
  A \emph{-neighborhood boundary position} denotes the first or
  last occurrence of a substring in a word. More precisely, a
  -neighborhood boundary position is of the form 
  with , ,  and . The interpretation of a -neighborhood boundary
  position  on a word  is defined
  as follows.
  
  Notice that  is undefined if the sequence 
  does not occur in
  . A -neighborhood boundary position can also be specified
  with respect to a position .
  
\end{defi}

Observe that -neighborhood boundary positions are identical to the
boundary positions from Definition \ref{def:boundary-position}. As before in
the case without successor, we build rankers out of these boundary
positions. The size of the boundary position neighborhoods grows linearly
from the first boundary position to the last one, reflecting the remaining
quantifier depth for successor moves at those positions.

\begin{defi}
  An \emph{-successor-ranker}  is a sequence of  neighborhood
  boundary positions, , where  is a
  -neighborhood boundary position and . The interpretation of an -successor-ranker  on a word
   is defined as follows.
  
  We denote the set of all -successor-rankers that are defined over a
  word  by , and set .
\end{defi}

Because we now have the additional atomic relation , we need to extend
our definition of order type as well.

\begin{defi} \label{def:suc-order-type} Let . The
  \emph{successor order type} of  and  is defined as
  
\end{defi}

With this new definition of -successor-rankers, our proofs for Lemmas
\ref{lem:ranker-sides}, \ref{lem:ranker-def}, \ref{lem:ranker-exp} and
Theorem \ref{thm:ranker-char} go through with only minor modifications.
Instead of working through all the details again, we simply point out the
differences.

First we notice that -successor-rankers are simply -rankers, so the
base case of all inductions remains unchanged. In the proofs of Lemmas
\ref{lem:ranker-sides}, \ref{lem:ranker-def} and \ref{lem:ranker-exp}, and
in the proof of ``(ii)  (i)'' from Theorem \ref{thm:ranker-char},
we argued that Delilah cannot reply with a position in a given section
because it does not contain a certain ranker and therefore it does not
contain the symbol used to define this ranker. Now we need to know more --
we need to show that Delilah cannot reply with a certain letter in a given
section that is surrounded by a specified neighborhood, given that this
section does not contain the corresponding successor-ranker. Whenever
Samson's winning strategy depends on the fact that an -successor-ranker
does not occur in a given section, he has  additional moves left. So if
Delilah does not reply with a position with the same letter and the same
neighborhood, Samson can point out a difference in the neighborhood with at
most  additional moves.

For the other direction of Theorem \ref{thm:ranker-char}, we need to make
sure that Delilah can reply with a position that is contained in the correct
interval, has the same symbol and is surrounded by the same neighborhood.
Where we previously defined the -ranker  or , we now include the
-neighborhood of the respective positions chosen by Samson. Thus we
make sure that Samson cannot point out a difference in the two words, and
Delilah still has a winning strategy. Thus we have the following three
theorems for \FOV[]{2}.


\begin{thm}[structure of {\FOVD[]{2}{n}}]
  \label{thm:suc-structure}
  Let  and  be finite words, and let . The following two
  conditions are equivalent.
  \begin{enumerate}[\em(i)]
  \item
    \begin{enumerate}[\em(a)]
    \item , and,
    \item for all  and for all ,\\
      
    \end{enumerate}
  \item 
  \end{enumerate}
\end{thm}

\begin{thm}[structure of {\FOVDA[]{2}{n}{m}}]
  \label{thm:suc-alt-structure} 
  Let  and  be finite words, and let  with . The
  following two conditions are equivalent.
  \begin{enumerate}[\em(i)]
  \item
    \begin{enumerate}[\em(a)]
    \item , and,
    \item for all  and for all ,\\
      , and,
    \item for all  and  such that  and  end with
      different directions, 
    \end{enumerate}
  \item 
  \end{enumerate}
\end{thm}

\begin{thm}[alternation hierarchy for {\FOV[]{2}}]
  \label{thm:suc-alt-hierarchy}
  Let  be a positive integer. There is a
  <,\suc and there are two languages  such that  separates  and ,
  but there is no <,\suc that separates  and
  .
\end{thm}


\begin{full}
\begin{proof}
  We use the same ideas as before in Theorem \ref{thm:alt-hierarchy}. We
  define example languages that now include an extra letter \texttt{b} to ensure
  that the successor predicate is of no use. As before, we inductively
  construct the words  and  and use them to define the
  languages  and .

   Finally we set  and . Notice that the \texttt{b}s are not necessary to
    distinguish between the two languages  and , and thus
    the proof of Lemma \ref{lem:alt-hierarchy-exp} goes through
    unchanged and we have a formula <,\suc that separates  and . To see that
    no <,\suc formula can separate  and ,
    we observe that any -neighborhood in the words 
    and  contains all \texttt{b}s except for at most one
    letter  for some . Thus the proof of
    Lemma \ref{lem:alt-hierarchy-nexp} goes through here as
    well.\qedconf
\end{proof}
\end{full}

\begin{conference}
  The proof of Theorem \ref{thm:suc-alt-hierarchy} is given in \cite{WI07},
  and mostly identical to the proof of Theorem \ref{thm:alt-hierarchy}. We use
   copies of an extra letter between any two letters in our example words,
  and thus ensure that the successor predicate is not useful.
\end{conference}


\section{Small Models and Satisfiability for \texorpdfstring{}{FOV2[<]}}
\label{sec:satisfiability}

\newcommand{\TILING}{\lang{TILING}}

The complexity of satisfiability for  was investigated in
\cite{EVW02}. There it is shown that any satisfiable 
formula has a model of size at most exponential in . It follows that
satisfiability for  is in \NEXP, and a reduction from \TILING{}
shows that satisfiability for  is \NEXP-complete. Using our
characterization of , Wilke
observed that satisfiability becomes -complete if we look at binary
alphabets only \cite{W07}. We generalize this observation and show that
satisfiability for  is -complete for any fixed alphabet
size. In contrast to this, satisfiability for  is
-complete even for binary alphabets \cite{EVW02}, since in the
presence of a successor predicate we can encode an arbitrary alphabet in
binary. Before we state and prove the two theorems of this section, we
prove a simple technical lemma first.

\begin{lem}
  \label{lem:equiv-word-replace}
  Let . If , then .
\end{lem}

\begin{proof}
  We argue that Delilah has a winning strategy for the game
  : If Samson places a pebble in  or , Delilah
  replies with the identical position in  or  in the other structure.
  If Samson places a pebble in  or , Delilah replies according to her
  winning strategy in the game . All of these moves
  obviously preserve the ordering of the pebbles, and thus Delilah wins.
\end{proof}


\begin{thm}[Small Model Property for Bounded Alphabets]
  \label{thm:small-model}
  Let  and let  be a formula over a -letter alphabet. If  is
  satisfiable, then  has a model of size .
\end{thm}

\begin{conference}
  The proof of Theorem \ref{thm:small-model} is presented in \cite{WI07}. We
  argue that any fixed word has as most  positions that can be
  reached with -rankers, and thus we have a word of size  that
  satisfies the given formula.
\end{conference}

\begin{full}
\begin{proof}
  Let  be an arbitrary model of . We use induction on  to
  show how to construct a new model of size  that satisfies
  . For , i.e. a single letter alphabet, 
  we observe that an -ranker can only point to a
  position within the first or last  letters of . We let  be a
  copy of  with all letters after the first  letters and before the
  last  letters removed. The words  and  agree on the existence
  and ordering of all -rankers, thus we can apply Theorem
  \ref{thm:ranker-char} and it follows that .

  For the inductive case, we partition  into segments, where each segment
  is a maximal sequence of the same letter. For example, the word
   has two segments,  and .
  First, we let  be a copy of  where we cut down all segments that
  are longer than  to exactly  letters. Since no -ranker can
  point to a position within any segment after the first  letters and
  before the last  letters of that segment, we have .

  Now we partition the word  such that , where  and for every ,  is a
  string of maximal length that uses exactly  different letters,  is
  a segment, and  is a string over at most a -letter alphabet.
  We observe that this partition is unique: If  is the last
  of the  letters in our alphabet to appear in , starting from
  the left, then  is the left-most segment of 's, and 
  is everything up to that segment. Now  is the left-most segment after
   of the letter that appears last after , and so on. We can point
  to a position in segment  with an -ranker, but no -ranker that
  starts with  can point to a position to the right of .
  Similarly, we partition , now starting from the right, such that , where  and for every ,  is a string of maximal length that uses exactly 
  different letters,  is a segment, and  is a string over at
  most a -letter alphabet. Again, this partition is unique and any
  -ranker that starts with  cannot point to a position to the
  left of . We also notice that both partitions have the same number
  of segments, i.e. , since any substring  from the first
  partition contains all letters of the alphabet and thus has to contain
  at least one segment  from the second partition, and vice versa.

  If both partitions use more than  segments, then the segment
   of the first partition occurs to the left of the segment  of
  the second partition. In this case we construct the word .  agrees with  on
  all -rankers, and thus . Every one of the strings
   and  uses at most  different
  letters, therefore we can apply the inductive hypothesis and replace each
  of these strings with an equivalent string of length , as
  explained in Lemma \ref{lem:equiv-word-replace}. Thus we
  have constructed a word of length  that satisfies .

  If the partitions have at most  segments, then we combine the
  two partitions such that , where , and for every ,  is one of the original
  segments  and . As above, we use the
  inductive hypothesis to replace all strings  with equivalent strings
  of length , and thus construct a new string of length 
  that satisfies .\qedconf
\end{proof}
\end{full}


\begin{thm}
  Satisfiability for  where the size of the alphabet is bounded
  by some fixed  is \NP-complete.
\end{thm}

\begin{proof}
  Membership in  follows immediately from Theorem \ref{thm:small-model}
  -- we nondeterministically guess a model of size  where  is the
  quantifier depth of the given formula, and verify that it is a model of
  the formula. Now we give a reduction from . Let  be a
  boolean formula in conjunctive normal form over the variables . We construct a  formula , where  says that every model has
  size exactly , and where we replace every occurrence of  in
   with a formula  of length  which says that the
  -th letter is a . The total length of  is , and  is satisfiable iff  is satisfiable.\qedconf
\end{proof}


\section{Conclusion}

We proved precise structure theorems for \FOV{2}, with and without the
successor predicate, that completely characterize the expressive power of
the respective logics, including exact bounds on the quantifier depth and on
the alternation depth. Using our structure theorems, we showed that the
quantifier alternation hierarchy for \FOV{2} is strict, settling an open
question from \cite{EVW97,EVW02}. Both our structure theorems and the
alternation hierarchy results add further insight to and simplify previous
characterizations of \FOV{2}. We hope that the insights gained in
our study of  on words will be useful in
future investigations of the trade-off between formula size and number of
variables.


\section*{Acknowledgment}

We would like to thank Thomas Wilke for pointing out the consequences of our
structural results to the satisfiability problem for . We are
also very thankful to two anonymous reviewers, whose detailed comments and
suggestions significantly improved the presentation of our results.

\begin{thebibliography}{10}

\bibitem{AI03}
{\sc Adler, M., and Immerman, N.}
\newblock An  lower bound on formula size.
\newblock {\em ACM Transactions on Computational Logic 4}, 3 (2003), 296--314.

\bibitem{BK78}
{\sc Brzozowski, J., and Knast, R.}
\newblock The dot-depth hierarchy of star-free languages is infinite.
\newblock {\em Journal of Computer and System Science 16\/} (1978), 37--55.

\bibitem{EVW97}
{\sc Etessami, K., Vardi, M.~Y., and Wilke, T.}
\newblock First-order logic with two variables and unary temporal logic.
\newblock In {\em I{EEE} {S}ymposium on {L}ogic in {C}omputer {S}cience\/}
  (1997).

\bibitem{EVW02}
{\sc Etessami, K., Vardi, M.~Y., and Wilke, T.}
\newblock First-order logic with two variables and unary temporal logic.
\newblock {\em Information and Computation 179}, 2 (2002), 279--295.

\bibitem{GS05}
{\sc Grohe, M., and Schweikardt, N.}
\newblock The succinctness of first-order logic on linear orders.
\newblock {\em Logical Methods in Computer Science 1}, 1:6 (2005), 1--25.

\bibitem{I99}
{\sc Immerman, N.}
\newblock {\em Descriptive Complexity}.
\newblock Springer, 1999.

\bibitem{IK89}
{\sc Immerman, N., and Kozen, D.}
\newblock Definability with bounded number of bound variables.
\newblock {\em Information and Computation 83}, 2 (1989), 121--139.

\bibitem{K68}
{\sc Kamp, J.~A.}
\newblock {\em Tense logic and the theory of linear order}.
\newblock PhD thesis, University of California, Los Angeles, 1968.

\bibitem{KW90}
{\sc Karchmer, M., and Wigderson, A.}
\newblock Monotone circuits for connectivity require super-logarithmic depth.
\newblock {\em SIAM Journal of Discrete Mathematics 3}, 2 (1990), 255--265.

\bibitem{MP71}
{\sc McNaughton, R., and Papert, S.~A.}
\newblock {\em Counter-free automata}.
\newblock MIT Press, Cambridge, MA, 1971.

\bibitem{PW97}
{\sc Pin, J.-E., and Weil, P.}
\newblock Polynomial closure and unambiguous product.
\newblock {\em Theory of Computing Systems 30\/} (1997), 1--39.

\bibitem{S76}
{\sc Sch\"utzenberger, M.~P.}
\newblock Sur le produit de concatenation non ambigu.
\newblock {\em Semigroup Forum 13\/} (1976), 47--75.

\bibitem{STV01}
{\sc Schwentick, T., Th\'erien, D., and Vollmer, H.}
\newblock Partially-ordered two-way automata: a new characterization of {DA}.
\newblock In {\em Developments in {L}anguage {T}heory\/} (2001).

\bibitem{ST02}
{\sc Straubing, H., and Th\'erien, D.}
\newblock Weakly iterated block products.
\newblock In {\em Latin {A}merican {T}heoretical {I}nformatics {C}onference\/}
  (2002).

\bibitem{TT01}
{\sc Tesson, P., and Th\'erien, D.}
\newblock Diamonds are forever: the variety {DA}.
\newblock In {\em Semigroups, {A}lgorithms, {A}utomata and {L}anguages\/}
  (2001).

\bibitem{TT07}
{\sc Tesson, P., and Th\'erien, D.}
\newblock Algebra meets logic: the case of regular languages.
\newblock {\em Logical Methods in Computer Science 3}, 1:4 (2007).

\bibitem{TW98}
{\sc Th\'erien, D., and Wilke, T.}
\newblock Over words, two variables are as powerful as one quantifier
  alternation.
\newblock In {\em A{CM} {S}ymposium on {T}heory of {C}omputing\/} (1998).

\bibitem{T82}
{\sc Thomas, W.}
\newblock Classifying regular events in symbolic logic.
\newblock {\em Journal of Computer and System Science 25\/} (1982), 360--376.

\bibitem{T84}
{\sc Thomas, W.}
\newblock An application of the {E}hrenfeucht-{F}ra\"{\i}ss\'e game in formal
  language theory.
\newblock {\em M\'emoires de la S.M.F. 16\/} (1984), 11--21.

\bibitem{WI07}
{\sc Weis, P., and Immerman, N.}
\newblock Structure theorem and strict alternation hiearchy for {FO} on
  words.
\newblock In {\em Computer {S}cience {L}ogic\/} (2007).

\bibitem{W07}
{\sc Wilke, T.}
\newblock Personal communication, 2007.

\end{thebibliography}
\vskip-40 pt
\end{document}
