
















\documentclass[twocolumn]{autart}    \usepackage{amsmath}
\usepackage{color}
\usepackage{graphicx}
\usepackage{amsfonts}
\usepackage{amssymb}
\usepackage{hyperref}
\usepackage{color}
\usepackage{graphicx,subfigure}


\newtheorem{theorem}{Theorem}
\newtheorem{definition}{Definition}
\newtheorem{proposition}{Proposition}
\newtheorem{lemma}{Lemma}
\newtheorem{corollary}{Corollary}
\newtheorem{example}{Example}
\newtheorem{remark}{Remark}
\newtheorem{assumption}{Assumption}


\newcommand{\rank} {\mbox{\rm rank}\,}
\newcommand{\diag} {\mbox{\rm diag}\,}
\newcommand{\grk} {\mbox{\rm grk}\,}


\begin{document}

\begin{frontmatter}


\title{Zeros of Networked Systems with Time-invariant  Interconnections} 

\thanks[footnoteinfo]{This paper was not presented at any IFAC
meeting. Corresponding author is Mohsen Zamani.}











\author[label1]{Mohsen Zamani,}
\author[label2]{Uwe Helmke,}
\author[label3]{Brian D. O. Anderson}
\address[label1]{Research School of Engineering, Australian National University, Canberra, ACT 0200, Australia. (e-mail: mohsen.zamani@anu.edu.au)}
\address[label2]{Institute of Mathematics, University of W\"{u}rzburg, 97074 W\"{u}rzburg, Germany (email: helmke@mathematik.uni-wuerzburg.de)}
\address[label3]{ Research School of Engineering, Australian National University, Canberra, ACT 0200, Australia and Canberra Research Laboratory, National ICT Australia Ltd., PO Box 8001, Canberra, ACT 2601, Australia. (e-mail:
brian.anderson@anu.edu.au) }



\begin{keyword}                           Networked Systems. Multi-agent systems. Zeros               \end{keyword}                             



\begin{abstract}
This paper studies zeros of networked linear systems with time-invariant interconnection  topology. While  the  characterization of zeros is given for both heterogeneous and homogeneous networks,   homogeneous networks are explored in greater detail. In the current paper, for homogeneous networks with time-invariant interconnection dynamics,  it is   illustrated how the zeros of each individual agent's system description
and zeros definable from the interconnection dynamics contribute to generating zeros of the whole network.  We also demonstrate how zeros of  networked systems and those of their  associated blocked versions are related.
\end{abstract}

\end{frontmatter}



\section{Introduction}\label{intro}
Recent developments of enabling technologies such as communication systems, cheap
computation equipment and sensor platforms have given great impetus to the creation
of  networked systems. Thus, this area has attracted significant
attention worldwide and researchers have studied networked systems from
different perspectives (see e.g. \cite{sinopoli2003distributed},
\cite{olfati2002distributed}, \cite{tanner2003stable}). In particular,
in view of the recent chain of events \cite{gorman2009electricity},
\cite{falliere2011w32} and \cite{rid2012cyber}, the issues of security and cyber
threats to the networked systems have gained  growing attention. This paper uses  system theoretic approaches  to deal with problems involved with the security of networks.

Recent works have shown that control theory can be used as an effective tool to detect and mitigate the effects of cyber  attacks on the networked systems;  see for example  \cite{sinapoli2012}, \cite{cardenas2011attacks}, \cite{gupta2010optimal}, \cite{amin:09}, \cite{govinndarasu}, \cite {Teixeira202} and the references listed therein. The authors of   \cite {Teixeira202} have introduced the concept of \textit{ zero-dynamics attacks}
and  shown how attackers can use knowledge of networks' zeros
to produce control commands such that they are not detected as security threats. Thus,  zeros of networks provide valuable
information relevant to detecting cyber attacks. Though  various aspects of the dynamics of networked systems have been extensively
studied in the literature \cite{Ren2007,Olfati2007,fax2004}, to the authors' best knowledge
the zeros of networked systems have not been studied in any detail \cite{zamanihlemke2013}.

This paper  examines  the zeros of networked systems in more depth. Our focus is on networks of finite-dimensional linear
dynamical systems that arise through static interconnections of a finite number of
such systems. Such models arise naturally in  applications
of linear  networked systems, e.g. for cyclic pursuit \cite{MarshallBrouckeFrancis};
 shortening flows in image processing \cite{Bruckstein}, or for the
 discretization of partial differential equations
 \cite{brockett-willems}.


Our ultimate goal is to analyze the zeros of networked systems with
periodic, or more  generally time-varying interconnection topology.
An important tool for this analysis is  blocking or lifting technique
for networks with time-invariant interconnections.  Note that blocking of linear time-invariant systems is  useful  in design of controllers for linear periodic systems  as shown by
\cite{chenB95} and \cite{Khargoneckar85}. References \cite{Bolzern86}, \cite{Grasselli88},
  \cite{zamani2011} and \cite{chen2010} have analyzed   zeros
    of blocked systems obtained from blocking of
  time-invariant  systems. Their works were  extended by
  \cite{zamani2011}, \cite{chen2010}.
However, these earlier contributions  do not take  any underlying network structure
  into consideration. In this paper,  we introduce some results that
  provide a first step in that direction.



The structure of this paper is as follows. First, in Section
\ref{sec:models} we introduce state-space and higher order
polynomial system models for time-invariant networks of linear
systems. A central result used is the strict system equivalence
between these different system representations.  Moreover, we
completely characterize  both finite and infinite zeros of arbitrary
heterogeneous networks. For homogeneous networks of identical SISO
systems more explicit results are provided in Section \ref{sec:homog}. Homogeneous networks with a
circulant coupling topology are studied as well. In Section  \ref{sec:blocking},
a relation between the transfer function of the blocked system and
the transfer function of the associated unblocked system is
explained.  We then relate the zeros of  blocked networked systems
to those  of the  corresponding unblocked systems, generalizing work by
\cite{zamani2011}, \cite{chen2010},\cite{MOHSEN-SCLpaper}.  Finally, Section \ref{sec:conclusion} provides the concluding remarks.



\section{Problem Statement and Preliminaries}
\label{sec:models}
We consider networks of  linear systems, coupled
through constant interconnection parameters. Each agent is assumed to
have the state-space representation as a linear discrete-time  system

Here,  ,  and  are the associated system matrices. We assume that each system is reachable and
observable and that the agents are interconnected by static
coupling laws

with ,   and  denoting an external input
applied  to the whole
network. Further, we assume that there is a -dimensional interconnected output  given by

Define ,
, 
and \textit{coupling matrices}

as well as \textit{node matrices}


Then the closed-loop system is

with matrices






One can also start by assuming that each  system (\ref{sys1}) is defined in terms of Rosenbrock-type equations \cite{Rosenbrock1970} i.e. by systems of higher order
difference equations


Here  denotes the forward shift operator that acts on
sequences of vectors  as . Furthermore,  denote polynomial matrices of
sizes  and , respectively. We always assume  that  is  \textit{nonsingular},
i.e. that  is not the zero polynomial. Moreover, the
system (\ref{SIC24}) is assumed to be strictly proper, i.e. we assume
that the associated transfer function

is strictly proper.  Following Fuhrmann \cite{Fuhrmann1977}, any
strictly proper system of higher order
difference equations has an associated state-space realization
, the so-called \textbf{shift realization}, such that
the polynomial matrices

 are strict system equivalent \cite{Fuhrmann1977}. If the  first order representation (\ref{sys1}) is strict
system equivalent to the higher order system (\ref{SIC24}) then of
course the associated transfer functions coincide, i.e. we have



Throughout this paper we assume that the first order and higher
order representations  i.e. the  systems (\ref{sys1}) and (\ref{SIC24}),  are chosen to be of minimal
order, respectively. This is equivalent to the controllability and
observability of the shift realizations (\ref{sys1}) associated with
these representations (\ref{SIC24}). It is also equivalent to the  simultaneous left coprimeness
of  and the right coprimeness of .
Proceeding as above, define polynomial matrices

and similarly for  and . Here .
Using this  notation,    we  write all  systems of  (\ref{SIC24})  in the  matrix form as

where  and similarly for  and .
Then we have the left- and right coprime factorizations of the
  {\it
  node transfer function} as
 
The interconnections are given, as before, by


The  resulting network representation then becomes
 
with the   {\it network transfer function} defined as



The connection between the state-space and the  polynomial matrix
representations (\ref{eq:system}) and (\ref{eq:polysystem}), respectively,
is clarified by the following result. This theorem implies that important
system-theoretic properties such as reachability and observability, as
well as the poles and zeros of the networked system (\ref{eq:system}) can all  be  characterized by the polynomial
system matrix (\ref{eq:polysystem}).\\

\begin{theorem}[\cite{fuhe2013}]\label{sysequiv}
The interconnected systems (\ref{eq:system}) and (\ref{eq:polysystem})
are strict system equivalent. In particular, for each  there exist unimodular polynomial matrices
 such that

\end{theorem}







As a consequence of Theorem \ref{sysequiv} we derive a complete characterization for the zeros of
the system (\ref{eq:system}). We first present an extension
of the classical
definition of the zeros \cite{kailath} to the higher
order system \eqref{SIC24}. Note that the \textbf{normal rank}  of a rational matrix
function  is defined as



\begin{definition}\label{def:def1}
Let  be polynomial matrices with

nonsingular such that the  node transfer function  is
strictly proper; let  be a constant matrix. A \textbf{finite zero} of the  polynomial system matrix

is any complex number
 such that

holds.    is said to have a  \textbf{zero at infinity} if

\end{definition}




As a consequence of Fuhrmann's result \cite{Fuhrmann1977}, a
 polynomial system matrix (\ref{eq:systemmatrix}) has a finite or
 infinite zero if and only if the polynomial matrix


of the associated shift realization
 has a finite or infinite
 zero. Theorem \ref{sysequiv} thus leads to a complete
 characterization
 of the zeros for the interconnected system \eqref{eq:system}  as stated in the subsequent theorem.  We emphasize that the characterization of the zeros in the
 subsequent Theorem \ref{MAINA}
holds for any interconnection matrices and does not require any assumptions on reachability or observability of
the network, except of those for the individual  node systems.\\

\begin{theorem}[\cite{fuhe2013}]
\label{MAINA}
Consider the strictly proper   node transfer function  with
minimal representations (\ref{eq:nodematrices}) as

Let  be any arbitrary constant interconnection matrices of the proper dimensions and

denote the network transfer function.
Assume that  is represented by  a polynomial left coprime matrix fraction
description (MFD) as



 Then
\begin{enumerate}
\item For all 


 \item For all 
 
\item
 has a finite zero at  if and only if





\item  has a zero at infinity if and only if

In particular, if  has full-row rank or full-column rank, then  has no infinite zero.
 \end{enumerate}
\end{theorem}


\section{Zeros of Homogeneous Networks}\label{sec:homog}

The preceding result has a nice simplification in the case of
\textbf{homogeneous networks  of SISO agents}, i.e. where the node systems  are single input single output systems with identical transfer function.  Let us define the  {\it interconnection transfer function} as
 

The next theorem relates the zeros of the system \eqref{eq:system} to those of the interconnection dynamics \footnote{The term interconnection dynamics
is partly a misnomer. There is no dynamics separate to that included within the agent description, and the interconnecting matrices are all constant. The transfer function  is a theoretical construct: it is the transfer function from  to  resulting when every system is replaced by .} defined by the quadruple . Before we provide this main result, we need to state the following lemma regarding the generic rank of .

\begin{lemma}\label{lem:normalrank}
Assume that   are scalar SISO systems with identical transfer function . Let  denote any constant interconnection matrices of the proper dimensions  and  be the interconnection transfer function. Then the following equality holds.

\end{lemma}
\noindent \textbf{Proof.}
Consider any coprime factorization  of the
strictly proper transfer function , having McMillan degree
. Define .  We know that

Then by applying the second part of  Theorem \ref{MAINA},  one obtains

By substituting the last equality of \eqref{eq:lem1eq2} into \eqref{eq:lem1eq1}, the result follows.

\hfill 


\begin{theorem}\label{MAIN-B1}
Assume that   are SISO systems with identical transfer
function .Then   has
a zero at infinity if and only if  has a
zero at infinity.
\end{theorem}


\noindent \textbf{Proof.}
By Lemma \ref{lem:normalrank}, the network transfer function matrix  and the interconnection transfer matrix  have the same normal rank. Using the conclusion of Theorem \ref{MAINA} (part \ref{partx}), the result follows.
\hfill 

Theorem \ref{MAIN-B1}   shows that the infinite zero structure of
a homogeneous network depends only upon the interconnection parameters
and not on the specific details of the node transfer function.
This is in contrast to the finite zero structure, as is shown by the
following result.

\begin{theorem}\label{MAIN-B}
Assume that   are
SISO systems with identical transfer function . Let  be a coprime polynomial factorization of  and define .  Let  denote any constant interconnection matrices of the proper dimensions. \begin{enumerate}

\item
  has a finite zero at  with  if and only if   is a finite zero of
.
\item   has a finite zero at  with  if and only if  has a zero at infinity.
\end{enumerate}
\end{theorem}


\noindent \textbf{Proof.}
We first prove the first part of the theorem.  By Lemma \ref{lem:normalrank} and Theorem \ref{MAINA},  is a zero of  if and
only if
 
For  this is equivalent to

i.e.   being a finite zero of .  For the second
part note that  is a zero of  if and only if  inequality \eqref{eq:sisorank} holds.   If , then by
coprimeness of  and  we have  and  therefore
\eqref{eq:sisorank} is equivalent to

This is equivalent to .
Thus a zero of the node transfer function  is a zero of
 if and only if  has
a zero at infinity. This completes the proof.
\hfill 






Now assume that  has full-column rank or full-row rank. Then the
homogeneous network realization 
has no zeros at infinity. Thus in this case the finite zeros of
 are exactly the preimages of the finite
zeros of  under the rational function .
We conclude with a result that is useful for the design of networks
with prescribed zero properties. The result below bears a certain
similarity with a result by Fax and Murray \cite{fax2004}.
As shown by them, a formation of  identical vehicles can be analyzed for stability by analyzing a single vehicle with the same dynamics modified   by only a scalar, which assumes values equal to the eigenvalues of the interconnection matrix. Such a result is to do with poles, linking those of the individual agent  and the overall system via the eigenvalues (which are pole-like) of the interconnection matrix.  Our result is to do with the zeros, but  still links those of the individual agent, those of the interconnection matrix (suitably interpreted) and those of the whole system.


With the help of the preceding results, we can now study two other
important properties of  networks, namely, losslessness and
passivity. It is well known, see  e.g. \cite{vaidyanathan1989role}
(Section II. B), that if all agent  transfer function matrices and the
system defined by the quadruple  are lossless, then the
system \eqref{eq:system} is lossless. We now provide an
improvement of this result for the case of SISO agents.\\




Recall  that a strictly proper real rational transfer
function  is called lossless \cite{vaid93} if all poles of  are in the open
unit disc and  holds for all . A key property used below is that  if  and  if .


\begin{theorem}\label{cor:homog-network} Assume that  has full-column   rank or full-row rank. Then
\begin{enumerate}
\item The homogeneous network 
  has no zeros at infinity. A complex number  is a finite zero of
   if and only if  is a
  finite zero of .
\item Assume that the agent transfer function  is \textbf{lossless}. Then  is a minimum phase
  network, i.e. all of its zeros have absolute value , if and only
  if  is minimum phase.
\end{enumerate}
\end{theorem}

\noindent \textbf{Proof.}
The first claim is an immediate consequence of Theorem
\ref{MAIN-B}. If  is lossless then   holds if and and
only if . Thus
  maps the complement
of the open unit disc onto itself.  Thus  if and only
if . Therefore  has a
finite  zero
 with  if and only if each  with  satisfies
 and is a zero
of . Note that for any finite , there is necessarily a   satisfying , since this is a polynomial equation for .  This proves the result.
\hfill 


We now extend the second part of the above corollary for  the choice of  transfer functions \cite{vaid93}.  Let us  recall that  is  passive  if and only if
\begin{enumerate}
\item
all poles of  are in 
\item
   .
\end{enumerate}
This implies
\begin{enumerate}
\item
   
\item
 If , then .
\end{enumerate}


\begin{corollary}
Assume that  has full-column rank or full-row rank and   is \textbf{passive}. Then   is a minimum phase network, i.e. all of its zeros have absolute value , if  is minimum phase.
\end{corollary}

\noindent \textbf{Proof.}
Suppose  is a finite zero of . Then  is a
finite zero of , i.e.,    is a finite zero of
. By the minimum phase assumption,  or
. Passivity of  thus implies .
\hfill 
\bigskip


\subsection{Design of Networks}
An important issue is the construction of network topologies so that
the resulting  network is zero-free, i.e. it does not have
any finite zeros (but still may have a zero at infinity). We derive a
simple sufficient condition for homogeneous networks. By Corollary
\ref{cor:homog-network}, the homogeneous network
 is zero-free if and only if 
is zero-free. For simplicity, we assume that there is a single
external input and a single external output associated with the
network, i.e. . Moreover, we assume . Thus the
interconnection transfer function  is
scalar strictly proper rational.
The next result characterizes which outputs of the SISO interconnected
system lead to a network without finite zeros, for given state and input
interconnection matrices.  


\begin{theorem}[SISO Design Condition]\label{DESIGN}
Assume that   are
identical minimal SISO systems with identical transfer function. Let  be reachable with . Then a  network output
 defines a minimal network realization
  without finite zeros if and only if
  has relative degree . \end{theorem}



\noindent \textbf{Proof.}
By Corollary \ref{cor:homog-network}, the homogeneous network
  has no finite zeros
if and only if this holds for . In the SISO case this
is equivalent to the transfer function  having no
zeros. By \cite{fuhe2013},  is minimal if and only
if  is minimal. In either case,  has
McMillan degree  and has no zeros if and only if the relative
degree of  is equal to .
\hfill 

\begin{figure}[!t]
\begin{center}
    \includegraphics[width=5cm,height=2cm]{threenodeszero.pdf}
    \caption{\emph{A homogenous network consisting of three SISO agents. The agents, the external input and measurement are depicted  by green, blue and red circles, accordingly. The whole network has two zeros at -1 and 1 when all weights are set to unity.}} \label{fig:threeagentszero}
\end{center}
\end{figure}




\begin{figure}[!t]
\begin{center}
    \includegraphics[width=5cm,height=2cm]{threenodesnozero.pdf}
    \caption{\emph{A homogenous network consisting of three SISO agents. The agents, the external input and measurement are depicted by green, blue and red circles, accordingly. The whole network is zero-free when all  weights are set to unity.}} \label{fig:threeagentsnozero}
\end{center}
\end{figure}



The above theorem characterizes when the SISO networked systems are
zero-free. We note that the condition is equivalent to the
sytem-theoretic condition that the closed loop
system  is feedback
irreducible; i.e. that  is observable
for all state feedback matrices .

The next example illustrates  that  the zeros of  the system (\ref{eq:system}) may drastically change  by replacing and adding   a link.



\begin{example}
Consider the network depicted in Fig. \ref{fig:threeagentszero} where the nodes are  simply double integrators. Note that there exist   bidirectional links between the  agents.  By assuming a unit weight on each link, it is easy to verify that for such a network  the interconnection matrices are   and . Moreover,  the interconnection dynamics
has a single zero at . Hence, by using   Theorem \ref{MAIN-B} it is easy to see that the whole network has two zeros at  and .   One can also observe that by adding an extra link in Fig. \ref{fig:threeagentszero} from  agent   to the measurement node, with the same set of interconnection matrices  as before except for  which assumes random values in its nonzero entries, the whole network becomes \textit{zero-free}. The same result holds i.e.  the resultant network is zero-free, when the topology  is modified according to Fig. \ref{fig:threeagentsnozero}.

\end{example}







\subsection{Circulant Homogeneous Networks}
Homogeneous networks with special coupling structures appear in many
applications, such as cyclic pursuit \cite{MarshallBrouckeFrancis};
shortening flows in image processing \cite{Bruckstein} or the
discretization of partial differential
equations~\cite{brockett-willems}.  Here, we characterize the zeros for
interconnections that have a  circulant
structure.
A homogeneous network is called \textbf{circulant}
if the state-to-state coupling matrix  is a circulant, i.e.

The book   \cite{davis} provides algebraic background on the circulant matrices.
A  basic fact on circulant matrices is that they are simultaneously diagonalizable by the \textbf{Fourier matrix}

where   denotes a primitive th root of
unity. Note, that  is both a unitary and a symmetric
matrix.
It is then easily seen that any circulant matrix
 has the form

 where

As a consequence of the preceding analysis we obtain the following result.

\begin{theorem} Suppose that the system in (\ref{eq:system}) is a circulant homogeneous network.  Let  be full rank and  and 
denote the complex roots of

Then

are the finite zeros of the homogeneous network .
\end{theorem}

\noindent \textbf{Proof.}
By Theorem \ref{MAIN-B}, we  conclude that  the  system defined by  has a finite nonzero zero  if and only if the following matrix pencil has less than full rank

Observe that the following equality holds

Note that multiplication of a matrix by non-singular matrices on the left and right respectively does not change the rank. This implies the result.
 \hfill 






\section{Zeros of Blocked Networked Systems}\label{sec:blocking}


The technique of blocking or lifting  a signal is well-known in
systems and control \cite{chenB95} and signal processing
\cite{vaid93}. In systems theory,  this  method
has been
mostly  exploited to  transform linear discrete-time periodic
systems into
linear time-invariant systems in order to apply the well-developed  tools for  linear time-invariant systems; see
\cite{bittanti09} and the literature therein. Here, we show how this
technique can be applied to the networked  systems of the form

with matrices

and the  network transfer function

Here ,  and  and  are block-diagonal.
Given an integer  as the block size,  we define for 
 
The \textbf{blocked system}  then is defined as \cite{bittanti09}

where

The transfer function
 of (\ref{eq:eq3}),
 see \cite{bittanti09}, \cite{Khargoneckar85}, has the circulant-like
 structure as

{\footnotesize
 }
where  and
.
It is worthwhile mentioning  that the blocked transfer function has the
structure of a generalized circulant matrix.  The theory of generalized
circulant matrices is very similar to that of classical circulant
matrices; see \cite{davis}. Using such techniques we obtain the following result.


In order to deal with the zeros of the system \eqref{eq:eq6},  we  first  need to review the  following result  from \cite{Varga2003}, obtained by specializing Lemma 1 of \cite{Varga2003} to the time-invariant case.
\begin{lemma}\label{lemvarga} \cite{Varga2003}
Let , ,  and . Furthermore,  define ,  and
  where
  denotes the Kronecker product and  denotes a complex number. Then there exist invertible  matrices  and  and  matrices  and    such that for all 

\end{lemma}

Using this lemma we introduce the following result.

\begin{proposition}\label{PROPOSITION}
Let  denote the Fourier matrix of the  proper dimension and  be a left
coprime factorization of the network transfer function.
Consider the system matrices

There exist invertible  matrices
 and  that  are invertible for all nonzero complex numbers
  such that
{\footnotesize

}
\end{proposition}




\noindent \textbf{Proof.}
First, observe that the following equality holds

where  is the Fourier matrix of the proper dimension. Furthermore, we have



where .

Now  by using  \eqref{eq:circul2} and  \eqref{eq:circul}, one can easily verify that  the following equality holds



Therefore, for any , , we have ,  and . Hence,


Now by substituting (\ref{eq:nice}) into the equation \eqref{eq:vargelemmaeq} and performing the required rows and  columns reordering,  the conclusion of the proposition becomes immediate. 

\hfill 


The preceding results imply the following  characterization  of
the finite zeros for the interconnected systems.  Thus consider the
interconnected system   defined
in (\ref{eq:system}). Let
 denote the associated
blocked system, defined as in \eqref{eq:eq6} and \eqref{eq:eq10}.
\begin{theorem}
 A complex number  is a finite zero of the blocked
network
 if and only if there
exists  with  such that  is a finite zero of
.
\end{theorem}
 \noindent \textbf{Proof.}
  \textbf{Necessity.} Suppose that  is a zero of the system matrix , then by recalling  the result  of Proposition \ref{PROPOSITION}, one can easily  see  that one or  more of diagonal blocks in \eqref{eq:blockedtransferdiag1} should have rank below their  normal rank i.e. there exist at least one -th root of  which is a zero  of the unblocked system. \\
  \textbf{Sufficiency.}  Suppose that  is a zero of the unblocked system \eqref{eq:eq3}. Then at least one of the diagonal blocks in \eqref{eq:blockedtransferdiag1} loses rank below its normal rank. Now, again  by using   \eqref{eq:blockedtransferdiag1}, one can conclude that  is a zero of . The latter implies that  must be a zero of the system .

\hfill 


The above theorem only treats the finite nonzero zeros. To treat the
other cases i.e. zeros at the origin and infinity, we recall the following
result from \cite{MOHSEN-SCLpaper}.

\begin{proposition}\label{thm:zeroatinfinity}
Consider the  unblocked networked  system \eqref{eq:eq3} with  transfer
function   and the  blocked networked  system  (\ref{eq:eq6})
with transfer function . Suppose that  the quadruple
 is minimal. Then
\begin{enumerate}
\item The system \eqref{eq:eq3} has a zero at    if  and only if the system (\ref{eq:eq6})  has a zero at .
\item The system \eqref{eq:eq3} has a zero at the origin if and only if the   the system (\ref{eq:eq6})  has a zero at  the origin.
\end{enumerate}
\end{proposition}

This implies the next characterization of the zeros for the systems \eqref{eq:eq3} and (\ref{eq:eq6}).

\begin{theorem}
Let  be full rank and 
a homogeneous network with SISO agents. Then the blocked network

has no zeros at infinity.  The finite zeros of
 are exactly all
 such that  is a finite zero of  for some
 .
\end{theorem}

 \noindent \textbf{Proof.}
The proof readily follows from Proposition \ref{thm:zeroatinfinity}
 and the first part of Theorem \ref{cor:homog-network}.
 \hfill 









\section{Conclusions}\label{sec:conclusion}
In this paper, we explored the zeros of  networks of linear
systems. It was assumed that the interaction topology is time-invariant.  The zeros  were  characterized for both homogeneous and
heterogeneous networks. In particular, it was  shown that for homogeneous
networks with  full rank direct feedthrough
matrix,  the finite zeros of the whole network  are
exactly the preimages of interconnection dynamics zeros  under the  inverse of an agent transfer
function. We then discussed the condition under which the networked systems  have  no finite nonzero zeros. Then
\textit{ generalized circulant matrices} were used for a concise
analysis of the  finite nonzero zeros of blocked networked  systems. Moreover, we recalled some results about their zeros at infinity and at the origin.  It was shown that the  networked systems have zeros at the origin  (infinity) if and only if their associated blocked systems have zeros at the origin (infinity). As a part of our future work  we will
address open problems such as the consideration of periodically varying network topologies and MIMO dynamics for each agents. Furthermore, as explained in the  illustrative example given in the current paper,  adding and removing links can dramatically change the  zero  structure. Thus, another interesting  research direction involves exploring how links in the  networked systems  can be  systematically  designed  such that the resultant networked systems   attain   a particular zero dynamics.





\bibliographystyle{plain}
\bibliography{acc2012bib2}           



\end{document} 