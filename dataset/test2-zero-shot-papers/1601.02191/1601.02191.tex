\documentclass[conference]{IEEEtran}
\setcounter{secnumdepth}{3}
\usepackage{ifpdf}
\usepackage{cite}
\ifCLASSINFOpdf
   \usepackage[pdftex]{graphicx}
   \graphicspath{{../pdf/}{../jpeg/}}
   \DeclareGraphicsExtensions{.pdf,.jpeg,.png}
\else
   \usepackage[dvips]{graphicx}
   \graphicspath{{../eps/}}
   \DeclareGraphicsExtensions{.eps}
\fi
\usepackage{epsf,psfrag,epsfig}
\def\scalefig#1{\epsfxsize #1\textwidth}

\usepackage[cmex10]{amsmath}
\usepackage{mathrsfs, amsthm,amsfonts, latexsym, amssymb}
\usepackage{algorithmic,algorithm}
\usepackage{hhline}\usepackage[english]{babel}
\usepackage[mathscr]{eucal}
\usepackage{graphicx,color}
\usepackage[font=small]{caption}
\usepackage[font=footnotesize]{subcaption}
\usepackage{multicol,multirow}
\usepackage[breaklinks]{hyperref}
\usepackage[hyphenbreaks]{breakurl}
\hyphenation{op-tical net-works semi-conduc-tor}
\newtheorem{lem}{Lemma}
\newtheorem{thm}{Theorem}
\IEEEoverridecommandlockouts
\begin{document}
\title{Stochastic Interchange Scheduling in the Real-Time Electricity Market}

\author{Yuting Ji, Tongxin Zheng, {\em Senior Member, IEEE}, and  Lang Tong, {\em Fellow, IEEE}\thanks {\scriptsize This work is supported in part by the DoE CERTS program and the National Science Foundation under Grant CNS-1135844. Part of this work appeared in \cite{JiTong15PESGM}.

Y. Ji and L. Tong are with the School of Electrical and Computer Engineering, Cornell University, Ithaca, NY 14853, USA (e-mail: yj246@cornell.edu; ltong@ece.cornell.edu).

T. Zheng is with ISO New England Inc., Holyoke, MA 01040, USA (e-mail: tzheng@iso-ne.com).}}
\maketitle

\begin{abstract}
The problem of multi-area interchange scheduling in the presence of stochastic generation and load is considered.   A new interchange scheduling technique based on a two-stage stochastic minimization of overall expected operating cost is proposed.  Because directly solving the stochastic optimization is intractable, an equivalent problem that maximizes the expected social welfare is formulated. The proposed technique leverages the operator's capability of forecasting locational marginal prices (LMPs) and obtains the optimal interchange schedule without iterations among operators.
\end{abstract}

\begin{IEEEkeywords}
Inter-regional interchange scheduling, multi-area economic dispatch, seams issue.
\end{IEEEkeywords}

\IEEEpeerreviewmaketitle
\section{Introduction}\label{sec:intro}
Since the restructuring of the electric power industry, independent system operators (ISOs) and regional transmission organizations (RTOs) have faced the seams issue characterized by  the inefficient transfer of power between neighboring regions.  Such inefficiency is caused  by incompatible market designs of independently controlled operating regions, inconsistencies of their scheduling protocols, and their different pricing models. The economic loss due to seams for the New York and New England customers is estimated at the level of \qp_1qp_2qp_1p_2qc_i(\cdot), i\in\{1,2\}c_i(\cdot)=g_i^\intercal H_i g_i +q_i g_iH_ic_i(\cdot)id_iiq12q>021g_iiF_iiQ\mathcal{G}_ii{S}_{ij}ij{S}_{qi}i\lambda_ii\mu_ii\mu_{q}(P_1)(P_1)g^*_i(q)i\in\{1,2\}iq12(P_2)qq(P_{21})(P_{22})qg^*_i(q)\lambda^*_i(q)\mu^*_i(q)i \in \{1, 2\}qi\in \{1, 2\}\lambda^*_i(q)\mu^*_i(q)(P_{2i})i \in \{1, 2\}\pi_i(q)iq\pi_2(q)\pi_1(q)\pi^*\piqQ{q_\text{TO}^*}\pi_i(q)i\pi_2(q)q=0\pi_1(0)<\pi_2(0)\pi_1(q)\pi_2(q)q_\text{TO}^*Q(P_1)(P_2)(P_1)qd_1d_2(P_{21})(P_{22})(P_{2i})qd_i\pi_i(q, d_i)qd_ic_i(q,d_i)(P_4)(P_4)\mathbb{E}_{d_2}[\pi_2(q,d_2)]\mathbb{E}_{d_1}[\pi_1(q,d_1)]\pi^*\piqQ{q_\text{STO}^*}q\pi_i(q,d_i)\mathbb{E}_{d_1}[\pi_1(q,d_1)]q\mathbb{E}_{d_2}[\pi_2(q,d_2)]q_\text{STO}^*(P_5)q(P_4)(P_5)(P_5)(P_4)(P_{21})(P_{22})q\leq Q(P_4)(P_5)(P_4)(P_5)(P_5)(P_4)(P_5)(P_{2i})i\in\{1,2\}q\leq Q(P_{2i})(P_{2i})(P_4)J(q)\pi_1(q,d_1)\pi_2(q,d_2)(P_5)s(q)(P_4)(P_5)q^{*}(P_4)\mu^{*}_{q}(P_4)q^\sharp(P_5)\mu^\sharp_{q}(P_5)q^{*}=q^\sharp\pi_i(q)d_i(P_{2i}), i\in\{1,2\}q\leq Q\pi_1(q)\pi_2(q)q^{*}=q^\sharpq^{*}=q^\sharp=Qq^{*}<Qq^\sharp <Q\mu_{q}^{*}=\mu_{q}^\sharp=0\mathbb{E}_{d_1}[\pi_1(q^{*},d_1)]=\mathbb{E}_{d_2}[\pi_2(q^{*},d_2)]\mathbb{E}_{d_1}[\pi_1(q^\sharp,d_1)]=\mathbb{E}_{d_2}[\pi_2(q^\sharp,d_2)]\mathbb{E}_{d_2}[\pi_2(q,d_2)]-\mathbb{E}_{d_1}[\pi_1(q,d_1)]=0q^{*}=q^\sharpq^{*}<q^\sharp =Q(P_1)q^\sharp(P_{21})(P_{22})q^\sharp\mu_{q}^\sharpg_1^*(q^{\sharp})g_2^*(q^{\sharp})\lambda_1^*(q^{\sharp})\mu_1^*(q^{\sharp}),\lambda_2^*(q^{\sharp}), \mu_2^*(q^{\sharp})(P_1)(P_1)(P_1)q^\sharp <q^{*}=Q(P_4)(P_5)\mathbb{E}_{d_1}[{\pi}_1(q,d_1)]\mathbb{E}_{d_2}[\tilde{\pi}_2(q,d_2)]\mathbb{E}_{d_2}[\pi_2(q,d_2)]\pi_\text{bid}(q)\mathbb{E}_{d_1}[\pi_1(q,d_1)]\mathbb{E}_{d_2}[\tilde{\pi}_2(q,d_2)]q^*_\text{SCTS}\mathbb{E}_{d_2}[\pi_2(q,d_2)]\mathbb{E}_{d_2}[\tilde{\pi}_2(q,d_2)]\mathbb{E}_{d_1}[\pi_1(q,d_1)]\pi^*{q^*_\text{SCTS}}\piqQ \mathbb{E}_{d_1}[\pi_1(q,d_1)] \mathbb{E}_{d_2}[\pi_2(q,d_2)]\mathbb{E}_{d_2}[\pi_2(0,d_2)]\mathbb{E}_{d_1}[\pi_1(0,d_1)]\pi_\text{bid}(q)(P_6)100200d_2=30d_5=250\renewcommand{\arraystretch}{1.5}\begin{array}{c} 0\leq g_1\leq 120\\c_1=0.01g^2_1+10g_1 \\ \end{array}\renewcommand{\arraystretch}{1.5}\begin{array}{c} 0\leq g_3\leq 200\\c_3=0.01g^2_3+40g_3\\  \end{array}\hspace{1em}\renewcommand{\arraystretch}{1.5}\begin{array}{c} 0\leq g_4\leq 100\\c_4=0.01g^2_4+30g_4\\  \end{array}\renewcommand{\arraystretch}{1.5}\begin{array}{c} 0\leq g_6\leq 200 \\c_6=0.01g^2_6+45g_6\\ \end{array}\mathcal{N}(55, 10^2)d_5=250d_5=200\pi_1^\text{TO}55\pi_1^\text{STO}\mathcal{N}(55, 10^2)-2.13/MWh, which means that the expected price of the importing region (region 2) is lower then that of the exporting region. This implies that the interchange is scheduled from a high cost region to a low cost region, which is counter intuitive. On the other hand, in the second example, the expected price difference at the interchange level of TO schedule is \q^*\mathbb{E}[\text{Cost}(q^*)]\mathbb{E}[\Delta\pi(q^*)]d_5=250166.56794.2-2.13162.86790.80d_5=200147.54621.62.38151.44608.60d_5=250d_5=200d_5=250d_5=200d_5=250d_5=200\sigmaw\sim \mathcal{N}(55, \sigma^2)\sigma\sigma55\sigma=050wpw=(10,10,10)w'=(100,200,200)p=(0.1,0.9)p=(0.5,0.5)p=(0.9,0.1)(91,181,181)(55,105,105)(19,29,29)q^*\mathbb{E}[\text{Cost}(q^*)]\mathbb{E}[\Delta\pi(q^*)]p=(0.1,0.9)268.6104751.6263.1104750.9p=(0.5,0.5)242.6109798.1236109797.1p=(0.9,0.1)197114806.8204114805.8p=(0.1,0.9)p=(0.5,0.5)p=(0.9,0.1)250q^*\mathbb{E}[\text{Cost}(q^*)]\mathbb{E}[\Delta\pi(q^*)]p=(0.1,0.9)250104754.8250104754.8p=(0.5,0.5)242.6109798.1236109797.1p=(0.9,0.1)197114806.8204114805.8p=(0.5,0.5)q^*\mathbb{E}[\text{Cost}(q^*)]109798.1109797.1(P_{2i})\mathcal{L}_{2i}, i\in\{1,2\}c_i(g_i^*(q))\pi_i(q)\pi_i(q)c_i(g_i^*(q))c_i(g_i^*(q))\pi_i(q)q\pi_1(q)\pi_2(q)\pi_i(q)q\leq Q\pi_i(q)q\leq Q$.
\end{proof}

\Urlmuskip=0mu plus 1mu\relax
\bibliographystyle{IEEEtran}
{\bibliography{reference}}
\end{document}
