In this section we analyze the structure of copyless CRA and develop some machinery that will be useful in Section~\ref{sec:nonexpressibility}. 
These results will help to understand the internal structure of copyless CRA.

\subsection{Normal form} Let  be a copyless CRA and let  be a predefined linear order over .
We say that a substitution  is in normal form with respect to  if  for all  and all .
In other words, all variables mentioned in  are greater or equal to  with respect to~.
Furthermore, we write that  is in \emph{normal form} with respect to  if every  is in normal form with respect to .
For example, the copyless CRAs in Example~\ref{ex:max-b-substrings} and~\ref{ex:no-poly-ambiguous} are in normal form with respect to the orders  and , respectively.
For the sake of simplification, we assume that every set of registers  is given with a linear order  and instead of writing that  or  are in normal form with respect to  we write in short that  or are in normal form.


\begin{example}\label{example:normal_form}


Consider the set of registers  with the order .
The copyless CRA  in Figure~\ref{fig:CRA2} is not in normal form, because of the -transitions.
On the other hand, the copyless CRA  is in normal form. For both automata we omit the initial states, initial valuations and final output functions because they are not relevant for the discussion.
\end{example}

\begin{figure}[t]
	\begin{center}
		\begin{tikzpicture}[-\string>,\string>=stealth',shorten \string>=1pt,auto,node distance=2.3cm,semithick,initial text={},scale=0.4,every node/.style={scale=0.8}]
		\tikzstyle{every state}=[fill=white,draw=black,text=black,style={font=}]
		
		\node[state,minimum size=2ex] 		(p) at (-9,-0.4) {};
		\node[state,minimum size=2ex] 		[right of=p](r) {};
		\node at () {};
		
		\path (p)   edge	[loop left]		node (s1) {}  (p)
		(p)   edge	[bend left]	node {}  (r)
		(r)   edge	[bend left]	node {}  (p)
		(r)   edge	[loop right]	node [align=center,below right]{}  (r);
		


		\node[state,minimum size=7ex] 		(p) at (8.5,-0.4) {};
		\node[state,minimum size=7ex] 		[right of=p](r) {\scriptsize{}};
		
		\node at () {};
		
		\path (p)   edge	[loop left]		node [above] (s1) {}  (p)
		(p)   edge	[bend left]	node {}  (r)
		(r)   edge	[bend left]	node {}  (p)
		(r)   edge	[loop right]	node [align=center]{}  (r);
		\end{tikzpicture}
\caption{Examples of copyless cost-register automata regarding normal form}\label{fig:CRA2}
	\end{center}
\end{figure}

In the previous example, the automaton  uses the registers  and  to count the number of 's and 's.
However, depending on the current state both registers have either the number of 's or the number of 's. 
It is clear that one would like to avoid this type of behavior in a theoretical analysis. Intuitively, one register should always contain the number of 's and the other register the number of 's.
One can clearly transform  to an automaton in normal form by exchanging the use of  and  in the transitions and encoding this permutation between registers in the states.
This is precisely the automaton  in Figure~\ref{fig:CRA2}. 
We generalize this idea for all copyless CRA in the next proposition.




\begin{proposition}
\label{proposition:normal_form}
	For every copyless CRA  there exists a copyless CRA in normal form  with the same set of registers such that they output the same ground expressions for all words and thus recognize the same function. The number of states in  can be bounded exponentially in the size of the automaton .
\end{proposition}

\begin{proof}
Let  be a copyless CRA. 
 We define a copyless CRA  in normal form such that  computes the same function as .
 The idea of  is to store a state  and a permutation  of the set  such that, if  is the current configuration of  over an input , then  is the configuration of  over  and . 
 In other words, the content of  is still defined by  but the value  of  is now in the register  for every .
 The permutation of registers' content will allow us to keep the normal form in . 
 Formally, let , where:
 
 is the set of states,  is the initial state where  is the identity permutation, and  is the initial function.
 For the sake of presentation, let us show how the run of  will correspond to the run of  before defining the transition function  and the output function~. 
 For an expression  and a permutation  over , we define  to be the expression  where the registers are replaced according to~.
 Let  and  be the configuration of  and , respectively, after reading . We will show:

Note that, for the word , this correspondence holds since  and  are the initial configuration of  and , respectively, and . 
We define  and show that if~(\ref{property}) always holds, then  and  computes the same function. 
Indeed, if the transition function  preserves~(\ref{property}), then:

This proves that the outputs of  and  are the same (provided that  preserves (1)).
It remains to define  such that  is a copyless CRA in normal form and its definition satisfies~(\ref{property}).

\newcommand{\sS}{S_{\sigma, \rho}}
\newcommand{\stau}{\tau_{\sigma, \rho}}

We need some additional definitions. 
For a copyless substitution  and a permutation  both over  we define the set , that is, the set of all variables  where  is not ground.
Further, define the set . For each , the set  is non-empty and thus has a least element. This motivates the function  that associates to each  its least element, formally:

One can easily check that  is a bijective function from  to .
It is surjective by the definition of  and , and injective by the copyless restriction over .
To see the last claim, recall that any copyless substitution satisfies  and, in particular,  for any permutation . 
Finally, given that  is a bijective function from  to , we can extend  to a bijection  such that  for every .
Of course, there might be many extensions of  to  but we can choose any extension (the decision is not important for the construction).

We have now all the ingredients to define the transition function . For every , , and permutation  over , if  then we define  such that

for every .
The substitution   is a copyless substitution for every copyless  because  and  are just permuting the variables. 
Therefore, we can conclude that  is copyless as well.

Our next step is to show that  is in normal form.
Recall that  is in normal form if for every  it holds that  for every .
We prove this by case analysis by considering whether  or not.
First suppose that . 
Since  is in the codomain of  and  is an extension of , we have that  by (\ref{property2}). 
Then if we replace  by , we get that .
In particular, we have that  for every .
Now suppose that .
This means that  is not in the codomain of  which implies that .
In other words,  is a ground expression and .
Since for both cases it holds that  for every  and , then we conclude that  is in normal form. 

For the last part of the proof, we show by induction that  satisfies the correspondence~(\ref{property}) between  and .
Let  and  be the configuration of  and , respectively, after reading . Assume by induction that  for every  and let:

be the transitions for  and , respectively,  after reading a new letter . 
We prove the correspondence~(\ref{property}) between  and  as follows:

This proves that the transition function  keeps the correspondence (\ref{property}) between  and . Since it also holds for the initial configuration then by induction it holds for all reachable configurations, which proves that the outputs of  and  are the same. 

So far, we have shown that  and  define the same function and every substitution on -transitions are in normal form. To conclude the proof we need to show that if every substitution on -transitions is in normal form, then every  is in normal form. We prove this by induction over the length of the word.
Assume that it holds for all words  of length at most  and let . Suppose we want to extend  with  and let . By definition, we know that . Set . By definition there exists a register  such that  and . Since  is in normal form, we conclude that . In other words,  for every  and, thus,  is in normal form. 	
\end{proof}





\subsection{Stable registers and reset substitutions}
\label{subsection:stable_reset}
Let  be a copyless CRA in normal form. 
During a run of  the content of its registers flows from higher to lower registers with respect to the total order . 
This does not necessarily mean that the content of all registers eventually reaches the -minimum register. 
For example, if all substitutions in  are of the form  for some , then each register will store just its own content during the whole run. 
Intuitively, in this example each register is ``stable'' with respect to the content flow of , since each register never passes its value to lower registers.  
This idea motivates the notion of \emph{stable registers}, which are essential to understand the behavior and output of copyless CRA.
Let  be a copyless substitution in normal form.
We say that a register  is -\emph{stable} (or stable on ) if .
More general, we write that a register  is \emph{stable} in  if   is -stable for all substitutions~.

Stable registers play a crucial role in the behavior of copyless CRA. 
Indeed, we will show that they are the only registers whose value always depends on the whole word, namely, we can always ``reset'' the value of non-stable registers to a constant. For instance, the automaton  in Example~\ref{ex:max-b-substrings} resets the register  to 0 each time the symbol  is read. On the other side, the register  is stable and it cannot be reset to a constant. In fact its value only grows or remains the same during the run of .


We formalize this idea of reseting the content of registers as follows: a substitution  is a \emph{reset} substitution if  for all non -stable registers . 
We say that a substitution  is an \emph{-reset substitution} if  for all non-stable registers ~in~.

We start with the following lemma that shows that the composition preserves stability between registers.
\begin{lemma}\label{lemma:stability}
Let  be two copyless substitution in normal form. For any register  it holds that  is stable on  and  if, and only if,  is -stable. 
\end{lemma}
\begin{proof}
Suppose that  is stable on  and . This means that  and  and, thus,  by composition, implying that  is -stable. 
For the other direction, suppose that .
Then we know that there exists  such that  and .
Since  and  are in normal form, then  which implies that  and, thus,  is stable on  and .
\end{proof}
By the previous lemma, it is enough to check that a register  is -stable for all transitions  to know whether  is stable in all substitutions from .

For  we say that  is a \emph{bottom strongly connected component} (BSCC) of  if (1) for every pair  there exists  such that  for some substitution  and (2) for every  and  if  then .
Intuitively a BSCC  of  is a set of mutually reachable states such that there is no word that leaves . 
We say that  is \emph{strongly connected} if the whole set  is a BSCC of .


\begin{proposition}
	\label{prop:reset_register}
	Let  be a copyless and strongly connected CRA in normal form.
	Then for all  there exists  and a substitution  such that  and  is an -reset substitution.
	Furthermore, there exists  containing all letters in .
\end{proposition}
For instance in Example~\ref{ex:max-b-substrings} it suffices to take the word  or any word that contains , given that the substitution defined by the -transition is an -reset substitution. For simplicity if , we write  and  instead of  and . Note that the additional requirement that  contains all letters in  is rather technical and its usefulness will be clear later in Section~\ref{sec:nonexpressibility}.

\begin{proof}[Proof (of Proposition~\ref{prop:reset_register})]
	Let  be all registers in  in the increasing order with respect to . We construct the word  by (inverse) induction starting from  and ending in .
	Specifically, for every  we define a word  and a substitution  such that the proposition holds for all non-stable registers  such that .
	Clearly, the proposition  will be shown by defining .
	
	We start with the base case  and consider whether  is stable or not.
	If  is stable, then take a word  and a substitution  such that  and  contains each letter in .
	Given that  is strongly connected we know that  and  always exists. 
	Then by defining  and , the proposition holds for the stable register .
	Now, suppose that  is non-stable which means that there exist a pair  and a word  such that  and  is non-stable in .
	Given that  is in normal form and  is the maximum register with respect to , this implies that .
	Pick two words  such that  and  for some substitutions  and , and  contains each letter in~.
	Again, we know that  and  always exists since  is strongly connected.
	Then define   and  .
	By construction, we know that  and  contains all letters in . 
	To prove that , notice that  is non-stable on .
	By Lemma~\ref{lemma:stability} this implies that  is non-stable on .
	Given that  is in normal form and  is the maximal register, we get that .
	
	For the inductive step, assume that  and  exist. We show how to construct  and  that satisfy the proposition for registers greater or equal than . 
	Again, we consider two cases depending on whether  is stable or not. 
	If  is stable, then by defining  and  the inductive step trivially holds.
Suppose that  is non-stable and, therefore, there exist a pair   and a word  such that  and  is non-stable on .
	Let  such that  and  for some substitutions  and .
	Recall that  and  exist because  is strongly connected. 
	Now, define:
	
	It is clear by construction that   and  contains all letter in 
	(because already  contains all letters in ). 
	To conclude the proof, we show that  for every non-stable register . 
	Let  be any non-stable register  (possibly ) and let . 
	First, note that all  are non-stable.
	Otherwise, if  is stable, then , which is a contradiction because  by the definition of being copyless.  
	Therefore, we have that every register in  is non-stable.
	Note also that .
	This is obvious when  because  is in normal form. Otherwise,  and we know that  because  is non-stable.
	Then we have that all registers in  are non-stable and strictly greater than . 
	By the induction assumption  for all . By composing  and , we conclude that .
\end{proof}




\subsection{Growing rate of stable registers in a cycle}
The behavior of cycles in a computation model is always important; most of the decidability results can be derived from a good understanding of its cyclic behavior. 
Here, we study how the content of stables registers behaves through cycles.
We say that a word  is a cycle over a state  in  if  for some substitution~. 
Of course, the iteration of a cycle  (i.e.  for any ) is also a cycle over  and it satisfies  for any , where  is  composed -times with itself. 
We will need the next result to show that by iterating cycles one can always ``reset'' the content of non -stable registers.  
\begin{lemma} \label{lemma:reseting-loop}
Let  be in normal form and  a copyless substitution. There exists  such that  is a reset substitution. 
\end{lemma}

\begin{proof}
Suppose  is non-stable over , i.e. , and assume that . Then consider the register . Since  is copyless and in normal form then . But since  and  is copyless then  (i.e.  cannot appear twice in ). 
Recall that  given that .
Combining both facts we get that  and because  is in normal form . 

So far, we have proved that  and . Also, we know that  is a copyless substitution in normal form. Then, we can repeat the same argument above for  and , forming an increasing sequence of registers:

The number of registers is finite so this sequence cannot be infinite. Thus there exists an  such that .
It is straightforward that we can take .
\end{proof}
By Lemma~\ref{lemma:reseting-loop} we know that a non-stable register  over  becomes a constant when  is iterated at least -times.
In the next proposition, we study the growing behavior of stable registers when a reset substitution is iterated. This will be useful to understand the behavior of copyless CRA inside their~cycles.
For this result we need additional assumptions that automata do not use  in their expressions and that the semiring  is non-zero (see Section~\ref{subsection:zeros} for details).
\begin{proposition}
	\label{prop:long_exp}
	Let  be a non-zero semiring and let  be a non-zero automaton
	in normal form. Let  be a reset substitution and  a -stable register. Then there exist  with  such that for every  we have:
	
\end{proposition}

Before proving Proposition~\ref{prop:long_exp}, we must show a straightforward fact about copyless expressions.
\begin{lemma}
	\label{lemma:easy_exp}
	Suppose the semiring  is a non-zero semiring and  is a non-zero automaton.
	Let  where  is a reset substitution and let  be a -stable and -stable register.
Then the expression  is equivalent to an expression of the form , where  and .
\end{lemma}

\begin{proof}
	We prove this by induction on the length of expression . For the base step take . Then  is equivalent to . Suppose we can write such an expression  for  of length at most . By the inductive assumption when  is of length  then  can be written in the form  or  for some expression . 
Since  is copyless then .
	Therefore, every  is non-stable in  and thus  for some constant  (by the non-zero assumptions on  and ). 
Thus,  can be rewritten as  or , respectively.
\end{proof}


\begin{proof}[Proof (of Proposition~\ref{prop:long_exp})]
	Since  is a copyless and reset substitution, then for every  either  or . 
	The expression  can be seen as  where every register  is replaced with . This is a copyless expression with only one variable  and it does not use  since  is non-zero. By Lemma~\ref{lemma:easy_exp} the expression  (setting ) can be rewritten in the form  for some  and .
	
	We prove the proposition by induction using the constants  from . For  the claim is trivially true and for  the expression  is of the desired form.
For the inductive step, we have . Then:
	
 \end{proof}



Proposition~\ref{prop:long_exp} shows how -stable registers grow with respect of the number of times that a cycle is iterated.
In particular, when  then -stable registers grows linearly because  operation is  in this semiring. 

The next result is a refinement of Proposition~\ref{prop:long_exp} for stable registers but in terms of the -semiring. This result will be crucial in the proof of Theorem~\ref{theorem:counterexample}.
Recall that, by Proposition~\ref{prop:reset_register}, for any  there exists a word  such that  and  is an -reset substitution.

\begin{lemma}
\label{lemma:loops}
Let  be a copyless, strongly connected and non-zero CRA in normal form over the  semiring. Furthermore, let  and  be a cycle in , namely,  for some reset substitution~.
For every  let  be the substitution such that . Then for every ,  and  big enough the following holds:

where  are constants that depend only on .
\end{lemma}

The additional substitution  will be necessary in Section~\ref{sec:nonexpressibility}. Intuitively, we will compose many substitutions and  will represent a substitution from previous compositions. \begin{proof}
Recall that we work with the  semiring. Let  and let  be a cycle such that  for some reset substitution .
By Proposition~\ref{prop:long_exp} we have that for every  it holds that:
\setlength{\jot}{4pt}

Recall that  is a non-zero CRA and in this semiring , thus . If  then the last equality is not immediate. It holds because:


Fix now a substitution .
Recall that  and  are the word and the substitution from Proposition~\ref{prop:reset_register} for~. 
Then for any , consider the substitution  such that .
We prove that  satisfies the lemma for  big enough.
For a non-stable register  in  the result is straightforward. 
Indeed,  is in normal form, and thus  contains just non-stable registers. 
This implies that:

Suppose now that  is a stable register in . We need to show that:

for  big enough and some constants  such that  do not depend on . 

The expression  is equivalent to . In other words, it is equivalent to the expression  where all registers  are substituted with . We start the proof by analyzing these expressions.

Let . If  is not a -stable register, then  and thus . 
Otherwise, by (\ref{eq:artic-version}) we know that  for some  and, thus,  by applying  on  we get: 

If  is a -stable register, but non-stable on  (i.e. non-stable in general) then  is equal to a constant and does not depend on . Thus we can estimate  by  and then (\ref{eq:artic-version2}) becomes:

Otherwise,  is a stable register and, moreover,  is a reset substitution. Then by Lemma~\ref{lemma:easy_exp} we know that  can be represented as:
\setlength{\jot}{4pt}

for some constants  not depending on  and where . Thus, by combining~(\ref{eq:artic-version2}) and~(\ref{eq:stable-simple-version}) we~get:
\setlength{\jot}{4pt}

Observe that a variable  is stable if, and only if, . We summarize the three possible cases in the next equation:



Now we can prove (\ref{eq:stable-cases}). Recall that the expression  is the expression  where all registers  are substituted with .
First notice that by Lemma \ref{lemma:easy_exp} the expression  is equivalent to .  We can do these estimations because  does not depend on .
By combining this equation with the definition of  we have:

Thus to finish the proof it suffices to show that:

for some constants . Indeed, if (\ref{equation:lemma_induction2}) holds, then . Similarly the value  can be estimated by the expression   so the last  operation is not needed. Notice that this does not change the constants  and , which proves that they do not depend on .

To conclude the proof, we show that (\ref{equation:lemma_induction2}) holds for a stable register . 
Recall that . 
We will show that (\ref{equation:lemma_induction2}) holds even if we change  to , where  is any copyless substitution in normal form and  is -stable. 
The proof is by induction on the size of . For the base step we must consider  (because  is stable). Then

For the inductive step, assume that~(\ref{equation:lemma_induction2}) holds for expressions  of length . We want to show that~(\ref{equation:lemma_induction2}) holds for an expression  of the form  where  is either  or ,  is -stable and  is an expression where all registers are non-stable (recall that  is copyless).
By unraveling , we get that:

In the final expression it is easy to check that  is equal to a constant or to  for some .
Indeed,  by (\ref{eq:sigma-tau-equation}) we know that  for non-stable registers. Since  is an expression over non-stable registers, one can show by induction over the size of  that   is of the form  for some constant  (but possibly ) and  big enough.

We assume that . Then we must consider two cases whether  is  or  operation. For the former we have:

and for the latter we have:

The number of induction steps depends on the size of the expression  which for  does not depend on . This concludes the proof. 
\end{proof}





