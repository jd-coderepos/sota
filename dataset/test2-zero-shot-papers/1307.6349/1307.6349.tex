In this section, we mainly focus on the techniques of estimating
moving speed and thus moving distance by CSI.
Theoretical basis is presented first, and then
 we present our algorithm implementation.


\subsection{The Electromagnetic Standing Wave Field}

Wireless radio propagation in compact environment could be modelled as a
superposition of large-scale path-loss, middle-scale shadowing, and small-scale
multipath fading~\cite{rappaport1996wireless}. For the multipath fading, it is
usually fitted to a statistical model called Rayleigh or Rician (Rayleigh fading
plus strong LoS components) distribution. The ripples-like deep fading shown in
Fig.\ref{fig:csisample} (b) and (c) are typical Rayleigh fading pattern.

Previous speed estimation methods, are based on some statistical properties of
Rayleigh distribution, \eg level crossing rate (LCR) or coherent time .
Although it has been experimentally validated that the distance between two
adjacent ripples (deep fadings) is about  ( is the carrier
wavelength) even in large-scale multipath environment (like Manhattan city)
\cite{sklar1997rayleigh}, no previous works explicitly exploits such 
fluctuation, since such fluctuation is encapsulated and blurred in a too general
model.

However, some detailed studies of radio
propagation\cite{braun1991physical,zonoozi1996shadow,chen1997sbr} have indicated
that, in a complex multipath environment the constructive or destructive
interferences of the large sum of reflected and scattered waves will generate a
\textit{standing waves field}, and the environment becomes a weak
\textit{Electromagnetic Cavity Resonator} (ECR) \cite{hill2009electromagnetic}
which hold standing waves in a very short time. According to basic physics of
wave propagation, the distance between two adjacent \textit{antinodes} (position
with maximum amplitude), \textit{towards any direction}, is , thereby
the experimentally observed  fluctuation.
Therefore, when a antenna traverse the indoor space with a speed , a
periodically ripples-like pattern with a frequency 
appears. Such simple relationship inspires us that the moving speed  could be
precisely estimated purely from the CSI, if we could precisely estimate .


\subsection{Theoretical Basis}
Wireless signal propagation in indoor environment can be well modelled
as Rician fading channel. In such a model, the CFR at the -th
subcarrier is described as \cite{ricianKestimate}

where the  represents the deterministic Line-of-Sight (LoS)
component,  represents the random multipath component. The
Rician factor  determines the power ratio between these two
components.

To simplify the system model, we will first focus on a simplified
model, Rayleigh fading channel, which is a specialized form of Rician
fading channel when . Generalized solution under Rician channel
will be discussed later.

\subsubsection{In Rayleigh Fading Channel}

In a multipath environment, there are lots of objects in the
environment that scatters the wireless signal. The received fading
envelop at an antenna will be the superposition of a large number of
these reflected and scattered waves. Since different position has
different constructive or destructive interference pattern among these
waves, the received signals amplitude at different locations become a
random variable. When a mobile antenna passes through the environment,
ripples-like deep fading appears in the instantaneous CFR as shown in
Figure \ref{fig:csisample} (b) and (c).

Assuming Wide Sense Stationary Uncorrelated Scattering (WSSUS)
 environment and the uniformly distributed angle of arrival (AoA) of
 multipath components, Clark \cite{Clark} has derived the
 auto-correlation of Rayleigh faded CIR  with motion at a scalar
 velocity  \textbf{towards any direction} is a \textit{zeroth-order Bessel
function
  of the first kind} that

where  is the bessel function with delay  when the
maximum Doppler shift is
 .
A general definition of the Bessel function, for integer values of 
 could be represented in an integral form:

For , the function can be further simplified to

There is a large positive value  , when ,
\eqqref{eq:bessel3} can be approximated by a periodical function
that

Substituting \eqqref{eq:bessel1} into Equation \eqqref{eq:bessel4}, we get:

Substituting the maxium Doppler frequency 
into  \eqqref{eq:bessel5}, we obtain:

where .

Apparently, , and  \eqqref{eq:bessel6}
tells a very elegant result that, \textbf{iff} ,


The periodicity of the auto-correlation function of  will
undoubtedly reflect on the original CIR , and CFR
. Thereby we could say, in a typical Rayleigh fading dominant
isotropically scattering environment, the distance 
between two adjacent deep fading(s) is a half-wavelength .
For example, In a 2.4G channel the wavelength , and in 5G channel the wavelength .

Thus, for the purpose of estimating moving distance,
 if there is a method to accurately count the number of deep fading(s)
  during a time interval , we are able to precisely
 estimate the moving distance  by


Noticing that the error caused by Doppler effect does exist in the
estimated distance .
However since Doppler frequency caused by human
 walking  is in tenth level, which is too small compared to the
 channel frequency.
Thus, the error caused by Doppler effect is ignored.


\subsubsection{In Rician Fading Channel}
distinction between generalized Rician fading and Rayleigh fading.
The ripple-like deep fading is caused by strong interference
 (constructive or destructive) in multipath environment. When In strong rician
fading environment,
   high  value will
 weaken the multipath effect and make it difficult to recognize and
 count the ripples.

However,according to our experiments in typical indoor environment, we found
that even in a
 always-LoS path, high value  rarely happens.
Figure \ref{fig:ricianfactor} plot the   values estimated from two
clients during the test using a moment-based estimator according to
\cite{ricianKestimate}.
The client in the path has always-LoS connection,
 while the LoS component is cut-off for the client placed in
 a metal-framed cubicle.

\begin{figure}[t]
\begin{center}
\includegraphics[scale=0.9]{ricianFactorK.pdf}
\end{center}
\caption{Estimated Rician Factor  during the test, where the mobile
  AP is moving towards the end. Higher  denotes stronger LoS
  components reception. The client in a path has always-LoS connection,
  while there is no LoS component for the client in the metal-framed
  cubicle.}
\label{fig:ricianfactor}
\end{figure}

Out of our expectation, comparing to the client in cubicle, the  value
 measured by the client in the path doesn't benefit much from the
 always-LoS connection.
Very large  values only happen occasionally at the
 beginning, while it increases very slowly even when the AP is quite
 close to the end.
The occasional large values  at prophase is simply caused by the indoor
structure at these locations, where additional propagation path is
introduced. These additional paths dilute and absorb some of the
multipath component.
Since the large value   rarely happens even in the condition with
LoS, the distance measured using the Rician fading could be approximated
according to the moving velocity before and after the Rician fading.

 


\subsection{Speed Estimation Algorithm}

As previously described, the speed estimation problem now becomes a
specific frequency estimation problem. We design a reasonable and effective
processing flow.  It includes Data Preprocessing, Noise Cancellation, Fading Enhancement, and Frequency Estimation.
the oscillation frequency .


\textit{Data Preprocessing:}
Every frame sent in 802.11n MCS rate at time  has an CSI . It
is a complex-number vector with a length of ,
where  and  are the number of MIMO spatial streams and
number of measured subcarriers across the Wi-Fi bandwidth. Every
complex value  describes the instantaneous amplitude
 and phase  of the underlying -th subcarrier.
In order to enable all 802.11n compatible devices to be ready for
speed estimation, we only use the first  spatial stream (first
 complex values in ) to estimate speed. Moreover, the
computation space is greatly reduced.
Since in multipath environment the phase  is uniformly
distributed between \cite{rappaport1996wireless}, which
provides no discriminative information.
Thus we drop the phase 
and only use amplitude  to estimate speed.

The amplitude matrix  is further
defined, where  is the th received column-wise
amplitude. Since the instantaneous reception rate of frames is
unstable due to the wireless traffic control,  is resampled
to a stable reception frequency  with the even interval between
each slot, and let  denote the resampled amplitude
matrix.


\textit{Noise Cancellation:}
Convolution based noise cancellation is applied on 
to filter out the high frequency noise, that
,

where  is a full- square matrix of size
. Currently in our system . This step is of great importance
according to the real data evaluation, since the following fading
enhancement and frequency estimation is quite sensitive to noise.

\textit{Fading Enhancement:}
An intuitive idea of enhance the fading is first-order
derivation of , however, first-order
derivation is quite sensitive to high frequency noise
rather than low frequency ripples. 
Another convolution is used to emphasize the fading that , where  is a Sobel-style
calculator that

Fig.~\ref{fig:power} shows the intermediate results After first 3 processing, and it is now suitable for frequency estimation.


\begin{figure}[t]
\begin{center}
\includegraphics[scale=0.5]{powerWaves2.pdf}
\vspace{-0.15in}
\caption{The effect of each step of processing.}
\label{fig:power}
\vspace{-0.2in}
\end{center}
\end{figure}

\textit{Frequency Estimation:}
Due to the MIMO configuration or other interference, deep fading(s) in
all subcarriers are not guaranteed to appear simultaneously, as shown
in Fig.~\ref{fig:csisample}, therefore the final decision of  are
based on the estimations of each underlying subcarriers.

Extracting  for -th subcarrier is equivalent to extracting
the expected frequency  in the spectral graph within a
frequency interval . According to
Eq. \eqref{eq:speed},  and  are set according to the
speed interval of human walking that

where the minimum speed  and the maximum speed  in
our system are set to 0.8m/s and 1.6m/s.

Short-Time Fourier Transformation (STFT) with 50\% overlapping window
is applied to obtain the Power Spectral Density (PSD) of -th
envelope of . It reveals the spectral density of subcarrier
 along with time. To reduce the jitter, the estimated  is
set to the weighed expectation of frequencies between  and


where  denotes the power of frequency .

\begin{figure}[tb]
\begin{center}
\includegraphics[scale=0.67]{STFT.pdf}
\end{center}
\vspace{-0.25in}
\caption{STFT result for 15 subcarrier at 5.8Ghz (Channel 161).}
\label{fig:stft}
\vspace{-0.1in}
\end{figure}

Figure \ref{fig:stft} shows the STFT result for -th subcarrier.
We can see very strong power around
50Hz. As , and it is quite close to
 the real walking speed at about . Two red bars denote the
 and , and they are set to 32 and 64 according the
 and  settings.
The final estimation of  is set to the median of all estimated 

Then the moving speed   is estimated by


It was worth to mention that although the Doppler effect does exist,
as discussed in Section \ref{sec:related}, the Doppler frequency is
 small comparing to the carrier frequency. Thus in current
system design, we did not consider the Doppler effect caused
by human walking.

\subsection{Start/Stop Detection}



Fig.~\ref{Fig:motionDetection} (a) presents a CSI sample, where the user
starts moving around 400th sample.  Observing the degree of disorder before
and after the start, we devise a correlation-based start/stop detection method.
The basis on the method is to leverage the rapid spatial de-correlation property
of CSI.  We find that the correlation coefficient  between consecutive CSI
samples will drop rapidly if the spatial distance  between them
is larger than  . Thus, there will be a rapid
co-efficiency raising or drop to check the ``moving'' and ``static''
status. Fig.~\ref{Fig:motionDetection} (b) shows the samples'
correlation matrix.
When device is static, stable and high correlation co-efficiency holds
the entire upper-left area, while it disappears immediately when the
device starts moving. According to our
experimental evaluation, a devices is said to be moving when 
drops below , and the final detected time  is quite close to
the actual time .

\begin{figure}[t]
\begin{center}
\begin{tabular}{cc}
\includegraphics[scale=0.45]{motionOrigin.pdf} &
\includegraphics[scale=0.45]{motionCorrGraph.pdf}\\
(a) CSI View & (b) Correlation Coefficient\\
\end{tabular}
\end{center}
\caption { (a)  the CSI image of a sample. The movement starts at about 400
time-slot. (b) the correlation matrix of the CSI images shown in (a), the black
cross denotes the detected start. }
\label{Fig:motionDetection}
\end{figure}


\subsection{Minimum Sampling Rate}
Similar to sensor-based system, sufficient CSI sampling rate is critical for accuracy.
Due to the un-equal time distribution between fading and non-fading,
 the Nyquist sampling rate of  is not
 sufficient.
We carried out experiments to find the minimum  that
 can guarantee good accuracy.
Evaluations are carried out in a wide range of channel frequencies
including 2.4G (channel 1), 5.2Ghz (channel 40), 5.5G (channel 100)
and the highest 5.8G(Channel 161).
During the experiment, testers are walking at the same speed around
  and the mobile device in their hands are constantly
 transmitting beacon frames at 500hz.
After the experiments, we simulate the sampling rate  from 20hz
 to 500hz by dropping frames uniformly.
 Fig.~\ref{fig:sampling} presents the results. We can see from the figure that
 the estimated speed  continuously climbs when  is higher than
 Nyquist rate , and the speed stops raising when  is about 4
 times of . More experiments in other situations have
 also confirmed the  sampling rate. Therefore if we set
  , the minimum sampling rate is only 100 (or 250)
 frames/s in 2.4G (or 5.8G) environment,  or equivalent to approximately
 40KBps or 100KBps traffic.

\begin{figure}[hptb]
\begin{center}
\includegraphics[scale=0.6]{samplingRate.pdf}
\end{center}
\vspace{-0.3in}
\caption{The measured speeds in different sampling rate.  The points
  denoted by  denotes the Nyquist sampling rate, while
   denotes the minimum require sampling rate.}
\label{fig:sampling}
\vspace{-0.1in}
\end{figure}

The traffic burstiness is another problem. The burstiness, which
happened frequently, is obviously against the CSI-based speed
estimation. Since the burstiness is usually short-time high-frequency
traffic phenomenon, a reasonable assumption could be made to ease this
problem: people's walking speed remain stable during the gap between
two burstiness. Fig.~\ref{fig:burstiness} presents our solution that
during each burstiness the speed is estimated, while in the gap the
speed is approximated as the average.
\begin{figure}[hptb]
\begin{center}
\includegraphics[scale=0.6]{burstiness.pdf}
\end{center}
\vspace{-0.3in}
\caption{The burstiness of wireless traffic in practical environment
  and the estimated speed. }
\label{fig:burstiness}
\vspace{-0.2in}
\end{figure}
