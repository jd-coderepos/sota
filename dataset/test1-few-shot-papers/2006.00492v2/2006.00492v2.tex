\documentclass[journal]{IEEEtran}
\usepackage{cite}
\usepackage{amsmath,amssymb,amsfonts}
\usepackage{graphicx}
\usepackage{textcomp}
\usepackage{xcolor}
\usepackage{caption}
\usepackage{subcaption}
\usepackage{multirow}
\usepackage{amssymb}
\usepackage{amsmath}
\usepackage{url}
\usepackage{soul}
\usepackage{color}
\usepackage{booktabs}
\usepackage{enumitem}
\usepackage{algorithm}
\usepackage[noend]{algpseudocode}

\definecolor{yellow}{rgb}{1.0, 0.65, 0.0}

\title{BiERU: Bidirectional Emotional Recurrent Unit \\for Conversational Sentiment Analysis}

\author{Wei Li,
        Wei Shao,
        Shaoxiong~Ji,
        and~Erik~Cambria,~\IEEEmembership{Senior Member,~IEEE}

\thanks{Wei Li, Nanyang Technological University, Singapore}
\thanks{Wei Shao (equal contribution), Peking University, China}\thanks{Shaoxiong Ji, Aalto University, Finland}

\thanks{Corresponding Author: Erik Cambria (cambria@ntu.edu.sg), Nanyang Technological University, Singapore}
}

\date{}

\begin{document}
\maketitle
\begin{abstract}
Sentiment analysis in conversations has gained increasing attention in recent years for the growing amount of applications it can serve, e.g., sentiment analysis, recommender systems, and human-robot interaction. The main difference between conversational sentiment analysis and single sentence sentiment analysis is the existence of context information which may influence the sentiment of an utterance in a dialogue. How to effectively encode contextual information in dialogues, however, remains a challenge. Existing approaches employ complicated deep learning structures to distinguish different parties in a conversation and then model the context information. In this paper, we propose a fast, compact and parameter-efficient party-ignorant framework named bidirectional emotional recurrent unit for conversational sentiment analysis. In our system, a generalized neural tensor block followed by a two-channel classifier is designed to perform context compositionality and sentiment classification, respectively. Extensive experiments on three standard datasets demonstrate that our model outperforms the state of the art in most cases.
\end{abstract}

\begin{IEEEkeywords}
Conversational sentiment analysis, emotional recurrent unit, contextual encoding, dialogue systems
\end{IEEEkeywords}

\section{Introduction}
\label{sec:intro}
\IEEEPARstart{S}{entiment} analysis and emotion recognition are of vital importance in dialogue systems and have recently gained increasing attention~\cite{maasur}. They can be applied to a lot of scenarios such as mining the opinions of  speakers in conversations, improving the feedback of robot agents, and so on. Moreover, sentiment analysis in live conversations can be used in generating talks with certain sentiments to improve human-machine interaction. Existing approaches to conversational sentiment analysis can be divided into party-dependent approaches, like DialogueRNN~\cite{majumder2019dialoguernn}, and party-ignorant approaches, such as AGHMN~\cite{jiao2019real}. Party-dependent methods distinguish different parties in a conversation while party-ignorant methods do not.  Both party-dependent and party-ignorant models are not limited to dyadic conversations. 
Nevertheless, party-ignorant models can be easily applied to multi-party scenarios without any adjustment. In this paper, we propose a fast, compact and parameter-efficient party-ignorant framework based on emotional recurrent unit (ERU), a recurrent neural network that contains a generalized neural tensor block (GNTB) and a two-channel feature extractor (TFE) to tackle conversational sentiment analysis.

Context information is the main difference between dialogue sentiment analysis and single sentence sentiment analysis tasks. It sometimes enhances, weakens, or reverses the raw sentiment of an utterance (Fig.~\ref{fig:DialogueSystem}). There are three main steps for sentiment analysis in a conversation: obtaining the context information, capturing the influence of the context information for an utterance, and extracting emotional features for classification. Existing dialogue sentiment analysis methods like c-LSTM~\cite{poria2017context}, CMN~\cite{hazarika2018conversational}, DialogueRNN~\cite{majumder2019dialoguernn}, and DialogueGCN~\cite{ghosal2019dialoguegcn} make use of complicated deep neural network structures to capture context information and describe the influence of context information for an utterance.

We redefine the formulation of conversational sentiment analysis and provide a compact structure to better encode the context information, capture the influence of context information for an utterance, and extract features for sentiment classification. According to Mitchell and Lapata~\cite{mitchell2010composition}, the meaning of a complete sentence must be explained in terms of the meanings of its subsentential parts, including those of its singular elements. Compositionality allows language to construct complicated meanings from its simpler terms. This property is often expressed in a manner of principle: the meaning of a whole is a function of the meaning of the components~\cite{partee1995lexical}. For conversation, the context of an utterance is composed of its historical utterances information. Similarly, context is a function of the meaning of its historical utterances. Therefore, inspired by the composition function in~\cite{partee1995lexical}, we design GNTB to perform context compositionality in conversation, which obtains context information and incorporates the context into utterance representation simultaneously, then employ TFE to extract emotional features. In this case, we convert the previous three-step task into a two-step task. Meanwhile, the compact structure reduces computational cost. To the best of our knowledge, our proposed model is the first to perform context compositionality in conversational sentiment analysis.

The GNTB takes the context and current utterance as inputs, and is capable of modeling conversations with arbitrary turns. It outputs a new representation of current utterance with context information incorporated (named `contextual utterance vector' in this paper). Then, the contextual utterance vector is further fed into TFE to extract emotional features. Here, we employ a simple two-channel model for emotion feature extraction.

The long short-term memory (LSTM) unit~\cite{hochreiter1997long} and one-dimensional convolutional neural network (CNN)~\cite{kim2014convolutional} are utilized for extracting features from the contextual utterance vector. Extensive experiments on three standard datasets demonstrate that our model outperforms state-of-the-art methods with fewer parameters. To summarize, the main contributions of this paper are as follows:

\begin{itemize}
    \item We propose a fast, compact and parameter-efficient party-ignorant framework based on emotional recurrent unit.
    \item We design generalized neural tensor block which is suitable for different structures, to perform context compositionality.
    \item Experiments on three standard benchmarks indicate that our model outperforms the state of the art with fewer parameters.
\end{itemize}

The remainder of the paper is organized as follows: related work is introduced in Section~\ref{sec:related}; the mechanism of our model is explained in Section~\ref{sec:method}; results of the experiments are discussed in Section~\ref{sec:experiments}; finally, concluding remarks are provided in Section~\ref{sec:conclusion}.

\begin{figure}[t]
\centering\includegraphics[width=\linewidth]{DialogueSystem_revised.png}
\caption{Illustration of dialogue system and the interaction between talkers.}
\label{fig:DialogueSystem}
\end{figure}

\section{Related~Work}
\label{sec:related}
Sentiment analysis is one of the key NLP tasks that has drawn great attention from the research community in the last decade~\cite{cambig}. Beside the basic task of binary polarity classification~\cite{camnt6}, sentiment analysis research has been carried out in many other related topics such as multimodal sentiment analysis~\cite{zadmul,ragima}, multilingual sentiment analysis~\cite{esucro}, aspect-based sentiment analysis~\cite{weiasp}, domain adaptation~\cite{banlex,xuuins}, rumors and fake news detection~\cite{akhnoo,reisup}, gender-specific sentiment analysis~\cite{mihwha,buktyp}, and multitask learning~\cite{yanseg}, including also applications of sentiment analysis in domains like healthcare~\cite{mahdet,qurmul}, political forecasting~\cite{ebrcha}, tourism~\cite{valsen}, customer relationship management~\cite{biicro}, stance classification~\cite{duucom}, and dialogue systems~\cite{wellea,schint}.

Sentiment analysis in dialogues, in particular, has become a new trend recently. 
Poria et al.~\cite{poria2017context} proposed context-dependent LSTM network to capture contextual information for identifying sentiment over video sequences, and Ragheb et al.~\cite{ragheb2019attention} utilized self attention to prioritize important utterances.
Memory networks~\cite{sukhbaatar2015end}, which introduce an external memory module, was applied to modeling historical utterances in conversations. For example, CMN~\cite{hazarika2018conversational} modeled dialogue histories into memory cells, ICON~\cite{hazarika2018icon} proposed global memories for bridging self- and inter-speaker emotional influences, and AGHMN~\cite{jiao2019real} proposed hierarchical memory network as utterance reader.
{Ghosal et al.~\cite{ghosal2020cosmic} incorporated commonsense knowledge to enhance emotion recognition.} 
Recent advances in deep learning were also introduced to conversational sentiment analysis like attentive RNN~\cite{majumder2019dialoguernn}, adversarial training~\cite{wang2019capturing}, and graph convolutional networks~\cite{ghosal2019dialoguegcn}.
Another emerging direction is to incorporate Transformer-based contextual embedding.
Zhong et al.~\cite{zhong2019knowledge} leveraged commonsense knowledge from external knowledge bases to enrich transformer encoder. 
Qin et al.~\cite{qin2020dcr} built a co-interactive relation network to model feature interaction from bidirectional encoder representations from transformers (BERT) for joint dialogue act recognition and sentiment analysis. 


Neural Tensor Networks (NTN)~\cite{socher2013reasoning} first proposed for reasoning over relational data are also related to our work. Socher et al.~\cite{socher2013recursive} further extended NTN to capture semantic compositionality for sentiment analysis. The authors proposed a tensor-based composition function to learn sentence representation recursively, which solves the issue when words function as operators that change the meaning of another word. 




\section{Method}
\label{sec:method}
\subsection{Problem Definition}

Given a multiple turns conversation , the task is to predict the sentiment labels or sentiment intensities of the constituent utterances . Taking the interactive emotional database IEMOCAP~\cite{busso2008iemocap} as an example, emotion labels include frustrated, excited, angry, neutral, sad, and happy.

In general, the task is formulated as a multi-class classification problem over sequential utterances while in some scenarios, it is regarded as a regression problem given continuous sentiment intensity. In this paper, utterances are pre-processed and represented as  using feature extractors described below.

\subsection{Textual Feature Extraction}
\label{sec:feature}
Following the tradition of DialogueRNN~\cite{majumder2019dialoguernn}, utterances are first embedded into vector space and then fed into CNNs~\cite{kim2014convolutional} for feature extraction. N-gram features are obtained from each utterance by applying three different convolution filters of sizes 3, 4, and 5, respectively. Each filter has 50 features-maps.~\cite{majumder2019dialoguernn} then use max-pooling followed by rectified linear unit (ReLU) activation~\cite{nair2010rectified} to process the outputs of convolution operation.

These activation values are concatenated and fed to a 100 dimensional fully connected layer whose outputs serve as the textual utterance representation. This CNN-based feature extraction network is trained at utterance level supervised by the sentiment labels.



\subsection{Our Model}

\begin{figure*}[tp]
\centering\includegraphics[width=0.95\textwidth]{model_final.pdf}
\caption{(a) Architecture of BiERU with global context. (b) Architecture of BiERU with local context. Here , , and  are forward contextual utterance vector, TFE, and GNTB, respectively.  and  stand for backward contextual utterance vector and ERU, respectively.  is the predicted possibility vector of sentiment labels. T refers to textual modality in this paper. In our model, we only focus on textual modality. The detailed structures of GNTB and TFE are shown in Fig.~\ref{fig:ntb_and_tfe}.}
\label{fig:model}
\end{figure*}
Our ERU is illustrated in Note 1 of Fig.~\ref{fig:model}, which consists of two components GNTB and TFE. As mentioned in the introduction, there are three main steps for conversational sentiment analysis, namely obtaining the context representation, incorporating the influence of the context information into an utterance, and extracting emotional features for classification. In this paper, the ERU is employed in a bidirectional manner (BiERU) to conduct the above sentiment analysis task, reducing some expensive computations and converting the previous three-step task into a two-step task as shown in Fig.~\ref{fig:model}.

Similar to bidirectional LSTM (BiLSTM)~\cite{graves2005framewise}, two ERUs are utilized for forward and backward passing the input utterances. Outputs from the forward and backward ERUs are concatenated for sentiment classification or regression. More concretely, the GNTB is applied to encoding the context information and incorporating it into an utterance simultaneously; while TFE takes the output of GNTB as input and is used to obtain emotional features for classification or regression.


\subsubsection{Generalized Neural Tensor Block}
The utterance vector  with the context information incorporated is named as contextual utterance vector  in this paper, where  is the dimension of  and . At time , GNTB (Fig.~\ref{fig:ntb_and_tfe}: (a)) takes  and  as inputs and then outputs , a contextual utterance vector. In this process, GNTB first extracts the context information from ; it then incorporates the context information into ; finally, contextual utterance vector  is obtained. The first step is to capture the context information and the second step is to integrate the context information into current utterance. The combination of these two steps is regarded as context compositionality in this paper. To the best of our knowledge, this is the first work to perform context compositionality in conversational sentiment analysis. GNTB is the core part that achieves the context compositionality. The formulation of GNTB is described below:


where  is the concatenation of  and ;  is an activation function, such as ,  and so on; the tensor  and the matrix  are the parameters used to calculate . Each slice  can be interpreted as capturing a specific type of context compositionality. Each slice  maps contextual utterance vector  and utterance vector  into the context compositionality space. Here we have  different context compositionality types, which constitutes -dimensional context compositionality space. The main advantage over the previous neural tensor networks (NTN)~\cite{socher2013reasoning}, which is a special case of the GNTB when  is set to , is that GNTB is suitable for different structures rather than only the recursive structure and the space complexity of GNTB is  compared with  in NTN. In order to further reduce the number of parameters, we employ the following low-rank matrix approximation for each slice :

where , ,  and .


\subsubsection{Two-channel Feature Extractor}

We utilize TFE to refine the emotion features from contextual vector . As shown in Fig.~\ref{fig:ntb_and_tfe}: (b), the TFE is a two-channel model, including a LSTM cell~\cite{hochreiter1997long} branch and a one-dimensional CNN~\cite{kim2014convolutional} branch. The two branches receive the same contextual utterance vector  and produce outputs that may contain complementary information~\cite{li2020user}.

At time , the LSTM cell takes hidden state , cell state  and the contextual utterance vector  as inputs, where  and  are obtained from the last time step . The outputs of the LSTM cell are updated hidden state  and cell state . The hidden state  is regarded as the emotion feature vector. The CNN receives  as input and outputs the emotion feature vector . Finally, the outputs of LSTM cell branch  and CNN branch  are concatenated into an emotion feature vector  which is also the output of ERU. The formulas of TFE are as follows:



\subsubsection{Sentiment Classification \& Regression}
Taking emotion feature  as input, we use a linear neural network  followed by a softmax layer to predict the sentiment labels, where  is the number of sentiment labels.

Then, we obtain the probability distribution  of the sentiment labels. Finally, we take the most possible sentiment class as the sentiment label of the utterance :


For sentiment regression task, we use a linear neural network  to predict the sentiment intensity. Then, we obtain the predicted sentiment intensity :

where , , ,  is a scalar and  is the predicted sentiment label for utterance .



\subsubsection{Training}
For classification task, we choose cross-entropy as the measure of loss, and use L2-regularization to relieve overfitting. The loss function is:


For regression task, we choose mean square error (MSE) to measure loss, and L2-regularization to relieve overfitting. The loss function is:

where N is the number of samples/conversations,  is the probability distribution of sentiment labels for utterance  of conversation ,  is the expected class label of utterance  of conversation ,  is the predicted sentiment intensity of utterance  of conversation ,  is the expected sentiment intensity of utterance  of conversation , c(i) is the number of utterances in sample ,  is the L2-regularization weight, and  is the set of trainable parameters. We employ stochastic gradient descent based Adam~\cite{kingma2014adam} optimizer to train our network.


\subsection{Bidirectional Emotion Recurrent Unit Variants}
Our model has two different forms according to the source of context information, namely bidirectional emotion recurrent unit with global context (BiERU-gc) and bidirectional emotion recurrent unit with local context (BiERU-lc).

\begin{figure}[!ht]
\centering\includegraphics[width=0.35\textwidth]{GNTB_and_TFE.pdf}
\caption{(a) GNTB when . (b) TFE. The input of LSTM and CNN is context utterance vector , and output is emotion features .}
\label{fig:ntb_and_tfe}
\end{figure}

\subsubsection{BiERU-gc}
According to equation (\ref{E1}), GNTB extracts the context information from , integrates the context information into , and thus obtains the contextual utterance vector . Based on the definition of contextual utterance vector,  is the utterance vector that contains information of  and . In this case, the contextual utterance vector  holds the context information from all the preceding utterances  in a recurrent manner. 
{Bidirectional neural networks have empirically gained improved performance than its counterpart with only forward propagation~\cite{schuster1997bidirectional}. As shown in Fig.~\ref{fig:model} : (a), we utilize the bidirectional setting to capture context information from surrounding utterances.}
The BiERU in Fig.~\ref{fig:model} :(a) is named as BiERU-gc.
\subsubsection{BiERU-lc}
Following equation (\ref{E1}), GNTB extracts the context information from the contextual utterance vector , and  contains the context information of all the preceding utterances  as mentioned above. If replacing  with  in equation (\ref{E1}) and (\ref{E2}),  contains the information of  and . In other words,  is not only an utterance vector, but also works as the context of . As shown in Fig.~\ref{fig:model} : (b), bidirectional ERU makes  obtain the future information . In this case, GNTB extracts the context information from  and , which are the adjacent utterances of . We name this model as BiERU-lc.

\section{Experiments}
\label{sec:experiments}
In this section, we conduct a series of comparative experiments to evaluate the performance of our proposed model (Codes are available on our GitHub\footnote{https://github.com/Maxwe11y/BiERU}) and perform a thorough analysis.

\subsection{Datasets}
We use three datasets for experiments, i.e., AVEC~\cite{schuller2012avec}, IEMOCAP~\cite{busso2008iemocap} and MELD~\cite{poria2018meld}, which are also used by some representative models such as DialogueRNN~\cite{majumder2019dialoguernn} and DialogueGCN~\cite{ghosal2019dialoguegcn}. We conduct the standard data partition rate (details in Table~\ref{tab:data}).\begin{table}[!htb]
    \centering
\begin{tabular}{|c|c|c|c|}
        \hline
        DATASET & Partition & \ Utterance Count & \ Dialogue Count \\
        \hline
        \hline
        \multirow{2}{*}{IEMOCAP} & train + val & 5810 & 120 \\
        \cline{2-4}
        & test & 1623 & 31 \\
        \hline
        \multirow{2}{*}{AVEC} & train + val & 4368 & 63 \\
        \cline{2-4}
        & test & 1430 & 32 \\
        \hline
        \multirow{2}{*}{MELD} & train + val & 11098 & 1153 \\
        \cline{2-4}
        & test & 2610 & 280 \\
        \hline
    \end{tabular}
\caption{Statistical information and data partition of datasets used in this paper.}
    \label{tab:data}
\end{table}{}

Originally, these three datasets are multimodal datasets. Here, we focus on the task of textual conversational sentiment analysis, and only use the textual modality to conduct our experiments.

\paragraph{IEMOCAP}
The IEMOCAP~\cite{busso2008iemocap} is a dataset of two-way conversations involved with ten distinct participators. It is recorded as videos where every video clip contains a single dyadic dialogue, and each dialogue is further segmented into utterances. Each utterance is labeled as one sentiment label from six sentiment labels, i.e., happy, sad, neutral, angry, excited and frustrated. The dataset includes three modalities: audio, textual and visual. Here, we only use textual modality data in experiments.

\paragraph{AVEC}
The AVEC dataset~\cite{schuller2012avec} is a modified version of the SEMAINE database~\cite{mckeown2011semaine} that contains interactions between human speakers  and robots. Unlike IEMOCAP, each utterance in the AVEC dataset is given an annotation every 0.2 second with one of four real valued attributes, i.e., valence (), arousal (), expectancy (), and power (). Our experiments use the processed utterance-level annotation~\cite{majumder2019dialoguernn}, and treat four affective attributes as four subsets for evaluation.

\paragraph{MELD}
The MELD~\cite{poria2018meld} is a multimodal and multiparty sentiment analysis/classification database. It contains textual, acoustic, and visual information for more than 13000 utterances from Friends TV series. The sentiment label of each utterance in a dialogue lies within one of the following seven sentiment classes: fear, neutral, anger, surprise, sadness, joy and disgust.
\subsection{Baselines~and~Settings}
To evaluate performance of our model, we choose the following models as strong baselines including the state-of-the-art methods.



\paragraph{c-LSTM~\cite{poria2017context}}
The c-LSTM uses bidirectional LSTM~\cite{hochreiter1997long} to learn contextual representation from the surrounding utterances. When combined with the attention mechanism, it becomes the c-LSTM+Att.



\paragraph{CMN~\cite{hazarika2018conversational}}
This model utilizes memory network and two different GRUs~\cite{cho2014learning} for two speakers for representation learning of utterance context from dialogue history.


\paragraph{DialogueRNN~\cite{majumder2019dialoguernn}}
It distinguishes different parties in a conversation interactively, with three GRUs representing the speaker states, context, and emotion.
It has several variants including DialogueRNN+Att with attention mechanism and bidirectional BiDialgoueRNN.


\paragraph{DialogueGCN~\cite{ghosal2019dialoguegcn}}
This model employs graph neural network based approach through which context propagation issue can be addressed, to detect sentiment in conversations.
\paragraph{AGHMN~\cite{jiao2019real}}
It utilizes hierarchical memory networks with BiGRUs for utterance reader and fusion, and attention mechanism for memory summarizing.

\paragraph{Settings}
All the experiments are performed using CNN extracted features as described in Method section. For fair comparison with the state-of-the-art DialogueRNN model, we use their utterance representation directly\footnote{Extracted features of two datasets are available at \url{https://github.com/senticnet/conv-emotion}.}.

To alleviate over-fitting, we employ
Dropout~\cite{srivastava2014dropout} over the outputs of GNTB and TFE. For the nonlinear activation function, we choose the sigmoid function for sentiment classification and the relu function for sentiment regression. Our model is optimized by an Adam optimizer~\cite{kingma2014adam}. Hyper-parameters are tuned manually. Batch size is set as 1. We set the rank for all the experiments to . Our model is implemented using PyTorch~\cite{paszke2019pytorch}. In Table \ref{tab:param}, we display the hyper-parameters of our BiERU-lc model on the three standard datasets.

\begin{table}[!htb]
    \centering
    \begin{tabular}{|c|c|c|c|}
        \hline
        \multirow{2}{*}{DATASET} & Dropout & \ Learning & \ Regularization \\
        & Rate & Rate & Weight\\
        \hline
        \hline
        IEMOCAP & 0.8 & 0.0001 & 0.001 \\
        \hline
        AVEC.VALENCE & 0.5 & 0.0001 & 0.0002 \\
        AVEC.AROUSAL & 0.8 & 0.0001 & 0.0002  \\
        AVEC.EXPECTANCY & 0.5 & 0.00005 & 0.0005  \\
        AVEC.POWER & 0.8 & 0.0001 & 0.0001  \\
        \hline
        MELD & 0.7 & 0.0005 & 0.001 \\
        \hline
    \end{tabular}
\caption{Hyper-parameters of our BiERU-lc model on different datasets.}
    \label{tab:param}
\end{table}{}


\linespread{1.25}
\begin{table*}[!ht]
    \centering
\begin{tabular}{|c||cc|cc|cc|cc|cc|cc|cc||c|}
    \hline
    \multirow{3}{*}{METHODS} & \multicolumn{14}{c||}{IEMOCAP} &
    \multicolumn{1}{c|}{MELD} \\
    \cline{2-16}
    & \multicolumn{2}{c|}{Happy} & \multicolumn{2}{c|}{Sad} & \multicolumn{2}{c|}{Neutral} & \multicolumn{2}{c|}{Angry} & \multicolumn{2}{c|}{Excited} & \multicolumn{2}{c|}{Frustrated} & \multicolumn{2}{c||}{Average} &
    \multicolumn{1}{c|}{Average} \\
    \cline{2-16}
	&	Acc. &	F1 &  Acc.	& F1 & Acc.	& F1 & Acc.	&	F1 &  Acc.	&	F1 & Acc.	&	F1 &  Acc.	&	F1  & Acc.	\\


\hline
c-LSTM & 30.56 & 35.63 & 56.73  & 62.90 & 57.55 & 53.00 & 59.41 & 59.24 & 52.84 & 58.85 & 65.88 & 59.41 & 56.32 & 56.19 & 57.5    \\
\hline
CMN & 25.00 & 30.38 & 55.92 & 62.41 & 52.86 & 52.39 & 61.76 & 59.83 & 55.52 & 60.25 & \textbf{71.13} & 60.69 & 56.56 & 56.13 & -    \\
\hline
DialogueRNN	&	25.69 & 33.18	&	75.10 & 78.80	&	58.59 & 59.21	&	64.71 & \textbf{65.28}	&	\textbf{80.27} & 71.86	&	61.15 & 58.91	&	63.40	&	62.75 &  56.1 \\
\hline
DialogueGCN	&	40.62  & \textbf{42.75}	&	\textbf{89.14} & \textbf{84.54}	&	61.92 & \textbf{63.54}	&	67.53 & 64.19	&	65.46 & 63.08	&	64.18 & \textbf{66.99}	&	65.25	&	64.18	& - \\
\hline
AGHMN	&	48.30 & \textbf{52.1}	&	68.30 & 73.3	&	61.60 & 58.4	&	57.50 & 61.9	&	\textbf{68.10} &  69.7	&	\textbf{67.10} & \textbf{62.3}	&	63.50	&	63.50  & 60.3 \\
\hline
BiERU-gc	&	\textbf{49.81} & 32.75	&	\textbf{81.26} & 82.37	&	\textbf{65.00} & \textbf{60.45}	&	\textbf{67.86} & \textbf{65.39}	&	63.14 & \textbf{73.29}	&	59.77 & 60.68	&	\textbf{65.35} & \textbf{64.24} & \textbf{60.7}    \\
\hline
BiERU-lc	& \textbf{55.44} & 31.56	&	80.19  & \textbf{84.13}	&	\textbf{64.73} & 59.66	&	\textbf{69.05} & 65.25	&	63.18 & \textbf{74.32}	&	61.06 & 61.54	&	\textbf{66.09}	&	\textbf{64.59}	& \textbf{60.9}\\
    \hline
    \end{tabular}
\linespread{1}
    \caption{Comparison with baselines on IEMOCAP and MELD datasets using textual modality. Average score of accuracy and f1-score are weighted. ``-'' represents no results reported in original paper.}
    \label{tab:results}
\end{table*}


\begin{table}[!ht]
    \centering
    \begin{tabular}{|c|c|c|c|c|}
    \hline
    \multirow{3}{*}{METHODS} & \multicolumn{4}{c|}{AVEC} \\
    \cline{2-5}
    & \multicolumn{1}{c|}{Valence} & \multicolumn{1}{c|}{Arousal} & \multicolumn{1}{c|}{Expectancy} & \multicolumn{1}{c|}{Power} \\
    \cline{2-5}
	&	r	&	r	&	r	&	r \\

\hline
c-LSTM & 0.16 & 0.25 & 0.24 & 0.10   \\
\hline
CMN &  0.23 & 0.29 &  0.26 & -0.02  \\
\hline
DialogueRNN	&	0.35	&	0.59	&	0.37	&	\textbf{0.37}   \\
\hline
BiERU-gc	& 0.30	&	0.63	&	0.36	&	0.36  \\
\hline
BiERU-lc	& \textbf{0.36}	&	\textbf{0.64}	&	\textbf{0.38}	&	\textbf{0.37} \\
    \hline
    \end{tabular}
\linespread{1}
    \caption{Comparison with baselines on AVEC dataset using textual modality.  stands for Pearson correlation coefficient.}
    \label{tab:result_avec}
\end{table}



\subsection{Results}
We compare our model with baselines on textual modality using three standard benchmarks. We run the experiment five times and report the average results. Overall, our model outperforms all the baseline methods including state-of-the-art models like DialogueRNN, DialogueGCN and AGHMN on these datasets, and markedly exceeds in some indicators as the results show in Table~\ref{tab:results}. 

For the IEMOCAP dataset as a classification problem, we use accuracy for each class, and weighted average of accuracy and f1-score for measuring the overall performance. As for the AVEC dataset, standard metrics for regression task including Pearson correlation coefficient () are used for evaluation. We use weighted average of accuracy as the measure of performance on MELD dataset.

\subsubsection{Comparison with the State of the Art}
We firstly compare our proposed BiERU with state-of-the-art methods DialogueGCN, DialogueRNN and AGHMN on IEMOCAP, AVEC and MELD, respectively.


\paragraph{IEMOCAP}
As shown in Table~\ref{tab:results}, our proposed BiERU-gc model exceeds the best model DialogueGCN by  and  in terms of weighted average accuracy and f1-score, respectively. And the BiERU-lc model pushes up state-of-the-art results by  and  for weighted average accuracy and f1-score, respectively. For all 14 indicators on IEMOCAP dataset, our models outperform at 7 indicators and has more balanced performances over these six classes. In particular, accuracy of ``happy'' of our proposed BiERU-lc is higher than the result of DialgoueGCN by . In the DialogueGCN model, the authors employ a two-layer graph convolutional network to model the interactions between speakers within a sliding window. For dyadic conversations, there are 4 different relations and context information is scattered into each relation. In this case, however, the context information is incomplete and inadequate for each relation. Besides, window size is fixed, which makes it inflexible to different scenarios. Our proposed models, to some extent, is more capable of capturing adequate context information. To sum up, the experimental results indicate that BiERU models can effectively capture the contextual information and extract rich emotion features to boost the overall performance and achieve relatively balanced results. 



\paragraph{AVEC}
Among these four attributes, our model outperforms DialogueRNN for ''valence'', ``arousal'' and ``expectancy'' attributes and obtains the same results on ''power'' attribute. The pearson correlation coefficient  of BiERU-gc is  higher than its counterpart in terms of ``arousal'' (Table~\ref{tab:result_avec}). As for the BiERU-lc model, it is  higher in . For the attributes ``expectancy'' and ''valence'', the BiERU-lc model is  higher in . As for the attribute``power'', although our best model does not outperform the state-of-the-art method, it surpasses most of the other baseline methods including CMN and c-LSTM.
Overall, the BiERU-lc model works well on all the attributes, considering the benchmark performances are very high.  As mentioned in part~\ref{tab:data} of section~\ref{sec:experiments}, AVEC is composed of conversations between human speakers and robots. Robots are not good at identifying global information and tend to respond to adjacent queries from human speakers. This is one possible reason that our BiERU-lc model has better performances than baselines and BiERU-gc since it is skilled at capturing local context information.


\paragraph{MELD}
Three factors make it considerably harder to model sentiment analysis on MELD in comparison with IEMOCAP and AVEC datasets. First, the average number of turns in a MELD conversation is 10 while it is close to 50 on IEMOCAP. Second, there are more than 5 speakers in most of the MELD conversations, which means most of the speakers only utter one or two utterances per conversation. What's worse, sentiment expressions rarely exist in MELD utterances and the average length of MELD utterances is much shorter than it is in IEMOCAP and AVEC datasets. For a party-dependent model like DialogueRNN, it is hard to model inter-dependency between speakers. We find that the performances of party-ignorant models such as c-LSTM and AGHMN are slightly better than party-dependent models on this dataset. Our BiERU models utilize GNTB to perform context compositionality and achieve the state-of-the-art average accuracy of , outperforming AGHMN by  and DialogueRNN by .

\subsubsection{Comparison between BiERU-gc and BiERU-lc}
The proposed two variants take different context inputs. The BiERU-gc model takes the output of GNTB at last time step and current utterance as the input of GNTB at current time step. And the BiERU-lc model uses the last utterance and current utterance as input of GNTB at current time step. According to experimental results in Tables~\ref{tab:results} and~\ref{tab:result_avec}, the overall performance of BiERU-lc is better than BiERU-gc.

For IEMOCAP datasets, the BiERU-lc model surpasses the BiERU-gc model by  and  in terms of weighted average accuracy and f1-score, respectively. For the AVEC and MELD datasets, BiERU-lc also outperforms its counterpart. One possible explanation is that context information of a contextual utterance vector in BiERU-gc comes from all utterances in the current conversation. However, in BiERU-lc, the context information comes from neighborhood utterances. In this case, context information of BiERU-gc contains redundant information and thus has a negative impact on emotion feature extraction.

\begin{figure}[!ht]
    \centering
	\includegraphics[width=0.45\textwidth]{cm-our.pdf}
	\linespread{1}
    \caption{Heat map of confusion matrix of BiERU-lc.}
\label{fig:heatmap}
\end{figure}{}


\subsection{Case~Study}
Figure~\ref{fig:case_study} illustrates a conversation snippet classified by our BiERU-lc method. In this snippet, person A is initially in a frustrated state while person B acts as a listener in the beginning. Then, person A changes his/her focus and questions person B on his/her job state. Person B tries to use his/her own experience to help person A get rid of the frustrating state. This snippet reveals that the sentiment of a speaker is relatively steady and the interaction between speakers may change the sentiment of a speaker. Our BiERU-lc method shows good ability in capturing the speaker's sentiment (turns 9, 11, 12, 14) and the interaction between speakers (turn 10). The sentiment in turn 13 is very subtle. Turn 13 contains a little bit of frustration since he/she is not satisfied with his/her job state. However, considering that person B attempts to help person A, turn 13 is more likely to be in a neutral stand. Besides, we also display the prediction results of baselines including the state of the art in Fig.~\ref{fig:case_study}. On the one hand, both the DialogueRNN and DialogueGCN models cannot successfully model the interaction between the two speakers in this dialogue snippet. On the other hand, the two baselines are more likely to classify a few consecutive utterances into the same emotion label, which indicates that they are insensitive to the context information. In contrast, our BiERU-gc model gets better results and detects the emotion shifting from frustrated to neutral and from frustrated to neutral. However, global context may contain noise information that is not related to the current utterance, which weakens the proportion of related information and makes the BiERU-gc model less sensitive to sentiment shifting. In this case, the BiERU-lc model obtains the best results on this dialogue snippet since it is more sensitive to the context information and has a better context compositionality ability in general.

\begin{figure*}[!ht]
    \centering
	\includegraphics[width=1.00\textwidth]{case_example.png}
	\linespread{1}
\caption{Illustration of a conversation snippet from IEMOCAP dataset.}
\label{fig:case_study}
\end{figure*}{}





\subsection{Visualization}
We use visualization to provide some insights of the proposed model. Firstly, we visualize the confusion matrix in the form of a heat map to describe the performance of our BiERU-lc model. The heat maps of BiERU-lc on the IEMOCAP dataset are shown in Fig.~\ref{fig:heatmap}.
Our model has a balanced performance over all the sentiment classes.

Secondly, we perform deeper analysis of our proposed model and DialogueRNN by visualizing the learned emotion feature representations on IEMOCAP as shown in Fig.~\ref{fig:Dimensionality Reduction of BiERU} and Fig.~\ref{fig:Dimensionality Reduction of DialogueRNN}. Vectors fed into the last dense layer followed by softmax for classification are regarded as emotion feature representations of utterances. We use principal component analysis~\cite{wold1987principal} to reduce the dimension of emotion representations from our model (BiERU-lc) and DialogueRNN. The emotion representation is reduced to be 3-dimensional. In Fig.~\ref{fig:Dimensionality Reduction of BiERU} and Fig.~\ref{fig:Dimensionality Reduction of DialogueRNN}, each color represents a predicted sentiment label and the same color means the same sentiment label. The figures show that our model outperforms on extracting emotion features of utterances labeled "happy", which is consistent with the results in Table 2. In detail, neutral is an intermediate emotion and every other emotion can smoothly transfer into neutral and vice versa. Therefore, in both our BiERU model and DialogueRNN model, neutral has more overlapping regions compared with other emotions. Compared with DialogueRNN, our model distinguishes happy \& excited, frustrated \& angry more clearly. Therefore, our model has the ability to learn better emotion features to some extent.

\begin{figure}[ht!]
\centering
\begin{subfigure}{0.35\textwidth}
    \includegraphics[width=\linewidth]{our.pdf}
    \subcaption{BiERU-lc}
    \label{fig:Dimensionality Reduction of BiERU}
\end{subfigure}
\begin{subfigure}{0.35\textwidth}
    \includegraphics[width=\linewidth]{rnn.pdf}
    \subcaption{DialogueRNN}
\label{fig:Dimensionality Reduction of DialogueRNN}
\end{subfigure}
\linespread{1}
\caption{Visualization of learned emotion features via dimensionality reduction.}
\label{fig:DR}
\end{figure}{}

\subsection{Efficiency~Analysis}
We analyze the efficiency of our proposed BiERU model by comparing it with two recent strong baselines. 
Two variants of our model, i.e., BiERU-gc and BiERU-lc, are included.
We choose DialogueRNN and DialogueGCN for comparison as these two are recent competitive methods with public source code. 
Our proposed model has advantages over DialogueRNN, in terms of convergence capacity, the number of trainable parameters, and training time.
In the comparison with DialogueGCN, our models take much less training time. 
Figure~\ref{fig:loss} shows the training curve with training and testing loss plotted. We utilize the same loss function for all the compared models.
Our BiERU-lc and BiERU-gc show comparable convergence speed with their counterparts, while DialogueRNN is prone to overfitting. 
{We also compare our model with AGHMN. The AGHMN model uses different loss function, making its loss quite different from our model, DialogueRNN and DialogueGCN. Thus, we do not include the AGHMN model in the training curve illustration.
}


{Our BiERU-gc has fewer trainable parameters and takes less training time than DialogueRNN and DialogueGCN.}
Moreover, BiERU-lc with low-rank matrix approximation has further reduced trainable parameters.
For 100D feature input in the IEMOCAP dataset, our model has about 0.5M parameters, while DialogueRNN requires around 1M. 
For 600D MELD dataset, DialogueRNN has 2.9M parameters, and our BiERU-lc only has 0.6M. 
With much fewer parameters, our model consequently trains faster than its counterpart as shown in Fig.~\ref{fig:time}, where training time is logged in a single NVIDIA Quadro M5000. 
Our BiERU model with either global or local context is more parameter-efficient and less time-consuming for training.
{When compared with AGHMN (UniF\_BiAGRU\_CNN), our model has slightly more trainable parameters. The number of trainable parameters of AGHMN is about 0.4M in both the IEMOCAP and MELD datasets. 
In terms of average training time per epoch, our model (34s per epoch) costs slightly more running time than AGHMN (19s per epoch) in the IEMOCAP dataset.
Moreover, we find that in the MELD dataset, the AGHMN model (93s per epoch for BiF\_AGRU\_CNN and 180s per epoch for UniF\_BiAGRU\_CNN) requires much more running time than our BiERU-lc (65s per epoch) and BiERU-gc (77s per epoch) model.} 


\begin{figure}[ht!]
\centering
\begin{subfigure}{0.35\textwidth}
    \includegraphics[width=\textwidth]{loss_latest.pdf}
    \caption{Training curve}
    \label{fig:loss}
\end{subfigure}{}
\begin{subfigure}{0.35\textwidth}
    \includegraphics[width=\textwidth]{time.pdf}
    \caption{Time consumption}
    \label{fig:time}
\end{subfigure}
\linespread{1}
\caption{Training curve and time consumption logged on a single GPU using the IEMOCAP dataset.}
\end{figure}{}

\subsection{Ablation~Study}
To further explore our proposed BiERU model, we perform ablation study on its two main components, i.e., GNTB and TFE. We conduct experiments on the IEMOCAP dataset with individual GNTB and TFE modules separately, and their combination, i.e., the complete BiERU. Experimental results on the IEMOCAP dataset are illustrated in Table~\ref{tab:ablation}.

The performance of sole GNTB or TFE is low in terms of accuracy and f1-score. The reason is that outputs of GNTB mainly contain context information and outputs of TFE lack context information. However, when these two modules are combined together as the BiERU model, the accuracy and f1-score increase dramatically, which proves the effectiveness of our BiERU model. More importantly, the GNTB and TFE modules couple significantly well to enhance the performance.
\begin{table}[!h]
\small
    \centering
    \begin{tabular}{|c|c|c|c|}
    \toprule
    GNTB & TFE & ACCURACY & F1-SCORE \\
    \midrule
     - & + & 55.45 &  55.17 \\
     + & - & 49.85 & 49.42\\
     + & + & 65.93 & 64.63 \\
    \bottomrule
    \end{tabular}
    \linespread{1}
    \caption{Results of ablated BiERU on the IEMOCAP dataset. Accuracy and F1-score are weighted average.}
    \label{tab:ablation}
\end{table}{}


\section{Conclusion}
\label{sec:conclusion}
In this paper, we proposed a fast, compact and parameter-efficient party-ignorant framework bidirectional emotional recurrent unit (BiERU) for sentiment analysis in conversations. Our proposed generalized neural tensor block (GNTB), skilled at context compositionality, reduced the number of parameters and was suitable for different structures. Additionally, our TFE is capable of extracting high-quality emotion features for sentiment analysis. We proved that it is feasible to both simplify the model structure and improve performance simultaneously.

Our model outperforms current state-of-the-art models on three standard datasets in most cases. In addition, our method has the ability to model conversations with arbitrary turns and speakers, which we plan to study further in the future. Finally, we also plan to adopt more recent emotion categorization models, e.g., the Hourglass of Emotions, to better distinguish between similar yet different emotions.


\section*{Acknowledgements}
This research is supported by the Agency for Science, Technology and Research (A*STAR) under its AME Programmatic Funding Scheme (Project \#A18A2b0046).

\bibliographystyle{IEEEtran}
\bibliography{ref-dial-emotion}

\end{document}
