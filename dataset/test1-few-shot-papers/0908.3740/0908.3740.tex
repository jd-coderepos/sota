\documentclass[11pt]{article}
\usepackage{geometry}
\usepackage{fullpage}
\textheight 8.9in 
\textwidth 6.5in
\usepackage{times}
\usepackage{graphicx}
\usepackage{amsmath}

\usepackage{amssymb}
\usepackage{amsthm}
\usepackage{epstopdf}
\DeclareGraphicsRule{.tif}{png}{.png}{`convert #1 `dirname #1`/`basename #1 .tif`.png}

\newtheorem{thm}{Theorem}[section]
\newtheorem{lem}[thm]{Lemma}
\newtheorem{cor}[thm]{Corollary}
\newtheorem{claim}[thm]{Claim}
\newtheorem{defn}[thm]{Definition}
\newtheorem{rem}[thm]{Remark}

\newcommand{\E}{\textrm{\textbf{E}}}
\newcommand{\mcA}{\mathcal{A}}
\newcommand{\valpha}{\vec{\alpha}}


\usepackage{hyperref}
\hypersetup{
bookmarksnumbered
}


\setcounter{page}{0}

\begin{document}
\title{An Oblivious -Approximation for Single Source
Buy-at-Bulk}
\author{Ashish Goel
\thanks{Departments of Management Science and Engineering, and by courtesy,
Computer Science, Stanford University. Email: ashishg@stanford.edu.
Research supported by an NSF ITR grant and the Stanford-KAUST alliance for academic excellence.}
\\Stanford University
\and Ian Post
\thanks{Department of Computer Science, Stanford University. Email: itp@stanford.edu.
Research supported by an NSF ITR grant and the Stanford-KAUST alliance for academic excellence.}
\\Stanford University}

\maketitle
\thispagestyle{empty}
\pdfbookmark[1]{Abstract}{MyAbstract}
\begin{abstract}
We consider the single-source (or single-sink) buy-at-bulk problem with an
unknown concave cost function.  We want to route a set of demands along a
graph to or from a designated root node, and the cost of routing  units
of flow along an edge is proportional to some concave, non-decreasing
function  such that .  We present a polynomial time algorithm
that finds a distribution over trees such that the expected cost of a tree
for any  is within an -factor of the optimum cost for that . The
previous best simultaneous approximation for this problem, even ignoring
computation time, was , where  is the
multi-set of demand nodes.

We design a simple algorithmic framework using the ellipsoid method that
finds an -approximation if one exists, and then construct a separation
oracle using a novel adaptation of the Guha, Meyerson, and Munagala
\cite{guha2001cfa} algorithm for the single-sink buy-at-bulk problem that
proves an  approximation is possible for all . The number of trees
in the support of the distribution constructed by our algorithm is at most
.
\end{abstract}
\newpage

\section{Introduction}

We study the single-source (or single-sink) buy-at-bulk network design problem with an unknown concave cost function. 
We are given
an undirected graph  with edge lengths  and a set of
demand nodes  with integer demands  and 
want to route these demands to a
designated root node  as cheaply as possible, where the cost of routing
along a particular edge is proportional to some function  of the amount of
flow sent along the edge. 
In many applications it is natural to assume that
 is a concave, non-decreasing function such that , capturing the
case where we benefit from some kind of economy of scale when aggregating
flows together.  We call such functions \emph{aggregation functions} and
define  as the set of all aggregation functions.

When the function  is given, the problem becomes the well-studied
single-sink buy-at-bulk (SSBaB) problem.  SSBaB is -hard, since it
contains the Steiner tree problem as a special case.  The problem was
introduced by Salman et al.~\cite{salman1997bbn} who gave algorithms for
special cases.  Awerbuch and Azar \cite{awerbuch1997bbn} gave an -approximation using metric tree embedding, which subsequently improved to
 using better metric embeddings~\cite{bartal1998aam,
fakcharoenphol2003tba}.  Building on their own work on hierarchical facility
location~\cite{guha2000hpa}, Guha, Meyerson, and Munagala (GMM) gave the first
constant-factor approximation \cite{guha2001cfa}, an algorithm that features
prominently in our results.  Recent work \cite{talwar2002ssb, gupta2003sab,
jothi2004iaa, grandoni2006ias} has reduced the approximation ratio to 24.92
and also provided an elegant cost-sharing framework for thinking about this
problem.

However, for some applications we may want to assume that  is unknown or is
known to vary over time. For instance, we may be aggregating observations in a
sensor network where we do not know the amount of redundancy among different
observations or where the redundancy is known to change. In this setting, it
is desirable to find a solution that is robust to changes in  and provides
a constant-factor approximation {\em simultaneously} for all .  Moreover, from a purely theoretical perspective, the existence
of a good algorithm that is independent of  reveals non-trivial structure
in the problem.

We will focus on randomized algorithms.
Given the concavity of , we may assume without loss of
generality that the optimal routing graph is a tree.  Let  be the
set of all trees in  spanning  and , and let  be the
optimal tree for some fixed .  We use the shorthand  to denote the
cost of  under , i.e.\  where  is the
amount of flow tree  routes on edge .  
There are two natural objectives which capture simultaneous approximation for multiple cost functions.  
First, we can try to minimize

which essentially gives a distribution over trees such that in expectation,
each function  is well-approximated.  Second, and much more difficult, we
can look for an algorithm that uses the objective

A bound on \eqref{exp_max} subsumes \eqref{max_exp} and proves there exists a
single tree that is simultaneously good for all . 
We call  the {\em oblivious} approximation ratio and  the {\em simultaneous} approximation ratio.
In this paper, we will
work with the weaker, oblivious objective \eqref{max_exp}.

Both objectives have been studied in the literature.  The tree embeddings used by Awerbuch and Azar
\cite{awerbuch1997bbn} give an  oblivious approximation,
which was later reduced to  \cite{bartal1998aam,
fakcharoenphol2003tba}.  Goel and Estrin \cite{goel2003soc} improved this to
 and also prove the same bound on the stronger
simultaneous objective.  Gupta et al.\ \cite{gupta2006ond} achieve a
 oblivious approximation for a generalization where
both the function and the demands are unknown.  Khuller et al.\
\cite{khuller1995bms} studied special case of simultaneously approximating
 and  for , i.e.\ the shortest-path and Steiner
trees, and prove an  simultaneous approximation.  These 2 functions
constitute opposite extremes of functions in , and one may wonder
if an  approximation for these 2 functions also works for all  lying ``in-between''.  However, it is not difficult to construct
a graph and a set of demands such that the shortest-path and Steiner trees are
identical, but this tree is an -approximation for other .  Enachescu et al.\ \cite{enachescu2005sfa} achieve an 
simultaneous value but only for grid graphs, assuming spatial
correlation among nearby nodes. This naturally leads to the following questions:

\begin{quote}
Is  achievable? If yes, is there a polynomial algorithm that
guarantees ?
\end{quote}

We answer both questions in the affirmative. We first write a simple LP
formulation of the problem and show that using the ellipsoid method on the
dual we can find an  approximation to the optimal ratio, whatever it
happens to be for a given problem instance. We also show that given an
appropriate separation oracle the optimum is constant and compute an explicit
distribution over  trees in polynomial time. This general
approach is along the lines of small metric tree embeddings~\cite{ccggp:tree98} and
oblivious congestion minimization~\cite{h:congestion08}.

Our key result is the construction of the necessary separation oracle
subroutine, running in polynomial time, that proves a constant is achievable.
We build our oracle around the GMM algorithm for SSBaB, using a modified
analysis to solve a different problem in which we bound the cost of the GMM
tree by a combination of different trees under different cost functions. 
\subsection{Organization of the Paper}

In Section \ref{algorithm_section} we present an LP formulation and a
framework using an approximate separation oracle that finds a
constant-factor approximation to the optimal oblivious approximation ratio.
In Section \ref{oracle_section} we present our primary result, which proves
the oblivious approximation ratio is constant and constructs the separation
oracle required by Section \ref{algorithm_section} assuming some extra
conditions on the input, and in Section \ref{regularization_section} we
complete the proof by showing those extra assumptions can be removed. We
conclude with some open problems (including whether  can be
achieved).

\section{LP Formulation and Algorithm Framework}
\label{algorithm_section}

Let  be the worst-case optimal oblivious ratio, i.e.\

where  is a distribution over .  In this section we discuss the problem of finding an -oblivious approximation if one exists.  

By losing a factor of  in the approximation ratio we can restrict our analysis to a smaller class of aggregation functions.
Let , the total amount of demand rounded up to the nearest power of 2.  We never route more than  flow on any edge, and  is integral, so we only care about  for integers .  Suppose , and .  By the monotonicity of , , and by the concavity of , , so with a loss of a factor of 2 we can interpolate between  and  and assume  is piecewise linear with breakpoints only at powers of 2.  Let  and  the optimal aggregation tree for .  We call  the -th atomic function following the terminology of Goel and Estrin \cite{goel2003soc}, and it is easy to see that any  that is linear between successive powers of 2 can be written as a linear combination of .  Therefore, it suffices to design an algorithm  minimizing .

Our algorithm makes use of the standard SSBaB problem where  is known.  We assume that  is given in the form of a set of  pipes , where the cost of routing  flow on pipe  is equal to .  Then  is defined as the cost of using the cheapest pipe for  flow: .  We assume that , and by concavity we can assume .  Define , the point at which the cost due to  begins to outweigh the cost due to .   We call  the \emph{capacity} of pipe ; the name arises from an alternate formulation
(equivalent up to a factor of 2)
of SSBaB where pipes have a fixed cost  for a fixed capacity .  Let  be the best-known approximation ratio for SSBaB.  Currently  using an algorithm by Grandoni and Italiano~\cite{grandoni2006ias}.

We also employ an approximation algorithm for a special case of SSBaB, the single-sink rent-or-buy (SSRoB) problem.  Here  is characterized by 2 pipes:  and , i.e.\ we can pay  to route  flow or pay  to route any amount of flow.  Let  be the best-known SSRoB approximation ratio.  Eisenbrand et al.~\cite{eisenbrand2008acf} give a -approximation.

If we can calculate  and  for every  and  then the following linear program finds the optimal distribution of trees.

In other words, we want a distribution  of trees minimizing .   However, this approach is not directly tractable, as  is -hard to find, and  is exponentially large.

We solve an SSRoB approximation for each  to get ---a -approximation---and replace  with  in the constraints, so that all quantities in the LP are polynomial-time computable.  
Now consider the dual of \eqref{primalLP}, which is given by

With an approximate separation oracle for the dual \eqref{dualLP}, we can approximate the solution in polynomial time using the ellipsoid method, and then transform it into an approximate solution to the primal \eqref{primalLP}.  More formally:

\begin{thm} \label{first_alg_thm} With a randomized -approximation to SSBaB, we can find a -approximation in expectation to the primal LP \eqref{primalLP} that runs in polynomial time with high-probability.
\end{thm}

The proof uses a SSBaB approximation algorithm to construct an approximate separation oracle for \eqref{dualLP}.  However, we will not prove this theorem because it is a special case of the following more general result, assuming that  is a constant which will follow from Theorem \ref{gmm_thm}.

\begin{thm}  \label{alg_thm}
If there exists a polynomial-time algorithm  and a given constant  such that ,  finds  such that  then we can construct an algorithm that runs in polynomial-time with high probability, makes  calls to  with high probability, and achieves an expected oblivious approximation ratio of  using a distribution over  trees.
\end{thm}

Proving that such an algorithm  exists for a constant  is the primary result of this paper and is discussed in sections \ref{oracle_section} and \ref{regularization_section}.

\begin{rem} If  is deterministic then the algorithm always runs in polynomial time and the expected ratio is , and if it is randomized then the algorithm runs in polynomial time with high probability and the expected ratio is .  For randomized  the ratio can also be reduced to  with a -factor increase in the runtime.
\end{rem}

\begin{proof}[Proof of Theorem \ref{alg_thm}]

Let  be a -approximation to  as above.  We construct an approximate separation oracle  for the dual \eqref{dualLP} as follows:

\begin{enumerate}

\item Check if .  If so, we have a violated constraint and are done.

\item \label{find_tree_step} Run  until it returns a tree  such that .

\item If , return .  Otherwise, return feasible.

\end{enumerate}

For a fixed , let  be the polytope defined by , and  for all .  We run the following procedure to find the desired distribution of trees:

\begin{enumerate}

\item Run the ellipsoid method to check the feasibility of , starting with the initial bounding box   and using  as the separation oracle.  It will terminate as infeasible.

\item Let  be the set of constraints returned by  proving  is infeasible.  It consists of , and  for  in some subset of trees .

\item In the dual LP (2), restrict the constraints to , and take the dual to get


\item Find a vertex optimal solution to \eqref{small_primal}, and return the distribution .
\end{enumerate}

First, we claim that  will find a violated constraint whenever  and will do so in polynomial time with high probability.  If   is violated, then we are done.  If not, we know  finds  such that 

By Markov's inequality , so with high probability  invocations of ---each running in polynomial time---suffice in step \ref{find_tree_step} of  to find a  satisfying .  Now if , the constraint  is violated.

With the necessary separation oracle, the ellipsoid algorithm can solve feasibility of  in  iterations, so using  it will conclude  is infeasible\footnote{In practice  may find violated constraints for , and we can do binary search to find the smallest infeasible .  However, we cannot improve the provable guarantee beyond , and this comes at a cost to the runtime.}.  
The set of constraints  returned by  during the execution constitutes a proof of infeasibility,
and  consists of , and  for each  in some set of trees .  

Consider writing \eqref{dualLP} with only the constraints in .  Taking the dual yields \eqref{small_primal}, which only has variables  for .  The ellipsoid algorithm concluded  is infeasible after  iterations, so  is only polynomially-large in the input size, implying we can solve \eqref{small_primal} exactly in polynomial time.

Find a vertex-optimal solution  to \eqref{small_primal}.  The constraints in  are enough to restrict the optimal dual objective to be at most , so by duality .  Therefore, for all 

Divide by  to get the oblivious ratio:
 

Moreover, we claim  is a distribution over only  trees.  The LP \eqref{small_primal} has  variables and  constraints, and the vertex-optimal solution  must have  tight constraints, implying at least  non-negativity constraints must be tight.  We know  is positive, so only at most  of the variables  can be non-zero.
\end{proof}

\section{The Separation Oracle Subroutine }
\label{oracle_section}

By Theorem \ref{first_alg_thm} we can find an -approximation to , whatever it may be, but it remains to prove that this optimal ratio is a constant.  In this section we construct the procedure  required by Theorem \ref{alg_thm} using the GMM algorithm for SSBaB.

Our contribution is adapting a special case of the analysis of the GMM algorithm, namely those cases that arise when , to solve a different problem--that of bounding the cost of the output by  rather than .  The GMM algorithm and proof works in stages and bounds the cost of the pipes laid in each stage by a different chunk of the optimal tree .  On the other hand, in our proof we bound the cost of each stage by the cost of a \emph{different} tree evaluated under a \emph{different} cost function.

\subsection{Background: The GMM Algorithm}

For completeness, we summarize the GMM algorithm and the key lemmas and definitions.  See the original paper \cite{guha2001cfa} for a thorough treatment.  We are given a graph, demands , and pipes  as described in Section \ref{algorithm_section}.  
We assume the costs of successive pipes differ ``significantly'': for some constant  such that , we have that  and .  For the SSBaB problem, it is easy to satisfy these constraints for arbitrary pipes with only an -factor loss.  For our problem, it is harder but still possible, and this is discussed in Section \ref{regularization_section}.  

We define  as the indifference point between pipe  and , which is the solution to the equation , and we define  as the solution to , which we interpret as the point at which pipe  becomes ``significantly'' cheaper than pipe .  It is easy to see that  for all .

The algorithm uses -approximations for Steiner tree and load-balanced facility location (LBFL), a generalization of the standard facility location problem.  In the LBFL problem we have a graph and demands as in SSBaB, a facility cost  for each node , and a lower bound  on the demand that a facility at  must service.  The objective is to choose facilities and routing paths so as to minimize the sum of the cost of the open facilities and the distances traveled by the demands to a servicing facility.  To approximate the LBFL we must relax the lower bound.  Using \cite{guha2000hpa} we can approximate the optimal LBFL cost to within  while reducing the lower bound by a factor of at most 3.  Here  denotes the best approximation to the normal facility location problem, currently  by Mahdian et al.\ \cite{mahdian2002iaa}.  We use  to denote the best approximation ratio for Steiner tree, currently  due to Robins and Zelikovsky \cite{robins2000ist}.

Now we can describe the GMM algorithm itself.  At stage , we lay pipe type , and we break each stage into a Steiner tree step and a ``shortest-path'' tree step based on whether the cost of pipe  is dominated by the term  or the term .  The effective demands will also change each stage.  Let  be the demand nodes at the start of stage , and  the stage  demand at .  Initially .

\begin{description}

\item{1. \emph{Steiner Tree: }}
Find a -approximate Steiner tree on  with edge cost per unit length .  Route all demands toward .  Cut the farthest-upstream edge with more than  flow, recalculate the flow, and repeat to get a forest with at least  flow at each root other than  and at most  flow on each edge.

\item{2. \emph{Consolidation:}}
Let  be a subtree not containing  and  the demand nodes in  it contains.  Choose  with probability  and route all demand in  back to  using pipe .

\item{3. \emph{Shortest Path Tree:}}
Approximately solve a LBFL problem with facility lower bound  and edge cost per unit length  on the \emph{original} demands  (not  and ).  This creates a forest of shortest-path trees with at least  flow at each root.  If  demand does not exist, route everything to .

\item{4. \emph{Consolidation:}}
Let  be subtree in the above forest servicing the demands  in .  Choose  with probability , and route the true, current demand  in  back to .  Let  be the set of nodes chosen for consolidation and  the demand at these nodes after consolidation.
\end{description}

Next, we mention the crucial lemmas in the GMM analysis used in our proof.  See \cite{guha2001cfa} for the proofs.

\begin{lem}[GMM Lemma 4.1] \label{consolidation_lem} 
Let  be the current demand at some  immediately after any consolidation step.  Then , i.e.\ the original demand.
\end{lem}

Using an algorithm that is a 3-approximation to the LBFL facility lower bounds, we have the following:

\begin{lem}[GMM Lemma 4.5] \label{current_demand_lem}
For every , we have .
\end{lem}

Define  to be the \emph{incremental} cost (due to ) of the pipes laid in the \emph{facility location} step in stage  and  to be the \emph{fixed} cost (due to ) of the pipes laid in the \emph{Steiner tree} step in stage .  All of the other costs incurred by the GMM algorithm can be bounded by  and , so our analysis need only consider these quantities:

\begin{lem}[GMM Lemmas 4.2, 4.4, and 4.8] \label{gmm_cost_lem}  Let  and  as defined above.
Then , where  is the final tree.
\end{lem}

\subsection{Adapting the GMM Algorithm}

From Theorem \ref{alg_thm} we are given  such that , and .  We want to find a tree  using the GMM algorithm such that . 
Define , the \emph{multi-level cost}, and , the concave cost function.  Using this notation our objective becomes to find  such that .  Define  as the number of non-zero , and for  define   where  is the index of the -th non-zero . 

First, we claim that given  we can define the pipes  used by the GMM algorithm, and given SSBaB pipes satisfying some minor conditions we can recover .  The following lemmas characterize the equivalence between the 2 types of parameters:

\begin{lem} \label{alphatodelta_lem}
Given  satisfying  with  non-zero , the SSBaB pipes  defined by  and  define the function .  That is, .
\end{lem}

\begin{lem} \label{deltatoalpha_lem}
Suppose we are given  SSBaB pipes  such that  and  is a power of 2 for all . For , let , , and  whenever  for all .  Then .
\end{lem}

\begin{proof}[Proof of Lemma \ref{alphatodelta_lem}]
By definition .  For any ,  is linear from  to  (we will assume  for consistency of notation), which will correspond to pipe .  For , the functions  have leveled off, and  are growing at rate 1.  Define  as the slope of  in this interval: .

Now we can define  to match  in the interval :

We also add a st pipe such that  and  to cover the interval after every  has leveled off.
Now, we claim :  for each  we know  whenever  by our choice of  and , and by the concavity of  for each  we have  when  or .  Therefore no other pipe can be cheaper in this interval.
Concavity also ensures that  and  for all , yielding valid SSBaB pipes.
\end{proof}

\begin{proof}[Proof of Lemma \ref{deltatoalpha_lem}]
Let  be the number of pipes, and , .  Since we never route more than  flow we may assume the cost function levels off at some , so that .  Define  for : when we change pipes at  the slope of  drops, which can occur only because the term  levels off.  Recover  by reversing the definitions in the proof of Lemma \ref{alphatodelta_lem}:  we have , so for  let .

We now show by induction that .  For the base case , we have

Now assume that for  that .  For , we know that .  Therefore,

We use that pipes  and  have equal cost at  in the first line and the induction hypothesis in the second line.
\end{proof}

We note that  corresponds not to a particular SSBaB pipe, but to a breakpoint between pipes:  when we switch from pipe  to  at  flow, the slope of  drops from  to , which is caused by the term  leveling off.

Given the above equivalence, we will use  and  interchangeably for the remainder of the paper, using whichever representation is more convenient and converting from one form to another using Lemmas \ref{alphatodelta_lem} and \ref{deltatoalpha_lem}.  However, the additional constraints that for some parameter  we have  and  for all pipes , will restrict the possible vectors  that can be run through the algorithm:

\begin{defn} Call  \emph{-regular} if the pipes found using Lemma \ref{alphatodelta_lem} satisfy  and . 
\end{defn}

We note the following constraints that -regularity imposes on :

\begin{lem} \label{alpha_delta_lem}
If , then  and .
\end{lem}

\begin{proof} 
These follow immediately from  and .
\end{proof}

\subsection{Approximation guarantee assuming regular }

We will first prove the existence of the separation oracle procedure  in Theorem \ref{alg_thm} for -regular  and later prove in Section \ref{regularization_section} that arbitrary  can be regularized with only an  change in  and :

\begin{thm} \label{gmm_thm}
Let  be -regular, and let , and .  Then the GMM algorithm finds a tree  such that .
\end{thm}

Roughly, our proof bounds the cost of the pipes laid in phase  of the algorithm by .  Using Lemma \ref{gmm_cost_lem} we concentrate on  and  and ignore the other costs.  
First, we bound the cost of the Steiner tree steps:

\begin{lem} 
Let  be the approximation ratio for Steiner tree.  Then we have .\label{steiner_cost_lem}
\end{lem}
\begin{proof}
We need to bound the cost of a Steiner tree spanning the current demands  with cost per unit length .  If , then  and we have nothing to bound, so assume .

We use the edges in .  Note that it spans  and hence , and let  be the subset of edges spanning these nodes.  
By Lemma \ref{current_demand_lem} each  has aggregated at least  demand.  At the end of the previous LBFL phase, we chose a node  for consolidation from the set of all  routing to facility  with probability .  An edge is in  only if some  routes through it, so by the union bound an edge carrying  demand in  is in  with probability at most .

The tree  pays  for any amount of flow, whereas  pays  to send  flow on .  Then the cost of  is


We need to bound  and .  For the former term, 

using that  by definition, the -regularity constraints on , and the fact that .
For the latter term,

using the formula for  in Lemma \ref{alphatodelta_lem} and -regularity.

Plug these into the final line in equation \eqref{steiner_cost_eq} above:


We lose another factor of  in approximating the Steiner tree.  Sum over all  to bound  by .
\end{proof}

Analyzing the LBFL step requires an additional lemma bounding the difference between  and :

\begin{lem}  For every , .
\label{b_g_lem}
\end{lem}

\begin{proof}
The bound  follows from Lemma 3.5 in GMM \cite{guha2001cfa}.  For the other inequality, from the definition of  and  we have

For the ratio of  terms,

Similarly, for the s,

Combining the 2 bounds,

\end{proof}

Now we can bound the LBFL cost :

\begin{lem} We have that  where  is the approximation ratio for the standard (non-load-balanced) facility location problem.
\label{facility_cost_lem}
\end{lem}

\begin{proof}
In the shortest path tree step, the GMM algorithm solves an LBFL problem on the original demands  with facility lower bound  and edge cost per unit length .  
We will construct a feasible solution using the edges of .
Orient the edges towards , and find the farthest upstream (i.e.\ away from ) edge routing at least   flow.  Cut the edge, and place a facility at the upstream node.  Subtract this flow from downstream edges, and repeat the procedure.  
If we finish with less than  flow at the root node, we route each demand still reaching the root from its source vertex along the tree to the nearest existing facility (according to distances in ).  Let  be the resulting forest, and note that it has at least  flow at each facility.

For an edge  let  be the amount  routes on  when the demands  are routed, and  the amount that  routes on .  We now show that .  If we finish cutting  with at least  at the root then all flows are a subset of the flows in  so .  If we end up with too little demand for a facility in the final step then some of those demands will not be flowing downstream towards  in .  For each edge they take towards , they are following the routing in , so .  For each  edge taken away from , we are no longer following , but we must be moving upstream towards the nearest facility.  This implies that in the tree  edge  carried more than  flow because all demand at the upstream facility flowed through  towards .  Since we are sending strictly less than  demand upstream we still have .

The forest  never routes more than  flow, so .  
When , , so .  
Since  levels off at , this may not hold for  , but by Lemma \ref{b_g_lem} .
Therefore  when .

Now let  be the flow  routes on edge  when the current, stage  demands  are used.  By Lemma \ref{consolidation_lem},  for each .  Summing over all the demands that contribute to an edge's flow, we have .

The cost of  with  cost per unit edge length is

using  and  from Lemma \ref{alpha_delta_lem}.

We can find an approximate LBFL solution that is a -approximation to the optimal cost and reduces the facility lower bound by a factor of at most .  Therefore


Sum over all values of  to bound the expected cost by .
\end{proof}


\begin{proof}[Proof of Theorem \ref{gmm_thm}]
Combining the bounds in Lemmas \ref{gmm_cost_lem}, \ref{facility_cost_lem}, and \ref{steiner_cost_lem}:

\end{proof}

This completes the analysis of  for -regular .  If arbitrary  can be -regularized for some  it follows that .

Recent algorithms for SSBaB are based on the 
Gupta, Kumar, and Roughgarden (GKR) algorithm \cite{gupta2003sab, gupta2007approximation}, which achieves a better approximation ratio than GMM with a simpler analysis, and one may wonder whether we could reap the same benefits by basing our proof around this algorithm instead.  One round of GKR is roughly equivalent to one round of GMM---starting with about  demand at a subset of nodes and ending with about  demand at a smaller subset---but the GKR analysis bounds the entire cost of a round using only one tree, whereas GMM requires two.  However, each tree required by GMM can be easily constructed from some  in , but building the tree needed by GKR and within the right bounds seems trickier.  Note that Lemmas \ref{steiner_cost_lem} and \ref{facility_cost_lem}
use two different trees,  and , analyzed in two different ways, either fixed or linear cost per edge.  Although this conveniently matches the GMM algorithm, it also required for the proof to work.  
Using only a single Steiner tree on a subset of the nodes as in GKR allows less flexibility,
so a proof may require a different approach or more substantial changes to the original GKR analysis.

\section {Handling Arbitrary }
\label{regularization_section}

Given any , where , defining , a concave cost function, and , the multi-level cost, we need to find regular  defining  and  such that  , and .  Then applying Theorem \ref{gmm_thm} to  gives , and 

satisfying the precondition of Theorem \ref{gmm_thm}.
Note that we can allow  to grow and  to shrink arbitrarily in the transformation to  and , but we need to bound increases in  and decreases in .  By scaling by  we may assume without loss of generality that .

First, we prove a simple bound on the change between each term  in .

\begin{lem}  For any  and any , .
\label{L_term_lem}
\end{lem}

\begin{proof}
Note  for .  Therefore 

\end{proof}

To regularize the values we run  through a series of three procedures, one for each of the following lemmas, 
each of which changes  to satisfy an additional set of constraints.  None of the procedures are conceptually difficult, but the details are quite intricate.  We will state the lemmas, give a brief sketch of the ideas, and present the complete proofs in the appendix.

The first lemma is only a helper used in satisfying the  constraints.  The proof serves as a warmup for the later lemmas, which use similar ideas but are more involved.

\begin{lem} 
\label{max_capacity_lem}
Given arbitrary , we can find  such that the corresponding ,, ,  satisfy , , and , where  is the number of pipes, and  is the total demand rounded up to a power of 2.
\end{lem}

The following 2 lemmas perform the actual regularization.

\begin{lem} 
\label{delta_lem}
Given  satisfying , we can find  
such that the corresponding ,, ,  satisfy
, , , and  for all .
\end{lem}

\begin{lem} 
\label{sigmalem}
Given  satisfying  and ,
we can find  such that such the corresponding ,, ,  satisfy
, , , and  for all .
\end{lem}

The proofs are based around the following idea: check if  or , and discard pipes that violate the constraints.  The additional difficulty, relative to the analysis of GMM, arises from the special form that  must satisfy and the need to bound the increase in .  When we remove pipes in general the indifference points between subsequent pipes will no longer be powers of 2, so  can no longer be defined in terms of .  We fix this by modifying the parameters of an offending pipe until the new breakpoint is a power of .  To avoid drastic changes in  or , we achieve this by holding the cost of the given pipe  fixed at its indifference point with either  of  and ``rotating'' the line  around this fixed point until the other indifference point is fixed.

Analyzing the increase in  caused by these procedures is the technical crux in the regularization analysis, as removing pipes can shift ``-mass'' in the multi-level cost onto much more expensive trees.  
We consider each pipe removal and the terms in  it affects.  If -mass is shifted from  to , where , then the current chunk of  has increased by .  If not, we show that the conditions requiring  imply there exist large terms in  above  that can absorb the increase with only an -factor loss.
We only charge against each -term  times during the entire regularization, so the total increase is bounded by .

We summarize the consequences of the regularization procedure below:

\begin{thm} The algorithm  required by Theorem \ref{alg_thm} exists for a constant , and the oblivious approximation ratio  is constant.
\end{thm}

\section{Open Problems}

A number of interesting open problems remain to be solved.  First, we have only achieved an -ratio for the objective , but Goel and Estrin \cite{goel2003soc} have shown an -approximation for the much harder objective , proving there exists a single tree that is \emph{simultaneously} an -approximation for all .  Achieving a constant for this stronger objective or showing a lower bound remains an important open question.  

Second, although our algorithm proves that an -approximate distribution exists, the ellipsoid algorithm tells us little about what these trees actually look like.  A combinatorial algorithm that yields insight as to the actual structure of these trees would also be of interest.  Third, we have made little attempt to optimize the constant  in the approximation ratio, and the resulting value is huge due to the regularization procedure.
Shaving large factors off our bound on  may be a simple question, and it would be particularly interesting to find an oblivious approximation algorithm that is competitive with standard SSBaB for known .

\pdfbookmark[1]{\refname}{My\refname}
\bibliographystyle{alpha}
\bibliography{ssbab}

\appendix

\section{Proofs of regularization lemmas}
\label{regularization_appendix}

\newtheorem*{thm_max_capacity}{Lemma \ref{max_capacity_lem}}

\begin{thm_max_capacity} 
Given arbitrary , we can find  such that the corresponding ,, ,  satisfy , , and , where  is the number of pipes, and  is the total demand rounded up to a power of 2.
\end{thm_max_capacity}

\begin{proof}
Let  be the first pipe such that .  Note  since .  Remove all pipes above .  
Now we modify the parameters of pipe  to satisfy the desired constraint.
Increase , while decreasing  so as to hold  fixed, until .  Geometrically, we are rotating the line  counter-clockwise around the point .  Let ,  be the new parameters for pipe .  Let  be the new cost function formed by modifying pipe  and removing pipes  and  the associated multi-level cost.

\begin{description}
\item{\emph{Claim:}}
The function  is concave, and  for all .

Initially  and , and we continuously decrease  while increasing .  We know , so if we decrease  to  the modified pipe  will match pipe .  However, we have that , so we stop before reaching that point.  Therefore  and , which implies  is concave since the switchover between pipes  and  is unchanged.
We only increased the rate of growth for , so  for all .


\item{\emph{Claim:}} The new multi-level cost  is at most .

There is a term  for each changeover between pipes as well as the implicit breakpoint at  when  levels off.
Increasing  and removing pipes  so that pipe  is used all the way to  corresponds in  to pushing -mass from the terms  onto the term  because .  

By the definition of  and  and Lemma \ref{alphatodelta_lem} we have 

The terms  are unchanged, and  drops due the decreased difference between  and .  There are no non-zero  between  and .  This gives us 


Next we use Lemma \ref{L_term_lem} to relate  and :

Finally, .
\end{description}
\end{proof}

\newtheorem*{thm_delta}{Lemma \ref{delta_lem}}

\begin{thm_delta} 
Given  satisfying , we can find  
such that the corresponding ,, ,  satisfy
, , , and  for all .
\end{thm_delta}

\begin{proof}
We repeat the following two steps until  for all .
\begin{description}
\item{1. \emph{Deletion Step:}} 
The basic idea here is the same as that used by GMM Lemma 3.2 \cite{guha2001cfa} to satisfy the constraints on the 's: whenever a pipe violates the constraint , we remove the pipe.

Let  be the smallest index such that , and let  be the smallest 
integer such that .  If such an  exists, then remove 
pipes , and change  in the interval  by using the cheaper of pipe  and 
.  If no such  exists then remove all pipes above , and replace them with pipe .
Note that this does not break the condition set in Lemma \ref{max_capacity_lem}.

\item{2. \emph{Rotation Step:}}
Pipes  and  now have equal cost at some point , but  may not be a power of , in which case  is no longer in the form , and  is no longer defined.  

We want to modify the pipes to change  while not affecting  or  too much.
As in Lemma \ref{max_capacity_lem}, we hold the cost of pipe  fixed when routing  flow (where we switch from  to ),
and reduce  until pipes  and  meet at the next power of 2, increasing  to maintain 's cost at .
This corresponds to rotating the line  clockwise around the point .  Let  and  be the new parameters for pipe .  
Note that  now has the proper structure again, and  and  are well-defined. We never increase  above  since we hold this point fixed when adjusting pipe .
\end{description}


\begin{figure}[htbp] 
 \centering
 \includegraphics[width=4in]{rotation} 
 \caption{To ensure the indifference point between pipes  and  is a power of 2 we ``rotate'' pipe  around it's starting point until it meets  at a power of 2.   }
 \label{rotation_fig}
\end{figure}

First, we bound the change to  in the rotation step.
This allows us to prove that the constraints on the 's are satisfied, and  decreases by at most an -factor.

\begin{description}

\item{\emph{Claim:}}
After rotation .

Before adjustment, we are indifferent between  and  at  
where . 
The difference in costs between  and  at  flow remains unchanged  
because we hold the cost of pipe  fixed at .
Let , the distance after  at which their costs are equal.
Before rotation, the pipes' costs approach each other at a rate of .
If we reduce  by a factor of , then , 
so it takes at least  for pipe k to grow from  to , during which pipe 's cost only increases, so pipe  does not surpass  until after .

The original pipe  met pipe  (now removed) at some point  before meeting  at .
Therefore , which implies .
After reducing  to , pipes  and  now meet after .
There must be a power of  between  and , and we reduce  only until we hit the next power of , so .

\item{\emph{Claim:}} When the procedure is finished  for all .

By the choice of , , using the previous claim.  
Further , so no previously-satisfied constraints are broken.  We renumber the pipes, and repeat the process for the next constraint violation.  When we are done, all the remaining pipes will satisfy .

\item{\emph{Claim:}} For all , .

Note that removing pipes  only changes  in the interval , and we only remove or adjust pipes in this interval once.  Initially, removing pipes can only increase , but then we reduce  by a factor of at most 3, which may decrease  by a factor of at most 3.
\end{description}

Now, we must bound the potential increase in .  
To avoid confusion due to relabeling indexes after removing pipes, we change notation slightly.  Suppose the procedure completes after  iterations.  
Let  be the final non-zero 's, and  the original 's.  
For  let  be the -terms affected by the th iteration of the procedure: either they are removed and merged into  or  if the constraint is already satisfied.
We need to analyze how mass is shifted between terms in .  Define , the portion of  that round  affects.


Consider round  in which we remove old pipes  and adjust .  The old  becomes .  Rotating  increases  because  but reduces the total -mass above  because , decreasing .  
The remaining -mass on ,  merges into  where  is somewhere between  and .
If mass from some  moves down to  where , then we can ignore it, as it will only reduce .  If it moves up, then we will charge the increase to some higher term in .

Let  be some small constant.  There are  cases to consider: either  or .  
\begin{description}
\item{\emph{Case 1}}: .

Intuitively, this means there is a big drop between  and , 
so  must be fairly large: .  We will charge any increase in  this iteration to the term .
Note that we are always in this case when we remove the last pipe because we can view the last pipe as intersecting a dummy pipe with  at .

In order to bound  by 
we must show that .
Note  is the cost at which the new, rotated pipe  surpasses the old pipe .  
New pipe  intersects pipe  before , and ,
so pipes  and  meet before  and  do.  
Therefore ,
and when we reduce  to fix the breakpoint we never need to raise  beyond  before hitting a power of .  Therefore


We can charge the increase in  to  in the current chunk , with a loss of , and this charge can only occur once for each .

\item{\emph{Case 2}:} .

In this case there is no large collection of mass that we can easily guarantee is above  in the current interval, but we do know there must be a lot of mass somewhere above  because  is large.  
The -mass  is ``used'' in the next iteration and contributes  to .
We know , which implies .
Now we can bound the increase

Therefore we can charge the increase in  due to iteration  to the portion  used in the next iteration. 
\end{description}

For a particular segment  of , the th iteration may been bounded by  increase in , and the th iteration may charge against a  increase.  Each type of charge can occur at most once per chunk.  Therefore the total increase in each piece, and hence the total increase in  is

This completes the proof.
\end{proof}


\newtheorem*{thm_sigma}{Lemma \ref{sigmalem}}


\begin{thm_sigma} 
Given  satisfying  and ,
we can find  such that such the corresponding ,, ,  satisfy
, , , and  for all .
\end{thm_sigma}

\begin{proof}  
The proof follows Lemma \ref{delta_lem} but moves backwards through the pipes rather than forwards.
\begin{description}
\item{1. \emph{Deletion Step:}}
Let  be the highest index such that , and  the smallest integer such that .  Such an  must exist because .  
Remove pipes , and replace them with the cheaper of pipes  and .

\item{2. \emph{Rotation Step:}}
As in Lemma \ref{delta_lem},  may no longer be a linear combination of terms  because the new indifference point may not be a power of 2.  We use a similar procedure as before to remedy this.
Hold pipe 's cost for  flow fixed, and reduce  while increasing  to maintain the invariant until  and  meet at a power of 2.  Geometrically we are rotating  counter-clockwise around .  Let ,  be the new parameters. Note that  and  are now well-defined.
\end{description}

First, we analyze the change to  and  required by the rotation step and use this result to prove the constraints on both the 's and 's are satisfied at the end without changing  too much.

\begin{description}
\item{\emph{Claim:}}
After rotation , and .

Suppose the unmodified pipe  and  meet at .  We will bound the adjustment required to guarantee they meet before .  
Reduce  to .
The modified pipe  has the same cost as the old at .
If  is the final pipe then from Lemma \ref{max_capacity_lem} we know D = .  
Otherwise, pipe  costs the same as  at , so we have that , using  (the constraint fixed in the previous iteration).  In either case .
Now,


The constraints on the s were satisfied before removing pipe , so .  This implies 

using .
We combine this with the bound on  to bound the change in : 


Now we have enough information to bound the new switchover point.

There must be a power of 2 between  and , so we need to reduce  by at most a factor of .
Finally, note that pipes 0 and 1 meet no sooner than 1, and  since it is always true that .  Therefore , and hence the new changeover point is at least 1, so we do not need to worry about a term .

\item{\emph{Claim:}} When the procedure finishes  and  for all .

We chose  such that , so .  Before starting, we had , and , which implies .  Note that the rotation step does not break any previously-satisfied constraints on larger 's.

\item{\emph{Claim:}} For all , .

Only 1 round affects the interval .  Removing pipes only increases , and if we adjust , then it decreases by a factor of at most , while  increases, so .
\end{description}

Now we analyze the increase in .  
First, unlike in Lemma \ref{delta_lem}, the rotation step works against us, and we need to bound the increase.
\begin{description}
\item{\emph{Claim:}}
Rotation only increases  by an -factor.

When adjusting pipe , we increase  without changing , which increases .  We have that , and , so

causing  to increase by at most .
\end{description}

Second, we need to bound the increase in  caused by removing pipes.
Let  be the number of iterations and final pipes and  the resulting non-zero 's.  Iteration , for , deletes pipes  which removes .  
Let  be the amount these contribute to .
Since it moves backwards through pipes the indices of new pipes are not fixed yet, but as labeled at the end, round  ensures  and creates a term  where .
  
The rotation step reduces both  and  which can only help in this step,
and we have already bounded the increase in  due to rotation, so we assume that no rotation is needed.  
This implies .  
As in Lemma \ref{delta_lem} we need to ensure that too much -mass does not move too high.

Let  be a small constant.  We need to consider two cases again: either  or .

\begin{description}
\item{\emph{Case 1}:} .

Intuitively, this means  is much larger than  because , 
so by the time pipe  catches up with pipe  or any later pipe, it has already covered an -fraction of the distance to .  Therefore, pushing mass from up to  increases  by only a constant factor.

We bound  by bounding the cost to which pipe  must grow before switching pipes.
Before removal the old pipe  crossed the new  at , 
so .  Pipe 's cost increases faster than 's and surpasses 's cost before .  Therefore .  

We know  or else it would not have been removed.  When  intersects  at  it has grown from  to at least  and therefore has covered at least

fraction of the distance to the indifference point between  and .
Therefore

Every other affected  is pushed up less than , so


\item{\emph{Case 2}:} .

In this case pipes  and  may meet very early, and  could be much bigger than .  Note that we are never in this case when .  We have that

After the next round---which we know occurs because ---
 will be the pipe preceding  (which is ).
Using , it is easy to see that

and from the formula for  we have


Combining the previous inequalities,

Now we can apply Lemma \ref{L_term_lem} to finish the bound:

Therefore we can charge the increase in  this iteration to  used in the next iteration.
\end{description}

For a particular chunk  of , round 's increase may be bounded by a -factor increase and round  may be bounded by a -factor increase.  Each charge only occurs once.
The rotation step adds another factor of  on top of this.
Therefore, the total growth of  is at most

\end{proof}
\end{document} 