\documentclass[a4paper]{article}


\usepackage{amsfonts}    
\usepackage{algorithm}
\usepackage[noend]{algorithmic}
\usepackage{geometry}
\usepackage{fullpage}
\usepackage{amsmath}
\usepackage{amsthm}
\usepackage{amssymb}
\usepackage{graphicx}
\usepackage{dsfont}
\usepackage{setspace}
\usepackage[shortlabels]{enumitem}
\setlist{nolistsep}


\newtheorem{theorem}{Theorem}
\newtheorem{lemma}[theorem]{Lemma}
\newtheorem{claim}{Claim}
\newtheorem{definition}{Definition}
\newtheorem{corollary}[theorem]{Corollary}
\newtheorem{conjecture}{Conjecture}
\newtheorem{observation}[theorem]{Observation}
\newtheorem{remark}{Remark}
\newtheorem{open}{Open Problem}




\newenvironment{indentpar}[1]{
 \begin{list}{}{\setlength{\leftmargin}{#1}
         \setlist{nolistsep} }\item[]}
 {\end{list}}

\renewcommand{\algorithmicrequire}{\textbf{Input:}}
\renewcommand{\algorithmicensure}{\textbf{Output:}}

\renewcommand{\labelitemi}{}
\renewcommand{\labelitemii}{}

\newcommand{\old}[1]{{}}
\newcommand{\versionA}[1]{{}}
\newcommand{\versionB}[1]{{#1}}

\newcommand{\verAlgA}[1]{{}}
\newcommand{\verAlgB}[1]{{#1}}


\newcommand{\bs}{\backslash}
\newcommand{\MST}{\textsc{MST}}
\newcommand{\MCRA}{\textsc{MinRange}}
\newcommand{\MCRAS}{\textsc{MinRangeSpanner}}
\newcommand{\AlgMCRA}{\textsc{1DMinRA}}
\newcommand{\AlgMCRAS}{\textsc{1DMinRASpanner}}
\newcommand{\AproxAlgMCRA}{\textsc{ApproxMinRA}}
\newcommand{\orderedline}{line alike}


\pagestyle{plain}





\title{On the Minimum Cost Range Assignment Problem
}


\author{Paz Carmi\footnote{Department of Computer Science, Ben-Gurion University of the Negev, Israel}
 and Lilach Chaitman-Yerushalmi
\footnote{Department of Computer Science, Ben-Gurion University of the Negev, Israel}}










\begin{document}
\maketitle



\begin{abstract}
We study the problem of assigning transmission ranges to radio stations  
placed arbitrarily in a -dimensional (-D) Euclidean space in order to achieve a strongly connected communication network
with minimum total power consumption.
The power required for transmitting in range  is proportional to ,
where  is typically between  and , depending on various environmental factors.
While this problem can be solved optimally in D, in higher dimensions it is known to be -hard for any .

For the D version of the problem, i.e., radio stations located on a line and , 
we propose an optimal -time algorithm. 
This improves the running time of the best known algorithm by a factor of .
Moreover, we show a polynomial-time algorithm for finding the minimum cost range assignment in D
whose induced communication graph is a -spanner, for any  .

In higher dimensions, finding the optimal range assignment is -hard;
however, it can be approximated within a constant factor. 
The best known approximation ratio is for the case , where the approximation ratio is .
We show a new approximation algorithm with improved approximation ratio of , where  is a small constant.
\end{abstract}





\section{Introduction}\label{sec:Intro}
A wireless ad-hoc  network is a self-organized decentralized network that consists of independent radio transceivers (transmitter/receiver) 
and does not rely on any existing infrastructure. The network nodes (stations) communicate over radio channels. 
Each node broadcasts a signal over a fixed range and any node within this transmission range receives the signal.
Communication with nodes outside the transmission range is done using multi-hops, i.e., intermediate nodes pass the message forward 
and form a communication path from the source node to the desired target node.
The twenty-first century witnesses widespread deployment of wireless networks for professional and private
applications. The field of wireless communication continues to experience unprecedented market growth. 
For a comprehensive survey of this field see~\cite{pahlavan2005}.

Let  be a set of points in the -dimensional Euclidean space representing radio stations.
A \emph{range assignment} for  is a function  that assigns each point a transmission range (radius). 
The cost of a range assignment, representing the power consumption of the network, is defined as 
 for some real constant , 
where  varies between  and values higher than , depending on different environmental factors~\cite{pahlavan2005}.


A range assignment  induces a \emph{directed communication graph}
, where  and  denotes the Euclidean distance between  and . 
A range assignment  is \emph{valid} if the induced (communication) graph  is strongly connected.
For ease of presentation, throughout the paper we refer to the terms `assigning a range  to a point ' 
and `adding a directed edge ' as equivalent.

We consider the -D \textsc{Minimum Cost Range Assignment} (\MCRA) problem,
that takes as input a set  of  points in ,
and whose objective is finding a valid \emph{range assignment} for 
of minimum cost.
This problem has been considered extensively in various settings, for different values of  and , 
with additional requirements and modifications.
Some of these works are mentioned in this section.


In \cite{Kirousis}, Kirousis et al. considered the D \MCRA\ problem (the radio stations are placed
arbitrarily on a line) and showed an -time algorithm which computes an optimal solution for the problem.
Later, Das et al.~\cite{DasGN07} improved the running time to . Here, we propose an -time exact algorithm, 
this improves the running time of the best known algorithm by a factor of  without increasing the space complexity. 
The novelty of our method lies in separating the range assignment into two, \emph{left} and \emph{right}, range assignments
(elaborated in Section~\ref{sec:Linear}). 
This counter intuitive approach allows us to achieve the aforementioned result and, moreover,
to compute an optimal range assignment in D with 
the additional requirement that the induced graph is a -spanner, for a given .


A directed graph  is a -spanner for a set , if for every two points  
there exists a path in  from  to  of length at most .
The importance of avoiding flooding the network when routing, 
was one of the reasons that led researchers to consider the combination of range assignment and -spanners, 
e.g.,~\cite{KarimRCK09,Shpungin,WYLX02,WangLi}, 
as well as the combination of range assignment and hop-spanners, e.g.,~\cite{Clementi00,Kirousis}.
While bounded-hop spanners bound the number of intermediate nodes forwarding a message, -spanners bound the relative distance a message is forward. 
For the D bounded-hop range assignment problem, Clementi et al.~\cite{Clementi00} showed a 2-approximation algorithm 
whose running time is .
To the best of our knowledge, we are the first to show an algorithm that computes an optimal solution for the 
range assignment with the additional requirement that the induced graph is a -spanner.  


While the D version of the \MCRA\ problem can be solved optimally,
for any  and , it has been proven to be -hard 
(in~\cite{Kirousis} for  and  and later in~\cite{Clementi} for  and ). 
However, some versions can be approximated within a constant factor. 
For  and any 
Kirousis et al.~\cite{Kirousis} gave a 2-approximation algorithm based on the minimum spanning tree 
(although they addressed the case of  their result holds for any ). 
The best known approximation ratio is for the case , where the approximation ratio is ~\cite{Ambuhl05}.
We show a new approximation algorithm for this case\footnote{Values of  smaller than  correspond to areas, 
such as, corridors and large open indoor areas~\cite{pahlavan2005}.} with improved approximation ratio of , for a suitable constant .
We do not focus on increasing  but rather on showing that there exists an approximation ratio for this problem 
that is strictly less than . This is in contrast to classic problems, such as metric TSP and strongly connected sub-graph problems, 
for which the  ratio bound has not yet been breached.








\section{Minimum Cost Range Assignment in 1D}\label{sec:Linear}
In the D version of the \MCRA\ problem, the input set  consists of points located on a line. 
For simplicity, we assume that the line is horizontal and for every ,  is to the left of .
Given two indices , we denote by  the subset .


We present two polynomial-time algorithms for finding optimal range assignments, 
the first, in Section~\ref{sec:LinMinCostRange}, for the basic D \MCRA \ problem,
and the second, in Section~\ref{subsec:spanner}, subject to the additional requirement 
that the induced graph is a -spanner (the D \MCRAS \ problem).
Our new approach for solving these problems requires introducing a variant of the \emph{range assignment}.
Instead of assigning each point in  a radius,
we assign each point two directional ranges, \emph{left range assignment}, ,
and \emph{right range assignment}, .
A pair of assignments  is called a \emph{left-right assignment}.
Assigning a point  a left range  and a right range 
implies that in the induced graph, ,  can reach every point to its left up to distance 
and every point to its right up to distance .
That is, ,
contains the directed edge  iff one of the following holds:
(i)  and , or \ 
(ii)  and .
The cost of an assignment , is defined as
.

Our algorithms find a \emph{left-right assignment} of minimum cost
that can be converted into a \emph{range assignment}  with the same cost
by assigning each point  a range .
Note that any valid \emph{range assignment} for  can be converted to a a \emph{left-right assignment}
with the same cost, by assigning every point , .
To be more precise, either  or  should be reduced to , 
where  is the farthest point in the directional range 
(for Lemma~\ref{lem:range_ij} to hold).
Therefore, a minimum cost \emph{left-right assignment}, implies a minimum cost \emph{range assignment}.

In addition to the  function, we define 
and refine the term of \emph{optimal solution} to include
only solutions that minimize 
among all solutions, , with minimum .





\subsection{An Optimal Algorithm for the 1D \MCRA \ Problem}\label{sec:LinMinCostRange}
Das et al.~\cite{DasGN07} state three basic lemmas regarding properties of an optimal range assignment.
The following three lemmas are adjusted versions of these lemmas for a \emph{left-right assignment}.
\begin{lemma}\label{lem:range_ij}
In an optimal solution  
for every ,
either   or  
and similarly, either  or  for some .
\end{lemma}
\begin{lemma}\label{lem:range_line}
Given three indices , consider an optimal solution for , 
denoted by , subject to the condition that  and ,
then, 
\begin{itemize}
	\item  for all ,  and ; and
   \item for all ,  and .
\end{itemize}
\end{lemma}
\begin{lemma}\label{lem:range_12}
In an optimal solution ,
 and .
\end{lemma}

Lemma~\ref{lem:range_ij} allows us to simplify the notation 
 for  and ,  
and write  for short.
We solve the \MCRA\ problem using dynamic programming.
Given , we denote by  the cost of an optimal solution for the sub-problem
defined by the input , subject to the condition that .
Note that the cost of an optimal solution for the whole problem is .

In Section~\ref{subsec:cubic} we present an algorithm with  running time and  space
(the same time and space as in~\cite{DasGN07}).
Then, in Section~\ref{subsec:Quad} we 
reduce the running time to .





\subsubsection{A Cubic-Time Algorithm }\label{subsec:cubic}
Algorithm~\AlgMCRA\ (Algorithm~\ref{alg}) applies dynamic programming to compute the values 
for every  and store them in a table, .
Finally, it outputs the value .
In our computation we use a -dimensional matrix, \emph{Sum}, storing 
for every  the sum .
\begin{algorithm}[htp]
\caption{\AlgMCRA()}\label{alg}
\begin{algorithmic}
	\FOR { \textbf{downto} }
			\FOR {  \textbf{downto}  }
\STATE	
				
				\ENDFOR
		\ENDFOR	
\FOR { \textbf{downto} }
\STATE
				-2mm]
																 	&  + \max\{ |v_i v_k|^\alpha, |v_k v_{k'}|^\alpha \}\}
															 \end{aligned}  &\: ,otherwise 
											\end{array}
										\right. OPT(i) = \min_{ \substack{i < k < n \\ k <  k' \leq n}}  
						\left\{	\sum_{m=i}^{k'-2}  |v_m v_{m+1}|^\alpha  + OPT(k'-1)  - |v_{k'-1} v_{k'}|^\alpha + \max\{ |v_i v_k|^\alpha, |v_k v_{k'}|^\alpha \}	\right\}.		 	OPT(i)=		
	        				\min_{i+1 < k' \leq n} 
										\left\{
										\begin{aligned}
										 \sum_{m=i}^{k'-2}  |v_m v_{m+1}|^\alpha  &+ OPT(k'-1) - |v_{k'-1} v_{k'}|^\alpha \\
									 	& + \max\{ |v_i v_{c(i,k')}|^\alpha, |v_{c(i,k')} v_{k'}|^\alpha \}
									 	 \end{aligned} 
									 	\right\}.				 	
 	
\begin{aligned}
	OPT(&i, i+1,\overrightarrow{\delta}, \overleftarrow{\delta}, \delta^i)= \overrightarrow{r} + \overleftarrow{r}, \text{where}\\
		  &\overrightarrow{r} =   	
								\left\{
											\begin{array}{ll}  
								        			 |i,i+1| \text{ (*\emph{assigning }*)}&\:  ,\overrightarrow{\delta}/|i,i+1|> t \\
															 0 &\: ,otherwise 
											\end{array}
								\right.
\end{aligned}
 	
\begin{aligned}
	&OPT(i, j,\overrightarrow{\delta}, \overleftarrow{\delta}, \delta^i)= \\
	&\min_{\substack{i \leq k \leq j \\ k \leq k' \leq j}}   	
					\left\{
						\begin{array}{ll}
							\begin{aligned}  
								    |i,i+1|^\alpha &+ OPT(i,\, i+1, \,|i,i+1|, \,|i+i,j|+\overleftarrow{\delta}, \,\varnothing) \\
								        					&+ OPT(i+1, \,j, \,\infty, \,\overleftarrow{\delta}-|i+1, i|, \,\varnothing) 
								  \end{aligned} &\:  ,i=k \4mm]
								  \begin{aligned}  
								    &\max\{|i,k|^\alpha, |k,k'|^\alpha\} \\
								    																	&+ OPT(i,\,i+1,\,\delta^i,\,|i+1,k|+|k,i|,\,\varnothing) \\
								        															&+ OPT(i+1,\,k,\,\infty,\,|k,i+1|,\,\infty) \\
								        															&+ OPT(k,\,k'-1,\,|k,k'-1|,\,\infty,\,|k,k+1|) \\
								        															&+ OPT(k'-1,\,j,\,|k'-1,i|+\overrightarrow{\delta},\,\overleftarrow{\delta}-|i,k'-1|,\,|k'-1,k|+|k,k'|). \\
								  \end{aligned} &\:  ,i \neq k \neq k' \\
							\end{array}
					\right.\\
\end{aligned}
largew(MST(S))MST(S)P_M1.5\simlargeP_M\simlargeMST(S)P_M\simlargeP_MMST(S)1\sim2mm]
\textbf{The main algorithm scheme:} \\
Compute four solutions and return the one of minimal cost.
In case of multiple assignments to a point in a solution, the maximal among the ranges counts.\\
\underline{Solution} (i): apply the \emph{Hub} algorithm.\\
\underline{Solution} (ii): apply a variant of the \emph{Hub} algorithm - find a point  that minimizes the value , 
where  and  are the endpoints of the path .
Assign  the range , direct  towards  and bi-direct all edges in .\1mm]
\textbf{For} every edge  do :
\begin{indentpar}{0.4cm}
			Let  and  be the two paths of , to the left and to the right of , respectively.\\
Apply the \emph{flatten procedure}  on  and  to obtain the sub-paths\\
 and , respectively.\\ 
			(* \textit{Note  has been changed during the \emph{flatten procedure}} *)\\
			\textbf{For} every  points 
			with  in the flattened sub-paths:
\begin{indentpar}{0.6cm}
	          In both solutions (iii) and (iv) direct the path  towards  
		        and for each point  with  direct  towards  and assign  a range .		
						Perform the least cost option among the following two, either 
						add the edge , or add the two edges, 
						one from  to  for  of minimal length
						and the other from  to  for  of minimal length.
						(see illustration in Fig.~\ref{fig:alg2}).
As for the two sub-paths  and ,
						assign them ranges as follows:\vspace{0.1cm} \\
\underline{Solution} (iii): apply the \emph{Hub} algorithm separately on each sub-path.\\ 
						\underline{Solution} (iv): apply the \emph{D RA} algorithm separately on each sub-path 
						with respect to the \emph{distance function }
						and perform the \emph{transformation } on the assignment received.
			\end{indentpar}
\end{indentpar}
 
\begin{figure}[bht]
    \centering
        \includegraphics[width=0.6\textwidth]{alg2.pdf}
    \caption{The two sub-paths  and  as defined in the algorithm.}
    \label{fig:alg2}
\end{figure}


\begin{description}[topsep=0.2cm, itemsep=0.1cm]

\item[The \emph{flatten} procedure .]
Let . Given a path , set . Let  be the maximal index such that .
If such index does not exist, let .
Else (), add the edge  to , remove the edge  from , 
move the sub-path  from  to the forest , 
and update . \
Finally, repeat with the sub-path  without initializing .
\end{description}




The definitions for  and  are given with respect to
the sub-paths  and , the definitions for the sub-path  and  are symmetric.
\begin{description}[topsep=0.2cm, itemsep=0.1cm]
\item[The distance function .]
For every two points  with  we define,


\item[The \emph{adjustment} transformation .]
Given an assignment , 
we transform it into an assignment . First, we assign ranges a follows:

\end{description}

where .
The multiplicity (by ) handles the gaps caused by points breaking the \emph{\orderedline\ } condition with respect to the Euclidean metric. 
The role of the additive part, together with
the second stage of the transformation, elaborated next, is to overcome the absence of points outside the path.
In the second stage, 
for every  with , let  be the minimal index
for which there exists  with ,
and let  be the maximal index
for which there exists  with ,
direct the sub-path between  and  towards .
See Fig.~\ref{fig:transformation} for illustration.
\begin{figure}[htb]
    \centering
        \includegraphics[width=0.78\textwidth]{transformation.pdf}
    \caption{An illustration of the second stage in the adjustment transformation .}
    \label{fig:transformation}
\end{figure}

The indexing of points in  and notations introduced in the algorithm are used throughout this section.
The tree , for , is sometimes denoted by  for short.
\subsection{The Validity of the Output}\label{sec:validity}
We consider each solution separately and show it forms a \emph{valid} assignment .


\noindent
\textbf{Validity of Solution (i).} \
Follows from the validity of the \emph{Hub} algorithm


\noindent
\textbf{Validity of Solution (ii).} \ 
The subgraph of  induced by the points of  is strongly connected due to the validity of the \emph{Hub} algorithm.
All trees in  are bi-directed trees sharing a common point with 
and therefore, the whole graph, , is strongly connected. 

\noindent
\textbf{Validity of Solution (iii).} \
Each tree in  induces a strongly connected subgraph of ,
thus, it is suffices to show the connectivity of the minor obtained from 
by contracting all trees of .
The subgraph of  induced by the points of the middle sub-path  
forms either one directed cycle or two directed cycles sharing common vertices. 
By the correctness of the \emph{Hub} algorithm, each of the sub-paths  and  induces 
a strongly connected subgraph of .
In addition, each of them shares a common vertex with the middle sub-path,
thus, the whole graph, , is strongly connected.

\noindent
\textbf{Validity of Solution (iv).} \
Since we already verified the validity of Solution (iii), 
we are only left to show that each of  and  induce strongly connected subgraphs in .
We consider the sub-path , while the case of  is symmetric.

Let  denote the assignments obtained by applying 
the D algorithm on the sub-path  with respect to .
Due to the validity of , the graph induced by  with respect to 
is strongly connected.
Let  be an edge in this graph, we show that there exists a directed path from  to  in . 
Assume w.l.o.g., .
By the definition of , there exist  and 
with , such that .
Since the final assignment  is obtained after applying the adjustment transformation ,
we have .
\begin{description}[topsep=0.2cm, itemsep=0.1cm]
\item[Case 1:]  or .\\
							 Assume, w.l.o.g., that the second condition holds, then  
							 and thus, by the definition of transformation , 
							 the directed path from  to  and the edge  are contained in .
							 Together with the directed path in  from  to the root  they form a cycle containing both  and .
\item[Case 2:] both  and .\\
							 Since , and  is the output sub-path after performing
							 the \emph{flatten} procedure, then .
							 Therefore, we have
							 
					
\end{description}




\subsection{The Approximation Ratio}

Let  denote the cost of the output of the algorithm for the input set 
and let  denote an optimal assignment for  of cost .
We show that .

Let , ,  denote the longest edge in 
and .
As shown in~\cite{Ambuhl05}, .
Next we show several upper bounds on the ratio , 
corresponding to the four solutions computed during the algorithms and finally conclude that the minimum
among them equals at most .\\label{eq:i}
\frac{SOL}{OPT} \leq \frac{W(1.5-r/2+l/2)}{W(1+l)} = \frac{1.5-(r-l)/2}{1+l}.

1.5-\epsilon < \frac{SOL}{OPT} < \frac{1.5-(r-l)/2}{1+l} < 1.5-r/2,  f(P_M)P_MfP_{e^l}P_{e^r}w(P_M)-2c_h W(1-1/c_s)P_M\tilde{c}\sim  \frac{1}{2}[ w(P_M)-c_h W(1-\frac{1}{c_s})+w(e_M) ]     SOL \leq  W(1-r+ \frac{1}{2}[(1-r)-c_h(1-\frac{1}{c_s})+l] +2r)
		 =  	W[1.5-\frac{1}{2}(c_h(1-\frac{1}{c_s})-l-r)]. \label{eq:ii}
\frac{SOL}{OPT} \leq \frac{1.5-\frac{1}{2}(c_h(1-\frac{1}{c_s})-l-r)}{1+l}.

1.5-\epsilon & < \frac{1.5-\frac{1}{2}(c_h(1-\frac{1}{c_s})-l-r)}{1+l} \ < \ 1.5-\frac{1}{2}(c_h(1-\frac{1}{c_s})-r) \\
						 & <	1.5-\frac{1}{2}(20\epsilon(\frac{1}{5})-2\epsilon) \ <	\ 1.5-\epsilon \qquad  \qquad \Rightarrow\ \text{ \ \ contradiction.}						 
f(P_M)(u,v)\in Q_{P_{e^l}}\cup Q_{P_{e^r}}\delta_{P_M}(u,v) \leq c_h W\tilde{c}\sim2mm]
\noindent
\textbf{Approximation bound for Solution (iii).} \
Preforming the \emph{Hub} algorithm on  and , separately, result
in two assignments with a total cost of at most .
Directing the path  and the trees in , and assigning all roots in  their assignment, 
together with adding the two edges,  for  of minimal length
and  for  of minimal length 
(or the edge  instead if it is cheaper)
costs at most \

Overall, we have a total cost of at most \


Since there is an edge connecting a point in  with a point in  (in some direction)
and an edge connecting a point in  with a point in  in
the optimal solution, we have ,
hence,



\noindent
\textbf{Approximation bound for Solution (iv).} \
Let  and  
denote the assignments obtained by applying the \emph{D RA} algorithm 
on  and , respectively, with respect to the distance function . 
Let  denote the union of the two assignments
and let  denote the cost of , i.e., .
\begin{claim}\label{cl:OPT'}
.
\end{claim}
\begin{proof}
We show that the optimal assignment  can be adjusted to an assignment\\
\mbox{}, valid for  and , separately, with respect to , of the same cost.\\
We define,

and for every  with , \ \  

Let  and  be two points in the same sub-path, w.l.o.g., , 
and let  be the path from  to  in .
Consider the sequence , obtained by replacing every  with ,
and every  with , for .
We prove the above sequence forms a path from  to  
in the graph induced by  with respect to  and , and conclude that  is valid and .
Consider a pair of consecutive nodes in the above sequence, . 
Note that if  (resp. ) for , than  (resp. ) for ; thus,
for every , by the definition of  and  we have, 
 
and the claim follows.
\end{proof}

Let  be the union of the assignments obtained
after applying the transformation  on each of  and , separately.
The following lemma bounds its cost.
\begin{lemma}\label{lem:g(rho)}

\end{lemma}
\begin{proof}
Consider applying transformation  on  and .
By multiplying the range of every point in  by 
we obtain an assignment of cost .
As for the additive part and the second stage, we analyze the cost with respect to , while the case for  is symmetric.
Every , is responsible for an additional of at most 
to the total cost,
where  and  are defined as in the definition of .
The first element in the summation is the range added to  itself,
and the second is the cost of directing the path between 
and  towards  (depicted in Fig.~\ref{fig:transformation} in black).

Allegedly, for computing  we should sum  over all  and add it to the cost ,
however, we observe that it is suffices to consider a point  only once
as , for the rightmost point  such that  and only once
as , for the leftmost point  such that .
Thus, we can charge  and  themselves once on 
each of the elements  and  in the overall summation.
Namely, charge every point  for a total range increase of .
Summing  over all  and adding the cost , using Claim~\ref{cl:OPT'} gives 

\end{proof}

Note that  has already assigned to every  an assignment greater than .
Directing all trees in  towards their roots,
directing the path  and adding the edge , adds to the cost
at most , and
together with Lemma~\ref{lem:g(rho)} we receive

\begin{lemma}\label{lem:approx2}
If condition L\ref{lem:sf}.2 holds for ,
then .
\end{lemma}
\begin{proof}
Assume towards contradiction that . 
By Corollary~\ref{cor:Soli},  and together with equation~(\ref{eq:iii}) we receive,

Replacing  and  with the above upper bounds in equation~(\ref{eq:v}) gives,

and by equation~(\ref{eq:iv}) we have,

The upper bound on  together with equation~(\ref{eq:i}) imply,

in contradiction to our choice of .
\end{proof}
We conclude with Theorem~\ref{theo:2Dmain}, derived from
Lemma~\ref{lem:sf} together with Lemmas~\ref{lem:approx1} and~\ref{lem:approx2}.

\begin{theorem}\label{theo:2Dmain}
Given a set  of points in  for ,
a minimum cost range assignment -approximation can be computed in polynomial time for , where . 
\end{theorem} 

The reader can notice that our algorithm yields a better approximation bound than stated in the above theorem.
However, we preferred the simplicity of presentation over a more complicated analysis resulting in a tighter bound.

\bibliographystyle{plain}


\begin{thebibliography}{10}

\bibitem{KarimRCK09}
K.~Abu-Affash, R.~Aschner, P.~Carmi, and M.~J. Katz.
\newblock Minimum power energy spanners in wireless ad-hoc networks.
\newblock In {\em INFOCOM}, 2010.

\bibitem{Ambuhl05}
C.~Amb\"{u}hl, A.~E.~F. Clementi, P.~Penna, G.~Rossi, and R.~Silvestri.
\newblock On the approximability of the range assignment problem on radio
  networks in presence of selfish agents.
\newblock {\em Theor. Comput. Sci.}, 343(1-2):27--41, October 2005.

\bibitem{Clementi}
A.~E.~F. Clementi, P.~Penna, and R.~Silvestri.
\newblock On the power assignment problem in radio networks.
\newblock {\em Mob. Netw. Appl.}, 9(2):125--140, April 2004.

\bibitem{Clementi00}
A.E.F. Clementi, A.~Ferreira, P.~Penna, S.~Perennes, and R.~Silvestri.
\newblock The minimum range assignment problem on linear radio networks.
\newblock In {\em ESA}, 2000.

\bibitem{DasGN07}
G.~K. Das, S.~C. Ghosh, and S.~C. Nandy.
\newblock Improved algorithm for minimum cost range assignment problem for
  linear radio networks.
\newblock {\em Int. J. Found. Comput. Sci.}, 18(3):619--635, 2007.

\bibitem{Kirousis}
L.~Kirousis, E.~Kranakis, D.~Krizanc, and A.~Pelc.
\newblock Power consumption in packet radio networks.
\newblock {\em Theoretical Computer Science}, 243(1-2):289--305, 2000.

\bibitem{pahlavan2005}
K.~Pahlavan.
\newblock {\em Wireless information networks}. 
\newblock John Wiley, Hoboken, NJ, 2005.

\bibitem{Shpungin}
H.~Shpungin and M.~Segal.
\newblock Near optimal multicriteria spanner constructions in wireless ad-hoc
  networks.
\newblock In {\em INFOCOM}, pages 163--171, 2009.

\bibitem{WYLX02}
Y.~Wang and X.-Y. Li.
\newblock Distributed spanner with bounded degree for wireless ad hoc networks.
\newblock In {\em IPDPS '2002:}, page 120, 2002.

\bibitem{WangLi} 
Y.~Wang and X.-Y. Li.
\newblock Minimum power assignment in wireless ad hoc networks with spanner
  property.
\newblock {\em Journal of Combinatorial Optimization}, 11(1):99--112, 2006.

\end{thebibliography}

  
\end{document}
