

We will now prove the positive results of Theorems \ref{thm:pos-paths}, \ref{thm:pos-fractional}, and \ref{thm:pos-discrete}.

\subsection{Discrete load balancing in paths and cycles}
\label{ssec:pos-paths}

We first give an algorithm that exactly matches the lower bound of Theorem~\ref{thm:neg-linear}.

\thmpospaths*

\paragraph{Infinite directed paths.}

We will first show how to do load balancing in an \emph{infinite path with a consistent orientation}. That is, each node  has a degree of , and it can refer to its \emph{left neighbour}  and \emph{right neighbour}  in a globally consistent manner.

We will interpret the path with tokens as a -dimensional grid, indexed by , where  is a node and  is a possible location for a token. We say that  is a \emph{slot}. Initially, slot  holds a \emph{token} if . Our plan is to move the tokens around in the grid so that we maintain the following stability conditions---see Figure~\ref{fig:slots} for an illustration.

\begin{figure}[ht]
    \centering
    \includegraphics[scale=1.2,page=10]{figs.pdf}
    \caption{Stability. Token  is -stable, as there exists a token in slot , where , i.e., the node  steps right from . Also  and  are -stable. However, this configuration is not -stable: token  is not -stable, as there is an empty slot . It can be verified that the configuration is -stable, -stable, and -stable.}\label{fig:slots}
\end{figure}

\begin{definition}
    A token in slot  is -stable if  or there is a token in slot . A configuration is -stable if all tokens are -stable. For a set , a configuration is -stable if it is -stable for all .
\end{definition}

We write . Initially, the configuration is -stable. If we can find a -stable configuration, we can construct a feasible solution to the load balancing problem by simply setting  to be equal to the number of tokens in slots .

However, we will now design an -time algorithm with a \emph{stronger} stability condition: it will compute a -stable configuration. Informally, we smooth out the load distribution so that the slope of the load curve is at most . This extra slack will be helpful when we eventually want to solve the problem in paths without consistent orientations.

This algorithm is based on the concept of \emph{pushes}. For a node  and integer , define the -diagonal of  as the following list of slots (see Figure~\ref{fig:push}):

In an -push we redistribute the tokens in each : if there are  tokens in , then we redistribute the tokens so that the first  elements of  are occupied and the remaining  elements are empty (see Figure~\ref{fig:push}). In essence, we let the tokens slide along each diagonal so that they are piled on the bottom of each diagonal.

\begin{figure}[ht]
    \centering
    \includegraphics[scale=1.2,page=11]{figs.pdf}
    \caption{Pushing. The -diagonal of  is highlighted. In a -push, we redistribute the tokens in each -diagonal so the end result will be a -stable configuration. Note that this configuration was already -stable and -stable, and it remained -stable and -stable after a -push. In general, whatever stability we have already achieved by pushing is never lost in subsequent pushes.}\label{fig:push}
\end{figure}

An -push can be efficiently implemented in time  with a distributed algorithm: for example, node  is responsible for redistributing the tokens in slots , and we first use  rounds so that each node  can discover everything related to , and then another  rounds so that node  can inform the relevant nodes regarding how to move tokens in .

Clearly, after an -push we will have an -stable configuration. The non-trivial part is that -pushes do not interfere with any stability that we have previously achieved.

\begin{restatable}{lemma}{lempush}\label{lem:push}
For every choice of integers  and , if a configuration is -stable, then it is still -stable after an -push.
\end{restatable}

\begin{proof}
The case  is trivial; hence we assume that . Consider slots  and ; see Figure~\ref{fig:pushlemma}. We need to argue that if  holds a token after an -push, then  will also hold a token after an -push. To this end, let  be the -diagonal that contains , and let  be the -diagonal that contains . Now by definition, after an -push, slot  is occupied if and only if there were at least  tokens in .

\begin{figure}[ht]
    \centering
    \includegraphics[scale=1.2,page=12]{figs.pdf}
    \caption{The proof of Lemma~\ref{lem:push}, for  and . The configuration was -stable, i.e., the gray -diagonals filled starting from the bottom. Now we do a -push, and want to argue that the configuration will be still -stable. Consider slot . It will be filled iff there are at least  tokens in the -diagonal . But this implies that there are at least  tokens in the diagonal , and hence the slot  will be filled, too.}\label{fig:pushlemma}
\end{figure}

The key observation is that -stability implies that for every token in  there is a token in , with the exception of the first token---if  holds a token and  then  holds a token as well. In particular, if there were at least  tokens in , there were at least  tokens in , and hence  will also hold a token.
\end{proof}

Now we can easily find a -stable configuration in time : the algorithm simply does an -push for each , sequentially, in an arbitrary order. We will call this algorithm~.


\paragraph{Finite directed paths and cycles.}

Algorithm  finds a -stable configuration in infinite directed paths in time . To handle \emph{finite} directed paths we could extend the algorithm and its analysis so that it takes into account the boundary effects. However, this would be a bit boring---instead, we will show that we can simply take  and use it as a black box.

Let us first adjust the stability condition so that it makes sense on finite paths: a token  is considered -stable also if node  does not exist.

Let  be the worst-case running time of . We use it to construct an algorithm  that finds a -stable configuration for a finite path , as follows:
\begin{enumerate}
    \item Check if the path is of length at most ; if so, we solve the problem by brute force in time , and stop.
    \item Each endpoint  gathers all tokens up to distance  and redistributes them so that all nodes within distance at most  from  have the same constant load; let us denote this constant .
    \item Construct a virtual graph  as follows: each endpoint  pretends that the path continues with infinitely many additional dummy nodes, each with the same constant load .
    \item Simulate algorithm  in the virtual graph .
    \item Discard the dummy nodes.
\end{enumerate}
It is easy to verify that the  will never move any tokens across an endpoint, as its neighbourhood was already well-balanced. Therefore if we remove the dummy nodes, we have a feasible solution for . Moreover, the running time of  is still .

It is also easy to see that  works correctly in directed \emph{cycles}; the first three steps simply do nothing as there are no endpoints.


\paragraph{Undirected paths and cycles.}

So far we have designed an algorithm  that finds a -stable configuration in paths and cycles with a globally consistent orientation. Now we show how to use it to design an algorithm  that finds a -stable configuration in paths and cycles without an orientation.

It can be shown that \emph{some} form of local symmetry-breaking is needed. We will use the familiar \emph{port-numbering model}: Each node  has up to two communication ports, labelled with  and . The ports are identified with the endpoints of the edges; each edge joins a pair of ports. The port numbers at the endpoints of an edge do not need to match---for example, an edge  may join  to  or  to .

In algorithm , we construct a \emph{virtual graph}  as shown in Figure~\ref{fig:virtpath}: Each node  splits itself in two virtual nodes,  and . The virtual nodes also have two ports. For each edge , depending on the type of  we connect the virtual nodes of  and  as follows:
\begin{itemize}[noitemsep]
    \item  joins  to : connect  to  and  to ,
    \item  joins  to : connect  to  and  to ,
    \item  joins  to : connect  to  and  to ,
    \item  joins  to : connect  to  and  to .
\end{itemize}
If  was a path with  nodes, then  consists of two disjoint paths with  nodes each. If  was an -cycle, then  consists of either one cycle with  nodes or two cycles with  nodes each.

\begin{figure}[ht]
    \centering
    \includegraphics[scale=1.2,page=13]{figs.pdf}
    \caption{Given any path  with some port numbering, we can construct a virtual graph  that consists of two paths, both of which have a \emph{consistent} port numbering.}\label{fig:virtpath}
\end{figure}

The key observation is that there is a \emph{consistent} port numbering in : port  of a virtual node is always connected to port  of an adjacent virtual node. We can now interpret the ports so that in each virtual node port  points ``left'' and port  points ``right''.

Each node first splits its input load arbitrarily between its virtual copies. Then we run algorithm  to find a -stable configuration in the virtual graph, and then map all tokens back to the original graph: the new load of  is the sum of the new loads of  and ; see Figure~\ref{fig:pathfix}.

\begin{figure}[ht]
    \centering
    \includegraphics[scale=1.2,page=14]{figs.pdf}
    \caption{The sum of two -stable configurations can be easily turned into a -stable configuration with local modifications.}\label{fig:pathfix}
\end{figure}

Now we have a configuration where the maximum load difference between a pair of adjacent nodes is . However, the load is \emph{approximately well-balanced}: a load difference of more than  implies a distance of at least . Therefore we can easily find a -stable configuration in  time with local operations (see Figure~\ref{fig:pathfix}). For example, we can apply a match-and-balance algorithm: find a maximal matching  of unhappy edges and move a token over each edge. Conveniently, all edges become happy, including those that were not in~. It is easy to find a maximal matching  in  time, as this is in essence maximal matching in a bipartite graph of maximum degree : on one side we have the nodes that are ``too low'' and on the other side we have the nodes that are ``too high'' in comparison with their neighbours.

In summary, we can find a -stable configuration in any path or cycle in time , and therefore we can do discrete load balancing in any path or cycle in time .


\subsection{Discrete load balancing in general graphs}\label{ssec:pos-discrete}

We will now show how to do discrete load balancing in graphs of maximum degree .

\thmposdiscrete*

Again, we will imagine that each node  has  \emph{slots}, labelled , and each token is placed in one of the slots. Initially slots  are occupied with tokens.

We define the \emph{(downward) cone}  of slot  as the set of slots  such that ; see Figure~\ref{fig:cone}. In the algorithm, if there is a token in  and all slots of the cone  are full, then we say that the token is \emph{stable}, and we \emph{freeze} it, i.e. it will never be moved again.

\begin{figure}[ht]
    \centering
    \includegraphics[scale=1.2,page=15]{figs.pdf}
    \caption{The downward cone  of the token  consists of the slots denoted with white boxes.}\label{fig:cone}
\end{figure}

In the algorithm we try to match the highest unfrozen tokens with the free slots in their cones. If they succeed then they move to these slots; otherwise they can be frozen.

We now give the pseudo-code of the algorithm in a centralised way, prove the correctness of the algorithm, and then show that it is actually a local algorithm. The algorithm proceeds as follows:
\begin{enumerate}
	\item All stable tokens of the initial configuration are frozen.
	\item For each :
	\begin{enumerate}
		\item Construct the virtual bipartite graph , where  consists of unfrozen tokens at level ,  consists of all empty slots at levels below , and there is an edge  if  is an empty slot in the cone of token .
		\item In , find a maximal matching .
		\item For every unfrozen token  at level : if the token is matched with a slot  in , move the token to slot , otherwise freeze it. 
		\item Collapse the tokens so that for each node  that holds  tokens, the tokens are in the slots .
	\end{enumerate}
\end{enumerate} 

First, remark that we maintain the invariant that at round , all load in slots at height~ either moves down or is safely frozen.  Indeed, if a token is not matched, then all slots in its cone will be full at the end of the loop, and if it is matched, it moves to a strictly lower level, thereafter the invariant is true for level  and maintained for the levels above. At the end of the algorithm all the tokens are frozen, thus the configuration is stable.

We stated the algorithm in a centralised manner, but it is actually local: The vertices only need the knowledge of their radius- neighbourhood to find their neighbours in graph . Graph  has a maximum degree of . Therefore we can find a maximal matching in  by simulating  rounds of the proposal algorithm \cite{hanckowiak98distributed} in the virtual graph . The simulation has a multiplicative  overhead---adjacent nodes in  are at distance  in graph . Finally, we have  iterations, giving the overall complexity of .


\subsection{Fractional load balancing in general graphs}\label{ssec:pos-fractional}

In fractional load balancing, we can use the same basic idea as what we had in the discrete case, but much faster:

\thmposfractional*

Our algorithm follows the same basic structure as the discrete algorithm of Section~\ref{ssec:pos-discrete}. However, in each bipartite virtual graph , we compute an -maximal fractional matching. With the algorithm by Khuller et al.~\cite{khuller94primal-dual}, this can be done in  rounds, which gives us an exponential speedup over the -round algorithm for maximal bipartite matching.

\paragraph{Almost maximal fractional matchings.}

Let the bipartite graph be  with  and  be the maximum degree. Each node has a \emph{capacity} . A fractional matching is a function  such that for each node, the sum  is at most . A fractional matching is \emph{-maximal} if

that is, there is no edge  with a value  that could be increased by more than .

\paragraph{Algorithm for fractional load balancing.}

Next we describe the algorithm for finding a fractional load balancing. As before, we have \emph{slots} labelled with ; here  is a node and  is the \emph{level} of the slot. However, now each slot may contain fractional units of load. We adapt the definition of stability to fractional load balancing in a natural manner: we say that  units of the load in slot  is stable if each  with  has at least  units of load, and each  with  is full. In the algorithm we will \emph{freeze} some parts of the load. We use  to denote the total amount of load in slot , and  to denote the amount of frozen load in slot . 

The algorithm would be simpler to analyse if we had a maximal fractional matching algorithm, but we only have an efficient -maximal one. Our strategy is to round the load, and by doing it, we accumulate \emph{surplus} and \emph{deficit}. This way we can analyse easily the iterations of the algorithm. We keep track of surplus and deficit, and at the end of the algorithm we readjust the loads.

The algorithm for fractional load balancing works as follows. First, double all load; this way we have  slots per node. Based on the new input, freeze all stable load. Then, for each  perform the following steps:
\begin{enumerate}
    \item Construct the virtual bipartite graph , where  consists of slots with unfrozen load at level ,  consists of all slots at levels below , and there is an edge  if  is a non-full slot in the cone of slot .
    \item Define the capacities of  as follows: the capacity of a node  is its unfrozen load , and the capacity of a node  is its free space .
    \item Find an -maximal fractional matching  in .
    \item\label{step:move} For each edge  of , move  units of load from slot  to slot .
    \item\label{step:freeze} For each  consider the two cases:
    \begin{itemize}
        \item If  has load at most , round it to 0 and freeze the load in . We create at most  units of deficit in slot  at the moment we freeze it.
        \item If  has strictly more than  load, then each unfrozen slot in the cone of  has at least  load. We round them to 1 and freeze all load at  and in . We create at most  units of surplus in each slot of  at the moment we freeze it.
    \end{itemize}
    \item Collapse all unfrozen load as low as possible.
\end{enumerate}
Finally, undo the rounding---remove the surplus and put back the deficit. Then normalise the output by dividing all load by two.

\paragraph{Analysis.}

Thanks to the rounding, the configuration after each iteration satisfies the same invariant as the discrete algorithm: at round , all load in slots at height  either moves down or is safely frozen. Then the configuration at the end of the iteration phase is stable and for each edge  we have . 

Each slot is involved at most once in the rounding, precisely at the moment we freeze it. Hence before normalisation, the difference between the real load and the rounded load is in  for each node. Therefore we have  for each edge  before normalisation and  after normalisation. We guarantee a feasible solution by choosing .

Khuller et al.~\cite{khuller94primal-dual} show how to find an -maximal fractional matching in time  in graphs of maximum degree . The virtual graph  has a maximum degree of , and there is an  overhead in the simulation of  in . Finally, we have  iterations in the algorithm; in total, the running time can be bounded by
 
This completes the proof of Theorem~\ref{thm:pos-fractional}.

