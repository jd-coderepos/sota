

\documentclass[conference,compsocconf]{IEEEtran}









\usepackage{graphicx}
\usepackage{amssymb}
\setcounter{tocdepth}{3}


\usepackage{algorithm}
\usepackage{algorithmic}

\usepackage{eso-pic}
\AddToShipoutPicture*{\small{\sffamily\raisebox{1.8cm}{\hspace{1.8cm}{978-1-4673-5828-6/13/\DT_{\tthres}N_{\twifi}T_{\tthres}D^{\tmin}_{\twifi}D_{\tmobile}T^{\tmin}_{\twifi, i}, T^{\tmax}_{\twifi, i}iT_{\tmobile}T_{{\mbox{\tiny next WiFi}}}R^{\tmin}_{\twifi, i}, R^{\tmax}_{\twifi, i}iR_{\tmobile}D^{\tcache}_{\twifi, \tnext}\OffsetD^{\tmin}_{\twifi}  \leftarrow \sum_{i \in N_{\twifi}} \left ( R^{\tmin}_{\twifi, i} \cdot T^{\tmin}_{\twifi, i} \right )T^{\tmin}_{\twifi}  \leftarrow \sum_{i \in N_{\twifi}} T^{\tmin}_{\twifi, i}D_{\tmobile} \leftarrow D-D^{\tmin}_{\twifi}T_{\tmobile} \leftarrow T_{\tthres}-T^{\tmin}_{\twifi}R_{\tmobile} \leftarrow  D_{\tmobile}/T_{\tmobile}D^{\tcache}_{\twifi, \tnext} \leftarrow  R^{\tmax}_{\twifi, \tnext} \cdot T^{\tmax}_{\twifi, \tnext}\Offset \leftarrow  R_{\tmobile} \cdot T_{ {\mbox{\tiny  next WiFi}}}D^{\tcache}_{\twifi, \tnext}\OffsetR_{\tmobile}\Offset><[0.9 \cdot 18, 1.1 \cdot 18]=[16.2, 19.8][0.6 \cdot M,1.4 \cdot M]MM, W,AM/4, M/3M/2, MW/4, W/3W/2, WA/4, A/3A/2, A\,\,$ \\
\begin{minipage}[b]{0.5\linewidth}
\centering
\hspace{-0.22in}
\includegraphics[width=1.7in]{ds_adsl.png}\\
{\footnotesize  \hspace{-0.22in}\small{(c) ADSL throughput}}
\end{minipage}
\begin{minipage}[b]{0.5\linewidth}
\centering
\hspace{-0.22in}
\includegraphics[width=1.7in]{ds_hotspots.png}\\
{\footnotesize  \hspace{-0.22in}\small{(c) \# of WiFi hotspots}}
\end{minipage}
\end{tabular}
\end{center}
\vspace{-.1 in}
\caption[]{\protect \small{Transfer delay as a function of mobile, WiFi, and ADSL throughput, and number of WiFi hotspots. Delay sensitive traffic.
}}
\label{fig:ds_thr}
\end{figure}


\medskip
\noindent
\emph{Time error:} Figure~\ref{fig:ds_error}(a) shows  that as the time error increases, the variability of the transfer delay increases slightly (the 95\% confidence interval is larger), but the average transfer delay for all schemes remains the same. Figure~\ref{fig:ds_error_energy}(a) shows the energy consumption as a function of time errors. Observe that the average energy efficiency gains are independent of the time errors and are relatively higher compared to the transfer delay gains: When prediction and prefetching are used the energy consumption is more than 40\% lower than when only the mobile network is used, whereas the transfer delay reduction  is approximately 27\%.

\medskip
\noindent
\emph{Throughput error:} Figure~\ref{fig:ds_error}(b) shows the influence of the throughput error on the transfer delay. As expected, the transfer delay gains are higher for lower throughput errors; however, observe that some gains still exist with prediction and prefetching even when the throughput error becomes very high (80\%). Also, observe that with a high throughput error the transfer delay when WiFi offloading without prediction and prefetching is used can be higher than when only the mobile network is used.
Figure~\ref{fig:ds_error_energy}(b) shows the energy consumption as a function of throughput errors. Observe that a higher throughput error reduces the energy efficiency gains, which however still remain high: with a 80\% throughput error, prediction and prefetching achieve lower energy consumption by approximately 30\% compared to the case where only the mobile network is used and 13\% when WiFi offloading is used without prediction and prefetching.




\begin{figure}[tb]
\begin{center}
\begin{tabular}{c}

\begin{minipage}[b]{0.5\linewidth}
\centering
\hspace{-0.22in}
\includegraphics[width=1.7in] {ds_timeerror.png}\\
{\footnotesize \hspace{-0.22in}\small{(a) Time error}}
\end{minipage}
\begin{minipage}[b]{0.5\linewidth}
\centering
\hspace{-0.22in}
\includegraphics[width=1.7in]{ds_threrror.png}\\
{\footnotesize  \hspace{-0.22in}\small{(b) Throughput error}}
\end{minipage}\

\end{tabular}
\end{center}
\vspace{-.1 in}
\caption[]{\protect \small{Transfer delay as a function of time and throughput error. Delay sensitive traffic.
}}
\label{fig:ds_error}
\vspace{-0.15in}
\end{figure}



\begin{figure}[tb]
\begin{center}
\begin{tabular}{c}

\begin{minipage}[b]{0.5\linewidth}
\centering
\hspace{-0.22in}
\includegraphics[width=1.7in] {ds_timeerror_energy.png}\\
{\footnotesize \hspace{-0.22in}\small{(a) Time error}}
\end{minipage}
\begin{minipage}[b]{0.5\linewidth}
\centering
\hspace{-0.22in}
\includegraphics[width=1.7in]{ds_threrror_energy.png}\\
{\footnotesize  \hspace{-0.22in}\small{(b) Throughput error}}
\end{minipage}\

\end{tabular}
\end{center}
\vspace{-.1 in}
\caption[]{\protect \small{Energy consumption as a function of time and throughput error. Delay sensitive traffic.
}}
\label{fig:ds_error_energy}
\vspace{-0.05in}
\end{figure}








\mynotex{
\begin{itemize}
\item Travel duration for route considered: 269 sec. Need to check that all transfer durations are smaller than this value.
\item Delay tolerant: amount of data offloading. Compare mobile+wifi \& mobile+wifi+prefetch. File sizes up to 50 MB. Performance metric: percentage of offloaded traffic.
\item Delay sensitive: transfer delay. Compare mobile-only, mobile+wifi, mobile+wifi+prefetching. Consider file sizes up to 30 MB. Performance metrics: delay, percentage of offloaded traffic. For prefetching: total buffer for cache.
\item Can possibly also consider battery consumption using numbers e.g. power/MB for mobile and for wifi.
\item How performance is influenced by time uncertainty in mobility prediction. This influences prefetching.
\item influence of number of wifi hotspots. N=3,6,9
\item delay of handover between mobile and wifi network not considered.

\end{itemize}

}


\mynotex{
Graphs:
\begin{itemize}
\item Percentage of offloaded traffic, in case of delay tolerant traffic. D=10,20,30,40 MB. Time error=10 \%, Radsl=5. Here we do not have mobile-only case, but rather compare prefetching with no-prefetching.
\item Percentage of offloaded traffic as function of time error, WiFi throughput error. Delay tolerant traffic. as before, we compare no-prefetching with prefetching.
\item Transfer delay as function of data object size for mobile+Wifi and mobile+Wifi+prefetching, in case of delay sensitive traffic. D=10,20,30,40 MB. Time error=10 \%, Radsl=5 Mb/s. Also add mobile-only for D=10-30 MB.
\item Cache size as function of data size and/or as function of time error. This is one or two graphs. D=10,20,30,40 MB. Time error=10, 30, 50,70,90 \%. Delay sensitive traffic.
\item Transfer delay as a function of time error, WiFi throughput error.
\item need graph showing influence of throughput errors. 20, 30, 40\%
\end{itemize}
}


\mynotex{
Graphs:
\begin{itemize}
\item Cumulative: delay sensitive: 2nd WiFi hotspot allows more data to be cached, compared to first.
\item delay sensitive: average improvement of prefetching over no prefetching is 24\%, and over mobile-only is 60\%
\item delay sensitive: time error affects conf interval. average performance improvements remain the same.
\item Delay tolerant: higher data size => lower performance of both prefetching and prediction-only. prediction only has no gains over no-prediction since throughput close to max mobile throughput is used
\item delay tolerant: low adsl=> prediction approaches no prediction. high adsl=> lower gain from prefetching
\item delay tolerant: higher time and throughput error affects prefetching and confidence interval. Higher time error => lower performance of prefetching. Still performance of prefetching remains higher compared to when no prediction and no prefetching is used. Performance of mobility prediction is affected less, since time error affects all segments, whereas prefetching depends on duration of mobile segment.
\item buffer requirements: higher for delay tolerant.
\end{itemize}
}





\section{Conclusions and Future Work}
\label{sec:conclusions}

We have presented a comprehensive evaluation of procedures that exploit mobility prediction and prefetching to enhance mobile data offloading, for both delay tolerant and delay sensitive traffic. Our evaluation is in terms of the amount of offloaded traffic, the data transfer delay, and the energy consumption, and shows how the performance depends on various factors, such as the data object size, the mobile, WiFi, and ADSL backhaul throughput, the number of WiFi hotspots, and the robustness of the proposed procedures to time and throughput estimation errors.
Future work includes implementing a prototype to demonstrate the gains of the proposed offloading procedures. Moreover, we are extending the procedures to  allow different tradeoffs between the delay, the amount of offloaded traffic, and the energy efficiency, and to exploit prediction and prefetching for streaming video.


\mynotex{
\begin{itemize}
\item streaming video
\item prototype
\end{itemize}
}

\mynotex{
\begin{itemize}
\item evaluate energy gains
\item implement prototype, extending the OptiPath application. In a first phase this will involve only the procedure to exploit mobility prediction, since prefetching needs support (caching) at the WiFi hotspots.
\end{itemize}
}










\bibliographystyle{abbrv}

{
\bibliography{pref} }


\end{document}
