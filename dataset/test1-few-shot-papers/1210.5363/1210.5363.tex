\documentclass[a4paper,10pt]{article}

\usepackage{fullpage}
\usepackage{xspace}
\usepackage{comment}










\usepackage{amsmath,amssymb,amstext,amsthm}


\newtheorem{theorem}{Theorem}
\newtheorem{lemma}[theorem]{Lemma}
\newtheorem{definition}[theorem]{Definition}
\newtheorem{corollary}[theorem]{Corollary}
\newtheorem{remark}[theorem]{Remark}
\newtheorem{proposition}[theorem]{Proposition}


\newcommand{\tudu}[1]{{\bf{TODO: #1}}}
\newcommand{\transname}[2]{-{\sc{transversal}}}
\newcommand{\himmersion}{{\sc{-Immersion}}\xspace}
\newcommand{\kpaths}{-{\sc{Disjoint Paths}}\xspace}
\newcommand{\dpath}{{\sc{Edge Disjoint Paths}}\xspace}
\newcommand{\toursection}{{\sc{Semi-complete Digraph Balanced Cut}}\xspace}
\newcommand{\vertexpath}{{\sc{Vertex Disjoint Paths}}\xspace}
\newcommand{\SAT}{{\sc{SAT}}\xspace}
\newcommand{\tcnfsat}{{\sc{3-CNF-SAT}}\xspace}
\newcommand{\compass}{\ensuremath{\textrm{NP} \subseteq \textrm{coNP}/\textrm{poly}}}
\newcommand{\cw}{\mathbf{cw}}
\newcommand{\ctw}{\mathbf{ctw}}
\newcommand{\pw}{\mathbf{pw}}
\newcommand{\N}{\ensuremath{\mathbb{N}}}
\newcommand{\Aa}{\ensuremath{\mathcal{A}}}
\newcommand{\Bb}{\ensuremath{\mathcal{B}}}
\newcommand{\Jj}{\ensuremath{\mathcal{J}}}
\newcommand{\Pp}{\ensuremath{\mathcal{P}}}
\newcommand{\Qq}{\ensuremath{\mathcal{Q}}}
\newcommand{\Ss}{\ensuremath{\mathcal{S}}}
\newcommand{\Ii}{\ensuremath{\mathcal{I}}}
\newcommand{\randfamily}{\ensuremath{\mathcal{F}}}


\newcommand{\WO}{{\sc{W[1]}}\xspace}
\newcommand{\FPT}{{\sc{FPT}}\xspace}
\newcommand{\NP}{{\sc{NP}}\xspace}
\newcommand{\XP}{{\sc{XP}}\xspace}
\newcommand{\MSO}{\ensuremath{\mathbf{MSO}}}
\newcommand{\ff}{\ensuremath{\varphi}}

\newcommand{\FF}{\ensuremath{\mathfrak{F}}}
\newcommand{\Ff}{\ensuremath{\mathcal{F}}}
\newcommand{\Gg}{\ensuremath{\mathcal{G}}}
\newcommand{\UU}{\ensuremath{\mathfrak{U}}}
\newcommand{\ptrace}{trace}
\newcommand{\ptraces}{traces}

\newcommand{\vf}[2]{\mathcal{V}_{(#1,#2)}}
\newcommand{\ef}[2]{\mathcal{E}_{(#1,#2)}}
\newcommand{\nequiv}{\not\equiv}

\newcommand{\cutfam}{\mathcal{N}}
\newcommand{\bipclass}{\mathcal{B}}
\newcommand{\match}{\mathfrak{M}}
\newcommand{\OPT}{\textrm{OPT}}
\newcommand{\buss}{\mathfrak{B}}
\newcommand{\rev}{{\rm{rev}}}



 
\newcommand{\defproblemu}[3]{
  \vspace{1mm}
\noindent\fbox{
  \begin{minipage}{0.95\textwidth}
  #1 \\
  {\bf{Input:}} #2  \\
  {\bf{Question:}} #3
  \end{minipage}
  }
  \vspace{1mm}
}
\newcommand{\defparproblem}[4]{
  \vspace{1mm}
\noindent\fbox{
  \begin{minipage}{0.96\textwidth}
  \begin{tabular*}{\textwidth}{@{\extracolsep{\fill}}lr} #1  & {\bf{Parameter:}} #3 \\ \end{tabular*}
  {\bf{Input:}} #2  \\
  {\bf{Problem:}} #4
  \end{minipage}
  }
  \vspace{1mm}
}

\renewcommand{\H}{\mathcal{H}}
\newcommand{\cO}{\mathcal{O}}
\newcommand{\sm}{\setminus}
\newcommand{\om}{\Omega}
\newtheorem{Rule}{Rule}
\newcommand{\eps}{\varepsilon}

\title{Computing cutwidth and pathwidth of semi-complete digraphs via degree orderings}

\date{}

\author{Micha\l{} Pilipczuk\thanks{Department of Informatics, University of Bergen, Norway, \texttt{michal.pilipczuk@ii.uib.no}}}


\begin{document}

\maketitle


\begin{abstract}
The notions of cutwidth and pathwidth of digraphs play a central role in the containment theory for tournaments, or more generally semi-complete digraphs, developed in a recent series of papers by Chudnovsky, Fradkin, Kim, Scott, and Seymour~\cite{ChudnovskyFS2011,ChudnovskySS2011,ChudnovskyS11,fradkin-seymourEDP,fradkin-seymour,kim-seymour-minors}. In this work we introduce a new approach to computing these width measures on semi-complete digraphs, via degree orderings. Using the new technique we are able to reprove the main results of~\cite{ChudnovskyFS2011,fradkin-seymour} in a unified and significantly simplified way, as well as obtain new results. First, we present polynomial-time approximation algorithms for both cutwidth and pathwidth, faster and simpler than the previously known ones; the most significant improvement is in case of pathwidth, where instead of previously known -approximation in fixed-parameter tractable time~\cite{my} we obtain a constant-factor approximation in polynomial time. Secondly, by exploiting the new set of obstacles for cutwidth and pathwidth, we show that topological containment and immersion in semi-complete digraphs can be tested in single-exponential fixed-parameter tractable time. Finally, we present how the new approach can be used to obtain exact fixed-parameter tractable algorithms for cutwidth and pathwidth, with single-exponential running time dependency on the optimal width.
\end{abstract}

\section{Introduction}

{\bf{Minors of (di)graphs.}} The {\em{Graph Minors}} series of Robertson and Seymour not only resolved the Wagner's Conjecture, but also provided a number of algorithmic tools for investigating topological structure of graphs. A useful concept that is repeatedly used in many arguments in this theory, is the WIN/WIN approach that can be formulated as follows. We introduce a graph width measure , usually defined as the optimal width of some decomposition. The crucial part is to prove a structural theorem that asserts that graphs with large measure  contain certain obstacles to admitting a simpler decomposition. These obstacles can be exploited in many ways: either we can immediately provide an answer to the algorithmic problem under consideration, or, for instance, find a vertex in the obstacle whose deletion does not change the answer. If no obstacle is present, we can find a decomposition of small width and try to apply algorithms based on the dynamic programming paradigm. The most classical measure  is the treewidth of the graph, accompanied by a grid minor as the obstacle; however, the framework has found applications in many other contexts as well.

Since the beginning of the graph minors project there was a thrilling question, to what extent similar results can be obtained in the directed setting. The answer is still unclear; however, it seems that there is no hope for a positive outcome in the general case. There are several reasons for this. Despite numerous attempts, we still lack any good width measure that would be an analogue of treewidth~\cite{ganian-i-kumple}; moreover, most of the containment problems are already NP-hard for small, constant-size graphs  that are to be embedded into input graph ~\cite{FortuneHW80}. Therefore, the scope of research moved to identifying subclasses of digraphs where construction of a sound containment theory could be possible. We invite an interested reader to the introduction of a recent work of Fomin and the current author~\cite{my} for a more detailed overview of the status of containment problems in digraphs.

\vskip 0.15cm

\noindent{\bf{Tournaments.}} In a recent series of papers, Chudnovsky, Fradkin, Kim, Scott, and Seymour~\cite{ChudnovskyFS2011,ChudnovskySS2011,ChudnovskyS11,fradkin-seymourEDP,fradkin-seymour,kim-seymour-minors} identify tournaments as a class where an elegant containment theory can be constructed. Recall that a tournament is a digraph where every two vertices are connected by an arc directed in one of two possible directions; the results hold also for a slightly more general class of {\em{semi-complete}} digraphs, where we additionally allow existence of two arcs directed in opposite directions. The central notions of the theory are two width measures of digraphs: {\em{cutwidth}} and {\em{pathwidth}}. The first one is based on the ordering of vertices and resembles classical cutwidth in the undirected setting~\cite{thilikos-i-kumple}, with the exception that only arcs directed forward in the ordering contribute to the width of a cut. The second one is a similar generalization of undirected pathwidth. See Section \ref{sec:width} for formal definitions.

Chudnovsky, Fradkin, and Seymour~\cite{ChudnovskyFS2011} prove a structural theorem that provides a set of obstacles for admitting an ordering of small (cut)width; a similar theorem for pathwidth was proven by Fradkin and Seymour~\cite{fradkin-seymour}. A large enough obstacle for cutwidth admits every fixed-size digraph as an immersion, and the corresponding is true also for pathwidth and topological containment. Basing on the first result, Chudnovsky and Seymour~\cite{ChudnovskyS11} were able to show that immersion is a well-quasi-order on the class of semi-complete digraphs. The same is not true for topological containment, but recently Kim and Seymour~\cite{kim-seymour-minors} introduced a relaxed notion of {\em{minor}} order, which indeed is a well-quasi-order of semi-complete digraphs.

As far as the algorithmic aspects of the work of Chudnovsky, Fradkin, Kim, Scott, and Seymour are concerned, the original proofs of the structural theorems can be turned into approximation algorithms, which given a semi-complete digraph  and an integer  find a decomposition of  of width , or provide an obstacle for admitting decomposition of width at most . For cutwidth the running time is polynomial, but for pathwidth it is  for some function ; this excludes usage of this approximation as a subroutine in any FPT algorithm\footnote{An algorithm for a parameterized problem is said to be {\em{fixed-parameter tractable}} (FPT), if it runs in time  on instances of size  and parameter , for some function  and constant .}, for instance, for topological containment testing. This gap has been bridged by Fomin and the current author~\cite{my} by presenting an FPT approximation algorithm for pathwidth that again either finds a path decomposition of  of width  or provides an obstacle for admitting path decomposition of width at most , but runs in  time. The approach in~\cite{my} is based on replacing the crucial part of the algorithm of Fradkin and Seymour that was implemented by a brute-force enumeration, by a more refined argument based on the colour coding technique.

As far as computation of optimal decompositions is concerned, by the well-quasi-ordering result of Chudnovsky and Seymour~\cite{ChudnovskyS11} we have that the class of semi-complete digraphs of cutwidth bounded by a constant is characterized by a finite set of forbidden immersions; the result of Kim and Seymour~\cite{kim-seymour-minors} proves that we can infer the same conclusion about pathwidth and minors. Having approximated the corresponding parameter, in FPT time we can check if any of these forbidden structures is contained in a given semi-complete digraph, using dynamic programming. This gives FPT exact algorithms computing cutwidth and pathwidth; however, they are both non-uniform ---  the algorithms depend on the set of forbidden structures which is unknown --- and non-constructive --- they provide just the value of the width measure, and not the optimal decomposition. To the best of author's knowledge, nothing better was known for these problems.

\vskip 0.15cm

\noindent{\bf{Our results and techniques.}} In this paper we present a new approach to computing cutwidth and pathwidth of semi-complete digraphs, via degree orderings. Using the new technique we are able to reprove the structural theorems for both parameters in a unified and simplified way, obtaining in both cases polynomial-time algorithms and, in case of pathwidth, a constant-factor approximation. The technique can be also used to develop FPT exact algorithm for both width measures with single-exponential dependency of the running time on the optimal width, as well as to trim the dependency of the running time on the size of the tested digraph in the topological containment and immersion tests to single-exponential. All the algorithms presented in this paper have quadratic dependency on the number of vertices of a given semi-complete digraph; note that this is {\em{linear}} in the input size. 

\begin{figure}
\begin{centering}
\noindent\begin{tabular}{| c | c | c |}
\hline
Problem & Previous results & This work \\
\hline
Cutwidth approximation &  time, width ~\cite{ChudnovskyFS2011} &  time, width  \0.2cm]
Pathwidth approximation &  time,  &  time,  \\
& width ~\cite{my} & width  \0.2cm]
Immersion &  time~\cite{ChudnovskyFS2011} &  time \|V^{-+}\setminus V^{++}|\leq d^+(w)-|V^{++}|\leq k+d^+(v)-|V^{++}|\leq k+1+|V^{+-}|\leq 2k.
|E(X,Y)| & \leq & |E(X_\alpha,Y)|+|E(X,Y_\beta)|+|E(X\setminus X_\alpha,Y\setminus Y_\beta)| \\
         & \leq & |E(X_\alpha,Y_\alpha)|+|E(X_\beta,Y_\beta)|+|E(X\setminus X_\alpha,Y\setminus Y_\beta)| \\
         & \leq & k+k+(10k+1)^2 = 100k^2+22k+1.
(m+2)^k \cdot (m+2)^m\cdot m^\ell\cdot m!\cdot (m+2)^\ell=2^{O((k+\ell+m)\log m)}.
Moreover, all of them can be enumerated in  time.
\end{lemma}
\begin{proof}
The consecutive terms correspond to:
\begin{enumerate}
\item the choice of mapping  on ;
\item for every element of , choice whether it will be the end of some subpath in some path signature, and in this case, the value of corresponding beginning (a vertex from  or );
\item for every pair composed in such manner, choice to which  it will belong;
\item the ordering of pairs along the path signatures;
\item for every , choice whether to append a pair of form  at the end of the signature , and in this case, the value of  (a vertex from  or ).
\end{enumerate}
It is easy to check that using all these information one can reconstruct the whole signature. For every object constructed in the manner above we can check in time polynomial in , whether it corresponds to a possible signature. This yields the enumeration algorithm.
\end{proof}

The set of possible signatures on separation  will be denoted by . We are now ready to present the dynamic programming routine.

\begin{lemma}\label{lem:dp}
There exists an algorithm, which given a digraph  with  vertices and  edges, and a semi-complete digraph  together with its path decomposition of width , checks whether  is topologically contained in  in time .
\end{lemma}
\begin{proof}
Let  be a path decomposition of  of width . Without loss of generality we assume that   is a nice path decomposition.

By Lemma~\ref{lem:signature-bound}, for every separation  with separator  the number of possible signatures is . We will consecutively compute the values of a binary table  with the  following meaning. For ,  tells, whether there exists a mapping  with the following properties:
\begin{itemize}
\item for every ,  if  and  if ;
\item for every ,  is a correct path trace with signature , beginning in  if  and anywhere in  otherwise, ending in  if  and anywhere in  otherwise;
\item path traces  are internally vertex-disjoint.
\end{itemize}
Such mapping  will be called a {\emph{partial expansion}} of  on .

For the first separation  we have exactly one signature with value , being the signature which maps all the vertices into  and all the arcs into empty signatures. The result of the whole computation should be the value for the signature for the last separation , which maps all vertices into  and arcs into signatures consisting of one pair . Therefore, it suffices to show how to fill the values of the table for {\bf{introduce vertex}} step and {\bf{forget vertex}} step.

\paragraph*{Introduce vertex step}

Let us introduce vertex  to the separation , i.e., we consider the new separation . Let . We show that  can be computed by analyzing the way signature  interferes with the vertex .

\begin{itemize}
\item[{\emph{Case :}}] , that is,  is not an image of  a  vertex of .
\begin{itemize}
\item[{\emph{Case :}}]  for some pair  and some . This means that the signature of the partial expansion truncated to separation  must look exactly like , but without this subpath of length zero. Thus , where  is constructed from  by deleting this pair from the corresponding signature.
\item[{\emph{Case :}}]  for some pair  and some . This means that the partial expansion truncated to separation  has to look the same but for the path corresponding to this very pair, which needs to be truncated by vertex . The new beginning has to be either a vertex in , or a forgotten vertex from . As  is semi-complete and  is a separation, there is an arc from  to every vertex of . Therefore, , where the disjunction is taken over all signatures  such that in  the pair  is substituted with , where  or  is any vertex of  such that there is an arc .
\item[{\emph{Case :}}]  for some pair  and some . Similarly as before, partial expansion truncated to separation  has to look the same but for the path corresponding to this very pair, which needs to be truncated by vertex . As  is a separation, the previous vertex on the path has to be in the separator . Therefore, , where the disjunction is taken over all signatures  such that in  the pair  is substituted with , where  is any vertex of  such that there is an arc .
\item[{\emph{Case :}}]  is not contained in any pair in any signature from . Either  lies on some path in the partial expansion, or it does not. In the first case the corresponding path in partial expansion on  has to be split into two subpaths, when truncating the expansion to . Along this path, the arc that was used to access  had to go from inside , due to  being a separation; however, the arc used to leave  can go to  and to any forgotten vertex from , as  is a separation and  is a semi-complete digraph. In the second case, the signature of the truncated expansion stays the same. Therefore, , where the disjunction is taken over all signatures  such that in  exactly one pair  is substituted with two pairs  and , where  with , whereas  or  with .
\end{itemize}
\item[{\emph{Case :}}]  for some . For every ,  has to be the beginning of the first pair of ; otherwise, . Similarly, for every ,  has to be the end of the last pair of ; otherwise, . Therefore, , where the disjunction is taken over all signatures  such that the first pairs of all  are truncated as in the case  for all  (or as in case , if the beginning and the end coincide), while the last pairs of all  are truncated as in the case  for all  (or as in case , if the beginning and the end coincide). Moreover, we impose a condition that .
\end{itemize}

\paragraph*{Forget vertex step}

Let us forget vertex  from the separation , i.e., we consider the new separation . Let ; we argue that  for some set , which corresponds to possible extensions of  to the previous, bigger separation. We now discuss, which signatures  are needed in  by considering all the signatures  partitioned with respect to behaviour on vertex .

\begin{itemize}
\item[{\emph{Case :}}] , that is,  is not in the image of .
\begin{itemize}
\item[{\emph{Case :}}]  for some pair  and some . This means that in the corresponding partial expansions  had to be left to ; however, in  there is no arc from  to . Therefore, in  we consider no signatures  of this type.
\item[{\emph{Case :}}]  for some pair  and some . If we are to consider  in , then  must prolong some path from the signature  in such a manner that  is its beginning. After forgetting  the beginning of this path belongs to the forgotten vertices; therefore, in  we consider only signatures  which differ from  on exactly one pair: in  there is  instead of  in .
\item[{\emph{Case :}}]  for some pair  and some . If we are to consider  in , then  must prolong some path from the signature  in such a manner, that  is its end. As , after forgetting the vertex  the path should still end in the separator. We obtain a contradiction; therefore, we take no such signature under consideration in the set .
\item[{\emph{Case :}}]  is not contained in any pair in any signature from . In this case  has to be simply some unused vertex or some vertex inside a path; therefore, we take under consideration in  exactly one signature .
\end{itemize}
\item[{\emph{Case :}}]  for some . We proceed similarly to cases  and . We consider in  signatures  that differ from  in following manner:  where  for exactly one ; for all edges  the first pair of  is of form , whereas the first pair of  is of form ; for all edges  the last pair of  is of form , whereas the last pair of  is of the form .
\end{itemize}

Updating table  for each separation requires at most
 time, and since the number of separations in the pathwidth decomposition is , the theorem follows.
\end{proof}

\end{document}
