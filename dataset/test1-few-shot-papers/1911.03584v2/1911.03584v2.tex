\documentclass{article} \usepackage{iclr2020_conference,times}


\usepackage{amsmath,amsfonts,bm}

\newcommand{\figleft}{{\em (Left)}}
\newcommand{\figcenter}{{\em (Center)}}
\newcommand{\figright}{{\em (Right)}}
\newcommand{\figtop}{{\em (Top)}}
\newcommand{\figbottom}{{\em (Bottom)}}
\newcommand{\captiona}{{\em (a)}}
\newcommand{\captionb}{{\em (b)}}
\newcommand{\captionc}{{\em (c)}}
\newcommand{\captiond}{{\em (d)}}

\newcommand{\newterm}[1]{{\bf #1}}


\def\figref#1{figure~\ref{#1}}
\def\Figref#1{Figure~\ref{#1}}
\def\twofigref#1#2{figures \ref{#1} and \ref{#2}}
\def\quadfigref#1#2#3#4{figures \ref{#1}, \ref{#2}, \ref{#3} and \ref{#4}}
\def\secref#1{section~\ref{#1}}
\def\Secref#1{Section~\ref{#1}}
\def\twosecrefs#1#2{sections \ref{#1} and \ref{#2}}
\def\secrefs#1#2#3{sections \ref{#1}, \ref{#2} and \ref{#3}}
\def\eqref#1{equation~(\ref{#1})}
\def\Eqref#1{Equation~(\ref{#1})}
\def\plaineqref#1{\ref{#1}}
\def\chapref#1{chapter~\ref{#1}}
\def\Chapref#1{Chapter~\ref{#1}}
\def\rangechapref#1#2{chapters\ref{#1}--\ref{#2}}
\def\algref#1{algorithm~\ref{#1}}
\def\Algref#1{Algorithm~\ref{#1}}
\def\twoalgref#1#2{algorithms \ref{#1} and \ref{#2}}
\def\Twoalgref#1#2{Algorithms \ref{#1} and \ref{#2}}
\def\partref#1{part~\ref{#1}}
\def\Partref#1{Part~\ref{#1}}
\def\twopartref#1#2{parts \ref{#1} and \ref{#2}}

\def\ceil#1{\lceil #1 \rceil}
\def\floor#1{\lfloor #1 \rfloor}
\def\1{\bm{1}}
\newcommand{\train}{\mathcal{D}}
\newcommand{\valid}{\mathcal{D_{\mathrm{valid}}}}
\newcommand{\test}{\mathcal{D_{\mathrm{test}}}}

\def\eps{{\epsilon}}


\def\reta{{\textnormal{}}}
\def\ra{{\textnormal{a}}}
\def\rb{{\textnormal{b}}}
\def\rc{{\textnormal{c}}}
\def\rd{{\textnormal{d}}}
\def\re{{\textnormal{e}}}
\def\rf{{\textnormal{f}}}
\def\rg{{\textnormal{g}}}
\def\rh{{\textnormal{h}}}
\def\ri{{\textnormal{i}}}
\def\rj{{\textnormal{j}}}
\def\rk{{\textnormal{k}}}
\def\rl{{\textnormal{l}}}
\def\rn{{\textnormal{n}}}
\def\ro{{\textnormal{o}}}
\def\rp{{\textnormal{p}}}
\def\rq{{\textnormal{q}}}
\def\rr{{\textnormal{r}}}
\def\rs{{\textnormal{s}}}
\def\rt{{\textnormal{t}}}
\def\ru{{\textnormal{u}}}
\def\rv{{\textnormal{v}}}
\def\rw{{\textnormal{w}}}
\def\rx{{\textnormal{x}}}
\def\ry{{\textnormal{y}}}
\def\rz{{\textnormal{z}}}

\def\rvepsilon{{\mathbf{\epsilon}}}
\def\rvtheta{{\mathbf{\theta}}}
\def\rva{{\mathbf{a}}}
\def\rvb{{\mathbf{b}}}
\def\rvc{{\mathbf{c}}}
\def\rvd{{\mathbf{d}}}
\def\rve{{\mathbf{e}}}
\def\rvf{{\mathbf{f}}}
\def\rvg{{\mathbf{g}}}
\def\rvh{{\mathbf{h}}}
\def\rvu{{\mathbf{i}}}
\def\rvj{{\mathbf{j}}}
\def\rvk{{\mathbf{k}}}
\def\rvl{{\mathbf{l}}}
\def\rvm{{\mathbf{m}}}
\def\rvn{{\mathbf{n}}}
\def\rvo{{\mathbf{o}}}
\def\rvp{{\mathbf{p}}}
\def\rvq{{\mathbf{q}}}
\def\rvr{{\mathbf{r}}}
\def\rvs{{\mathbf{s}}}
\def\rvt{{\mathbf{t}}}
\def\rvu{{\mathbf{u}}}
\def\rvv{{\mathbf{v}}}
\def\rvw{{\mathbf{w}}}
\def\rvx{{\mathbf{x}}}
\def\rvy{{\mathbf{y}}}
\def\rvz{{\mathbf{z}}}

\def\erva{{\textnormal{a}}}
\def\ervb{{\textnormal{b}}}
\def\ervc{{\textnormal{c}}}
\def\ervd{{\textnormal{d}}}
\def\erve{{\textnormal{e}}}
\def\ervf{{\textnormal{f}}}
\def\ervg{{\textnormal{g}}}
\def\ervh{{\textnormal{h}}}
\def\ervi{{\textnormal{i}}}
\def\ervj{{\textnormal{j}}}
\def\ervk{{\textnormal{k}}}
\def\ervl{{\textnormal{l}}}
\def\ervm{{\textnormal{m}}}
\def\ervn{{\textnormal{n}}}
\def\ervo{{\textnormal{o}}}
\def\ervp{{\textnormal{p}}}
\def\ervq{{\textnormal{q}}}
\def\ervr{{\textnormal{r}}}
\def\ervs{{\textnormal{s}}}
\def\ervt{{\textnormal{t}}}
\def\ervu{{\textnormal{u}}}
\def\ervv{{\textnormal{v}}}
\def\ervw{{\textnormal{w}}}
\def\ervx{{\textnormal{x}}}
\def\ervy{{\textnormal{y}}}
\def\ervz{{\textnormal{z}}}

\def\rmA{{\mathbf{A}}}
\def\rmB{{\mathbf{B}}}
\def\rmC{{\mathbf{C}}}
\def\rmD{{\mathbf{D}}}
\def\rmE{{\mathbf{E}}}
\def\rmF{{\mathbf{F}}}
\def\rmG{{\mathbf{G}}}
\def\rmH{{\mathbf{H}}}
\def\rmI{{\mathbf{I}}}
\def\rmJ{{\mathbf{J}}}
\def\rmK{{\mathbf{K}}}
\def\rmL{{\mathbf{L}}}
\def\rmM{{\mathbf{M}}}
\def\rmN{{\mathbf{N}}}
\def\rmO{{\mathbf{O}}}
\def\rmP{{\mathbf{P}}}
\def\rmQ{{\mathbf{Q}}}
\def\rmR{{\mathbf{R}}}
\def\rmS{{\mathbf{S}}}
\def\rmT{{\mathbf{T}}}
\def\rmU{{\mathbf{U}}}
\def\rmV{{\mathbf{V}}}
\def\rmW{{\mathbf{W}}}
\def\rmX{{\mathbf{X}}}
\def\rmY{{\mathbf{Y}}}
\def\rmZ{{\mathbf{Z}}}

\def\ermA{{\textnormal{A}}}
\def\ermB{{\textnormal{B}}}
\def\ermC{{\textnormal{C}}}
\def\ermD{{\textnormal{D}}}
\def\ermE{{\textnormal{E}}}
\def\ermF{{\textnormal{F}}}
\def\ermG{{\textnormal{G}}}
\def\ermH{{\textnormal{H}}}
\def\ermI{{\textnormal{I}}}
\def\ermJ{{\textnormal{J}}}
\def\ermK{{\textnormal{K}}}
\def\ermL{{\textnormal{L}}}
\def\ermM{{\textnormal{M}}}
\def\ermN{{\textnormal{N}}}
\def\ermO{{\textnormal{O}}}
\def\ermP{{\textnormal{P}}}
\def\ermQ{{\textnormal{Q}}}
\def\ermR{{\textnormal{R}}}
\def\ermS{{\textnormal{S}}}
\def\ermT{{\textnormal{T}}}
\def\ermU{{\textnormal{U}}}
\def\ermV{{\textnormal{V}}}
\def\ermW{{\textnormal{W}}}
\def\ermX{{\textnormal{X}}}
\def\ermY{{\textnormal{Y}}}
\def\ermZ{{\textnormal{Z}}}

\def\vzero{{\bm{0}}}
\def\vone{{\bm{1}}}
\def\vmu{{\bm{\mu}}}
\def\vtheta{{\bm{\theta}}}
\def\va{{\bm{a}}}
\def\vb{{\bm{b}}}
\def\vc{{\bm{c}}}
\def\vd{{\bm{d}}}
\def\ve{{\bm{e}}}
\def\vf{{\bm{f}}}
\def\vg{{\bm{g}}}
\def\vh{{\bm{h}}}
\def\vi{{\bm{i}}}
\def\vj{{\bm{j}}}
\def\vk{{\bm{k}}}
\def\vl{{\bm{l}}}
\def\vm{{\bm{m}}}
\def\vn{{\bm{n}}}
\def\vo{{\bm{o}}}
\def\vp{{\bm{p}}}
\def\vq{{\bm{q}}}
\def\vr{{\bm{r}}}
\def\vs{{\bm{s}}}
\def\vt{{\bm{t}}}
\def\vu{{\bm{u}}}
\def\vv{{\bm{v}}}
\def\vw{{\bm{w}}}
\def\vx{{\bm{x}}}
\def\vy{{\bm{y}}}
\def\vz{{\bm{z}}}

\def\evalpha{{\alpha}}
\def\evbeta{{\beta}}
\def\evepsilon{{\epsilon}}
\def\evlambda{{\lambda}}
\def\evomega{{\omega}}
\def\evmu{{\mu}}
\def\evpsi{{\psi}}
\def\evsigma{{\sigma}}
\def\evtheta{{\theta}}
\def\eva{{a}}
\def\evb{{b}}
\def\evc{{c}}
\def\evd{{d}}
\def\eve{{e}}
\def\evf{{f}}
\def\evg{{g}}
\def\evh{{h}}
\def\evi{{i}}
\def\evj{{j}}
\def\evk{{k}}
\def\evl{{l}}
\def\evm{{m}}
\def\evn{{n}}
\def\evo{{o}}
\def\evp{{p}}
\def\evq{{q}}
\def\evr{{r}}
\def\evs{{s}}
\def\evt{{t}}
\def\evu{{u}}
\def\evv{{v}}
\def\evw{{w}}
\def\evx{{x}}
\def\evy{{y}}
\def\evz{{z}}

\def\mA{{\bm{A}}}
\def\mB{{\bm{B}}}
\def\mC{{\bm{C}}}
\def\mD{{\bm{D}}}
\def\mE{{\bm{E}}}
\def\mF{{\bm{F}}}
\def\mG{{\bm{G}}}
\def\mH{{\bm{H}}}
\def\mI{{\bm{I}}}
\def\mJ{{\bm{J}}}
\def\mK{{\bm{K}}}
\def\mL{{\bm{L}}}
\def\mM{{\bm{M}}}
\def\mN{{\bm{N}}}
\def\mO{{\bm{O}}}
\def\mP{{\bm{P}}}
\def\mQ{{\bm{Q}}}
\def\mR{{\bm{R}}}
\def\mS{{\bm{S}}}
\def\mT{{\bm{T}}}
\def\mU{{\bm{U}}}
\def\mV{{\bm{V}}}
\def\mW{{\bm{W}}}
\def\mX{{\bm{X}}}
\def\mY{{\bm{Y}}}
\def\mZ{{\bm{Z}}}
\def\mBeta{{\bm{\beta}}}
\def\mPhi{{\bm{\Phi}}}
\def\mLambda{{\bm{\Lambda}}}
\def\mSigma{{\bm{\Sigma}}}

\DeclareMathAlphabet{\mathsfit}{\encodingdefault}{\sfdefault}{m}{sl}
\SetMathAlphabet{\mathsfit}{bold}{\encodingdefault}{\sfdefault}{bx}{n}
\newcommand{\tens}[1]{\bm{\mathsfit{#1}}}
\def\tA{{\tens{A}}}
\def\tB{{\tens{B}}}
\def\tC{{\tens{C}}}
\def\tD{{\tens{D}}}
\def\tE{{\tens{E}}}
\def\tF{{\tens{F}}}
\def\tG{{\tens{G}}}
\def\tH{{\tens{H}}}
\def\tI{{\tens{I}}}
\def\tJ{{\tens{J}}}
\def\tK{{\tens{K}}}
\def\tL{{\tens{L}}}
\def\tM{{\tens{M}}}
\def\tN{{\tens{N}}}
\def\tO{{\tens{O}}}
\def\tP{{\tens{P}}}
\def\tQ{{\tens{Q}}}
\def\tR{{\tens{R}}}
\def\tS{{\tens{S}}}
\def\tT{{\tens{T}}}
\def\tU{{\tens{U}}}
\def\tV{{\tens{V}}}
\def\tW{{\tens{W}}}
\def\tX{{\tens{X}}}
\def\tY{{\tens{Y}}}
\def\tZ{{\tens{Z}}}


\def\gA{{\mathcal{A}}}
\def\gB{{\mathcal{B}}}
\def\gC{{\mathcal{C}}}
\def\gD{{\mathcal{D}}}
\def\gE{{\mathcal{E}}}
\def\gF{{\mathcal{F}}}
\def\gG{{\mathcal{G}}}
\def\gH{{\mathcal{H}}}
\def\gI{{\mathcal{I}}}
\def\gJ{{\mathcal{J}}}
\def\gK{{\mathcal{K}}}
\def\gL{{\mathcal{L}}}
\def\gM{{\mathcal{M}}}
\def\gN{{\mathcal{N}}}
\def\gO{{\mathcal{O}}}
\def\gP{{\mathcal{P}}}
\def\gQ{{\mathcal{Q}}}
\def\gR{{\mathcal{R}}}
\def\gS{{\mathcal{S}}}
\def\gT{{\mathcal{T}}}
\def\gU{{\mathcal{U}}}
\def\gV{{\mathcal{V}}}
\def\gW{{\mathcal{W}}}
\def\gX{{\mathcal{X}}}
\def\gY{{\mathcal{Y}}}
\def\gZ{{\mathcal{Z}}}

\def\sA{{\mathbb{A}}}
\def\sB{{\mathbb{B}}}
\def\sC{{\mathbb{C}}}
\def\sD{{\mathbb{D}}}
\def\sF{{\mathbb{F}}}
\def\sG{{\mathbb{G}}}
\def\sH{{\mathbb{H}}}
\def\sI{{\mathbb{I}}}
\def\sJ{{\mathbb{J}}}
\def\sK{{\mathbb{K}}}
\def\sL{{\mathbb{L}}}
\def\sM{{\mathbb{M}}}
\def\sN{{\mathbb{N}}}
\def\sO{{\mathbb{O}}}
\def\sP{{\mathbb{P}}}
\def\sQ{{\mathbb{Q}}}
\def\sR{{\mathbb{R}}}
\def\sS{{\mathbb{S}}}
\def\sT{{\mathbb{T}}}
\def\sU{{\mathbb{U}}}
\def\sV{{\mathbb{V}}}
\def\sW{{\mathbb{W}}}
\def\sX{{\mathbb{X}}}
\def\sY{{\mathbb{Y}}}
\def\sZ{{\mathbb{Z}}}

\def\emLambda{{\Lambda}}
\def\emA{{A}}
\def\emB{{B}}
\def\emC{{C}}
\def\emD{{D}}
\def\emE{{E}}
\def\emF{{F}}
\def\emG{{G}}
\def\emH{{H}}
\def\emI{{I}}
\def\emJ{{J}}
\def\emK{{K}}
\def\emL{{L}}
\def\emM{{M}}
\def\emN{{N}}
\def\emO{{O}}
\def\emP{{P}}
\def\emQ{{Q}}
\def\emR{{R}}
\def\emS{{S}}
\def\emT{{T}}
\def\emU{{U}}
\def\emV{{V}}
\def\emW{{W}}
\def\emX{{X}}
\def\emY{{Y}}
\def\emZ{{Z}}
\def\emSigma{{\Sigma}}

\newcommand{\etens}[1]{\mathsfit{#1}}
\def\etLambda{{\etens{\Lambda}}}
\def\etA{{\etens{A}}}
\def\etB{{\etens{B}}}
\def\etC{{\etens{C}}}
\def\etD{{\etens{D}}}
\def\etE{{\etens{E}}}
\def\etF{{\etens{F}}}
\def\etG{{\etens{G}}}
\def\etH{{\etens{H}}}
\def\etI{{\etens{I}}}
\def\etJ{{\etens{J}}}
\def\etK{{\etens{K}}}
\def\etL{{\etens{L}}}
\def\etM{{\etens{M}}}
\def\etN{{\etens{N}}}
\def\etO{{\etens{O}}}
\def\etP{{\etens{P}}}
\def\etQ{{\etens{Q}}}
\def\etR{{\etens{R}}}
\def\etS{{\etens{S}}}
\def\etT{{\etens{T}}}
\def\etU{{\etens{U}}}
\def\etV{{\etens{V}}}
\def\etW{{\etens{W}}}
\def\etX{{\etens{X}}}
\def\etY{{\etens{Y}}}
\def\etZ{{\etens{Z}}}

\newcommand{\pdata}{p_{\rm{data}}}
\newcommand{\ptrain}{\hat{p}_{\rm{data}}}
\newcommand{\Ptrain}{\hat{P}_{\rm{data}}}
\newcommand{\pmodel}{p_{\rm{model}}}
\newcommand{\Pmodel}{P_{\rm{model}}}
\newcommand{\ptildemodel}{\tilde{p}_{\rm{model}}}
\newcommand{\pencode}{p_{\rm{encoder}}}
\newcommand{\pdecode}{p_{\rm{decoder}}}
\newcommand{\precons}{p_{\rm{reconstruct}}}

\newcommand{\laplace}{\mathrm{Laplace}} 

\newcommand{\E}{\mathbb{E}}
\newcommand{\Ls}{\mathcal{L}}
\newcommand{\R}{\mathbb{R}}
\newcommand{\emp}{\tilde{p}}
\newcommand{\lr}{\alpha}
\newcommand{\reg}{\lambda}
\newcommand{\rect}{\mathrm{rectifier}}
\newcommand{\softmax}{\mathrm{softmax}}
\newcommand{\sigmoid}{\sigma}
\newcommand{\softplus}{\zeta}
\newcommand{\KL}{D_{\mathrm{KL}}}
\newcommand{\Var}{\mathrm{Var}}
\newcommand{\standarderror}{\mathrm{SE}}
\newcommand{\Cov}{\mathrm{Cov}}
\newcommand{\normlzero}{L^0}
\newcommand{\normlone}{L^1}
\newcommand{\normltwo}{L^2}
\newcommand{\normlp}{L^p}
\newcommand{\normmax}{L^\infty}

\newcommand{\parents}{Pa} 

\DeclareMathOperator*{\argmax}{arg\,max}
\DeclareMathOperator*{\argmin}{arg\,min}

\DeclareMathOperator{\sign}{sign}
\DeclareMathOperator{\Tr}{Tr}
\let\ab\allowbreak
 
\usepackage{hyperref}
\usepackage{url}

\usepackage{booktabs}
\usepackage{graphicx}
\usepackage{xcolor}
\usepackage{cleveref}
\usepackage{multirow}
\usepackage{enumerate}
\usepackage{amsthm}
\usepackage{floatrow}
\usepackage{float}
\usepackage{bbm}
\usepackage{caption}
\newfloatcommand{capbtabbox}{table}[][\FBwidth]

\iclrfinaltrue
\usepackage[bottom]{footmisc}

\usepackage{todonotes}
\setlength{\marginparwidth}{3.4cm}

\definecolor{red}{HTML}{e74c3c}
\definecolor{blue}{HTML}{3498db}
\definecolor{green}{HTML}{2ecc71}

\DeclareMathOperator*{\concat}{concat}
\newtheorem{theorem}{Theorem}
\newtheorem{lemma}{Lemma}
\newtheorem{remark}{Remark}

\def\vdelta{{\bm{\delta}}}
\def\vDelta{{\bm{\Delta}}}
\newcommand{\sDelta}{\Delta\!\!\!\!\Delta}

\def\qry{{\!\textit{qry}}}
\def\key{{\!\textit{key}}}
\def\val{{\!\textit{val}}}
\def\out{{\textit{out}}}




\title{On the Relationship between Self-Attention and Convolutional Layers}


\author{Jean-Baptiste Cordonnier, Andreas Loukas \& Martin Jaggi\\
{\'E}cole Polytechnique F\'ed\'erale de Lausanne (EPFL)\\
\texttt{\{first.last\}@epfl.ch}
}


\newcommand{\fix}{\marginpar{FIX}}
\newcommand{\new}{\marginpar{NEW}}

\newcommand{\githuburl}{\href{https://github.com/epfml/attention-cnn}{\tt github.com/epfml/attention-cnn}}
\newcommand{\interactiveurl}{\href{https://epfml.github.io/attention-cnn}{\tt epfml.github.io/attention-cnn}}


\begin{document}


\maketitle

\begin{abstract}
Recent trends of incorporating attention mechanisms in vision have led researchers to reconsider the supremacy of convolutional layers as a primary building block.
Beyond helping CNNs to handle long-range dependencies,
\cite{ramachandran2019standaloneselfattention} showed that attention can completely replace convolution and achieve state-of-the-art performance on vision tasks. This raises the question: do learned attention layers operate similarly to convolutional layers?
This work provides evidence that attention layers can perform convolution and, indeed, they often learn to do so in practice.
Specifically, we prove that a multi-head self-attention layer with sufficient number of heads is at least as expressive as any convolutional layer. Our numerical experiments then show that self-attention layers attend to pixel-grid patterns similarly to CNN layers, corroborating our analysis. Our code is publicly available\footnote{Code: \githuburl{}. Website: \interactiveurl{}.}.
\end{abstract}

\section{Introduction}
\vspace{-.3mm}

Recent advances in Natural Language Processing (NLP) are largely attributed to the rise of the \textit{transformer}~\citep{vaswani17attentionisallyouneed}.
Pre-trained to solve an unsupervised task on large corpora of text, transformer-based architectures, such as GPT-2 \citep{radford2018gpt2}, BERT \citep{devlin2018bert} and Transformer-XL \citep{dai2019transformerxl}, seem to possess the capacity to learn the underlying structure of text and, as a consequence, to learn representations that generalize across tasks.
The key difference between transformers and previous methods, such as recurrent neural networks \citep{lstm97} and convolutional neural networks (CNN), is that the former can simultaneously attend to every word of their input sequence.
This is made possible thanks to the \textit{attention mechanism}---originally introduced in Neural Machine Translation to better handle long-range dependencies~\citep{Bahdanau2015attention}.
With self-attention in particular, the similarity of two words in a sequence is captured by an attention score measuring the distance of their representations. The representation of each word is then updated based on those words whose attention score is highest.




Inspired by its capacity to learn meaningful inter-dependencies between words, researchers have recently considered utilizing self-attention in vision tasks.
Self-attention was first added to CNN by either using channel-based attention \citep{DBLP:conf/cvpr/HuSS18} or non-local relationships across the image \citep{nonlocal2018wang}.
More recently, \citet{belloAttentionAugmentedConvolutional2019} augmented CNNs by replacing some convolutional layers with self-attention layers, leading to improvements on image classification and object detection tasks.
Interestingly, \cite{ramachandran2019standaloneselfattention} noticed that, even though state-of-the art results are reached when attention and convolutional features are combined, under same computation and model size constraints, self-attention-\emph{only} architectures also reach competitive image classification accuracy.


\textit{These findings raise the question, do self-attention layers process images in a similar manner to convolutional layers?}
From a theoretical perspective, one could argue that transfomers have the capacity to simulate any function---including a CNN. Indeed, \cite{perez2019turingcomplete} showed that a multi-layer attention-based architecture with additive positional encodings is Turing complete under some strong theoretical assumptions, such as unbounded precision arithmetic.
Unfortunately, universality results do not reveal how a machine solves a task, only that it has the capacity to do so. Thus, the question of how self-attention layers actually process images remains open.


\paragraph{Contributions.}
In this work, we put forth theoretical and empirical evidence that self-attention layers can (and do) learn to behave similar to convolutional layers:
\vspace{-2mm}
\begin{itemize}
\item[I.] From a theoretical perspective, we provide a constructive proof showing that self-attention layers can express any convolutional layers.
\end{itemize}
\vspace{-2mm}
Specifically, we show that a single multi-head self-attention layer using relative positional encoding can be re-parametrized to express any convolutional layer. \vspace{-2mm}
\begin{itemize}
\item[II.] Our experiments show that the first few layers of attention-only architectures~\citep{ramachandran2019standaloneselfattention} do learn to attend on grid-like pattern around each query pixel, similar to our theoretical construction.
\end{itemize}
Strikingly, this behavior is confirmed both for our quadratic encoding, but also for relative encoding that is learned.
Our results seem to suggest that localized convolution is the right inductive bias for the first few layers of an image classifying network.
We provide an interactive website\footnote{\ificlrfinal \interactiveurl{} \else URL available after deanonymization, preview at \url{https://drive.google.com/file/d/1METSetroUA2qd2slol9wt7YxucJslAmF/} \fi} to explore how self-attention exploits localized position-based attention in lower layers and content-based attention in deeper layers.
For reproducibility purposes, our code is publicly available. 

\section{Background on Attention Mechanisms for Vision}
\label{sec:background}

We here recall the mathematical formulation of self-attention layers and emphasize the role of positional encodings.








\subsection{The Multi-Head Self-Attention Layer}
\label{ssec:background_self_attention}

Let  be an input matrix consisting of  tokens in of  dimensions each.
While in NLP each token corresponds to a word in a sentence, the same formalism can be applied to any sequence of  discrete objects, e.g. pixels. A self-attention layer maps any query token  from  to  dimensions as follows:

where we refer to the elements of the  matrix

as \textit{attention scores} and the softmax output\footnote{} as \textit{attention probabilities}.
The layer is parametrized by a query matrix , a key matrix  and a value matrix .For simplicity, we exclude any residual connections, batch normalization and constant factors.

A key property of the self-attention model described above is that it is equivariant to reordering, that is, it gives the same output independently of how the  input tokens are shuffled.
This is problematic for cases we expect the order of things to matter.
To alleviate the limitation, a \emph{positional encoding} is learned for each token in the sequence (or pixel in an image), and added to the representation of the token itself before applying self-attention

where  contains the embedding vectors for each position. More generally,  may be substituted by any function that returns a vector representation of the position.

It has been found beneficial in practice to replicate this self-attention mechanism into \emph{multiple heads}, each being able to focus on different parts of the input by using different query, key and value matrices.
In multi-head self-attention, the output of the  heads of output dimension  are concatenated and projected to dimension  as follows:

and two new parameters are introduced: the projection matrix  and a bias term .




\subsection{Attention for Images}

Convolutional layers are the \textit{de facto} choice for building neural networks that operate on images. We recall that, given an image tensor  of width , height  and  channels, the output of a convolutional layer for pixel  is given by

where  is the  weight tensor
\footnote{To simplify notation, we index the first two dimensions of the tensor from  to .},  is the bias vector and the set

contains all possible shifts appearing when convolving the image with a  kernel.





In the following, we review how self-attention can be adapted from 1D sequences to images.

With images, rather than tokens, we have query and key pixels . Accordingly, the input is a tensor  of dimension  and each
attention score associates a query and a key pixel. 

To keep the formulas consistent with the 1D case, we abuse notation and slice tensors by using a 2D index vector: if , we write  and  to mean  and , respectively.
With this notation in place, the multi-head self attention layer output at pixel  can be expressed as follows:

and accordingly for the multi-head case.



\subsection{Positional Encoding for Images}
\label{ssec:relative_position_encoding}

There are two types of positional encoding that has been used in transformer-based architectures: the \textit{absolute} and \textit{relative} encoding (see also \Cref{tab:relwork_attention} in the Appendix).

With absolute encodings, a (fixed or learned) vector  is assigned to each pixel . The computation of the attention scores we saw in \cref{eq:att_coeff} can then be decomposed as follows:

where  and  correspond to the query and key pixels, respectively.

The relative positional encoding was introduced by \cite{dai2019transformerxl}. The main idea is to only consider the position difference between the query pixel (pixel we compute the representation of) and the key pixel (pixel we attend) instead of the absolute position of the key pixel:

In this manner, the attention scores only depend on the shift .
Above, the learnable vectors  and  are unique for each head, whereas for every shift  the relative positional encoding  is shared by all layers and heads.
Moreover, now the key weights are split into two types:  pertain to the input and  to the relative position of pixels.




\section{Self-Attention as a Convolutional Layer}
\label{sec:attention_can_implement_cnn}

This section derives sufficient conditions such that a multi-head self-attention layer can simulate a convolutional layer.
Our main result is the following:


\begin{theorem}
  A multi-head self-attention layer with  heads of dimension , output dimension~ and a relative positional encoding of dimension 
  can express any convolutional layer of kernel size 
  and  output channels.
  \label{thm:the_theorem}
\end{theorem}


The theorem is proven constructively by selecting the parameters of the multi-head self-attention layer so that the latter acts like a convolutional layer.
In the proposed construction, the attention scores of each self-attention head should attend to a different relative shift within the set  of all pixel shifts in a  kernel. The exact condition can be found in the statement of Lemma~\ref{lemma:1}.

Then, Lemma~\ref{lemma:2} shows that the aforementioned condition is satisfied for the relative positional encoding that we refer to as the \textit{quadratic encoding}:

The learned parameters  and  determine the center and width of attention of each head, respectively. On the other hand,  is fixed and expresses the relative shift between query and key pixels.

It is important to stress that the above encoding is not the only one for which the
conditions of Lemma~\ref{lemma:1} are satisfied. In fact, in our experiments, the relative encoding learned by the neural network also matched the conditions of the lemma (despite being different from the quadratic encoding). Nevertheless, the encoding defined above is very efficient in terms of size, as only  dimensions suffice to encode the relative position of pixels, while also reaching similar or better empirical performance (than the learned one).

The theorem covers the general convolution operator as defined in \cref{eq:conv}.
However, machine learning practitioners using differential programming frameworks \citep{paszke2017automatic,tensorflow2015-whitepaper} might question if the theorem holds for all hyper-parameters of 2D convolutional layers:
\begin{itemize}
  \item \emph{Padding}: a multi-head self-attention layer uses by default the \texttt{"SAME"} padding while a convolutional layer would decrease the image size by  pixels. The correct way to alleviate these boundary effects is to pad the input image with  zeros on each side. In this case, the cropped output of a MHSA and a convolutional layer are the same.
  \item \emph{Stride}: a strided convolution can be seen as a convolution followed by a fixed pooling operation---with computational optimizations. \Cref{thm:the_theorem} is defined for stride 1, but a fixed pooling layer could be appended to  the Self-Attention layer to simulate any stride.
  \item \emph{Dilation}: a multi-head self-attention layer can express any dilated convolution as each head can attend a value at any pixel shift and form a (dilated) grid pattern.
\end{itemize}


\paragraph{Remark for the 1D case.} Convolutional layers acting on sequences are commonly used in the literature for text~\citep{kim-2014-convolutional}, as well as audio~\citep{oord2016wavenet} and time series~\citep{franceschi2019unsupervised}.
Theorem~\ref{thm:the_theorem} can be straightforwardly extended to show that multi-head self-attention with  heads can also simulate a 1D convolutional layer with a kernel of size  with  output channels using a positional encoding of dimension . Since we have not tested empirically if the preceding construction matches the behavior of 1D self-attention in practice, we cannot claim that it actually learns to convolve an input sequence---only that it has the capacity to do so.


\subsection*{Proof of Main Theorem}

\begin{figure}[t]
\centering
\vspace{-2mm}
  \includegraphics[width=1.01\linewidth]{figs/attention_cnn_v3}
  \caption{Illustration of a Multi-Head Self-Attention layer applied to a tensor image .
  Each head~ attends pixel values around shift  and learn a filter matrix .
  We show attention maps computed for a query pixel at position .\vspace{-3mm}}
  \label{fig:attention}
\end{figure}

The proof follows directly from Lemmas~\ref{lemma:1} and~\ref{lemma:2} stated below:



\begin{lemma}
Consider a multi-head self-attention layer consisting of  heads,  and let  be a bijective mapping of heads onto shifts. Further, suppose that for every head the following holds: 
Then, for any convolutional layer with a  kernel and  output channels, there exists  such that

for every .\label{lemma:1}
\end{lemma}

\begin{proof}
Our first step will be to rework the expression of the Multi-Head Self-Attention operator from  \eqref{eq:attention} and \eqref{eq:multi-head} such that the effect of the multiple heads becomes more transparent:

Note that each head's value matrix  and each block of the projection matrix  of dimension  are learned.
Assuming that , we can replace each pair of matrices by a learned matrix  for each head.
We consider one output pixel of the multi-head self-attention:

Due to the conditions of the Lemma, for the -th attention head the attention probability is one when

and zero otherwise.
The layer's output at pixel  is thus equal to

For , the above can be seen to be equivalent to a convolutional layer expressed in eq.~\ref{eq:conv}: there is a one to one mapping (implied by map ) between the matrices  for  and the matrices  for all 
\end{proof}

\paragraph{Remark about  and .}
It is frequent in transformer-based architectures to set , hence . In that case,  can be seen to be of rank , which does not suffice to express every convolutional layer with  channels. Nevertheless, it can be seen that any  out of  outputs of  can express the output of any convolutional layer with  output channels. To cover both cases, in the statement of the main theorem we assert that the output channels of the convolutional layer should be . In practice, we advise to concatenate heads of dimension  instead of splitting the  dimensions among heads to have exact re-parametrization and no ``unused'' channels.




\begin{lemma}
There exists a relative encoding scheme 
with  and parameters  with  such that, for every  there exists some vector  (conditioned on ) yielding
 if  and zero, otherwise.
\label{lemma:2}
\end{lemma}
\vspace{-1em}
\begin{proof}
We show by construction the existence of a  dimensional relative encoding scheme yielding the required attention probabilities.

As the attention probabilities are independent of the input tensor , we set  which leaves only the last term of \cref{eq:att_rel}.
Setting  to the identity matrix (with appropriate row padding), yields
 where .
Above, we have assumed that  such that no information from  is lost.

Now, suppose that we could write:

for some constant .
In the above expression, the maximum attention score over  is  and it is reached for  with . On the other hand, the  coefficient can be used to scale arbitrarily the difference between  and the other attention scores.

In this way, for , we have

and for , the equation becomes

exactly as needed to satisfy the lemma statement.

What remains is to prove that there exist  and  for which \cref{eq:iso_att_decomposed} holds.
Expanding the RHS of the equation, we have

Now if we set

and

then

which matches \cref{eq:iso_att_decomposed} with  and the proof is concluded.
\end{proof}

\paragraph{Remark on the magnitude of .}
The exact representation of one pixel requires  (or the matrices  and ) to be arbitrary large, despite the fact that the attention probabilities of all other pixels converge exponentially to 0 as  grows.
Nevertheless, practical implementations always rely on finite precision arithmetic for which a constant  suffices to satisfy our construction. For instance, since the smallest positive \texttt{float32} scalar is approximately , setting  would suffice to obtain hard attention.


\section{Experiments}
\label{sec:experiments}

The aim of this section is to validate the applicability of our theoretical results---which state that self-attention \emph{can} perform convolution---and to examine whether self-attention layers in practice do actually learn to operate like convolutional layers when trained on standard image classification tasks.
In particular, we study the relationship between self-attention and convolution with \textit{quadratic} and \textit{learned} relative positional encodings. We find that, for both cases, the attention probabilities learned tend to respect the conditions of Lemma~\ref{lemma:1}, supporting our hypothesis.




\subsection{Implementation Details}
\label{ssec:experiment_setup}

We study a fully attentional model consisting of six multi-head self-attention layers.
As it has already been shown by \cite{belloAttentionAugmentedConvolutional2019} that
combining attention features with convolutional features improves performance on Cifar-100 and ImageNet, we do not focus on attaining state-of-the-art performance. Nevertheless, to validate that our model learns a meaningful classifier, we compare it to the standard ResNet18 \citep{He2015resnet} on the CIFAR-10 dataset \citep{cifar10}.
In all experiments, we use a  invertible down-sampling \citep{jacobsen2018irevnet} on the input to reduce the size of the image. As the size of the attention coefficient tensors (stored during forward) scales quadratically with the size of the input image, \emph{full} attention cannot be applied to bigger images.  The fixed size representation of the input image is computed as the average pooling of the last layer representations and given to a linear classifier.

We used the PyTorch library \citep{paszke2017automatic} and based our implementation on PyTorch Transformers\footnote{\href{https://github.com/huggingface/pytorch-transformers}{\tt github.com/huggingface/pytorch-transformers}}.
We release our code on Github\footnote{\ificlrfinal \githuburl{} \else URL available after deanonymization.\fi} and hyper-parameters are listed in \Cref{tab:hyper-parameter} (Appendix).



\begin{figure}
\begin{floatrow}\ffigbox{\includegraphics[width=0.95\linewidth]{plots/learning_curve/learning_curve_small.pdf}\vspace{-1em}
}{\caption{Test accuracy on CIFAR-10.\vspace{-1em}}\label{fig:learning_curve}}
\capbtabbox{\resizebox{1\linewidth}{!}{\begin{tabular}{lrll}
\toprule
 Models                    &   accuracy & \# of params   & \# of FLOPS   \\
\midrule
 ResNet18                  &      0.938 & 11.2M         & 1.1B         \\
 SA quadratic emb.         &      0.938 & 12.1M         & 6.2B         \\
 SA learned emb.           &      0.918 & 12.3M         & 6.2B         \\
 SA learned emb. + content &      0.871 & 29.5M         & 15B          \\
\bottomrule
\end{tabular}}}{\vspace{3em}\caption{Test accuracy on CIFAR-10 and model sizes. SA stands for Self-Attention.\vspace{-1em}}\label{tab:parameter_size}}
\end{floatrow}
\end{figure}

\vspace{-2mm}
\paragraph{Remark on accuracy.}
To verify that our self-attention models perform reasonably well, we display in \Cref{fig:learned_attention_map_data} the evolution of the test accuracy on CIFAR-10 over the 300 epochs of training for our self-attention models against a small ResNet (\Cref{tab:parameter_size}).
The ResNet is faster to converge, but we cannot ascertain whether this corresponds to an inherent property of the architecture or an artifact of the adopted optimization procedures.
Our implementation could be optimized to exploit the locality of Gaussian attention probabilities and reduce significantly the number of FLOPS.
We observed that learned embeddings with content-based attention were harder to train probably due to their increased number of parameters.
We believe that the performance gap can be bridged to match the ResNet performance, but this is not the focus of this work.

\vspace{-2mm}



\subsection{Quadratic Encoding}
\label{ssec:verifying_theory}
\vspace{-2mm}
As a first step, we aim to verify that, with the relative position encoding introduced in \eqref{eq:quadposembedding}, attention layers learn to behave like convolutional layers.
We train nine attention heads at each layer to be on par with the  kernels used predominantly by the ResNet architecture.
The center of attention of each head  is initialized to .

\Cref{fig:iso_during_training} shows how the initial positions of the heads (different colors) at layer 4 changed during training.
We can see that after optimization, the heads attend on specific pixel of the image forming a grid around the query pixel.
Our intuition that Self-Attention applied to images learns convolutional filters around the queried pixel is confirmed.


\begin{figure}
  \includegraphics[width=.9\linewidth]{plots/epochs_iso_layer_4_small.png}
  \caption{Centers of attention of each attention head (different colors) at layer 4 during the training with quadratic relative positional encoding.
    The central black square is the query pixel, whereas solid and dotted circles represent the 50\% and 90\% percentiles of each Gaussian, respectively.\vspace{-3mm}}
  \label{fig:iso_during_training}
\end{figure}


\Cref{fig:iso_attention_final} displays all attention head at each layer of the model at the end of the training.
It can be seen that in the first few layers the heads tend to focus on local patterns (layers 1 and 2), while deeper layers (layers 3-6) also attend to larger patterns by positioning the center of attention further from the queried pixel position.
We also include in the Appendix a plot of the attention positions for a higher number of heads ().
\Cref{fig:iso_many_heads} displays both local patterns similar to CNN and long range dependencies.
Interestingly, attention heads do not overlap and seem to take an arrangement maximizing the coverage of the input space.


\begin{figure}
  \includegraphics[width=1\linewidth]{plots/final_iso_small.png}
  \caption{Centers of attention of each attention head (different colors) for the 6 self-attention layers using quadratic positional encoding.
  The central black square is the query pixel, whereas solid and dotted circles represent the 50\% and 90\% percentiles of each Gaussian, respectively.\vspace{-1.5mm}}
  \label{fig:iso_attention_final}
\end{figure}



\subsection{Learned Relative Positional Encoding}


We move on to study the positional encoding used in practice by fully-attentional models on images.


We implemented the 2D relative positional encoding scheme used by \citep{ramachandran2019standaloneselfattention,belloAttentionAugmentedConvolutional2019}:
we learn a  position encoding vector for each row and each column pixel shift.
Hence, the relative positional encoding of a key pixel at position  with a query pixel at position  is the concatenation of the row shift embedding  and the column shift embedding  (where ).
We chose  in the experiment.
We differ from their (unpublished) implementation in the following points:
(\emph{i}) we do not use convolution stem and ResNet bottlenecks for downsampling, but only a  invertible downsampling layer \citep{jacobsen2018irevnet} at input,
(\emph{ii}) we use  instead of  backed by our theory that the effective number of learned filters is .



At first, we discard the input data and compute the attention scores solely as the last term of eq.~(\ref{eq:att_rel}).
The attention probabilities of each head at each layer are displayed on \Cref{fig:learned_attention_map}.
The figure confirms our hypothesis for the first two layers and partially for the third: even when left to learn the positional encoding scheme from randomly initialized vectors, certain self-attention heads (depicted on the left) learn to attend to individual pixels, closely matching the condition of Lemma 1 and thus Theorem 1.
At the same time, other heads pay attention to horizontally-symmetric but non-localized patterns, as well as to long-range pixel inter-dependencies.



\begin{figure}
\RawFloats
\centering
  \includegraphics[width=.95\linewidth]{plots/learned_attention_maps_small.png}
  \caption{Attention probabilities of each head (\emph{column}) at each layer (\emph{row}) using learned relative positional encoding without content-based attention. The central black square is the query pixel. We reordered the heads for visualization and zoomed on the 7x7 pixels around the query pixel.
  \vspace{-3mm}}
  \label{fig:learned_attention_map}

\vspace*{\floatsep}
\vspace*{\floatsep}
\vspace*{\floatsep}

  \includegraphics[width=.95\linewidth]{plots/average_attention_with_content.pdf}
  \caption{Attention probabilities for a model with 6 layers (\emph{rows}) and 9 heads (\emph{columns}) using learned relative positional encoding and content-content based attention.
  Attention maps are averaged over 100 test images to display head behavior and remove the dependence on the input content.
  The black square is the query pixel.
  More examples are presented in Appendix~\ref{ssec:appendix_content}.
  \vspace{-5mm}}
  \label{fig:learned_attention_map_data}
\end{figure}



We move on to a more realistic setting where the attention scores are computed using both positional and content-based attention (i.e.,  in \citep{ramachandran2019standaloneselfattention}) which corresponds to a full-blown standalone self-attention model.

The attention probabilities of each head at each layer are displayed in \Cref{fig:learned_attention_map_data}.
We average the attention probabilities over a batch of 100 test images to outline the focus of each head and remove the dependency on the input image.
Our hypothesis is confirmed for some heads of layer 2 and~3: even when left to learn the encoding from the data, certain self-attention heads only exploit position-based attention to attend to distinct pixels at a fixed shift from the query pixel reproducing the receptive field of a convolutional kernel.
Other heads use more content-based attention (see \Cref{fig:extra_data_1,fig:extra_data_2,fig:extra_data_3} in Appendix for non-averaged probabilities) leveraging the advantage of Self-Attention over CNN which does not contradict our theory.
In practice, it was shown by \cite{belloAttentionAugmentedConvolutional2019} that combining CNN and self-attention features outperforms each taken separately.
Our experiments shows that such combination is learned when optimizing an unconstrained fully-attentional model.





The similarity between convolution and multi-head self-attention is striking when the query pixel is slid over the image:
the localized attention patterns visible in \Cref{fig:learned_attention_map_data} follow the query pixel.
This characteristic behavior materializes when comparing \Cref{fig:learned_attention_map_data} with the attention probabilities at a different query pixel (see \Cref{fig:average_attention_3_3} in Appendix).
Attention patterns in layers 2 and 3 are not only localized but stand at a constant shift from the query pixel, similarly to convolving the receptive field of a convolutional kernel over an image.
This phenomenon is made evident on our interactive website\footnote{\ificlrfinal \interactiveurl{} \else URL available after deanonymization, preview at \url{https://drive.google.com/file/d/1METSetroUA2qd2slol9wt7YxucJslAmF/} \fi}.
This tool is designed to explore different components of attention for diverse images with or without content-based attention.
We believe that it is a useful instrument to further understand how MHSA learns to process images.

\section{Related Work}

In this section, we review the known differences and similarities between CNNs and transformers.



The use of CNN networks for text---at word level \citep{DBLP:journals/corr/GehringAGYD17} or character level \citep{kim-2014-convolutional}---is more seldom than transformers (or RNN).
Transformers and convolutional models have been extensively compared empirically on tasks of Natural Language Processing and Neural Machine Translation.
It was observed that transformers have a competitive advantage over convolutional model applied to text \citep{vaswani17attentionisallyouneed}.
It is only recently that \cite{belloAttentionAugmentedConvolutional2019,ramachandran2019standaloneselfattention} used transformers on images and showed that they achieve similar accuracy as ResNets.
However, their comparison only covers performance and number of parameters and FLOPS but not expressive power.



Beyond performance and computational-cost comparisons of transformers and CNN, the study of expressiveness of these architectures has focused on their ability to capture long-term dependencies \citep{dai2019transformerxl}.
Another interesting line of research has demonstrated that transformers are Turing-complete \citep{universalTransformers,perez2019turingcomplete}, which is an important theoretical result but is not informative for practitioners.
To the best of our knowledge, we are the first to show that the class of functions expressed by a layer of self-attention encloses all convolutional filters.



The closest work in bridging the gap between attention and convolution is due to \cite{outerproduct2019andreoli}.
They cast attention and convolution into a unified framework leveraging tensor outer-product.
In this framework, the receptive field of a convolution is represented by a ``basis'' tensor .
For instance, the receptive field of a classical  convolutional kernel would be encoded by  for .
The author distinguishes this \emph{index-based} convolution with \emph{content-based} convolution where  is computed from the value of the input, e.g., using a key/query dot-product attention.
Our work moves further and presents sufficient conditions for relative positional encoding injected into the input content (as done in practice) to allow \emph{content-based} convolution to express any \emph{index-based} convolution.
We further show experimentally that such behavior is learned in practice.





\section{Conclusion}
\label{sec:discussion}

We showed that self-attention layers applied to images can express any convolutional layer (given sufficiently many heads) and
that fully-attentional models learn to combine local behavior (similar to convolution) and global attention based on input content.
More generally, fully-attentional models seem to learn a generalization of CNNs where the kernel pattern is learned at the same time as the filters---similar to deformable convolutions \citep{dai2017deformable,Zampieri2019masterthesis}.
Interesting directions for future work include translating existing insights from the rich CNNs literature back to transformers on various data modalities, including images, text and time series.

\subsubsection*{Acknowledgments}
Jean-Baptiste Cordonnier is thankful to the Swiss Data Science Center (SDSC) for funding this work.
Andreas Loukas was supported by the Swiss National Science Foundation (project “Deep Learning for Graph Structured Data”, grant number PZ00P2 179981).

\newpage

\bibliography{bibliography}
\bibliographystyle{iclr2020_conference}

\newpage

\appendix

{\LARGE \textsc{Appendix}}





\section{More Examples with Content-based Attention}
\label{ssec:appendix_content}

We present more examples of attention probabilities computed by self-attention model. \Cref{fig:average_attention_3_3} shows average attention at a different query pixel than \Cref{fig:learned_attention_map_data}. \Cref{fig:extra_data_3,fig:extra_data_2,fig:extra_data_1} display attention for single images.
\vspace{-1em}
\begin{figure}[H]
\centering
  \includegraphics[width=0.8\linewidth]{plots/average_attention_with_content_3.pdf}
  \vspace{-.5em}
  \caption{Attention probabilities for a model with 6 layers (\emph{rows}) and 9 heads (\emph{columns}) using learned relative positional encoding and content-content attention.
  We present the average of 100 test images.
  The black square is the query pixel.}
  \label{fig:average_attention_3_3}
\end{figure}


\begin{figure}[H]
\centering
  \vspace{-1em}
  \includegraphics[width=0.8\linewidth]{plots/attention_maps_learned_data/attention_qk_qr_i19_x10_y4.pdf}
  \vspace{-.5em}
  \caption{Attention probabilities for a model with 6 layers (\emph{rows}) and 9 heads (\emph{columns}) using learned relative positional encoding and content-content based attention. The query pixel (black square) is on the frog head.}
  \label{fig:extra_data_3}
\end{figure}

\begin{figure}[H]
\centering
  \includegraphics[width=0.8\linewidth]{plots/attention_maps_learned_data/attention_qk_qr_i56_x2_y5.pdf}
  \vspace{-.5em}
  \caption{Attention probabilities for a model with 6 layers (\emph{rows}) and 9 heads (\emph{columns}) using learned relative positional encoding and content-content based attention. The query pixel (black square) is on the horse head.}
  \label{fig:extra_data_1}
\end{figure}


\begin{figure}[H]
\centering
  \includegraphics[width=0.8\linewidth]{plots/attention_maps_learned_data/attention_qk_qr_i56_x12_y1.pdf}
  \vspace{-.5em}
  \caption{Attention probabilities for a model with 6 layers (\emph{rows}) and 9 heads (\emph{columns}) using learned relative positional encoding and content-content based attention. The query pixel (black square) is on the building in the background.}
  \label{fig:extra_data_2}
\end{figure}

\newpage


\section{Hyper-parameters used in our Experiments}

\begin{table}[h!]
  \centering
  \begin{tabular}{lc}
    \toprule
    Hyper-parameters&\\
    \midrule
    number of layers              & 6\\
    number of heads               & 9\\
    hidden dimension              & 400\\
    intermediate dimension        & 512\\
    invertible pooling width      & 2\\
    dropout probability           & 0.1\\
    layer normalization epsilon   & \\
    \midrule
    number of epochs              & 300\\
    batch size                    & 100\\
    learning rate                 & 0.1\\
    weight decay                  & 0.0001\\
    momentum                      & 0.9\\
    cosine decay                  & \checkmark \\
    linear warm up ratio          & 0.05\\
    \bottomrule
  \end{tabular}
  \caption{Self-attention network parameters}
  \label{tab:hyper-parameter}
\end{table}


\section{Positional Encoding References}

\begin{table}[h!]
  \centering
  \begin{tabular}{lcccc}
    \toprule
    \multirow{2}{*}{Model}&\multicolumn{3}{c}{type of positional encoding}&\multirow{2}{*}{relative}\\
    \cmidrule(r){2-4}
    &sinusoids&learned&quadratic\\
    \midrule
    \cite{vaswani17attentionisallyouneed} & \checkmark\\
    \cite{radford2018gpt2} & & \checkmark\\
    \cite{devlin2018bert} & & \checkmark\\
    \cite{dai2019transformerxl} & \checkmark & & & \checkmark \\
    \cite{yang2019xlnet}  & \checkmark & & & \checkmark \\
    \midrule
    \cite{belloAttentionAugmentedConvolutional2019} & & \checkmark && \checkmark \\
    \cite{ramachandran2019standaloneselfattention} & & \checkmark && \checkmark \\
    Our work & & \checkmark & \checkmark & \checkmark \\
    \bottomrule
  \end{tabular}
  \caption{Types of positional encoding used by transformers models applied to text (\emph{top}) and images (\emph{bottom}).
  When multiple encoding types have been tried, we report the one advised by the authors.}
  \label{tab:relwork_attention}
\end{table}



\section{Generalized \Cref{lemma:1}}

We present a generalization of \Cref{lemma:1} that replaces the necessity of hard attention (to single pixels) by a milder assumption: the attention probabilities should span the grid receptive field. The conditions of this Lemma are still satisfied by \Cref{lemma:2}, hence \Cref{thm:the_theorem} follows.


\begin{lemma}
Consider a multi-head self-attention layer consisting of  heads,  and let  be a pixel indexing.
Then, for any convolutional layer with a  kernel and  output channels, there exists  and  such that

for every 
if and only if, for all ,
\footnote{the vectorization operator  flattens a matrix into a vector}

\label{lemma:1_generalized}
\end{lemma}

\begin{proof}
Our first step will be to rework the expression of the Multi-Head Self-Attention operator from  \eqref{eq:attention} and \eqref{eq:multi-head} such that the effect of the multiple heads becomes more transparent:

Note that each head's value matrix  and each block of the projection matrix  of dimension  are learned.
Assuming that , we can replace each pair of matrices by a learned matrix  for each head.
We consider one output pixel of the multi-head self-attention and drop the bias term for simplicity:

with .
We rewrite the output of a convolution at pixel  in the same manner:

Equality between equations~(\ref{eq:mhsa})~and~(\ref{eq:conv}) holds for any input  if and only if the linear transformations for each pair of key/query pixels are equal, i.e. .
We vectorize the weight matrices into matrices of dimension  as  and .
Hence, to show that  for all , we must show that  for all .

\begin{figure}
  \centering
  \includegraphics[width=1.\linewidth]{figs/generalization.pdf}
  \caption{Factorization of the vectorized weight matrices  and  used to compute the output at position  for an input image of dimension . On the \emph{left}: a convolution of kernel , on the \emph{right}: a self-attention with  heads. ,  in both cases.}
  \label{fig:generalization}
\end{figure}


The matrix  has a restricted support: only the columns associated with a pixel shift  in the receptive field of pixel  can be non-zero.
This leads to the factorization  displayed in Figure~\ref{fig:generalization} where  and . Given an ordering of the shifts  indexed by , set  and .
On the other hand, we decompose

  with
   
   and
    .


The proof is concluded by showing that  is a necessary and sufficient condition
for the existence of a  such that any  can be written as .

\textbf{Sufficient.}
Given that , there exists  such that  and a valid decomposition is  which gives .

\textbf{Necessary.}
Assume there exists  such that  and 
and set  to be a row of .
Then,  for any  and there is no possible decomposition.


\end{proof}




\section{Generalized Quadratic Positional Encoding}
\label{ssec:non_isotropic}

We noticed the similarity of the attention probabilities in the quadratic positional encoding (\Cref{sec:attention_can_implement_cnn}) to isotropic bivariate Gaussian distributions with bounded support:

Building on this observation, we further extended our attention mechanism to non-isotropic Gaussian distribution over pixel positions.
Each head is parametrized by a center of attention  and a covariance matrix  to obtain the following attention scores,

where, once more, .
The last term can be discarded because the softmax is shift invariant and we rewrite the attention coefficient as a dot product between the head target vector  and the relative position encoding  (consisting of the first and second order combinations of the shift in pixels ):


\paragraph{Evaluation.}
We trained our model using this generalized quadratic relative position encoding.
We were curious to see if, using the above encoding the self-attention model would learn to attend to non-isotropic groups of pixels---thus forming unseen patterns in CNNs.
Each head was parametrized by  and  to ensure that the covariance matrix remained positive semi-definite.
We initialized the center of attention to  and  so that initial attention probabilities were close to an isotropic Gaussian.
\Cref{fig:noniso_attention_final} shows that the network did learn non-isotropic attention probability patterns, especially in high layers.
Nevertheless, the fact that we do not obtain any performance improvement seems to suggest that attention non-isotropy is not particularly helpful in practice---the quadratic positional encoding suffices.

\begin{figure}[h]
  \includegraphics[width=1\linewidth]{plots/final_noniso.png}
  \caption{Centers of attention of each attention head (different colors) for the 6 self-attention layers using non-isotropic Gaussian parametrization.
    The central black square is the query pixel, whereas solid and dotted circles represent the 50\% and 90\% percentiles of each Gaussian, respectively.}
  \label{fig:noniso_attention_final}
\end{figure}



\paragraph{Pruning degenerated heads.}
Some non-isotropic attention heads attend on ``non-intuitive'' patches of pixels:
either attending a very thin stripe of pixels, when  was almost singular, or attending all pixels uniformly, when  was close to  (i.e. constant attention scores).
We asked ourselves, are such attention patterns indeed useful for the model or are these heads degenerated and unused?
To find out, we pruned all heads having largest eigen-values smaller than  or condition number (ratio of the biggest and smallest eigen-values) greater than .
Specifically in our model with 6-layer and 9-heads each, we pruned  heads from the first to the last layer.
This means that these layers cannot express a  kernel anymore.
As shown in yellow on \cref{fig:learning_curve}, this ablation initially hurts a bit the performance, probably due to off biases, but after a few epochs of continued training with a smaller learning rate (divided by 10) the accuracy recovers its unpruned value.
Hence, without sacrificing performance, we reduce the size of the parameters and the number of FLOPS by a fourth.

\begin{figure}\begin{floatrow}\ffigbox{\includegraphics[width=1\linewidth]{plots/learning_curve/learning_curve_small_appendix.pdf}}{\caption{Evolution of test accuracy on CIFAR-10.
  Pruned model (\emph{yellow}) is continued training of the non-isotropic model (\emph{orange}).}\label{fig:learning_curve_with_noniso}}
\capbtabbox{\resizebox{1\linewidth}{!}{\begin{tabular}{lrll}
\toprule
 Models                        &   accuracy & \# of params   & \# of FLOPS   \\
\midrule
 ResNet18                      &      0.938 & 11.2M         & 1.1B         \\
 SA quadratic emb.             &      0.938 & 12.1M         & 6.2B         \\
 SA quadratic emb. gen.        &      0.934 & 12.1M         & 6.2B         \\
 SA quadratic emb. gen. pruned &      0.934 & 9.7M          & 4.9B         \\
 SA learned emb.               &      0.918 & 12.3M         & 6.2B         \\
 SA learned emb. + content     &      0.871 & 29.5M         & 15B          \\
\bottomrule
\end{tabular}}}{\vspace{3em}\caption{Number of parameters and accuracy on CIFAR-10 per model. SA stands for Self-Attention.}\label{tab:parameter_size_with_noniso}}
\end{floatrow}
\end{figure}








\section{Increasing the Number of Heads}

For completeness, we also tested increasing the number of heads of our architecture from 9 to 16.

\begin{figure}[h]
  \includegraphics[width=1\linewidth]{plots/final_noniso_manyheads.png}
  \caption{Centers of attention for 16 attention heads (different colors) for the 6 self-attention layers using quadratic positional encoding.
    The central black square is the query pixel, whereas solid and dotted circles represent the 50\% and 90\% percentiles of each Gaussian, respectively.}
  \label{fig:iso_many_heads}
\end{figure}

Similar to Figure~\ref{fig:iso_attention_final}, we see that the network distinguishes two main types of attention patterns. Localized heads (i.e., those that attend to nearly individual pixels) appear  more frequently in the first few layers. The self-attention layer uses these heads to act in a manner similar to how convolutional layers do. Heads with less-localized attention become more common at higher layers.




\end{document}
