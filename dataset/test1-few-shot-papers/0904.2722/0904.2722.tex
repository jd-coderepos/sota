
In this section, we first introduce our model for evaluating the probability of a distributed denial of service attack (DDoS) caused by Byzantine nodes in a P2P network. We then present results for two distinct scenarios.

\subsection{Model}\label{sect:model}

We consider a directed graph with a set of nodes . A {\em source} node has a large file to be sent to \emph{receiver} nodes. The file is divided into  packets. To do so, the source connects to a subset of nodes, , chosen uniformly at random, and sends each of them a different random linear combination of the original file packets. To ensure that enough degrees of freedom exist in the network, . We refer to the nodes in  as {\em level-s} nodes. A {\em tracker} node keeps track of the list of \emph{informed nodes}, , \emph{i.e.}, nodes that keep an information packet.

For a receiver to retrieve the file, it connects to a subset of nodes , chosen uniformly at random, with . We refer to the nodes in  as \emph{level-r} nodes. Note that there may be an overlap between level-s and level-r. In each time slot, one of the uninformed level-r nodes, , contacts the tracker to retrieve a random list of  informed nodes, where . The node   then connects to these informed nodes through a secure overlay connection, retrieves their  packets, and stores a single random linear combination of these packets. During the same time slot, the tracker updates its list of informed nodes to . This process is repeated for all nodes in , and then all level-r nodes forward their stored packets to the receiver. In order to maximize the probability of storing linearly independent combinations in level-r nodes and ensure decodability at the receiver, we set . Although we assume that each node in level-s and level-r stores only one packet, the model can be easily generalized to account for higher numbers. An example of this network model is shown in Figure \ref{network_setup}.  Note that the tracker is considered to be a trusted party in our model -- in fact, as in the case of most P2P protocols, a dishonest tracker would yield a protocol failure with overwhelming probability.

\begin{figure}[tbp]
\begin{center}
\includegraphics[scale=0.5]{network_setup.eps}
\caption{Network model. The source is connected to the level-s nodes, and the receiver is connected to the level-r nodes. The dark nodes are the informed nodes. The level-r nodes take turns to contact the tracker, and connects to  level-s nodes based on the list returned by the tracker. Here, nodes  and  has completed this process, and the other level-r nodes have not.}
\label{network_setup}
\end{center}
\vspace*{-0.5cm}
\end{figure}

We define an \emph{Information Contact Graph}  to denote the evolving graph formed in the above process, where  is the list of informed nodes and  is the set of overlay links that connect the level-s and level-r nodes.
The probability that a node becomes a Byzantine attacker is . An attacker corrupts the packet it stores by generating arbitrary content while complying to the standard packet format. A node independently decides whether it becomes Byzantine at the start of the file dissemination process according to  and stays that way throughout the process. We define an indicator variable  which is 1 if node  is Byzantine and 0 otherwise. The tracker has no information about which nodes are Byzantine. A \emph{contaminated packet} is a packet that is either directly corrupted by an attacker, or is a linear combination that involves at least one contaminated packet. A \emph{contaminated node} is a node that stores a contaminated packet. The \emph{blocking probability}  is the probability that the receiver collects at least one contaminated packet, and thus, is unable to decode the file. This is equivalent to the probability that the attacker successfully carries out a DDoS attack.

\subsection{Analysis of Impact of Byzantine Attacks}

 We now evaluate the {\em blocking probability} at the receiver. We then consider the {\em expected number of contaminated nodes} at any given time. First, we introduce necessary definitions, as follows.
 We define an indicator variable  which is equal to  if node  is contaminated at time  and  otherwise.
  is a random variable for the number of contaminated nodes in , and  is the number of uncontaminated nodes. The function  denotes the hypergeometric distribution, in which
 
 Let  denote the number of informed Byzantine nodes at time , that is, the number of Byzantine nodes in .  has a binomial distribution with parameters .


We consider two scenarios. In \theorref{th:1}, for simplicity, we consider a static informed nodes list, in which the list kept by the tracker is fixed to . In this case, level-r nodes only connect to level-s nodes. Second, in \theorref{th:2}, we generalize to the case in which the tracker updates its list of informed nodes to , as stated in Section \ref{sect:model}.


\begin{theorem}[Static Informed Nodes List]\label{th:1}
Let  be an information contact graph in which nodes in  only connect to nodes in . Then its blocking probability  is given by:

\vspace{-0.3cm}


where
\vspace{-0.3cm}


\end{theorem}

\vspace{0.2cm}

\begin{proof}
We consider two disjoint subsets of :  the set of informed nodes at , that is, , and the uninformed nodes, that is, .
Let  be a random variable for the number of nodes in .  has a hypergeometric distribution, .

We first consider . Given  and , the probability that  is uncontaminated is equal to the probability that it is not initially Byzantine, which is equal to .
Then, the probability that all nodes in  are uncontaminated is:

\vspace{-0.6cm}


Now, at each timeslot , a node  becomes informed. For  to be uncontaminated, it must not be Byzantine and it must connect to  uncontaminated nodes. Then,

\vspace{-0.6cm}


It follows that the probability that all nodes in  are uncontaminated at time  is:

\vspace{-0.7cm}


Note that since  nodes are added, the information dissemination process ends at . Now, the probability that only uncontaminated nodes exist in  at time , conditioned on  and , is:

\vspace{-0.5cm}






 has a binomial distribution,  has a hypergeometric distribution and they are independent of each other. Taking out these two conditions, the probability that all nodes in  are uncontaminated is:

\vspace{-0.5cm}


It follows that the blocking probability is .
\end{proof}

We now consider that the list of informed nodes at the tracker is , that is, it is updated with each new informed level-r node.

\begin{theorem}[Evolving informed nodes list]\label{th:2}

Let  be an information contact graph in which  are to be added to the graph by connecting to nodes in . Then its blocking probability  is:

\vspace{-0.5cm}


where

\vspace{-1.1cm}


\vspace{0.05cm}

\end{theorem}

\begin{proof}
Recall from \theorref{th:1} that we consider two disjoint subsets of , that is,  and . As before,  is the number of nodes in .
Again, at time , the probability that all nodes in  are uncontaminated given  and  is .




We now consider the nodes in  and assume .
At each time step, there are  contaminated nodes and  uncontaminated nodes in .
The probability of obtaining a contaminated node at time  is only dependent on  and , and thus, we can model these probabilities by Markov chains  , in which  represents the set of states and  represents the matrix of transition probabilities. A state in  is represented by . Transitions from  are only possible to  and to .
It is also important to note that the depth of the Markov chain is equal to . The transition probabilities from  when adding a node  are  and .  is illustrated in ~\fig{fig:Markov} for .

\begin{figure}[tbp]
\centering
\includegraphics[width=12cm]{markov.eps}
\caption{ Markov diagram for the dissemination process, .  The transitions to the left (dotted arrows) represent the addition of an uncontaminated node, and the transitions to the right (filled arrows) represent the addition of a contaminated node. The grey states are considered in computing , that is, the states in which no contaminated nodes are added.}
\label{fig:Markov}
\vspace{-0.5cm}
\end{figure}


Let us denote  as  and , it follows that . Now let   denote the probability of being in state  at time .
 can be defined recursively as:
 \vspace{-0.1cm}
 

Now, consider that node  is active at time . The probability of  being uncontaminated is the probability that it is not Byzantine and does not connect to contaminated nodes. Thus,

\vspace{-0.75cm}

\VONEJSAC{

}

Now, notice that the probability of only having uncontaminated nodes at time  is the probability of, starting in state , ending in state  after  steps: in that case, no contaminated node is added to the network. The probability of this event, conditioned on  and , is

\vspace{-0.7cm}


Combining the results for sets  and , we have that the probability that no contaminated nodes exist in  given that  and  is given by

\vspace{-0.5cm}


Finally, it follows that the blocking probability at time  is

\vspace{-0.5cm}

\end{proof}

\vspace{-0.2cm}

The results from \emph{Theorems 1} and \emph{2} are illustrated in \fig{fig:theorems}. Note that even for a small , the blocking probability  is very high. Even for the case in \theorref{th:1},  grows exponentially. This is because it is sufficient for a single level-r node to connect to a Byzantine node in level-s to contaminate the receiver. Figure \ref{fig:theorems} indicates that  grows faster for the evolving informed node list than for the static informed node list). This is due to the fact that as more nodes are added to the network, the presence of contaminated nodes becomes more likely, and thus, the probability that a level-r node connects to at least one contaminated node increases. The probability  also increases with other parameters such as , , and  since they increase the probability of level-r nodes connecting to contaminated nodes.


From the above proofs, it follows that the number of contaminated nodes in , is dependent on the random variable . We now perform an analysis of the expected number of contaminated nodes in the network  conditioned on .

First, we consider the case of the static informed nodes list, conditioned on . It is clear that . Now, at each time step , one contaminated node is added to  with probability  and thus . It follows that


In the case of the evolving informed nodes list, since the states of  are representative of the number of contaminated nodes in the network,  has a direct correspondence to the expected state the Markov Chain is in after  time steps; therefore:

\vspace{-0.5cm}



In order to visualize these results, we take the expected value of  for the set of parameters chosen in \fig{fig:theorems}, which is equal to 1. Then, we plot  for the static and evolving informed node lists. It is shown in \fig{fig:exp} that the expected number of contaminated nodes in the static case is linear with time. For small probabilities , the  is higher for the evolving case; as  increases, the values for both cases become similar.





\begin{figure}[tbp]
\centering
\includegraphics[width=11cm]{blockProb_N30N_s5N_r6d_3_final.eps}
\caption{Blocking probability in function of  for , ,  and . The results for the static and evolving informed nodes list are shown in full and dashed, respectively.}
\label{fig:theorems}
\vspace{-0.4cm}
\end{figure}

\begin{figure}[tbp]
\centering
\includegraphics[width=11cm]{expNrCont_N30N_s5N_r6d_3Y_1_final.eps}
\caption{Expected number of contaminated nodes in function of time, for , , ,  and . The results for the static and evolving informed nodes list are shown in full and dashed, respectively.}
\label{fig:exp}
\vspace{-0.4cm}
\end{figure}

