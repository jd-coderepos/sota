\documentclass{llncs}
\usepackage[english]{babel}
\usepackage{graphicx}
\usepackage{amssymb, amsmath}

\parskip=4pt





\newtheorem{claimnum}{Claim}






\title{Edge-b-coloring Trees
\thanks{Partially supported by CNPq/Brazil.}
}
\author{Victor Campos \and Ana Silva}
\institute{ParGO Group - Parallellism, Graphs and Optimization\\Universidade Federal do Cear\'a, Fortaleza, Brazil\\\email{campos@lia.ufc.br,\ anasilva@mat.ufc.br}}

\newcommand{\C}{\mathcal{C}}

\begin{document}

\maketitle

\begin{abstract}
A b-coloring of the vertices of a graph is a proper coloring where each color class contains a vertex which is adjacent to at least one vertex in each other color class. The b-chromatic number of  is the maximum integer  for which  has a b-coloring with  colors. This problem was introduced by Irving and Manlove in 1999, where they showed that computing  is -hard in general and polynomial-time solvable for trees. Since then, a number of complexity results were shown, including NP-hardness results for chordal graphs (Havet et. al., 2011) and line graphs (Campos et. al., 2015). In this article, we present a polynomial time algorithm that solves the problem restricted to claw-free block graphs, an important subclass of chordal graphs and line graphs. This is equivalent to solving the edge coloring version of the problem restricted to trees.
\end{abstract}


\section{Introduction}

Let  be a simple graph\footnote{The graph terminology used in this paper follows \cite{BM08}.} 
and let  be a proper coloring of
. If a vertex  in a color class  of  has no neighbors in another color class , then another proper coloring of  can be obtained by simply changing the color of  to . Therefore, if this holds for every vertex of color class , then we can obtain a proper coloring that uses fewer colors.  Because finding the chromatic number of a graph is -hard, one cannot expect to always get a good result by iteratively applying this method.

On the basis of this idea, Irving and Manlove introduced the notion of
b-coloring \cite{Irving.Manlove.99}.  Intuitively, a b-coloring
is a proper coloring that cannot be improved by the described heuristic,
and the b-chromatic number  measures the worst possible such coloring. Finding 
 was proved to be -hard in general graphs \cite{Irving.Manlove.99}, 
and remains so even when restricted to bipartite graphs \cite{KRATOCHVIL.etal.02}, chordal
 graphs \cite{HLS.11}, or line graphs \cite{Campos.etal.15}.  In the same article, Irving and Manlove also introduced a simple upper bound for , defined as follows. The \emph{-degree of } is the maximum integer  for which there are at least 
 edges of degree at least ; we denote it by . 
Because the existence of a b-coloring with  colors forces the existence of  vertices of degree , we get: 

They then proved that  whenever  is a tree. Since then, it has been discovered that the graph needs only to be ``locally acyclic'' in order to have this property, in other words, it is known that  whenever  has girth at least~7 \cite{Campos.Lima.Silva.15}. A number of other results investigate graphs with high b-chromatic number when compared to ; for instance \cite{BMZ.09}, \cite{Cabello.Jakovac.11}, \cite{KM.02}, and \cite{KRATOCHVIL.etal.02}, among others. A natural question is whether this property carries on to the edge version of the problem, that is, whether the b-chromatic number of the line graph  of a tree is also at least . The answer to this question is ``yes'', if  is the line graph of a caterpillar \cite{Campos.etal.15}, but it is ``no'' for general trees, as can be seen in \cite{MS.12}. Nevertheless, here we show that deciding whether the b-chromatic number of  is at least  can be done in polynomial time, when  is the line graph of a tree. 
This, together with the result on caterpillars, are the only positive results on a subclass of chordal graphs, up to our knowledge. We believe that our algorithm can be adapted to work on any block graph. However, the generalization to bigger subclasses of chordal graphs, for instance interval graphs, poses much harder difficulties.

Our algorithm was inspired on a previous fixed-parameter algorithm for block graphs presented in~\cite{Silva.10} (we mention that the fixed-parameter decision problem is open for general graphs).
Consider  with cardinality , and, for each , consider a subset  of size . Let  be the graph obtained from  by turning each subset  into a clique; also, let  be the precoloring of  where each  is colored with a distinct color, and no other vertex is colored. Note that if  can be extended to the whole graph , then the same extension is a b-coloring of  having  as basis. In~\cite{Marx.07}, Marx proves that if  is a chordal graph, then deciding whether the precoloring can be extended to  can be done in polynomial time. Sampaio and Silva observed that if  is a block graph, then the obtained graph  is chordal. They presented their result separatedly in~\cite{HLS.11} and \cite{Silva.10}, and in \cite{Silva.10} it is observed that this leads to a fixed-parameter-algorithm for block graphs: it suffices to test whether the extension exists, for every possible subset , and for every family of subsets . 


Here, we consider line graphs of trees (which equals claw-free block graphs) and use a flow network representation of a b-coloring of  (this is similar to the one used by Marx in~\cite{Marx.07}); then, we use a dynamic programming algorithm to eliminate the need for an exponential number of tests. We believe that our algorithm can be made to work on general block graphs, and maybe even on larger classes, provided that the auxiliary graph constructed as in the previous paragraph is still chordal. However, we do not believe that this class is much larger; for instance, if  has a , (), in its neighborhood and , then already we get a  in the auxiliary graph. 

Before we start, we need some formal definitions. Let  be a graph. A -coloring of  is a function , and it is said to be \emph{proper} if , whenever . Consider  to be a proper -coloring of . The \emph{color class  (of )} is the subset of vertices . We say that vertex  \emph{realizes color } (or that \emph{color  is realized by }) if  and  is adjacent to at least one vertex of color class , for every . We say that a subset \emph{ realizes distinct colors} if each vertex of  realizes a color, and the set of realized colors equals . If  is such that every color is realized by some vertex, then we say that  is a \emph{b-coloring of }. Also, a subset containing exactly one vertex that realizes , for each color class , is called a \emph{basis of }. A vertex  is said to be \emph{-dense} if , and the set of all -dense vertices of  is denoted by . Clearly, if  is a b-coloring of  with  colors with basis , then . The \emph{b-chromatic number of }, denoted by , is the maximum integer  for which  has a b-coloring with  colors. 

A graph is called a \emph{block graph}  if its 2-connected components are cliques, and it is \emph{claw-free} if it does not have an induced subgraph isomorphic to  (the complete bipartite graph with parts of size 1 and 3, also known as ``claw''). It is well known, and not hard to verify, that the class of claw-free block graphs equals the class of line graphs of trees. In the remainder of the text, we refer only to claw-free block graphs, and consider the graph to be rooted at some block. 

Here, we want to decide, given a claw-free block graph  and an integer , whether . Let  denote the size of a largest clique in . Clearly, the answer is always ``no'' when , and, because  is a chordal graph (which means that ), the answer is ``yes'' when . Thus, from now on we suppose that . 
To solve the problem, we compute the maximum size of a subset  that realizes distinct colors on a -coloring of . Clearly, we will get  if and only if the maximum size of such a set is . The idea is then to compute this value for smaller subgraphs of  and then try to combine these solutions. For this, we need a more convenient way to represent a solution. In the next section, given a subset  that can realize distinct colors in a -coloring, we show another representation of such a -coloring related to . Then, in the following section, we show how to combine the partial solutions. 

\section{Representation of a -coloring that realizes  colors}

Before we start, we need some further definitions. Consider a claw-free block graph  rooted at some block , an integer , and a subset . For any block  different from the root, denote by  the \emph{parent block of }, and by  the cut vertex that separates  from . Finally, denote by  the subgraph rooted at . 

The \emph{flow network related to } is denoted by  and is obtained as follows (denote by  the capacity function). Consider a block  of . Add nodes  to , and set the capacity of each node to one, except for  which has capacity . Denote this set of nodes by  and call node  the \emph{cash-node (of )}. Suppose, without loss of generality, that , , are all the cut vertices in  different from . For each , let  be the block containing  different from  (i.e.,  and ). \begin{itemize}
 \item[(I)] If  (observe Figure \ref{xnotinW}) - add node  with capacity , and the following arcs:\begin{enumerate}
  \item , for all ; 
  \item , for all ; 
  \item , , and .
 \end{enumerate}

 \begin{figure}\label{xnotinW}
\begin{center}
\includegraphics[height=3cm]{xinotinW.eps} 
\end{center}
\caption{Flow network representation when . The arcs represent the existence of every arc between the corresponding subsets of nodes.}
\end{figure}

   
 \item[(II)] If  (observe Figure \ref{xinW}) - add node  with capacity , a source node  that sends out 1 unit of flow, and the following arcs:
 \begin{enumerate}
  \item  and , for all ;
  \item  and , for all ; and
  \item  and .
 \end{enumerate}
 
 \begin{figure}\label{xinW}
\begin{center}
\includegraphics[height=3.5cm]{xinW.eps}
\end{center}
\caption{Flow network representation when . The arcs represent the existence of every arc between the corresponding subsets of nodes.}
\end{figure}

\item[] Nodes of type  are also called \emph{cash-nodes}.
\end{itemize}

Finally, we set  to be the sinks. The network will admit a flow if each flow-unit coming from a source can be rooted to the sinks. More formally, write , , and let  denote the set of non-source nodes of . A \emph{flow in } is a collection of directed paths  of  such that:  starts at  and ends at , for every ; and , for every . We denote the value  by .
We want to prove that there exists a -coloring where  realizes distinct colors if and only if  admits a flow.

Intuitively, one can see the flow paths as colors that are forced to vertices because of the precoloring. When no implications exist, we can ``store flow'' in the cash-nodes, which means that no vertex in that block must necessarily have that color. That is why the flow path containing node  above must also contain node , i.e., the color of a node cannot change from one block to the other. Also, if , then the colors that do not necessarily appear in block  must appear in block  in order for  to realize its color. Note that some of the colors that appear in  are allowed to appear in  but up to a limit set by the capacity of node , which depends on the degree of . This is because if too many colors are repeated in the neighborhood of  then it may be impossible for  to realize its color. In what follows, we give a formal definition of these ideas.

Let  be a flow in . For each vertex , let  represent the set of nodes of  related to , i.e., , for every block  containing , and  if . Let  be a -coloring of . We say that  is \emph{saturated by  (in )} if , and that  is \emph{flow colored} (in  according to ) if  is saturated by . If  is a proper -coloring such that every saturated vertex is flow colored, then we say that  is a \emph{flow coloring of }.





\begin{theorem}\label{thm:flow}
Let , and . If  has a flow , then there exists a flow coloring  of  where each  realizes a distinct color. Conversely, if  is a coloring of  such that each  realizes a distinct color, then  is a flow coloring of , for some flow  in .
\end{theorem}
\begin{proof}
The proof is by induction on the height of . If , because , we get  and the theorem is vacuously true. So, let  be the cut vertices of , and, for every , let  be the block containing  other than ,  be subgraph of  rooted at ,  be the set , and  be the subnetwork of  related to . 

First, consider a flow  in  and, for each , let  be the flow  restricted to . By induction hypothesis, there exists a flow coloring  of  such that each  realizes a distinct color. We can suppose that the colors realized in  are distinct whenever , as otherwise it suffices to rename some colors. Let  be the coloring (not necessarily proper) obtained by the union  (it is well defined since  whenever ). In order to obtain a flow coloring, we need to ensure that: 1) each saturated  is flow colored; 2) the coloring is proper; and 3) that each  realizes a distinct color.
 
First, consider a saturated vertex . By the capacity of nodes in  and the construction of , one can verify that  is saturated by at most one path, say . We want to ensure that  is colored with . If  is not a cut vertex, then it is not colored in  and we can just color  with ; so, suppose that , for some . If , note that the path  must be in  and, by induction hypothesis,  is flow colored.
So, consider that , and suppose that . By induction hypothesis, we know that  is not saturated in , i.e.,  is not contained in . Also, because  is not adjacent to , we know that  also does not contain , which implies that . The following claim ensures us that we can change  to color  with .

\begin{claimnum}\label{claim1}
 Let  be nonsaturated in  with , and let  such that . Then, we can change the color of  from  to  in .
\end{claimnum}
\begin{proof}
Note that if , we can simply switch colors  and  in ; so suppose otherwise. Because  is a flow coloring of  with respect to , which means that it is proper and that every saturated vertex is flow colored, we know that path  contains cash-node ; hence, . Let  be the cut vertex that separates  from , and recall that , since  is proper. Denote by  the graph rooted at  and by  the graph . If  and , then we can switch colors  and  in . So, we analyse the following cases:

\begin{enumerate}
  \item : Let  be the block containing  different from . Because  is not adjacent to , we get that path  must contain , for some , which means that color  is repeated in . Thus, we can just switch colors  and  in ;
	
  \item : Note that this implies  since . Let . If there exists some color  such that  and , then we switch colors  and  in  and colors  and  in . Thus, suppose otherwise. This means that every color appears in  or in , i.e., 
  Because the capacity of  is , we get that at least  paths starting at  must pass through nodes in  . Also, because  is a flow coloring, at most  nodes in  are contained in paths starting at . This means that every node in  that receives color from  is flow colored, a contradiction since  and  is not flow colored.
\end{enumerate}

\end{proof}

Now, we want to ensure that  is a proper coloring. For this, consider  to be an unsaturated vertex colored with some color that also appears in . Note that, if there is any conflict, such a vertex must exist, since every saturated vertex is flow colored. We show that  there exists a color , and apply Claim \ref{claim1} to change the color of  to  in , thus eliminating the conflict. So suppose otherwise, i.e., that . Also, let  denote the set , and observe that . Thus, every color appears in  or in , i.e., . However, by the capacity of  and the fact that  has a conflicting color, we get that , a contradiction. Note that a similar argument can be applied if  has a color of , or if  is not colored. Because of this, we can suppose that every unsaturated vertex of  is colored with a color not in 







Finally, we need to prove that we can realize a disctinct color in each vertex of . Because  and , we know that . Consider , and denote by  the set of colors that do not appear in , and by  the set of colors . We change  in order to decrease the number of colors in . First, note that, for every , either  intersects  or it contains  and, hence, must intersect . Therefore, since we already know that every vertex is flow colored, we get that . Now, consider any , and let  be the set of colors of  that are repeated in , i.e., . By the capacity of node , we know that . Therefore, some color  not in  must be repeated in the neighborhood of . Since  is also not in , we can just switch colors  and  in .

Now, we prove the second part of the theorem. Let  be a coloring of  where each  realizes a distinct color and let  be the coloring  restricted to , for each . By induction hypothesis,  is the flow coloring of a flow in ; because , , are pairwise disjoint, we make an abuse of language and denote by  the path starting in  and ending in . We want to extend these paths into a flow in  for which  is a flow coloring. First, we just extend them without paying atention to the nodes' capacities. Consider any  and let  be equal to , if there exists  such that ; otherwise, let  be equal to . We want to ensure that path  contains . Let  be such that . If , by induction hypothesis, we know that  contains  and we just add  to ; hence, suppose . Because  is a proper coloring and by induction hypothesis, we know that , and: either (1)  contains , for some ; or (2)  contains . If (1) occurs and either  or , or if (2) occurs and , then add  and  to . If (1) occurs, , and , then add  to . Finally, if (2) occurs and , because  realizes its color and , we must have that ; hence, we can just add  to .

It remains to prove that the capacities are respected. Because , whenever , we get that , for all . So, it remains to prove that the capacities of the following cash-nodes are respected:
\begin{itemize}
 \item cash-node , when : its capacity is  and it is violated only if all  and no flow path passes through . However,  is colored with some color, which contradicts the fact that  is a flow coloring;
 
 \item cash-node , when : in this case, its capacity is . Because  is not adjacent to  nor to , we know that each path passing through  must come from a node  and end at a node . Since  are flow colored, this means that each path passing through  defines a color that is being repeated in . Since  realizes a color in , we know that at most  colors are repeated in its neighborhood, i.e., that ;
 
 \item cash-node : each flow path ending at  defines a color in  that is not used in . Because  is a clique, there are at most  such colors.
\end{itemize}
\end{proof}

\section{Finding the maximum }\label{sec:dynalg}

Now, we want to find tha maximum size of a subset  that realizes distinct colors. We solve the problem for each subgraph rooted at some block, starting by the leaf nodes and going up towards the root . Recall that , which implies that the answer is always zero on the leaf nodes. For simplicity, suppose we are at the root , where  connects  to its parent block, if it exists, and  are all the cut vertices in , for some . We denote by  the family of flow networks . We make an abuse of language and say that there exists a flow  in  if  is a flow in , for some , and the \emph{value of } is given by . Also, for each , we denote by  the block containing  other than . Now, let  and . We say that a flow  \emph{realizes } if , and , and we define:

\begin{itemize}
\item[*] : maximum value  for which there exists a flow  in  of value  that realizes .
\end{itemize}

We want to compute table  using tables . Note that no vertex in  can define a source in these subsolutions since they have degree at most  in the related subgraphs. Therefore, we need to investigate the possibility of adding a subset of  as new sources. However, there is an exponential number of subsets to investigate. Because of this, we need some auxiliary tables. For each , let  be the graph which corresponds to the component of  containing , and let  be the family of flow networks . 
Now, consider , , , and .  We say that a flow  \emph{realizes } if , , and . Then, we define:

\begin{itemize}
\item[*] : maximum value  for which there is a flow in  of value  that realizes .
\end{itemize}
 
Because  equals , we get:


In what follows, sometimes we implicitly assume that a flow strictely contained in some other exists. That holds because of the following proposition (it suffices to ignore one of the paths).

\begin{proposition}\label{prop:1lessflow}
If there exists a flow  in , where  is rooted at , then there exists a flow  in , .
\end{proposition}


Now, considering that we know tables  and , we want to compute , where , , and . For this, we need to analyse what types of solutions in  and in  can be combined. Consider  realizing entry  of , where , and  realizing entry  of . Let the values of  be  and , respectively. We want to construct a flow  that realizes . 
Let  and  be such that  is a flow in  and  is a flow in , where  is the subgraph rooted at . Intuitively, what we do is applying Proposition \ref{prop:1lessflow} to accomodate  into a flow in , for some . We also try to increase the combined flow's value by adding  to . Below, by ``push flow into node '', we mean that we increase the corresponding path to a path with extremity in . Observe that every path in  already has an extremity in ; therefore, we try to push flow  into  taking into consideration  and the values .


First, we try to push the flow  into  without trying to add vertex  as a new source. In this case, all the flow not in  can be pushed into whatever unsaturated node in  by passing through . So, if either  or , we know that the produced flow will not realize . This is also the case when one of the following situations occurs. If the amount of flow in  is not sufficient to satisfy the following lacks: of  in ; of  in ; and of  in . Or if , , and the amount of flow  in  is not sufficient to satisfy the lack in . We then say that  are \emph{weakly -compatible} if 

\begin{itemize}
  \item[W1] ; \item[W2] ; \item[W3] , and . \end{itemize}



If  and  are weakly -compatible, their \emph{combined value}, denoted by , is the amount of flow that can be sent to . By applying Proposition \ref{prop:1lessflow}, this either equals the number of nodes, or the sum  minus the quantity of flow sent to satisfy the lacks. That is: . 

Now, if we want to turn  into a source, we have to pick entries in  of type  and push the flow  into , and the flow  into . Suppose we can push  units of flow through . By similar arguments, we need: the amount of flow lacking in  to be nonnegative and to be satisfiable by the amount of flow in  plus ; and the amount of flow lacking in  to be non-negative and to be satisfiable by the amount of flow in  minus . For this, we define  to be \emph{strongly -compatible} if  and there exists  such that:

\begin{itemize}
\item[S1] ;
  \item[S2] ; and \item[S3] . \end{itemize}

Again, if  and  are strongly -compatible, we want their \emph{combined value}, , to be the amount of flow that can be sent to . Because  units of flow are necessarily sent to , we know that this is either , or  minus the flow used to satisfy the lack. This gives us that: , where  (minimum between the amount of flow remaining in  and the amount of flow that we can still push through ). If  are either weakly or strongly -compatible, we say that they are \emph{-compatible} (or just ``compatible'' if there is no ambiguity). Finally, we prove that these definitions completely describe our solution set.

\begin{lemma}
There exists a flow  that realizes  of value  if and only if there are flows  and  such that  are -compatible and .
\end{lemma}
\begin{proof}
: Consider entries  in  and  in  such that  realize , respectively. Let  have value  and  have value . First, suppose they are weakly compatible. Observe that the flow paths in  with extremity in  can actually end in any subset of these vertices, the same being valid for the flow paths in  with extremity in . We construct a flow in  that realizes  of value  as follows. 
\begin{enumerate}
  \item Push  units of flow from  into , then into any subset of  unsaturated nodes in . This is possible by (W1), which implies , and by (W3), which ensures that there is a sufficient amount of flow to be pushed.
  \item Push  units of flow from  into , then into . This is possible by (W2) and (W3).
  \item If , then we know that  is already satisfied by . Otherwise, push  units of flow from  into , then into . This is possible by (W3).
  \item Push flow  directly into , if there is any. If  and , in which case , we push 1 unit of flow from  into . This is possible by (W3).
  \item Push the remaining flow into , decreasing the amount if needed, i.e., if there is more flow than nodes. This can be done by Proposition \ref{prop:1lessflow}. By the previous steps, we know that the remaining amount is . 
\end{enumerate}

Now, suppose that  are strongly compatible. This implies . We construct a flow in  where  is a source as explained below. First, note that  is -dense, by .

\begin{enumerate}
  \item Create a new source  and push the new unit of flow into , then into . This is possible since .
  \item Push  units of flow from , and  other nodes in , into ; this is possible by (S2).
  \item Push  units of flow from  into ; this is possible by (S3);
  \item Finally, push as much flow as possible from what remains in  into , decreasing the amount of flow if needed. It is possible to verify that this is equal to .
\end{enumerate}

: Consider a flow in  that realizes entry , and let  be  restricted to , respectively. In , let , , , and . Clearly,  realizes entry  in . Now, let , and  be the value of  minus . Finally, let  be either the amount of flow received by , if  is not a source in , or 0 otherwise. Clearly,  realizes entry  in . We need to prove that  are compatible and . Consider two cases: 

\begin{itemize}
  \item  is not a source in : clearly,  and , i.e., (W1) and (W2) hold. Also, by definition, we know that the flow at  that do not come from  must come from , that is, the first part of (W3) holds. Finally, if either  or , then  and (W3) follows. So, suppose that , and . Observe that, in this case,  receives flow from some node in , which implies .
  
  \item  is a source in : Let  be the amount of flow sent from  to . By definition, we know that  and  is the sum of  and  on these nodes; therefore,  and . Also, the flow in  not coming from  and , must come from  and , i.e., , and  holds. Similarly, the flow in  not coming from  must come from , i.e.,  and  holds. Finally, (S1) holds because of the capacity of node  and it is not hard to verify that  equals .
\end{itemize}
\end{proof}

Consider  to be  compatible. Note that only conditions (W3) and (S3) depend on the values , and that, if they hold for , they also hold for bigger values. Also, the larger these values are, the larger is the combined value of  and . Therefore, it indeed suffices to investigate tables  and  in order to compute . This gives us the complexity presented in the next corollary. We realize that this complexity can be refined, however here we are more concerned about the theoretical aspect of the problem.

\begin{corollary}
Given a claw-free block graph  and an integer , where , it can be decided whether  in time . 
\end{corollary}
\begin{proof}
Each table  has size , and each table  has size . Also, deciding whether entries  of  and  of  are -compatible takes time . Therefore, computing an entry  of  takes time  and, because we need to compute  entries, for each  such that , it takes time  to compute . Finally, since there are  blocks in , the theorem follows.
\end{proof}




\begin{thebibliography}{99}

\bibitem{BMZ.09}
M. Blidia, F. Maffray, and Z. Zemir. On b-colorings in regular graphs.
{\it Discrete Applied Math.} 157 (2009) 1787--1793.

\bibitem{BM08}
A. Bondy and U.S.R. Murty.
Graph Theory, Spring-Verlag Press, 2008. 

\bibitem{Cabello.Jakovac.11}
S. Cabello and M. Jakovac. 
On the b-chromatic number of regular graphs. 
{\it Discrete Applied Math.}, 159 (2011) 1303--1310, 2011.

\bibitem{Campos.etal.15}
V.~Campos, C.~Lima, N.A.~Martins, L.~Sampaio, M.C.~Santos and A.~Silva. The b-chromatic index of graphs. Discrete Mathematics 338 (2015) 2072--2079.

\bibitem{Campos.Lima.Silva.15}
V.~Campos, V.~Lima and A.~Silva.
Graphs of girth at least 7 have high b-chromatic number. 
{\it European Journal of Combinatorics} 48 (2015) 154--164.

\bibitem{HLS.11}
F.~Havet, C.~Linhares-Sales and L.~Sampaio. b-coloring of tight graphs. {\it Discrete Appl. Math.} 160 (18) (2012) 2709--2715.

\bibitem{Irving.Manlove.99}
R.W.~Irving and D.F.~Manlove. 
 The b-chromatic number of a graph. 
{\it Discrete Appl. Math.} 91 (1999) 127--141.

\bibitem{KM.02}
M. Kouider and M. Mah\'eo.
 Some bounds for the b-chromatic number of a graph. 
 {\it Discrete Mathematics}, 256 (2002) 267--277.

\bibitem{KRATOCHVIL.etal.02}
J.~Kratochv\'{\i}l,  Zs.~Tuza,  and M.~Voigt. 
 On the b-chromatic number of graphs. 
 {\it Lecture Notes In Computer Science} 2573 (2002) 310--320.

\bibitem{Maffray.Silva.12}
F.~Maffray and A.~Silva.
b-colouring outerplanar graphs with large girth.
{\it Discrete Mathematics} 312 (10) (2012) 1796--1803.

\bibitem{Marx.07} 
    D.~Marx.
   Precoloring extension on chordal graphs.
 {\it Graph Theory in Paris, Proceedings of a Conference in Memory of Claude Berge} (2007) 255--270.


\bibitem{MS.12}
F.~Maffray and A.~Silva.
b-colouring outerplanar graphs with large girth.
{\it Discrete Mathematics} 312 (10) 1796--1803, 2012.


\bibitem{Silva.10}
A.~Silva. 
 The b-chromatic number of some tree-like graphs. 
 PhD Thesis, Universit\'e de Grenoble, 2010.
\end{thebibliography}

\end{document}