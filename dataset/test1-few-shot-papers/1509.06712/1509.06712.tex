
In this section we consider a  variant of the Bin Packing problem (BP)  \cite{Srikantaiah:2008},  which is the key sub-problem of the application  investigated here. 
We denote by  the integer sizes of the  items such that . A bin  is characterised by an integer capacity , a non-negative fixed cost  and a  non-negative  cost  for each unit of used capacity. We denote by  the characteristics of the  bins. A bin is used when it contains at least one item. Its cost is a linear function , where  is  the total size of the items  in bin . The total load  is denoted by  and the maximum capacity by . The problem is to assign each item to a bin subject to the capacity constraints so that we minimise  the sum of the costs of all bins. We refer to this problem as the \emph{Bin Packing with Usage Cost} problem (BPUC).
BP  is a special case of BPUC where all  are set to 1 and all  to 0. The following example shows that a good solution for BP might not yield a good solution for  BPUC. 


\begin{example}\label{ex1} In Figure~\ref{example1}, Scenario~1,   B =\{(9,0,1),(3,0,2),(3,0,2),(3,0,2),(3,0,2)\} and S = \{2,2,2,2,3,3,3\}. Notice that . The packing  : \{\{2,2,2,2\}, \{3\}, \{3\}, \{3\}, \{\}\} is using the minimum number of bins and has a  cost of 26 (8*1 + 3*2 + 3*2 + 3*2). The packing : \{\{3,3,3\}, \{2\}, \{2\}, \{2\}, \{2\}\} is using one more bin but has a  cost of 25 (9 + 2*2 + 2*2 + 2*2 + 2*2). Here, () is better than () and using the minimum number of bins is not a good strategy. 
Now  change the last unit cost to  (see Figure~\ref{example1}, Scenario 2). The cost of  remains unchanged since it does not use bin number 5 but the cost of  increases to 27, and thus   is now better than .
\end{example}
\begin{figure}[h]
\centering
\includegraphics[width=12cm]{exemplePacking1.png}
\caption{Example of optimal solutions in two scenarios of costs. In Scenario 1, the best solution has no waste on the cheapest bin. In Scenario 2, it does not fill completely the cheapest bin.}
\label{example1}
\end{figure}

\noindent\textbf{Literature Review.} 
A first relevant extension of BP for the current paper is called Variable Size Bin-Packing, where bins have different capacities and the problem is to minimise the sum of the wasted space over all used bins~\cite{Monaci2002}. It can be seen as a special case of BPUC where all  and . 
Recent lower bounds and an exact approach are examined in~\cite{Haouari:2011}.
A generalisation to any kind of fixed cost is presented in~\cite{Crainic2011}, which can be seen as a special case of BPUC where all . 
Concave costs of bin utilisation  studied in~\cite{Li2006} are more general than the linear cost functions of BPUC. However~\cite{Li2006} does not consider bins of different capacities and deals with the performance of classical BP heuristics whereas we are focusing on lower bounds and exact algorithms.
Secondly, BP with general cost structures have been introduced in~\cite{AnilyBS94} and studied in~\cite{Epstein:2012}. The authors investigated BP with non-decreasing and concave cost functions of the number of items put in a bin. They extend it with profitable optional items in~\cite{Baldi2012}. Their framework can model a fixed cost but does not relate to bin usage.






\newenvironment{equ}{\vspace{0.1cm}\vspace{1mm}}

\section{Basic Formulation and  Lower Bounds of BPUC}
 Numerous linear programming models have been proposed for BP ~\cite{DBLP:journals/eor/Carvalho02}. We first present a 
 formulation for  BPUC.
For each bin a binary variable  is set to 1 if bin  is used in the packing,
and 0 otherwise. For each item  and each bin   a binary variable  is set to 1 if item  is packed into bin , and 0 otherwise. 
We add non-negative variables  representing the load  of each bin j. The  model is as follows:

Constraint (1.1) states that each item is assigned to  one bin whereas (1.2) and (1.3) enforce the capacity of the bins. We now  characterise the linear relaxation of the model. 
Let  be a real number associated with bin . If bin  is filled completely,  is the cost of one unit of space in bin . We sort the bins by non-decreasing : ;  is a permutation of the bin indices . Let   be the minimum number of bins such that .
\begin{proposition}
Let  be the optimal value of the linear relaxation of the formulation~(1). We have  with 
. 
\end{proposition}
 \begin{proof}
  because of the constraint , so .  is the quantity minimising  under the constraints  where each . To minimise the quantity  we must split W over the  related to the smallest coefficients . Hence, . \qed
 \end{proof} 
 is a lower bound of BPUC  that can be computed in . Also notice that   is  the bound that we get by solving the linear relaxation of formulation (1).
\begin{proposition}
  is the optimal value of the linear relaxation of the formulation~(1). 
\end{proposition}
\begin{proof} For all , we set each  to  and  to .  We fix  to  and  to . For all  we set  and . Constraints (1.3) are thus satisfied. Finally we fix  for all  so that constraints (1.2) and (1.1) are satisfied. This is a feasible solution of the linear relaxation of (1) achieving an objective value of . We have, therefore,  and consequently  from Proposition~1.  \qed
 \end{proof}
Adding the constraint  for each item  and bin , strengthens the linear relaxation only if . Indeed, the solution given in the proof is otherwise feasible for the constraint, ( and for  we have  if ). 




\paragraph{Dominance and symmetries.} Formulation~(1) can be used to solve the problem exactly and strengthened to take some bin and item symmetries into account. Typically, if bins  and  are such that ,  and  then bin  is said to dominate bin  and the constraints  as well as  can be added to rule out some dominated solutions. 
Suppose now that we have  items that have the same size, any permutation of the items in a feasible solution remains feasible and have the same overall cost. A simple way to break this symmetry is to replace for each bin , the k variables   corresponding to the identical items by a variable  giving the number of the k items assigned to  (and adding ). We also perform in practice another simple pre-processing by tightening the capacities of the bins using dynamic programming. Given the item sizes and for each bin , one can compute efficiently (when  is sufficiently small) the largest integer less than or equal to  that is also equal to the sum of a subset of . 






\section{Two Extended Formulations of BPUC}
\label{af}


\paragraph{\textbf{The Cutting Stock Model}}
The  formulation of Gilmore and Gomory for the cutting stock problem~\cite{gilmore1961linear} can  be  adapted for BPUC. The items of equal size are now grouped and for  different sizes we denote the number of items of sizes   by   respectively.
A cutting pattern for bin  is a combination of item sizes that fits into bin  using no more than  items of size . 
In the -th pattern of bin , the number of items of size  that are in the pattern is denoted . 
Let  be the set of all patterns for bin . 
The cost of the -th pattern (assumed to be non empty) of bin  is therefore equal to . The cutting stock formulation is using a variable  for the -th pattern of bin :

Constraint (2.1) states that each item has to appear in a pattern (thus in a bin) and (2.2) enforces one pattern to be designed for each bin (convexity constraints). A pattern  for bin  is valid if  and all  are integers such that . The sets  have an exponential size so the linear relaxation of this model can be solved using column generation. The pricing step is a knapsack problem that can be solved efficiently by dynamic programming if the capacities are small enough. The pricing problem for bin  can be written as follows (where  is the dual variable associated to the d-th constraint (2.1) and  is the dual variable related to j-th constraint (2.2)):

In practice, we solve the pricing problem using Dynamic Programming in .
\paragraph{\textbf{The \arcflow\ Model}}
Carvalho introduced an elegant \arcflow\ model
 for BP~\cite{Carvalho99,DBLP:journals/eor/Carvalho02}. His model explicitly uses each unit of capacity  of the bins. In the following we show how to adapt it for BPUC.
Consider a multi-graph , where  is the set  of  nodes labelled from  to  and a final node labelled , and   is the set of  two kinds of edges. An edge   between two nodes labelled  and  represents the use of an item of size . 
 An edge of  for each bin  represents the usage  of the bin , and therefore . An example of such a graph is shown in Figure~\ref{example2}(a).
\begin{figure}[t!]
\centering
\includegraphics[width=12cm]{arcflows2.png}
\caption{(a) An example of the graph underlying the \arcflow\ model for ,  so that . A packing is shown using a dotted line:  is put in the first bin for a cost of 7,  is in the second bin for a cost of 7 and  in the last bin for a cost of 18.
(b) The graph underlying the \arcflow\ model after the elimination of symmetries.}
\label{example2}
\end{figure}
Notice that this formulation has symmetries since a packing can be encoded by many different paths.
Some reduction rules were given by Carvalho~\cite{Carvalho99}, which help in reducing such symmetries (see Figure~\ref{example2}(b)).







BPUC can be seen  as a minimum cost flow between 0 and  with constraints enforcing the number of edges of a given length used by the flow to be equal to the number of items of the corresponding size. We have variables  for each edge  as well as variables  for each pair of bin  and . 
The  cost of using an edge  for bin  with  is  and . The model is as follows:

Constraint (3.1) enforces the flow conservation at each node, and Constraint (3.2) states that each bin should be used exactly once. Constraint (3.3) ensures that all the items are packed, while Constraint (3.4) enforces that bin  is not used beyond its capacity . A solution can be obtained again by decomposing the flow into paths. The number of variables in this model is in  and the number of constraints is . Although its
LP  relaxation is stronger than that of Model~(1), it remains dominated by that of Model~(2).
\begin{proposition}. The optimal value of the linear relaxation of (3) is less than  the optimal value of the linear relaxation of (2). 
\end{proposition}
 \begin{proof} Let  be a solution of the linear relaxation of (2). Each pattern  is mapped to a path of the \arcflow\ model. A fractional value  is added on the arcs corresponding to the item sizes of the pattern (the value of the empty patterns for which all  is put on the arcs ). The flow conservation (3.1) is satisfied by construction, so is (3.2) because of (2.2) and so are the demand constraints (3.3) because of (2.1). Any solution of (2) is thus encoded as a solution of (3) for the same cost so . \qed
 \end{proof}


\begin{proposition}  can be stronger than  \emph{i.e.} there exist instances such that .  
\end{proposition}

 \begin{proof} Consider the following instance:  and . Two items of size 1 occurs so that ,  corresponding to . The two  bins have to be used and the first \emph{dominates} the second (the maximum possible space is used in bin 1 in any optimal solution) so the optimal solution is the packing  (cost of 12). Let's compute the value of . It must fill the first bin  with the pattern  for a cost of . Only three possible patterns can be used to fill the second bin: ,  and  (a valid pattern  is such that ). The best solution is using  and  taking both a 0.5 value to get a total cost . The \arcflow\ model uses a path to encode the same first pattern [1,1] for bin 1. But it can build a path for bin 2 with a  unit of flow taking three consecutive arcs of size 1 to reach a better cost of . This path would be a pattern [3,0] which is not valid for (\ref{bp-model}). So  and . 
 \qed
 \end{proof}


The \arcflow\ model  may use a path containing more than  arcs of size  with a positive flow whereas no such patterns exist in (3) \textit{because} the sub-problem is subject to the constraint . The cutting stock formulation used in \cite{Carvalho99}  ignores this constraint and therefore the bounds are claimed to be equivalent.






\section{Extending the Bin Packing Global Constraint}

A bin packing global constraint was introduced for constraint programming  by \cite{DBLP:conf/cp/Shaw04} and discussions about its filtering can be found in \cite{Schaus2009,DBLP:conf/aaai/ReginR11}. We present an  extension of this global constraint to handle BPUC. The scope and parameters are as follows:

Variables ,  and  denote  the bin assigned to item , the load of bin , and the number of bins used, respectively. These  variables are also used by the \textsc{BinPacking} constraint. Variables  and  are due to the cost. They denote whether bin  is open, and the cost of the packing. The last two arguments refers to BPUC and give the size of the items as well as the costs (fixed and unit). In the following,   (resp. ) denotes the lower (resp. upper) bound of variable .   \\

\noindent\textbf{Cost-based Propagation using .}
The characteristics of the bins of the restricted  BPUC  problem  based on the current state of the domains of the variables is  denoted by , and defined by  where  is the remaining capacity, and  is the remaining fixed cost  that is set to 0 if the bin is known to be open. The total load that remains to be allocated to the bins is denoted . Notice that we use the lower bounds of the loads rather than the already packed items. We assume it is strictly better due to the reasoning of the bin packing constraint.

\paragraph{Lower bound of z} 
The first propagation rule is the update of the lower bound   of  denoted by . The bound is computed by summing the cost due to open bins and minimum loads with the value of  on the remaining problem. 
It gives a maximum possible increase in cost denoted by :


\noindent\emph{Bounds of the load variables.} 
We use the notation  to denote the units of space used by   on bin . The bins  are fully used so , for bin  we have  and .
We define the bin packing problem  obtained by excluding the space supporting the lower bound . 
 The resulting bins are defined as  where  for all ,  and  for all . Lower and upper bounds  of loads are adjusted with rules (). 




Let  be the largest quantity that can be removed from a bin , with ,
and put at the other cheapest possible place without overloading . Consequently, when ,  is the largest value in  such that . When , the same reasoning can be done by setting  in . 

Similarly, let  be the largest value in  that can  be put on a bin , with , without triggering a contradiction with the remaining gap of cost.  is thus the largest value in  such that .

\noindent\emph{Channelling.} The constraint ensures two simple rules relating the load and open-close variables (a bin of zero load can be open):



\smallskip\noindent\emph{Bounds of the open-close variables.} The propagation rule  for  can derive  from (\ref{r2}), which in turn (because of the channelling between  and ) will force a bin to open \emph{i.e.}    will change  to . To derive that a  has to be fixed to , we can use  similarly to the reasoning presented for the load variables (checking that the increase of cost for opening a bin remains within the gap). 

Tightening the bounds of the load variables can trigger the existing filtering rules of the bin packing global constraint thus forbidding or committing items to bins. Notice that items are only increasing the cost indirectly by increasing the loads of the bins because the cost model is defined by the state of the bins (rather than the items). The cost-based propagation on  is thus performed by the bin packing global constraint solely as a consequence of the updates on the bin related variables, i.e.\  and .\\

\smallskip\noindent\emph{Filtering the load variables with dynamic programming.} Another filtering rule based on dynamic programming can be added to the Bin-Packing global constraint. We can simply update  (resp. ) to the smallest (resp. largest) integer greater than (resp. less than) or equal to the current value of  (resp. ) that can be reached with the remaining items that can still go on bin . We can solve this problem by dynamic programming (taking into account the items already assigned to ). This is generally  very costly in practice and is not performed by default by the bin-packing global constraint. Nevertheless, in this case, one can notice that  strongly relies on the accuracy of    and  and we observed that it can help significantly in practice for some hard instances. This technique has been originally proposed by \cite{DBLP:journals/anor/Trick03} for knapsack constraints. \\

\begin{example}\label{ex2}Let's consider the following instance:  B =\{(9,9,5), (3,1,5), (7,14,3), (5,1,10), (12,12,10)\} and S = \{3,5,5,5\}. 
The total load is therefore  and the values or  sorted increasingly are . The lower bound is therefore . Figure~\ref{example2} is  visualising   and shows the optimal solution. Propagation on the bounds of the load variables by dynamic programming is not performed here for the illustrative purposes.



\begin{figure}[t!]
\centering
\includegraphics[width=11cm]{exemplelb1.png}
\caption{Visualisation of the lower bound Lb and optimal solution for the instance of example \ref{ex2}. The bins have been sorted by increasing .} \label{example2}
\end{figure}

Assuming that we have an upper bound  of value 130, the gap is therefore initially equal to 130 - 99 = 31. 
Initially  as  and . 
Afterwards, the propagation is able to deduce  as well as  as illustrated on the left of Figure~\ref{example3}. Indeed  would lead to a lower bound of  thus overloading . Similarly  would give  and with  we get . At this stage we know that bins 3 and 1 are open in any solution of cost less than 130. This is affecting the propagation as the fixed cost of bins 1 and 3 are now  included in  and propagation is strengthened.
The  cost due to open bins is  and due to  lower bounds of loads of bins 1 and 3 is .
Consequently,  and .
Therefore, 
the  values are now ranked: 
. Notice that the ordering on bins have changed.
The new value of  is  and thus 
the new value of   is . 
When the fix point is reached we know that  as well as .
 
\begin{figure}[t!]
\centering
\includegraphics[width=12cm]{newfigure-crop.pdf} \caption{Propagation of lower and upper bounds of the load variables. 
The state of the domains at the end of the first iteration is depicted on the left.
The lower bounds of  and  are increased to 1 and 
 the upper bound of  is decreased to .
The state of the domains when the fix point is reached is shown on the right.}
\label{example3}
\end{figure}

Let's now add the propagation using dynamic programming ( would typically be reduce to 5). The lower bound observed at the root node is 119.66 whereas the one obtained by linear programming \emph{i.e.} with formulation~(\ref{lp-model}) (including the initial tightening of the capacities by dynamic programming) is 114.4.
\end{example}


\noindent\textbf{Algorithms and Complexity}. Assuming that  and  are available,  can be computed in  time. Firstly we compute the  values corresponding to   for all bins. Secondly, we sort the bins in non-decreasing . Finally, the bound is computed by iterating over the sorted bins and the complexity is dominated by the sorting step. After computing , the values  (the permutation of the bins) such that  are available as well as the critical  and . The update of  and   can then be done in  as shown in Figure~\ref{propalgo}. Notice that this process can be repeated until a fix point is reached.

Algorithm \texttt{UpdateMinimumLoad} of Figure~\ref{propalgo} is used to update  the lower bounds of the load variables 
for each bin   such that . Recall that  is the number of bins supporting the lower bound .
If  (resp.\ ) then the algorithm tries to find  the largest quantity, denoted by , that can be removed
from the bin  and can be put on the bins , where  (resp.\ ),
such that the increment in the cost remains smaller than the  (Lines~4--10). 
The algorithm loops over the bins as long as the   has not reached its maximum amount on bin  \emph{i.e.}  and the gap has not been overloaded.
When the increment in cost exceeds the gap then the minimum load if known.
Similarly, \texttt{UpdateMinimumLoad} of Figure~\ref{propalgo} is used to update  the upper bounds of the load variables 
for each bin   such that .

\begin{figure}[t]
\begin{small}
\centering

  \begin{minipage}{.45\textwidth}
  \centering


\begin{coden}
{ Algorithm 1: UpdateMinimumLoad} \\
  {\bf Input}:  with , ,  \\
  {\bf Output}: a lower bound of  \\
  ~1.  ; ; ; \\
  ~2.  {\bf If} () \{;\} \\
  ~3.  {\bf While} ( \&\& ) \\
  ~4.   \> \> ;\\~5.   \> \> ;   \\ 
  ~6.   \> \> {\bf If} () \\ 
  ~7.   \> \> \> \> ;\\
  ~8.   \> \> \> \> {\bf return} ;\\
  ~9.   \> \> ; \\ 
  ~10. \> \> ;  ; \\
  ~11. {\bf return}  \\
  \end{coden}

\end{minipage}
\end{small}
\begin{minipage}{0.45\textwidth}  
  \centering
  \vspace{-0.3cm}

\begin{small}
\begin{coden}
{Algorithm 2: UpdateMaximumLoad}\\
   {\bf Input}:  with  , , \\
   {\bf Output}: an upper bound of  \\
  ~1.  ; ; ; \\
  ~2.  {\bf If} () \{; ;\} \\
  ~3.  {\bf While} ( \&\& ) \\
  ~4.   \> \> ;   \\ ~5.   \> \> ;   \\ 
  ~6.   \> \> {\bf If}  \\ 
  ~7.   \> \> \> \> ;\\
  ~8.   \> \> \> \> {\bf return} ;\\
  ~9.   \> \> ; \\
  ~10. \> \> ; ; \\
  ~11. {\bf return} 
  \end{coden}
\end{small}
\end{minipage}
\caption{Propagation algorithms for updating the lower and upper bounds of the load variables \label{propalgo}}
\end{figure}

\subsection{Dominance and symmetries.} The dominance and symmetries previously mentioned can be similarly eliminated here. We go a step further and eliminate some dominances during search. In particular, for two bins ,  that are known to be open ()  such that  and , we can enforce  in the remaining sub-tree. Regarding item symmetries, we prefer to avoid changing the domain's definitions (as done for Formulation~(\ref{lp-model})) to keep the semantics of the global constraint unchanged. Instead, each time a bin is proved to be infeasible for an item, the same value is removed from all the ungrounded items of the same size.
 
\subsection{Search.} The propagation scheme relies on the knowledge of good upper bounds so the search process is driven by this need. We branch first on the bin variables  before the item's variables . We branch on the cheapest bins first by selecting the  with minimum  value and setting it to 1. The  variables are therefore sorted lexicographically by non-decreasing  values. Once all  variables are grounded, we select the open bin  with the smallest slope () and assign to it the largest item that can fit in a perfect packing of the bin. We thus check by dynamic programming that the item selected belongs to a subset of items that perfectly (up to its capacity) fill the bin. The search is binary, \emph{i.e.} on the left branch we enforce  and on the right branch we simply have .

\subsection{Propagating a stronger lower bound.} The lower bound  obtained by solving the linear relaxation of the cutting stock model can also be propagated by the constraint. However current restrictions of the domains have to be taken into account when computing the bound during search. The master and pricing problem are affected as we apply the bound by taking into account items already assigned (the capacity of the bins is reduced accordingly), and bins known to be open/close and currently possible items for each bin. The bound thus benefits from all other reasoning that acts on the domains. Since the pricing problem is time-consuming we found that the following techniques are important for efficiency:
\begin{itemize}
\item We keep the columns of the previous call to the cutting stock model that are still compatible (feasible) with respect to the current domains.
\item When solving the pricing problem for a bin, an upper bound is first computed by sorting the items from the most beneficial to the least and adding them greedily as long as the capacity is not overloaded. If the column obtained has a negative reduced cost, the dynamic programming algorithm is not called and the algorithm moves on. This is a very common technique in column generation that pays off when the pricing is costly and the master very easy to solve (because it usually increases the number of iterations). The dynamic programming algorithm is called at least once to prove that no negative reduced cost columns still exist and ensure the validity of the bound.
\end{itemize}


