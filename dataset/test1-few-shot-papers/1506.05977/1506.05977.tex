\documentclass{llncs}

\usepackage[utf8]{inputenc}
\usepackage{amsmath,amssymb}
\usepackage{xspace}
\usepackage{dsfont}
\usepackage[capitalise]{cleveref}
\usepackage[ruled,vlined]{algorithm2e}
\usepackage{graphicx}
\usepackage[caption=false,font=footnotesize]{subfig} 
\usepackage[T1]{fontenc}
\usepackage{longtable}
\usepackage{url}
\usepackage{color}
\usepackage{soul}

\usepackage{xcolor}\usepackage[inline,nomargin,author=]{fixme}
\fxsetface{inline}{\rmfamily}
\newcommand{\fix}[1]{{\fxfatal{\hl{#1}}}}




\newcommand{\ori}[1]{\smash{\overrightarrow{\!#1}}}
\newcommand{\todo}[2]{{#1}'s \fxnote{\hl{#2}}}
\newcommand{\loghole}{log-hole\xspace}

\newcommand{\hide}[1]{}

\frenchspacing

\title{Enumerating Cyclic Orientations of a Graph}

\author{Alessio Conte\inst{1} \and Roberto Grossi\inst{1} \and Andrea Marino\inst{1} \and Romeo Rizzi\inst{2}}
\institute{Universit\`a di Pisa, \email{conte,grossi,marino@di.unipi.it} \and Universit\`a di Verona, \email{rizzi@di.univr.it}}



\pagestyle{plain}


\begin{document}
\maketitle

\begin{abstract}
  Acyclic and cyclic orientations of an undirected graph have been
  widely studied for their importance: an orientation is acyclic if it
  assigns a direction to each edge so as to obtain a directed acyclic
  graph (DAG) with the same vertex set; it is cyclic otherwise. As far
  as we know, only the enumeration of acyclic orientations has been
  addressed in the literature. In this paper, we pose the problem of
  efficiently enumerating all the \emph{cyclic} orientations of an
  undirected connected graph with  vertices and  edges,
  observing that it cannot be solved using algorithmic techniques
  previously employed for enumerating acyclic orientations.  We show
  that the problem is of independent interest from both combinatorial
  and algorithmic points of view, and that each cyclic orientation can
  be listed with  delay time. Space usage is  with
  an additional setup cost of  time before the enumeration
  begins, or  with a setup cost of  time.
\end{abstract}


\section{Introduction}
\label{sec:intro}
Given an undirected graph  with  vertices and 
edges, an orientation transforms  into a directed graph 
by assigning a direction to each edge. That is, an orientation of 
is the directed graph  such that the vertex set
 is the same as , and the edge set  is an orientation
of : exactly one direction between  and
 holds for any undirected edge .  An
orientation  is \emph{acyclic} when  does not
contain any directed cycles, so  is a directed acyclic graph
(DAG); otherwise we say that the orientation  is cyclic.


Acyclic orientations of undirected graphs have been studied in
depth. Many results concern the number of such orientations:
Stanley~\cite{stanley87} shows how, given a graph  and its
chromatic number , the number of acyclic orientations of  can
be computed by using the chromatic polynomial (a special case of
Tutte's polynomial). Other results rely on the so called acyclic
orientation game: Alon et al.~\cite{alon1995acyclic} inquire about the
number of edges that have to be examined in order to identify an
acyclic orientation of a random graph ; Pikhurko~\cite{CPC:6776456}
gives an upper bound on this number of edges in general graphs.  The
counting problem is known to be \#P-complete~\cite{linial1986hard} and
enumeration algorithms that list all the acyclic orientations of a
graph are given in \cite{Barbosa199971} and \cite{Squire1998275}.

We consider \emph{cyclic} orientations, which have been also studied
from many points of view. Counting them is
co-\#P-complete~\cite{linial1986hard}. In Fisher et
al.~\cite{Fisher199773}, given a graph G and an acyclic orientation of
it, the number of \emph{dependent edges}, i.e. edges generating a
cycle if reversed, has been studied.
This number of edges implicitly gives a hint on the number of
cyclic orientations in a graph. 
In~\cite{Richardson:1996:DAD:2074284.2074338} an algorithm has been
given to make inference about causal structure in cyclic
graphs. In~\cite{Spirtes:1995:DCG:2074158.2074214} directed cyclic
graphs are used to model economic processes, and an algorithm is given
to characterize conditional independence constraints of these
processes.

In this paper we address the problem of enumerating all the cyclic
orientations of a graph.
\begin{problem}
  Enumerating the set of all the directed graphs  that are
  cyclic orientations of an undirected graph .
\label{problem1}
\end{problem}

We analyze the cost of an enumeration algorithm for
Problem~\ref{problem1} in terms of its \emph{setup} cost, meant as the
initialization time before the algorithm is able to lists the
solutions, and its \emph{delay} cost, which is the worst-case time
between any two consecutively enumerated solutions
(e.g. \cite{JohnsonP88}). We are interested in algorithms with
guaranteed  delay, where the  notation ignores
polylogs.

A naive solution to Problem~\ref{problem1} uses the fact that
enumeration algorithms exist for listing acyclic
orientations~\cite{Barbosa199971,Squire1998275}. It enumerates the
cyclic orientations by difference, namely, by enumerating all the
 possible edge orientations and removing the  acyclic
ones. However, this solution does not guarantee any polynomial delay, as
the number  if cyclic orientations can be much
larger or much smaller than the number  of acyclic ones.  For
example, a tree with  edges has  and . On the
other extreme of the situation, we have cliques. An oriented clique is
also called a \textit{tournament}, and a \textit{transitive
  tournament} is a tournament with no cycles. A clique of  nodes
(and  edges) can generate  different
tournaments, out of which exactly  will be transitive
tournaments~\cite{moon1968topics}. As  grows faster than , we
have that the ratio  tends to 0 for increasing ,
where .

To the best of our knowledge, an enumeration algorithm for
Problem~\ref{problem1} with guaranteed  delay is still
missing. We provide such an algorithm in this paper, namely, listing
each cyclic orientation with  delay time. Space usage is
 memory cells with a setup cost of  time, or 
memory cells with a setup cost of  time. Interestingly, 
Problem~\ref{problem1} reveals to be a rich source for enumerations techniques, and our solution
offers new combinatorial and algorithmic techniques when compared to
previous work on the enumeration of acyclic
orientations~\cite{Barbosa199971,Squire1998275}.

In the following, for the sake of clarity, we will call \emph{edges}
the elements of  (undirected graph) and \emph{arcs} the elements of
 (directed graph).  We will assume that the graph in input
 is connected and we will denote as  and 
respectively its number of nodes and edges.

The paper is organized as follows. Section~\ref{sec:overview}
 gives an overview of our enumeration algorithm. Section~\ref{sec:setup} describes the initialization steps, and shows how to reduce the problem from the input graph to a suitable multi-graph that guarantees to have a chordless cycle (hole) of logarithmic size. The latter is crucial to obtain the claimed delay.
Section~\ref{sec:enum} shows how to enumerate in the multigraph and obtain the cyclic orientations for the input graph. Section~\ref{sec:absorb} describes how to absorb the setup cost using more space. Finally, some conclusions are drawn in Section~\ref{sec:conclusions}.



\section{Algorithm overview}
\label{sec:overview}

The intuition behind our algorithm for an undirected connected graph  is the following one. Suppose that  is cyclic, otherwise there are no cyclic orientations. Consider one cycle\footnote{This will actually be a chordless cycle of logarithmic size (called \loghole).}  in : we can orient its edges in two ways so that the resulting  is a directed cycle. At this point, \emph{any} orientation of the remaining edges, e.g. those in , will give a cyclic orientation of . Thus, the interesting cases are when the resulting  is not a directed cycle. 

The idea is first to generate all possible orientations of the edges in , and then assign some suitable orientations to the edges in . This guarantees that we have at least two solutions for each orientation of , namely, setting the orientation of  so that  is one of the two possible directed cycles. Yet this is not enough as we could have a cyclic orientation even if  is \emph{not} a directed cycle. 

Therefore we must consider some cases. One easy case is that the orientation of  already produces a directed cycle: any orientation of  will give a cyclic orientation of . Another easy case, as seen above, is for the two orientations of  such that  is a directed cycle:  any orientation of  will give a cyclic orientation of . It remains the case when the orientations of both  and  are individually acyclic: when put together, we might have or not a directed cycle in the resulting orientation of . To deal with the latter case, we need to ``massage''  and transform it into a multigraph as follows. We refer the reader to Algorithm~\ref{alg:structure}. 

\textbf{Algorithm setup} is performed as described in Section~\ref{sec:setup}. We preprocess  with dead-ends removal and edge chain compression. The result is an undirected connected multigraph , where the edges are labeled as \emph{simple} and \emph{chain}. After that we find a chordless cycle of logarithmic size  in , and remove  from , obtaining the labeled multigraph , where .  

\textbf{Enumerating cyclic orientations} described in Section~\ref{sec:enum} exploits the property (which we will show later) that finding cyclic orientations of  corresponds to finding particular orientations in , called \emph{extended cyclic orientations}, and of , called \emph{legal orientations}. In the \textbf{for} loop, these orientations of  and  are enumerated so as to find all the cyclic orientations of . As we will see for the latter task, it is important to have  of logarithmic size to guarantee our claimed delay.

\begin{algorithm}[t]
\KwIn{An undirected connected graph }
\KwOut{All the cyclic orientations }
\smallskip
\textbf{Algorithm setup (Section~\ref{sec:setup}):}\\
Remove dead-ends (nodes of degree ) recursively from \\
 replace 's maximal paths of degree-2 nodes with chain edges\\
 a \loghole of \\
 delete the edges of  from , i.e.  \3mm]
\setcounter{AlgoLine}{0}
  \SetKwProg{myproc}{Procedure}{}{}
  \myproc{\proc{}}{
    


    \eIf{ }
        {output  and its set  of broken edges}
        {
            , \:  
            
             updated by adding the arc \;
            \If{ is cyclic or has positive reachability test on }{
                \proc (\\
            }
            
             updated adding the arc \;
            \If{ is cyclic or has positive reachability test on }{
                \proc (\\
            }
            
            \If{ is a chain edge}
            {
            \If{ is cyclic or has positive reachability test on }{
                \proc (\\ 
                }
            }
        }
}
\caption{Returning all legal orientations of }
\label{alg:ternpart}
\end{algorithm}










\begin{lemma}[Correctness]
\begin{enumerate}
\item All the extended cyclic orientations of  are output. 
\item Only extended cyclic orientations of  are output.
\item There are no duplicates.
\end{enumerate}
\end{lemma}
\begin{proof}
\begin{enumerate}
\item Any extended cyclic orientation  can be seen as the union  of  and , which are two edge disjoint directed subgraphs. Our algorithm enumerates all the extended orientations of , and, for each of them, all legal extended orientations : if there is a cycle in  involving only edges in  all the extended orientations of  are legal; otherwise just the extended orientations  of  whose arcs create a cycle in  are legal. Hence any extended cyclic orientation  is output.
\item Any output solution is an extended orientation: each edge in  and in  has exactly one direction or is broken. Moreover, any output solution contains at least a cycle: if there is not a cycle in , a cycle is created involving the edges in .
\item All the extended orientations  in output differ for at least an arc in  or an arc in . Hence there are no duplicate solutions.
\end{enumerate}
\qed
\end{proof}

As a result, we obtain delay .
\begin{lemma}
The extended cyclic orientations of  can be enumerated with delay  and space .
\label{lem:delay2}
\end{lemma}
\begin{proof}
Finding extended orientations  of  can be done with  delay. Every time a new  has been generated, we apply Algorithm~\ref{alg:ternpart}. By Lemma~\ref{lem:delaytern} we output the first cyclic orientation  of  with delay  and the remaining ones with delay . Hence the maximum delay between any two consecutive solutions is . The space usage is linear in all the phases: in particular in Algorithm~\ref{alg:ternpart} the space is , because of the reachability matrix , which is smaller than .
\qed
\end{proof}

Applying \Cref{lem:delay2} and \Cref{lem:ext}, and considering the setup cost in Section~\ref{sec:setup} ( and ), we can conclude as follows.

\begin{theorem}
\label{the:delay}
Algorithm~\ref{alg:structure} lists all cyclic orientations of  with setup cost ,
and delay . The space usage is  memory cells.
\end{theorem}


\section{Absorbing the setup cost}
\label{sec:absorb}

In this section, we show how to modify our approach to get a setup time equal to the delay, requiring space .

\begin{theorem}
\label{the:absorb}
All cyclic orientations of  can be listed with setup cost , delay , and space usage of  memory cells.
\end{theorem}

We use  and  for brevity.  Let  be the
following algorithm that takes  time to generate 
solutions, each with  delay, starting from any given
cycle of size . This cycle is found by performing a BFS
on an arbitrary node , and identifying the shortest cycle 
containing . Note that  is a \loghole as required. Now, if
, we stop the setup and run the algorithms in the
previous sections setting . The case of interest in this
section is when . We take a cyclic orientation
 of , and then  arbitrary orientations of the edges
in .  The setup cost is  time and we can easily
output each solution in  delay. We denote this set of
 solutions by .

Also, let  be the algorithm behind Theorem~\ref{the:delay}, with a setup cost of  and   delay (i.e. Algorithm~\ref{alg:structure}). We denote the time taken by  to list the first  solutions, including the  setup cost, by , and this set of  solutions by . Since  and  can have nonempty intersection, we want to avoid duplicates. 

We show how to obtain an algorithm  that lists all the cyclic orientations without duplicates with  setup cost and delay, using  space. Even though the delay cost of  is larger than that of  and  by a constant factor, the asymptotic complexity is not affected by this constant, and remains . 

Algorithm  executes simultaneously and independently the two algorithms  and . Recall that these two algorithms take  time in total to generate  and  with   delay. However those in  are produced after a setup cost of .  Hence  slows down on purpose by a constant factor , thus requiring  time: it has time to find the distinct solutions in  and build a dictionary  on the solutions in . (Since an orientation can be represented as a binary string of length , a binary trie can be employed as dictionary , supporting each dictionary operation in  time.) During this time,  outputs the  solutions from  with a delay of  time each, while storing the rest of solutions of  in a buffer . 

After  time, the situation is the following: Algorithm  has output the  solutions in  with  setup cost and delay. These solutions are stored in , so we can check for duplicates. We have buffered at most  solutions of  in . Now the purpose of  is to continue with algorithm  alone, with  delay per solution, avoiding duplicates. Thus for each solution given by , algorithm  suspends  and waits so that each solution is output in  time: if the solution is not in ,  outputs it; otherwise  extracts one solution from the buffer  and outputs the latter instead. Note that if there are still  duplicates to handle in the future, then  contains exactly  solutions from   (and  is empty when  completes its execution). Thus,  never has to wait for a non-duplicated solution. The delay is the maximum between  and the delay of , hence . The additional space is dominated by that of , namely,  memory cells to store up to  solutions.


We also have an amortized cost using the lemma below, where  and . 


\begin{lemma}
Listing all the extended cyclic orientations of  with delay  and setup cost  implies that the average cost per solution is .
\label{lem:preproc}
\end{lemma}
\begin{proof}
We perform a BFS on an arbitrary node , and identify the shortest cycle  that contains . This costs  time. 
Note that  is a hole (i.e. it has no chords). Note that a minimum cycle in  either is  or contains a node in : hence we perform all the BFSs from each node in , as explained in~\cite{itai1978finding} with an overall cost of . The number of extended orientations of  is at least .
Our setup cost is , with , and the number of solutions is at least . The overall average cost per solution is at most

\qed
\end{proof}

\section{Conclusions}
\label{sec:conclusions}

In this paper we considered the problem of efficiently enumerating cyclic orientations of graphs. The problem is interesting from a combinatorial and algorithmic point of view, as the fraction of cyclic orientations over all the possible orientations can be as small as 0 or very close to 1. We provided an efficient algorithm to enumerate the solutions with delay  and overall complexity , with  being the number of solutions.

\bibliographystyle{abbrv}
\begin{thebibliography}{10}

\bibitem{alon1995acyclic}
N.~Alon and Z.~Tuza.
\newblock The acyclic orientation game on random graphs.
\newblock {\em Random Structures \& Algorithms}, 6(2-3):261--268, 1995.

\bibitem{Barbosa199971}
V.~C. Barbosa and J.~L. Szwarcfiter.
\newblock Generating all the acyclic orientations of an undirected graph.
\newblock {\em Information Processing Letters}, 72(1):71 -- 74, 1999.

\bibitem{B78}
B.~Bollobas.
\newblock {\em Extremal Graph Theory}.
\newblock Dover Publications, Incorporated, 2004.

\bibitem{EP62}
P.~Erd{\H{o}}s and L.~P{\'o}sa.
\newblock On the maximal number of disjoint circuits of a graph.
\newblock {\em Publ. Math. Debrecen}, 9:3--12, 1962.

\bibitem{Fisher199773}
D.~C. Fisher, K.~Fraughnaugh, L.~Langley, and D.~B. West.
\newblock The number of dependent arcs in an acyclic orientation.
\newblock {\em Journal of Combinatorial Theory, Series B}, 71(1):73 -- 78,
  1997.

\bibitem{itai1978finding}
A.~Itai and M.~Rodeh.
\newblock Finding a minimum circuit in a graph.
\newblock {\em SIAM Journal on Computing}, 7(4):413--423, 1978.

\bibitem{JohnsonP88}
D.~S. Johnson, C.~H. Papadimitriou, and M.~Yannakakis.
\newblock On generating all maximal independent sets.
\newblock {\em Inf. Process. Lett.}, 27(3):119--123, 1988.

\bibitem{linial1986hard}
N.~Linial.
\newblock Hard enumeration problems in geometry and combinatorics.
\newblock {\em SIAM Journal on Algebraic Discrete Methods}, 7(2):331--335,
  1986.

\bibitem{moon1968topics}
J.~Moon.
\newblock {\em Topics on tournaments}.
\newblock Athena series: Selected topics in mathematics. Holt, Rinehart and
  Winston, 1968.

\bibitem{CPC:6776456}
O.~Pikhurko.
\newblock Finding an unknown acyclic orientation of a given graph.
\newblock {\em Combinatorics, Probability and Computing}, 19:121--131, 1 2010.

\bibitem{Richardson:1996:DAD:2074284.2074338}
T.~Richardson.
\newblock A discovery algorithm for directed cyclic graphs.
\newblock In {\em Proceedings of the Twelfth International Conference on
  Uncertainty in Artificial Intelligence}, UAI'96, pages 454--461, San
  Francisco, CA, USA, 1996. Morgan Kaufmann Publishers Inc.

\bibitem{Spirtes:1995:DCG:2074158.2074214}
P.~Spirtes.
\newblock Directed cyclic graphical representations of feedback models.
\newblock In {\em Proceedings of the Eleventh Conference on Uncertainty in
  Artificial Intelligence}, UAI'95, pages 491--498, San Francisco, CA, USA,
  1995. Morgan Kaufmann Publishers Inc.

\bibitem{Squire1998275}
M.~B. Squire.
\newblock Generating the acyclic orientations of a graph.
\newblock {\em Journal of Algorithms}, 26(2):275 -- 290, 1998.

\bibitem{stanley87}
R.~Stanley.
\newblock Acyclic orientations of graphs.
\newblock In I.~Gessel and G.-C. Rota, editors, {\em Classic Papers in
  Combinatorics}, Modern Birkh\"auser Classics, pages 453--460. Birkh\"auser
  Boston, 1987.

\end{thebibliography}

\end{document}
