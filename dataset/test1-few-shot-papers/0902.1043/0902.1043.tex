
\documentclass{article}
\usepackage{graphicx}
\usepackage{amssymb,amsmath,mathrsfs,amsthm}
\usepackage{subfigure}
\usepackage{xspace}
\usepackage{enumerate}


\newtheorem{theorem}{Theorem}
\newtheorem{observation}[theorem]{Observation}
\newtheorem{definition}[theorem]{Definition}
\newtheorem{proposition}[theorem]{Proposition}
\newtheorem{lemma}[theorem]{Lemma}


\newcommand{\AAA}{\mathcal{A}} \newcommand{\BBB}{\mathcal{B}}
\newcommand{\CCC}{\mathcal{C}} \newcommand{\DDD}{\mathcal{D}}
\newcommand{\EEE}{\mathcal{E}} \newcommand{\FFF}{\mathcal{F}}
\newcommand{\GGG}{\mathcal{G}} \newcommand{\HHH}{\mathcal{H}}
\newcommand{\III}{\mathcal{I}} \newcommand{\JJJ}{\mathcal{J}}
\newcommand{\KKK}{\mathcal{K}} \newcommand{\LLL}{\mathcal{L}}
\newcommand{\MMM}{\mathcal{M}} \newcommand{\NNN}{\mathcal{N}}
\newcommand{\OOO}{\mathcal{O}} \newcommand{\PPP}{\mathcal{P}}
\newcommand{\QQQ}{\mathcal{Q}} \newcommand{\RRR}{\mathcal{R}}
\newcommand{\SSS}{\mathcal{S}} \newcommand{\TTT}{\mathcal{T}}
\newcommand{\UUU}{\mathcal{U}} \newcommand{\VVV}{\mathcal{V}}
\newcommand{\WWW}{\mathcal{W}} \newcommand{\XXX}{\mathcal{X}}
\newcommand{\YYY}{\mathcal{Y}} \newcommand{\ZZZ}{\mathcal{Z}}

\newcommand{\varsubscript}[2]{{#1}_{\mbox{\footnotesize #2}}}
\newcommand{\BC}{{\mathcal B}^+}
\newcommand{\BT}{{\mathcal B}^\div}
\newcommand{\portaledges}{\varsubscript{E}{portal}}
\newcommand{\outerface}{\varsubscript{f}{out}}
\newcommand{\MGthin}{\varsubscript{\MG}{thin}}
\newcommand{\preprocess}{\mbox{\sc Preprocess}}
\newcommand{\planarize}{\mbox{\sc Planarize}}
\newcommand{\thinning}{\mbox{\sc Thinning}}

\newcommand{\longvar}[1]{\mathop{\mathrm{#1}}\nolimits}
\newcommand{\OPT}{\longvar{OPT}}
\newcommand{\LB}{\longvar{LB}}
\newcommand{\PB}{\longvar{PB}}
\newcommand{\MG}{\longvar{MG}}
\newcommand{\CG}{\longvar{CG}}
\newcommand{\SPT}{\longvar{SPT}}

\newcommand{\tw}{\operatorname{tw}}
\newcommand{\ltw}{\operatorname{ltw}}
\newcommand{\ltwr}{\operatorname{ltw}_r}
\renewcommand{\deg}{\operatorname{deg}}
\newcommand{\dist}{\operatorname{dist}}
\newcommand{\ratio}{\operatorname{ratio}}
\newcommand{\goal}{\operatorname{opt}}

\newcommand{\vrt}[1]{V(#1)}
\newcommand{\vrtA}{\vrt{A}}
\newcommand{\vrtB}{\vrt{B}}
\newcommand{\vrtD}{\vrt{D}}
\newcommand{\vrtH}{\vrt{H}}
\newcommand{\vrtG}{\vrt{G}}
\newcommand{\vrtT}{\vrt{T}}
\newcommand{\edge}[1]{E(#1)}
\newcommand{\edgeA}{\edge{A}}
\newcommand{\edgeB}{\edge{B}}
\newcommand{\edgeG}{\edge{G}}
\newcommand{\edgeD}{\edge{D}}
\newcommand{\edgeH}{\edge{H}}
\newcommand{\edgeT}{\edge{T}}

\newcommand{\specialgraph}[2]{\mathrm{\bold #1}_{#2}}
\newcommand{\Kn}[1]{\specialgraph{K}{#1}}
\newcommand{\Kmn}[2]{\specialgraph{K}{#1,#2}}
\newcommand{\Pathn}[1]{\specialgraph{P}{#1}}
\newcommand{\Cyclen}[1]{\specialgraph{C}{#1}}
\newcommand{\Gin}{\varsubscript{G}{in}}
\newcommand{\Gthin}{\varsubscript{G}{thin}}
\newcommand{\Gspan}{\varsubscript{G}{span}}

\newcommand{\Oof}{\mathcal{O}} 
\newcommand{\OHof}{\mathcal{O}_H} 
\newcommand{\oeff}{o^{\mbox{\tiny eff}}}
\newcommand{\set}[1]{\ensuremath{\{#1\}}}
\newcommand{\length}[1]{\ensuremath{\ell(#1)}}
\newcommand{\ceil}[1]{\left\lceil#1\right\rceil}
\newcommand{\floor}[1]{\left\lfloor#1\right\rfloor}
\renewcommand{\bar}[1]{\overline{#1}}
\renewcommand{\hat}[1]{\widehat{#1}}
\renewcommand{\epsilon}{\varepsilon}
\renewcommand{\emptyset}{\varnothing} 

\newcommand{\CompClass}[1]{\ensuremath{\mathsf{#1}}\xspace}
\newcommand{\NP}{\CompClass{NP}}
\newcommand{\NPO}{\CompClass{NPO}}
\newcommand{\PTIME}{\CompClass{P}}
\newcommand{\PSPACE}{\CompClass{PSPACE}}
\newcommand{\APX}{\CompClass{APX}}
\newcommand{\RP}{\CompClass{RP}}
\newcommand{\PTAS}{\CompClass{PTAS}}
\newcommand{\EPTAS}{\CompClass{EPTAS}}
\newcommand{\FPTAS}{\CompClass{FPTAS}}
\newcommand{\FPT}{\CompClass{FPT}}
\newcommand{\EPT}{\CompClass{EPT}}
\newcommand{\EXPT}{\CompClass{EXPT}}
\newcommand{\SUBEPT}{\CompClass{SUBEPT}}
\newcommand{\ETH}{\CompClass{ETH}}
\newcommand{\XP}{\CompClass{XP}}
\newcommand{\WONE}{\CompClass{W[1]}}
\newcommand{\WTWO}{\CompClass{W[2]}}
\newcommand{\WP}{\CompClass{W[P]}}
\newcommand{\poly}{\ensuremath{\mathrm{poly}}}
\newcommand{\opoly}{n^{\Oof(1)}}

\newcommand{\myproblemname}[1]{\ensuremath{\mbox{\sc #1}}\xspace}
\newcommand{\steiner}{\myproblemname{Steiner Tree}}
\newcommand{\steinerforest}{\myproblemname{Steiner Forest}}
\newcommand{\tsp}{\myproblemname{Tsp}}
\newcommand{\subtsp}{\myproblemname{Subset Tsp}}
\newcommand{\survive}{\myproblemname{Survivable Network}}




\begin{document}

\title{Polynomial-time approximation schemes for subset-connectivity
    problems in bounded-genus graphs}



\author{Glencora Borradaile \and
        Erik D.\ Demaine \and
        Siamak Tazari
}









\maketitle

\begin{abstract}
  We present the first polynomial-time approximation schemes (PTASes)
  for the following subset-connectivity problems in edge-weighted
  graphs of bounded genus: Steiner tree, low-connectivity
  survivable-network design, and subset TSP.  The schemes run in  time for graphs embedded on both orientable and
  nonorientable surfaces.  This work generalizes the PTAS frameworks
  of Borradaile, Klein, and Mathieu (2007) from
  planar graphs to bounded-genus graphs: any future problems shown to
  admit the required structure theorem for planar graphs will
  similarly extend to bounded-genus graphs.
\end{abstract}




\section{Introduction}

In many practical scenarios of network design, input graphs
have a natural drawing on the sphere or equivalently the plane.
In most cases, these embeddings have few crossings, either to avoid digging
multiple levels of tunnels for fiber or cable or to avoid building overpasses
in road networks.  But a few crossings are common, and can easily come in
bunches where one tunnel or overpass might carry several links or roads.
Thus we naturally arrive at graphs of small (bounded) genus,
which is the topic of this work.

We develop a  framework for subset-connectivity problems on
edge-weighted graphs of bounded genus.  In general, we are given a
subset of the nodes, called \emph{terminals}, and the goal is to
connect the terminals together with some substructure of the graph by
using cost within  of the minimum possible cost.  Our
framework applies to three well-studied problems in this framework.
In \steiner, the substructure must be connected, and thus forms a
tree.  In \subtsp, the substructure must be a cycle; to guarantee
existence, the cycle may traverse vertices and edges multiple times,
but pays for each traversal.  In -edge-connectivity
\survive, the substructure must have  edge-disjoint
paths connecting vertices  and~, where each ; we allow the substructure to include multiple copies of an
edge in the graph, but pay for each copy.  In particular, if 
for all terminals  and~, then we obtain the Steiner tree
problem; if  for all terminals  and~, then we obtain
the minimum-cost -edge-connected multi-subgraph problem.

Our framework yields the first  for all of these problems in
bounded-genus graphs.  These es are efficient, running in
 time for graphs embedded on orientable surfaces and
nonorientable surfaces.  (We usually omit the mention of
 and  by assuming  and  are
constant, but we later bound  as singly exponential in
a polynomial in~ and  and  as singly exponential
in .)  In contrast, the problems we consider are -complete
(and constant-factor-approximable) for general graphs.

We build upon the recent  framework of Borradaile, Klein, and
Mathieu~\cite{BorradaileKM09} for subset-connectivity problems on
planar graphs.  In fact, our result is strictly more general: any
problem to which the previous planar-graph framework applies
automatically works in our framework as well, resulting in a  for
bounded-genus graphs.  For example, Borradaile and
Klein~\cite{BorradaileKlein08} have recently given a  for the
-edge-connectivity \survive problem
using the planar framework.  This will imply a similar result in
bounded genus graphs.  In contrast to the planar-graph framework, our
es have the attractive feature that they run correctly on all
graphs with the performance degrading with the genus.

Our techniques for attacking bounded-genus graphs include two recent
results: decompositions into bounded-treewidth graphs via
contractions~\cite{DemaineHM07} and fast algorithms for finding the
shortest noncontractible cycle~\cite{CabelloChambers07}.  We also use
a simplified version of an algorithm for finding a short sequence of
loops on a topological surface~\cite{EricksonWhittlesey05}, and
sophisticated dynamic programming. Our aim is to prove the following theorem:

\begin{theorem}\label{thm:main_ptas} 
  There exists a  for the \steiner, \subtsp, and
  -edge-connected \survive problems in edge-weighted graphs of
  genus  with running time .
\end{theorem}




\section{Preliminaries}

All graphs  have  vertices,  edges and are undirected
with edge lengths (weights).  The length of an edge , subgraph , and set
of subgraphs  are denoted ,  and
, respectively.  The shortest distance between
vertices  and  in a graph  is denoted .  The
boundary of a graph  embedded in the plane is denoted by . For an edge , we define the operation of \emph{contracting}
 as identifying  and  and removing all loops and duplicate edges.

We use the basic terminology for embeddings as outlined
in~\cite{MoharThomassen01}. In this paper, an embedding refers to a
\emph{2-cell embedding}, i.e. a drawing of the vertices and faces of
the graph as points and arcs on a surface such that every face is
homeomorphic to an open disc. Such an embedding can be described
purely combinatorially by specifying a \emph{rotation system}, for the
cyclic ordering of edges around vertices of the graph, and a
\emph{signature} for each edge of the graph; we use this notion of a
\emph{combinatorial embedding}. A combinatorial embedding of a graph
 naturally induces such a 2-cell embedding on each subgraph of
. We only consider compact surfaces without boundary. When we refer
to a planar embedding, we actually mean an embedding in the 2-sphere.
If a surface contains a subset homeomorphic to a M\"obius strip, it is
\emph{nonorientable}; otherwise it is \emph{orientable}. For a 2-cell
embedded graph  with  facial walks, the number  is called the Euler genus of the surface. The Euler genus is equal
to twice the usual genus for orientable surfaces and equals the usual
genus for nonorientable surfaces. The \emph{dual} of an embedded
graph  is defined as having the set of faces of  as its vertex
set and having an edge between two vertices if the corresponding faces
of  are adjacent.  We denote the dual graph by  and
identify each edge of  with its corresponding edge in .  A
cycle of an embedded graph is \emph{contractible} if it can be
continuously deformed to a point; otherwise it is
\emph{noncontractible}. The operation of \emph{cutting along a
  2-sided cycle}  is essentially: partition the edges adjacent to
 into left and right edges and replace  with two copies 
and , adjacent to the left or right edges, accordingly. The
inside of these new cycles is ``patched'' with two new faces. If the
resulting graph is disconnected, the cycle is called
\emph{separating}, otherwise \emph{nonseparating}.  Cutting along a
1-sided cycle  on nonorientable surfaces is defined similarly,
only that  is replaced by one bigger cycle  that contains every
edge of  exactly twice.  See~\cite[pages
  105--106]{MoharThomassen01} for further technical details.

Next we define the notions related to treewidth as introduced by
Robertson and Seymour~\cite{GraphMinors02}. A \emph{tree
  decomposition} of a graph  is a pair , where
 is a tree and  is a family of
subsets of , called \emph{bags}, such that
\begin{enumerate} \item every vertex of  appears in some bag of ;
\item for every edge  of , there exists a bag that contains
  both  and ;
\item for every vertex  of , the set of bags that contain 
  form a connected subtree  of .
\end{enumerate}
The \emph{width} of a tree decomposition is the maximum size of a bag
in  minus . The \emph{treewidth} of a graph , denoted by
, is the minimum width over all possible tree decompositions
of . 

The input graph is  and has genus ; the
terminal set is . We assume  is equipped with a combinatorial
embedding; such an embedding can be found in linear time, if the genus
is known to be fixed, see~\cite{Mohar99}. Let  be the considered
subset-connectivity problem. In Section~\ref{sec:preprocess}, we show
how to find a subgraph  of , so that for  any -approximate solution of  in
 also exists in . Hence, we may use  instead of 
in the rest of the paper. Note that as a subgraph of ,  is
automatically equipped with a combinatorial embedding.

Let  denote the length of a optimal Steiner tree spanning
terminals~. We define  to be the length of an
optimal solution to problem .  For the problems that we
solve, we require that  and in
particular that .  The
constant  will be used in Section~\ref{sec:genus-mg} and is equal
to  for both the subset TSP and -edge-connectivity
problems.  This requirement is also needed for the planar case;
see~\cite{BorradaileKlein08}.  Because , upper bounds
in terms of  hold for all the problems herein.  As a result, we
can safely drop the  subscript throughout the paper.

We show how to obtain a  solution for
a fixed constant .  To obtain a 
solution, we can simply use  as input to the
algorithm.



\section{Mortar Graph and Structure Theorem}
\label{sec:mg}

In~\cite{BorradaileKM09}, Borradaile, Klein
and Mathieu developed a  for the Steiner tree problem in planar
graphs.  The method involves finding a grid-like subgraph called the
{\em mortar graph} that spans the input terminals and has length
.  The set of feasible Steiner trees is restricted to those
that cross between adjacent faces of the mortar graph only at a small
number (per face of the mortar graph) of pre-designated vertices
called {\em portals}.  A Structure Theorem guarantees the existence of
a nearly optimal solution (one that has length at most
) in this set.  We review the details that are
relevant to this work and generalize to genus- graphs.

Here we define the mortar graph in such a way that generalizes to
higher genus graphs.  A path  in a graph  is {\em
  -short in } if for every pair of vertices  and 
on , the distance from  to  along  is at most
 times the distance from  to  in :
.  Given a graph 
embedded on a surface and a set of terminals , a mortar graph is a
subgraph of  with the following properties:

\begin{definition}[Mortar Graph and Bricks]\label{def:mg}
  Given a graph  embedded on a surface of genus , a set of
  terminals , and a number , consider a
  subgraph  of  spanning  such that
  each facial walk of  encloses an area homeomorphic to an open
  disk. For each face  of , we construct a \emph{brick}  of
   by cutting  along the facial walk ;  is the
  subgraph of  embedded inside the face, including . We
  denote this facial walk as the \emph{mortar boundary} 
  of . We define the \emph{interior} of  as  without the
  edges of . We call  a \emph{mortar graph} if for
  some constants  and  (to be
  defined later), we have  and every brick 
  satisfies the following properties:
  \begin{enumerate} \item  is planar.
  \item The boundary of  is the union of four paths in the
    clockwise order
    , , , .
  \item Every terminal of  that is in  is on  or on .
  \item  is 0-short in , and every proper subpath of  is
    -short in .
  \item There exists a number  and vertices  ordered from left to right along  such
    that, for any vertex  of , the distance from
     to  along  is less than  times the
    distance from  to  in : .
  \end{enumerate}  
\end{definition}

\begin{figure}
  \centering
  \subfigure[]{\includegraphics[scale=0.9]{BD-gin}}\hfil\hfil
  \subfigure[]{\includegraphics[scale=0.9]{BD-MG}}\hfil\hfil
  \subfigure[]{\includegraphics[scale=0.9]{BD-bricks}}\hfil\hfil
  \subfigure[]{\includegraphics[scale=0.9]{BD-PCG}}\hfil\hfil
  \subfigure[]{\includegraphics[scale=0.9]{BD-brick-contract}}
  \caption{(a)~An input graph  with mortar graph  given by bold
    edges in (b). (c)~The set of bricks corresponding to  (d) A
    portal-connected graph, . The portal edges are
    grey.  (e)~ with the bricks contracted, resulting in
    .  The dark vertices are brick vertices. }
  \label{fig:mg} 
\end{figure} 

The mortar graph and the set of bricks are illustrated in
Figures~\ref{fig:mg}~(a),~(b) and~(c).  Constructing the mortar graph
for planar graphs first involves finding a 2-approximate Steiner tree
~\cite{Mehlhorn88} and cutting open the graph along  creating a new face  and then: 

\begin{enumerate} \item Finding shortest paths between certain vertices of .  These paths
  result in the  and  boundaries of the bricks.\label{mg:step-3}
\item Finding shortest paths between vertices of the paths found in
  Step~\ref{mg:step-3}.  These paths are called {\em columns}, do not
  cross each other, and have a natural order. \label{mg:step-4} 
\item
  Taking every th path found in Step~\ref{mg:step-4}.  These
  paths are called {\em supercolumns} and form the  and 
  boundaries of the bricks. We sometimes refer to  as the
  \emph{spacing} of the supercolumns.
  \label{mg:step-5} 
\end{enumerate} 

The mortar graph is composed of the edges of  (equivalently, )
and the edges found in Steps~\ref{mg:step-3} and~\ref{mg:step-5}.
In~\cite{BorradaileKM09}, it is shown that the total length of the
mortar graph edges is at most . For the purposes of this paper,
we bound the length of the mortar graph in terms of .  The
following theorem can be easily deduced from~\cite{Klein06}
and~\cite{BorradaileKM09}:

\begin{theorem}[\cite{Klein06,BorradaileKM09}] \label{thm:planar-mg}
  Let  and  be a planar graph with outer face
   containing the terminals  and such that
  , for some constant
  . For , there is a
  mortar graph  containing  whose
  length is at most  and whose supercolumns have length
  at most  with spacing .  The mortar graph can be
  found in  time.
\end{theorem} 

\subsection{A mortar graph for bounded-genus graphs: Overview}\label{sec:genus-mg}

We use Theorem~\ref{thm:planar-mg} to prove the existence of a mortar graph for
genus- embedded graphs. This section is devoted to
proving the following theorem: 

\begin{theorem}
  \label{thm:genus-mortar-graph} Let an embedded edge-weighted
  graph  of Euler genus , a subset of its vertices , an , and  be given. For , there is a mortar graph  of
   such that the length of  is  and the
  supercolumns of  have length  with spacing
  . The mortar
  graph can be found in  time.
\end{theorem}

Let  be the input graph of genus  and  be the
terminal set. In a first preprocessing step, we delete a number of
unnecessary vertices and edges of  to obtain a graph  of
genus  that still contains every
-approximate solution for terminal set  for all  while fulfilling certain bounds on the length of
shortest paths. In the next step, we find a \emph{cut graph}  of
 that contains all terminals and whose length is bounded by a
constant times . We cut the graph open along , so that it
becomes a planar graph with a simple cycle  as boundary, where
the length of  is twice that of . See
Figure~\ref{fig:cutgraph} for an illustration. Afterwards, the
remaining steps of building the mortar graph can be the same as in the
planar case, by way of Theorem~\ref{thm:planar-mg}.

For an edge  in , we let 
 
and say that 
is \emph{at distance}  from . If the root vertex
represents a contracted graph , we use the same terminology with
respect to .

\subsection{Preprocessing the input graph} \label{sec:preprocess}

Our first step is to apply the following preprocessing procedure:

\begin{center} \fbox{
    \begin{minipage}[h]{0.99\linewidth}
      \noindent\textbf{Algorithm . } \\
      \begin{tabular}{ll}
        \textit{Input. } & an arbitrary graph , terminals , a constant \\
        \textit{Output. } & a preprocessed subgraph of \\
      \end{tabular} \-2ex]
      \begin{enumerate} \item Apply  and let  be the obtained subgraph.
      \item Find a 2-approximate Steiner tree  for  and contract it to a  vertex .
      \item Find a shortest paths tree  rooted at .
      \item Uncontract  and set . {\em ( is a
        spanning tree of )}
      \item Find a spanning tree  in . {\em ( is a spanning tree of )}
      \item Let .
      \item Return  together with .
      \end{enumerate}
    \end{minipage} }
\end{center}

\begin{lemma} \label{lem:cut-graph} The algorithm 
  returns a cut graph  such that cutting
   open along  results in a planar graph  with a face
   whose facial walk 
\begin{enumerate} \item[(i)] is a simple cycle; 
\item[(ii)] contains all terminals (some terminals might appear more than
  once as multiple copies might be created during the cutting process); and
\item[(iii)] has length .
\end{enumerate}
The algorithm can be implemented in linear time.
\end{lemma}
\begin{proof}
  Clearly,  is tree-cotree decomposition of  and
  so, by Lemma~\ref{lem:eppstein},  is a cut graph. By
  Euler's formula, we get that , the Euler genus of . 

  Each
  edge  completes a (nonseparating,
  not necessarily simple) closed walk as follows: a shortest path  from
   to , the edge , a shortest path  from  to 
  and possibly a path  in . By
  Proposition~\ref{prop:preprocess}, we know that  is at distance
  at most  from  and so, both  and , and at least one of
   have length at most . Hence, we have that . Because there are (exactly)  such cycles in , we get
  that
  
  Since  is a connected cut graph and , cutting  open along  results in a connected
  planar graph with boundary .  Each edge of  appears
  twice in  and each edge of  is derived from ,
  so  (see
  Fig.~\ref{fig:cutgraph}).



  As mentioned in the previous section,  and  can be computed in
  linear time on bounded-genus
  graphs~\cite{HenzingerKRS97,TazariMuellerh09-DAM}.  can be obtained,
  for example, by a simple breadth-first-search in the dual. The remaining
  steps can also easily be implemented in linear time.\qed
\end{proof}

\subsection{Proof of Theorem~\ref{thm:genus-mortar-graph}} 

We complete the construction of a mortar graph for genus- embedded
graphs.

Let  be the result of planarizing  as guaranteed by
Lemma~\ref{lem:cut-graph}.   is a planar graph with
boundary  such that  spans  and has length .  Let  be the mortar graph guaranteed by
Theorem~\ref{thm:planar-mg} as applied to  with  as
its outer face.  Every edge of  corresponds to an edge of .
Let  be the subgraph of  composed of edges corresponding to
.  Every face  of  (other than ) corresponds to a
face  of  and the interior of  is homeomorphic to a disk
on the surface in which  is embedded.  It is easy to verify that
 is indeed a mortar graph of ; and the length bounds
specified in the statement of the theorem follow directly from
Theorem~\ref{thm:planar-mg} and the bound on the length of
. \qed

\subsection{Structure Theorem}\label{sec:structure-thm}

Along with the mortar graph, Borradaile et~al.~\cite{BorradaileKM09}
define an operation  called {\em brick-copy} that allows a
succinct statement of the Structure Theorem.  For each brick , a
subset of  vertices are selected as {\em portals} such that
the distance along  between any vertex and the closest
portal is at most .  For every brick ,
embed  in the corresponding face of  and connect every portal
of  to the corresponding vertex of  with a zero-length {\em
  portal edge}: this defines .   is
illustrated in Figure~\ref{fig:mg}~(d). We denote the set of all
portal edges by . The following simple observation, proved
in~\cite{BorradaileKM09} holds also for bounded-genus graphs:

\begin{observation}[\cite{BorradaileKM09}] \label{lem:soln} If  is a
  connected subgraph of , then  is a connected subgraph of  spanning the same vertices
  of .
\end{observation} 

The following Structure Theorem is the heart of the correctness of the
es.  

\begin{theorem}[Structure Theorem]
  Let  be one of the subset-connectivity problems \steiner,
  -edge-connectivity \survive, or \subtsp. Let  be
  an edge-weighted graph embedded on a surface,  a
  given set of terminals, and . Let
   be a corresponding mortar graph of weight at
  most  and supercolumns of weight at most  with spacing . There exist constants
   and  depending
  polynomially on  and  such that
   where  is an absolute constant.  Here  for \steiner and -edge
  connectivity \survive and  for \subtsp.
  (Recall that  and  depend polynomially on
   and  by Theorem~\ref{thm:genus-mortar-graph}.)
\end{theorem}

It is due to our special way of defining and constructing a mortar
graph for bounded-genus graphs that this theorem follows immediately
as for the planar cases: the crucial point here is that our bricks are
always planar -- even when the given graph is embedded in a surface of
higher genus. The Structure Theorem for \steiner is proved
in~\cite{BorradaileKM09}, the case of -edge-connectivity
\survive is studied in~\cite{BorradaileKlein08}, and we show that the
theorem holds for \subtsp in Section~\ref{sec:subtsp}. Note
  that for \subtsp, it is possible to obtain a singly exponential
  algorithm by following the spanner construction of
  Klein~\cite{Klein06} after performing the planarizing step
  (Lemma~\ref{lem:cut-graph}). Our presentation here is chosen to
  unify the methods for all problems studied.

The Structure Theorem essentially says that there is a constant
 depending polynomially on  such that in
finding a near-optimal solution to , we can restrict our attention
to . Whenever we wish to apply our framework to a
new problem, it is essential to prove a similar structure theorem for
the considered problem.

\section{Obtaining es for bounded-genus graphs}

We present two methods of obtaining polynomial-time approximation
schemes.  The first is a generalization of the framework of
Klein~\cite{Klein06} for planar graphs that is based on finding a
\emph{spanner} for a problem, a subgraph containing a nearly optimal
solution having length .  In Section~\ref{sec:spanner} we
show how to find such a spanner and in Section~\ref{sect:ptas_spanner}
we generalize Klein's framework to higher genus graphs using the
techniques of Demaine et~al.~\cite{DemaineHM07}.  In the second
method, dynamic programming is done over the bricks of the mortar
graph.  This generalizes the framework of Borradaile
et~al.~\cite{BorradaileKM09} for planar graphs to higher genus
graphs.  While both methods result in  algorithms, the
first method is doubly exponential in a polynomial in  and
 and the second is singly exponential.

\subsection{Spanner for Subset-Connectivity Problems}\label{sec:spanner}

A spanner is a subgraph of length
 that contains a -approximate
solution.  Here we show how to find a spanner for bounded-genus graphs
and the subset-connectivity problems considered in this paper. After a
mortar graph is computed, the construction is, in fact, exactly the
same as in the planar cases, namely:
\begin{quote}
  For each brick  defined by  and for each subset  of the
  portals of , find the optimal Steiner tree of  in  (using
  the method of Erickson et~al.~\cite{EricksonMV87}).  The spanner  is
  the union of all these trees over all bricks plus the edges of the
  mortar graph.
\end{quote}
To prove the correctness of our spanner theorem for the case of
-edge-connectivity \survive, we need to appeal to the
following result of Borradaile and Klein, which we have simplified the
statement of here:

\begin{theorem}[{\cite[Theorem~5]{BorradaileKlein08}}]\label{thm:2ec-trees}
  Consider an instance of the -edge connectivity problem.
  There is a feasible solution  to this instance that is a subgraph of
   such that 
  \begin{itemize}
  \item  where  is an absolute
    constant, and
  \item the intersection of  with any brick  is a set of 
    trees the set of leaves of which are portals.
  \end{itemize}
\end{theorem}



\begin{theorem}[Spanner Theorem]\label{thm:spanner}
  Let  be an edge-weighted graph embedded on a surface of Euler
  genus  and  a given set of terminals. There
  exists a spanner  such that
  \begin{description} \item[ is spanning:]  contains a -approximate solution to
    \steiner, -edge-connected \survive, and \subtsp; and
  \item[ is short:] ;
  \end{description}
  where the function  is singly exponential in a
  polynomial in  and , and  is an absolute
  constant. The spanner can be found in  time.
\end{theorem}


\begin{proof}
  Given a mortar graph  as guaranteed by
  Theorem~\ref{thm:genus-mortar-graph}, a spanner is constructed
  as specified above.  As in~\cite{BorradaileKM09}, the time to find
   is .  It was proved in~\cite{BorradaileKM09}
  that .
  Therefore,  and
   (recall that  and
   depend polynomially on  and ). 
  
  Now we show that  contains a near-optimal solution to each
  problem.
  For \steiner, the proof follows directly from the Structure Theorem:
  the intersection of a minimal solution in  with a
  brick  is a forest whose leaves are portals.

  For -edge-connected \survive, we appeal to
  Theorem~\ref{thm:2ec-trees}: By the Structure Theorem, there
  is a solution  in  that has length at most
  .  For each brick , let  be the
  intersection of  with .  
   is the union of trees.
Replace each tree with the Steiner
  tree spanning the same subset as found in the spanner construction.
  Let  be the graph resulting from all such replacements:
  .  By
  Observation~\ref{lem:soln}, the edges of  induce
  a solution to the problem of length at most .

  For \subtsp, the proof is similar.  By the Structure Theorem, there
  is a tour  of the terminals  in  that has length at
  most .  For each brick , let  be a
  connected component of the intersection of  with .  Because
  the terminals are in  and not in ,  is a path between
  portals of : replace  with the Steiner tree (i.e.\ a shortest
  path) connecting these two portals found in the construction of the
  spanner\footnote{Note that to construct a spanner for \subtsp, we
    need only shortest paths between pairs of portals.}.  Let
   be the tour resulting from all these replacements: .  Appealing to
  Observation~\ref{lem:soln}, the edges of  induce
  a solution of length at most .\qed
\end{proof}

\subsection{\PTAS via Spanner}\label{sect:ptas_spanner}

In order to apply the \PTAS framework of Klein~\cite{Klein08} to
bounded-genus graphs, we need the following Contraction Decomposition
Theorem due to Demaine et al.:

\begin{theorem}[{\cite[Theorem~1.1]{DemaineHM07}}] \label{thm:genus-baker}
  For a fixed genus , and any integer  and for every
  graph  of Euler genus at most , the edges of  can be
  partitioned into  sets such that contracting any one of the sets
  results in a graph of treewidth at most .  Furthermore,
  such a partition can be found in  time.
\end{theorem}
Recent techniques~\cite{CabelloChambers07} for finding shortest
noncontractible cycles of embedded graphs have improved the above
running time to .\footnote{We would like to thank Jeff
  Erickson for pointing out in private communication that the
  algorithm given in~\cite{CabelloChambers07} works for both
  orientable and nonorientable surfaces.}

We review the four steps of the framework in our setting:
\begin{description}
\item[1. Spanner Step:] Find a spanner  of  according to
  Theorem~\ref{thm:spanner}.
\item[2. Thinning Step:] For  (where
   is the function given in Theorem~\ref{thm:spanner}),
  let  be the partition of the edges of  as
  guaranteed by Theorem~\ref{thm:genus-baker}.  Let  be the set
  in the partition with minimum weight: .  Let  be the graph obtained from  by contracting the
  edges of .  By Theorem~\ref{thm:genus-baker},  has
  treewidth at most .
\item[3. Dynamic Programming Step:] Use dynamic programming (see,
  e.g.~\cite{KorachSolel90}) to find the optimal solution to the
  problem in .
\item[4. Lifting Step:] Convert this solution to a solution in  by
  incorporating some of the edges of .  For \steiner, at most one
  copy of each edge of  is introduced to maintain
  connectivity~\cite{BorradaileKM09}. In the case of -edge connected
  \survive, at most two copies of each edge of  are
  required~\cite{BorradaileKlein08}. For \subtsp, the method was
  explained in~\cite{Klein06}.
\end{description}

\paragraph{Analysis of the running time.}
By Theorem~\ref{thm:spanner}, the spanner step takes
 time (with singly exponential dependence
on polynomials in  and ).  By
Theorem~\ref{thm:genus-baker}, thinning takes time  using~\cite{CabelloChambers07}. Dynamic programming takes time
: because  is
singly exponential in polynomials in  and , this
step is doubly exponential in polynomials in  and .
Lifting takes linear time.  Hence, the overall running time is
.

\subsection{\PTAS via Dynamic Programming over the Bricks} \label{sec:dp}

In~\cite{BorradaileKM09}, Borradaile et al.\ present a \PTAS that is
singly exponential in a polynomial in  for \steiner in
planar graphs. The idea is to incorporate the spanner step into the
dynamic programming step and to use a somewhat modified thinning
step. To this end, the operator \emph{brick-contraction}  is
defined to be the application of the operation  followed by
contracting each brick to become a single vertex of degree at most
 (see Figure~\ref{fig:mg}(e)). The thinning algorithm
decomposes the mortar graph  into parts  of
\emph{bounded dual radius} (implying bounded treewidth). Applying
 to each part maintains bounded dual radius. The algorithm computes
optimal Steiner trees inside the bricks using the method
of~\cite{EricksonMV87} only at the leaves of the dynamic programming tree, thus
eliminating the need of an a-priori constructed spanner. The
interaction between subproblems of the dynamic programming is
restricted to the portals, of which there are few.

For embedded graphs with genus , the concept of bounded dual
radius does not apply in the same way.  We deal with treewidth
directly and obtain the following algorithm:
we apply the Contraction Decomposition
Theorem~\ref{thm:genus-baker}~\cite{DemaineHM07} to  and
contract a set of edges  in . However, we apply a
special weight to portal edges so as to prevent them from being
included in . Also, in , we slightly modify the
definition of contraction: after contracting an edge, we do not delete
parallel portal edges.  Because portal edges connect the mortar graph
to the bricks, they are not parallel in the graph in which we find a
solution via dynamic programming. The details are given below.


\begin{center} \fbox{
    \begin{minipage}[h]{0.95\linewidth}
      \noindent\textbf{Algorithm . } \\
      \begin{tabular}{ll}
        \textit{Input. } & a graph  of fixed genus , a mortar graph  of \\
        \textit{Output. } & a set ,\\
                          & a tree decomposition  of \\
      \end{tabular} \
  L &\leq \length{\MG} + \sum_F \length{\partial F}\theta \leq \alpha
  \OPT + \theta \sum_F \length{ \partial F}\\
  &\leq \alpha \OPT + \theta
  \cdot 2\alpha \OPT \leq 3 \theta \alpha \OPT \, .
  
    \length{\widehat {\cal P}_{S\vee N}} \leq
    (1+\epsilon)\length{{\cal P}_{S\vee N}}. \label{eq:S}
  
    \length{\widehat {\cal X}_i} \leq \length{{\cal X}_i} + 5 \length{Q_i} \leq \length{{\cal
        X}_i} + 5\epsilon \length{P_i} \leq (1+5\epsilon) \length{{\cal X}_i} \label{eq:Q}
  
    \length{\widehat {\cal P}_{S\wedge N}} \leq (1+5\epsilon)\length{{\cal P}_{S \wedge N}}.\label{eq:NS}
  
  \length{T_1} \leq (1+3\epsilon) \length {T}.\label{eq:s1}

  \length{T_2} \leq (1+ 5 \epsilon) \length{T_1}.\label{eq:s2}
\label{eq:s3}
  \length{T_3} = \length{T_2}.
\label{eq:s4}
  \length{\widehat T}\leq \length{T_3}+ \sum_{B \in {\cal B}} \sum_{v
    \in V_B} (\length{P_v} + \length{e} + \length{P'_v}),

  \sum_{B \in {\cal B}} \sum_{v \in V_B} \length{P_v} +
  \length{e} + \length{P'_v} 
  &=& 2 \sum_{B \in {\cal B}} \sum_{v \in V_B} \length{P_v}\mbox{,
    because }\ell({\mbox{portal edges}})=0 \\
  &\leq& 2 \sum_{B \in {\cal B}} \sum_{v \in V_B} \length{\partial
    B}/\theta\mbox{, by the choice of portals} \\
  &\leq& 2 \sum_{B \in {\cal B}}
  \frac{\beta}{\theta} \length{\partial B}\mbox{, by
    Theorem~\ref{thm:tsp-prop}}\\
  &\leq& 2 \frac{\beta}{\theta} 2\alpha (\epsilon^{-1},g)
  \OPT\mbox{, by Theorem~\ref{thm:genus-mortar-graph}}\\
  &\leq& \epsilon\, \OPT\mbox{, for , as required.}


Combining Equations~(\ref{eq:s1}),~(\ref{eq:s2}),~(\ref{eq:s3})
and~(\ref{eq:s4}), we obtain .
The Structure Theorem is proved for the \subtsp problem.


\section{Conclusion and Outlook}

We presented a framework to obtain es on bounded-genus
graphs for subset-connectivity problems, where we are given a graph
and a set of terminals and require a certain connectivity among the
terminals. Specifically, we obtained the first  for \steiner on
bounded-genus graphs running in -time with a constant
that is singly exponential in  and the genus of the
graph. Our method is based on the framework of Borradaile et
al.~\cite{BorradaileKM09} for planar graphs; in fact, we generalize
their work in the sense that basically any problem that is shown to
admit a  on planar graphs using their framework easily
generalizes to bounded-genus graphs using the methods presented in
this work. In particular, this gives rise to es in
bounded-genus graphs for \subtsp (Section~\ref{sec:subtsp}),
-edge-connected \survive~\cite{BorradaileKlein08}, and
also \steinerforest~\cite{BateniHM10}.

A natural question is to ask what other classes of graphs admit a
\PTAS for the problems discussed in this work. An important
generalization of bounded-genus graphs are proper classes of graphs
that are closed under taking \emph{minors}. Such \emph{-minor-free
  graphs} have earned much attention in recent years. Many hard
optimization problems have been shown to admit es and
fixed-parameter algorithms on these classes of graphs; see,
e.g.,~\cite{DemaineHajiaghayi05-PTAS,Grohe03}. But subset-connectivity
problems, specifically \subtsp and \steiner, remain important open
problems~\cite{Grohe03,DemaineHM07}. Both a spanner theorem and a
contraction decomposition theorem are still missing for the
-minor-free case. Very often, results on -minor-free graphs are
first shown for planar graphs, then extended to bounded-genus graphs,
and finally obtained for -minor-free graphs. This is due to the
powerful decomposition theorem of Robertson and
Seymour~\cite{GraphMinors16} that essentially says that every
-minor-free graph can be decomposed into a number of parts that are
``almost embeddable'' in a bounded-genus surface. We conjecture that
our framework extends to -minor-free graphs via this decomposition
theorem. The advantage of our methodology is that handling weighted
graphs and subset-type problems are naturally incorporated,
and thus it might be possible to combine all the steps for a
potential \PTAS into a single framework for -minor-free graphs
based on what we presented in this work.  Hence, whereas our
work is an important step towards this generalization, still a number
of hard challenges remain; see also~\cite{DemaineHM07} for a further
discussion on this matter.






\begin{thebibliography}{10}


\bibitem{ArnborgProskurowski89}
Arnborg, S., Proskurowski, A.: Linear time algorithms for {NP}-hard problems
  restricted to partial k-trees.
\newblock Discrete Applied Mathematics \textbf{23}(1), 11--24 (1989)

\bibitem{Arora03}
Arora, S.: Approximation schemes for {}-hard geometric optimization
  problems: a survey.
\newblock Mathematical Programming \textbf{97}(1--2), 43--69 (2003)

\bibitem{BateniHM10}
Bateni, M., Hajiaghayi, M., Marx, D.: Approximation schemes for steiner forest
  on planar graphs and graphs of bounded treewidth.
\newblock In: STOC '10: Proceedings of the 42nd annual ACM Symposium on Theory
  of Computing, p. to appear. ACM (2010)

\bibitem{BorradaileKlein08}
Borradaile, G., Klein, P.: The two-edge connectivity survivable network problem
  in planar graphs.
\newblock In: ICALP '08: Proceedings of the 35th International Colloquium on
  Automata, Languages and Programming, \emph{LNCS}, vol. 5125, pp. 485--501.
  Springer (2008)

\bibitem{BorradaileKM09}
Borradaile, G., Klein, P.N., Mathieu, C.: An {} approximation
  scheme for {S}teiner tree in planar graphs.
\newblock ACM Transactions on Algorithms \textbf{5}(3) (2009)

\bibitem{CabelloChambers07}
Cabello, S., Chambers, E.W.: Multiple source shortest paths in a genus g graph.
\newblock In: SODA '07: Proceedings of the 18th annual ACM-SIAM Symposium on
  Discrete Algorithms, pp. 89--97. SIAM (2007)

\bibitem{DemaineHajiaghayi05-PTAS}
Demaine, E.D., Hajiaghayi, M.: Bidimensionality: new connections between {FPT}
  algorithms and {PTASs}.
\newblock In: SODA '05: Proceedings of the 16th annual ACM-SIAM Symposium on
  Discrete Algorithms, pp. 590--601 (2005)

\bibitem{DemaineHM07}
Demaine, E.D., Hajiaghayi, M., Mohar, B.: Approximation algorithms via
  contraction decomposition.
\newblock In: SODA '07: Proceedings of the 18th annual ACM-SIAM Symposium on
  Discrete Algorithms, pp. 278--287. SIAM (2007)

\bibitem{Eppstein03}
Eppstein, D.: Dynamic generators of topologically embedded graphs.
\newblock In: SODA '03: Proceedings of the 14th annual ACM-SIAM Symposium on
  Discrete Algorithms, pp. 599--608. SIAM (2003)

\bibitem{EITTWY92}
Eppstein, D., Italiano, G., Tamassia, R., Tarjan, R., Westbrook, J., Yung, M.:
  {Maintenance of a minimum spanning forest in a dynamic planar graph}.
\newblock J. Algorithms \textbf{13}(1), 33--54 (1992).
\newblock Special issue for 1st SODA

\bibitem{EricksonWhittlesey05}
Erickson, J., Whittlesey, K.: Greedy optimal homotopy and homology generators.
\newblock In: SODA '05: Proceedings of the 16th annual ACM-SIAM Symposium on
  Discrete Algorithms, pp. 1038--1046. SIAM (2005)

\bibitem{EricksonMV87}
Erickson, R.E., Monma, C.L., Arthur F.~Veinott, J.: Send-and-split method for
  minimum-concave-cost network flows.
\newblock Math. Oper. Res. \textbf{12}(4), 634--664 (1987)

\bibitem{Grohe03}
Grohe, M.: Local tree-width, excluded minors, and approximation algorithms.
\newblock Combinatorica \textbf{23}(4), 613--632 (2003)

\bibitem{HenzingerKRS97}
Henzinger, M.R., Klein, P.N., Rao, S., Subramanian, S.: Faster shortest-path
  algorithms for planar graphs.
\newblock Journal of Computer and System Sciences \textbf{55}(1), 3--23 (1997)

\bibitem{Klein06}
Klein, P.N.: A subset spanner for planar graphs, with application to subset
  {TSP}.
\newblock In: STOC '06: Proceedings of the 38th annual ACM Symposium on Theory
  of Computing, pp. 749--756 (2006)

\bibitem{Klein08}
Klein, P.N.: A linear-time approximation scheme for {TSP} in undirected planar
  graphs with edge-weights.
\newblock SIAM J. Comput. \textbf{37}(6), 1926--1952 (2008)

\bibitem{KorachSolel90}
Korach, E., Solel, N.: Linear time algorithm for minimum weight {S}teiner tree
  in graphs with bounded treewidth.
\newblock Manuscript (1990)

\bibitem{Mehlhorn88}
Mehlhorn, K.: A faster approximation algorithm for the {Steiner} problem in
  graphs.
\newblock Information Processing Letters \textbf{27}, 125--128 (1988)

\bibitem{Mohar99}
Mohar, B.: A linear time algorithm for embedding graphs in an arbitrary
  surface.
\newblock SIAM Journal on Discrete Mathematics \textbf{12}(1), 6--26 (1999)

\bibitem{MoharThomassen01}
Mohar, B., Thomassen, C.: Graphs on Surfaces.
\newblock The John Hopkins University Press (2001)

\bibitem{GraphMinors16}
Robertson, N., Seymour, P.: Graph minors. {XVI.} {E}xcluding a non-planar
  graph.
\newblock J. Comb. Theory Ser. B \textbf{89}(1), 43--76 (2003)

\bibitem{GraphMinors02}
Robertson, N., Seymour, P.D.: Graph minors. {II}. {A}lgorithmic aspects of
  tree-width.
\newblock Journal of Algorithms \textbf{7}(3), 309--322 (1986)

\bibitem{TazariMuellerh09-DAM}
Tazari, S., {M{\"u}ller-Hannemann}, M.: Shortest paths in linear time on
  minor-closed graph classes, with an application to {S}teiner tree
  approximation.
\newblock Discrete Applied Mathematics \textbf{157}, 673--684 (2009)

\end{thebibliography}





\end{document}
