\documentclass{article}

\PassOptionsToPackage{square}{natbib}





\usepackage[final]{neurips_2018}

\usepackage[utf8]{inputenc} \usepackage[T1]{fontenc}    \usepackage{hyperref}       \usepackage{url}            \usepackage{booktabs}       \usepackage{amsfonts}       \usepackage{nicefrac}       \usepackage{microtype}      \usepackage{url}
\usepackage{latexsym}
\usepackage{lipsum}
\usepackage{mathtools}
\usepackage{amssymb}
\usepackage{hyperref}
\usepackage{array}
\usepackage{float}
\usepackage[font={small}]{caption}
\newcommand*\samethanks[1][\value{footnote}]{\footnotemark[#1]}
\title{Densely Connected Attention Propagation \\ for Reading Comprehension}



\author{
  \:\:Yi Tay, \:\:Luu Anh Tuan, Siu Cheung Hui and Jian Su \\
  Nanyang Technological University, Singapore\\
  Institute for Infocomm Research, Singapore \\
  \texttt{ytay017@e.ntu.edu.sg}\\
  \texttt{at.luu@i2r.a-star.edu.sg}\\
  \texttt{asschui@ntu.edu.sg} \\
  \texttt{sujian@i2r.a-star.edu.sg} \\
}

\begin{document}


\maketitle

\begin{abstract}

We propose \textsc{DecaProp} (\textit{Densely Connected Attention Propagation}), a new densely connected neural architecture for reading comprehension (RC). There are two distinct characteristics of our model. Firstly, our model densely connects all pairwise layers of the network, modeling relationships between passage and query across all hierarchical levels. Secondly, the dense connectors in our network are learned via attention instead of standard residual skip-connectors. To this end, we propose novel \textit{Bidirectional Attention Connectors} (BAC) for efficiently forging connections throughout the network. We conduct extensive experiments on four challenging RC benchmarks. Our proposed approach achieves state-of-the-art results on all four, outperforming existing baselines by up to  in absolute F1 score.
\end{abstract}


\section{Introduction}
The dominant neural architectures for reading comprehension (RC) typically follow a standard `encode-interact-point' design \citep{wang2016machine,seo2016bidirectional,DBLP:journals/corr/XiongZS16,wang2017gated,kundu2018amanda}. Following the embedding layer, a compositional encoder typically encodes  (query) and  (passage) individually. Subsequently, an (bidirectional) attention layer is then used to model interactions between . Finally, these attended representations are then reasoned over to find (point to) the best answer span. While, there might be slight variants of this architecture, this overall architectural design remains consistent across many RC models.

Intuitively, the design of RC models often possess some depth, i.e., every stage of the network easily comprises several layers. For example, the R-NET \citep{wang2017gated} architecture adopts three BiRNN layers as the encoder and two additional BiRNN layers at the interaction layer. BiDAF \citep{seo2016bidirectional} uses two BiLSTM layers at the pointer layer, etc. As such, RC models are often relatively deep, at the very least within the context of NLP.

Unfortunately, the depth of a model is not without implications. It is well-established fact that increasing the depth may impair gradient flow and feature propagation, making networks harder to train \citep{he2016deep,DBLP:journals/corr/SrivastavaGS15,DBLP:conf/cvpr/HuangLMW17}. This problem is prevalent in computer vision, where mitigation strategies that rely on shortcut connections such as Residual networks \citep{he2016deep}, GoogLeNet \citep{szegedy2015going} and DenseNets \citep{DBLP:conf/cvpr/HuangLMW17} were incepted. Naturally, many of the existing RC models already have some built-in designs to workaround this issue by shortening the signal path in the network. Examples include attention flow \citep{seo2016bidirectional}, residual connections \citep{xiong2017dcn+,yu2018qanet} or simply the usage of highway encoders \citep{DBLP:journals/corr/SrivastavaGS15}. As such, we hypothesize that explicitly improving information flow can lead to further and considerable improvements in RC models.

A second observation is that the flow of  representations across the network are often well-aligned and `synchronous', i.e.,  is often only matched with  at the same hierarchical stage (e.g., only after they have passed through a fixed number of encoder layers). To this end, we hypothesize that increasing the number of interaction interfaces, i.e., matching in an asynchronous, cross-hierarchical fashion, can also lead to an improvement in performance.

Based on the above mentioned intuitions, this paper proposes a new architecture with two distinct characteristics. Firstly, our network is densely connected, connecting every layer of  with every layer of . This not only facilitates information flow but also increases the number of interaction interfaces between . Secondly, our network is densely connected by \textit{attention}, making it vastly different from any residual mitigation strategy in the literature. To the best of our knowledge, this is the first work that explicitly considers attention as a form of skip-connector.

Notably, models such as BiDAF incorporates a form of attention propagation (flow). However, this is inherently unsuitable for forging dense connections throughout the network since this would incur a massive increase in the representation size in subsequent layers. To this end, we propose efficient \textit{Bidirectional Attention Connectors} (BAC) as a base building block to connect two sequences at arbitrary layers. The key idea is to compress the attention outputs so that they can be small enough to propagate, yet enabling a connection between two sequences. The propagated features are collectively passed into prediction layers, which effectively connect shallow layers to deeper layers. Therefore, this enables multiple bidirectional attention calls to be executed without much concern, allowing us to efficiently connect multiple layers together.


Overall, we propose \textsc{DecaProp} (Densely Connected Attention Propagation), a novel architecture for reading comprehension. \textsc{DecaProp} achieves a significant gain of  absolute improvement in F1 score over the existing state-of-the-art on four challenging RC datasets, namely NewsQA \citep{trischler2016newsqa}, Quasar-T \citep{dhingra2017quasar}, SearchQA \citep{dunn2017searchqa} and NarrativeQA \citep{kovcisky2017narrativeqa}.

\section{Bidirectional Attention Connectors (BAC)}
This section introduces the Bidirectional Attention Connectors (BAC) module which is central to our overall architecture. The BAC module can be thought of as a \textit{connector} component that connects two sequences/layers.

The key goals of this module are to (1) connect any two layers of  in the network, returning a residual feature that can be propagated\footnote{Notably, signals still have to back-propagate through the BAC parameters. However, this still enjoys the benefits when connecting far away layers and also by increasing the number of pathways.} to deeper layers, (2) model cross-hierarchical interactions between  and (3) minimize any costs incurred to other network components such that this component may be executed multiple times across all layers.

Let  and  be inputs to the BAC module. The initial steps in this module remains identical to standard bi-attention in which an affinity matrix is constructed between . In our bi-attention module, the affinity matrix is computed via:

where  is a standard dense layer with ReLU activations and  is the dimensionality of the vectors. Note that this is the scaled dot-product attention from \cite{vaswani2017attention}. Next, we learn an alignment between  as follows:

where  are the aligned representations of the query/passsage respectively. In many standard neural QA models, it is common to pass an augmented\footnote{This refers to common element-wise operations such as the subtraction or multiplication.} matching vector of this attentional representation to subsequent layers. For this purpose, functions such as  have been used \citep{wang2016machine}. However, simple/naive augmentation would not suffice in our use case. Even without augmentation, every call of bi-attention returns a new  dimensional vector for each element in the sequence. If the network has  layers, then connecting\footnote{See encoder component of Figure \ref{overview} for more details.} all pairwise layers would require  connectors and therefore an output dimension of . This is not only computationally undesirable but also require a large network at the end to reduce this vector. With augmentation, this problem is aggravated. Hence, standard birectional attention is not suitable here.

To overcome this limitation, we utilize a parameterized function  to compress the bi-attention vectors down to scalar.

where  is the output (for each element in ) of the BAC module. This is done in an identical fashion for  and  to form  for each element in Q. Intuitively  where  are the learned scalar attention that is propagated to upper layers. Since there are only three scalars, they will not cause any problems even when executed for multiple times. As such, the connection remains relatively lightweight. This compression layer can be considered as a defining trait of the BAC, differentiating it from standard bi-attention.

Naturally, there are many potential candidates for the function . One natural choice is the standard dense layer (or multiple dense layers). However, dense layers are limited as they do not compute dyadic pairwise interactions between features which inhibit its expressiveness. On the other hand, factorization-based models are known to not only be expressive and efficient, but also able to model low-rank structure well.

To this end, we adopt factorization machines (FM) \citep{rendle2010factorization} as . The FM layer is defined as:

where ,  and . The output  is a scalar. Intuitively, this layer tries to learn pairwise interactions between every  and  using factorized (vector) parameters . In the context of our BAC module, the FM layer is trying to learn a low-rank structure from the `match' vector (e.g.,  or [). Finally, we note that the BAC module takes inspiration from the main body of our CAFE model \citep{tay2017compare} for entailment classification. However, this work demonstrates the usage and potential of the BAC as a residual connector.

\begin{figure}[ht]
  \centering
  \includegraphics[width=1.0\textwidth]{images/bac3.pdf}
  \caption{High level overview of our proposed Bidrectional Attention Connectors (BAC). BAC supports connecting two sequence layers with attention and produces connectors that can be propagated to deeper layers of the network.}\label{bac}
  \end{figure}

\section{Densely Connected Attention Propagation (\textsc{DecaProp})}
In this section, we describe our proposed model in detail. Figure \ref{overview}
depicts a high-level overview of our proposed architecture.

\begin{figure}[ht]
  \centering
  \includegraphics[width=1.0\textwidth]{images/decaprop5.pdf}
  \caption{Overview of our proposed model architecture}\label{overview}
  \end{figure}

\subsection{Contextualized Input Encoder}
The inputs to our model are two sequences  and  which represent passage and query respectively. Given , the task of the RC model is to select a sequence of tokens in  as the answer. Following many RC models, we enhance the input representations with (1) character embeddings (passed into a BiRNN encoder), (2) a binary match feature which denotes if a word in the query appears in the passage (and vice versa) and (3) a normalized frequency score denoting how many times a word appears in the passage. The Char BiRNN of  dimensions, along with two other binary features, is concatenated with the word embeddings , to form the final representation of  dimensions.

\subsection{Densely Connected Attention Encoder (\textsc{DecaEnc})}
The \textsc{DecaEnc} accepts the inputs  and  from the input encoder. \textsc{DecaEnc} is a multi-layered encoder with  layers. For each layer, we pass  into a bidirectional RNN layer of  dimensions. Next, we apply our attention connector (BAC) to  where  represents the hidden state outputs from the BiRNN encoder where the RNN cell can either be a GRU or LSTM encoder. Let  be the input dimensions of  and , then this encoder goes through a process of  in which the BiRNN at layer  consumes the propagated features from layer .

Intuitively, this layer models  whenever they are at the same network hierarchical level. At this point, we include `asynchronous' (cross hierarchy) connections between  and . Let  denote the representations of  at layer . We apply the Bidirectional Attention Connectors (BAC) as follows:

where  represents the BAC component. This densely connects all representations of  and  across multiple layers.  represents the generated features for each  combination of . In total, we obtain  compressed attention features for each word. Intuitively, these features capture fine-grained relationships between  at different stages of the network flow. The output of the encoder is the concatenation of all the BiRNN hidden states  and  which is a matrix of  dimensions.


\subsection{Densely Connected Core Architecture (\textsc{DecaCore})}
This section introduces the core architecture of our proposed model. This component corresponds to the interaction segment of standard RC model architecture.
\paragraph{Gated Attention} The outputs of the densely connected encoder are then passed into a standard gated attention layer. This corresponds to the \textit{`interact'} component in many other popular RC models that models  interactions with attention. While there are typically many choices of implementing this layer, we adopt the standard gated bi-attention layer following \citep{wang2017gated}.

where  is the sigmoid function and  are dense layers with ReLU activations. The output  is the query-dependent passage representation.


\paragraph{Gated Self-Attention} Next, we employ a self-attention layer, applying Equation (\ref{gated_ca}) yet again on , matching  against itself to form , the output representation of the core layer. The key idea is that self-attention models each word in the query-dependent passsage representation against all other words, enabling each
word to benefit from a wider global view of the context.


\paragraph{Dense Core} At this point, we note that there are two intermediate representations of , i.e., one after the gated bi-attention layer and one after the gated self-attention layer. We denote them as  respectively. Unlike the Densely Connected Attention Encoder, we no longer have two representations at each hierarchical level since they have already been `fused'. Hence, we apply a one-sided BAC to all permutations of  and . Note that the one-sided BAC only outputs values for the left sequence, ignoring the right sequence.

where  represents the connection output and  is the one-sided BAC function. All values of  are concatenated to form a matrix  of , which is then concatenated with  to form . This final representation is then passed to the answer prediction layer.



\subsection{Answer Pointer and Prediction Layer}
Next, we pass  through a stacked BiRNN model with two layers and obtain two representations,  and  respectively.

The start and end pointers are then learned via:

where  are parameters of this layer.  To train the model, following prior work, we minimize the sum of negative log probabilities of the start and end indices:

where  is the number of samples,  are the true start and end indices.  is the k-th value of the vector . The test span is chosen by finding the maximum value of  where .
\section{Experiments}
This section describes our experiment setup and empirical results.
\subsection{Datasets and Competitor Baselines}
We conduct experiments on four challenging QA datasets which are described as follows:

\paragraph{NewsQA} This challenging RC dataset \citep{trischler2016newsqa}  comprises  QA pairs. Passages are relatively long at about  words on average. This dataset has also been extensively used in benchmarking RC models. On this dataset, the key competitors are BiDAF \citep{seo2016bidirectional}, Match-LSTM \citep{wang2016machine}, FastQA/FastQA-Ext \citep{weissenborn2017making}, R2-BiLSTM \citep{weissenborn2017reading}, AMANDA \citep{kundu2018amanda}.
\paragraph{Quasar-T} This dataset \citep{dhingra2017quasar} comprises  factoid-based QA pairs and is constructed using ClueWeb09 as its backbone corpus. The key competitors on this dataset are BiDAF and the Reinforced Ranker-Reader (R \citep{wang2017r}.
Several variations of the ranker-reader model (e.g., SR, SR), which use the Match-LSTM underneath, are also compared against.
\paragraph{SearchQA} This dataset \citep{dunn2017searchqa} aims to emulate the search and retrieval process in question answering applications. The challenge involves reasoning over multiple documents. In this dataset, we concatenate all documents into a single passage context and perform RC over the documents. The competitor baselines on this dataset are Attention Sum Reader (ASR) \citep{kadlec2016text}, Focused Hierarchical RNNs (FH-RNN) \citep{ke2018focused}, AMANDA \citep{kundu2018amanda}, BiDAF, AQA \citep{buck2017ask} and the Reinforced Ranker-Reader (R \citep{wang2017r}.
\paragraph{NarrativeQA} \citep{kovcisky2017narrativeqa} is a recent QA dataset that involves comprehension over stories. We use the summaries setting\footnote{Notably, a new SOTA was set by \citep{hu2018attention} after the NIPS submission deadline.} which is closer to a standard QA or reading comprehension setting. We compare with the baselines in the original paper, namely Seq2Seq, Attention Sum Reader and BiDAF. We also compare with the recent BiAttention + MRU model \citep{tay2018multi}.

As compared to the popular SQuAD dataset \citep{rajpurkar2016squad}, these datasets are either (1) more challenging\footnote{This is claimed by authors in most of the dataset papers.}, involves more multi-sentence reasoning or (2) is concerned with searching across multiple documents in an `open domain' setting (SearchQA/Quasar-T). Hence, these datasets accurately reflect real world applications to a greater extent. However, we regard the concatenated documents as a single context for performing reading comprehension. The evaluation metrics are the EM (exact match) and F1 score. Note that for all datasets,
we compare all models solely on the RC task. Therefore, for fair comparison we do not compare with algorithms that use a second-pass answer re-ranker \citep{wang2017evidence}. Finally, to ensure that our model is not a failing case of SQuAD, and as requested by reviewers, we also include development set scores of our model on SQuAD.
\subsection{Experimental Setup}
Our model is implemented in Tensorflow \citep{tensorflow2015-whitepaper}. The sequence lengths are capped at  for NewsQA, SearchQA, Quasar-T and NarrativeQA respectively. We use Adadelta \citep{zeiler2012adadelta} with  for NewsQA, Adam \citep{DBLP:journals/corr/KingmaB14} with  for SearchQA, Quasar-T and NarrativeQA. The choice of the RNN encoder is tuned between GRU and LSTM cells and the hidden size is tuned amongst . We use the \textsc{cuDNN} implementation of the RNN encoder. Batch size is tuned amongst . Dropout rate is tuned amongst  and applied to all RNN and fully-connected layers. We apply variational dropout \citep{gal2016theoretically} in-between RNN layers. We initialize the word embeddings with  GloVe embeddings \citep{DBLP:conf/emnlp/PenningtonSM14} and are fixed during training. The size of the character embeddings is set to  and the character RNN is set to the same as the word-level RNN encoders. The maximum characters per word is set to . The number of layers in \textsc{DecaEnc} is set to  and the number of factors in the factorization kernel is set to . We use a learning rate decay factor of  and patience of  epochs whenever the EM (or ROUGE-L) score on the development set does not increase.

\begin{table}[t]
\small
\centering
    \begin{minipage}[t]{.46\linewidth}
    \centering
       \begin{tabular}{lcccc}
    \hline
          & \multicolumn{2}{c}{Dev} & \multicolumn{2}{c}{Test} \\
          \cline{2-5}
    Model & EM    & F1    & EM    & F1 \\
    \hline
Match-LSTM  & 34.4  & 49.6  & 34.9  & 50.0 \\
    BARB  & 36.1  & 49.6  & 34.1  & 48.2 \\
    BiDAF & N/A    & N/A     & 37.1  & 52.3 \\
    Neural BoW & 25.8  & 37.6  & 24.1  & 36.6 \\
    FastQA & 43.7  & 56.4  & 41.9  & 55.7 \\
    FastQAExt & 43.7  & 56.1  & 42.8  & 56.1 \\
    R2-BiLSTM & N/A     & N/A     & 43.7  & 56.7 \\
    AMANDA & 48.4  & 63.3  & 48.4  & 63.7 \\
\hline
\textsc{DecaProp} & \textbf{52.5}  & \textbf{65.7}  & \textbf{53.1}  & \textbf{66.3} \\
    \hline


    \end{tabular}

     \caption{Performance comparison on NewsQA dataset. }
 \label{newsqa}
    \end{minipage}\hfill
    \begin{minipage}[t]{.46\linewidth}
\centering
     \begin{tabular}{lcccc}
     \hline
          & \multicolumn{2}{c}{Dev} & \multicolumn{2}{c}{Test} \\

             \cline{2-5}
          & EM    & F1    & EM    & F1 \\
       \hline
GA    & 25.6  & 25.6  & 26.4  & 26.4 \\
    BiDAF & 25.7  & 28.9  & 25.9  & 28.5 \\
    SR & N/A     & N/A     & 31.5  & 38.5 \\
    SR & N/A    & N/A     & 31.9  & 38.8 \\
    R & N/A     & N/A    & 34.2  & 40.9 \\
    \hline
    \textsc{DecaProp}& \textbf{39.7}    & \textbf{48.1}  & \textbf{38.6}  & \textbf{46.9} \\
    \hline

    \end{tabular}

    \caption{Performance comparison on Quasar-T dataset. }
\label{quasar}\end{minipage}




\centering
    \begin{minipage}[t]{.46\linewidth}
    \centering
        \begin{tabular}{lrrrr}
        \hline
          & \multicolumn{2}{c}{Dev} & \multicolumn{2}{c}{Test} \\
            \cline{2-5}
          & \multicolumn{1}{c}{Acc} & \multicolumn{1}{c}{F1} & \multicolumn{1}{c}{Acc} & \multicolumn{1}{c}{F1} \\
          \hline
    TF-IDF max & 13.0    & \multicolumn{1}{l}{N/A} & 12.7  & \multicolumn{1}{l}{N/A} \\
    ASR   & 43.9  & 24.2  & 41.3  & 22.8 \\
    FH-RNN &  49.6 & 56.7 & 46.8 &53.4\\
    AMANDA & 48.6  & 57.7  & 46.8  & 56.6 \\
    \hline
    \textsc{DecaProp} & \textbf{64.5}  & \textbf{71.9}  & \textbf{62.2}  & \textbf{70.8} \\
    \hline

    \end{tabular}

      \caption{Evaluation on \textit{original setting}, Unigram Accuracy and N-gram F1 scores on SearchQA dataset.}
        \label{searchqa1}
    \end{minipage}\hfill
    \begin{minipage}[t]{.46\linewidth}
\centering

    \begin{tabular}{lrrrr}

    \hline
          & \multicolumn{2}{c}{Dev} & \multicolumn{2}{c}{Test} \\
          \cline{2-5}
          & \multicolumn{1}{c}{EM} & \multicolumn{1}{c}{F1} & \multicolumn{1}{c}{EM} & \multicolumn{1}{c}{F1} \\
          \hline
    BiDAF & 31.7  & 37.9  & 28.6  & 34.6 \\
    AQA   & 40.5  & 47.4  & 38.7  & 45.6 \\
    R  & N/A      & N/A       & 49.0    & 55.3 \\
    \hline
    \textsc{DecaProp} & \textbf{58.8}  & \textbf{65.5}  & \textbf{56.8}  & \textbf{63.6} \\
    \hline

    \end{tabular}

     \caption{Evaluation on Exact Match and F1 Metrics on SearchQA dataset.}
  \label{tab:searchqa2}\end{minipage}
\centering

    \begin{tabular}{ccccc}
      \hline
      & \multicolumn{4}{c}{Test / Validation}\\
          & BLEU-1 & BLEU-4 & METEOR & ROUGE-L \\
          \hline
    Seq2Seq & 15.89 / 16.10 & 1.26 / 1.40 & 4.08 / 4.22 & 13.15 / 13.29 \\
    Attention Sum Reader & 23.20 / 23.54 & 6.39 / 5.90 & 7.77 / 8.02 & 22.26 / 23.28 \\
    BiDAF  & 33.72 / 33.45 & 15.53 / 15.69 & 15.38 / 15.68 & 36.30 / 36.74 \\
    BiAttention + MRU &  - / 36.55 & - /19.79 &  - / 17.87 & - / 41.44  \\
    \hline
    \textsc{DecaProp} & \textbf{42.00 / 44.35} & \textbf{23.42 / 27.61} & \textbf{23.42 / 21.80} & \textbf{40.07 / 44.69} \\
\hline
    \end{tabular}

  \caption{Evaluation on NarrativeQA (Story Summaries).}
  \label{tab:NarrativeQA}\end{table}

\begin{table*}
  \small
  \centering
\begin{tabular}{ccc}
\hline
Model & EM & F1 \\
\hline
DCN \citep{DBLP:journals/corr/XiongZS16} & 66.2 & 75.9 \\
DCN + CoVE \citep{mccann2017learned} & 71.3 & 79.9 \\
R-NET (Wang et al.) \citep{wang2017gated} & 72.3 & 80.6 \\
R-NET (Our re-implementation) & 71.9 & 79.6 \\
\textsc{DecaProp} (This paper) & 72.9 & 81.4 \\
QANet \citep{yu2018qanet} & 73.6 & 82.7 \\
\hline
\end{tabular}
\caption{Single model dev scores (published scores) of some representative models on SQuAD. }
\label{tab:squad}
\end{table*}

\section{Results}
Overall, our results are optimistic and promising, with results indicating that \textsc{DecaProp} achieves state-of-the-art performance\footnote{As of NIPS 2018 submission deadline.} on all four datasets.

\paragraph{NewsQA} Table \ref{newsqa} reports the results on NewsQA. On this dataset, \textsc{DecaProp} outperforms the existing state-of-the-art, i.e., the recent AMANDA model by ( EM /  F1). Notably, AMANDA is a strong neural baseline that also incorporates gated self-attention layers, along with question-aware pointer layers. Moreover, our proposed model also outperforms well-established baselines such as Match-LSTM (+ EM / + F1) and BiDAF ( EM / + F1).
\paragraph{Quasar-T} Table \ref{quasar} reports the results on Quasar-T. Our model achieves state-of-the-art performance on this dataset, outperforming the state-of-the-art R (Reinforced Ranker Reader) by a considerable margin of  EM /  F1. Performance gain over standard baselines such as BiDAF and GA are even larger ( F1).
 \paragraph{SearchQA} Table \ref{searchqa1}
 and Table \ref{tab:searchqa2} report the results\footnote{
 The original SearchQA paper \citep{dunn2017searchqa}, along with AMANDA \citep{kundu2018amanda} report results on Unigram Accuracy and N-gram F1. On the other hand, \citep{buck2017ask} reports results on overall EM/F1 metrics. We provide comparisons on both.} on SearchQA. On the original setting, our model outperforms AMANDA by  EM and  in terms of F1 score. On the overall setting, our model outperforms both AQA (+ EM / + F1) and Reinforced Reader Ranker (+ EM / + F1). Both models are reinforcement learning based extensions of existing strong baselines such as BiDAF and Match-LSTM.
 \paragraph{NarrativeQA} Table \ref{tab:NarrativeQA} reports the results on NarrativeQA. Our proposed model outperforms all baseline systems (Seq2Seq, ASR, BiDAF) in the original paper. On average, there is a  improvement across all metrics.
\paragraph{SQuAD} Table \ref{tab:squad} reports dev scores\footnote{Early testing of our model was actually done on SQuAD. However, since taking part on the heavily contested public leaderboard requires more computational resources than we could muster, we decided to focus on other datasets. In lieu of reviewer requests, we include preliminary results of our model on SQuAD dev set.} of our model against several representative models on the popular SQuAD benchmark. While our model does not achieve state-of-the-art performance, our model can outperform the base R-NET (both our implementation as well as the published score). Our model achieves reasonably competitive performance.


\subsection{Ablation Study}
We conduct an ablation study on the NewsQA development set (Table \ref{tab:ablation}). More specifically, we report the development scores of seven ablation baselines. In (1), we removed the entire \textsc{DecaProp} architecture, reverting it to an enhanced version of the original R-NET model\footnote{For fairer comparison, we make several enhancements to the R-NET model as follows: (1) We replaced the additive attention with scaled dot-product attention similar to ours. (2) We added shortcut connections after the encoder layer. (3) We replaced the original Pointer networks with our BiRNN Pointer Layer. We found that these enhancements consistently lead to improved performance. The original R-NET performs at  lower on NewsQA.}. In (2), we removed \textsc{DecaCore} and passed  to the answer layer instead of . In (3), we removed the \textsc{DecaEnc} layer and used a 3-layered BiRNN instead. In (4), we kept the \textsc{DecaEnc} but only compared layer of the same hierarchy and omitted cross hierarchical comparisons. In (5), we removed the Gated Bi-Attention and Gated Self-Attention layers. Removing these layers simply allow previous layers to pass through. In (6-7), we varied , the number of layers of \textsc{DecaEnc}. Finally, in (8-9), we varied the FM with linear and nonlinear feed-forward layers.



 \begin{table}[htb]
  \begin{minipage}[c]{.46\linewidth}
    \centering
    \small
     \begin{tabular}{lcc}
     \hline
     Ablation & \multicolumn{1}{c}{EM} & \multicolumn{1}{c}{F1} \\
     \hline
     (1) Remove All (R-NET) &  48.1 & 61.2 \\
     (2) w/o \textsc{DecaCore}  & 51.5  & 64.5 \\
     (3) w/o \textsc{DecaEnc} & 49.3 & 62.0 \\
     (4) w/o Cross Hierarchy &   50.0    & 63.1  \\
         (5) w/o Gated Attention  & 49.4& 62.8 \\
    (6) Set \textsc{DecaEnc}  & 50.5  & 63.4 \\
     (7) Set \textsc{DecaEnc}  & 50.7  & 63.3 \\
  (8) DecaProp (Linear d->1) & 50.9 & 63.0 \\
(9) DecaProp (Nonlinear d->d->1) & 48.9 & 60.0 \\
     \hline
     Full Architecture ()& \textbf{52.5}  & \textbf{65.7} \\
     \hline
     \end{tabular}\vspace{0.2em}
     \caption{Ablation study on NewsQA development set.}
     \label{tab:ablation}
 \end{minipage}\hfill
 \begin{minipage}[c]{.5\linewidth}
   \centering
   \includegraphics[width=0.7\textwidth]{images/devem.pdf}
   \caption{Development EM score (\textsc{DecaProp} versus R-NET) on NewsQA.}
   \label{rnet}
 \end{minipage}
 \end{table}





From (1), we observe a significant gap in performance between \textsc{DecaProp} and R-NET. This demonstrates the effectiveness of our proposed architecture. Overall, the key insight is that all model components are crucial to \textsc{DecaProp}.  Notably, the \textsc{DecaEnc} seems to contribute the most to the overall performance. Finally, Figure \ref{rnet} shows the performance plot of the development EM metric (NewsQA) over training. We observe that the superiority of \textsc{DecaProp} over R-NET is consistent and relatively stable. This is also observed across other datasets but not reported due to the lack of space.


\section{Related Work}
In recent years, there has been an increase in the number of annotated RC datasets such as SQuAD \citep{rajpurkar2016squad}, NewsQA \citep{trischler2016newsqa}, TriviaQA \citep{joshi2017triviaqa} and RACE \citep{lai2017race}. Spurred on by the avaliability of data, many neural models have also been proposed to tackle these challenges. These models include BiDAF \citep{seo2016bidirectional}, Match-LSTM \citep{wang2016machine}, DCN/DCN+ \citep{DBLP:journals/corr/XiongZS16,xiong2017dcn+}, R-NET \citep{wang2017gated}, DrQA \citep{chen2017reading}, AoA Reader \citep{cui2016attention}, Reinforced Mnemonic Reader \citep{hu2017mnemonic}, ReasoNet \citep{shen2017reasonet}, AMANDA \citep{kundu2018amanda}, R Reinforced Reader Ranker \citep{wang2017r} and QANet \citep{yu2018qanet}. Many of these models innovate at either (1) the bidirectional attention layer (BiDAF, DCN), (2) invoking multi-hop reasoning (Mnemonic Reader, ReasoNet), (3) reinforcement learning (R, DCN+), (4) self-attention (AMANDA, R-NET, QANet) and finally, (5) improvements at the encoder level (QANet). While not specifically targeted at reading comprehension, a multitude of pretraining schemes \citep{mccann2017learned,peters2018deep,radford2018improving,1810.04805} have recently proven to be very effective for language understanding tasks.

Our work is concerned with densely connected networks aimed at improving information flow \citep{DBLP:conf/cvpr/HuangLMW17,DBLP:journals/corr/SrivastavaGS15,szegedy2015going}. While most works are concerned with computer vision tasks or general machine learning, there are several notable works in the NLP domain. \cite{ding2018densely} proposed Densely Connected BiLSTMs for standard text classification tasks. \citep{tay2018co} proposed a co-stacking residual affinity mechanims that includes all pairwise layers of a text matching model in the affinity matrix calculation. In the RC domain, DCN+ \citep{xiong2017dcn+} used Residual Co-Attention encoders. QANet \citep{yu2018qanet} used residual self-attentive convolution encoders. While the usage of highway/residual networks is not an uncommon sight in NLP, the usage of bidirectional attention as a skip-connector is new. Moreover, our work introduces new cross-hierarchical connections, which help to increase the number of interaction interfaces between .





\section{Conclusion}
We proposed a new \textit{Densely Connected Attention Propagation} (\textsc{DecaProp}) mechanism. For the first time, we explore the possibilities of using birectional attention as a skip-connector. We proposed Bidirectional Attention Connectors (BAC) for efficient connection of any two arbitary layers, producing connectors that can be propagated to deeper layers. This enables a shortened signal path, aiding information flow across the network. Additionally, the modularity of the BAC allows it to be easily equipped to other models and even other domains. Our proposed architecture achieves state-of-the-art performance on four challenging QA datasets, outperforming strong and competitive baselines such as Reinforced Reader Ranker (), AMANDA, BiDAF and R-NET.

\section{Acknowledgements}
This paper is partially supported by Baidu I\textsuperscript2R Research Centre, a joint laboratory between Baidu and A-Star I\textsuperscript2R. The authors would like to thank the anonymous reviewers of NeuRIPS 2018 for their valuable time and feedback!

\bibliographystyle{plainnat}
\bibliography{nips2018}

\end{document}
