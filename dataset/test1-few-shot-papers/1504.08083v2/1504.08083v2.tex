\documentclass[10pt,twocolumn,letterpaper]{article}

\usepackage{iccv}
\usepackage{times}
\usepackage{epsfig}
\usepackage{graphicx}
\usepackage{amsmath}
\usepackage{amssymb}
\usepackage{amssymb}
\usepackage{url}
\usepackage{booktabs}
\usepackage{multirow}
\usepackage{subfigure}
\usepackage{lib}
\usepackage{makecell}
\usepackage{booktabs}







\usepackage[pagebackref=true,breaklinks=true,letterpaper=true,colorlinks,bookmarks=false]{hyperref}

\iccvfinalcopy 

\def\iccvPaperID{1698} \def\httilde{\mbox{\tt\raisebox{-.5ex}{\symbol{126}}}}

\newcommand{\X}{\xspace}
\newcommand{\pool}[1]{\xspace}
\newcommand{\conv}[1]{\xspace}
\newcommand{\maxp}[1]{\xspace}
\newcommand{\fc}[1]{\xspace}
\newcommand{\vggsixteen}{VGG16\xspace}
\newcommand{\roi}{RoI\xspace}
\newcommand{\Sm}{{\bf S}\xspace}
\newcommand{\Med}{{\bf M}\xspace}
\newcommand{\Lg}{{\bf L}\xspace}
\newcommand{\ZF}{{\bf ZF}\xspace}
\newcommand{\ms}{ms\xspace}
\newcolumntype{x}{>\small c}
\newcolumntype{L}[1]{>{\raggedright\let\newline\\\arraybackslash\hspace{0pt}}m{#1}}
\newcolumntype{C}[1]{>{\centering\let\newline\\\arraybackslash\hspace{0pt}}m{#1}}
\newcolumntype{R}[1]{>{\raggedleft\let\newline\\\arraybackslash\hspace{0pt}}m{#1}}


\pagestyle{empty}

\begin{document}

\title{Fast R-CNN}

\author{Ross Girshick\\
Microsoft Research\\
{\tt\small rbg@microsoft.com}
}

\maketitle
\thispagestyle{empty}

\begin{abstract}
This paper proposes a Fast Region-based Convolutional Network method \emph{(Fast R-CNN)} for object detection.
Fast R-CNN builds on previous work to efficiently classify object proposals using deep convolutional networks.
Compared to previous work, Fast R-CNN employs several innovations to improve training and testing speed while also increasing detection accuracy.
Fast R-CNN trains the very deep \vggsixteen network 9\X faster than R-CNN, is 213\X faster at test-time, and achieves a higher mAP on PASCAL VOC 2012.
Compared to SPPnet, Fast R-CNN trains \vggsixteen 3\X faster, tests 10\X faster, and is more accurate.
Fast R-CNN is implemented in Python and C++ (using Caffe) and is available under the open-source MIT License at \url{https://github.com/rbgirshick/fast-rcnn}.
\end{abstract}
 \section{Introduction}

Recently, deep ConvNets \cite{krizhevsky2012imagenet,lecun89e} have significantly improved image classification \cite{krizhevsky2012imagenet} and object detection \cite{girshick2014rcnn,overfeat} accuracy.
Compared to image classification, object detection is a more challenging task that requires more complex methods to solve.
Due to this complexity, current approaches (\eg, \cite{girshick2014rcnn,he2014spp,overfeat,Zhu2015segDeepM}) train models in multi-stage pipelines that are slow and inelegant.

Complexity arises because detection requires the accurate localization of objects, creating two primary challenges.
First, numerous candidate object locations (often called ``proposals'') must be processed.
Second, these candidates provide only rough localization that must be refined to achieve precise localization.
Solutions to these problems often compromise speed, accuracy, or simplicity.


In this paper, we streamline the training process for state-of-the-art ConvNet-based object detectors \cite{girshick2014rcnn,he2014spp}.
We propose a single-stage training algorithm that jointly learns to classify object proposals and refine their spatial locations.


The resulting method can train a very deep detection network (\vggsixteen \cite{simonyan2015verydeep}) 9\X faster than R-CNN \cite{girshick2014rcnn} and 3\X faster than SPPnet \cite{he2014spp}.
At runtime, the detection network processes images in 0.3s (excluding object proposal time) while achieving top accuracy on PASCAL VOC 2012 \cite{Pascal-IJCV} with a mAP of 66\% (vs. 62\% for R-CNN).\footnote{All timings use one Nvidia K40 GPU overclocked to 875 MHz.}



\subsection{R-CNN and SPPnet}
The Region-based Convolutional Network method (R-CNN) \cite{girshick2014rcnn} achieves excellent object detection accuracy by using a deep ConvNet to classify object proposals.
R-CNN, however, has notable drawbacks:
\begin{enumerate}
\itemsep0em
\item {\bf Training is a multi-stage pipeline.}
R-CNN first fine-tunes a ConvNet on object proposals using log loss.
Then, it fits SVMs to ConvNet features.
These SVMs act as object detectors, replacing the softmax classifier learnt by fine-tuning.
In the third training stage, bounding-box regressors are learned.
\item {\bf Training is expensive in space and time.}
For SVM and bounding-box regressor training, features are extracted from each object proposal in each image and written to disk.
With very deep networks, such as \vggsixteen, this process takes 2.5 GPU-days for the 5k images of the VOC07 trainval set.
These features require hundreds of gigabytes of storage.
\item {\bf Object detection is slow.}
At test-time, features are extracted from each object proposal in each test image.
Detection with \vggsixteen takes 47s / image (on a GPU).
\end{enumerate}

R-CNN is slow because it performs a ConvNet forward pass for each object proposal, without sharing computation.
Spatial pyramid pooling networks (SPPnets) \cite{he2014spp} were proposed to speed up R-CNN by sharing computation.
The SPPnet method computes a convolutional feature map for the entire input image and then classifies each object proposal using a feature vector extracted from the shared feature map.
Features are extracted for a proposal by max-pooling the portion of the feature map inside the proposal into a fixed-size output (\eg, ).
Multiple output sizes are pooled and then concatenated as in spatial pyramid pooling \cite{Lazebnik2006}.
SPPnet accelerates R-CNN by 10 to 100\X at test time.
Training time is also reduced by 3\X due to faster proposal feature extraction.

SPPnet also has notable drawbacks.
Like R-CNN, training is a multi-stage pipeline that involves extracting features, fine-tuning a network with log loss, training SVMs, and finally fitting bounding-box regressors.
Features are also written to disk.
But unlike R-CNN, the fine-tuning algorithm proposed in \cite{he2014spp} cannot update the convolutional layers that precede the spatial pyramid pooling.
Unsurprisingly, this limitation (fixed convolutional layers) limits the accuracy of very deep networks.

\subsection{Contributions}
We propose a new training algorithm that fixes the disadvantages of R-CNN and SPPnet, while improving on their speed and accuracy.
We call this method \emph{Fast R-CNN} because it's comparatively fast to train and test.
The Fast R-CNN method has several advantages:
\begin{enumerate}
  \itemsep0em
  \item Higher detection quality (mAP) than R-CNN, SPPnet
  \item Training is single-stage, using a multi-task loss
  \item Training can update all network layers
  \item No disk storage is required for feature caching
\end{enumerate}

Fast R-CNN is written in Python and C++ (Caffe \cite{jia2014caffe}) and is available under the open-source MIT License at \url{https://github.com/rbgirshick/fast-rcnn}.
 \section{Fast R-CNN architecture and training}

\figref{arch} illustrates the Fast R-CNN architecture.
A Fast R-CNN network takes as input an entire image and a set of object proposals.
The network first processes the whole image with several convolutional (\emph{conv}) and max pooling layers to produce a conv feature map.
Then, for each object proposal a region of interest (\emph{\roi}) pooling layer extracts a fixed-length feature vector from the feature map.
Each feature vector is fed into a sequence of fully connected (\emph{fc}) layers that finally branch into two sibling output layers: one that produces softmax probability estimates over  object classes plus a catch-all ``background'' class and another layer that outputs four real-valued numbers for each of the  object classes.
Each set of  values encodes refined bounding-box positions for one of the  classes.

\begin{figure}[t!]
\centering
\includegraphics[width=1\linewidth,trim=0 24em 25em 0, clip]{figs/arch.pdf}
\caption{Fast R-CNN architecture. An input image and multiple regions of interest ({\roi}s) are input into a fully convolutional network. Each \roi is pooled into a fixed-size feature map and then mapped to a feature vector by fully connected layers (FCs). The network has two output vectors per \roi: softmax probabilities and per-class bounding-box regression offsets. The architecture is trained end-to-end with a multi-task loss.}
\figlabel{arch}
\end{figure}


\subsection{The \roi pooling layer}
The \roi pooling layer uses max pooling to convert the features inside any valid region of interest into a small feature map with a fixed spatial extent of  (\eg, ), where  and  are layer hyper-parameters that are independent of any particular \roi.
In this paper, an \roi is a rectangular window into a conv feature map.
Each \roi is defined by a four-tuple  that specifies its top-left corner  and its height and width .

\roi max pooling works by dividing the  RoI window into an  grid of sub-windows of approximate size  and then max-pooling the values in each sub-window into the corresponding output grid cell.
Pooling is applied independently to each feature map channel, as in standard max pooling.
The \roi layer is simply the special-case of the spatial pyramid pooling layer used in SPPnets \cite{he2014spp} in which there is only one pyramid level.
We use the pooling sub-window calculation given in \cite{he2014spp}.




\subsection{Initializing from pre-trained networks}
We experiment with three pre-trained ImageNet \cite{imagenet_cvpr09} networks, each with five max pooling layers and between five and thirteen conv layers (see \secref{setup} for network details).
When a pre-trained network initializes a Fast R-CNN network, it undergoes three transformations.

First, the last max pooling layer is replaced by a \roi pooling layer that is configured by setting  and  to be compatible with the net's first fully connected layer (\eg,  for \vggsixteen).

Second, the network's last fully connected layer and softmax (which were trained for 1000-way ImageNet classification) are replaced with the two sibling layers described earlier (a fully connected layer and softmax over  categories and category-specific bounding-box regressors).

Third, the network is modified to take two data inputs: a list of images and a list of {\roi}s in those images.


\subsection{Fine-tuning for detection}
Training all network weights with back-propagation is an important capability of Fast R-CNN.
First, let's elucidate why SPPnet is unable to update weights below the spatial pyramid pooling layer.

The root cause is that back-propagation through the SPP layer is highly inefficient when each training sample (\ie \roi) comes from a different image, which is exactly how R-CNN and SPPnet networks are trained.
The inefficiency stems from the fact that each \roi may have a very large receptive field, often spanning the entire input image.
Since the forward pass must process the entire receptive field, the training inputs are large (often the entire image).

We propose a more efficient training method that takes advantage of feature sharing during training.
In Fast R-CNN training, stochastic gradient descent (SGD) mini-batches are sampled hierarchically, first by sampling  images and then by sampling  {\roi}s from each image.
Critically, {\roi}s from the same image share computation and memory in the forward and backward passes.
Making  small decreases mini-batch computation.
For example, when using  and , the proposed training scheme is roughly 64\X faster than sampling one {\roi} from  different images (\ie, the R-CNN and SPPnet strategy).

One concern over this strategy is it may cause slow training convergence because {\roi}s from the same image are correlated.
This concern does not appear to be a practical issue and we achieve good results with  and  using fewer SGD iterations than R-CNN.

In addition to hierarchical sampling, Fast R-CNN uses a streamlined training process with one fine-tuning stage that jointly optimizes a softmax classifier and bounding-box regressors, rather than training a softmax classifier, SVMs, and regressors in three separate stages \cite{girshick2014rcnn,he2014spp}.
The components of this procedure (the loss, mini-batch sampling strategy, back-propagation through \roi pooling layers, and SGD hyper-parameters) are described below.

\paragraph{Multi-task loss.}
A Fast R-CNN network has two sibling output layers.
The first outputs a discrete probability distribution (per \roi), , over  categories.
As usual,  is computed by a softmax over the  outputs of a fully connected layer.
The second sibling layer outputs bounding-box regression offsets, , for each of the  object classes, indexed by .
We use the parameterization for  given in \cite{girshick2014rcnn}, in which  specifies a scale-invariant translation and log-space height/width shift relative to an object proposal.

Each training \roi is labeled with a ground-truth class  and a ground-truth bounding-box regression target .
We use a multi-task loss  on each labeled {\roi} to jointly train for classification and bounding-box regression:

in which  is log loss for true class .

The second task loss, , is defined over a tuple of true bounding-box regression targets for class , , and a predicted tuple , again for class .
The Iverson bracket indicator function  evaluates to 1 when  and 0 otherwise.
By convention the catch-all background class is labeled .
For background {\roi}s there is no notion of a ground-truth bounding box and hence  is ignored.
For bounding-box regression, we use the loss

in which

is a robust  loss that is less sensitive to outliers than the  loss used in R-CNN and SPPnet.
When the regression targets are unbounded, training with  loss can require careful tuning of learning rates in order to prevent exploding gradients.
\eqref{smoothL1} eliminates this sensitivity.

The hyper-parameter  in \eqref{loss} controls the balance between the two task losses.
We normalize the ground-truth regression targets  to have zero mean and unit variance.
All experiments use .

We note that \cite{erhan2014scalable} uses a related loss to train a class-agnostic object proposal network.
Different from our approach, \cite{erhan2014scalable} advocates for a two-network system that separates localization and classification.
OverFeat \cite{overfeat}, R-CNN \cite{girshick2014rcnn}, and SPPnet \cite{he2014spp} also train classifiers and bounding-box localizers, however these methods use stage-wise training, which we show is suboptimal for Fast R-CNN (\secref{multitask}).

\paragraph{Mini-batch sampling.}
During fine-tuning, each SGD mini-batch is constructed from  images, chosen uniformly at random (as is common practice, we actually iterate over permutations of the dataset).
We use mini-batches of size , sampling  {\roi}s from each image.
As in \cite{girshick2014rcnn}, we take 25\% of the {\roi}s from object proposals that have intersection over union (IoU) overlap with a ground-truth bounding box of at least .
These {\roi}s comprise the examples labeled with a foreground object class, \ie .
The remaining {\roi}s are sampled from object proposals that have a maximum IoU with ground truth in the interval , following \cite{he2014spp}.
These are the background examples and are labeled with .
The lower threshold of  appears to act as a heuristic for hard example mining \cite{lsvm-pami}.
During training, images are horizontally flipped with probability .
No other data augmentation is used.

\paragraph{Back-propagation through \roi pooling layers.}




Back-propagation routes derivatives through the \roi pooling layer.
For clarity, we assume only one image per mini-batch (), though the extension to  is straightforward because the forward pass treats all images independently.

Let  be the -th activation input into the \roi pooling layer and let  be the layer's -th output from the -th \roi.
The \roi pooling layer computes , in which .
 is the index set of inputs in the sub-window over which the output unit  max pools.
A single  may be assigned to several different outputs .

The \roi pooling layer's \texttt{backwards} function computes partial derivative of the loss function with respect to each input variable  by following the argmax switches:

In words, for each mini-batch \roi  and for each pooling output unit , the partial derivative  is accumulated if  is the argmax selected for  by max pooling.
In back-propagation, the partial derivatives  are already computed by the \texttt{backwards} function of the layer on top of the \roi pooling layer.

\paragraph{SGD hyper-parameters.}
The fully connected layers used for softmax classification and bounding-box regression are initialized from zero-mean Gaussian distributions with standard deviations  and , respectively.
Biases are initialized to .
All layers use a per-layer learning rate of 1 for weights and 2 for biases and a global learning rate of .
When training on VOC07 or VOC12 trainval we run SGD for 30k mini-batch iterations, and then lower the learning rate to  and train for another 10k iterations.
When we train on larger datasets, we run SGD for more iterations, as described later.
A momentum of  and parameter decay of  (on weights and biases) are used.





\subsection{Scale invariance}
We explore two ways of achieving scale invariant object detection: (1) via ``brute force'' learning and (2) by using image pyramids.
These strategies follow the two approaches in \cite{he2014spp}.
In the brute-force approach, each image is processed at a pre-defined pixel size during both training and testing.
The network must directly learn scale-invariant object detection from the training data.

The multi-scale approach, in contrast, provides approximate scale-invariance to the network through an image pyramid.
At test-time, the image pyramid is used to approximately scale-normalize each object proposal.
During multi-scale training, we randomly sample a pyramid scale each time an image is sampled, following \cite{he2014spp}, as a form of data augmentation.
We experiment with multi-scale training for smaller networks only, due to GPU memory limits.

\section{Fast R-CNN detection}
Once a Fast R-CNN network is fine-tuned, detection amounts to little more than running a forward pass (assuming object proposals are pre-computed).
The network takes as input an image (or an image pyramid, encoded as a list of images) and a list of  object proposals to score.
At test-time,  is typically around , although we will consider cases in which it is larger ( k).
When using an image pyramid, each \roi is assigned to the scale such that the scaled \roi is closest to  pixels in area \cite{he2014spp}.


For each test \roi , the forward pass outputs a class posterior probability distribution  and a set of predicted bounding-box offsets relative to  (each of the  classes gets its own refined bounding-box prediction).
We assign a detection confidence to  for each object class  using the estimated probability .
We then perform non-maximum suppression independently for each class using the algorithm and settings from R-CNN \cite{girshick2014rcnn}.

\subsection{Truncated SVD for faster detection}
\seclabel{svd}
For whole-image classification, the time spent computing the fully connected layers is small compared to the conv layers.
On the contrary, for detection the number of {\roi}s to process is large and nearly half of the forward pass time is spent computing the fully connected layers (see \figref{timing}).
Large fully connected layers are easily accelerated by compressing them with truncated SVD \cite{Denton2014SVD,Xue2013svd}.

In this technique, a layer parameterized by the  weight matrix  is approximately factorized as

using SVD.
In this factorization,  is a  matrix comprising the first  left-singular vectors of ,  is a  diagonal matrix containing the top  singular values of , and  is  matrix comprising the first  right-singular vectors of .
Truncated SVD reduces the parameter count from  to , which can be significant if  is much smaller than .
To compress a network, the single fully connected layer corresponding to  is replaced by two fully connected layers, without a non-linearity between them.
The first of these layers uses the weight matrix  (and no biases) and the second uses  (with the original biases associated with ).
This simple compression method gives good speedups when the number of {\roi}s is large.


 \section{Main results}

\begin{table*}[t!]
\centering
\renewcommand{\arraystretch}{1.2}
\renewcommand{\tabcolsep}{1.2mm}
\resizebox{\linewidth}{!}{
  \begin{tabular}{@{}L{2.5cm}|L{1.2cm}|r*{19}{x}|x@{}}
method & train set & aero      & bike      & bird      & boat      & bottle     & bus        & car        & cat        & chair      & cow        & table      & dog        & horse      & mbike      & persn     & plant      & sheep      & sofa       & train      & tv         & mAP       \\
\hline
SPPnet BB \cite{he2014spp} &
07  diff &
73.9 &
72.3 &
62.5 &
51.5 &
44.4 &
74.4 &
73.0 &
74.4 &
42.3 &
73.6 &
57.7 &
70.3 &
74.6 &
74.3 &
54.2 &
34.0 &
56.4 &
56.4 &
67.9 &
73.5 &
63.1 \\
R-CNN BB \cite{rcnn-pami} &
07 &
73.4 &
77.0 &
63.4 &
45.4 &
\bf{44.6} &
75.1 &
78.1 &
79.8 &
40.5 &
73.7 &
62.2 &
79.4 &
78.1 &
73.1 &
64.2 &
\bf{35.6} &
66.8 &
67.2 &
70.4 &
\bf{71.1} &
66.0 \\
\hline
FRCN [ours] &
07 &
74.5 &
78.3 &
69.2 &
53.2 &
36.6 &
77.3 &
78.2 &
82.0 &
40.7 &
72.7 &
67.9 &
79.6 &
79.2 &
73.0 &
69.0 &
30.1 &
65.4 &
70.2 &
75.8 &
65.8 &
66.9 \\
FRCN [ours] &
07  diff &
74.6 &
\bf{79.0} &
68.6 &
57.0 &
39.3 &
79.5 &
\bf{78.6} &
81.9 &
\bf{48.0} &
74.0 &
67.4 &
80.5 &
80.7 &
74.1 &
69.6 &
31.8 &
67.1 &
68.4 &
75.3 &
65.5 &
68.1 \\
FRCN [ours] &
07+12 &
\bf{77.0} &
78.1 &
\bf{69.3} &
\bf{59.4} &
38.3 &
\bf{81.6} &
\bf{78.6} &
\bf{86.7} &
42.8 &
\bf{78.8} &
\bf{68.9} &
\bf{84.7} &
\bf{82.0} &
\bf{76.6} &
\bf{69.9} &
31.8 &
\bf{70.1} &
\bf{74.8} &
\bf{80.4} &
70.4 &
\bf{70.0} \\
\end{tabular}
}
\vspace{0.05em}
\caption{{\bf VOC 2007 test} detection average precision (\%). All methods use \vggsixteen. Training set key: {\bf 07}: VOC07 trainval, {\bf 07  diff}: {\bf 07} without ``difficult'' examples, {\bf 07+12}: union of {\bf 07} and VOC12 trainval.
SPPnet results were prepared by the authors of \cite{he2014spp}.}
\tablelabel{voc2007}
\end{table*}
 \begin{table*}[t!]
\centering
\renewcommand{\arraystretch}{1.2}
\renewcommand{\tabcolsep}{1.2mm}
\resizebox{\linewidth}{!}{
\begin{tabular}{@{}L{2.5cm}|L{1.2cm}|r*{19}{x}|x@{}}
method & train set & aero      & bike      & bird      & boat      & bottle     & bus        & car        & cat        & chair      & cow        & table      & dog        & horse      & mbike      & persn     & plant      & sheep      & sofa       & train      & tv         & mAP       \\
\hline
BabyLearning &
Prop. &
77.7 &
73.8 &
62.3 &
48.8 &
45.4 &
67.3 &
67.0 &
80.3 &
41.3 &
70.8 &
49.7 &
79.5 &
74.7 &
78.6 &
64.5 &
36.0 &
69.9 &
55.7 &
70.4 &
61.7 &
63.8 \\
R-CNN BB \cite{rcnn-pami} &
12 &
79.3 &
72.4 &
63.1 &
44.0 &
44.4 &
64.6 &
66.3 &
84.9 &
38.8 &
67.3 &
48.4 &
82.3 &
75.0 &
76.7 &
65.7 &
35.8 &
66.2 &
54.8 &
69.1 &
58.8 &
62.9 \\
SegDeepM &
12+seg &
\bf{82.3} &
75.2 &
67.1 &
50.7 &
\bf{49.8} &
71.1 &
69.6 &
88.2 &
42.5 &
71.2 &
50.0 &
85.7 &
76.6 &
81.8 &
69.3 &
\bf{41.5} &
\bf{71.9} &
62.2 &
73.2 &
\bf{64.6} &
67.2 \\
\hline
FRCN [ours] &
12 &
80.1 &
74.4 &
67.7 &
49.4 &
41.4 &
74.2 &
68.8 &
87.8 &
41.9 &
70.1 &
50.2 &
86.1 &
77.3 &
81.1 &
70.4 &
33.3 &
67.0 &
63.3 &
77.2 &
60.0 &
66.1 \\
FRCN [ours] &
07++12 &
82.0 &
\bf{77.8} &
\bf{71.6} &
\bf{55.3} &
42.4 &
\bf{77.3} &
\bf{71.7} &
\bf{89.3} &
\bf{44.5} &
\bf{72.1} &
\bf{53.7} &
\bf{87.7} &
\bf{80.0} &
\bf{82.5} &
\bf{72.7} &
36.6 &
68.7 &
\bf{65.4} &
\bf{81.1} &
62.7 &
\bf{68.8} \\
\end{tabular}
}
\vspace{0.05em}
\caption{{\bf VOC 2010 test} detection average precision (\%).
BabyLearning uses a network based on \cite{Lin2014NiN}.
All other methods use \vggsixteen. Training set key: {\bf 12}: VOC12 trainval, {\bf Prop.}: proprietary dataset, {\bf 12+seg}: {\bf 12} with segmentation annotations, {\bf 07++12}: union of VOC07 trainval, VOC07 test, and VOC12 trainval.
}
\tablelabel{voc2010}
\end{table*}
 \begin{table*}[t!]
\centering
\renewcommand{\arraystretch}{1.2}
\renewcommand{\tabcolsep}{1.2mm}
\resizebox{\linewidth}{!}{
  \begin{tabular}{@{}L{2.5cm}|L{1.2cm}|r*{19}{x}|x@{}}
method & train set & aero      & bike      & bird      & boat      & bottle     & bus        & car        & cat        & chair      & cow        & table      & dog        & horse      & mbike      & persn     & plant      & sheep      & sofa       & train      & tv         & mAP       \\
\hline
BabyLearning &
Prop. &
78.0 &
74.2 &
61.3 &
45.7 &
42.7 &
68.2 &
66.8 &
80.2 &
40.6 &
70.0 &
49.8 &
79.0 &
74.5 &
77.9 &
64.0 &
35.3 &
67.9 &
55.7 &
68.7 &
62.6 &
63.2 \\
NUS\_NIN\_c2000 &
Unk. &
80.2 &
73.8 &
61.9 &
43.7 &
\bf{43.0} &
70.3 &
67.6 &
80.7 &
41.9 &
69.7 &
51.7 &
78.2 &
75.2 &
76.9 &
65.1 &
\bf{38.6} &
\bf{68.3} &
58.0 &
68.7 &
63.3 &
63.8 \\
R-CNN BB \cite{rcnn-pami} &
12 &
79.6 &
72.7 &
61.9 &
41.2 &
41.9 &
65.9 &
66.4 &
84.6 &
38.5 &
67.2 &
46.7 &
82.0 &
74.8 &
76.0 &
65.2 &
35.6 &
65.4 &
54.2 &
67.4 &
60.3 &
62.4 \\
\hline
FRCN [ours] &
12 &
80.3 &
74.7 &
66.9 &
46.9 &
37.7 &
73.9 &
68.6 &
87.7 &
41.7 &
71.1 &
51.1 &
86.0 &
77.8 &
79.8 &
69.8 &
32.1 &
65.5 &
63.8 &
76.4 &
61.7 &
65.7 \\
FRCN [ours] &
07++12 &
\bf{82.3} &
\bf{78.4} &
\bf{70.8} &
\bf{52.3} &
38.7 &
\bf{77.8} &
\bf{71.6} &
\bf{89.3} &
\bf{44.2} &
\bf{73.0} &
\bf{55.0} &
\bf{87.5} &
\bf{80.5} &
\bf{80.8} &
\bf{72.0} &
35.1 &
\bf{68.3} &
\bf{65.7} &
\bf{80.4} &
\bf{64.2} &
\bf{68.4} \\
\end{tabular}
}
\vspace{0.05em}
\caption{{\bf VOC 2012 test} detection average precision (\%).
BabyLearning and NUS\_NIN\_c2000 use networks based on \cite{Lin2014NiN}.
All other methods use \vggsixteen. Training set key: see \tableref{voc2010}, {\bf Unk.}: unknown.
}
\tablelabel{voc2012}
\end{table*}

 
Three main results support this paper's contributions:
\begin{enumerate}
  \itemsep0em
  \item State-of-the-art mAP on VOC07, 2010, and 2012
  \item Fast training and testing compared to R-CNN, SPPnet
  \item Fine-tuning conv layers in \vggsixteen improves mAP
\end{enumerate}

\subsection{Experimental setup}
\seclabel{setup}
Our experiments use three pre-trained ImageNet models that are available online.\footnote{\url{https://github.com/BVLC/caffe/wiki/Model-Zoo}}
The first is the CaffeNet (essentially AlexNet \cite{krizhevsky2012imagenet}) from R-CNN \cite{girshick2014rcnn}.
We alternatively refer to this CaffeNet as model \Sm, for ``small.''
The second network is VGG\_CNN\_M\_1024 from \cite{Chatfield14}, which has the same depth as \Sm, but is wider.
We call this network model \Med, for ``medium.''
The final network is the very deep \vggsixteen model from \cite{simonyan2015verydeep}.
Since this model is the largest, we call it model \Lg.
In this section, all experiments use \emph{single-scale} training and testing (; see \secref{scale} for details).

\subsection{VOC 2010 and 2012 results}
On these datasets, we compare Fast R-CNN (\emph{FRCN}, for short) against the top methods on the \texttt{comp4} (outside data) track from the public leaderboard (\tableref{voc2010}, \tableref{voc2012}).\footnote{\url{http://host.robots.ox.ac.uk:8080/leaderboard} (accessed April 18, 2015)}
For the NUS\_NIN\_c2000 and BabyLearning methods, there are no associated publications at this time and we could not find exact information on the ConvNet architectures used; they are variants of the Network-in-Network design \cite{Lin2014NiN}.
All other methods are initialized from the same pre-trained \vggsixteen network.


Fast R-CNN achieves the top result on VOC12 with a mAP of 65.7\% (and 68.4\% with extra data).
It is also two orders of magnitude faster than the other methods, which are all based on the ``slow'' R-CNN pipeline.
On VOC10, SegDeepM \cite{Zhu2015segDeepM} achieves a higher mAP than Fast R-CNN (67.2\% vs. 66.1\%).
SegDeepM is trained on VOC12 trainval plus segmentation annotations; it is designed to boost R-CNN accuracy by using a Markov random field to reason over R-CNN detections and segmentations from the OP \cite{o2p} semantic-segmentation method.
Fast R-CNN can be swapped into SegDeepM in place of R-CNN, which may lead to better results.
When using the enlarged 07++12 training set (see \tableref{voc2010} caption), Fast R-CNN's mAP increases to 68.8\%, surpassing SegDeepM.

\subsection{VOC 2007 results}
On VOC07, we compare Fast R-CNN to R-CNN and SPPnet.
All methods start from the same pre-trained \vggsixteen network and use bounding-box regression.
The \vggsixteen SPPnet results were computed by the authors of \cite{he2014spp}.
SPPnet uses five scales during both training and testing.
The improvement of Fast R-CNN over SPPnet illustrates that even though Fast R-CNN uses single-scale training and testing, fine-tuning the conv layers provides a large improvement in mAP (from 63.1\% to 66.9\%).
R-CNN achieves a mAP of 66.0\%.
As a minor point, SPPnet was trained \emph{without} examples marked as ``difficult'' in PASCAL.
Removing these examples improves Fast R-CNN mAP to 68.1\%.
All other experiments use ``difficult'' examples.

\subsection{Training and testing time}
Fast training and testing times are our second main result.
\tableref{timing} compares training time (hours), testing rate (seconds per image), and mAP on VOC07 between Fast R-CNN, R-CNN, and SPPnet.
For \vggsixteen, Fast R-CNN processes images 146\X faster than R-CNN without truncated SVD and 213\X faster with it.
Training time is reduced by 9\X, from 84 hours to 9.5.
Compared to SPPnet, Fast R-CNN trains \vggsixteen 2.7\X faster (in 9.5 vs. 25.5 hours) and tests 7\X faster without truncated SVD or 10\X faster with it.
Fast R-CNN also eliminates hundreds of gigabytes of disk storage, because it does not cache features.

\begin{table}[h!]
\begin{center}
\setlength{\tabcolsep}{3pt}
\renewcommand{\arraystretch}{1.2}
\resizebox{\linewidth}{!}{
\small
\begin{tabular}{l|rrr|rrr|r}
  & \multicolumn{3}{c|}{Fast R-CNN} & \multicolumn{3}{c|}{R-CNN} & \multicolumn{1}{c}{SPPnet} \\
  & \Sm & \Med & \Lg & \Sm & \Med & \Lg & \Lg \\
\hline
train time (h) & \bf{1.2} & 2.0 & 9.5 &
22 & 28 & 84 & 25 \\
train speedup & \bf{18.3\X} & 14.0\X & 8.8\X &
1\X & 1\X & 1\X & 3.4\X \\
\hline
test rate (s/im) & 0.10 & 0.15 & 0.32 &
9.8 & 12.1 & 47.0 & 2.3 \\
~ with SVD & \bf{0.06} & 0.08 & 0.22 &
- & - & - & - \\
\hline
test speedup & 98\X & 80\X & 146\X &
1\X & 1\X & 1\X & 20\X \\
~ with SVD & 169\X & 150\X & \bf{213\X} &
- & - & - & - \\
\hline
VOC07 mAP & 57.1 & 59.2 & \bf{66.9} &
58.5 & 60.2 & 66.0 & 63.1 \\
~ with SVD & 56.5 & 58.7 & 66.6 &
- & - & - & - \\
\end{tabular}
}
\end{center}
\caption{Runtime comparison between the same models in Fast R-CNN, R-CNN, and SPPnet.
Fast R-CNN uses single-scale mode.
SPPnet uses the five scales specified in \cite{he2014spp}.
Timing provided by the authors of \cite{he2014spp}.
Times were measured on an Nvidia K40 GPU.
}
\tablelabel{timing}
\vspace{-1em}
\end{table}












\paragraph{Truncated SVD.}
Truncated SVD can reduce detection time by more than 30\% with only a small (0.3 percentage point) drop in mAP and without needing to perform additional fine-tuning after model compression.
\figref{timing} illustrates how using the top  singular values from the  matrix in {\vggsixteen}'s fc6 layer and the top  singular values from the  fc7 layer reduces runtime with little loss in mAP.
Further speed-ups are possible with smaller drops in mAP if one fine-tunes again after compression.


\begin{figure}[h!]
\centering
\includegraphics[width=0.49\linewidth,trim=3em 2em 0 0, clip]{figs/layer_timing.pdf}
\includegraphics[width=0.49\linewidth,trim=3em 2em 0 0, clip]{figs/layer_timing_svd.pdf}
\caption{Timing for \vggsixteen before and after truncated SVD.
Before SVD, fully connected layers fc6 and fc7 take 45\% of the time.}
\figlabel{timing}
\end{figure}



\subsection{Which layers to fine-tune?}
For the less deep networks considered in the SPPnet paper \cite{he2014spp}, fine-tuning only the fully connected layers appeared to be sufficient for good accuracy.
We hypothesized that this result would not hold for very deep networks.
To validate that fine-tuning the conv layers is important for \vggsixteen, we use Fast R-CNN to fine-tune, but \emph{freeze} the thirteen conv layers so that only the fully connected layers learn.
This ablation emulates single-scale SPPnet training and \emph{decreases mAP from 66.9\% to 61.4\%} (\tableref{whichlayers}).
This experiment verifies our hypothesis: training through the \roi pooling layer is important for very deep nets.

\begin{table}[h!]
\begin{center}
\setlength{\tabcolsep}{3pt}
\renewcommand{\arraystretch}{1.1}
\small
\begin{tabular}{l|rrr|r}
  & \multicolumn{3}{c|}{layers that are fine-tuned in model \Lg} & SPPnet \Lg \\
  &  fc6 &  conv3\_1 &  conv2\_1 &  fc6 \\
\hline
VOC07 mAP & 61.4 & 66.9 & \bf{67.2} & 63.1 \\
test rate (s/im) & \bf{0.32} & \bf{0.32} & \bf{0.32} & 2.3 \\
\end{tabular}
\end{center}
\caption{Effect of restricting which layers are fine-tuned for \vggsixteen.
Fine-tuning  fc6 emulates the SPPnet training algorithm \cite{he2014spp}, but using a single scale.
SPPnet \Lg results were obtained using five scales, at a significant (7\X) speed cost.}
\tablelabel{whichlayers}
\vspace{-0.5em}
\end{table}

Does this mean that \emph{all} conv layers should be fine-tuned? In short, \emph{no}.
In the smaller networks (\Sm and \Med) we find that conv1 is generic and task independent (a well-known fact \cite{krizhevsky2012imagenet}).
Allowing conv1 to learn, or not, has no meaningful effect on mAP.
For \vggsixteen, we found it only necessary to update layers from conv3\_1 and up (9 of the 13 conv layers).
This observation is pragmatic: (1) updating from conv2\_1 slows training by 1.3\X (12.5 vs. 9.5 hours) compared to learning from conv3\_1;
and (2) updating from conv1\_1 over-runs GPU memory.
The difference in mAP when learning from conv2\_1 up was only  points (\tableref{whichlayers}, last column).
All Fast R-CNN results in this paper using \vggsixteen fine-tune layers conv3\_1 and up; all experiments with models \Sm and \Med fine-tune layers conv2 and up.
 \begin{table*}[t!]
\begin{center}
\setlength{\tabcolsep}{5pt}
\renewcommand{\arraystretch}{1.1}
\small
\begin{tabular}{l|rrrr|rrrr|rrrr}
  & \multicolumn{4}{c|}{\Sm} & \multicolumn{4}{c|}{\Med} & \multicolumn{4}{c}{\Lg}  \\
\hline
multi-task training? &
&
\checkmark &
&
\checkmark &
&
\checkmark &
&
\checkmark &
&
\checkmark &
&
\checkmark
\\
stage-wise training? &
 &
 &
\checkmark &
&
 &
 &
\checkmark &
&
 &
 &
\checkmark &
\\
test-time bbox reg? & & & \checkmark & \checkmark & & & \checkmark & \checkmark & & & \checkmark & \checkmark \\
VOC07 mAP & 52.2 & 53.3 & 54.6 & \bf{57.1} & 54.7 & 55.5 & 56.6 & \bf{59.2} & 62.6 & 63.4 & 64.0 & \bf{66.9} \\
\end{tabular}
\end{center}
\caption{Multi-task training (forth column per group) improves mAP over piecewise training (third column per group).}
\tablelabel{multitask}
\vspace{-0.5em}
\end{table*}

\section{Design evaluation}

We conducted experiments to understand how Fast R-CNN compares to R-CNN and SPPnet, as well as to evaluate design decisions.
Following best practices, we performed these experiments on the PASCAL VOC07 dataset.


\subsection{Does multi-task training help?}
\seclabel{multitask}
Multi-task training is convenient because it avoids managing a pipeline of sequentially-trained tasks.
But it also has the potential to improve results because the tasks influence each other through a shared representation (the ConvNet) \cite{caruana1997multitask}.
Does multi-task training improve object detection accuracy in Fast R-CNN?

To test this question, we train baseline networks that use only the classification loss, , in \eqref{loss} (\ie, setting ).
These baselines are printed for models \Sm, \Med, and \Lg in the first column of each group in \tableref{multitask}.
Note that these models \emph{do not} have bounding-box regressors.
Next (second column per group), we take networks that were trained with the multi-task loss (\eqref{loss}, ), but we \emph{disable} bounding-box regression at test time.
This isolates the networks' classification accuracy and allows an apples-to-apples comparison with the baseline networks.


Across all three networks we observe that multi-task training improves pure classification accuracy relative to training for classification alone.
The improvement ranges from  to  mAP points, showing a consistent positive effect from multi-task learning.

Finally, we take the baseline models (trained with only the classification loss), tack on the bounding-box regression layer, and train them with  while keeping all other network parameters frozen.
The third column in each group shows the results of this \emph{stage-wise} training scheme: mAP improves over column one, but stage-wise training underperforms multi-task training (forth column per group).

\subsection{Scale invariance: to brute force or finesse?}
\seclabel{scale}
We compare two strategies for achieving scale-invariant object detection: brute-force learning (single scale) and image pyramids (multi-scale).
In either case, we define the scale  of an image to be the length of its \emph{shortest} side.

All single-scale experiments use  pixels;
 may be less than  for some images as we cap the longest image side at  pixels and maintain the image's aspect ratio.
These values were selected so that \vggsixteen fits in GPU memory during fine-tuning.
The smaller models are not memory bound and can benefit from larger values of ; however, optimizing  for each model is not our main concern.
We note that PASCAL images are  pixels on average and thus the single-scale setting typically upsamples images by a factor of 1.6.
The average effective stride at the \roi pooling layer is thus  pixels.

In the multi-scale setting, we use the same five scales specified in \cite{he2014spp} () to facilitate comparison with SPPnet.
However, we cap the longest side at  pixels to avoid exceeding GPU memory.

\begin{table}[h!]
\begin{center}
\setlength{\tabcolsep}{4.7pt}
\renewcommand{\arraystretch}{1.1}
\small
\begin{tabular}{l|rr|rr|rr|r}
 & \multicolumn{2}{c|}{SPPnet \ZF}  & \multicolumn{2}{c|}{\Sm} & \multicolumn{2}{c|}{\Med} & \Lg \\
\hline
scales & 1 & 5 & 1 & 5 & 1 & 5 & 1 \\
test rate (s/im) & 0.14 & 0.38 & \bf{0.10} & 0.39 & 0.15 & 0.64 & 0.32 \\
VOC07 mAP & 58.0 & 59.2 & 57.1 & 58.4 & 59.2 & 60.7 & \bf{66.9}
\end{tabular}
\end{center}
\caption{Multi-scale vs. single scale.
SPPnet \ZF (similar to model \Sm) results are from \cite{he2014spp}.
Larger networks with a single-scale offer the best speed / accuracy tradeoff.
(\Lg cannot use multi-scale in our implementation due to GPU memory constraints.)
}
\tablelabel{scales}
\vspace{-0.5em}
\end{table}

\tableref{scales} shows models \Sm and \Med when trained and tested with either one or five scales.
Perhaps the most surprising result in \cite{he2014spp} was that single-scale detection performs almost as well as multi-scale detection.
Our findings confirm their result: deep ConvNets are adept at directly learning scale invariance.
The multi-scale approach offers only a small increase in mAP at a large cost in compute time (\tableref{scales}).
In the case of \vggsixteen (model \Lg), we are limited to using a single scale by implementation details. Yet it achieves a mAP of 66.9\%, which is slightly higher than the 66.0\% reported for R-CNN \cite{rcnn-pami}, even though R-CNN uses ``infinite'' scales in the sense that each proposal is warped to a canonical size.

Since single-scale processing offers the best tradeoff between speed and accuracy, especially for very deep models, all experiments outside of this sub-section use single-scale training and testing with  pixels.

\subsection{Do we need more training data?}
\seclabel{moredata}
A good object detector should improve when supplied with more training data.
Zhu \etal \cite{devaMoreData} found that DPM \cite{lsvm-pami} mAP saturates after only a few hundred to thousand training examples.
Here we augment the VOC07 trainval set with the VOC12 trainval set, roughly tripling the number of images to 16.5k, to evaluate Fast R-CNN.
Enlarging the training set improves mAP on VOC07 test from 66.9\% to 70.0\% (\tableref{voc2007}).
When training on this dataset we use 60k mini-batch iterations instead of 40k.

We perform similar experiments for VOC10 and 2012, for which we construct a dataset of 21.5k images from the union of VOC07 trainval, test, and VOC12 trainval.
When training on this dataset, we use 100k SGD iterations and lower the learning rate by  each 40k iterations (instead of each 30k).
For VOC10 and 2012, mAP improves from 66.1\% to 68.8\% and from 65.7\% to 68.4\%, respectively.


\subsection{Do SVMs outperform softmax?}
Fast R-CNN uses the softmax classifier learnt during fine-tuning instead of training one-vs-rest linear SVMs post-hoc, as was done in R-CNN and SPPnet.
To understand the impact of this choice, we implemented post-hoc SVM training with hard negative mining in Fast R-CNN.
We use the same training algorithm and hyper-parameters as in R-CNN.
\begin{table}[h!]
\begin{center}
\setlength{\tabcolsep}{6pt}
\renewcommand{\arraystretch}{1.1}
\small
\begin{tabular}{l|l|r|r|r}
  method & classifier & \Sm & \Med & \Lg \\
\hline
R-CNN \cite{girshick2014rcnn,rcnn-pami} & SVM & \bf{58.5} & \bf{60.2} & 66.0 \\
\hline
FRCN [ours] & SVM & 56.3 & 58.7 & 66.8 \\
FRCN [ours] & softmax & 57.1 & 59.2 & \bf{66.9} \\
\end{tabular}
\end{center}
\caption{Fast R-CNN with softmax vs. SVM (VOC07 mAP).}
\tablelabel{svm}
\vspace{-0.5em}
\end{table}

\tableref{svm} shows softmax slightly outperforming SVM for all three networks, by  to  mAP points.
This effect is small, but it demonstrates that ``one-shot'' fine-tuning is sufficient compared to previous multi-stage training approaches.
We note that softmax, unlike one-vs-rest SVMs, introduces competition between classes when scoring a \roi.

\subsection{Are more proposals always better?}



There are (broadly) two types of object detectors: those that use a \emph{sparse} set of object proposals (\eg, selective search \cite{UijlingsIJCV2013}) and those that use a \emph{dense} set (\eg, DPM \cite{lsvm-pami}).
Classifying sparse proposals is a type of \emph{cascade} \cite{Viola01} in which the proposal mechanism first rejects a vast number of candidates leaving the classifier with a small set to evaluate.
This cascade improves detection accuracy when applied to DPM detections \cite{UijlingsIJCV2013}.
We find evidence that the proposal-classifier cascade also improves Fast R-CNN accuracy.

Using selective search's \emph{quality mode}, we sweep from 1k to 10k proposals per image, each time \emph{re-training} and \emph{re-testing} model \Med.
If proposals serve a purely computational role, increasing the number of proposals per image should not harm mAP.
\begin{figure}[h!]
\centering
\includegraphics[width=1\linewidth,trim=0em 0em 0 0, clip]{figs/proposals.pdf}
\caption{VOC07 test mAP and AR for various proposal schemes.}
\figlabel{proposals}
\end{figure}

We find that  mAP rises and then falls slightly as the proposal count increases (\figref{proposals}, solid blue line).
This experiment shows that swamping the deep classifier with more proposals does not help, and even slightly hurts, accuracy.

This result is difficult to predict without actually running the experiment.
The state-of-the-art for measuring object proposal quality is Average Recall (AR) \cite{Hosang15proposals}.
AR correlates well with mAP for several proposal methods using R-CNN, \emph{when using a fixed number of proposals per image}.
\figref{proposals} shows that AR (solid red line) does not correlate well with mAP as the number of proposals per image is varied.
AR must be used with care; higher AR due to more proposals does not imply that mAP will increase.
Fortunately, training and testing with model \Med takes less than 2.5 hours.
Fast R-CNN thus enables efficient, direct evaluation of object proposal mAP, which is preferable to proxy metrics.


We also investigate Fast R-CNN when using \emph{densely} generated boxes (over scale, position, and aspect ratio), at a rate of about 45k boxes / image.
This dense set is rich enough that when each selective search box is replaced by its closest (in IoU) dense box, mAP drops only 1 point (to 57.7\%, \figref{proposals}, blue triangle).

The statistics of the dense boxes differ from those of selective search boxes.
Starting with 2k selective search boxes, we test mAP when \emph{adding} a random sample of  dense boxes.
For each experiment we re-train and re-test model \Med.
When these dense boxes are added, mAP falls more strongly than when adding more selective search boxes, eventually reaching 53.0\%.

We also train and test Fast R-CNN using \emph{only} dense boxes (45k / image).
This setting yields a mAP of 52.9\% (blue diamond).
Finally, we check if SVMs with hard negative mining are needed to cope with the dense box distribution.
SVMs do even worse: 49.3\% (blue circle).



\subsection{Preliminary MS COCO results}
We applied Fast R-CNN (with \vggsixteen) to the MS COCO dataset \cite{coco} to establish a preliminary baseline.
We trained on the 80k image training set for 240k iterations and evaluated on the ``test-dev'' set using the evaluation server.
The PASCAL-style mAP is 35.9\%; the new COCO-style AP, which also averages over IoU thresholds, is 19.7\%.
 \section{Conclusion}

This paper proposes Fast R-CNN, a clean and fast update to R-CNN and SPPnet.
In addition to reporting state-of-the-art detection results, we present detailed experiments that we hope provide new insights.
Of particular note, sparse object proposals appear to improve detector quality.
This issue was too costly (in time) to probe in the past, but becomes practical with Fast R-CNN.
Of course, there may exist yet undiscovered techniques that allow dense boxes to perform as well as sparse proposals.
Such methods, if developed, may help further accelerate object detection.
 
\paragraph{Acknowledgements.}
I thank Kaiming He, Larry Zitnick, and Piotr Doll{\'a}r for helpful discussions and encouragement.

{\small
\bibliographystyle{ieee}
\bibliography{main}
}

\end{document}
