\documentclass[preprint,fleqn,1p]{elsarticle}

\usepackage{amssymb}
\usepackage{amsmath}
\usepackage{amsthm}

\newtheorem{thm}{Theorem}
\newtheorem{lem}{Lemma}
\newtheorem{cla}{Claim}
\newtheorem{de}{Definition}
\newtheorem{rem}{Remarks}
\newtheorem{note}{Notes}
\newtheorem{prop}{Proposition}
\newtheorem{cor}{Corollary}
\newtheorem{exam}{Example}


\def\rnum#1{\resizebox{0.5em}{\height}{\expandafter{\romannumeral #1}}}
\def\Rnum#1{\resizebox{0.5em}{\height}{\uppercase\expandafter{\romannumeral #1}}}


\journal{Theoretical Computer Science}

\begin{document}

\begin{frontmatter}


\title{On the Properties of Language Classes Defined by Bounded Reaction Automata}

\author[label1]{Fumiya Okubo}
\ead{f.okubo@akane.waseda.jp}
\author[label2]{Satoshi Kobayashi}
\ead{satoshi@cs.uec.ac.jp}
\author[label3]{Takashi Yokomori\corref{cor1}}
\ead{yokomori@waseda.jp}
\cortext[cor1]{Corresponding author}

\address[label1]{Graduate School of Education,
Waseda University, 1-6-1 Nishiwaseda, Shinjuku-ku, Tokyo 169-8050,
Japan}
\address[label2]{Graduate School of Informatics and Engineering,
University of Electro-Communications, 1-5-1 Chofugaoka, Chofu-shi, Tokyo 182-8585, Japan}
\address[label3]{Department of Mathematics,  Faculty of Education and Integrated Arts and Sciences,
Waseda University, 1-6-1 Nishiwaseda, Shinjuku-ku, Tokyo 169-8050,
Japan}

\begin{abstract}
Reaction automata are a formal model that has been introduced to  investigate the computing powers of interactive behaviors of biochemical reactions(\cite{OKY:12}). Reaction automata are language acceptors with multiset rewriting mechanism whose basic frameworks are based on reaction systems introduced in \cite{ER:07a}.  

In this paper we continue the investigation of reaction automata with  a  focus on the formal language theoretic properties of subclasses of reaction automata, called {\it linear-bounded} reaction automata (LRAs) and {\it exponentially-bounded} reaction automata (ERAs).    Besides LRAs, we newly introduce an extended model (denoted by -LRAs) by allowing -moves in the accepting process of reaction, and 
investigate the closure properties of language classes accepted by 
both LRAs and -LRAs.   Further, we  establish  new relationships of language classes accepted by LRAs and by ERAs  with the Chomsky hierarchy. 
The main results include the following : \\
\quad (\,i\,) the class of languages accepted by -LRAs forms an AFL with additional closure properties,\\
\quad (\rnum{2})  any recursively enumerable language can be expressed as  
a homomorphic image of a language accepted by an LRA,\\
\quad (\rnum{3}) the class of languages accepted by ERAs coincides with the class of context-sensitive languages. 
\end{abstract}

\begin{keyword}
biochemical reaction model; \  bounded reaction 
automata; \ abstract family of languages; \ closure property
\end{keyword}

\end{frontmatter}

\section{Introduction}
There exist two major categories in the research of mathematical modeling of biochemical reactions.  One is an analytical framework based on ordinary differential equations (ODEs) in which macroscopic behaviors of molecules are formulated as ODEs by means of approximating a massive number of molecules (or molecular concentration) by a continuous quantity. The other is a discrete framework based on the multiset rewriting in which a set of various sorts of molecular species in small quantities is represented by a multiset and a biochemical reaction is simulated by replacing the multiset  with another one, under a prescribed condition (\cite{AV:11, CPRS:01, KMP:01, KTZ:09b, Set:01}). 

Among many models that have been investigated from the 
viewpoint of the latter category mentioned above,  Ehrenfeucht and Rozenberg have introduced a formal model called {\it reaction systems} for investigating interactive behaviors between biochemical reactions 
 in which two basic components (reactants and inhibitors) play a key role as a regulation mechanism in controlling biochemical functionalities 
(\cite{ER:07a,ER:07b,ER:09}).   In the same framework, they also introduced the notion of time into reaction systems and investigated notions such as reaction times, creation times of compounds and so forth.  Rather recent two papers  \cite{EMR:10,EMR:11} continue the investigation of reaction systems, with the focuses on combinatorial properties of functions defined by random reaction systems and on the dependency relation between the power of defining functions and  the amount of available resource.    
In the theory of reaction systems, a biochemical reaction is formulated as a triple , where  is the set of molecules called {\it reactants},  is the set of molecules called {\it inhibitors}, and  is the set of molecules called {\it products}. Let  be a set of molecules, and the result of applying a 
 reaction  to , denoted by , is given by  
if  is enabled by  (i.e., if  completely includes  
and excludes ). Otherwise, the result is empty.  Thus,  if  is enabled on , and  otherwise. The result of applying a reaction  is extended to the set of reactions , denoted by , and an interactive process 
consisting of a sequence of 's is properly introduced and investigated. 

Inspired by the works of reaction systems,  we have introduced in \cite{OKY:12} computing devices called {\it reaction automata} and  showed  that they are computationally universal by proving that any recursively enumerable language is accepted by a reaction automaton.  
The notion of reaction automata may be regarded as an extension of reaction systems in the sense that our reaction automata deal with {\it multisets} rather than (usual) sets  as reaction systems do, in the sequence of computational process.  
However,  reaction automata are introduced as computing devices that accept the sets of {\it string objects} (i.e., languages over an alphabet).  This feature of a string accepting device based on multiset 
computing can be realized by introducing a simple idea of feeding an input to the device from the environment and by employing a special encoding technique.  


In reaction systems, a number of working assumptions are adopted among which  there are two to be remarked : Firstly, the {\it threshold supply} of elements (molecules) requires that for each element, either enough quantity of it is always supplied to react or it is not present at all. (Thus, reaction systems work with sets rather than multisets.) Secondly, the {\it non-permanency of elements} means that any element not involved in the active reaction ceases  to exist. (Thus, each element has a limited life-span of the unit time.)  
In contrast,  reaction automata assume  properties rather orthogonal to those features of reaction systems: They are defined as computing devices that deal with multisets (rather than sets) in the computing  process of biochemical interactions. It is also assumed that each element is sustained for free if it is not invloved in the reaction.

Before introducing the formal definition of reaction automata in the later  section, we want to describe with an example how a reaction automaton behaves in an interactive way with a given input.  
Figure \ref{graphic} illustrates an intuitive idea of the behavior of a reaction automata , where 
  is the set of objects with the input alphabet , 
 is the set of reactions, 
  is the initial multiset, and  is the special object to indicate a final multiset. 
Note that in a reaction , multisets  and  are  represented by string forms, while  is given as a set.  
In  Figure \ref{graphic},  each reaction  
is applied to a multiset (of a test tube) after receiving an input symbol (if any is provided from the environment). We assume in this example that no input symbol being fed implies that the input has already been completed. Thus, for instance, when  is applied to the initial multiset  without any input symbol being fed, which implies that 
the input is the empty string  and that it is  accepted by . For each , let . Then,  reactions  and  are enabled by the multiset  only when inputs  and  are received,  which result in producing  and , respectively.   
Once applying  to  has brought  into ,  has no possibility of applying  furthermore, because of its inhibitor .  Afterwards, a successful reaction process can continue only when either  or  is fed, and only an input sequence of  followed by  
eventually leads the reaction process to a multiset  from which no further multiset is derived and this reaction process terminates. 
One may easily see that  accepts the language . One important assumption we would like to remark is that reaction automata allow a multiset of reactions  to apply to a multiset of objects  in an exhaustive manner (what is called  {\it maximally parallel manner}),  and  the interactive process sequence of computation is nondeterministic in that the reaction result from   may produce more than one product. The details are formally described in the sequel.

\begin{figure}[t]
\centerline{
\includegraphics[scale=0.27]{anbncn_new1.eps}}
\caption{A graphic illustration of interactive biochemical reaction processes for accepting  the language  in terms of the reaction automaton  . }
\label{graphic}
\end{figure}

In this paper we  continue the  investigation of reaction automata   
with a  focus on the formal language theoretic properties of subclasses of reaction automata, called {\it linear-bounded} reaction automata (LRAs) and {\it exponentially-bounded} reaction automata (ERAs).  Besides LRAs, we will newly introduce an extended model 
(denoted by -LRAs) by allowing -moves in the accepting process of reaction, and investigate the closure properties of language classes   and - 
accepted by  LRAs and -LRAs, respectively.  
 We also investigate the relationships of language classes 
  and  (the class of languages accepted by ERAs) with the Chomsky hierarchy. 



This paper is organized as follows.  After preparing the basic notions and notations from formal language theory in Section 2, we formally describe  the notion of reaction automata (RAs)  and introduce several subclasses of reaction automata such as LRAs, -LRAs and ERAs, based on their volume (space) complexity in Section 3. Then,  the closure properties of the language classes  and - 
are investigated in  Section 4 and Section 5,  respectively. It is shown  that - forms an AFL with some additional closure properties.  In Section 6, we also establish  the relations of language classes   and  to the classes in the Chomsky hierarhy.  Specifically, we show that  any recursively enumerable language can be expressed as  a homomorphic image of a language in . It is also shown that the language class    coincides with the class of context-sensitive languages.  
Finally,  concluding remarks as well as future research topics are briefly discussed in Section 7.



\section{Preliminaries}

We assume that the reader is familiar with the basic notions of formal language theory.  For unexplained details, refer  to~\cite{HMU:03}. 

Let  be a finite alphabet. For a set , the cardinality of  is denoted by . The set of all finite-length strings over  is denoted by . The empty string is denoted by .  For a string  in ,   denotes the length of , while for a symbol  in  we denote  by   the number of occurences of  in .  

A morphism  such that  for all  is called a \textit{coding}, and it is a \textit{weak coding} if  for all . A weak coding is a \textit{projection} if  for each . 

We use the basic notations and definitions regarding multisets that follow~\cite{CMM:01,KMP:01}.
A {\it multiset} over an alphabet  is a mapping , 
where  is the set of non-negative integers and for each ,  represents the number of occurrences of  in the multiset .  
The set of all multisets over  is denoted by , including the empty multiset denoted by , where  for 
all .  A multiset  may be represented as a vector, , for an ordered set .  We can also represent the multiset  by any permutation of the string . Conversely, with any string  one can associate the multiset  defined by  for each . In this sense, we often identify a multiset  with its string representation  or any permutation of .  Note that the string representation of  is , i.e., .  

A usual set  is regarded as a multiset  such that 
  if   is in  and  otherwise.  In particular, 
 for each symbol , a multiset  is often denoted by  itself.

For two multisets ,  over ,  we define one relation and three operations as follows: 

A multiset  is called a {\it multisubset} of  if .  
The sum for a family of multisets  is also denoted by . For a multiset  and ,  is defined by  for each . The {\it weight} of a multiset  is .

We introduce an injective function  that maps a string to a multiset in the following manner: 


Let us denote by  (resp. ) 
 the class of regular (resp. linear context-free, context-free, context-sensitive, recursively enumerable) languages.

\section{Reaction Automata and Bounded Variants}

Inspired by the works  of reaction systems, we have introduced the notion of reaction automata in \cite{OKY:12} by extending sets in each reaction to multisets. Here, we start by recalling basic notions concerning reaction automata and their restricted variants called {\it bounded reaction automata}. 


\begin{de}
{\rm 
For a set ,  a {\it reaction} in  is a 3-tuple  of finite multisets, such that ,  and .
}
\end{de}
The multisets  and  are called the {\it reactant} of  and the {\it product} of , respectively, while the set  is called the {\it inhibitor} of . These notations are extended to a multiset of reactions as follows:    For a set of reactions  and a multiset  over ,  
 

In what follows,  we usually identify the set of reactions  with the set of labels of reactions in , and often use the symbol  as a finite alphabet.

\begin{de}
{\rm 
Let  be a set of reactions in  and   be a multiset of reactions 
over .  Then, for  a finite multiset , we say that \\
(1)  is {\it enabled by}  if  and , \\
(2)   is {\it enabled by  in maximally parallel manner}   
if there is no   such that ,  and   and   are enabled by  .   \\
(3)  By  we denote the set of all multisets of reactions  which are enabled by  in maximally parallel manner.\\
(4) The {\it results of  on }, denoted by , is defined as follows: 
 
Note that we have  if . Thus, if no multiset of reactions  is enabled by  in maximally parallel manner, then  remains unchanged.  
}
\end{de}

\begin{note}
{\rm 
 (\,i\,)\ As is mentioned earlier, the definition of the results of  on  given in (4) is in contrast to the original one in \cite{ER:07a},  because 
we  adopt the assumption that  
any element that is not a reactant for any active reaction {\it does} remain in the result after the reaction.\\
(\rnum{2})\ In general,  may contain more than one element, and therefore, so may .\\
(\rnum{3})\ For simplicity,  is often represented as a string rather than a set.
}
\end{note}

We are now in a position to introduce the notion of reaction automata.

\begin{de}{\rm 
{(Reaction Automata)}\ A {\it reaction automaton} (RA)  is a 5-tuple , where
\begin{itemize}
\item  is a finite set,  called the {\it background set of}  ,
\item  is called the {\it input alphabet of}  , 
\item  is a finite set of reactions in ,
\item  is an {\it initial multiset},
\item  is a special symbol which indicates the final state.
\end{itemize}
}
\end{de}

\begin{de}{\rm 
Let  be an RA and .  An {\it interactive process in  with input } is an infinite sequence  
, where 

In order to represent an interactive process , we also use 
the ``arrow notation'' for  : 
.   By  we denote the set of all interactive processes in  with input .}
\end{de}


For an interactive process  in  with input , if  for some , then we have that   and . In this case, considering the smallest , we say that  {\it converges on}  
(at the -th step). If an interactive process  converges 
on , then  is called the {\it converging state} of  and each  of  is omitted for .

\begin{de}{\rm 
Let  be an RA. Then, we define:  

The {\it language accepted by} , denoted by , is defined as follows:
}

\end{de}

Let  be an RA.  
Motivated by the notion of a workspace for a phrase-structure grammar (\cite{AS:73}), we define: for  with , and for  in ,

Further, the {\it workspace of}  {\it for}  is defined as:

 
\begin{de}{\rm Let   be a function defined on .\\
(1)\ An RA   is {\it -bounded} if for any  with ,    is bounded by . \\
(2)\ If a function  is a constant  (resp. linear, polynomial,  exponential), 
then   is termed -bounded (resp. linearly-bounded,  polynomially-bounded,  exponentially-bounded), and denoted by 
-RA (resp. -RA, -RA, -RA). Further, 
the class of languages accepted by -RAs (resp. -RAs,  -RAs, -RAs, arbitrary RAs) is denoted by - 
(resp. ). 
}
\end{de}

\begin{prop}{\rm (Theorem 3 in \cite{OKY:12})} The following inclusions hold {\rm :}  \\
{\rm (1)}. - {\rm (for each )}. \\
{\rm (2)}. . \\
{\rm (3)}.  {\rm (}{\rm )} and  are incomparable.
\label{prop-ra}
\end{prop}

\begin{exam} \rm{
Let   be an LRA defined  as follows:

Then, it holds that . Figure \ref{fig-c2n} illustrates the interactive process in  with the input . } \label{exam-c2n}
\end{exam}

\begin{figure}[t]
\centerline{
\includegraphics[scale=0.55]{lra-c2n.eps}}
\caption{Reaction diagram for accepting  in .}
\label{fig-c2n}
\end{figure}


\section{The closure properties of }

We investigate the closure properties of the class  under various language operations. To this aim, it is convenient to prove the following that one may call {\it normal form lemma} for a bounded class of RAs. 

In what follows, we assume that (i) the symbols (such as ,etc.) used in the construction for the background set in the proof denote mutually disjoint sets, and (ii) the symbols (such as ,etc.) are newly introduced in the proof.

\begin{de} \rm{
An -bounded RA  is said to be in} \it{normal form} \rm{if  appears  only in a converging state of an interactive process.}
\end{de}


\begin{lem}
For an -bounded RA , there exists an -bounded RA  such that  and  appears  only in a convergeing state of an interactive process. \label{lem-nf}
\end{lem}

\begin{proof}
For an LRA , construct an RA  and a mapping  as follows:

and

Let  with . Then, there exists an interactive process  which converges on  if and only if there exists  which converges on   such that

Note that (i) , . (ii) there may be  in  with , , but  cannot be derived from the corresponding state  in , because the blocking symbol  exists in . Moreover, the workspace of  is obviously -bounded.
\end{proof}

\begin{thm}
 is closed under union, intersection, concatenation, derivative, -free morphisms, -free gsm-mappings and shuffle.
\end{thm}

\begin{proof}
Let  and  be LRAs in normal form with . Moreover, let , ,   and  are defined as follows:

for . It is important in the proof of ``union'', ``intersection'', ``concatenation'' and ``shuffle'' parts, that  and  are disjoint.

[union] We construct an RA  as follows:

Let  and let , . Then, for  or , there exists an interactive process  which converges on  if and only if there exists  such that

(Note that either  or  includes  and , respectively, if and only if  includes .) 

Hence, it holds that  and the workspace of  is linear-bounded.

[intersection] In the LRA  constructed in the proof of ``union'' part,  we replace (i) 
 by , and 
(ii)  by . Then, it is easily seen that  that 
 holds.

[concatenation] We construct an RA  as follows:

Let  with ,  and . Then, for  and , there exists an interactive process  which converges on  if and only if there exists  such that

Note that for  in , a rule in  and  is nondeterministically chosen to be applied in the next step. If a rule in  is chosen,  is in . 

Hence, it holds that  and the workspace of  is linear-bounded.

[shuffle] We construct an RA  as follows:

Let  with ,  and let . Then, for  and , there exists an interactive process  which converges on  if and only if there exists  such that

where  or  and . Note that  () means that only  (resp. ) advances to the next step and the value of  (resp. ) is increased by one. 
 
Hence, it holds that  and the workspace of  is linear-bounded.

[right derivative]  For an LRA  in normal form and , construct an RA  and a mapping  as follows:

and

Let  with , . Then, there exists an interactive process  which converges on  if and only if there exists  such that

Hence, it holds that  and the workspace of  is linear-bounded.

[left derivative] Let  be an LRA in normal form and  and  for , .  Construct an RA  and a mapping  as follows:

and

Let  with , .  Then, there exists an interactive process  which converges on  if and only if there exists  such that

Hence, it holds that  and the workspace of  is linear-bounded.

[-free gsm-mappings] For an LRA  in normal form and a gsm-mapping , construct an RA  and a mapping   as follows:

and


Then, for an input ,  there exists   which converges on , and  
, where ,   if and only if  there exists the interactive process  in  such that

and  is a converging state in .  
Hence, it holds that  and the workspace of  is linear-bounded.

[-free morphisms] Since  is closed under -free gsm-mappings, it is also closed under -free morphisms.
\end{proof}

In order to prove some of the negative closure properties of , the following two lemmas are of crucially importance.

\begin{lem}{\rm (Lemma 1 in \cite{OKY:12})}
For an alphabet  with , let  be an injection such that for any ,  is bounded by a polynomial of . Then, there is no PRA  such that . 
\label{lem-ww}
\end{lem}

\begin{lem} 
. 
\label{lem-nww}
\end{lem}

\begin{proof}
Let  and  be an LRA defined as follows:

Let  be an input string. The string  is accepted by  in the following manner: 
\begin{enumerate}
\item Applying  and , the length of  is counted by the number of . 
\item Applying  or ,  is rewritten by . 
\item Applying , ,  and , the length of  is counted by the number of . If  or  is applied, then the interactive process enters the next step. \item Applying  and , it is confirmed that  by consuming . 
\item Applying  and , it is confirmed that . 
\item Applying  and , it is confirmed that  by consuming .
\end{enumerate}

Therefore, it holds that . Note that . Since  is closed under union and includes all regular language,  is in .
\end{proof}

\begin{thm}
 is not closed under complementation, quotient by regular languages, morphisms or gsm-mappings.
\end{thm}

\begin{proof}
From Lemma \ref{lem-nww}, , while from Lemma \ref{lem-ww}, . Hence,  is not closed under complementation. From Corollary \ref{cor-re-lra}, it obviously follows that  is not closed under quotient by regular languages, morphisms or gsm-mappings.
\end{proof}


\section{The closure properties of -}

As is seen in the previous section, it remains open whether or not 
the class  is closed under several basic operations 
such as Kleene closures  or inverse homomorphism. 

In this section, we shall prove that if the -move is allowed in the phase of input mode in the transition process of 
reactions, then  the obtained class of languages (- introduced below) accepted in that manner shows in turn positive closure properties under  those basic operations. 


\begin{de}{\rm 
Let  be an RA. An interactive process} \it{in the -input mode} \rm{in  with input  is a sequence  
, where  with  for ,

By  we denote the set of all interactive processes} \it{in the -input mode} \rm{in  with input .}
\end{de}

\begin{de}{\rm 
Let  be an RA. Then, we define: 

The {\it language accepted by}  {\it in the -input mode}, denoted by 
, is defined as follows:
}
\end{de}

\begin{de}{\rm 
The class of languages accepted by RAs (-RAs, -RAs,  -RAs and -RAs)} \it{in the -input mode} {\rm  is denoted by - 
(resp. --, -, - and -). 
}
\end{de}

In what follows, we focus on dealing with - and continue investigating the clouser properties of the class of languages. 
As a result, it is shown that the class forms an AFL, i.e., an abstract family of languages. 

\begin{thm}
For any LRA , there exists an LRA  such that .
\end{thm}

\begin{proof}
Let  be a new alphabet.
For an LRA  in normal form, construct an RA  and a mapping  as follows:

and

Note that once  is inputted before an element  in an interactive process,  cannot be consumed since  will have to be introduced by  in the next step, which implies that no -input is  allowed before an element  in a successful interactive process in .
\end{proof}

\begin{de} \rm{
An -bounded RA  is said to be in} \it{-normal form} \rm{if  appears  only in a converging state of an interactive process in the -input mode.}
\end{de}

\begin{lem}
For an -bounded RA , there exists an -bounded RA  such that  and  appears  only in a converging state of an interactive process.
\end{lem}

\begin{proof}
For an -bounded RA , construct an RA  and a mapping  as follows:

and

When  is inputted in an interactive process,  exclusively or  has to be used in the next step. Using  implies that the input of the string terminates, while using  implies that the input of the string continues.
The rest of the key issue is proved in a similar manner to Lemma \ref{lem-nf}.
\end{proof}



\begin{thm}
- is closed under union, intersection, concatenation, Kleene , Kleene , derivative, -free morphisms, inverse morphisms, -free gsm-mappings and shuffle.
\end{thm}

\begin{proof}

[union, concatenation and shuffle] Using the same construction as the proof of Theorem 1, the claims are immediately proved. 

[intersection] Let  and  be LRAs in -normal form with . Moreover, let ,  be alphabets and  be a mapping defined as follows:

for . 

Then, we construct an RA  as follows:

Let . 
Moreover, let  be a part of  and  be a part of  for . We assume that . Then, they are imitated in  as follows:


The other direction of the proof is shown in the similar manner.
Hence, it holds that  and the workspace of  is linear-bounded.

[Kleene ] Let  be an LRA in -normal form and . Construct an RA  and a mapping  as follows:

and

Let  with .
Then, we can easily see that there exist the interactive processes  and  which converge on  and , respectively, if and only if there exists the interactive process  such that

Hence, it holds that .
In a similar manner, we can prove that  for  and .
Then, it holds that  and the workspace of  is linear-bounded.

[Kleene ] For LRAs  and  in the proof of ``Kleene '' part, it holds that .  Since - is closed under intersection with regular languages, it is also closed under  Kleene .

[right derivative]  For an LRA  in -normal form and , construct an RA  and a mapping  as follows:

and

Note that because of the inhibitor of a reaction in , a reaction in  must be used after feeding the input. 
Hence, the rest of the proof is similar to the case for the ordinary input mode.

[left derivative] For an LRA  in -normal form and , construct an RA  and a mapping  as follows:

and

Note that because of the inhibitor of a reaction in , each reaction in  must be used before starting the input except .

Let  with . Then, there exists an interactive process  which converges on  if and only if there exists  such that

for some .
Hence, it holds that  and the workspace of  is linear-bounded.

[inverse morphisms] Let  be an LRA in normal form and  be a morphism defined as  or , for  and . Moreover, let  and .  Construct an RA  and a mapping  as follows:

and

Let . Hence,  is included in .
Moreover, let  be a part of . For , it is imitated in  as follows:


The other direction of the proof is shown in the similar manner. Hence, it holds that  and the workspace of  is linear-bounded.

[-free morprhisms]
We first show that - is closed under codings.
For an LRA  in -normal form and a coding , construct an RA  and a mapping   as follows:

and

Then, it holds that  and the workspace of  is linear-bounded.

In Theorem 3.7.1 of \cite{SG:75}, it is shown that each family closed under inverse morphisms, intersection with regular languages and codings is also closed under -free morprhisms. Hence, - is closed under -free morprhisms.

[-free gsm-mappings]
Since every trio is closed under -free gsm-mappings (\cite{AS:73}), - is closed under -free gsm-mappings.
\end{proof}

We shall show that - shares common negative closure properties with . The manner of proving those results is almost parallel to that of proofs for  presented in the previous section. In order to make this paper self-contained,  below we give the proof of the following lemma that is a -version of 
Lemma 2 (i.e., of Lemma 1 in \cite{OKY:12}). 
 
\begin{lem}
For an alphabet  with , let  be an injection such that for any ,  is bounded by a polynomial of . Then, there is no PRA  such that . 
\label{lem-ww-lambda}
\end{lem}

\begin{proof}
Assume that there is a {\it poly}-RA  such that .  Let ,  and the input string be  with . 

Since  is bounded by a polynomial of ,  is also bounded by a polynomial of .  Hence, for each  in an interactive process , it holds that  for some polynomial  from the definition of a {\it poly}-RA.

Let . Then, it holds that

where  denotes the number of repeated combinations of  things taken  at a time. Therefore, there is a polynomial  such that . Since it holds that , if  is sufficiently large,  we obtain the inequality . 

For , let , i.e.,  is the set of multisets in  which appear immediately after inputing  in . From the fact that  and  is an injection, we can show that for any two distinct strings ,  and  are incomparable. This is because if , then the string  is in , which means that  and contradicts that  is an injection. 

Since for any two distinct strings ,  and  are incomparable and , it holds that 

However, from the pigeonhole principle, the inequality  contradicts that for any two distinct strings , . Hence, there is no LRA  such that .
\end{proof}

\begin{thm}
- is not closed under complementation, quotient by regular languages, morphisms or gsm-mappings.
\end{thm}

\begin{cor}
- is an AFL, but not a full AFL.
\end{cor}

\begin{flushleft}
Remark: We note the class - could be proved to be an AFL in the same manner as -.
\end{flushleft}

\section{Further characterizations of  and }

In this section, we develop further characterizations concerning  and  in relation to the Chomsky hierarchy, and show two interesting results. One is concerned with a representation theorem for the class  in terms of , and the other is a new characterization of  with .  


\begin{thm}
For any context-sensitive language , there exists an LRA  such that  if and only if  {\rm (}or {\rm )} with  and .
\end{thm}

\begin{proof}
From Proposition \ref{prop-ra}, let  be an ERA which accepts . Then, construct an RA  as follows:


Note that the reactions - are almost the same as the ones of Example \ref{exam-c2n}. Therefore, the total number of  appearing in a interactive process of  is  (see Example \ref{exam-c2n} and Figure \ref{fig-c2n}). 
On the other hand, the total number of  appearing in a interactive process of  is , which is derived by the reactions . Using the reaction , it is confirmed that if  is accepted by , then .
Hence, it holds that  if and only if  with .  

Since the workspace of  for  is bounded by an exponential function with respect to the length , the workspace of  for  is bounded by a linear function with respect to the length .  

For the case , we can prove in a similar manner.
\end{proof}

\begin{thm}[Theorem 3.12 in \cite{PRS:98}]
For any recursively enumerable language , context-sensitive language  such that  if and only if  (or ) for some  and . 
\end{thm}

\begin{cor}
For any recursively enumerable language , there exists an LRA  such that  if and only if  (or ) for some  and . 
\end{cor}

\begin{cor}
(i) For any recursively enumerable language , there exists an LRA  such that  (or ) for some regular language . \\
(ii) For any recursively enumerable language , there exists an LRA  such that  for some projection .
\label{cor-re-lra}
\end{cor}

\begin{thm}
For a language ,  is a context-sensitive language if and only if  is accepted by an ERA.
\end{thm}

\begin{proof}
The claim of ``only if'' part holds by Proposition \ref{prop-ra}.

Let  be an ordered alphabet and  be an ERA. Assume that for an input , the workspace of  is bounded by the exponential function , where  are constants. Then, we shall construct the nondeterministic -tape linear-bounded automaton  in which the length of each tape is bounded by  for some constant .   imitates an interactive process  in the following manner:  
\begin{enumerate}
\item At first, Tape- has the input  and Tape- has the number of  in  (for )  represented by -ary number. Tape- is used to count the number of computation step of .
\item Let  be the current multiset in . When  reads the symbol  in the input, add one to the Tape-. Then, by checking all tapes except Tape-, compute an element of  in the nondeterministic way and rewrite the contents in the tapes. After reading through the input ,  computes an element of  in the nondeterministic way and rewrite the contents in the tapes.
\item After reading through the input , if  and , then  accepts . In the case where (i)  and , (ii)  exceeds  or (iii) the number of computation step exceeds  for  and some constant ,  rejects .
\end{enumerate}
Since we use -ary number for counting the number of symbols, the length   of each tape is enough to memorize  with . 
In the case where  never stops with the input , there exists a cycle of configurations in the computation of . Since the number of all possible s during the computation is bounded by  for  and some constant  (see the equation () in the proof of Lemma \ref{lem-ww-lambda}),  the length of Tape- to count the number of steps of computation is bounded by .
Therefore, it holds that .
\end{proof}


Table 1 summarizes the results of closure properties of both 
 and -, while Figure \ref{hierarchy} 
illustrates the relationship between language classes defined by 
a various types of bounded reaction automata and the Chomsky hierarchy.




\begin{table}
\caption{Closure properties of  and -.}
\begin{center}
\begin{tabular}{|l|c|c|} \hline
language operations & \  \ & \ - \  \\ \hline \hline
union & Y & Y \\ \hline 
intersection & Y & Y \\ \hline
complementation & N & N \\ \hline
concatenation & Y & Y \\ \hline
Kleene  & ? & Y \\ \hline
Kleene  & ? & Y \\ \hline
(right \& left) derivative & Y & Y \\ \hline
(right \& left) quotient by regular languages & N & N \\ \hline
-free morphisms & Y & Y \\ \hline
morphisms & N & N \\ \hline
inverse morphisms & ? & Y \\ \hline
-free gsm-mappings & Y & Y \\ \hline
gsm-mappings & N & N \\ \hline
shuffle & Y & Y \\ \hline
\end{tabular}
\end{center}
\end{table}

\begin{figure}[t]
\centerline{
\includegraphics[scale=0.55]{hierarchy2.eps}}
\caption{The hierarchy of the language classes accepted by bounded reaction automata, where .}
\label{hierarchy}
\end{figure}


\section{Conclusion}

We have continued the investigation of reaction automata (RAs)  with  a focus on the formal language theoretic properties of subclasses of RAs, called  linear-bounded RAs (LRAs) and  exponentially-bounded RAs (ERAs).   Besides LRAs, we have newly introduced an extended model 
(denoted by -LRAs) by allowing -moves in the accepting process of reaction, and investigated the closure properties of language classes  and -. 
We have shown the following :  \\
\quad (\,i\,) the class  - forms an AFL with additional closure properties, \\
\quad (\rnum{2})  any recursively enumerable language can be expressed as  
a homomorphic image of a language in , \\
\quad (\rnum{3}) the class    coincides with the class of context-sensitive languages. 

Considering  the definitions of  the existing acceptors in the classical automata theory, the result (\,i\,) suggests that it may be  reasonably justifiable to allow each  subclass of bounded RAs to have -transitions in its definition. 
 From the result (\rnum{2}) and the closure properties (shown in Table 1), it is interesting to see that the class    (or -) is closer to the class  rather than the class  in its language theoretic property.  Further,  
an intriguing  result (\rnum{3}) together with the result that  (in \cite{OKY:12}) may provide a new insight into the theory of computational complexity.  

Many subjects remain to be investigated. First, it is open whether or not  is equal to -, whose positive answer immediately settles open problems of the closure properties on  
 (see Table 1).  Also, we do not know the proper inclusion relation between  and . Secondly, it is interesting to explore the relationship between RAs and other computing devices that are based on the multiset rewriting, such as a variety of P-systems and their variants (\cite{PP:11}),  Petri net models (\cite{HM:01}).  Also, it would be useful to develop a method for simulating a variety of chemical reactions in the real world by the use of the framework based on reaction automata.


\begin{thebibliography}{00}

\bibitem{AV:11}
A. Alhazov, S. Verlan,   
\newblock Minimization strategies for maximally parallel multiset rewriting systems, {\it Theoretical Computer Science} vol.412,  pp.1587-1591, 2011.


\bibitem{CPRS:01}
C. Calude, Gh. P\u aun, G. Rozenberg and A. Salomaa (Eds.),  
\newblock {\it Multiset Processing}, LNCS 2235, 
Springer, 2001.

\bibitem{CMM:01}
E. Csuhaj-Varju, C. Martin-Vide, V. Mitrana,  
\newblock Multiset Automata, in: {\it Multiset Processing}, C. Calude, Gh. P\u aun, G. Rozenberg, A. Salomaa (Eds.), LNCS 2235, 
Springer,  pp.69-83, 2001.


\bibitem{ER:07a}
A. Ehrenfeucht, G. Rozenberg,   
\newblock Reaction systems, {\it Fundamenta Informaticae} vol.75,  pp.263-280, 2007.

\bibitem{ER:07b}
A. Ehrenfeucht, G. Rozenberg,   
\newblock Events and modules in reaction systems, {\it Theoretical Computer Science} vol.376,  pp.3-16, 2007.

\bibitem{ER:09}
A. Ehrenfeucht, G. Rozenberg,   
\newblock Introducing time in reaction systems, {\it Theoretical Computer Science} vol.410,  pp.310-322, 2009.

\bibitem{EMR:10}
A. Ehrenfeucht, M. Main, G. Rozenberg,   
\newblock Combinatorics of life and death in reaction systems, {\it Intern. J. of Foundations of Computer Science} vol.21,  pp.345-356, 2010.

\bibitem{EMR:11}
A. Ehrenfeucht, M. Main, G. Rozenberg,   
\newblock Functions defined by reaction systems, {\it Intern. J. of Foundations of Computer Science} vol.22,  pp.167-178, 2011.

\bibitem{SG:75}
S. Ginsburg,   
\newblock {\it Algebraic and automata-theoretic properties of formal languages},  North-Holland, Amsterdam, 1975.

\bibitem{HM:01}
Y. Hirshfeld, F. Moller,   
\newblock Pushdown automata, multiset automata, and Petri nets, {\it Theoretical Computer Science} vol.256,  pp.3-21, 2001.



\bibitem{HMU:03}
J.E. Hopcroft, T. Motwani, J.D. Ullman,   
\newblock {\it Introduction to automata theory, language and computation} - 2nd ed, 
\newblock Addison-Wesley, 2003.


\bibitem{KMP:01}
M. Kudlek, C. Martin-Vide, Gh. P\u aun,  
\newblock Toward a formal macroset theory, in: {\it Multiset Processing}, C. Calude, Gh. P\u aun, G. Rozenberg, A. Salomaa (Eds.), LNCS 2235, 
Springer,  pp.123-134, 2001.


\bibitem{KTZ:09b}
M. Kudlek, P. Totzke, G. Zetzsche,  
\newblock Properties of multiset language classes defined by multiset pushdown automata, {\it Fundamenta Informaticae}  vol.93,  pp.235-244, 2009.

\bibitem{OKY:12}
F. Okubo, S. Kobayashi, T. Yokomori,   
\newblock Reaction Automata, arXiv:1111.5038v2, submitted for publication.


\bibitem{PP:11}
Gh. P\u aun, M.J. P\'erez-Jim\' enez,  P and dP automata: A survey, Lecture Notes in Computer Science 6570, Springer, pp.102-115, 2011. 

\bibitem{PRS:98}
Gh. P\u aun, G. Rozenberg, A. Salomaa, {\it DNA Computing: New Computing Paradigms}, Springer-Verlag Berlin, Inc., 1998.

\bibitem{AS:73}
A. Salomaa,   
\newblock {\it Formal Languages}, Academic Press, New York, 1973.



\bibitem{Set:01}
Y. Suzuki, Y. Fujiwara, J. Takabayashi, H. Tanaka,  
\newblock Artificial Life Applications of a Class of P Systems, in: {\it Multiset Processing}, C. Calude, Gh. P\u aun, G. Rozenberg, A. Salomaa (Eds.), LNCS 2235, 
Springer,  pp.299-346, 2001.

\end{thebibliography}


\end{document}
