\documentclass[11pt,draftclsnofoot,onecolumn]{IEEEtran}
\IEEEoverridecommandlockouts                              	
\overrideIEEEmargins

\title{Asymptotic Stability and Decay Rates of Homogeneous Positive Systems With\\ Bounded and Unbounded Delays}

\author{Hamid Reza Feyzmahdavian, Themistoklis Charalambous, and Mikael Johansson
\thanks{Department of Automatic Control, School of Electrical Engineering and ACCESS Linnaeus Center, Royal Institute of Technology (KTH), SE-100 44 Stockholm, Sweden.
Emails: {\tt \{hamidrez, themisc, mikaelj\}@kth.se.}}
}



\usepackage[T1]{fontenc}
\usepackage{graphicx}
\graphicspath{{figures/}}
\usepackage{amssymb,amsmath,amsfonts}
\usepackage{bm}
\usepackage{epsfig}
\usepackage{epstopdf}
\usepackage{verbatim}
\usepackage[usenames,dvipsnames]{color}
\usepackage{flushend}
\usepackage{cite}



\newtheorem{theorem}{\normalfont \bfseries Theorem}
\let\oldtheorem\theorem
\renewcommand{\theorem}{\oldtheorem\em}


\newtheorem{definition}{\normalfont \bfseries Definition}
\let\olddefinition\definition
\renewcommand{\definition}{\olddefinition\em}

\newtheorem{proposition}{\normalfont \bfseries Proposition}
\let\oldproposition\proposition
\renewcommand{\proposition}{\oldproposition\em}

\newtheorem{corollary}{\normalfont \bfseries Corollary}
\let\oldcorollary\corollary
\renewcommand{\corollary}{\oldcorollary\em}

\newtheorem{example}{\normalfont \bfseries Example}
\let\oldexample\example
\renewcommand{\example}{\oldexample\em}

\newtheorem{remark}{\normalfont \bfseries Remark}
\let\oldremark\remark
\renewcommand{\remark}{\oldremark\em}

\newtheorem{assumption}{\normalfont \bfseries Assumption}
\let\oldassumption\assumption
\renewcommand{\assumption}{\oldassumption\em}

\def\bnull{\bm{0}}
\def\r{\bm{r}}
\def\x{\bm{x}}
\def\y{\bm{y}}
\def\v{\bm{v}}
\def\f{\bm{f}}
\def\g{\bm{g}}
\def\bphi{\bm{\varphi}}



\begin{document}

\maketitle




\begin{abstract}
There are several results on the stability of nonlinear positive systems in the presence of time delays. However, most of them assume that the delays are constant. This paper considers {\em time-varying}, possibly unbounded, delays and establishes asymptotic stability and bounds the decay rate of a significant class of nonlinear positive systems which includes positive linear systems as a special case. Specifically, we present a necessary and sufficient condition for delay-independent stability of continuous-time positive systems whose vector fields are cooperative and homogeneous. We show that global asymptotic stability of such systems is independent of the magnitude and variation of the time delays. For various classes of time delays, we are able to derive explicit expressions that quantify the decay rates of positive systems. We also provide the corresponding counterparts for discrete-time positive systems whose vector fields are non-decreasing and homogeneous.
\end{abstract}




\section{Introduction}\label{sec:intro}

Many real-world processes in areas such as economics, biology, ecology and communications deal with physical quantities that cannot attain negative values. The state trajectories of dynamical models characterizing such processes should thus be constrained to stay within the positive orthant. Such systems are commonly referred to as {\em positive systems} \cite{Smith:95,Farina:00,Haddad:10}. Due to their importance and broad applications, a large body of literature has been concerned with the analysis and control of positive systems (see, e.g., \cite{Jacquez:93,Valcher:97,Van:98,Leenheer:01,Aeyels:02,Dashkovski:06,Rami:07,Mason:07,Ngoc:09,Fainshil:09,2009:Florian,Ruffer:10,Rantzer:11,Tanaka:11,Fornasini:11,
li:11,Ebihara:13,Feyzmahdavian:13-0,Briat:13} and references therein).

In distributed systems where exchange of information is involved, delays are inevitable. For this reason,  a considerable effort has been devoted to characterizing the stability and performance of systems with delays (see, e.g., \cite{Hale:93,Fridman:02,Wu:04,Gao:07,He:07,Peet:09,Sipahi:11} and references therein). Recently, the stability of delayed positive linear systems has received significant attention \cite{Ngoc:06,Liu:09,Rami:09,Liu:10,Haddad:04} and it has been shown that such systems are insensitive to certain classes of time delays, in the sense that a positive linear system with time delays is asymptotically stable if the corresponding delay-free system is asymptotically stable. This is a surprising property, since the stability of general dynamical systems typically depends on the magnitude and variation of the time delays.

While the asymptotic stability of positive linear systems in the presence of time delays has been thoroughly investigated, the theory for {\em nonlinear} positive systems is considerably less well-developed (see, e.g., \cite{Haddad:04,Mason:09,Vahid:10} for exceptions). In particular, \cite{Mason:09} showed that the asymptotic stability of a particular class of nonlinear positive systems whose vector fields are {\em cooperative} and {\em homogenous} of degree zero does not depend on the magnitude of {\em constant} delays. A similar result for cooperative systems that are homogeneous of any degree was given in \cite{Vahid:10}, also under the assumption of constant delays. Extensions of these results to {\em time-varying} delays are, however, not trivial. The main reason for this is that the proof technique in \cite{Mason:09,Vahid:10} relies on a fundamental monotonicity property of trajectories of cooperative systems, which does not hold when the delays are time-varying. To the best of our knowledge, there have been rather few studies on stability of nonlinear positive systems with time-varying delays (see, e.g., \cite{Feyzmahdavian:14,Ngoc:13,Feyzmahdavian:14-1}).

At this point, it is worth noting that the results for positive linear systems cited above consider {\em bounded} delays. However, in some cases, it is not possible to a priori guarantee that the delays will be bounded, but the state evolution might be affected by the entire history of states. It is then natural to ask if the insensitivity properties of positive linear systems with respect to time delays will hold also for {\em unbounded} delays. In \cite{Liu:2011}, it was shown that, for a particular class of unbounded delays, this is indeed the case. Extensions of this result to more general classes of unbounded delays were given in \cite{Sun:12,Feyzmahdavian:13} for continuous- and discrete-time positive linear systems, respectively. However, \cite{Liu:2011,Sun:12,Feyzmahdavian:13} did not quantify how various bounds on the delay evolution impact the decay rate of positive linear systems.

This paper establishes delay-independent stability of a class of nonlinear positive systems, which includes positive linear systems as a special case, and allows for time-varying, possibly unbounded, delays. The proof technique, which uses neither the Lyapunov-Krasovskii functional method widely used to analyse positive systems with constant delays \cite{Haddad:04} nor the approach used in \cite{Mason:09,Vahid:10}, allows us to impose minimal restrictions on the delays. Specifically, we make the following contributions:
\begin{itemize}
\item We derive a set of necessary and sufficient conditions for delay-independent global stability of  continuous-time positive systems whose vector fields are cooperative and homogeneous of {\em arbitrary degree} and  discrete-time positive systems whose vector fields are non-decreasing and homogeneous of {\em degree zero}. We demonstrate that such systems are insensitive to a general class of time delays which includes bounded and unbounded time-varying delays.
\item When the asymptotic behaviour of the time delays is known, we obtain conditions to ensure global -stability in the sense of \cite{Chen:07,LLiu:08,LLiu:10}. These results allow us to quantify the decay rates of positive systems for various classes of (possibly unbounded) time-varying delays.
\item For bounded delays and a particular class of unbounded delays, we present explicit bounds on the decay rates. These bounds quantify how the magnitude of bounded delays and the rate at which the unbounded delays grow large affect the decay rate.
\item We also show that discrete-time positive systems whose vector fields are non-decreasing and homogeneous of {\em degree greater than zero} are locally asymptotically stable under delay-independent global stability conditions that we have derived.
\end{itemize}

The remainder of the paper is organized as follows. In Section \ref{sec:preliminaries}, we introduce the notation and review some preliminaries that are essential for the development of the results in this paper. Our main results for continuous- and discrete-time nonlinear positive systems are stated in Sections \ref{sec:Continuous-Time Case} and \ref{sec:Discrete-Time Case}, respectively. An illustrative example, justifying the validity of our results, is presented in Section \ref{sec:examples}. Finally, concluding remarks are given in Section \ref{sec:conclusions}.



\section{Notation and Preliminaries}\label{sec:preliminaries}

\subsection{Notation}

Vectors are written in bold lower case letters and matrices in capital letters. We let , , and  denote the set of real numbers, natural numbers, and the set of natural numbers including zero, respectively. The nonnegative orthant of the {\em n}-dimensional real space  is represented by . The {\em i}th component of a vector  is denoted by , and the notation  means that  for all components . If  is a vector in , the notation  indicates that all components of  are positive. Given a vector , the weighted  norm is defined by

For a matrix ,  denotes the real-valued entry in row  and column . A matrix  is said to be {\em nonnegative} if  for all  and . It is called {\em Metzler} if  for all . Given an -tuple  of positive real numbers and ,  the {\em dilation map}  is given by

If , the dilation map is called the {\em standard dilation map}. For a real interval ,  denotes the space of all real-valued continuous functions on  taking values in . The upper-right Dini-derivative of a continuous function  at  is defined by

where  means that  approaches zero from the right-hand side.

\subsection{Preliminaries}

Next, we review the key definitions and results necessary for developing the main results of this paper. We start with the definition of {\em cooperative} vector fields.

\begin{definition}
A continuous vector field  which is continuously differentiable on  is said to be cooperative if the Jacobian matrix  is Metzler for all .
\end{definition}
Cooperative vector fields satisfy the following property.

\begin{proposition}\textup{\textbf{\cite[Remark 3.1.1]{Smith:95}}}
\label{Proposition 0}
Let  be cooperative. For any two vectors  and  in  with  and , we have .
\end{proposition}
The following definition introduces {\em homogeneous} vector fields.

\begin{definition}
For any , the vector field   is said to be homogeneous of degree  with respect to the dilation map   if

\end{definition}
Finally, we define {\em non-decreasing} vector fields.

\begin{definition}
A vector field  is said to be non-decreasing on  if  for any  such that .
\end{definition}



\section{Continuous-Time Homogeneous Cooperative Systems}\label{sec:Continuous-Time Case}

\subsection{Problem Statement}

Consider the continuous-time dynamical system
0.05cm]
\x\bigl(t\bigr)=\bphi\bigl(t\bigr),&t\in[-\tau_{\max},0],
\end{array}
\right.

\label{Unbounded Assumption 1}
\lim_{t \rightarrow +\infty} t-\tau(t)=+\infty.

\label{Power Control}
\dot{x_i}\bigl(t\bigr)=-x_i\bigl(t\bigr)+\sum_{\substack{j=1\\j\neq i}}^n a_{ij}x_j\bigl(t-\tau(t)\bigr),\quad i=1,\ldots,n.

t-\tau(t)\geq t_1,\quad \forall t\geq t_2.

\label{Unbounded Assumption 1.1}
\sup_{t>T}\frac{\tau(t)}{t}=\alpha.

\lim_{t \rightarrow +\infty} t-\tau(t)&=&\lim_{t \rightarrow +\infty}\ln(t+1)=+\infty,\\
\lim_{k \rightarrow +\infty} \frac{\tau(t)}{t}&=&\lim_{t \rightarrow +\infty} \frac{t-\ln(t+1)}{t}=1,

\tau_{\max}=-\inf_{0\leq t\leq T_0}\biggl\{t-\tau(t)\biggr\}.

\label{Proposition 2.0}
\begin{split}
\forall i\in\{1,\ldots,n\},\;&\forall \x \in \mathbb{R}^n_+\;:\;x_i=0 \Rightarrow f_i(\x)\geq 0,\\
&\forall \x \in \mathbb{R}^n_+, \hspace{1.95cm} \g(\x)\geq 0,
\end{split}

\label{Example 0}
\f(x_1,x_2)=\begin{bmatrix} 1 & 0\0.05cm]  e & 0 \end{bmatrix}\begin{bmatrix} x_1 \\ x_2 \end{bmatrix},

\label{Example 0.1}
\tau(t)=
\begin{cases}
0, & 0\leq t\leq 1, \\
t-1,  & 1\leq t\leq 2, \\
1, & 2\leq t.
\end{cases}

x_1(t)&=&x_1(0)e^t,\quad \hspace{3.83cm} 0\leq t,\\
x_2(t)&=&
\begin{cases}
x_2(0)+(e-1)(e^t-1)x_1(0), & 0\leq t\leq 1, \\
x_2(0)+(e^2t-e^t+1-e)x_1(0),  & 1\leq t\leq 2, \\
x_2(0)+(e^2-e+1)x_1(0), & 2\leq t.
\end{cases}

V(\x)=\max_{1\leq i\leq n}\left({\frac{x_i}{v_i}}\right)^{\frac{r_{\max}}{r_i}},

\label{Level Sets}
S(m)=\biggl\{\x\in \mathbb{R}^n_+  \;\bigl|\; V(\x) \leq  \gamma^m\|\bphi\| \biggr\},\quad m\in \mathbb{N}_0,

\label{Initial Condition}
\|\bphi\|=\sup_{-\tau_{\max} \leq s\leq 0}V(\bphi(s)),

S(0)\supset \cdots \supset S(m)\supset S(m+1)\supset \cdots,

\label{Theorem 3-0}
\f(\v)+\g(\v)<\bnull.

\dot{\x}\bigl(t\bigr)=\f\bigl(\x(t)\bigr)+\sum _{q=1}^s\g_q\bigl(\x(t-\tau_q(t))\bigr).

\f(\v)+\sum _{q=1}^s \g_q(\v)<\bnull.

\|\x(t)\|\leq \frac{M}{\mu(t)},\quad t>0,

\label{Theorem 3-1-0}
\f(\v)+\g(\v)<\bnull.

\left(\frac{r_{\max}}{r_i}\right)\left(\left(\frac{f_i(\v)}{v_i}\right)+
\left(\lim_{t\rightarrow \infty}\frac{\mu(t)}{\mu(t-\tau(t))}\right)^{\frac{r_i+p}{r_{\max}}}\left(\frac{g_i(\v)}{v_i}\right)\right)+\lim_{t\rightarrow \infty}\frac{\dot{\mu}(t)}{\left(\mu(t)\right)^{1-\frac{p}{r_{\max}}}}<0,

\left({\frac{x_i(t)}{v_i}}\right)^{\frac{r_{\max}}{r_i}}=O\left(\mu^{-1}(t)\right),\quad t\geq 0,

\label{Bounded Delay}
0\leq \tau(t)\leq \tau_{\max}, \quad t\geq 0.

\label{Corollary 3-11}
\left({\frac{x_i(t)}{v_i}}\right)^{\frac{r_{\max}}{r_i}}= O\left(e^{-\eta t}\right),\quad t\geq 0,

\label{Corollary 3-1}
\left(\frac{r_{\max}}{r_i}\right)\left(\left(\frac{f_i(\v)}{v_i}\right)+\biggl(e^{\eta_i\tau_{\max}}\biggr)^{\frac{r_i}{r_{\max}}}
\left(\frac{g_i(\v)}{v_i}\right)\right)+\eta_i=0.

\label{Corollary 3-1-1-0}
\left({\frac{x_i(t)}{v_i}}\right)^{\frac{r_{\max}}{r_i}}=O\left((\theta t+1)^{\frac{-r_{\max}}{p}}\right),\quad t\geq 0,

\label{Corollary 3-1-1}
\frac{f_i(\v)}{v_i}+\frac{g_i(\v)}{v_i}+\theta_i\frac{r_{i}}{p}=0.

\left({\frac{x_i(t)}{v_i}}\right)^{\frac{r_{\max}}{r_i}}= O\left(t^{-\xi}\right),\quad t\geq 0,

\label{Corollary 4-1}
\left(\frac{f_i(\v)}{v_i}\right)+\biggl(\frac{1}{1-\alpha}\biggr)^{\frac{r_i}{r_{\max}}\xi_i}\left(\frac{g_i(\v)}{v_i}\right)=0;

\left({\frac{x_i(t)}{v_i}}\right)^{\frac{r_{\max}}{r_i}}=O\left(t^{\frac{-r_{\max}}{p}\beta}\right),\quad t\geq 0,

\label{Corollary 4-1-1}
\left(\frac{f_i(\v)}{v_i}\right)+\biggl(\frac{1}{1-\alpha}\biggr)^{(1+\frac{r_i}{p})\beta}\left(\frac{g_i(\v)}{v_i}\right)<0,

\label{System 4}
{\mathcal G}_{L}:
& \left\{
\begin{array}[l]{ll}
\dot{\x}\bigl(t\bigr)=A\x\bigl(t\bigr)+B\x\bigl(t-\tau(t)\bigr),& t\geq 0,\
We then have the following special case of Theorem \ref{Theorem 3}.

\begin{corollary}
\label{Corollary 6}
Consider the positive linear system  given by (\ref{System 4}) where  is Metzler and  is nonnegative. Then,  is globally asymptotically stable for all time delays satisfying Assumption \ref{Assumption 5} if and only if there exists a vector  such that

\end{corollary}

Corollary \ref{Corollary 6} shows that if the positive linear system (\ref{System 4}) without delay is stable, it remains asymptotically stable under all bounded and unbounded time-varying delays satisfying Assumption \ref{Assumption 5}. Note that the stability condition (\ref{LP}) is a linear programming (LP) feasibility problem in  which can be verified numerically in polynomial time.

\begin{remark}
Since  is Metzler and  is nonnegative,  is Metzler. It follows from \cite[Proposition 2]{Rantzer:11} that the linear inequality (\ref{LP}) holds if and only if  is Hurwitz, \textit{i.e.}, all its eigenvalues have negative real parts.
\end{remark}

While the asymptotic stability of the positive linear system  given by (\ref{System 4}) with time-varying delays satisfying Assumption \ref{Assumption 5} has been investigated in \cite{Sun:12}, the impact of time delays on the decay rate has been missing. Theorem \ref{Theorem 3-1} helps us to find guaranteed decay rates of  for different classes of time delays. Specifically, Corollaries \ref{Corollary 3} and \ref{Corollary 4} show that  is exponentially stable if time-varying delays are bounded, and power-rate stable if delays are unbounded and satisfy Assumption \ref{Assumption 5.1}. Therefore, not only do we extend the result of \cite{Sun:12} to general homogeneous cooperative systems (not necessarily linear), but we also provide explicit bounds on the decay rate of positive linear systems.

\begin{remark}
In \cite[Example 4.5]{Liu:13}, it was shown that a positive linear system with unbounded delays satisfying Assumption \ref{Assumption 5.1} may converge slower than any exponential function. However, an upper bound for the decay rate was not derived in~\cite{Liu:13}. Corollary \ref{Corollary 4} reveals that under Assumption \ref{Assumption 5.1} on delays, the decay rate of positive linear systems is upper bounded by a polynomial function of time.
\end{remark}



\section{Discrete-time Homogeneous Non-Decreasing Systems}\label{sec:Discrete-Time Case}

\subsection{Problem Statement}

Next, we consider the discrete-time analog of (\ref{System 3}):
0.05cm]
\hspace{0.3cm}\x\bigl(k\bigr)\hspace{0.3cm}=\bphi\bigl(k\bigr), &k\in\{-d_{\max},\ldots,0\}.
\end{array}
\right.

\label{Unbounded Assumption:1}
\lim_{k \rightarrow +\infty} k-d(k)=+\infty.

d_{\max}=-\inf_{0\leq k\leq T_0}\biggl\{k-d(k)\biggr\}.

f(x)=2x,\;g(x)=-x,\;d(k)=\frac{1}{2}\left(1-(-1)^k\right),\quad k\in\mathbb{N}_0.

\label{Theorem 1-0}
\f(\v)+\g(\v)<\v.

\bnull \leq \bphi(k)\leq \v,\quad \forall k\in\{-d_{\max},\ldots,0\}.

\|\x(k)\|\leq \frac{M}{\mu(k)},\quad k\in\mathbb{N},

\label{Theorem 1-2-0}
\f(\v)+\g(\v)<\v.

\left(\lim_{k\rightarrow \infty}\frac{\mu(k+1)}{\mu(k)}\right)^{\frac{r_i}{r_{\max}}}\left(\frac{f_i(\v)}{v_i}\right)+
\left(\lim_{k\rightarrow \infty}\frac{\mu(k+1)}{\mu(k-d(k))}\right)^{\frac{r_i}{r_{\max}}}\left(\frac{g_i(\v)}{v_i}\right)<1,

\left({\frac{x_i(k)}{v_i}}\right)^{\frac{r_{\max}}{r_i}}=O\left(\mu^{-1}(k)\right),\quad k\in \mathbb{N},

\label{Unbounded Assumption:2}
\sup_{k>T}\frac{d(k)}{k}=\alpha.

\label{Theorem 1-3-0}
\left(\frac{f_i(\v)}{v_i}\right)+\left(\frac{1}{1-\alpha}\right)^{\frac{r_i}{r_{\max}}\xi_i}\left(\frac{g_i(\v)}{v_i}\right)=1,\quad i=1,\ldots,n.

\left({\frac{x_i(k)}{v_i}}\right)^{\frac{r_{\max}}{r_i}}=O\left(k^{-\xi}\right),\quad k\in \mathbb{N},

\label{System 2}
{\Sigma}_L:
& \left\{
\begin{array}[l]{ll}
{\x}\bigl(k+1\bigr)=A\x\bigl(k\bigr)+B\x\bigl(k-d(k)\bigr), &k\in \mathbb{N}_0,\
In terms of (\ref{System 1}),  and . It is easy to verify that if  are nonnegative matrices, Assumption \ref{Assumption 3} is satisfied. Therefore, Theorem \ref{Theorem 1} can help us to derive a necessary and sufficient condition for delay-independent stability of (\ref{System 2}). Specifically, we note the following.

\begin{corollary}
\label{Theorem 2}
Consider the discrete-time positive linear system  given by~(\ref{System 2}) where  and  are nonnegative. Then, there exists a vector  such that

if and only if  is globally asymptotically stable for all time delays satisfying Assumption \ref{Assumption 1}.
\end{corollary}

\begin{remark}
For the positive linear system (\ref{System 2}),  and  are nonnegative, so  is also nonnegative. According to property of nonnegative matrices \cite{Berman:79},\cite[Proposition 1]{Rantzer:11}, there exists a vector  satisfying (\ref{LP-d}) if and only if all eigenvalues of  are strictly inside the unit circle.
\end{remark}

\begin{remark}
In \cite{Feyzmahdavian:13}, it was shown that discrete-time positive linear systems are insensitive to time delays satisfying Assumption \ref{Assumption 1}. Theorem \ref{Theorem 1} shows that a similar delay-independent stability result holds for nonlinear positive systems whose vector fields are non-decreasing and homogeneous of degree zero. Furthermore, the impact of various classes of time delays on the convergence rate of positive linear systems has been missing in \cite{Feyzmahdavian:13}, whereas Theorem \ref{Theorem 1-2} provides explicit bounds on the decay rate that allow us to quantify the impact of bounded and unbounded time-varying delays on the decay rate.
\end{remark}



\section{An Illustrative Example}\label{sec:examples}

Consider the continuous-time system (\ref{System 3}) with
0.05cm] x_1^2x_2-4x_2^2 \end{bmatrix},\quad \g(x_1,x_2)=\begin{bmatrix}x_1x_2\
Both  and  are homogeneous of degree  with respect to the dilation map  with . Moreover,  is cooperative and  is non-decreasing on . Since ,  it follows from Theorem \ref{Theorem 3} that the homogeneous cooperative system (\ref{Example 2}) is globally asymptotically stable for any time delays satisfying Assumption \ref{Assumption 5}. Now, consider the specific time-varying delay , . As  for all , Corollary \ref{Corollary 3} can help us to calculate an upper bound on the decay rate of (\ref{Example 2}). Using  and , the solutions to (\ref{Corollary 3-1-1}) are , , which implies that

Thus, the solution  of (\ref{Example 2}) satisfies

Figure \ref{Figure 3} gives the simulation results of the actual decay rate of the homogeneous cooperative system~(\ref{Example 2}) and the guaranteed decay rate we calculated, when the initial condition is , for all .
\begin{figure}[h]
\centering
\includegraphics [width=0.7\columnwidth]{Figure3.eps}
\caption{Comparison of guaranteed upper bound and actual decay rate of the homogeneous cooperative system (\ref{Example 2}) corresponding to the initial condition , .}
\label{Figure 3}
\end{figure}


\section{Conclusions}\label{sec:conclusions}

This paper has been concerned with delay-independent stability of a significant class of nonlinear (continuous- and discrete-time) positive systems with time-varying delays. We derived a set of necessary and sufficient conditions for global asymptotic stability of continuous-time homogeneous cooperative systems of arbitrary degree and discrete-time homogeneous non-decreasing systems of degree zero with bounded and unbounded time-varying delays. These results show that the asymptotic stability of such systems is independent of the magnitude and variation of the time delays. However, we also observed that the decay rates of these systems depend on how fast the delays can grow large. We developed two theorems for global -stability of positive systems that quantify the convergence rates for various classes of time delays. For discrete-time homogeneous non-decreasing systems of degree greater than zero, we demonstrated that the origin is locally asymptotically stable under global asymptotic stability conditions that we derived.



\bibliographystyle{IEEEtran}
\bibliography{bibliografia}

\end{document}
