\documentclass[11pt]{amsart}
\usepackage[leqno]{amsmath}
\usepackage{epsfig}
\usepackage{graphicx}
\usepackage{cite}
\usepackage[utf8]{inputenc}
\usepackage{algorithm, algorithmic}
\usepackage{subfigure}
\pagestyle{plain} \pagenumbering{arabic} \oddsidemargin0.3cm
\evensidemargin0.3cm \topmargin1cm \headheight0cm \headsep5mm
\topskip0cm \textheight21.6cm \textwidth15.8cm
\footskip1.5cm
\renewcommand\baselinestretch{1.05}

\sloppy
\usepackage{epsfig}
\usepackage{color}
\usepackage{amsmath}
\usepackage{amssymb}
\usepackage{graphicx,color, algorithmic, algorithm}

\newtheorem{theorem}{Theorem}


\begin{document}

\thispagestyle{empty}

	\centerline{\Large\bf Embedding into the rectilinear plane in optimal  time}

	\vspace{6mm}

	\centerline{{\sc Nicolas Catusse, Victor Chepoi,} and {\sc Yann Vax\`es}}

	\vspace{3mm}
	\medskip
	\centerline{Laboratoire d'Informatique Fondamentale,}
	\centerline{Universit\'e d'Aix-Marseille,}
	\centerline{Facult\'e des Sciences de Luminy,} \centerline{F-13288
	Marseille Cedex 9, France} \centerline{catusse,chepoi,vaxes@lif.univ-mrs.fr}

	\vspace{15mm}
	\begin{footnotesize} \noindent {\bf Abstract.}
 In this paper, we present
an optimal  time algorithm for deciding if a metric space
 on  points can be isometrically embedded into the plane
endowed with the -metric. It improves the
 time algorithm of J. Edmonds (2008). Together with
some ingredients introduced by Edmonds, our algorithm uses the
concept of tight span and the injectivity of the -plane. A
different  time algorithm was recently proposed by D.
Eppstein (2009).
\end{footnotesize}


\section{Introduction}

Deciding if a finite metric space  admits an isometric embedding or an embedding with a small distortion
into a given geometric space (usually  endowed with some norm-metric) is a classical question
in distance geometry  which has some applications in theoretical computer science, visualization, and data analysis.
The first question can be answered in polynomial time
if  is endowed with the Euclidean metric due to classical results of Menger and Sch\"onberg \cite{DeLa}.
On the other hand, by a result of Frechet \cite{DeLa}, any metric space can be isometrically embedded into some
 with the -metric. However, it is NP-hard  to decide if a metric space isometrically
embeds into some  endowed with the  (alias rectilinear or Manhattan) metric \cite{AvDe,DeLa}. More recently,
Edmonds \cite{Ed} established that it is even NP-hard to decide if a metric space embeds into  with
-metric (a similar question for  with
-metric is still open).  In case of  both - and -metrics are equivalent because the
second metric can be obtained from the first one by a rotation of the plane by   and then by a shrink by a factor
 The embedding problem for the rectilinear plane was investigated in the papers \cite{BaCh_six,MaMa}, which ultimately
show that a metric space  embeds into the -plane if and only if any subspace
with at most six points does \cite{BaCh_six}  (a similar result for embedding into the -grid was obtained in \cite{BaCh_grid}).
As a consequence, it is possible to decide in polynomial time if a finite metric space embeds into the -plane.
Edmonds \cite{Ed} presented an  time algorithm for this problem and very recently we learned that
Eppstein \cite{Epp} described an optimal  time algorithm (for earlier algorithmic results, see also \cite{ChTr}).
In this note, independently of \cite{Epp}, we describe a simple and optimal algorithm for this embedding problem,
which is different from that of \cite{Epp}.

We conclude this introductory section with a few definitions. In the
sequel, we will denote by  or by  the -metric
and by  the -metric. A metric space 
is {\it isometrically embeddable} into a host metric space 
if there exists a map  such that
 for all  In this case
we say that   is a subspace of  A {\it retraction} 
of  a metric space  is an idempotent nonexpansive mapping of
 into itself, that is,  with
 for all  The
subspace  of  induced by the image of  under  is
referred to as a {\it retract} of  Let  be a metric
space.  The {\it (closed) ball} and the {\it sphere} of center 
and radius  are the sets  and
 respectively. The \emph{interval}
between two points  of  is the set   Any ball of 
is an axis-parallel cube. A subset  of  is \emph{gated} if for
every point  there exists a (unique) point  the
\emph{gate} of  in ,  such that  for all  \cite{DrSch}. The intersection of gated sets is also gated.  A
{\it geodesic} in a metric space is the isometric image of a line
segment. A metric space is called {\it geodesic} (or {\it
Menger-convex}) if any two points are the endpoints of a geodesic.


For a point  of  denote by  the
four quadrants of  defined by the vertical and horizontal
lines passing via the point  and labeled counterclockwise.
Any interval  of the rectilinear plane  is an
axis-parallel rectangle which can be
reduced to a horizontal or vertical segment. Any ball of  is a lozenge obtained from an axis-parallel square by a
rotation by  degrees. In the rectilinear plane, any halfplane defined
by a vertical or a horizontal line is gated. As a consequence,
axis-parallel rectangles, quadrants, and strips of   are gated as intersections of such halfplanes.




\section{Tight spans}

A metric space  is called {\it hyperconvex} (or {\it
injective})  \cite{ArPa,Is} if any family of closed balls
 with centers  and radii  
satisfying  for all  has a nonempty
intersection, that is,  is a geodesic space such that the
closed balls have the Helly property. Since the closed
balls of  are axis-parallel boxes, the
metric spaces  and 
are hyperconvex. It is well known \cite{ArPa} that  is
hyperconvex iff it is an absolute retract, that is,
 is a retract of every metric space into which it embeds
isometrically.  As shown by Isbell \cite{Is} and Dress \cite{Dr},
for every metric space  there exists the smallest injective
space  extending  referred to as the {\it injective
hull} \cite{Is}, or {\it tight span} \cite{Dr} of  The tight
span of a finite metric space  can be defined as follows. Let
 be the set of functions  from  to  such
that

\medskip
(1) for any  in   and

(2) for each  in  there exists  in  such that 

\medskip
One can interpret  as the distance from  to . Then (1)
is just the triangle inequality.  Taking  in (1), we infer that
 for all   The requirement (2) states that
 is minimal, in the sense that no value  can be reduced
without violating the triangle inequality. We can endow  with
the -distance: given two functions  and  in
 define  The resulting metric
space  is injective and  is called the
{\it tight span} of  There is an isometric embedding of 
into its tight span . Moreover, {\it any isometric embedding
of  into an injective metric space  can be extended
to an isometric embedding of  into } i.e.,
 is the smallest injective space into which  embeds isometrically. 



In general, tight spans are hard to
visualize. Nevertheless, if , Dress \cite{Dr} completely
described  via the interpoint-distances of . For example,
if  say   then  consists of three line
segments joined at a (Steiner) point, with the points of  at the
ends of the arms (see Fig. \ref{TS_3_4} (a)). The lengths of these
segments are  where
 is the Gromov product
of  with the couple  ( and  are defined
in a similar way). Notice that one of the values
 may be 0, in this case one point is
located between two others. If   then the generic form of
 is a rectangle  endowed with the -metric, together
with a line segment attached by one end to each corner of this
rectangle (see Fig. \ref{TS_3_4} (b)). The four points of  are the
outer ends of these segments. The lengths of these segments and the
sides of the rectangle can be computed in constant time from the
pairwise distances between the points of ;  for exact
calculations see \cite{Dr}. It may happen that  degenerates
into a segment or a point. Finally, there are three canonical types
of tight spans of 5-point metric spaces precisely described in
\cite{Dr} (see also Fig. \ref{TS_5} for an illustration). Each of
them consists of four or five rectangles, five segments, and
eventually one rectangular triangle, alltogether constituting a
2-dimensional cell complex. All sides of the cells can be computed
in constant  time as described in \cite{Dr}. It was also noticed in
\cite{Dr} that if for each quadruplet  of a finite metric space
 the rectangle  is degenerated, then 
isometrically embeds into a (weighted) tree and its tight span
 is a tree-network.



\begin{figure}
\begin{center}
\begin{tabular}{cc}
    \includegraphics[scale=0.5]{tight_span_3_noms.eps} & \hspace{2cm}
\includegraphics[scale=0.5]{tight_span_4_noms.eps} \vspace{-3mm}\\
(a) & \hspace{2cm}(b) \\
\end{tabular}
\end{center}
\caption{Tight span of 3- and 4-point metric space.}
\label{TS_3_4}
\end{figure}

\begin{figure}
\begin{center}
\begin{tabular}{ccc}
    \includegraphics[scale=0.5]{tight_span_5_1} &
    \includegraphics[scale=0.5]{tight_span_5_2} &
    \includegraphics[scale=0.5]{tight_span_5_3} \\
\end{tabular}
\end{center}
\caption{The three canonical types of tight span of 5-point metric space.}
\label{TS_5}
\end{figure}


From the construction of tight spans of 3- and 4-point metric spaces
immediately follows that any metric space  with at most 4
points and its tight span  can be isometrically
embedded into the -plane as shown in Fig. \ref{TS_3_4} (b). This is
no longer true for metric spaces on 5 points:
to embed, some cells of the tight span must be degenerated. If
 and the rectangle  is non-degenerated, one can easily
show that  isometrically embeds  into the -plane only as
an axis-parallel rectangle. Therefore, if additionally the four line
segments of  are also non-degenerated, then up to a rotation
of the plane by   and   admit exactly two
isometric embeddings into the -plane. If one corner of 
is a point of   and the embedding of the rectangle  is
fixed, then there exist three types of isometric embeddings of 
and  into the rectilinear plane: two segments of  are
embedded as axis-parallel segments and the third one as a segment
whose slope has to be determined. Analogously, if two incident
corners of  are points of  the two segments of  are
either embedded as axis-parallel segments, or one as a horizontal or
vertical segment and another one as segment whose slope has to be
determined. Note also that from the combinatorial characterization
of finite metric subspaces of the -plane presented in
\cite{BaCh_six} immediately follows that a tree-metric  is
isometrically embeddable into the -plane if and only if the
tree-network  has at most four leaves. Finally note that since
 is injective, by minimality property of tight
spans,  is an isometric subspace of the -plane for any
finite subspace  of .


\section{Algorithm and its correctness}


\subsection{Outline of the algorithm} Let  be a metric space with  points, called {\it terminals}. Set .
Our algorithm first finds in  time a
quadruplet  of  whose tight span contains a nondegenerated rectangle . If such a quadruplet
does not exists, then  is a tree-metric and  is a tree-network.  If this
tree-network contains more than four leaves, then  cannot be isometrically embedded into the -plane,
otherwise such an embedding can be easily derived. Given the required
quadruplet  we consider any isometric embedding  of  and of its tight span into the -plane as illustrated in Fig. \ref{TP*}
and partition the
remaining points of  into groups depending on their location in the regions of the plane defined by the rectangle  and the segments
of . The exact location of points of  in these regions is uniquely
determined except the four quadrants defined by . At the second stage, we replace the quadruplet  by another quadruplet   by picking one furthest
from  point of  in each of these quadrants.  We show that the rectangle  is contained in the rectangle  moreover,
for any isometric embedding  of  and  into the -plane,
the quadrants defined by two opposite corners
are empty (do not contains other terminals of ).  Again the location of the points of  in all regions of the plane except the two opposite
quadrants is uniquely determined. To compute the location of the remaining terminals in these two quadrants we adapt the second part of the algorithm of
Edmonds \cite{Ed}: we construct on these terminals a graph as in \cite{Ed}, partition it into connected components, separately determine the location of
the points of each component, and then combine them into
a single chain of components in order to obtain a global isometric embedding  of  extending  or to decide that it does not exist.

Now, we briefly overview the algorithms of Edmonds \cite{Ed} and
Eppstein \cite{Epp}. Edmonds \cite{Ed} starts by picking two
diametral points  of  These two points can be embedded into
the -plane in an infinite number of different ways. Each
embedding defines  an axis-parallel rectangle   whose
half-perimeter  is exactly  Using the distances of  and
 to the remaining points of  Edmonds computes a list 
of linear size of possible values of the sides of the rectangle
 For each value  from this list, the algorithm of
\cite{Ed} decides in  time if there exists an isometric
embedding of  such that one side of the rectangle  has
length . For this, it partitions the points of  into
groups, depending on their location in the regions of the plane
determined by  In order to fix the positions of points in one
of these regions, Edmonds \cite{Ed} defines a graph whose connected
components are also used  in our algorithm. While sweeping through
the list  the algorithm of \cite{Ed} update this graph and
its connected components in an efficient way. Notice that the second
part of our algorithm is similar to that from \cite{Ed}, but
instead of trying several sizes of the rectangle  we use the
tight spans to provide us with a single rectangle, ensuring some
rigidity in the embedding of the remaining  points. The algorithm of
Eppstein \cite{Epp} is quite different in spirit from our algorithm
and that of Edmonds \cite{Ed}. Eppstein \cite{Epp} first
incrementally constructs  in  time a planar rectangular
complex which is the tight span of the input metric space  or
decide that the tight span of  is not planar. In the second stage
of the algorithm, he decides in  time if this planar
rectangular complex can be isometrically embedded into the
-plane or not.


\subsection{Computing the quadruplet }\label{Pcirc} For each  set  We start by computing the tight span of the first four points of  If this tight span is not degenerated then we return the quadruplet  as  Now suppose that the tight span of the first  points of  is a tree-network 
with at most four leaves. This means that  contains one or two ramification points (which are not necessarily points of ) having degree at most 4,
all remaining terminals of  are either leaves or vertices of degree two of  We say that two terminals of  are consecutive in 
if the segment connecting them in  does not contain other points of  Note that  contains  at most  of consecutive
pairs. For each pair  of consecutive terminals of  we compute the Gromov product  of  with  Let  be the pair of consecutive points of  minimizing the Gromov product  Let  be the point of the segment  of  located at distance  from  and at distance  from  ( may coincide with one of the points  or ).

Denote by  the tree-network obtained from  by adding the segment  of length  By running  Breadth-First-Search on   rooted at   we check if  for any terminal  of  If this holds for all  then the tight span of  is the tree-network  If  contains more than 4 leaves, then we return the answer ``not''and the algorithm halts. Otherwise, if  then we return the answer  ``yes'' and an isometric embedding of  and its tight
span  in the -plane, else, if   we consider the next point  Finally, if  is the first point of  such that 
then we assert that {\it the tight span of the quadruplet  is non-degenerated and we return it as } Suppose by way of contradiction that  is a tree.
Since  realizes  and  realizes
, the subtree of  spanned by the terminals
 and  is isometric to the subtree of 
spanned by the same terminals. On the other hand,  contains a point  located at
distance  and  from
 and  respectively.
This means that  is isometric to the subtree of 
spanned by the vertices  and  (see Fig.~\ref{tree_A_i}) contrary to the
assumption that 
Hence, this inequality implies indeed that  is not a tree. Finally note that dealing with a current
point  takes time linear in  thus the whole
algorithm for computing the quadruplet  runs in  time.

\begin{figure}
\begin{center}
    \includegraphics[scale=0.5]{tree.eps}
\end{center}
\caption{The tree-network }
\label{tree_A_i}
\end{figure}




\subsection{Classification of the points of  with respect to the rectangle of }\label{Pcircb} Let 
be the quadruplet whose tight span  is non-degenerated.  Let
 be one of the two possible isometric embeddings of the rectangle  of  and consider a complete or a partial isometric embedding of  such that  is embedded as 
Denote by
 the four (closed) quadrants defined by the four consecutive corners  of  labeled in such a way that
the point  must be located in the quadrant  Let also  and  be the remaining half-infinite strips.
Since we know how to construct in constant time the tight span of  a 5-point
metric space, we can compute the distances from all terminals  of  to the corners of the rectangle  (and hence to the corners of ) in total  time. With some abuse of notation,
we will denote the -distance from  to the corner  of  by 
Since  is gated,
from the distances of  to the corners  of  we can compute the gate of   in  Consequently, for each point  we can decide in which of
the nine regions of the plane will belong its location  under any isometric embedding  of  subject to the assumption that  is embedded as .
If  belongs to one of the four half-strips or to , then we can also easily find
the exact location itself: this can be done by using either the gate of  in  or the fact that inside these five regions the intersection of the four -spheres centered at the corners of  and having the distances from respective corners to  as radii is a single point. So, it remains to decide the locations
of points assigned to the four quadrants  and   For any point 
which must be located in the quadrant  the set of possible locations of  is either empty (and no isometric embedding exists) or a segment  of  consisting of all points  such that 

Notice that for any quadruplet  of terminals such that  is assigned to the quadrant   {\it the rectangle  belongs to the tight span  of } Indeed, for any point  and any point  of  we have  where  is selected in such a way that  and  are opposite corners of  From injectivity of the -plane and the characterization of tight spans we conclude that all points of  belong
to  establishing in particular that this tight span is also non-degenerated.


\subsection{The quadruplet  and its properties}\label{P} Let 
be the quadruplet of  where  is a point of  which must
be located in the quadrant  and is maximally distant
from the corner  of . As we established
above, the tight span of  is non-degenerated,  moreover the
rectangle  contains the rectangle  As we also
noticed, there exists a constant number of ways in which we can
isometrically embed   into the -plane. Further we proceed
in the following way: we pick an arbitrary isometric embedding
 of  and try to extend it to an isometric embedding
 of the whole metric space  in the -plane. If
this is possible for some embedding of , then the algorithm
returns the answer ``yes'' and an isometric embedding of 
otherwise the algorithm returns the answer ``not''. Let  denote
the image of  under 

\begin{figure}
\begin{center}
\includegraphics[scale=0.5]{table_cases.eps}
\end{center}
\caption{Possible isometric embeddings of }
\label{TP*}
\end{figure}





We call a terminal  of  {\it fixed} by the embedding  if either 
is a corner of the rectangle  or the segment of  incident to  is embedded by
 as a horizontal or a vertical segment; else we call  {\it free}.
The embedding of a free terminal  is not exactly determined but is restricted
to a segment  consisting  of the points of the quadrant defined by 
and having the same -distance to  We call the terminals
  {\it incident} and
the terminals  {\it opposite}.
From the isometric embedding of  we conclude that at most one of
two incident terminals can be free. Moreover, if a terminal  of  is
fixed but is not a corner of  then at least one of the two terminals
incident to  is also fixed. If all four tips of  are non-degenerated,
then all four terminals of  are fixed. If only three tips of  are
non-degenerated then at most one terminal of  is free, all remaining
terminals are fixed. If only two tips of  are non-degenerated,
then either they correspond to incident terminals, one of which is
fixed and another one is free or to two opposite terminals which are both free.
Finally, if only one tip of the tight span is non-degenerated, then it corresponds
to a free terminal, all other terminals of  are corners of  and therefore
are fixed (see Fig. \ref{TP*} for the occurring  possibilities).


\begin{figure}
\begin{center}
\includegraphics[scale=0.6]{case1a_regions.eps}
\hspace*{2cm}
\includegraphics[scale=0.6]{case2b_regions.eps}
\end{center}
\caption{The partition of the plane into half-strips and quadrants.}
\label{regions}
\end{figure}


 Denote by  the smallest axis-parallel rectangle containing  and the fixed terminals
 of  Fig. \ref{regions} illustrates  for two cases from Fig. \ref{TP*}
 (if a terminal is free, then the respective corner of  is also a corner of
 ). Let  be the corners of  labeled in such a way that  is the corner of
 corresponding to the point  and to the corner 
of  Denote by  the quadrants of
 defined by
 the corners of  and by  the remaining half-infinite strips.
 Again, as in the case of the quadruplet , by building the tight spans
 of  for all terminals  we can compute in total
 linear time the distances from all such points  to the corners of  (and to
 the corners of ). From these four distances and the distances of  to the
 terminals of the quadruplet  we can determine in which of the nine regions
  of the plane must be located 
 Moreover, if  is assigned to the rectangle  or to one of the four
 half-strips  then we can conclude that, in the region 
 in which  assigned, the intersection
 of the four spheres centered at the terminals of  and having the distances
 from respective points to  as radii is either empty or a single point. The
 sphere centered at a free terminal  is needed only to decide the location of
  in the quadrant  of the plane having the same apex  as the quadrant 
 and which is opposite to   ( is a corner of ). But in this case,
 instead of considering the sphere of radius  centered at 
 we consider the sphere of radius  and centered
 at  indeed, both these spheres have the same intersection with .





We are now ready to prove the following property of the quadruplet  {\it among the four quadrants
 and  defined by  two opposite quadrants are empty,} i.e., they do not contain terminals
of . First note that by inspecting the different cases listed in Fig.~\ref{TP*} one can check that the
two neighbors  and  of a free point  are both
fixed; let say  and  are fixed. Now, suppose by way of contradiction that a terminal  must be located
in the quadrant  This means that its gate in the rectangle  is the corner of  corresponding to . Since in any embedding  of 
that extends the chosen embedding of  the terminal  is located in  we deduce that  
On the other hand, the inclusion  follows directly from the definition of  and the fact
that  is fixed. Now, from the inclusions , we obtain that  and,
since  is closer to  than to  we get a contradiction with the choice of 
establishing that indeed  does not contain any point of  The same argument
shows that  is empty as well. Note that actually we proved that any quadrant  corresponding to a
fixed terminal  of  is empty.


\begin{figure}
\begin{center}
\includegraphics[scale=0.6]{free_regions.eps}
\end{center}
\caption{On possible locations of terminals in  and }
\label{zone}
\end{figure}




\subsection{Locating in the non-empty quadrants  and }\label{Q}

As we have showed in previous subsection, any isometric embedding  of  extending the embedding  of  locates each terminal  of 
in one and the same of the nine regions defined by  Moreover, if  must be located in the rectangle  or in one of the four half-strips 
then this location  is uniquely determined from the distances to the terminals of  and to the corners of  We also established that
only one or two opposite quadrants defined by , say  and  can host terminals of  see Fig. \ref{zone}. We will show now how
to find the exact location of the set  of terminals assigned to  (the set  of terminals which must be located in  is treated analogously).

Note that independently of how the extension  of  is chosen, for each terminal  the -distance  from the location of  to the corner  of  is one and the same, which we denote by  The value of    can be easily computed because  lies between  and  for any : for example, we can set    Then the set of all possible locations  of  is the {\it level segment}  which is the intersection of   with the sphere  of radius   centered at 

To compute the locations of the terminals of  in the quadrant , we adapt to the -plane
the definition of a graph (which we denote by ) defined by Edmonds \cite{Ed} in the -plane.
Two terminals  are adjacent in  if and only if   Equivalently
 with  are adjacent in  iff  cannot be located between  and
   Denote by  the connected components of the graph 
They have the following useful properties established in Lemmata 3-5 of \cite{Ed}:





\medskip\noindent
(1) Each component  is {\it rigid,} i.e., once the location
of any point  of   has been fixed, the locations of the remaining points of 
are also fixed (up to symmetry with respect to the line parallel to the bisector of   and
passing via );

\medskip\noindent
(2) The components  of the graph  can be numbered so that the
points of each  appear consecutively in the list of points  
sorted in increasing order of their distances  to 

\medskip\noindent
(3) For a component  of  let  be the smallest
axis-parallel rectangle containing  for an isometric embedding 
of  in the -plane. Let  be the upper right corner
of  Then the embedding of  preserves
the distances between all pairs of points lying in different
components if and only if for every pair of consecutive components
 and  the rectangle  lies entirely in the
quadrant 

\begin{figure}
    \begin{minipage}[b]{.48\linewidth}
        \centering
        \includegraphics[scale=0.5]{fixed_points.eps}
        \caption{ and  are fixed by  and }
        \label{points_fixes}
    \end{minipage} \hfill
    \begin{minipage}[b]{.48\linewidth}
        \centering
        \includegraphics[scale=0.5]{fixed_v.eps}
        \caption{ is fixed by }
        \label{fixe_v}
    \end{minipage}
\end{figure}


\medskip
The location in the quadrant  of some terminals of  (and
therefore of the connected components containing them) can be fixed
by terminals already located in the two half-strips incident to
 We say that a terminal  is {\it fixed by a
terminal}  already located in  if the intersection
of the segment  with the sphere  is a
single point. Note that if  is fixed by a terminal located
in  then  is also fixed by the upmost terminal 
located in this half-strip. Analogously, if  is fixed by a
terminal of   then  is also fixed by the rightmost terminal
 located in  Therefore by considering the intersections
of the segments  with the spheres
 and  we can
decide in linear time which terminals of  are fixed by 
and  and find their location in  (for an illustration, see
Fig. \ref{points_fixes}). According to property (1), if a terminal
of a connected component of  is fixed, then the location of the
whole component is also fixed (up to symmetry).  Let  be the
connected component of  containing the furthest from 
terminal  fixed by  or  say by  (therefore
the location of  is fixed). We assert that {\it  all terminals
of  are also fixed by .} Indeed, pick
such a terminal  From property (2) we conclude that  and from the definition of  we deduce that  must
be located in the axis-parallel rectangle  and
therefore below . Since  is fixed by   must be
located below , whence  also must be located below  We
can easily see that the intersection of  with the sphere
 is a single point, i.e.  is also fixed
by   (see Fig. \ref{fixe_v}).

\begin{figure}
    \begin{center}
    \includegraphics[scale=0.6]{blocs.eps}
    \end{center}
    \caption{On the assemblage of blocks }
    \label{blocs}
\end{figure}


It remains to locate in  the terminals of the 
components  We compute separately an
isometric embedding of each component  for 
For this, we fix arbitrarily the location of the first two
points  of  in the segments  and  so that to
preserve the distance  (the terminals of  are ordered
by their distances to ). By property (1) of \cite{Ed}, the
location of the remaining points of  is uniquely determined and
each  point  of  will be located in its level segment 
Let  be the resulting embedding of .  Denote by
 the smallest axis-parallel rectangle (alias {\it box})
containing the image  of  Let  and 
denote the lower left and the upper right corners of   Note
that  belongs to the -interval between  and the image
 of any terminal  of  while the
-interval between  and  will contain the images of
all terminals of  Therefore if we set
 and
 where  is any
terminal of  then in all isometric embeddings of  in
which all terminals  are located on  the points
 and  must be located on the level segments  and
  defined as the intersections of the quadrant  with
the spheres  and 



By properties (2) and (3) of \cite{Ed}, in order to define a single
isometric embedding of the components  we now
need to assemble the boxes  (by moving
their terminals along the level segments) in such a way that {\it
for two consecutive components  and  the box
 lies entirely in the quadrant }  We assert
that this is possible if and only if {\it for each pair of
consecutive boxes   the
inequality  holds.}  Indeed, if
 then translating  
along the segment  we can locate its corner 
in the quadrant  and thus satisfy the embedding
requirement.  Conversely, if  holds,
then   cannot belong to the quadrant 
independently of the positions of  and  on their level
segments. This local condition depends only of the values of
 and is independent of the actual
location of the boxes   As a result, the
algorithm that embeds the boxes  is very
simple. For each  we compute the box  and
the values of  and  If
 for some  then return the
answer ``there is no isometric embedding of  extending the
embedding  of ". Otherwise, having already located
the box  by what has been shown above, the intersection of the
quadrant  with the level segment  is
non-empty. Therefore we can translate  in such a way that
its lower left corner  becomes a point of this
intersection.

In this way, we obtain an embedding of  and
 satisfying the conditions (1)-(3), thus an
isometric embedding of the metric space 
in  Analogously, by constructing the graph  and
its components, either we obtain a negative answer or we return an
isometric embedding  of the metric space defined by the non-fixed
components of  in the quadrant  Denote by  the
embedding of  which coincides with  on  with these
two embeddings on the non-fixed components of  and  and
with the already computed fixed locations of the terminals assigned
to  to the half-strips  and to the fixed
connected components of the graphs  and  In  we
test if  is an isometric embedding of  into the
-plane. If the answer is negative, then we return ``there is no
isometric embedding of  extending the embedding 
of ", otherwise we return  as an isometric embedding.
The algorithm returns the global answer ``not'' if for all possible
embeddings   of  it returns the negative answer.
From what we established follows that in this case  is not
isometrically embeddable into the -plane.


\subsection{Algorithm and its complexity}

We conclude the paper with a description of the main steps of the algorithm and their complexity.


\medskip
{\footnotesize
\noindent{{\sf Algorithm} \bf Embedding into the -plane}\\
{\bf Input:} A metric space  on  points\\
{\bf Output:} An isometric embedding  of  into 
or the answer ``not'' if it does not exist\\
{\bf Step 1.}  Find a quadruplet  of  whose tight span contains
a rectangle. If  does not exist, then  is a tree. If 
 has more than 4 leaves, then return ``not", else return an embedding of  and .\\
{\bf Step 2.}  Pick any embedding of  and for each terminal of 
determine in which of the nine regions of the plane it must be located.
  Using this partition of  define the quadruplet .\\
{\bf Step 3.}  Embed  and its tight span  into the -plane in all
possible different ways.  Try to extend each of these embeddings
  to an isometric embedding of  following the rules (a)-(g).
  If all of these attempts return the answer ``not'', then return the answer ``not", else return one of the obtained  embeddings.
\begin{itemize}
\item[(a)]  Given an embedding  of  for each terminal  of  determine in which of the nine regions defined by the rectangle  will be located  in any isometric embedding extending  
\item[(b)]  Locate the terminals assigned to the rectangle  and the four half-strips  
\item[(c)]  Define the sets of terminals  and  assigned to the quadrants  and  construct the graphs  and  and their connected components;
\item[(d)]  Find the terminals of  fixed by  and their location in . Do a similar thing for 
\item[(e)]  Find an isometric embedding of each component  of  not containing already fixed terminals so that its terminals are located on their level segments. Do a similar thing for ;
\item[(f)]  Test if the free components  of  satisfy the condition   for  If not, then return the answer ``not", else locate consecutively the boxes  in such a way that   is located in  and fix in this way the position of all terminals of  Do a similar thing for the free components of 
\item[(g)]  Verify if the resulting embedding of  extending  is an isometric embedding of . If ``yes'', then return it as a resulting isometric embedding, otherwise return the answer ``there is no isometric embedding of  extending the embedding ''.
\end{itemize}
}


\medskip
In Subsection \ref{Pcirc} we established that the quadruplet
 if it exists, can be computed in  time. If
 does not exists, then the tree-network 
(constructed within the same time bounds) is the tight span of
 Embedding  (if it has at most 4 leaves) in the
-plane can be easily done in linear time. As shown in
Subsection \ref{Pcircb}, Step 2 can be implemented in linear time.
There exists a constant number of ways in which the quadruplet 
and its tight span can be isometrically embedded in the -plane.
Therefore, to show that
Step 3 has complexity , it suffices to estimate the total
complexity of the steps (a)-(g) for a fixed embedding  of
 Step (a) is similar to Step 2, thus its complexity is
linear. The exact location of each terminal in the half-strips or in
 is determined as the intersection of two spheres, therefore
step (b) is also linear. Defining the graph  and computing its
connected components can be done in  time. Thus step (c)
has complexity  Steps (d) and (e) can be implemented in an
analogous way as (b), thus their complexity is  Testing the
condition in step (f) and assembling the free components into a
single chain is linear as well. Finally, step (g) requires 
time. Therefore, the total complexity of the algorithm is 
Summarizing, here is the main result of this note:

\begin{theorem} For a metric space  on  points, it is possible to decide in optimal  time if  is isometrically embeddable into the -plane and to find such an embedding if it exists.

\end{theorem}



\begin{thebibliography}{99}
\bibitem{ArPa} N. Aronszajn and P. Panitchpakdi, Extensions of uniformly continuous transformations
and hyperconvex metric spaces, {\it Pacific J. Math.} {\bf 6} (1956), 405--439.
\bibitem{AvDe} D. Avis, M. Deza, The cut cone,  embeddability, complexity and multicommodity flows, {\it Networks} {\bf 21} (1991),
595--617.
\bibitem{BaCh_six} H.--J. Bandelt and V. Chepoi,  Embedding metric spaces in the rectilinear
plane: a six--point criterion, {\it Discr. Comput. Geom.} {\bf 15} (1996) 107--117.
\bibitem{BaCh_grid} H.-J. Bandelt, V. Chepoi, Embedding into the rectilinear grid, {\it Networks}  {\bf 32} (1998), 127--132.
\bibitem{ChTr} G.E. Christopher and M.A. Trick, Faster decomposition of totally decomposable metrics with applications,
Preprint, Carnegie Mellon University (1996).
\bibitem{DeLa}  M. Deza, M. Laurent, {\it Geometry of Cuts and
Metrics,}  Springer-Verlag, Berlin, 1997.
\bibitem{Dr} A.W.M. Dress, Trees, tight extensions of metric spaces, and the cohomological
dimension of certain groups,  {\it Adv. Math.}  {\bf 53} (1984), 321--402.
\bibitem{DrSch}  A. Dress and R. Scharlau, Gated sets in metric spaces,
    {\it Aequat. Math.} {\bf 34} (1987), 112--120.
\bibitem{Ed} J. Edmonds. Embedding into  is easy, embedding into  is NP-complete, \emph{Discr. Comput. Geom.} {\bf 39} (2008), 747--765.
\bibitem{Epp} D. Eppstein, Optimally fast incremental Manhattan plane embedding and planar tight span construction, Electronic preprint arXiv:0909.1866v1, (2009).
\bibitem{Is} J. Isbell, Six theorems about metric spaces, {\it Comment. Math.
Helv.} {\bf 39} (1964), 65--74.
\bibitem{MaMa} S.M. Malitz and J.I. Malitz,  A bounded compactness theorem for
--embeddability of metric spaces in the plane, {\it Discr. Comput. Geom.}
{\bf 8} (1992) 373--385.

\end{thebibliography}




\end{document}
