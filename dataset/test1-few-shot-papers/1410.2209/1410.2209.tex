\subsection{Proofs}

\begin{proof}[of Theorem~\ref{fft:exp}]
Let ,  be the ring of integers modulo . Since each coefficient of  is a positive integer less than , it \newb{suffices} to find , where . Note that one can perform arithmetic operations in  in time . Since ,  and  can be multiplied in time  (see, e.g., Exercise~17.24 in~\cite{S2009}).
\end{proof}

\begin{proof}[of Theorem~\ref{fft:poly}]
Assume one is to find the coefficient  of the monomial  in .
By the prime number theorem, there exist \newb{at least}  primes in the interval . Using a deterministic polynomial primality-testing algorithm (e.g., the AKS algorithm~\cite{AKS2004}) one can find primes  in this interval in time . By the Chinese remainder theorem, it suffices to find  modulo  in the desired time and space.

We show how to find  modulo . Since , one can factor  in deterministic time  using trivial division. Given the factorization of \newb{, one finds a primitive root of unity in  in  steps as follows. For every , one checks if  for all , this takes  steps.
Now the} coefficient  of  in  equals 

Since  can be evaluated in  in  time and space, one needs  time and only  space to find one coefficient of~.
\end{proof}

For a subset , let  denote the characteristic
vector of the set  (i.e.,  if and only if ). 
In the analysis below we sometimes identify a bit vector  with a non-negative integer between  and  that it represents.

For a bit vector , we denote the Hamming weight of , \newb{i.e., the number of 's in }, by . Note the following simple fact: for any two non-negative integers
 and ,

and the equality holds if and only if there are no carries in .



\begin{lemma}
\label{lemma:binary}
Let  be polynomials of  variables  with non-negative coefficients. Let also 

Then  contains the monomial 
if and only if  contains the monomial .
\end{lemma}

\begin{proof}
One direction of this statement is straightforward.
If  contains the (multilinear) monomial , then
 contain multilinear monomials  such that
. Then 
 and


For the reverse direction, assume that  contains
the monomial . Because of the term ,
there exist  monomials  such that 
. In other words, the total number of 's in 
all characteristic vectors of 's at most~. Moreover,

From (\ref{eq:carries}), one concludes that the equality  (\ref{eq:sum})
is only possible when 
and there are no carries in  (\ref{eq:sum}).
This in turn implies that  is a partition of~.
\end{proof}

\begin{proof}[of Theorem~\ref{fft:multi}]
Let for , , where 's are multilinear polynomials. Let also 

From Lemma~\ref{lemma:binary},  contains the monomial   if and only if  contains the monomial  .

Now it suffices to show how to efficiently find the sum of coefficients of the monomials  for . Indeed, since all the coefficients are positive, the binary search on  gives us the smallest  s.t.  contains , and the coefficient of this monomial in . 

It is easy to see that the degree of  in  does not exceed . Similarly, the degree of  does not exceed . Therefore, in order to obtain univariate polynomials we can use Kronecker substitution~\cite{K1882}. Namely, we replace  by , and  by . Thus, for each  we consider a univariate polynomial :

It it easy to see that the coefficient of  (where ) in  equals the coefficient of  in .
In other words, we associate an integer from  with each monomial , where  is a monomial over . This integer is an encoding of  and  in  bits, s.t. the first  bits indicate elements of , the next  bits are zeros, then  bits are the binary expansion of ,  zeros again, and the last  bits encode .
We need to find the sum of coefficients of  in  for .
 In order to do that, we multiply  by . The coefficient of 
 in the obtained polynomial  is equal to the sum of coefficients of  in  for all .

Note that the degree of  is . Now Theorem~\ref{fft:exp} and Theorem~\ref{fft:poly} finish the proof. Namely, to prove , we just apply the FFT from Theorem~\ref{fft:exp}  times. In order to prove , we note that  can be easily obtained from the values .
\end{proof}




\begin{definition}
For a matrix  let

\end{definition}

\begin{definition}
A matrix  is called \emph{row-normalized} if  for all .
\end{definition}
\begin{remark}
In the analysis below we will need the following simple estimates. Let . Then
\begin{enumerate}
\item  if and only if 
\item .
\item  (assuming ).
\end{enumerate}
\end{remark}

The following fact is well known so we state it without a proof.

\begin{lemma}
\label{lemma:system}
The expansion of  in powers of  as  has the minimal value of the sum of digits  if and only if  (i.e.,  is written in the numeral system of base ).
\end{lemma}


\begin{lemma}
\label{lemma:coding}
Let  and 
and  be row-normalized matrices.
If
, 
,
,
,
then .
\end{lemma}

\begin{proof}
The claim follows from Lemma~\ref{lemma:system}.
Since for all , ,  has the minimal sum of digits in base  system. This in turn implies that for each , . Then the first columns of matrices  and  are equal modulo , because parities of rowcodes depend only on the first column. Since  we conclude that the first columns of  and  are equal.

Now each  has the minimal sum of digits in the system of base , which means that  has the minimal possible value for these . It follows from Lemma~\ref{lemma:system} that each  must be equal to .
\end{proof}

\begin{definition} 
An injective function  is called a \emph{matrix representation} of a set . For such  and , a~\emph{characteristic matrix}  is defined as follows:  if and only if .
\end{definition}

\begin{proof}[of Theorem~\ref{thm:maininfants}]
Let  be a {-system of families with infants}
for a \newb{}. Append arbitrary elements from  to families so that the size of each family equals  and the families are still disjoint (this is possible since ). Denote the union of  families by  and the rest of  by~. 
For each family~, fix an order of its elements such that
the th element is~. Now consider a matrix representation  defined as follows.  If  is the th element of , then .
We encode each monomial  by parts. We encode elements from  using the standard technique from Lemma~\ref{lemma:binary}. To encode elements from  we use the characteristic matrix . Note that  is a {row-normalized matrix}, because if  contains an infant  of a family~, then it must contain at least one other element from the same row.
Consider the following polynomials for : 
is equal to

where  and  is an integer from  to .

We claim that  contains the monomial 
if and only if  contains the monomial
 
Indeed, as it was shown in Lemma~\ref{lemma:binary},  corresponds to partitions of . Lemma~\ref{lemma:coding} implies that only partitions of  may have the term 
 Note that the degrees of  in  are bounded from above by , the degree of  is bounded by , the degree of  is bounded by , the degree of  is bounded by , \newb{the degree of  is bounded by }.  Now we can apply Kronecker substitution as in the proof of Theorem~\ref{fft:multi}:


\newb{}

The running time of FFT is bounded by the degree of the resulting univariate polynomial, i.e.
\newb{

}
The required statements now follow from Theorems \ref{fft:exp} and \ref{fft:poly}.
\end{proof}





