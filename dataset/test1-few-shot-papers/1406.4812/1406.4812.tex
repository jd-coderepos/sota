\documentclass[conference]{IEEEtran}
\IEEEoverridecommandlockouts
\usepackage{cite}
\usepackage{amsmath,amssymb,amsfonts,bm}
\usepackage{amsthm}
\usepackage{algorithmic}
\usepackage{graphicx}
\usepackage{textcomp}
\usepackage{xcolor}
\usepackage{cuted}
\usepackage[section]{placeins}
\usepackage{stfloats}
\usepackage{caption}
\makeatletter
  \newcommand\tabcaption{\def\@captype{table}\caption}
\makeatother
\newtheorem{theorem}{Theorem}
\newtheorem{remark}{Remark}
\newtheorem{example}{Example}
\def\BibTeX{{\rm B\kern-.05em{\sc i\kern-.025em b}\kern-.08em
    T\kern-.1667em\lower.7ex\hbox{E}\kern-.125emX}}
\begin{document}

\title{A benchmark generator for boolean quadratic programming\\
}

\author{\IEEEauthorblockN{Xiaojun Zhou}
\IEEEauthorblockA{\textit{School of Automation} \\
\textit{Central South University}\\
Changsha, China \\
410083}
\and
\IEEEauthorblockN{Tingwen Huang}
\IEEEauthorblockA{\textit{School of Engineering} \\
	\textit{Texas A\&M University at Qatar}\\
	Doha, Qatar \\
	PO Box 23874}
}

\maketitle

\begin{abstract}
For boolean quadratic programming (BQP), we will show that there is no duality gap between the primal and dual problems under some conditions by using the classical Lagrangian duality.
A benchmark generator is given to create random BQP problems which can be solved in polynomial time. Several numerical examples are generated to demonstrate the effectiveness of the proposed method.
\end{abstract}
\setlength{\parskip}{4pt}
\begin{IEEEkeywords}
Lagrangian duality, Boolean quadratic programming, Polynomial time, Benchmark generator
\end{IEEEkeywords}

\section{Introduction}
Boolean quadratic programming (BQP) has found many applications (see \cite{billionnet2007using} the  references therein), for example, the Boolean networks \cite{cheng2010linear}, and the well-known maxcut problem \cite{marti2009advanced}, can be formulated as BQP. Various approaches have been used to solve BQP, including semidefinite programming (SDP) relaxation \cite{poljak1995recipe}, second order cone programming (SOCP) relaxation  \cite{kim2001second}, decomposition method \cite{elloumi2000decomposition} and randomized heuristics \cite{festa2002randomized}. However, these approaches are only able to find inexact solutions to BQP. On the other hand, methods based on branch-and-cut \cite{leyffer2001integrating} can find exact solutions, but they are usually time-comsuming. 
Therefore, there exists a necessity to evaluate the performance of approaches to solving BQP. 
To provide testbed for these approaches, a benchmark generator for BQP problems is proposed in this study. 

\section{Theoretical basics}
Considering the following boolean quadratic programming (BQP) problem

where,  is a given indefinite matrix,  is a given nonzero vector.

By introducing the Lagrange multiplier  associated with each constraint , the Lagrangian  can be defined as

where,  is a vector with all entries one, and  is

here,  is a diagonal matrix with its diagonal entries .

The Lagrangian dual function can be obtained by


Let define the following dual feasible space

then the Lagrangian dual function can be written explicitly as

associated with the Lagrangian equation


Finally, the Lagrangian dual problem can be obtained as


\begin{theorem}\label{the1}
If  is a critical point of the Lagrangian dual function and , then the corresponding
 is a global solution to the BQP problem.
\end{theorem}
\begin{proof}
The derivative of  gives

where, . Since  is a critical point of the Lagrangian dual function, we have .

To continue, we have

that is to say,  is the global minimizer of the BQP problem.
This completes the proof.
\end{proof}
\begin{remark}
Under the conditions in Theorem \ref{the1}, it is easy to verify that

which shows that there is no duality gap between the primal and dual problems.
\end{remark}

\section{A benchmark generator for BQP}
As stated above, if there exists a critical point of Lagrangian dual function in , zero duality gap will be met.
In this section, we will construct a benchmark generator for BQP problem such that these conditions are satisfied.

The inverse problem can be simplified as follows

where, .

Suppose  is a freely random symmetric matrix, to guarantee , let  satisfy

to make sure that  is a diagonally dominant matrix.

Suppose  is a freely random ``true" solution, then  should satisfy


The matlab scripts for generating a benchmark BQP are given in the following
\begin{verbatim}
function [Q,c,x,lambda] = generate_Qc(n)
base = 10;
Q = base*randn(n);
Q = round((Q + Q')/2);
lambda = zeros(n,1);
x = round(rand(n,1));
x = 2*x - 1;
lambda = sum(abs(Q),2);
c = (Q + diag(lambda))*x;
\end{verbatim}
where, \textit{base} is set to control the range of elements in .\\
\section{Numerical experiments}
The Lagrangian dual problem can be reformulated as the following SDP problem easily via Schur complement \cite{boyd1994linear}


In the next, several examples are created by the above mentioned generator function \textit{generate\_Qc}.
All of the experiments are run on MATLAB R2010b on Intel(R) Core(TM) i3-2310M CPU @2.10GHz under Window 7 environment.
The SDPT3 \cite{toh1999sdpt3} is used as a solver embedded in YALMIP \cite{lofberg2004yalmip} for the SDP problem.
\begin{example}\label{exa1}

\end{example}
By solving the corresponding Lagrangian dual problem in 0.9204 seconds, we can get  and  =.

\begin{strip}
\begin{example}\label{exa2}
 
\end{example}
\end{strip}

By solving the Lagrangian dual problem  in 0.9204 seconds, we can get

and  .

\begin{strip}
\begin{example}\label{exa3}
\setcounter{MaxMatrixCols}{20}
 
\end{example}
\setlength{\parskip}{4pt}
\captionsetup[table]{labelformat=simple ,labelsep=quad}
\renewcommand{\thetable}{\arabic{table}}
\end{strip}


By solving the Lagrangian dual problem  in 0.9204 seconds, we can get

and .

\begin{remark}
	All of the tested problems are solved in 0.9204 seconds, which can be regarded as an indicator for polynomial time complexity.
\end{remark}

\begin{example}
\textup{Other large random BQP problems are created by the same procedure, and the corresponding running time for solving these problems can be found in Fig. \ref{fig_runtime}.}
\end{example}

\begin{figure}[htbp]
	\centerline{\includegraphics[width=8cm,height=6cm]{running_time.pdf}}
	\caption{Running time for other large random BQP cases}
	\label{fig_runtime}
\end{figure}



\section{Conclusion}
In this short note, a benchmark generator for boolean quadratic programming (BQP) was implemented in Matlab. And these BQP instances can be used for evaluating the performance of various approaches. 


\begin{thebibliography}{00}
\bibitem{billionnet2007using}
Billionnet A, Elloumi S. Using a mixed integer quadratic programming solver for the unconstrained quadratic 0-1 problem. Mathematical Programming, 109(1): 55--68, 2007.
\bibitem{cheng2010linear}
Cheng D, Qi H. A linear representation of dynamics of Boolean networks. IEEE Transactions on Automatic Control, 55(10): 2251-2258, 2010.
\bibitem{marti2009advanced}
Marti R, Duarte A, Laguna M. Advanced scatter search for the max-cut problem. INFORMS Journal on Computing, 21(1): 26-38, 2009.
\bibitem{poljak1995recipe}
Poljak S, Rendl F, Wolkowicz H. A recipe for semidefinite relaxation for (0, 1)-quadratic programming. Journal of Global Optimization, 7(1): 51-73, 1995.
\bibitem{kim2001second}
Kim S, Kojima M. Second order cone programming relaxation of nonconvex quadratic optimization problems. Optimization Methods and Software, 15(3-4): 201--224,  2001.
\bibitem{elloumi2000decomposition}
Elloumi S, Faye A, Soutif E. Decomposition and linearization for 0-1 quadratic programming. Annals of Operations Research, 99(1-4): 79-93, 2000.
\bibitem{festa2002randomized}
Festa P, Pardalos P M, Resende M G C, et al. Randomized heuristics for the MAX-CUT problem. Optimization methods and software, 17(6): 1033--1058, 2002.
\bibitem{leyffer2001integrating}
Leyffer S. Integrating SQP and branch-and-bound for mixed integer nonlinear programming[J]. Computational optimization and applications, 18(3): 295--309, 2001.

\bibitem{boyd1994linear} Boyd S P, Ghaoui L EI, Feron E and Balakrishnan V, Linear matrix inequalities in system and control theory, Philadelphia: SIAM, 1994.
\bibitem{toh1999sdpt3} Toh K C, Todd M J and T{\"u}t{\"u}nc{\"u} R H, SDPT3-a MATLAB software package for semidefinite programming, version 1.3, Optimization methods and software, 11:545--581, 1999.
\bibitem{lofberg2004yalmip} Lofberg J, YALMIP: A toolbox for modeling and optimization in MATLAB, IEEE International Symposium on Computer Aided Control Systems Design, pp. 284--289, 2004.

\end{thebibliography}
\end{document}
