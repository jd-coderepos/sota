\documentclass[11pt,letter]{article}

\usepackage{amsmath}
\usepackage{xspace}
\usepackage{amssymb}
\usepackage{epsfig}
\usepackage{graphicx}

\newtheorem{Thm}{Theorem}
\newtheorem{Lem}[Thm]{Lemma}
\newtheorem{Cor}[Thm]{Corollary}
\newtheorem{Prop}[Thm]{Proposition}
\newtheorem{Claim}{Claim}
\newtheorem{Obs}{Observation}
\newtheorem{Def}{Definition}
\newtheorem{Fact}{Fact}
\newtheorem{Case}{Case}
\newenvironment{proof}{\noindent {\textbf{Proof }}}{ \medskip}

\newcommand {\Comments} {\vspace{1em} \noindent \textbf{Comments:}}
\newcommand\mbZ{\mbox{}}
\newcommand\mbC{\mbox{}}
\newcommand\mcX{\mathcal{X}}
\newcommand\mcY{\mathcal{Y}}
\newcommand\B{\{0,1\}}     
\newcommand\Bn{\{0,1\}^n}
\newcommand\BntB{\{0,1\}^n\rightarrow \{0,1\}}
\newcommand {\ie} {\textit{i.e.}\xspace}
\newcommand {\st} {\textit{s.t.}\xspace}
\newcommand {\WP} {with probability\xspace}
\newcommand\defeq{\stackrel{\mathrm{\scriptsize def}}{=}}

\newcommand\ket[1]{| #1 \rangle}
\newcommand\bra[1]{\langle #1 |}
\newcommand\qip[2]{\langle #1 | #2 \rangle}

\addtolength{\oddsidemargin}{-.75in}
\addtolength{\evensidemargin}{-.75in}
\addtolength{\textwidth}{1.5in}
\addtolength{\topmargin}{-.75in} 
\addtolength{\textheight}{1.5in}

\bibliographystyle{alpha}
\pagenumbering{arabic} 
\pagestyle{plain}

\begin{document}
\title{Fixed Parameter Tractable Algorithm for Firefighting Problem}
\author{Ming Lam Leung\thanks{Department of Computer Science and Engineering, The Chinese University of Hong Kong, Shatin, Hong Kong SAR, China}}
\maketitle



\abstract{The firefighter problem is defined as below. A fire initially breaks out at a vertex r on a graph G. In each step, a firefighter chooses to protect one vertex, which is not yet burnt. And the fire spreads out to its unprotected neighboring vertices afterwards. The objective of the problem is to choose a sequence of vertices to protect, in order to save maximum number of vertices from the fire.\\

In this paper, we will introduce a parameter k into the firefighter problem and give several FPT algorithms using a random separation technique of Cai, Chan and Chan. We will prove firefighter problem is FPT on general graph if we take total number of vertices burnt to be a parameter. If we parameterize the number of protected vertices, we discover several FPT algorithms of the firefighter problem on degree bounded graph and unicyclic graph. Furthermore, we also study the firefighter problem on weighted and valued graph, and the problem with multiple fire sources on degree-bounded graph.}


\section{Introduction}
The {\bf firefighter problem} is a discrete-time game on a graph G defined as below. Initially, a fire breaks out at a vertex r of G. For each time step, a firefighter choose one vertex not yet on fire to protect it (that vertex remains protected thereafter), and then the fire spreads from the vertices on fire to all of their unprotected neighbours. This process continues round by round. The game ends when the fire cannot spread anymore, i.e. all of the neighbours of the burnt vertices are already protected. Then the vertices not on fire are considered saved. The objective of the problem is to choose a sequence of vertices to protect, in order to save maximum number of vertices in the graph from the fire.\\

The firefighter problem is introduced by Hartnell \cite{Har95} in 1995 and can be used to model the spread of fire, diseases, computer viruses in a marco-control level. The firefighter problem is shown to be NP-complete even for trees of maximum degree 3 by Finbow et al. \cite{FKMR07} Actually, Hartnell and Li \cite{HL9} proved that a simple greedy method for trees is a 0.5-approximation algorithm, and MacGillivray and Wang \cite{MW03} have solved the problem in polynomial time in some subclass of trees. Recently, Cai, Verbin and Yang \cite{CVY08} have obtained a (1-1/)-approximation algorithm for trees based on a linear programming relaxation and randomized rounding, and they have also considered fixed parameter tractability of the problem on trees. Besides, Cai and Wang \cite{C09} also study the defending ability of a graph as a whole by considering the average percentage of vertices the firefighter can save. Furthermore, various aspects of the problem for d-dimensional grids have been considered by Develin and Hartke \cite{DH07}, Fogarty \cite{Fog03}, Wang and Moeller \cite{WM02}, and Cai and Wang \cite{C09}, among others.\\

In this report, we consider fixed parameter tractability of the firefighter problem on various graphs other than trees or  d-dimensional-grid. We establish several FPT algorithms of the problem, either use number of vertices burnt or number of vertices protected to be parameter. Our main results are:

\begin{enumerate}
\item Parameter k on number of vertices burnt
\begin{enumerate}
\item Firefighter problem on general graph is FPT
\end{enumerate}
\item Parameter k on number of vertices protected (i.e. k = no. of rounds)
\begin{enumerate}
\item Firefighter problem on degree-bounded graph is FPT

\item  Firefighter problem on unicyclic graph is FPT

\end{enumerate}
\item Reduction of Firefighter problem on weighted and valued graph
\item Firefighter problem with multiple fire sources, multiple protection and multiple burning can be reduced to the original firefighter problem
\end{enumerate}

The main tool we use is the  method of Cai, Chan and Chan \cite{CCC06}.  In the rest of the paper, we fix our notations and give definitions in Section 2. We then show that the parameterized firefighter problem with number of vertices burnt as parameter is FPT. We then establish FPT algorithms on degree bounded graph (Section 4) and unicyclic graph (Section 5). We also study the possibility of generalizing the algorithm on unicyclic graph to the family of tree+b edges. In Section 6, we will discuss the possibility of finding a solution in the general firefighter problem on the weighted and unequal valued graph, using local replacement. Finally, we will explore further in the firefighter problem with multiple fire sources, multiple protection and multiple burning in Section 7.



\section{Definitions and Notation}
Here we define some terms. Let  be a undirected graph with a source vertex , which is the origin of the fire. A vertex is  once it is on fire, and  once it is protected by the firefighter, and  if it is not burnt at the end of the game. We assume the game ends at time step , which means that the fire can no longer spread to any unburnt vertices after the firefighter protected t vertices. A  for the  problem is a sequence  of protected vertices of  such that vertex  is protected at time , where . An  of a  problem is a  maximizing the total number of vertices saved on the graph .\\

We let  and  be the vertex set and edge set of graph , and we let  and  be the number of vertices and edges of the graph  respectively, i.e. . For any set of vertices , We denote  to be the set of neighbours of . We also define .\\

Define  be the set of edges in the induced subgraph containing vertices , and  be the set of cross edges linking the vertex set  and . Define .\\

We define  to be a graph  constructed by merging the vertices in set  in  into one single vertex. For example, \{u,v\} is a graph constructed by merging the vertices  and  into one vertex.\\

A graph  is said to be  if the degree of each vertex on  is bounded by a fixed constant . 
A graph  is a  graph if the graph contains only 1 cycle, i.e. it has n vertices and n edges. Note that a  graph can also be seen as a tree plus one extra edge.\\

In this paper, we mainly consider FPT algorithms for the following three versions of the  problem. 
\begin{enumerate}
\item \textbf{At Most  Vertices Burnt}: At most  vertices are burnt at the end of the game. We ask if there is a strategy saving at least  vertices.

\item \textbf{Exactly  Vertices Burnt}: The parameter  is the number of vertices burnt at the end of the game. We ask if there is a strategy saving at least  vertices.

\item \textbf{Maximum -Vertex Protection}: The parameter  is the upper bound of number of protected vertices. We need to find a strategy to protect at most  vertices in order to maximize the number of vertices saved.

The main tool we use is the  method of Cai, Chan and Chan \cite{CCC06}. This technique produces randomized algorithms in FPT time. We can derandomize the algorithms using  \cite{NSS95}. Naor et al. give a construction of a (n,t)-universal set of cardinality  in time . A (,)- set is a set of binary vectors of length  while all the  configurations appear in the set for every subset of size t of the indices.\\

\end{enumerate}
We are going to construct a FPT algorithm to solve the problems \textbf{At Most  Vertices Burnt} and \textbf{Exactly  Vertices Burnt} on general graph in Section 3. And then we will focus on the problem \textbf{Maximum -Vertex Protection} in Section 4 and 5, which is proven to be W[1]-hard in general graph. We will introduce FPT algorithm to solve it on degree bounded graph in Section 4, and show that it is FPT on unicyclic graph in Section 5. Besides, we study the reduction of the problem on weighted and valued graph and discuss about the case with multiple fire sources, multiple protection and multiple burning in Section 6 and 7.\\

The complexity of all FPT algorithms introduced in this report are summarized in Table 1.\\

\begin{table}[ht]
\caption{Summary of FPT algorithms} \centering \begin{tabular}{c c c c} \hline\hline Problem & Randomised & Deterministic  &  \\ [0.5ex] \hline \textbf{At Most  Vertices Burnt} &   &  \\ \textbf{Exactly  Vertices Burnt} &   &  \\
\textbf{Maximum -Vertex Protection} \\
Degree-Bounded Graph & O() & --- \\
Unicyclic Graph &  &  \\
Tree+b edges&  &  \\
\hline \end{tabular}
\label{table:nonlin} \end{table}



\section{At Most k Vertices Burnt on general graph}
In this section, we are considering the \textbf{At Most k Vertices Burnt} problem on general graph. We call a strategy S satisfying if it saves at least  vertices, i.e. the total number of vertices burnt is at most . The goal of the problem is to find a satisfying strategy if one exists.\\

The algorithm first colors each vertex of G randomly and independently by either green or red with probability . We call a coloring of G a  if there exists a satisfying strategy  such that all vertices in  are green, and all the burnt vertices are red.\\

\begin{center}
\includegraphics{firefighter1.jpg}\\
\underline{Figure 1: A good coloring example of \textbf{At Most k Vertices Burnt}}
\end{center}

Given a good coloring of , we can find the specific satisfying strategy  as the following. First, starting from the source s, do breath-first-search only on the red vertices to get a red subtree  rooted at s. It takes linear time on number of vertices and edges of the subgraph . After the BFS, all of the neighbours of  have to be green, and  must be a satisfying strategy if the coloring is good. If , we check if the green vertex set  is a valid strategy set for the problem using the procedure below:

\begin{center}
\fbox{
\begin{minipage}[l1pt]{5.50in}
{\bf  Procedure for verifying if  is a satisfying strategy set} \\

{\bf Input}: graph G, source s, vertex set  to be verified, and burnt vertex set \\
{\bf Output}: If true, output the sequence of satisfying strategy in correct order. If false, output "no".
\begin{enumerate}
	\item Do BFS on the subgraph  starting from s, find the distance  of the each vertices  on  from the source  on the subgraph .  
	\item Sort the vertices  according to  in ascending order. Store the sorted sequence in an array  
	\item for i = 1 to  do\\
                      	 If the distance from the source of the i-th vertex , then the solution is not valid, output "no" and halt.
            \item Output the sequence  as the corresponding firefighting strategy in correct order.
    
\end{enumerate}
\end{minipage}
}
\end{center}

Notice that  in the context here. Therefore, the subgraph  in step 1, is nothing but the induced subgraph  with the edge set  removed.

The above procedure is correct based on the following fact.

\begin{Fact} If  is a satisfying strategy in correct order, the distance between the source  and the i-th protected vertices  on the subgraph  has to be at least .
\end{Fact}

Fact 1 is obviously true as the fire will spread out by one unit in the subgraph , after the firefighter protect one vertex at each time step. If the distance between  and the i-th protected vertices  on the subgraph  is less than , the vertex  will be burnt by the fire before time step .\\

Before calculating the complexity of the above algorithm, we have the following observation.

\begin{Obs}
If the number of the vertices burnt is upper bounded by k, there exists a satisfying strategy  with at most k protected vertices. i.e. .
\end{Obs}

The above statement is true because the number of protected vertices is equal to the total time step t. Moreover, each time step at least one vertex is burnt if the game is not ended. Therefore, . \\

Since the size of  and  are both bounded by k, we have a good coloring with  colored red and  colored green with probability at least . Since the BFS algorithm and distance finding is of linear time, and the sorting procedure in step 2 of the verifying procedure takes only O() time, therefore the \textbf{Procedure for verifying if  is a satisfying strategy set} takes linear time. After all, there exists a randomized FPT algorithm solving the problem in O() time.\\ 

Moreover, we can use  universal set to derandomize it and get a deterministic FPT algorithm in O() time. This result matches the previous result of the model \textbf{Saving All But  Vertices} in the special case of firefighter on tree \cite{CVY08}, which gives us a result of  time as  for trees. \\

Furthermore, we can modify the above algorithm a little bit to solve the \textbf{Exactly  Vertices Burnt} problem with the same time complexity. In the above algorithm, we run the \textbf{Procedure for verifying if  is a satisfying strategy set} if . If we modify the algorithm as we run the procedure only if , then we can solve the \textbf{Exactly  Vertices Burnt} problem on general graph with the same time complexity.



\section{Maximum -Vertex Protection on degree bounded graphs}

In this section, we are considering the \textbf{Maximum -Vertex Protection} problem on degree bounded graphs. For the \textbf{Maximum -Vertex Protection} problem, we call a strategy  satisfying if the strategy  contains at most  vertices, and it saves maximum vertices on the graph  from the fire. Note that the fire can no longer spread at time step t+1 after the firefighter protects the last vertex , therefore \textbf{Maximum -Vertex Protection} can be also regarded as a parameterized version of the firefighter problem when we bound the total number of time step . \\

First, we define a type of subgraph called . Given the firefighter problem on a graph  with a strategy  with a sequence of protected vertices , interconnect the source  and vertex set  by a tree  of shortest , where the  is defined as the sum over the pairwise distances between the source  and each vertex  on the subgraph , with constraint that all vertices  have to be a leaf of the tree . Then  is known as the  of the strategy . If the source  and the vertex set  are given, the  of the strategy  can be found in linear time by doing Breath-First Search starting from the source , until all the vertices  are reached.\\

A  of a strategy  on the graph  is the  with minimum number of vertices. Therefore, on a  of a strategy , all the unnecessary edges and vertices which helps nothing on connecting the source  with  have to be removed from the tree. From this definition, we can see that a  of a strategy  is consist of either  or  leaves. Given that the number of protected vertices of an optimal strategy  is at most , we have the following lemma:

\begin{Lem}
In the firefighter problem with an optimal strategy  of at most k vertices, the number of vertices of the  of  is at most .
\end{Lem}

\begin{proof}
Since  is an optimal strategy with at most k vertices, for each vertex , the distance from the source  to  on the subgraph  is at most . Otherwise the fire cannot reach  before time step +1, so the firefighter does not need to protect  before time step k+1. Then the firefighter can earn one more turn to protect an extra vertex  on the -th round, and protect  on the -th round, which contradicts with the assumption that  is an optimal strategy.\\

Since a  of  is a tree minimizing the , which is the sum over all pairwise distances between the source  and each vertex  on the subgraph . As the distance from the source  to  on the subgraph  is at most , the  is at most . And the number of edges of the  is upper bounded by the , so the number of vertices of it is at most .
\end{proof}

Using Lemma 1, we can design a randomized FPT algorithm for the \textbf{Maximum -Vertex Protection} on degree bounded graph using random separation. The main idea of the algorithm is to randomly separate the graph G into 3 parts, one part contains the optimal strategy , one part contains the source  and all the internal nodes of the  of , the third part contains the neighbours of ( of ). \\

The algorithm first colors the source  red, then colors each vertex of G randomly and independently by either green, red or yellow with probability . We call a coloring of G a  if there exists a maximum k-vertex optimal strategy  such that all vertices in  are green, and the tree  of  are red and the neighbours of the tree  are yellow. Notice that the number of vertices in  is at most k. By Lemma 1, we also know that the number of vertices in  is at most . If the graph  is a degree bounded graph with degree at most , then . By random separation, we can have a  with probability at least . \\

\begin{center}
\includegraphics{firefighter2.jpg}\\
\underline{Figure 2: A good coloring example on degree bounded graph}
\end{center}

Given a good coloring of , we can find the specific satisfying strategy  as the following. First, starting from the source , do breath-first-search only on the red vertices to get a red subgraph  rooted at s. Since  is a red tree containing the source , and all the neighbour of this red tree  are either yellow or green, but not red. Therefore, doing BFS from  can help us to locate the tree , i.e. . It takes linear time on number of vertices and edges of the subgraph . After the BFS, all of the green neighbours of  must be a satisfying strategy if the coloring is good, as all the non-solution neighbours of the red tree  are all colored yellow. Let  be the green neighbours of T'. If , we check if the green vertex set  is a valid strategy set for the problem using the {\bf  Procedure for verifying if  is a satisfying strategy set} introduced in Section 3, with the input burnt vertex set . If the coloring is good, this procedure will output  in correct order as the optimal firefighting strategy . \\

Similar to the time complexity analysis in Section 3, we can easily show the complexity of the above randomized algorithm is O().



\section{Maximum -Vertex Protection on unicyclic graphs}

In this Section, we are considering the \textbf{Maximum -Vertex Protection} problem on  . A  is a graph containing only 1 cycle. A  can also be seen as a tree plus one extra edge. Given a unicyclic graph  with a fire source , let n be the number of vertices and edges of . We would like to locate the optimal strategy  with at most k vertices which saves maximum number of vertices.\\

In order to solve the problem on unicyclic graphs, we need to refer to a previous result of \textbf{Maximum -Vertex Protection} problem on trees, given by Cai, Verbin and Yang\cite{CVY08}. In their paper, they introduce a FPT algorithm using  to solve the \textbf{Maximum -Vertex Protection} firefighter problem on trees. Their algorithm randomly and independently colors each vertex into either green or red. For a , all vertices in the satisfying strategy  are green and all of their ascendants of vertices in  on the tree are red. Given a , they can use greedy method to locate the k solution one-by-one in correct order by considering each level of the tree. They proved that this randomized algorithm runs in time . And it can be derandomized using  introduced by Verbin \cite{Ver}, which helps us to get a deterministic algorithm that runs in time .\\

We are going to use their result to show the \textbf{Maximum -Vertex Protection} firefighter problem on unicyclic graphs are also FPT. The main idea of our algorithm is to separate the original problem into three cases, and handle them one by one. In each case, by observing the sequence of vertices being burnt, we find to transform the unicyclic graph  to a tree  by removing vertices and edges on the cycle. Finally we can use the FPT algorithm given by Cai, Verbin and Yang to solve the firefighter problem on tree  and get the optimal solution of the graph .\\

In the beginning, we need to find the cycle  by doing Depth-First-Search from the source . It takes only linear time. Then we can locate the vertex set  of the cycle and the minimum distance from  to cycle . Let  be the minimum distance from  to cycle . Generally, the fire source does not have to locate on the cycle, i.e. .\\

 Let  be the vertex on the cycle  which is nearest to the source . Notice that  is unique. If , then the fire source  locate on the cycle and . Define path P to be the path connecting the source  to , where path P contains  edges and  vertices, including  and . Let  contains r+1 vertices, where . (If  or , then  is not a simple graph. It contains either a loop or multiple edges.) We name the sequence of vertices on cycle  to be , starting with  with clockwise direction.\\

Among all the vertices on the cycle ,  is the one closest to the fire. If  is saved, all the  can be saved as  block the only route that the fire can enter cycle . It means that the fire can enter cycle C only if  is burnt. Therefore, we try to use two FPT algorithms to consider the problem in the 3 cases below separately, and then combine them to get the optimal solution.



\begin{Case}
 and the whole cycle  will be saved at the end of the game.
\end{Case}

In case 1, we assume  and the whole cycle  will be saved at the end no matter what happens. To satisfy the criteria of this case, we modify graph G into a graph , by replacing the whole cycle  by a single vertex , linking to  individual vertices by  edges. Therefore, if G contains n vertices and n edges,  is a tree containing  vertices and  edges. Then we have to use the following fact:

\begin{Fact}
The optimal solution  of \textbf{Maximum -Vertex Protection} firefighter problem on   , with constraint that  has to be saved at the end of the game, is equivalent to the optimal solution  of \textbf{Maximum -Vertex Protection} firefighter problem on tree .
\end{Fact}

Fact 2 is obviously true as  in  is so important that a firefighter must keep it away from the fire, otherwises at least  vertices will be burnt. Therefore the 2 situations is totally equivalent. By solving the \textbf{Maximum -Vertex Protection} on the tree  using Cai, Verbin and Yang's FPT algorithm, we can get the solution  of the problem on unicyclic graph  based on the requirement of case 1.

\begin{center}
\includegraphics{firefighter3.jpg}\\
\underline{Figure 3: Whole cycle C will be burnt}
\end{center}

\begin{Case}
 and the whole cycle  will be burnt at the end of the game.
\end{Case}

 In case 2, we assume the whole cycle  will be burnt at the end, therefore the firefighter should not protect any vertices on path  and cycle . Since the fire will reach cycle  starting from , and none of the vertices on cycle  get protected, so we have the following fact:

\begin{Fact}
If the fire burnt all the vertices on cycle , the last vertex being burnt on cycle C, is either  or .
\end{Fact}

Let  be the edge on the cycle  connecting the 2 vertices  and . When the fire pass along edge , the whole cycle  is completely burnt already. Therefore, we can first transform G to a tree  by deleting edge . Then we use random separation method to solve the \textbf{Maximum -Vertex Protection} firefighter problem on the tree . To guarantee the whole cycle  will be burnt at the end, we colors all the vertices on the path  and cycle  red, before randomly coloring all the other vertices in green or red. Then the optimal solution  of the problem on the tree  is exactly the optimal solution  of the problem on the graph  based on the requirement of case 2.



\begin{Case}
 will be burnt but at least one vertex on the cycle  will be saved at the end of the game.
\end{Case}

This is the most complicated case to consider among all the cases. As  will be burnt, which means the whole path P will be burnt, and the fire can enter cycle  and burn some of the vertices on . However, the criteria "at least one vertex on the cycle  will be saved" tells us that there exists at least one protected vertex on the cycle , unless no other protected vertices can save vertices on  from the fire.\\

The main idea to solve this case is to exhaust all the possible combinations of protected vertices located on cycle  in the optimal solution , then we can figure out which vertices on cycle C will be saved, and delete them from the graph to get a tree. To ensure the algorithm runs in FPT time, we need the following important fact:

\begin{Fact}
Let  be the optimal strategy of the problem on the unicyclic graph  with a cycle , then the number of vertices in  is at most 2.
\end{Fact}

\begin{proof}
If  is saved, we only need to protect one vertex on the path P in order to save the whole cycle , therefore we never need to protect any vertices on  except for , then at most one protected vertex on  would be enough.\\

Consider the case that  is burnt. Let  be the number of vertices in  located on cycle . We call those vertices . Assume . We can always find two vertices in , say  and , cut the cycle  into two half. Let's call these two half  and , with  contain the vertex . \\

Consider another vertex , if  is on , then  is already saved by the 2 protected vertices  and  as they block the only two route that the fire can enter . It is no point for the firefighter to include  as a protected vertex in the optimal strategy. If  is on , then either  and  save , or  and  save  from the fire, either  or  becomes pointless and can be removed from the optimal strategy . i.e. the set  should contain at most 2 vertices.
\end{proof}

Fact 4 tells us that only 2 protected vertices on the cycle  is sufficient for us to construct an optimal strategy . Besides, by the requirement of case 3, at least one vertex on the cycle  has to be protected. Exhaust for all pairs of vertices  and  on the cycle , where  can be equal to . We assume  and  are the only protected vertices in  on the cycle , and they cut the cycle  into two half  and , with  located on the half . We can easily see that the vertices on  will be all burnt and the vertices on  will be all saved. \\

For each pair of ,transform the graph  to a tree  by removing all the vertices and edge on , and also all the subtrees originally rooted at the vertices on . Then we can use random separation to solve the problem on the graph , by pre-coloring the paths  and  to red, and  to blue. Then the optimal strategy  of the problem on  is exactly the optimal strategy of the problem on graph , condition on the criteria of case 3 and both  and  are the element of .\\
 
\begin{center}
\includegraphics{firefighter4.jpg}\\
\underline{Figure 4: Exhaustion for vertex  and }
\end{center}

In the exhaustion in case 3 , the choices of  and  are both bounded by the size of the cycle. Therefore we need to use Cai, Verbin and Yang's FPT algorithm O() times in each choice of . As all the transformations from a graph  to a tree  mentioned above take only linear time. Combining the algorithms in all 3 cases, finally we construct a FPT algorithm runs in time . And it can be derandomized using  to get a deterministic algorithm that runs in time .\\

Actually, the algorithm above can be generalized to solve the firefighter problem in the family of graph , for any small constant . In unicyclic graph, , and the randomized FPT algorithm runs in time . Generally, one can use a similar approach to construct an algorithm solving the firefighter problem on  in time . Unfortunately, it is not FPT time if both  and  are parameters of problem, as the complexity depends on , but the algorithm is also efficient when  is a small fixed constant.\\

Here we omit the details on constructing algorithm of the \textbf{Maximum -Vertex Protection} firefighter problem on  in  time. The main idea is to modify the exhaustion in case 3, from 2 vertices  on cycle  to  vertices  on the cyclic subgraph based on the fact below:

\begin{Fact}
Let  be the optimal strategy of the problem on the  graph , there exists a subgraph  where  is a forest, and the number of vertices in  is at most 2b.
\end{Fact}

The above fact can be deduced by fact 4 by considering the  smallest cycles induced by the extra  edges.\\

As Cai, Verbin and Yang show in their paper that the \textbf{Saving k Vertices} firefighter problem on trees has a polynomial kernel of size  \cite{CVY08}, one may be curious if the problem on  graph also has a polynomial kernel or not. If there exists a polynomial kernel, we can reduce the complexity of the above algorithm to FPT time even for large b. However, since both the \textbf{Saving k Vertices} and \textbf{Maximum -Vertex Protection} firefighter problem on general were proven to be W[1]-hard, and any general graph can be consider as a  graph if  is not bounded, therefore it seems very uneasy to find a polynomial kernel in this situation.



\section{Weighted and unequal valued graph}
In the previous discussion, we only consider the firefighter problem on unweighted graph, i.e. all the edges have equal distance, therefore the fire spreads out uniformly to the unprotected neighbours of the burnt vertices at each time step. Besides, all the vertices are assumed to be equally valuable, and all the edges are set to have no values. In this section, we will focus on the firefighter problem on weighted graph and unequal valued graph.\\

We define the weighted firefighter problem as follow: A fire initially breaks out at a vertex r on a weighted graph G. In each time step , a firefighter chooses to protect one vertex , which is not yet burnt. Then for each edge  with an integral edge weight , where  is already on fire at time  and  is not yet on fire, the fire spreads out from  to  along the edge e and burn  at time . The objective of the problem is to choose a sequence of vertices  to protect, in order to save maximum number of vertices from the fire. The original firefighter problem on unweighted graph is the weighted firefighter problem with all the edge weights equal to 1.\\

For any weighted graph  of maximum weight , we try to transform  to an unweighted graph  by local replacement. For each edge  with edge weight , we contracts these edges and combine each (u,v) pair to be a single vertex. For each edge  with edge weight , we replace it by a path  of  edges by adding  vertices between u and v. If  follows certain properties, for instance,  is degree bounded or unicyclic, then we can try to find the solution of the problem on graph  using random separation by pre-coloring all the newly-added vertices to yellow (or red in unicyclic case). Then we can apply the known FPT algorithm on unweighted graph to the weighted firefighter problem, where the time complexity would be almost the same (with a multiplication factor of ).\\

Now we consider the case that each vertices on the graph  have unequal value, i.e. some vertices are more valuable that we have more motivation to save them from the fire. Also, we assumes the edges contain values too, therefore we are motivated to save valuable edges. A edge  is said to be saved if both vertices u and v are saved, otherwises it is a burnt edge. This model is more realistic in the application of firefighter problem on diseases or computer-viruses control. Here we assign each vertex  a "value" , each edge  a "value" , and the objective of the  problem is to choose a sequence of vertices  to protect, in order to maximize the total value Z of saved vertices and saved edges. The original equally valued firefighter problem is the valued firefighter problem with all the vertex values equal to 1 and all the edge values equal to 0. \\

For any valued graph  of maximum value , we try to transform  to an equally valued graph  by local replacement. For each edge  with edge value , we replace it by a path  of  "value 0" edges by adding  "value 1" vertices between u and v. For each vertex  with value , we replace it by a path  of  "value 1" vertices by adding  "value 0" edges. After all, we can also try to find the solution of the problem on graph  using random separation by pre-coloring all the newly-added vertices to yellow (or red in unicyclic case), which runs in the same time complexity of the originally problem (with a multiplication factor of ), similar to what we mention above in the weighted graph case.\\

Notice that the above algorithm does not work if there exists vertex with value 0. Actually, in the special case of firefighter problem on trees, the well-known \textbf{Saving k Leaves} problem is special case of firefighter problem on valued graph where internal vertices have value 0. And the parametric dual \textbf{Saving all But k Leaves} is NP-complete even for k=0, shown by Finbow et. al. \cite{FKMR07}, so there is no FPT algorithm for the problem \textbf{Saving all But k Leaves} unless P=NP. In Section 3 of this report, we show that \textbf{k Vertices Burnt} of equally valued firefighter problem on general graph is FPT. If there exists a reduction method to reduce any valued graph  with 0-value vertices to an equally valued graph, then we show \textbf{Saving all But k Leaves} is FPT and prove P=NP. Therefore, it is not surprising that the above algorithm cannot reduce a valued graph with 0-value vertices to equal valued graph.



\section{Multiple fire sources, multiple protection and multiple burning}
In this final section, we are considering the firefighter problem with multiple fire sources and multiple firefighters. Multiple fire sources means that there are more than one sources (say g sources) on fire initially. Multiple protection means that the firefighter can protect more than one vertices, say p vertices, before the fire spreads out in each time step. Multiple protection means that the fire spreads out by several units length, say h units, after firefighter's action in each time step.\\



The firefighter problem with multiple fire sources on general graph is nothing but just an ordinary firefighter problem by combining all the g fire sources into one single source. Let  be the g sources on the graph ,  we define . The result of \textbf{At Most k Vertices Burnt} problem on general graph  in Section 3 is the same in the result of the problem on graph . For the \textbf{Maximum -Vertex Protection} problem on degree bounded graph, we can also combine all the g fire sources into one single source. The only difference is the maximum degree of the source in G' is no longer , but becomes . Therefore the complexity of the randomized algorithm is O().\\

\begin{center}
\includegraphics[width=90mm]{firefighter5.jpg}\\
\underline{Figure 5: Transformation from 2 fire sources on tree to 1 fire source on unicyclic graph}
\end{center}

Similarly, by combining all the g fire sources into one single source, the firefighter problem with multiple fire sources on  or  can be transform to the ordinary firefighter problem on  and  graph respectively. For example, consider the firefighter problem with 2 fire sources on a tree T,  by combining the two fire sources  and  into a single vertex, the tree T is transform to be an unicyclic graph . Then we can use the FPT algorithm mentioned in Section 5 to solve it in  time. If  is a small fixed constant, the firefighter problem with multiple fire sources on   can be transformed to the firefighter problem with single fire source on , and there exists an algorithm runs in  time.\\



The firefighter problem with multiple protection and multiple burning is also very simple. By changing the parameter  to be total number of rounds, we define a firefighter problem called  \textbf{Maximum -Step Protection} with multiple protection and multiple burning as follow: \\

We want to find an optimal strategy  = , where  is the total number of time steps of the game and  is the number of vertices a firefighter can protect in each time step. This optimal strategy  can save maximum number of the vertices from the fire, where the fire burns  layers of neighbours of the vertices on fire in each time step.\\

Without multiple protection and multiple burning, the problem \textbf{Maximum -Step Protection} can be reduced to  \textbf{Maximum -Vertex Protection}. Otherwise, we can solve this problem in FPT time using the similar strategy as above. For example, considering the \textbf{Maximum -Step Protection} firefighter problem with multiple protection and multiple burning on degree bounded graphs, we can treat the problem as an ordinary \textbf{Maximum -Vertex Protection} firefighter problem. Using the algorithm provided on Section 4, with  strategy vertices green,  BFS Burning Tree vertices red, and  neighbouring vertices yellow as good coloring. Then we can do a BFS from the source , and check if all the green leaves satisfy the criteria of optimal strategy or not. Notice that the {\bf  Procedure for verifying if  is a satisfying strategy set} given in Section 3 also has to be modified a little bit as below:\\

\begin{center}
\fbox{
\begin{minipage}[l1pt]{5.50in}
{\bf  Modified Procedure for verifying if  is a satisfying strategy set for multiple protection and burning} \\

{\bf Input}: graph G, source s, integer p and h, vertex set  to be verified, and burnt vertex set \\
{\bf Output}: If true, output the sequence of satisfying strategy in correct order. If false, output "no".
\begin{enumerate}
	\item Do BFS on the subgraph  starting from s, find the distance  of the each vertices  on  from the source  on the subgraph .  
	\item Sort the vertices  according to  in ascending order. Store the sorted sequence in an array  
	\item for i = 1 to  do\\
                      	 If the distance from the source of the i-th vertex {\bf satisfies the relation }, then the solution is not valid, output "no" and halt.
            \item Output the sequence  as the corresponding firefighting strategy in correct order.
    
\end{enumerate}
\end{minipage}
}
\end{center}

Therefore, we can solve the \textbf{Maximum -Step Protection} firefighter problem with multiple protection and multiple burning in the degree bounded graph in FPT time. Similarly, by modifying the algorithm in Section 5, one can also construct an FPT algorithm for the \textbf{Maximum -Step Protection} in the unicyclic graph.

\section{Acknowledgments}
I gratefully acknowledge the support of Computer Science and Engineering Department, the Chinese University of Hong Kong. I have to thank Professor Leizhen Cai, for the innoviative teaching and offering continuing advice and guidance. I am also thankful to my master supervisor, Professor Shengyu Zhang, who have been giving me much support and encouragement on my study and research.



\bibliography{firefighter}
\bibliographystyle{amsplain}

\end{document}
