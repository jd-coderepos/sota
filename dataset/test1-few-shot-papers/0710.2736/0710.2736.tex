\documentclass[11pt]{article}
\usepackage{amsfonts}
\usepackage{amsmath,amssymb}
\usepackage{epsfig}
\pagestyle{empty} \setlength{\parindent}{12pt}
\setlength{\parskip}{3pt plus1pt minus2pt}
\setlength{\baselineskip}{20pt plus2pt minus1pt}
\setlength{\textheight}{24.5true cm} \setlength{\textwidth}{16.8cm}
\setlength{\topmargin}{-20mm} \setlength{\columnsep}{8mm}
\setlength{\evensidemargin}{0.5mm} \setlength{\oddsidemargin}{0.5mm}
\def\baselinestretch{1.4}
\raggedbottom \headheight 0pt \columnsep=0.8cm \raggedbottom
\textwidth=16.5cm \textheight=24.5cm \topmargin=-0.7cm
\evensidemargin=0.5cm \oddsidemargin=-0.2cm
\renewcommand \baselinestretch{1.25}
\pagestyle{plain}

\renewcommand{\thefootnote}{\fnsymbol{footnote}}
\def\dref#1{(\ref{#1})}

\newlength{\gH}
\setlength{\gH}{50ex}
\newlength{\gV}
\setlength{\gV}{40ex}

\begin{document}

\begin{center}
{{\Large \bf  norm performance index of synchronization and
optimal control synthesis of complex networks}}
 \footnote{{\footnotesize This work is supported by both the National Science
 Foundation of China under grants 60674093, 60334030 and

\quad the City University of Hong Kong under the Research
Enhancement Scheme and SRG grant 7002134. }}
\end{center}
\vskip 0.3cm
\begin{center}
  Chao Liu  \footnote{
{\footnotesize Corresponding author: chaoliu@pku.edu.cn}},
    \quad Zhisheng Duan ,
    \quad Guanrong Chen ,
    \quad Lin Huang 
\end{center}
\begin{center}
{\small \it   State Key Laboratory for Turbulence and
Complex Systems, Department of Mechanics and \\Aerospace
Engineering, College of Engineering, Peking University, Beijing
100871, P. R. China
\\  Department of Electronic Engineering, City University of Hong Kong, Hong Kong, SAR, P. R. China}
\end{center}

\vspace*{1\baselineskip}
\begin{center}
{\begin{minipage}{127mm} {\small {\bf Abstract.} In this paper, the
synchronizability problem of dynamical networks is addressed, where
better synchronizability means that the network synchronizes faster
with lower-overshoot. The  norm of the error vector  is
taken as a performance index to measure this kind of
synchronizability. For the equilibrium synchronization case, it is
shown that there is a close relationship between the  norm of
the error vector  and the  norm of the transfer function 
of the linearized network about the equilibrium point. Consequently,
the effect of the network coupling topology on the  norm of the
transfer function  is analyzed. Finally, an optimal controller is
designed, according to the so-called \textit{LQR} problem in modern
control theory, which can drive the whole network to its equilibrium
point and meanwhile minimize the  norm of the output of the
linearized network. }

\vspace*{0.5\baselineskip} {{\bf Keywords.} network,
 synchronizability,  norm,  norm, \textit{LQR} problem, optimal control.}

\end{minipage}
}
\end{center}

\abovedisplayskip=0.13cm \abovedisplayshortskip=0.06cm
\belowdisplayskip=0.13cm \belowdisplayshortskip=0.08cm

\def\dis{\displaystyle}




\section{Introduction}

\quad The topic of synchronizability of dynamical networks has
attracted increasing interest recently (see
\cite{bern05,hong04,nish03,wu03,wu05,wu06,zhou06} and references
therein), which is mostly referred to as how easy it is for the
network to synchronize. Technically, this is a problem whether there
is a wide range of coupling strength in which the synchronization is
stable, and the wider the range, the better the synchronizability of
the network. Accordingly, for some cases when the synchronized
region is unbounded, a better synchronizability means that the
synchronization can be achieved with a smaller coupling strength
\cite{wu03}. Generally, the synchronizability of a network depends
on the underlying coupling topology of the network. Studies show
that the ability for a network to synchronize is related to the
spectral properties of the outer coupling matrix of the network
\cite{wu03,wu05} and thus influenced by the structural properties of
the network, such as average distance \cite{zhou06}, degree
homogeneity \cite{nish03}, clustering coefficient \cite{wu06},
degree correlation \cite{bern05}, betweenness centrality
\cite{hong04}, etc.

In this paper, the synchronizability problem of dynamical networks
is considered from a different point of view: suppose that the
coupling strength of a network belongs to the range in which the
synchronization is stable; then, how fast the synchronization will
be achieved? This problem is important for the reason that in
practical engineering implementation (such as communications via
chaotic synchronization), synchronization is expected to be achieved
not only easily but also swiftly. In \cite{nish06}, it is
qualitatively pointed out that networks with diagonalizable outer
coupling matrices may synchronize faster than the ones with
non-diagonalizable outer coupling matrices. Clearly, a measure is
needed for a quantitative description of the swiftness of network
synchronization. To meet this objective, the  norm of the error
vector , denoted as , is taken in this paper as a
performance index of this kind of network synchronizability. As will
be seen later, the quantity  presents a suitable measure of
both swiftness and overshoot (referring to the largest difference
among various node dynamics before the synchronization is achieved):
the smaller the quantity , the faster with smaller
overshoot the network synchronization.

Furthermore, as shown by the numerical examples given below, the
quantity  is influenced by the coupling topology of the
network. Thus, the investigation on the relationship between
 and the network structure is of significance. In this
paper, for the case that the synchronous state is an equilibrium
point, it is pointed out that  is upper-bounded by the
product of the vector 2-norm of the initial error vector  and
the  norm of the transfer function , denoted as
 or simply , of the linearized network about
the equilibrium point. Thus, the smaller the , the smaller
the  as well. As pointed out in \cite{duan07}, the
relationship between  and the network structure is quite
complicated. Under some assumptions, it is proved in this paper (see
Theorem 1 and Example 4) that  will not increase as the
real eigenvalues of the symmetrical outer coupling matrix increase.

For a linear time-invariant system, the linear quadratic regulator
problem, or simply the \textit{LQR} problem, is a classical problem
in modern control theory. The objective of the \textit{LQR} problem
is to find an optimal control law  such that the state 
is driven into a (small) neighborhood of the origin while minimizing
a quadratic performance ( performance) index on  and . In
fact, the \textit{LQR} problem is posed traditionally as the
minimization problem of the  norm of the regulator output of
the system. In this paper, based on the techniques of the
\textit{LQR} problem, an optimal controller design is developed so
as to drive the network dynamics onto some homogenous stationary
states while minimizing the  norm of the output of the
linearized network.

The rest of the paper is organized as follows. In Section 2, some
preliminary definitions and lemmas necessary for successive
development are presented. In Section 3, some numerical examples are
provided to illustrate that the quantity  presents a
suitable measure of both swiftness and overshoot of the network
synchronization. For the equilibrium synchronization case, the
relationship between  and the network structure is
investigated in Section 4. Based on the results of the \textit{LQR}
problem, an \textit{LQR} optimal controller is proposed in Section
5. The paper is concluded by the last Section.


\section{Preliminaries}

\quad  is an infinite-dimensional Hilbert space, which
consists of all square-integrable and Lebesgue measurable functions
defined on an interval  with the scalar inner product

while if the functions are vector or matrix-valued, the inner
product is defined as

where  denotes complex conjugate transpose, and the induced
norm is defined as

for .


Consider a continuous-time linear system,

where ,  and  are the state, input and output of the
system, respectively, and ,  and  are given
constant matrices. The transfer function from  to  is
. If  is stable, then the  norm of
system \dref{pre1} is represented by the  norm of the transfer
function , which is defined by

It can be proved that  equals the overall output
energy of the system response to the impulse input. For computing
, the following formula is convenient.


\textbf{Lemma 1} \cite{zhou96}: If  is stable, then
, where matrix  is the
solution to the following Lyapunov equation:

Equivalently,


The so-called linear quadratic regulator (\textit{LQR}) problem is
an optimal control problem with a quadratic performance ( norm)
criterion. For the linear time-invariant system

where  is arbitrarily given, the regulator problem refers to
finding a control function  defined on , which can be
a function of the state , such that the state  is driven
into a (small) neighborhood of origin at time , . Since
every physical system has energy limitation, and large control
action (even if it is realizable) can easily drive the system out of
its valid region, certain limitations have to be imposed on the
control in practical engineering implementation. For these reasons,
the regulator problem is usually posed as an optimal control problem
with a certain combined performance index on  and . Focusing
on the infinite time regulator problem (i.e., )
and without loss of generality assuming , the \textit{LQR}
problem is formulated as follows: Find a control  defined on
 such that the state  is driven to the origin at
 and the following performance index is
minimized:

for some  and . Here,  emphasizes that the
control energy has to be finite, i.e., .
Moreover, it is assumed that

Then, \dref{pre5} can be factored as

and \dref{pre4} can be written as

Traditionally, the \textit{LQR} problem is posed as the following
minimization problem:



For the above \textit{LQR} problem, the following lemma is useful.


\textbf{Lemma 2 } \cite{zhou96}: Suppose that in \dref{pre6}:

(A1)  is stabilizable;

(A2)  has full column rank with  being unitary;

(A3)  is detectable;

(A4)  has full column rank for all .\\
Then, there exists a unique optimal control  for the
\textit{LQR} problem \dref{pre6}, where

and  is the stabilizing solution to the following Riccati
equation:

Moreover, the minimized  norm of the output  is given by

where  is the transfer function of the system

with  being the identity matrix and  the impulse
function.

\section{ norm performance index of network synchronization}

\quad Consider a network of  identical dynamical nodes, described
by

where  governs the dynamics of each individual node, 
is the coupling strength,  is the inner linking matrix, and
 is the outer coupling matrix.

Let

and

denote the error vector of network \dref{net1}. Then, network
\dref{net1} is said to achieve (asymptotical) synchronization if


Let , or simply  when no confusion
may be caused, denote the  norm of  on a given interval
. Then, according to \dref{L2norm},


In fact,  represents the energy of the error signal 
on the interval , which is in proportion to the area between
the error function  and the time axis. Hence,  can be
used as a quantitative measure of the swiftness and overshoot of the
network synchronization. In what follows, two examples are first
given for illustration.

\textbf{Example 1:} Suppose that each single node in network
\dref{net1} is a Chua's oscillator. In the dimensionless form,
Chua's oscillator is described by

where  is a piecewise linear function:


Take parameters ,
and , so that Chua's oscillator \dref{chua1} generates a
double-scroll chaotic attractor.
Set  in \dref{net1}. Fig. 1 shows the
different synchronization performances of network \dref{net1} with
the same coupling strength  () but different
coupling configurations. The corresponding values of 
are computed numerically as given in Table 1.

\begin{center}
\vskip -0.5cm
 \unitlength=1cm
 \qquad \hbox{\hspace*{0.1cm} \epsfxsize5cm \epsfysize5cm
\epsffile{3statechuaN20nearcs6.eps} \;\quad \epsfxsize5cm
\epsfysize5cm \epsffile{3statechuaN20starcs6.eps}\;\quad
\epsfxsize5cm \epsfysize5cm \epsffile{3statechuaN20glocs6.eps}}
\end{center}

\vskip -0.7cm\quad\; {\small Nearest-neighbor coupling
}\qquad\qquad\quad {\small Star-shaped coupling
}\qquad\qquad\qquad\; {\small Global coupling}

\begin{center}
\vskip -0.5cm
 \unitlength=1cm
 \qquad \hbox{\hspace*{0.1cm} \epsfxsize5cm \epsfysize5cm
\epsffile{echuaN20nearcs6.eps} \;\quad \epsfxsize5cm \epsfysize5cm
\epsffile{echuaN20starcs6.eps}\;\quad \epsfxsize5cm \epsfysize5cm
\epsffile{echuaN20glocs6.eps}}
\end{center}

\vskip -0.7cm\quad\; {\small Nearest-neighbor coupling
}\qquad\qquad\quad {\small Star-shaped coupling
}\qquad\qquad\qquad\; {\small Global coupling}

\qquad\quad {\small Fig. 1 \quad Synchronization of the state
variables of network \dref{net1} with 20 nodes and \vskip
-0.2cm\qquad\qquad\qquad\quad  the same coupling strength in
different coupling configurations.}

\begin{center}
\vskip 0.2cm
\begin{tabular}{|c|c|c|c|}\hline
Coupling & Nearest-neighbor & Star-shaped & Global
\\\hline  & 9.2744 & 0.8030 & 0.4142
\\\hline
\end{tabular}
\vskip 0.2cm \quad {\small Table 1 \quad Values of . }
\end{center}

As to the case of equilibrium synchronization of network
\dref{net1}, since the synchronous state  is known, the error
vector can be defined in the following way:


\textbf{Example 2:} Consider a Lur'e system,

where  is the state,  is the measured output,

and the nonlinear function .

A network with system \dref{lure1} as individual nodes is given as
follows:

where ,  and
.

Network \dref{lure2} can be viewed as a large-scale system with
measured output and feedback. Assume that system \dref{lure2} is
observable. Then, the states of network nodes achieve
synchronization if and only if all the outputs of network nodes
achieve synchronization. Thus, one only needs to consider the
outputs of network \dref{lure2}.

Replace  with  and set  in \dref{chua4}. Then,
 and . Let  and
 in \dref{lure1}. Fig. 2 shows the different performances
of outputs of network \dref{lure2} with the same coupling strength
 (=2) but different coupling configurations. The
corresponding values of  are listed in Table 2.

\begin{center}
\vskip -0.5cm
 \unitlength=1cm
 \qquad \hbox{\hspace*{0.1cm} \epsfxsize5cm \epsfysize5cm
\epsffile{ytN20nearlurea10b3cs2.eps} \;\quad \epsfxsize5cm
\epsfysize5cm \epsffile{ytN20starlurea10b3cs2.eps}\;\quad
\epsfxsize5cm \epsfysize5cm \epsffile{ytN20glolurea10b3cs2.eps}}
\end{center}

\vskip -0.7cm\quad\; {\small Nearest-neighbor coupling
}\qquad\qquad\quad {\small Star-shaped coupling
}\qquad\qquad\qquad\; {\small Global coupling}


\qquad\quad {\small Fig. 2 \quad Synchronization of the outputs of
network \dref{lure2} with 20 nodes having \vskip
-0.2cm\qquad\qquad\qquad\quad\;\;  the same coupling strength but
different coupling configurations.}

\begin{center}
\vskip 0.2cm
\begin{tabular}{|c|c|c|c|}\hline
Coupling  & Nearest-neighbor & Star-shaped & Global
\\\hline  & 0.7296 & 0.5768 & 0.3990
\\\hline
\end{tabular}
\vskip 0.2cm \quad {\small Table 2 \quad Values of . }
\end{center}
\hfill 

\textbf{Remark 1:} As indicated by the above examples, in the same
situation of stable synchronization, the performances of
synchronization of networks can be quite different.

\textbf{Remark 2:} It is also clear that swiftness and overshoot are
important indexes for describing the synchronous behaviors. The
 norm of the error vector , i.e.,  as defined in
\dref{net3} or \dref{chua4} for the equilibrium synchronization
case, can properly measure the swiftness and overshoot of the
network synchronization: the smaller the , the faster with
lower overshoot the network synchronization. Thus, the quantity
 can be taken as a performance index of network
synchronizability.

\textbf{Remark 3:} The quantity  is influenced by many
factors of the network, such as

(1) network structure, particularly the coupling strength 
and the outer coupling matrix ;

(2) dynamical components, particularly the individual dynamics
determined by , the synchronous state , and the inner
linking matrix .

\section{Local synchronization to network equilibrium point}

\quad In the literature where the eigenvalues of the outer coupling
matrix are used to measure the network synchronizability, it is a
topic of great interest to investigate the relationship between the
eigenvalues and the network structure thereby finding suitable ways
to enhance the synchronizability. In this section, for the case of
equilibrium synchronization, the relationship between the quantity
 and some network parameters is explored.


\subsection{ as a constraint of }

\quad Suppose that each single node in network \dref{net1} has a
measured output . Then, the equations of network \dref{net1}
can be written as follows:

If , then  is just the state of the th node.

Let  denote an equilibrium point of the individual node,
satisfying . Then, the linearized equations of \dref{lsyn1} about the
synchronous solutions  are as follows:

where  () and  are the state error vector
and the output error vector to the th node, respectively. Viewing
the impulse function  as an input to system
\dref{lsyn2}, system \dref{lsyn2} can be equivalently written as

Using the Kronecker product, the error system \dref{lsyn3} can be
rewritten as

where ,  and .

As in Example 2, suppose that system \dref{lsyn4} is observable.
Then,  can be taken as a measure of the swiftness and
overshoot of the synchronization of network \dref{lsyn1}. Let 
denote the transfer function of the error equation \dref{lsyn4} from
 to . Then


Let the transfer function  be given in \dref{lsyn5}, then
\textbf{Lemma 3:} The inequality

holds, where  is the  norm of the error vector 
on the interval ,  is the  norm of the
transfer function , and  is the vector 2-norm
of the initial error vector .

{\bf Proof:} Since , ,
where  denotes the corresponding bilateral Laplace transform
of . Then, by Parseval's identity,

\hfill 


Thus, the quantity  is upper-bounded by the product
. Since  is the given initial error
vector,  can be taken as a constraint of . In
fact, as introduced in Sec.1,  equals the overall output
energy of the system response to the impulse input.

\textbf{Remark 4:} The advantages of using the quantity 
include:

(1)  can be numerically computed;

(2) the synchronizability is affected by many factors of a network,
while  can be seen as an overall reflection of these
network factors consisting of both structural and dynamical ones.

\textbf{Example 3:} Consider Example 2 again. The linearized
equation of network \dref{lure2} about the equilibrium point  is given as follows:


The corresponding values of  with the three different
network configurations are listed in Table 3.

\begin{center}
\vskip 0.2cm
\begin{tabular}{|c|c|c|c|}\hline
Coupling  & Nearest-neighbor & Star-shaped & Global
\\\hline
 & 2.5091 & 2.4383 & 1.2565
\\\hline
 & 0.7296 & 0.5768 & 0.3990
\\\hline
\end{tabular}
\vskip 0.2cm \quad {\small Table 3 \quad  Values of  of
network \dref{lure2} with different network configurations.}
\end{center}


\subsection{Relationship between  and network structure}

\quad Example 3 shows that  is influenced by the network
configurations. Thus, the relationship between  and the
network structure is a problem deserving further investigation.

In this section, it is always assumed that the outer coupling matrix
 is symmetrical. Then, there exists a unitary matrix  such that , where
 is a
diagonal matrix with the diagonal entries
, being the eigenvalues of matrix .
Let , ,
 and .
Then, the error equation \dref{lsyn4} can be rewritten as follows:

Note that system \dref{analyH1} is composed of  uncoupled
subsystems:

where  is the th component of the input vector
.
Let

be a matrix with the th diagonal entry being 1 and all the other
entries being zero. Then,  and
.

From \dref{analyH2}, the condition for ensuring the stable
synchronous state

is that the  matrices

are all stable.

Let  denote the transfer function of \dref{analyH1} from
 to . Then

Since both  and  are
unitary transformations, one has

Let  denote
the transfer function of the th subsystem in \dref{analyH2}. Then

and


\textbf{Assumption 1:} Suppose that the outer coupling matrix  is
symmetrical, diffusive and irreducible, with the off-diagonal
entries  and the diagonal entries
, for .

The outer coupling matrix  satisfying Assumption 1 has a zero
eigenvalue of multiplicity , and its other eigenvalues are all
positive.

\textbf{Theorem 1:} Suppose that the inner linking matrix
, where  is a constant and  the identity matrix.
Suppose that the outer coupling matrix  satisfies Assumption 1
with the eigenvalues given as follows:

Then

where , and


{\bf Proof:} Let  be an arbitrary positive definite matrix such
that

By the assumption that , one has

Then

The above inequality and Lemma 1 together leads to the assertions of
the theorem. \hfill 

\textbf{Remark 5:} Since ,  in
\dref{analyH7} represents the  norm of the transfer function of
each individual node in the network.

\textbf{Remark 6:} Theorem 1 provides a simple way for comparing
 of a network with its different possible structures. It
can be deduced from Theorem 1 that  will not increase as
the eigenvalues of the outer coupling matrix increase.

\textbf{Example 4:} Consider Example 2 again. The eigenvalues of the
global coupling matrix  are  and
, while the eigenvalues of the
star-shaped coupling matrix  are ,
 and . Let 
and  denote the transfer functions of the global coupling
and the star-shaped coupling networks, respectively. Then, by
Theorem 1,

It is consistent with the numerical results given in Example 3.

For Example 2, through integral computations, an analytical
expression of  can be obtained as follows:

By \dref{analyH8} and \dref{analyH4}, Table 4 gives the different
values of  and  of network
\dref{lure2} as the node number  increases.

\begin{center}
\begin{tabular}{|c|c|c|c|c|c|c|}\hline
N & 1 & 10 & 100 & 1000 &  \\\hline  &
1.0801 & 1.3190 & 1.4492 & 1.4696  & 1.2910
\\\hline  & 1.0801 & 2.2619 & 6.9840 & 22.0434 &  \\\hline
\end{tabular}
\vskip 0.2cm \quad {\small Table 4 \quad Values of  of
network \dref{lure2} with different network configurations.}
\end{center}
\hfill 


\section{Optimal controller design}

\quad So far, the pinning control strategy is extensively used for
 achieving synchronization of dynamical networks
\cite{li04,wang02}. The main advantage of the pinning control
strategy is that only a few network nodes are needed to be
controlled. AS far as the control effects (referring to the
swiftness and overshoot of the synchronization) and the control cost
(represented by the  norm of the control input) are concerned,
however, pinning control may not be the best choice. It is revealed
by the \textit{LQR} problem, as illuminated in \dref{pre6}, when
matrices  and  are properly selected, both the control effects
and the control cost can be simultaneously evaluated by the 
norm of the measured output  of the controlled network. The
smaller the , the better the control effects and the lower
the control cost. In this section, based on the optimal solution to
the \textit{LQR} problem, an \textit{LQR} optimal controller is
designed for network \dref{net1}, which can drive the network onto
some homogenous stationary states while minimizing the quantity
.


Suppose that the controlled network is given as follows:

where , ,
 are given constant matrices,
 is the controller to be designed, and
 is the measured output of the LQR
problem. As in Sec. 4, let  denote an equilibrium point of the
individual node and let  where
 be the state error vector.

Suppose that the controller is a feedback law of the state error,
i.e., , where  is the feedback gain matrix to be
determined. Then, by the Kroneck product, the equation of the
linearized system of \dref{opticon1} about the synchronous solution
 is given as follows:

where ,
,  and .
Furthermore, suppose that the matrices  and 
satisfy the assumptions (A1-A4) as given in Lemma 2. Then, by Lemma
2, the feedback gain  for the \textit{LQR} problem of the
linearized system \dref{opticon2} is obtained as

where  is the stabilizing solution to the following Riccati
equation:


\textbf{Remark 7:} The optimal controller  with  given
in \dref{opticon3} has the property that it controls network
\dref{opticon1} to the synchronous state  and
meanwhile minimizes the  norm of the output  of the
linearized system \dref{opticon2}.

As the node number  of network \dref{opticon1} increases, the
direct computation of the feedback gain  as given in
\dref{opticon3} will become harder. For the case when the outer
matrix  is symmetrical, the feedback gain  can be
alternatively given in terms of the  uncoupled subsystems. The
main advantage of this approach is that the decentralized feedback
gain  will be given only based on the information of the th
subsystem, .

By using the unitary transformation, as used in Sec. 4, system
\dref{opticon2} becomes

where , ,
 and . The  uncoupled controlled subsystems
 are as follows:

where . By the analysis in Sec.
4, one has


Suppose that the constant matrices  and  in
\dref{opticon6} satisfy the assumptions (A1-A4) as given in Lemma 2,
for  Then, the controller  to the th
subsystem can be given as follows:

where  is the stabilizing solution to the following Riccati
equation:

for . Moreover, the feedback gain matrix  is
given by



\textbf{Example 5:} Consider Example 1 again. Suppose that the
network is with the global coupling configuration. The objective is
to design a controller such that network \dref{net1} with Chua's
oscillators can be driven to the synchronous solution
.

It is easy to deduce that
 Consequently, the
linearized equation of the controlled network is given by

where the constant matrices  in this example.

By the above analysis, the \textit{LQR} optimal controller can be
designed by solving the Riccati equation \dref{opticon4}. For
comparison, a pinning control strategy is also considered, where 
nodes are randomly selected to be pinned by controllers
 with  and .

Fig. 3 shows the different performances of the synchronous behavior
of the two controlled networks, by the \textit{LQR} optimal
controller and the pinning controller, respectively. The
corresponding values of  are listed in Table 5.

\begin{center}
\unitlength=1cm
 \qquad \hbox{\hspace*{0.1cm} \epsfxsize7cm \epsfysize5cm
\epsffile{conchualqrBCDI.eps} \;\quad \epsfxsize7cm \epsfysize5cm
\epsffile{conchuapinBCDI.eps}}
\end{center}

\vskip -0.7cm\qquad\qquad\qquad\qquad\; {\small \textit{LQR} optimal
control}\qquad\qquad\qquad\qquad\qquad\qquad\; {\small pinning
control}

\qquad\qquad\qquad\qquad\qquad {\small Fig. 3 \quad Different
effects of the \textit{LQR} and pinning controllers.}

\begin{center}
\vskip 0.2cm
\begin{tabular}{|c|c|c|c|}\hline
Controller & \textit{LQR} optimal control & Pinning control
\\\hline  & 0.1309 & 4.9799

\\\hline
\end{tabular}
\vskip 0.2cm \quad {\small Table 5 \quad Values of . }
\end{center}
\hfill 

\section{Conclusion}

\quad In this paper, synchronizability of dynamical networks is
considered based on some new measures: the swiftness and overshoot
of the network synchronization. The quantity , which
represents the  norm of the synchronization error vector
, is taken as the performance index of this kind of
synchronizability. It has been shown by several numerical examples,
 presents a suitable measure of both swiftness and
overshoot of network synchronization: the smaller the values of
, the faster with smaller overshoot the network
synchronization. For the case when the synchronous state is an
equilibrium point,  is upper-bounded by the product of the
vector 2-norm of the initial error vector  and the  norm
of the transfer function , denoted as , of the
linearized network about the equilibrium point. The relationship
between  and the network structure has also been
discussed. Under some assumptions, it has been proved that
 will not increase as the real eigenvalues of the outer
coupling matrix increase. Finally, based on the techniques of the
\textit{LQR} control theory, an optimal controller has been
suggested to drive the network onto some homogenous stationary
states, which , in the mean time, can minimize the  norm of the
output of the linearized network. Further research along this
direction seems to be quite promising as long as the network energy
and performance are concerned, therefore deserves further efforts.






\vskip 2mm

\begin{thebibliography}{99}

\bibitem{bern05}
M. di Bernardo, F. Garofalo and F. Sorrentino, Synchronizability of
degree correlated networks, {\it arXiv: cond-mat/0504335}.

\bibitem{duan07}
Z. S. Duan, J. Z. Whang, G. R. Chen and L. Huang, Complexity in
linearly coupled dynamical networks: some unusual phenomena in
energy accumulation, 2007, {\it submitted}.

\bibitem{hong04}
H. Hong, B. J. Kim, M. Y. Choi and H. Park, Factors that predict
better synchronizability on complex networks, {\it Phys. Rev. E},
vol. 69, 2004, 067105.



\bibitem{hu98}
G. Hu, J. Z. Yang and W. J. Liu, Instability and controllability of
linearly coupled oscillators: eigenvalues analysis, {\it Phys. Rev.
E}, vol. 58, no. 4, 1998, pp. 4440-4453.

\bibitem{li04}
X. Li, X. F. Wang and G. R. Chen, Pinning a complex dynamical
network to its equilibrium, {\it IEEE Trans. Circuits Syst.-I },
vol. 51, no. 48, 2004, pp. 2074-2087.

\bibitem{nish03}
T. Nishikawa, A. E. Motter, Y. C. Duan and F. C. Hoppensteadt,
Heterogeneity in oscillator networks: are smaller worlds easier to
synchronize? {\it Phys. Rev. Lett.}, vol. 91, no. 1, 2003, 014101.

\bibitem{nish06}
T. Nishikawa and A. E. Motter, Synchronization is optimal in
nondiagonalizable networks, {\it Phys. Rev. Lett.}, vol. 91, no. 1,
2003, 014101.

\bibitem{wang02}
X. F. Wang and G. R. Chen, Pinning control of scale-free dynamical
networks, {\it Physica A}, vol. 310, 2002, pp. 521-531.

\bibitem{wu94}
C. W. Wu and L. O. Chua, A unified framework for synchronization and
control of dynamical systems, {\it Int. J. Bifurc. Chaos}, vol. 4,
no. 4, 1994, pp. 979-989.

\bibitem{wu03}
C. W. Wu, Perturbation of coupling matrices and its effect on the
synchronizability in arrays of coupled chaotic systems, {\it Phys.
Lett. A}, vol. 319, 2003, pp. 495-503.

\bibitem{wu05}
C. W. Wu, Synchronizability of networks of chaotic systems coupled
via a graph with a prescribed degree sequence, {\it Phys. Lett. A},
vol. 346, 2005, pp. 281-287.

\bibitem{wu06}
X. Wu, B. Wang, T. Zhou, W. Wang, M. Zhao and H. Yang,
Synchronizability of highly clusterd scale-free networks, {\it
Chinese Phys. Lett.}, vol. 23, no. 4, 2006, pp. 1046-1049.

\bibitem{zhou96}
K. M. Zhou, J. C. Doyle and K. Glover, Robust and optimal control,
{\it Englewood Cliffs: Prentice-Hall}, 1996.

\bibitem{zhou06}
T. Zhou, M. Zhao, and B. H. Wang, Better synchronizability predicted
by crossed double cycle, {\it Phys. Rev. E}, vol. 73, 2006, 037101.




\end{thebibliography}

\end{document}
