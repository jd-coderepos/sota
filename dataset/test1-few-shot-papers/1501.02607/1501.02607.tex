

One of the main technical results for parity automata is the so-called ``Simulation Theorem'':

\begin{theorem}[\cite{Walukiewicz96,Walukiewicz02}]\label{thm:origsimulation}
	Every automaton  is equivalent (over all models) to a \emph{non-deterministic} automaton .
\end{theorem}

Very informally, an automaton  is called non-deterministic when in every acceptance game , if \abelard can choose to play both  and  at a given moment, then . That is, \abelard's power boils down to being a \emph{pathfinder} in . He chooses the elements of  whereas the state of  is `fixed' by the valuation played by \eloise, for every given . For a formal definition we refer the reader to Definition~12 and Lemma~19--20 in~\cite{Walukiewicz96}.

To get a better picture of what non-determinism means, it is good to do the following: first observe that if we fix a strategy  for \eloise for the game  then the whole game can be represented by a tree, whose nodes are the different admissible moves for \abelard. We assume that the automaton is clear from context and denote such a tree by . Figure~\ref{fig:strategies1} shows the move-tree for \abelard for some fixed strategy  for \eloise. Each branch of  represents a possible -guided match.



\begin{figure}[h]
\centering
\begin{tikzpicture}
\begin{scope}[
->,>=latex,
level/.style={level distance=1.5cm, sibling distance=2cm/#1}
]
\node {}
    child {
        node {}
            child {
                node {}
edge from parent
            }
            child {
                node {}
edge from parent
            }
            edge from parent 
    }
    child {
        node {}
        child {
                node {}
                edge from parent
            }
            child {
                node {}
                edge from parent
            }
        edge from parent         
    };
\end{scope}

\begin{scope}[
xshift=7cm,
level/.style={level distance=1.5cm, sibling distance=3cm/#1},
level 1/.style={sibling distance=2cm}
]
\node {}
    child {
        node {}
            child {
                node {}
                edge from parent
            }
            child {
                node {}
                edge from parent
            }
edge from parent 
    }
    child {
        node {}
        edge from parent 
    }
    child {
        node {}
            child {
                node {}
                edge from parent
            }
            child {
                node {}
                edge from parent
            }
            child {
                node {}
                edge from parent
            }
        edge from parent         
    };
\end{scope}
\end{tikzpicture}
\caption{A tree  and  for a fixed strategy for \eloise.}
\label{fig:strategies1}
\end{figure}
 
Observe that in this figure the chosen strategy for \eloise is \emph{not} non-deterministic. The admissible moves which violate this condition are underlined. On trees, the notion of non-deterministic strategy can be rephrased as ``every element  occurs at most once as an admissible move for \abelard.''

\begin{remark}
	The terminology ``non-deterministic'' may seem confusing, given that  is certainly more ``deterministic''  than  (from the point of view of \abelard.) However, the terminology is sensible when seen from the following perspective: we say that a finite state automaton (on words) is \emph{deterministic} when the next state is uniquely determined by the current state (and the input); on the other hand, they are called \emph{non-deterministic} when \eloise can choose between different transitions, leading to the next state; finally, \emph{alternating} finite state automata are a generalization where the next state is chosen by a complex interaction of \eloise and \abelard.
Going back to parity automata, the above theorem then says that every alternating parity automaton is equivalent to a non-deterministic automaton. In light of our brief discussion, it should be clear that non-deterministic automata are ``more deterministic'' than alternating automata.
\end{remark}

Unfortunately, the transformation performed by Theorem~\ref{thm:origsimulation} does not preserve the weakness and additivity conditions (see~\cite[Remark~3.5]{Zanasi:Thesis:2012}), and therefore does not give us non-deterministic automata for the class .

In this section we provide, for every automaton  and , an automaton  which, although not being fully non-deterministic, is specially tailored to prove the closure of  under finite chain projection.
The state space of the construction  will be the disjoint union of two parts:
\begin{itemize}
	\itemsep 0 pt 
	\item A \emph{non-deterministic} part based on ; and
	\item An \emph{alternating} part based on .
\end{itemize}
The non-deterministic part will basically be a powerset construction of , and contain the initial state of the automaton. It will have very nice properties enforced by construction:
\begin{itemize}
 	\itemsep 0 pt
 	\item It will behave non-deterministically,
 	\item The parity of every element will be  (trivially satisfying the weakness condition); and 
 	\item The transition map of every element will be completely additive in .
\end{itemize} 
The alternating part will be a copy of  modified such that it cannot be used to read nodes colored with the propositional variable . The automaton  will therefore be based both on states from  and on ``macro-states'' from . Moreover, the transition map of  will be defined such that once a match goes from the non-deterministic part to the alternating part, then it cannot come back (see Fig.~\ref{fig:twopart} for an illustration). Successful runs of  will have the property of processing only a \emph{finite} amount of the input being in a macro-state and all the rest behaving exactly as  (but without reading any~).



\begin{figure}[h]
\centering
\begin{tikzpicture}[node distance=1.3cm,>=stealth',bend angle=45,auto]
  \tikzstyle{node}=[circle,fill=black,minimum size=5pt,inner sep=1pt]
  \tikzstyle{place}=[circle,thick,draw=blue!75,fill=blue!20,minimum size=6mm]
  \tikzstyle{iplace}=[place,draw=red!75,fill=red!20]
  \tikzstyle{inter}=[dashed,draw=black!50,->]

  \begin{scope}
\node [node] (w1)                                    {};
    \node [node] (c1) [below of=w1]                      {};
    \node [node] (s)  [below of=c1] {};
    \node [node] (c2) [below of=s]                       {};
    \node [node] (w2) [below of=c2]                      {};

    \node [node] (e1) [left of=c1] {}
      edge [pre,bend left]                  (w1)
      edge [post,bend right]                (s)
      edge [post]                           (c1);

    \node [node] (e2) [left of=c2] {}
      edge [pre,bend right]                 (w2)
      edge [post,bend left]                 (s)
      edge [post]                           (c2);

    \node [node] (l1) [right of=c1] {}
      edge [pre]                            (c1)
      edge [pre,bend left]                  (s)
      edge [post,bend right] (w1);

    \node [node] (l2) [right of=c2] {}
      edge [pre]                            (c2)
      edge [pre,bend right]                 (s)
      edge [post,bend left]         (w2);
  \end{scope}

  \begin{scope}[xshift=-6cm]
\node [iplace,tokens=1] (w1')   {}
      edge [inter,bend angle=20, bend left] (e1);

    \node [place,tokens=2] (c1') [below of=w1'] {}
    	edge [inter,bend angle=30, bend right] (l1);

    \node [place,tokens=2] (s1') [below of=c1',xshift=-5mm]      {};
    \node [place,tokens=3]
                      (s2') [below of=c1',xshift=5mm] {};
    \node [place,tokens=4]     (c2') [below of=s1',xshift=5mm]                      {}
    	edge [inter,bend angle=20, bend right] (l2);

    \node [place,tokens=3]
                      (w2') [below of=c2']                                 {};

    \node [place,tokens=3] (e1') [left of=c1'] {}
      edge [pre,bend left]                  (w1')
      edge [post]                           (s1')
      edge [pre]                            (s2')
      edge [post]                           (c1')
      edge [inter,bend angle=20, bend right] (c1);

    \node [place,tokens=3] (e2') [left of=c2'] {}
      edge [pre,bend right]                 (w2')
      edge [post]                           (s1')
      edge [pre]                            (s2')
      edge [post]                           (c2')
      edge [inter,bend angle=20, bend right] (c2);

    \node [place,tokens=3] (l1') [right of=c1'] {}
      edge [pre]                            (c1')
      edge [pre]                            (s1')
      edge [post]                           (s2')
      edge [post,bend right]  (w1');

    \node [place,tokens=3] (l2') [right of=c2'] {}
      edge [pre]                            (c2')
      edge [pre]                            (s1')
      edge [post]                           (s2')
      edge [post,bend left]        (w2')
      edge [inter,bend angle=20, bend right] (w2);
  \end{scope}

  \begin{pgfonlayer}{background}
    \filldraw [line width=4mm,join=round,black!10]
(w1'.north  -| l1.east) rectangle (w2'.south  -| e1.west)
      (w1'.north -| l1'.east) rectangle (w2'.south -| e1'.west);
  \end{pgfonlayer}
\end{tikzpicture}
\caption{Two-part construction, initial state in red (illustrative)}
\label{fig:twopart}
\end{figure}
 
The key property of , which we will use to prove the closure under finite chain projection, is that for every tree  and proposition , the following holds:


This finishes the intuitive explanations and we now turn to the necessary definitions to prove the results. The first step is to define a translation on the sentences associated with the
transition map of the original additive-weak automaton, which will aid us to define the transition map of the non-deterministic part. Henceforth, we use the notation  to denote the set .

\begin{definition}\label{def:nbl}
Let  be of the shape  for some  and . We say that  is a \emph{non-branching lifting} of  if
\begin{enumerate}[(i)]
	\itemsep 0 pt
	\item  for some ,
	\item For every , either:
			(a) , or 
			(b)  and .
	\item Case~(ii.b) occurs at most once.
\end{enumerate}
Consider now  of the shape . We say that  is a \emph{non-branching lifting} of  if  and 
\begin{enumerate}[(i)]
	\itemsep 0 pt
\item For every , either:
(a) , or 
(b)  is a non-branching lifting of .
\item Case~(i.b) occurs at most once.
\end{enumerate}
Observe that every such  is completely additive in .
\end{definition}

\begin{definition}\label{def:bigpsi}
Let . Let  be a color and  be a macro-state. First consider the formula

By Corollary~\ref{cor:ofoepositivenf} there is a formula  such that  and  is in the basic form  where
.
We define

Observe that  and  is completely additive in . The latter is because the non-branching liftings have this property, which is preserved by disjunction.
\end{definition}

\noindent
We are finally ready to define the two-part construct.

\begin{definition}\label{def:fc2}
Let  belong to  and let  be a propositional variable. We define the \emph{two-part construct of  with respect to } as the automaton  given by:

\end{definition}

The next definition introduces some notions of strategies that are closely related to our desiderata on the two-part construct.

\begin{definition}\label{def:StratfunctionalFinitary}
Given an automaton , a subset  of the states of , and a tree~; a strategy  for \eloise in  is called:
\begin{itemize}
	\itemsep 0pt
\item \emph{Functional in } (denoted `F in ', for short) if for each node  there is at most one  such that  belongs to .
\item \emph{Non-branching in } (denoted `NB in ', for short) if all the nodes of  with a state from  belong to the same branch of .
\item \emph{Well-founded in } (denoted `WF in ', for short) if the set of nodes of  with a state from  are all contained in a well-founded subtree of .
\end{itemize}
\end{definition}





\begin{figure}\centering
\begin{tikzpicture}
\begin{scope}[
level/.style={level distance=1.5cm, sibling distance=3cm/#1},
level 1/.style={sibling distance=2.5cm}
]
\node {}
    child {
        node {}
            child {
                node {}
                edge from parent
            }
            edge from parent 
    }
    child {
        node {}
            child {
                node {}
                edge from parent
            }
            child {
                node {}
                edge from parent
            }
        edge from parent         
    };
\end{scope}



\begin{scope}[
xshift=5.5cm,
level/.style={level distance=1.5cm, sibling distance=3cm/#1},
level 1/.style={sibling distance=2.5cm}
]
\node {}
    child {
        node {}
            child {
                node {}
                edge from parent
            }
        edge from parent         
    };
\end{scope}
\end{tikzpicture}
\caption{Functional and functional+non-branching strategies.}
\end{figure}
 
Before proving that the two-part construct satisfies nice properties we need a few propositions. The following two lemmas show how to go from admissible moves in the two-part construct to the original automaton and vice-versa.

\begin{lemma}[functional to alternating]\label{lem:os-nd-to-alt}
	Given an automaton , a macro-state  and a color  such that  for some ; there is  such that
		\begin{enumerate}[(i)]
			\itemsep 0 pt
			\item  for all ,
			\item If  then
				, or
				 for some  such that .
		\end{enumerate}
\end{lemma}
\begin{proof}
	Define  as
	
	Recall that
	
As a first step, let  where  is a non-branching lifting of some .

	\begin{claimfirst}
		.
	\end{claimfirst}
\begin{pfclaim}
		We show that . This is enough because . The only interesting case is that of the predicates in . Observe that in  there can be at most one . If there are none then  and we are done. Suppose that  occurs in , then  occurs in  in the same place. It is clear from the definition of  that if  then  for every .
	\end{pfclaim}
It is direct from the claim that  for all .
\end{proof}

\begin{lemma}[alternating to functional]\label{lem:os-alt-to-nd}
	Let  belong to ,  be a macro-state, and  be a color. Let  be a family of valuations such that  for each .
Then, for every  with  there is a valuation  such that
		\begin{enumerate}[(i)]
			\itemsep 0 pt
			\item ,
			\item If  then
				 for some .
			\item If  then
				 for some , .
			\item For every  we have that  iff .
		\end{enumerate}
\end{lemma}
\begin{proof}
	We use an auxiliary valuation  defined as .

	\begin{claimfirst}
		.
	\end{claimfirst}
\begin{pfclaim}
		Observe that for every ,  we have  then by monotonicity we get that  for every .
	\end{pfclaim}
Define the valuation , using the alternative marking representation , as follows:
	
	and recall that  and
	
Assume that , we show that  for some non-branching lifting of . If  is empty then  and as  is itself a non-branching lifting of  and , we can conclude that  and we are done.

	If  we proceed as follows: first recall that the shape of  is . Let  be the set of sorts to which  belongs. We show that  for some non-branching lifting  of . This will be enough, since it is easy to show that in that case  is a non-branching lifting of  (by Definition~\ref{def:nbl}) and .

	Our hypothesis is that , which gives a full description of the elements of  with sorts . Namely, if we restrict to the elements of sorts , then:
\begin{itemize}
		\itemsep 0pt
		\item There are  such that  has type ,
		\item Every  which is not among  satisfies some type in .
	\end{itemize}
We consider the following two cases:
	\begin{enumerate}[(1)]
		\item Suppose that  for some ; without loss of generality assume that . In this case it is easy to see that , which is a non-branching lifting of .
\item Suppose that  for all . Then,  must have some type . The key observation is that . Hence, there is some  such that . Observe now that if we `switch' the elements  and  we end up in case (1).\qedhere
	\end{enumerate}
\end{proof}

\begin{remark}
	Observe that, without loss of generality, \eloise can always choose to play \emph{minimal} valuations. That is, in a basic position  she plays a valuation  such that for every  and , the element  belongs to  only if it is strictly needed to make  true. That is, she plays valuations  such that
	
	for all  and .
	In what follows we assume that \eloise plays minimal valuations and we call such strategies minimal. For more detail we refer the reader to~\cite[Proposition~2.13]{Zanasi:Thesis:2012}.
\end{remark}

\noindent
Finally we can state and prove the properties of the two-part construct.

\begin{theorem}\label{thm:fc2}
Let ,  be a propositional variable, and  be a -tree. The following holds:
\begin{enumerate}
  \itemsep 0 pt
  \item\label{point:fc2funConstrAut}
  .
\item\label{point:fc2funConstrStrategy}
  Every winning strategy for  in
  the game  can be assumed to be functional, non-branching and well-founded in .
\item\label{point:fc2funConstrEquiv}
   accepts  iff 
   accepts  for some finite chain .
\end{enumerate}
\end{theorem}
\begin{proof}
\eqref{point:fc2funConstrAut} The key observation is that  is completely additive in .

\noindent
\eqref{point:fc2funConstrStrategy} We treat the properties separately:
\begin{itemize}
	\itemsep 0 pt
\item \textit{Functional in }: Suppose that  is a position of an -guided match where the proposed valuation  is such that  and  for distinct  and some . Let  be a disjunct of  witnessing . As  is a non-branching lifting, the element  has to be witness for exactly one type  with . As we assume that \eloise plays minimal strategies then we can assume that  only, among . Therefore,  cannot be required to be a witness for both  and  at the same time.
\item \textit{Non-branching in }: This is direct from the syntactical form of . Observe that in each disjunct, at most one element of  can occur. Assuming that  plays minimal strategies then she always proposes a valuation  where  is a quasi-atom.
\item \textit{Well-founded in }: The game starts in  and, as the parity of  is , it can only stay there for finitely many rounds. This means that, as  is winning, every branch of  has to leave  at some finite stage. 
\end{itemize}

\noindent
\eqref{point:fc2funConstrEquiv}
\fbox{}
Let  and therefore . Given a winning strategy  for \eloise in  we construct a winning strategy  for \eloise in . We define it inductively for a match  of . While playing  we maintain a bundle (set)  of -guided shadow matches. We use  to denote the bundle at round . We maintain the following condition () for every round along the play:
\begin{enumerate}[1.]
  \item If the current basic position in  is of the form , then (a) for every  there is an -guided shadow match  such that the current basic position is ; moreover, (b)  is not -free.
\item Otherwise, (a)  and the position in both  and  is of the form ; and moreover, (b)  is -free.
\end{enumerate}
Intuitively, in order to simulate  with  we have to keep two things in mind: (1)  can only read  while it is in the non-deterministic part; and (2) every choice of \abelard in  corresponds to a match that has to be won by \eloise; these parallel matches are kept track of in  and represented as a macro state in . Therefore, we want the simulation to stay in the non-deterministic part while we could potentially read some  in  (condition 1). Whenever there are no more 's to be read, we can relax and behave exactly as  (condition 2).

We only consider the case where  is not -free, and hence,  is non-empty. Otherwise it can be easily seen that \eloise can win  using the same strategy . Assume then that  is not -free. At round  we initialize the bundle  with the -guided match  at basic position . It is clear that~(1) holds. For the inductive step we divide in cases:

\begin{itemize}
	\item If (2) holds we are given a bundle  such that both  and  are in position . We define  as  for this position. To see that this is an admissible move in  observe that  is -free and therefore . Now it is \abelard's turn to make a move in . By definition of , the formula  belongs to  and hence the next position in  will be of the form  with  -free. We replicate the move in the shadow match  and hence (2) is preserved.
\item If (1) holds at round  of  we are given -guided matches  such that for the current position  and for each  we have . 
	For every match , the strategy  provides a valuation  which is an admissible move in this match. Define p and observe that, as  is a chain,  will be either a singleton or empty. Using Lemma~\ref{lem:os-alt-to-nd} with  and  we can combine these valuations into an admissible move  in .

	To prove that~() is preserved we distinguish cases as to \abelard's move: first suppose that \abelard chooses a position of the form . Because of Lemma~\ref{lem:os-alt-to-nd}(ii) we know that  for some . That is, we can replicate this move in one of the shadow matches . We do that and set  hence validating (2a). To see that (2b) is also satisfied observe that Lemma~\ref{lem:os-alt-to-nd}(iv) implies that  is -free.

	For the other case, suppose that \abelard chooses a position of the form . Similar to the last case, this time using Lemma~\ref{lem:os-alt-to-nd}(iii), we can trace every  back to some match . We define  as the match  extended with \abelard's move . Finally we let , which validates (1a). To see that (1b) is also satisfied observe that Lemma~\ref{lem:os-alt-to-nd}(iv) implies that  is not -free.
\end{itemize}
Now we prove that  is actually winning. It is clear that \eloise wins every finite full -guided match (because the moves are admissible). Now suppose that an -guided match is infinite. By hypothesis the extension of  in  is a finite chain, so after a finite amount of rounds we arrive to an element  such that  is -free. This means --because of ()-- that the automaton stays in  only for a finite amount of steps and then moves to , at a position  which is \emph{winning} for \eloise. From there on the match  and  are exactly the same and, as \eloise wins  (which is -guided for a winning strategy ), she also wins .



\begin{figure}[h]
\centering
\begin{tikzpicture}[
level/.style={sibling distance=70mm/#1},
level 1/.style={sibling distance=40mm},
level 2/.style={sibling distance=25mm},
level 3/.style={sibling distance=20mm},
every node/.style={draw=black!50,circle,fill=black!50,inner sep=1pt},
every edge/.style={draw=black!50},
aedge/.style={draw=black!50,thick},
ndnode/.style={draw=blue!75,circle,fill=blue!75,inner sep=2pt},
ndedge/.style={draw=blue!75,very thick},
end/.style={isosceles triangle,fill=black!20,rotate=+90,minimum width=1cm,line width=0pt}
]
\node [ndnode] {}
  child[ndedge] {
    node[ndnode,fill=yellow] {}
    child {
      node[ndnode] {}
      child[aedge] {
        node {}
      	child {node[end] {}}
      } 
      child {
        node[ndnode,fill=yellow] {}
        child[draw=red!75] {
            node[end,fill=red!50,draw=red!75] {}
        }
      }
    }
    child[aedge] {
      node[fill=yellow] {}
      child {
        node {}
          child {
            node[end] {}
          }
      }
    }
  }
child[aedge] {
	  node {}
	  child {
      node {}
		  child {node[end] {}}
	  } 
	  child {
      node[fill=yellow] {}
	  	child {node[end] {}}
	  }
	};
\end{tikzpicture}
\caption{F+NB mode (blue), alternating mode (red) and  (yellow).}
\label{fig:xp}
\end{figure}
 
\medskip
\noindent
\fbox{}
Given a winning strategy  for \eloise in  we construct a winning strategy  for \eloise in  where  for some finite chain , which we promptly define.
Using Theorem~\ref{thm:fc2}(\ref{point:fc2funConstrStrategy}) we assume that  is functional, non-branching and well-founded in  and define the set  as follows:

The fact that  is non-branching (in ) makes  a chain, and well-foundedness makes it finite. As an illustration, Fig.~\ref{fig:xp} represents a possible tree  where the path induced by positions of the form  is drawn with a thicker stoke.

Next, we define the strategy inductively for a match  of . While playing  we maintain an -guided shadow match . We maintain the following condition () for every round along the play: let  be the current position in , then one of the following conditions holds:
\begin{enumerate}[1.]
	\itemsep 0pt
	\item The current basic position in  is of the form  with ,
\item The current basic position in  is also . \end{enumerate}
At round  the matches  and  are in position  and  respectively, therefore (1) holds. For the inductive step we divide in cases:

\begin{itemize}
	\item If (2) holds, the match  is in position . For this position, we let  be defined as . Observe that it must be the case that , otherwise \eloise wouldn't have an admissible move  in . Given this, and assuming that \eloise plays minimal strategies, \eloise can use the same  in . It is easy to see that we can replicate \abelard's next move in the shadow match.
\item If (1) holds, the matches  and  are respectively in position  and  with . The strategy  provides a valuation  which is admissible in . Using Lemma~\ref{lem:os-nd-to-alt} we can get a valuation  which is admissible in  --see item (i). Suppose now that \abelard chooses  as a next position in . Using Lemma~\ref{lem:os-nd-to-alt}(ii) we know that either (a)  or, (b) there is some  with  and . In both cases we have a way to replicate \abelard's move in  and preserve ().
\end{itemize}
To see that  is winning we proceed similar to the other direction.
\end{proof}

\paragraph{Historical remarks and related results.}
The idea of a Simulation Theorem goes back to (at least) Safra~\cite{Safra:1988} and Muller and Schupp~\cite{MullerSimulation}. In the first case, Safra used an augmented state space to convert non-deterministic B\"uchi automata into deterministic automata. In the latter, Muller and Schupp also use an augmented state space to convert alternating tree automata to non-deterministic tree automata.

The idea to use a two-part automata to preserve the weakness condition was introduced in~\cite{Zanasi:Thesis:2012,DBLP:conf/lics/FacchiniVZ13}, although the authors claim that some concepts were already present in~\cite{MullerSaoudiSchupp92}. In~\cite{Zanasi:Thesis:2012,DBLP:conf/lics/FacchiniVZ13} the authors use an automaton based on  and  with a non-parity acceptance condition. This automaton is then converted to a parity automaton with a standard trick. Using  as the state space instead of  is necessary to correctly keep track of \emph{infinite} runs of the automata. The first explicit use of  as the state space of such automata seems to be in~\cite[Section~9.6.2]{ArnoldN01}.

Observe, however, that in the non-deterministic part of our constructions the parity is uniformly  and therefore therefore any infinite run which stays in that part will be a rejecting run. Using this observation, we give a slightly simpler construction based on  and . In this respect, the proofs are cleaner and we avoid a non-parity acceptance condition.

The second critical element of this section is the enforcing of the additivity condition on the non-deterministic component. The core of this idea was developed in~\cite{LICS14,DBLP:journals/corr/CarreiroFVZ14} where it is applied for another notion called ``continuity.''
