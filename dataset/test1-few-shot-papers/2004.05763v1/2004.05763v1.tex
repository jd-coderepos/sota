\documentclass[10pt,twocolumn,letterpaper]{article}

\usepackage{cvpr}
\usepackage{times}
\usepackage{epsfig}
\usepackage{graphicx}
\usepackage{amsmath}
\usepackage{amssymb}
\usepackage{soul}
\usepackage{multirow}
\usepackage{csquotes}
\usepackage{enumerate}
\usepackage[normalem]{ulem}
\usepackage{algorithm}
\usepackage{algorithmicx}
\usepackage{algpseudocode}
\usepackage{booktabs}
\usepackage{array}
\usepackage{overpic}
\usepackage{rotating}



\usepackage[pagebackref=false,breaklinks=true,letterpaper=true,colorlinks,bookmarks=false]{hyperref}

\cvprfinalcopy 

\def\cvprPaperID{2097} \def\httilde{\mbox{\tt\raisebox{-.5ex}{\symbol{126}}}}

\def\ourmodel{\textit{UC-Net}}
\ifcvprfinal\pagestyle{empty}\fi

\graphicspath{{./Imgs/}}
\DeclareGraphicsExtensions{.jpg,.pdf,.png}
\begin{document}

\title{UC-Net: Uncertainty Inspired RGB-D Saliency Detection \\via Conditional Variational Autoencoders
}

\author{
Jing Zhang\quad
Deng-Ping Fan\thanks{Corresponding author: Deng-Ping Fan \emph{(dengpfan@gmail.com)}}\quad
Yuchao Dai\quad
Saeed Anwar\\
Fatemeh Sadat Saleh\quad
Tong Zhang\quad
Nick Barnes\\
 Australian National University \quad
 CS, Nankai University \quad
 Northwestern Polytechnical University \\
 ACRV \quad
 Data61 \quad 
 Inception Institute of Artificial Intelligence (IIAI), Abu Dhabi, UAE\\
}



\def\JZ#1{{\color{red}{\bf [Jing:} {\it{#1}}{\bf ]}}}
\def\YD#1{{\color{blue}{\bf [Yuchao:} {\it{#1}}{\bf ]}}}
\def\SA#1{{\color{red}{\bf [Saeed:} {\it{#1}}{\bf ]}}}
\def\NB#1{{\color{green}{\bf [Nick:} {\it{#1}}{\bf ]}}}
\newcommand{\FS}[1]{\textcolor{magenta}{{\bf #1}}}
\newcommand{\fs}[1]{\textcolor{magenta}{ #1}}
\newcommand{\fdp}[1]{{\textcolor{red}{#1}}}
\def\TZ#1{{\color{cyan}{\bf [Tong:} {\it{#1}}{\bf ]}}}
\maketitle
\thispagestyle{empty}

\begin{abstract}

In this paper, we propose the first framework (\textbf{\ourmodel})~to employ uncertainty for RGB-D saliency detection by learning from the data labeling process. Existing RGB-D saliency detection methods treat the saliency detection task as a point estimation problem, and produce a single saliency map following a deterministic learning pipeline. 
Inspired by the saliency data labeling process, we propose probabilistic RGB-D saliency detection network via conditional variational autoencoders to model human annotation uncertainty
and generate multiple saliency maps for each input image by sampling in the latent space. 
With the proposed saliency consensus process, we are able to generate an accurate saliency map based on these multiple predictions. 
Quantitative and qualitative evaluations on six challenging benchmark datasets against 18 competing algorithms demonstrate the effectiveness of our approach in learning the distribution of saliency maps, leading to a new state-of-the-art in RGB-D saliency detection\footnote{Our code is publicly available at: \url{https://github.com/JingZhang617/UCNet}.}.
 



\end{abstract}

\section{Introduction}
Object-level visual saliency detection involves separating the most conspicuous objects that attract humans from the background \cite{itti_saliency,achanta2009frequency,Iter_Coop_CVPR,Zhang_2018_CVPR,Liu_2019_ICCV,F3Net_aaai2020,jing2020weakly}. Recently, visual saliency detection from RGB-D images have attracted lots of interest due to the importance of depth information in human vision system and the popularity of depth sensing technologies \cite{dmra_iccv19,zhao2019Contrast}. Given a pair of RGB-D images, the task of RGB-D saliency detection aims to predict a saliency map by exploring the complementary information between color image and depth data. 



The de-facto standard for RGB-D saliency detection is to train a deep neural network using ground truth (GT) saliency maps provided by the corresponding benchmark datasets, where the GT saliency maps are obtained through human consensus or by the dataset creators \cite{sip_dataset}. 
Building upon large scale RGB-D datasets, deep convolutional neural network based models \cite{Fu2020JLDCF,dmra_iccv19,chen2019three,han2017cnns} have made profound progress in learning the mapping from an RGB-D image pair to the corresponding GT saliency map. Considering the progress for RGB-D saliency detection under this pipeline, in this paper, we would like to argue that this pipeline fails to capture the \textit{uncertainty} in labeling the GT saliency maps. 

According to research in human visual perception \cite{scanpath}, visual saliency detection is subjective to some extent. Each person could have specific preferences in labeling the saliency map (which has been previous discussed in user-specific saliency detection \cite{ITTI20001489}). Existing approaches to RGB-D saliency detection treat saliency detection as a point estimation problem, and produce a single saliency map for each input image pair following a \textit{deterministic} learning pipeline, which fails to capture the stochastic characteristic of saliency, and may lead to a partisan saliency model as shown in second row of Fig. \ref{fig:inconsistent_ef_sod}. 
Instead of obtaining only a single saliency prediction (point estimation), we are interested in how the network produces multiple predictions (distribution estimation), which are then processed further to generate a single prediction in a similar way to how the GT saliency maps are created.






\begin{figure}[t!]
	\centering
    \small
	\begin{overpic}[width=.98\columnwidth]{FirstImage-min}
    \end{overpic}
    \caption{Provided GT compared with UC-Net (ours) predicted saliency maps. For images with a single salient object (1  row), we can produce consistent prediction. When multiple salient objects exist (2 row),
we can produce diverse predictions.
}
\label{fig:inconsistent_ef_sod}
    \vspace{-4mm}
\end{figure}

In this paper, inspired by human perceptual uncertainty, we propose a conditional variational autoencoders \cite{structure_output} (CVAE) based RGB-D saliency detection model \ourmodel~to produce multiple saliency predictions by modeling the distribution of output space as a generative model conditioned on the input RGB-D images to account for the human uncertainty in annotation.

However, there still exists one obstacle before we could apply the probabilistic framework, that is existing RGB-D benchmark datasets generally only provide a single GT saliency map for each RGB-D image pair. 
To produce diverse and accurate predictions\footnote{Diversity of prediction is related to the content of image. Image with clear content may lead to consistent prediction (1 row in Fig. \ref{fig:inconsistent_ef_sod}), while complex image may produce diverse predictions (2 row of Fig. \ref{fig:inconsistent_ef_sod}).}, we resort to the ``hide and seek'' \cite{hide_and_seek-iccv2017} principle following the orientation shifting theory \cite{ITTI20001489} by iteratively hiding the salient foreground from the RGB image for testing, which forces the deep network to learn the saliency map with diversity. 
Through this iterative hiding strategy, we obtain multiple saliency maps for each input RGB-D image pair, which reflects the diversity/uncertainty from human labeling.

Moreover, depth data in the RGB-D saliency dataset can be noisy, and a direct fusion of RGB and depth information may overwhelm the network to fit noise. To deal with the noisy depth problem, a depth correction network is proposed as an auxiliary component to produce depth images with rich semantic and geometric information. We also introduce a saliency consensus module to mimic the majority voting mechanism for saliency GT generation.


Our main contributions are summarized as: 1) We propose a conditional probabilistic RGB-D saliency prediction model that can produce diverse saliency predictions instead of a single saliency map;
2) We provide a mechanism via saliency consensus to better model how saliency detection works; 3) We present a depth correction network to decrease noise that is inherent in depth data; 4) Extensive experimental results on six RGB-D saliency detection benchmark  datasets demonstrate the effectiveness of our \ourmodel.





\section{Related Work}











\subsection{RGB-D Saliency Detection}
Depend on how the complementary information between RGB images and depth images is fused, existing RGB-D saliency detection models can be roughly classified into three categories: early-fusion models \cite{qu2017rgbd}, late-fusion models \cite{wang2019adaptive,han2017cnns} and cross-level fusion models \cite{dmra_iccv19,chen2018progressively,chen2019multi,chen2019three,zhao2019Contrast}. 
Qu \etal \cite{qu2017rgbd} proposed an early-fusion model to generate feature for each superpixel of the RGB-D pair, which was then fed to a CNN to produce saliency of each superpixel. Recently, Wang~\etal~\cite{wang2019adaptive} introduced a late-fusion network (\ie AFNet) to fuse predictions from the RGB and depth branch adaptively. In a similar pipeline, Han \etal \cite{han2017cnns} fused the RGB and depth information through fully connected layers.
Chen \etal \cite{chen2019multi} used a multi-scale multi-path network for different modality information fusion. Chen \etal \cite{chen2018progressively} proposed a complementary-aware RGB-D saliency detection model by fusing features from the same stage of each modality with a complementary-aware fusion block.
Chen \etal \cite{chen2019three} presented attention-aware cross-level combination blocks for multi-modality fusion.
Zhao \etal \cite{zhao2019Contrast} integrated a contrast prior to enhance depth cues, and employed a fluid pyramid integration framework to achieve multi-scale cross-modal feature fusion.
To effectively incorporate geometric and semantic information within a recurrent learning framework, Li \etal~\cite{dmra_iccv19} introduced a depth-induced multi-scale RGB-D saliency detection network.




\begin{figure*}[!htp]
   \begin{center}
{\includegraphics[width=0.85\linewidth]{train_rgbd-min}} 
   \end{center}
   \vspace{-6pt}
   \caption{Network training pipeline. Four main modules are included, namely a LatentNet (PriorNet  and PosteriorNet
), a SaliencyNet, a DepthCorrectionNet
and a PredictionNet. The LatentNet maps the RGB-D image pair  (or together with GT  for the PosteriorNet) to low dimensional Gaussian latent variable .  The DepthCorrectionNet
refines the raw depth with a semantic guided loss. The SaliencyNet takes the RGB image and the refined depth as input to generate a saliency feature map. The PredictionNet takes both stochastic features and deterministic features to produce a final saliency map. We perform saliency consensus in the testing stage, as shown in Fig. \ref{fig:testing_overview} to generate the final saliency map according to the mechanism of GT saliency map generation.}






\label{fig:overview}
   \vspace{-4mm}
\end{figure*}

\begin{figure}[!htp]
\begin{center}
   {\includegraphics[width=1\linewidth]{testing_rgbd-min}} 
   \end{center}
   \vspace{-7pt}
   \caption{Overview of the proposed framework during testing. We sample the PriorNet multiple times to generate diverse and accurate predictions. The saliency consensus module is then used to obtain the majority voting of the final predictions.}
   \label{fig:testing_overview}
   \vspace{-4mm}
\end{figure}





\subsection{VAE or CVAE based Deep Probabilistic Models}
Ever since the seminal work by Kingma \etal \cite{vae_bayes_kumar} and Rezende \etal \cite{pmlr-v32-rezende14}, variational autoencoder (VAE) and its conditional counterpart CVAE \cite{structure_output} have been widely applied in various computer vision problems. 
To train a VAE, a reconstruction loss and a regularizer are needed to penalize the disagreement of the prior and posterior distribution of the latent representation. 
Instead of defining the prior distribution of the latent representation as a standard Gaussian distribution, CVAE utilizes the input observation to modulate the prior on Gaussian latent variables to generate the output.
In low-level vision, VAE and CVAE have been applied to the tasks such as image background modeling~\cite{SuperVAE_AAAI19}, latent representations with sharp samples~\cite{pixel_vae}, difference of motion modes~\cite{MT-VAE}, medical image segmentation model~\cite{PHiSeg2019}, and modeling inherent ambiguities of an image~\cite{probabilistic_unet}.
Meanwhile, VAE and CVAE have been explored in more complex vision tasks such as uncertain future forecast~\cite{ContrastiveVAE,vae_future}, human motion prediction~\cite{aliakbarian2019learning}, and shape-guided image generation~\cite{Esser_2018_CVPR}. Recently, VAE algorithms have been extened to 3D domain targeting applications such as 3D meshes deformation~\cite{Tan_2018_CVPR}, and point cloud instance segmentation~\cite{Yi_2019_CVPR}. 




To the best of our knowledge, CVAE has not been exploited in saliency detection. Although Li \etal \cite{SuperVAE_AAAI19} adopted VAE in their saliency prediction framework, they used VAE to model the image background, and separated salient objects from the background through the reconstruction residuals. In contrast, we use CVAE to model labeling variants, indicating human uncertainty of labeling. We are the first to employ CVAE in saliency prediction network by considering the human uncertainty in annotation.

































\section{Our Model}\label{sec:OurApproach}
In this section, we present our probabilistic RGB-D saliency detection model based on a conditional variational autoencoder, which learns the distribution of saliency maps rather than a single prediction.
Let  be the training dataset, where  denotes the RGB-D input (consisting of the RGB image  and the depth image ),  denotes the ground truth saliency map. 
The whole pipeline of our model during training and testing are illustrated in Fig.~\ref{fig:overview} and Fig.~\ref{fig:testing_overview}, respectively. 

Our network is composed of five main modules: 1) LatentNet (PriorNet and PosteriorNet) that maps the RGB-D input  (for PriorNet) or  and  (for PosteriorNet) to the low dimensional latent variables  ( is dimension of the latent space); 2) DepthCorrectionNet that takes  and  as input to generate a refined depth image ; 3) SaliencyNet that maps the RGB image  and the refined depth image  to saliency feature maps ; 4) PredictionNet that employs stochastic features  from LatentNet and deterministic features  from SaliencyNet to produce our saliency map prediction ; 5) A saliency consensus module in the testing stage that mimics the mechanism of saliency GT generation to evaluate the performance with the provided single GT saliency map . We will introduce each module as follows.















\subsection{Probabilistic RGB-D Saliency Model via CVAE}








The Conditional Variational Autoencoder (CVAE) modulates the prior as a Gaussian distribution with parameters conditioned on the input data . There are three types of variables in the conditional generative model:
conditioning variable  (RGB-D image pair in our setting), latent variable , and output variable . 
For the latent variable  drawn from the Gaussian distribution , the output variable  is generated from ,
then the posterior of  is formulated as . The loss of CVAE is defined as:

where  is the likelihood of  given latent variable  and conditioning variable , the Kullback-Leibler Divergence  works as a regularization loss to reduce the gap between the prior  and the auxiliary posterior . 
In this way, CVAE aims to model the log likelihood  under encoding error .
Following the standard practice in conventional CVAE \cite{structure_output}, we design a CVAE-based RGB-D saliency detection network, and describe each component of our model in the following.


\noindent\textbf{LatentNet:} 
We define  as PriorNet that maps the input RGB-D image pair  to a low-dimensional latent feature space, where  is the parameter set of PriorNet. With the same network structure and provided GT saliency map , we define  as PosteriorNet, with  being the posterior net parameter set. 
In the LatentNet (PriorNet and PosteriorNet), we use five convolutional layers to map the input RGB-D image  (or concatenation of  and  for the PosteriorNet) to the latent Gaussian variable
,
where ,  , representing the mean and standard deviation of the latent Gaussian variable, as shown in Fig. \ref{fig:encoder_latent}.







\begin{figure}[t!]
	\centering
    \small
	\begin{overpic}[width=1.0\columnwidth]{RGBD_latent-min}
    \end{overpic}
	\caption{Detailed structure of LatentNet,
where  is dimension of the latent space, \enquote{c1\_4K} represents a  convolutional layer of output channel size , \enquote{GAP} is global average pooling.}
    \vspace{-4mm}
    \label{fig:encoder_latent}
\end{figure}

Let us define parameter set of PriorNet and PosteriorNet as  and  respectively. The KL-Divergence in Eq. \eqref{CVAE_equation} is used to measure the distribution mismatch between the prior net  and posterior net , or how much information is lost when using  to represent .
Typical using of CVAE involves multiple versions of ground truth  \cite{probabilistic_unet} to produce informative , with each position in  represents possible labeling variants or factors that may cause diverse saliency annotations. As we have only one version of GT, directly training with the provided single GT may fail to produce diverse predictions as the network will simply fit the provided annotation .



\noindent\textit{Generate Multiple Predictions:} To produce diverse and accurate predictions, we propose an iterative hiding technique inspired by \cite{hide_and_seek-iccv2017} following the orientation shifting theory \cite{ITTI20001489} to generate more
annotations as shown in Fig. \ref{fig:iterative_label_generation}. We iteratively hide the salient region in the RGB image with mean of the training dataset. The RGB image and its corresponding GT are set as the starting point of the \enquote{new label generation} technique. We first hide the ground truth salient object in the RGB image, and feed the modified image to an existing RGB saliency detection model \cite{BASNet_Sal} to produce a saliency map and treat it as one candidate annotation.
We repeat salient object hiding technique three times for each training image\footnote{We found that usually after three times of hiding, there exists no salient objects in the hidden image.} to obtain four different sets of annotations in total (including the provided GT), and we term this dataset as \enquote{AugedGT}, which is our training dataset.




During training, different annotations (as shown in Fig. \ref{fig:iterative_label_generation}) in  can force the PriorNet  to encode labeling variants of a given input . As we have already obtained diverse annotations with the proposed hiding technique,
we are expecting the network to produce diverse predictions for images with complicated context. During testing, we can obtain one stochastic feature  (input of the \enquote{PredictionNet}) of channel size  each time we sample as shown in Fig. \ref{fig:testing_overview}.



\begin{figure}[thp]
	\centering
    \small
	\begin{overpic}[width=1.0\columnwidth]{LabelGeneration-min}
    \end{overpic}
    \caption{New label generation. The 1 row: we iteratively hide the predicted salient region, where no region is hidden in the first image. The 2 row: the corresponding GT of the hidden image.}
   \label{fig:iterative_label_generation}
   \vspace{-1mm}
\end{figure}





\noindent\textbf{SaliencyNet:}
We design SaliencyNet to produce a deterministic saliency feature map  from the input RGB-D data, where the refined depth data comes from the DepthCorrectionNet. We use VGG16 \cite{VGG} as our encoder, and remove layers after the fifth pooling layer. To enlarge the receptive field, we follow DenseASPP~\cite{denseaspp} to obtain feature map with the receptive field of the whole image on each stage of the VGG16 network. We then concatenate those feature maps and feed it to another convolutional layer to obtain
.
The detail of the SaliencyNet is illustrated in Fig. \ref{fig:saliency_feature_net}, where \enquote{c1\_M} represents convolutional layer of kernel size , and  is channel size of .









\begin{figure}[tbp]
	\centering
    \small
	\begin{overpic}[width=1.0\columnwidth]{SaliencyFeature-min}
    \end{overpic}
    \caption{SaliencyNet, where \enquote{S1} represents the first stage of the VGG16 network, \enquote{daspp} is the DenseASPP module \cite{denseaspp}.}
   \label{fig:saliency_feature_net}
   \vspace{-4mm}
\end{figure}


\noindent\textit{Feature Expanding:} Statistics ( in particular) from the LatentNet (PriorNet during testing as shown in Fig. \ref{fig:testing_overview} \enquote{Sampling}, or PosteriorNet during training in Fig. \ref{fig:overview}) form the input to 
the Feature Expanding module. Given a pair of  in each position of the  dimensional vector, we obtain latent vector , where . To fuse with deterministic feature , we expand  to feature map of the same spatial size as  by defining  as two-dimensional Gaussian noise map. With , we can obtain 
a 
(size of the latent space) 
channel stochastic feature  representing labeling variants.


\noindent\textbf{PredictionNet:}
The LatentNet produces stochastic features  representing labeling variants, while the SaliencyNet outputs deterministic saliency features  of input . We propose the PredictionNet, as shown in Fig. \ref{fig:overview} to fuse features from mentioned branches.
A naive concatenation of  and  may lead the network to learn only from the deterministic features,
thus fail to model labeling variants. Inspired by~\cite{aliakbarian2019learning}, 
we mix  and  channel-wise; thus, the network cannot distinguish between features of the deterministic branch and the probabilistic branch.
We concatenate  and  to form a  channel feature map . We define  dimensional variable  (a learnable parameter) representing possible ranking of , and then  is mixed channel-wisely according to  
to obtain the mixed feature .
Three  convolutional layers with output channel sizes of , are included in the PredictionNet to map  to a single channel saliency map . 
During testing, with multiple stochastic features , we can obtain multiple predictions by sampling  from the LatentNet  multiple times.











\subsection{DepthCorrectionNet}
Two main approaches are employed to acquire depth data for RGB-D saliency detection: through depth sensors such as Microsoft Kinect, \eg, DES \cite{cheng2014depth}, and NLPR \cite{peng2014rgbd} datasets; or computing depth from stereo cameras, examples of such datasets are SSB \cite{niu2012leveraging} and NJU2K \cite{NJU2000}. Regardless of the capturing technique, noise is inherent in the depth data. 
We propose a semantic guided depth correction network to produce refined depth information 
as shown in Fig. \ref{fig:overview}, termed as \enquote{DepthCorrectionNet}. The encoder part of the DepthCorrectionNet is the same as the \enquote{SaliencyNet}, while the decoder part is composed of four sequential convolutional layers and bilinear upsampling operation.




We assume that edges of the depth map should be aligned with edges of the RGB image. We adopt the boundary IOU loss \cite{Luo2017CVPR} as a regularizer for DepthCorrectionNet to achieve a refined depth, which is guided by intensity of the RGB image. The full loss for DepthCorrectionNet is defined as:

where  is the smooth  loss between the refined depth  and the raw depth ,  is the boundary IOU loss between the refined depth  and intensity  of the RGB image .
Given the predicted depth map  and intensity of RGB image , we follow \cite{Luo2017CVPR} to compute the first-order derivatives of  and . Subsequently, we calculate the magnitude  and  of the gradients of  and , and define the boundary IOU loss as:





\subsection{Saliency Consensus Module}
Saliency detection is subjective to some extent, and it is common to have multiple annotators to label one image, and the final ground truth saliency map is obtained through majority voting strategy \cite{sip_dataset}.
Although it is well known in the saliency detection community about how the ground truth is acquired; yet, there exists no research on embedding this mechanism into deep saliency frameworks. \textit{Current models define saliency detection as a point estimation problem instead of a distribution estimation problem}. We, instead, use CVAE to obtain the saliency distribution. Next, we embed saliency consensus into our probabilistic framework to compute the majority voting of different predictions in the testing stage as shown in Fig. \ref{fig:testing_overview}.








During testing, we sample PriorNet with fixed  and  to obtain a stochastic feature . With each  and deterministic feature  from SaliencyNet, we obtain one version of saliency prediction . To obtain  different predictions , we sample PriorNet  times.
We simultaneously feed these multiple predictions to the saliency consensus module to obtain the consensus of predictions.

Given multiple predictions , where , we first compute the binary\footnote{As the GT map , we produce series of binary predictions with each one representing annotation from one saliency annotator.} version  of the predictions by performing adaptive threshold \cite{borji2015salient} on .
For each pixel , we obtain a  dimensional feature vector . We define  as a one-channel saliency map representing majority voting of .
We define an indicator  representing whether the binary prediction is consistent with the majority voting of the predictions. If , then . Otherwise, . We obtain one gray saliency map after saliency consensus as:



































\subsection{Objective Function} 
At this stage, our loss function is composed of two parts \ie  and . Furthermore, we propose to use the smoothness loss \cite{UnsupeGodard} as a regularizer to achieve edge-aware saliency detection, based on the assumption of inter-class distinction and intra-class similarity.
Following \cite{occlusion_aware}, we define first-order derivatives of the saliency map in the smoothness term as

where  is defined as ,
 is the predicted saliency map at position , and  is the image intensity,  indexes over partial derivative on  and  directions. We set  following \cite{occlusion_aware}.


Both the smoothness loss (Eq. \eqref{smoothness_loss}) and the boundary IOU loss (Eq. \eqref{boundary_iou_loss}) need intensity . We convert the RGB image  to a gray-scale intensity image  as \cite{Saliency_preserving_eccv}:

where ,  and  represent the color components in the linear color space after Gamma function been removed from the original color space.  is achieved via:

where  is the original red channel of image , and we compute  and  in the same way as Eq. \eqref{gamma_extension}.

With smoothness loss , depth loss  and CVAE loss , our final loss function is defined as:

In our experiments, we set .



\noindent\textbf{Training details:} We set channel size of  as , and scale of latent space as . We trained our model using Pytorch, and initialized the encoder of SaliencyNet and DepthCorrectionNet with VGG16 parameters pre-trained on ImageNet. Weights of new layers were initialized with , and bias was set as constant. We used the Adam method with momentum 0.9 and decreased the learning rate 10\% after each epoch. The base learning rate was initialized as 1e-4. The whole training took 13 hours with training batch size 6 and maximum epoch 30 on a PC with an NVIDIA GeForce RTX GPU. For input image size , the inference time is 0.06s on average.











\section{Experimental Results}


\begin{table*}[t!]
  \centering
  \scriptsize
  \renewcommand{\arraystretch}{1.1}
  \renewcommand{\tabcolsep}{0.9mm}
  \caption{Benchmarking results of ten leading handcrafted feature-based models and eight deep models on six RGBD saliency datasets.   denote larger and smaller is better, respectively. Here, we adopt mean  and mean .
}\label{tab:BenchmarkResults}
  \begin{tabular}{rr|cccccccccc|cccccccc|c}
  \hline
&  &\multicolumn{10}{c|}{Handcrafted Feature based Models}&\multicolumn{8}{c|}{Deep Models}&\multicolumn{1}{c}{} \\
    & Metric &
   LHM  & CDB  & DESM & GP    &
   CDCP & ACSD & LBE & DCMC & MDSF   & SE   & DF   & AFNet& CTMF & MMCI & PCF   & TANet& CPFP & DMRA & \ourmodel \\
   &  & \cite{peng2014rgbd}        & \cite{liang2018stereoscopic}       & \cite{cheng2014depth}          & \cite{ren2015exploiting}              &
        \cite{zhu2017innovative}   & \cite{NJU2000}                 & \cite{feng2016local}  & \cite{cong2016saliency}
        & \cite{song2017depth}   & \cite{guo2016salient} &\cite{qu2017rgbd}       & \cite{wang2019adaptive}& \cite{han2017cnns}    & \cite{chen2019multi}
         & \cite{chen2018progressively}  &\cite{chen2019three}   &   \cite{zhao2019Contrast} & \cite{dmra_iccv19} & Ours \\
  \hline
  \multirow{4}{*}{\textit{NJU2K} \cite{NJU2000}}
    &     & .514 & .632 & .665 & .527 & .669 & .699 & .695 & .686 & .748 & .664 & .763 & .822 & .849 & .858 & .877 & .879 & .878 & .886& \textbf{.897}\\
    &      & .328 & .498 & .550 & .357 & .595 & .512 & .606 & .556 & .628 & .583 & .653 & .827 & .779 & .793 & .840 & .841 & .850 & .873 & \textbf{.886}\\
    &        & .447 & .572 & .590 & .466 & .706 & .594 & .655 & .619 & .677 & .624 & .700 & .867 & .846 & .851 & .895 & .895 & .910 & .920 & \textbf{.930}\\
    &  & .205 & .199 & .283 & .211 & .180 & .202 & .153 & .172 & .157 & .169 & .140 & .077 & .085 & .079 & .059 & .061 & .053 & .051& \textbf{.043} \\ \hline
\multirow{4}{*}{\textit{SSB} \cite{niu2012leveraging}}
    &     & .562 & .615 & .642 & .588 & .713 & .692 & .660 & .731 & .728 & .708 & .757 & .825 & .848 & .873 & .875 & .871 & .879 & .835 & \textbf{.903}\\
    &      & .378 & .489 & .519 & .405 & .638 & .478 & .501 & .590 & .527 & .611 & .617 & .806 & .758 & .813 & .818 & .828 & .841 & .837 & \textbf{.884}\\
    &        & .484 & .561 & .579 &.508  & .751 & .592 & .601 & .655 & .614 & .664 & .692 & .872 & .841 & .873 & .887 & .893 & .911 & .879 & \textbf{.938}\\
    &  & .172 & .166 & .295 & .182 & .149 & .200 & .250 & .148 & .176 & .143 & .141 & .075 & .086 & .068 & .064 & .060 & .051 & .066 & \textbf{.039}\\ \hline
\multirow{4}{*}{\textit{DES} \cite{cheng2014depth}}
    &     & .578 & .645 & .622 & .636 & .709 & .728 & .703 & .707 & .741 & .741 & .752 & .770 & .863 & .848 & .842 & .858 & .872 & .900 & \textbf{.934}\\
    &      & .345 & .502 & .483 & .412 & .585 & .513 & .576 & .542 & .523 & .618 & .604 & .713 & .756 & .735 & .765 & .790 & .824 & .873 & \textbf{.919}\\
    &        & .477 & .572 & .566 & .503 & .748 & .613 & .650 & .631 & .621 & .706 & .684 & .809 & .826 & .825 & .838 & .863 & .888 & .933 & \textbf{.967}\\
    &  & .114 & .100 & .299 & .168 & .115 & .169 & .208 & .111 & .122 & .090 & .093 & .068 & .055 & .065 & .049 & .046 & .038 & .030 & \textbf{.019}\\ \hline
\multirow{4}{*}{\textit{NLPR} \cite{peng2014rgbd}}
    &     & .630 & .632 & .572 & .655 & .727 & .673 & .762 & .724 & .805 & .756 & .806 & .799 & .860 & .856 & .874 & .886 & .888 & .899 & \textbf{.920}\\
    &      & .427 & .421 & .430 & .451 & .609 & .429 & .636 & .542 & .649 & .624 & .664 & .755 & .740 & .737 & .802 & .819 & .840 & .865 & \textbf{.891}\\
    &        & .560 & .567 & .542 & .571 & .782 & .579 & .719 & .684 & .745 & .742 & .757 & .851 & .840 & .841 & .887 & .902 & .918 & .940 & \textbf{.951}\\
    &  & .108 & .108 & .312 & .146 & .112 & .179 & .081 & .117 & .095 & .091 & .079 & .058 & .056 & .059 & .044 & .041 & .036 & .031 & \textbf{.025}\\ \hline
\multirow{4}{*}{\textit{LFSD} \cite{li2014saliency}}
    &     & .557 & .520 & .722 & .640 & .717 & .734 & .736 & .753 & .700 & .698 & .791 & .738 & .796 & .787 & .794 & .801 & .828 & .847& \textbf{.864} \\
    &      & .396 & .376 & .612 & .519 & .680 & .566 & .612 & .655 & .521 & .640 & .679 & .736 & .756 & .722 & .761 & .771 & .811 & .845 & \textbf{.855}\\
    &        & .491 & .465 & .638 & .584 & .754 & .625 & .670 & .682 & .588 & .653 & .725 & .796 & .810 & .775 & .818 & .821 & .863 & .893 & \textbf{.901}\\
    &  & .211 & .218 & .248 & .183 & .167 & .188 & .208 & .155 & .190 & .167 & .138 & .134 & .119 & .132 & .112 & .111 & .088 & .075 & \textbf{.066}\\ \hline
\multirow{4}{*}{\textit{SIP} \cite{sip_dataset}}
    &     & .511 & .557 & .616 & .588 & .595 & .732 & .727 & .683 & .717 & .628 & .653 & .720 & .716 & .833 & .842 & .835 & .850 & .806 & \textbf{.875}\\
    &      & .287 & .341 & .496 & .411 & .482 & .542 & .572 & .500 & .568 & .515 & .465 & .702 & .608 & .771 & .814 & .803 & .821 & .811 & \textbf{.867}\\
    &        & .437 & .455 & .564 & .511 & .683 & .614 & .651 & .598 & .645 & .592 & .565 & .793 & .704 & .845 & .878 & .870 & .893 & .844 & \textbf{.914}\\
    &  & .184 & .192 & .298 & .173 & .224 & .172 & .200 & .186 & .167 & .164 & .185 & .118 & .139 & .086 & .071 & .075 & .064 & .085 & \textbf{.051}\\
\hline
  \end{tabular}
\end{table*}



  




\begin{figure*}[thp!]
	\centering
    \small
	\begin{overpic}[width=\textwidth]{PRCurve-min}
    \end{overpic}
     \caption{E-measure (1 row) and F-measure (2 row) curves on four testing datasets.} \label{fig:E_F_measure_show}
  \vspace{-4mm}
\end{figure*}



\subsection{Setup}
\noindent\textbf{Datasets:}
We perform experiments on six datasets including five widely used RGB-D saliency detection datasets (namely NJU2K \cite{NJU2000}, NLPR \cite{peng2014rgbd}, SSB \cite{niu2012leveraging}, LFSD \cite{li2014saliency}, DES \cite{cheng2014depth}) and one newly released dataset (SIP \cite{sip_dataset}). 



         
\noindent\textbf{Competing Methods:}
We compare our method with 18 algorithms, including ten handcrafted conventional methods and eight deep RGB-D saliency detection models. 







\begin{figure*}[t!]
	\centering
    \small
	\begin{overpic}[width=\textwidth]{SaliencyCompare-min}
    \end{overpic}
\caption{\small Comparisons of saliency maps. \enquote{MH1} and \enquote{MH2} are two predictions from M-head. \enquote{DP1} and \enquote{DP2} are predictions of two random MC-dropout during test. \enquote{Ours(1)} and \enquote{Ours(2)} are two predictions sampled from our CVAE based model.
Different from M-head and MC-dropout, which produce consistent predictions for ambiguous images (5 row), \ourmodel~can produce diverse predictions.
}
\vspace{-4mm}
\label{fig:saliency_compare}
\end{figure*}

\noindent\textbf{Evaluation Metrics:}
Four evaluation metrics are used, including two widely used: 1) Mean Absolute Error (MAE ); 2) mean F-measure () and two recently proposed: 3) Enhanced alignment measure (mean E-measure, ) \cite{Fan2018Enhanced} and 4) Structure measure (S-measure, ) \cite{fan2017structure}.
















\subsection{Performance Comparison}


\noindent\textbf{Quantitative Comparison:}
We report performance of our method and competing methods in Table \ref{tab:BenchmarkResults}. It shows that our method consistently achieves the best performance on all datasets, especially on SSB \cite{niu2012leveraging} and SIP \cite{sip_dataset}, our method achieves significant S-measure, E-measure, and F-measure performance boost and a decrease in MAE by a large margin. We show E-measure and F-measure curves of competing methods and ours in Fig. \ref{fig:E_F_measure_show}. 
We observe that our method produces not only stable E-measure and F-measure but also best performance.






\noindent\textbf{Qualitative Comparisons:}
In Fig. \ref{fig:saliency_compare}, we show five images comparing results of our method with one newly released RGB-D saliency detection method (DMRA \cite{dmra_iccv19}), and two widely used methods to produce structured outputs, namely M-head \cite{Rupprecht2016LearningIA} and MC-dropout \cite{kendall2015bayesian} (we will discuss these two methods in detail in the ablation study section). We design both M-head and MC-dropout based structured saliency detection models by replacing CVAE with M-head and MC-dropout respectively. 
Results in Fig. \ref{fig:saliency_compare} show that our method can not only produce high accuracy predictions (compared with DMRA \cite{dmra_iccv19}), but also diverse predictions (compared with M-head based and MC-dropout based models) for images with complex background (image in the first and last rows).







\subsection{Ablation Study}\label{sec:AblationStudy}
We carried out eight experiments (shown in Table \ref{tab:ablation_study}) to thoroughly analyse our framework, including network structure (\enquote{M1}, \enquote{M2}, \enquote{M3}), probabilistic model selection (\enquote{M4}, \enquote{M5}, \enquote{M6}), data source selection (\enquote{M7}) and effectiveness of the new label generation technique (\enquote{M8}).
We make the number bold
when it's better than ours.




\begin{table}[t!]
  \centering
  \scriptsize
  \renewcommand{\arraystretch}{1.1}
  \renewcommand{\tabcolsep}{1.1mm}
  \caption{Ablation study on RGB-D saliency datasets. 
  \vspace{2mm}
}\label{tab:ablation_study}
  \begin{tabular}{lr|cccccccccccc}
  \hline
& Metric & \ourmodel &M1  & M2 & M3  & M4     & M5 & M6 & M7& M8& M9 \\
  \hline
  \multirow{4}{*}{\begin{sideways}\textit{NJU2K} \cite{NJU2000}\end{sideways}}
    &     & .897 & .866 & .893 & \textbf{.905} & .871  & .885 & .881& .893& .838& .866  \\
    &      & .886  & .858 & \textbf{.887} & .884 & .851  & .878& .878 & .884& .787& .812    \\
    &        & .930  & .905 & .930 & .927 & .910  & .923& .927 & \textbf{.932}& .840& .866   \\
    &  & \textbf{.043}  & .060 & .046 & .045 & .059  & .047 & .046 & .044& .084& .075 \\ \hline
\multirow{4}{*}{\begin{sideways}\textit{SSB} \cite{niu2012leveraging}\end{sideways}}
    &    & \textbf{.903} & .854 & .893 & .900 & .867  & .891 & .893& .898& .855& .872  \\
    &     & \textbf{.884}   & .831 & .876 & .868 & .834  & .864 & .876 & .882& .793& .805    \\
    &       & \textbf{.938}   & .894 & .911 & .922 & .907  & .921& .931 & .934& .854& .870   \\
    &  & \textbf{.039}  & .060 & .043 & .047 & .057 & .047 & .043 & .040& .073& .068  \\ \hline
\multirow{4}{*}{\begin{sideways}\textit{DES} \cite{cheng2014depth}\end{sideways}}
    &  & \textbf{.934}   & .876 & .896 & .928 & .897  & .911 & .896& .918& .811& .911  \\
    &    & \textbf{.919}    & .844 & .868 & .902 & .867  & .897& .868 & .904& .724& .843    \\
    &     & \textbf{.967}     & .906 & .928 & .947 & .930  & .945& .928 & .953& .794& .910   \\
    &  & \textbf{.019}  & .035 & .026 & .024 & .033  & .024& .026 & .023& .065& .036  \\ \hline
\multirow{4}{*}{\begin{sideways}\textit{NLPR} \cite{peng2014rgbd}\end{sideways}}
    &  & \textbf{.920}   & .878 & .919 & .918 & .890  & .899 & .910 & .915& .850 & .883  \\
    &   & .891     & .846 & \textbf{.897} & .878 & .845  & .875 & .867 & .889 & .759 & .795    \\
    &      & .951    & .911 & \textbf{.953} & .941 & .924  & .937 & .933 & .951 & .841 & .883   \\
    &  & .025  & .039 & \textbf{.024} & .029 & .037 & .029  & .028 & .025 & .057 & .045 \\ \hline
\multirow{4}{*}{\begin{sideways}\textit{LFSD} \cite{li2014saliency}\end{sideways}}
    &  & \textbf{.864}   & .799 & .847 & .862 & .820  & .838 & .847 & .853 & .729& .823 \\
    &    & \textbf{.855}    & .791 & .838 & .841  & .802 & .833  & .838 & .848& .661& .779   \\
    &     & \textbf{.901}     & .829 & .879 & .885 & .865 & .875  & .879 & .891 & .720& .818  \\
    &  & \textbf{.066}  & .101 & .079 & .075 & .093 & .079  & .079 & .073  & .145& .108\\ \hline
\multirow{4}{*}{\begin{sideways}\textit{SIP} \cite{sip_dataset}\end{sideways}}
    &  & \textbf{.875}   & .846 & .867 & .870 & .851  & .859 & .867 & .865 & .810& .845 \\
    &    & \textbf{.867}    & .837 & .860 & .848 & .821 & .853  & .860 & .855 & .751& .795  \\
    &     & \textbf{.914}     & .884 & .908 & .901 & .893 & .905  & .908 & .908 & .816& .852 \\
    &  & \textbf{.051}  & .068 & .056  & .059 & .067 & .057  & .056 & .056& .094& .079 \\ \hline
\end{tabular}
    \vspace{-4mm}
\end{table} 

\noindent\textbf{Scale of Latent Space:}
We investigate the influence of the scale of the Gaussian latent space  in our network. In this paper, after parameter tuning, we find  works best. We show performance with  as \enquote{M1}. Performance of \enquote{M1} is worse than our reported results, which indicates that scale of the latent space is an important parameter in our framework. We further carried out more experiments with , and found relative stable predictions with . 


\noindent\textbf{Effect of DepthCorrectionNet:}
To illustrate the effectiveness of the proposed DepthCorrectionNet, we remove this branch and feed the concatenation of the RGB image and depth data to the SaliencyNet, shown as \enquote{M2}, which is worse than our method. On DES~\cite{cheng2014depth} dataset, we observe the proposed solution achieves around 4\% improvement on S-measure, E-measure and F-measure, which demonstrates the effectiveness of the depth correction net.


\noindent\textbf{Saliency Consencus Module:}
To mimic the saliency labeling process, we embed a saliency consensus module during test in our framework (as shown in Fig. \ref{fig:testing_overview}) to obtain the majority voting of the multiple predictions. We remove it from our framework and test the network performance by random sample from the latent PriorNet , and performance is shown in \enquote{M3}, which is the best compared with competing methods. While, with the saliency consensus module embedded, we achieve even better performance, which illustrates effectiveness of the saliency consencus module.

\noindent\textbf{VAE  CVAE:}
We use CVAE to model labeling variants, and a PosteriorNet is used to estimate parameters for the PriorNet.
To test how our model performs with prior of  as a standard normal distribution,
and the posterior of  as . 
VAE performance is shown as \enquote{M4}, which is comparable with SOTA RGB-D models. With the CVAE \cite{structure_output} based model proposed, we further boost performance of \enquote{M4}, which proves effectiveness of the our solution.

\noindent\textbf{Multi-head  CVAE:}
Multi-head models \cite{Rupprecht2016LearningIA} generate multiple predictions with different decoders and a shared encoder, and the loss function is always defined as the closest of the multiple predictions. We remove the LatentNet, and copy the decoder of the SaliencyNet multiple times to achieve multiple predictions (\enquote{M5} in this paper). We report performance in \enquote{M5} as mean of the multiple predictions. \enquote{M5} is better than SOTA models (\eg, DMRA) while there still exists gap between M-head based method (\enquote{M5}) and our CVAE based model (\ourmodel).

\noindent\textbf{Monte-Carlo Dropout  CVAE:}
Monte-Carlo Dropout \cite{kendall2015bayesian} uses dropout during the testing stage to introduce stochastic to the network. We follow \cite{kendall2015bayesian} to remove the LatentNet, and use dropout in the encoder and decoder of the SaliencyNet in the testing stage. We repeats five times of random dropout (dropout ratio = 0.1), and report the mean performance as \enquote{M6}. Similar to \enquote{M5}, \enquote{M6} also achieves the best performance comparing with SOTA models (\eg, CPFP and DMRA), while the proposed CVAE based model achieves even better performance.

\noindent\textbf{HHA  Depth:}
HHA \cite{gupta2014learning} is a widely used technique that encodes the depth data to three channels: \textbf{h}orizontal disparity, \textbf{h}eight above ground, and the \textbf{a}ngle the pixel’s local surface normal makes with the inferred gravity direction. HHA is widely used in RGB-D related dense prediction models \cite{Du_2019_CVPR, han2017cnns} to obtain better feature representation. To test if HHA also works in our scenario, we replace depth with HHA, and performance is shown in \enquote{M7}. We observe similar performance achieved with HHA instead of the raw depth data.

\noindent\textbf{New Label Generation:}
To produce diverse predictions, we follow \cite{hide_and_seek-iccv2017} and generate diverse annotations for
the training dataset. To illustrate the effectiveness of this strategy, we train with only the SaliencyNet to produce single channel saliency map with RGB-D image as input for simplicity.
\enquote{M8} and \enquote{M9} represent using the provided training dataset and augmented training data respectively. We observe performance improvement of \enquote{M9} compared with \enquote{M8},
which indicates effectiveness of the new label generation technique. 







\section{Conclusion}
\vspace{-5pt}
Inspired by human uncertainty in ground truth (GT) annotation, we proposed the first uncertainty network named \textit{\textbf{UC-Net}} for RGB-D saliency detection based on a conditional variational autoencoder.
Different from existing methods, which generally treat saliency detection as a point estimation problem, we propose to learn the distribution of saliency maps. 
Under our formulation, our model is able to generate multiple labels which have been discarded in the GT annotation generation process through saliency consensus. 
Quantitative and qualitative evaluations on six standard and challenging benchmark datasets demonstrated the superiority of our approach in learning the distribution of saliency maps.
In the future, we would like to extend our approach to other saliency detection problems (\eg, VSOD~\cite{fan2019shifting}, RGB SOD~\cite{fan2018salient,zhao2019egnet}, Co-SOD~\cite{fan2020taking}).
Furthermore, we plan to capture new datasets with multiple human annotations to further model the statistics of human uncertainty in interactive image segmentation~\cite{fClick20CVPR}, camouflaged object detection~\cite{fan2020Camouflage}, \etc.







\vspace{-5pt}
\small{\vspace{.1in}\noindent\textbf{Acknowledgments.}\quad
This research was supported in part by Natural  Science  Foundation  of  China  grants  (61871325, 61420106007, 61671387), the Australia Research Council Centre of Excellence for Robotics Vision (CE140100016), and the National Key Research and Development Program of China under Grant 2018AAA0102803. We thank all reviewers and Area Chairs for their constructive comments.}

{\small
\bibliographystyle{ieee_fullname}
\bibliography{RGBD_Saliency}
}

\end{document}
