\documentclass[12pt]{article}


\usepackage{xspace}
\usepackage{epsfig}
\usepackage{epsf}
\usepackage{latexsym}
\usepackage{amsmath}
\usepackage{times}

\usepackage{amssymb}
\usepackage{epsfig}
\usepackage{psfrag}


\setlength{\textwidth}{6.7in}
\setlength{\textheight}{8.5in}
\setlength{\topmargin}{-0.2in}
\setlength{\oddsidemargin}{-0.3in}
\setlength{\headsep}{0in}
 \renewcommand{\baselinestretch}{.95}
\setlength{\topsep}{0in}


\newcommand{\ignore}[1]{ }

\newcommand{\MST}{\mathit{MST}}
\newcommand{\OPT}{\mathit{OPT}}
\newcommand{\debt}{\mathit{debt}}
\newcommand{\opt}{\mathit{opt}}
\newcommand{\gainh}{\widehat{\mathit{gain}}}


\sloppy

\def\qed{ \ \vrule width.2cm height.2cm depth0cm\smallskip}

\def\proof{\noindent {\bf Proof. \ }}
\def\sketch{\noindent{\bf Proof sketch. \ }}
\def\Fproof{\noindent {\bf First proof. \ }}
\def\Sproof{\noindent {\bf Second proof. \ }}

\newtheorem{proposition}{Proposition}
\newtheorem{lemma}[proposition]{Lemma}
\newtheorem{fact}[proposition]{Fact}
\newtheorem{theorem}[proposition]{Theorem}
\newtheorem{corollary}[proposition]{Corollary}
\newtheorem{claim}[proposition]{Claim}
\newtheorem{definition}[proposition]{Definition}

\def\hq{{\hat{q}}}
\def\hQ{{\hat{Q}}}
\def\hS{{\hat{S}}}
\def\tT{{\tilde{T}}}
\def\tY{{\tilde{Y}}}
\def\be{{\bar{e}}}
\def\bY{{\bar{Y}}}
\def\hY{{\hat{Y}}}
\def\hR{{\hat{R}}}
\def\hT{{\hat{T}}}
\def\tR{{\tilde{R}}}
\def\tS{{\tilde{S}}}
\def\bB{{\bar{B}}}
\def\bQ{{\bar{Q}}}
\def\jb{{\bar{j}}}

\newcommand{\cD}{{\cal D}}
\newcommand{\cB}{{\cal B}}
\newcommand{\cM}{{\cal M}}
\newcommand{\cA}{{\cal A}}
\newcommand{\ZZ}{{\mathbb{Z}}}


\newcommand{\eps}{\epsilon}
\newcommand{\ib}{\bar{i}}
\newcommand{\ih}{\widehat{i}}
\newcommand{\nh}{\widehat{n}}
\newcommand{\yh}{\widehat{y}}
\newcommand{\Kh}{\widehat{K}}
\newcommand{\Qh}{\widehat{Q}}



\begin{document}


\title{1.85 Approximation for Min-Power Strong Connectivity}
\author{ {G. Calinescu}
\thanks{Department of Computer Science,
Illinois Institute of Technology, Chicago, IL 60616, USA. 
{\tt calinescu@iit.edu}. Research supported in part by NSF grants
CCF-0515088 and NeTS-0916743, and the ERC StG project PAAl no. 259515.
}}\date{}
\maketitle




\begin{abstract}
Given a directed
simple graph  and a cost function ,
the {\em power} of a vertex  in a directed spanning subgraph  is given by
, and corresponds to the
energy consumption required for wireless node  to transmit to all nodes
 with .
The {\em power} of  is given by .

Power Assignment seeks to minimize  while 
satisfies some connectivity constraint. In this paper, we assume
 is bidirected (for every directed edge , the opposite edge exists
and has the same cost), while  is required to be strongly connected.
This is the original power assignment problem introduced by Chen and Huang
in 1989,
who proved that a bidirected minimum spanning tree
has approximation ratio at most 2 (this is tight).
In Approx 2010, we introduced a Greedy approximation algorithm 
and claimed a ratio of .  Here we improve the analysis to ,
combining techniques from Robins-Zelikovsky (2000) for Steiner Tree,
and Caragiannis, Flammini, and Moscardelli (2007) for the broadcast version
of Power Assignment, together with a simple idea inspired by
Byrka, Grandoni, Rothvo\ss, and Sanit\`{a} (2010).

The proof also shows that a natural linear programming relaxation,
introduced by us in 2012, has the same  integrality gap.
\end{abstract}

\section{Introduction}

There has been a surge of research in Power Assignment problems since 2000
(among the earlier papers are
\cite{ramanathan00topology,WLBW01,HajiaghayiIM03})
This class of problems take as input a directed
simple graph  and a cost function .
The {\em power} of a vertex  in a directed spanning simple 
subgraph  of  is given by
, and corresponds to the
energy consumption required for wireless node  to transmit to all nodes
 with .
The {\em power} (or {\em total power})
of  is given by .

The study of the min-power power assignment 
was started by Chen and Huang \cite{CH89}, which consider, as we do,
the case when  is bidirected,
(that is,  if and only if , and if weighted, the two
edge have the same cost;
this case was sometimes called ``symmetric" or  ``undirected" in the literature)
while  is required to be strongly connected. 
We call this problem {\em Min-Power Strong Connectivity}.
We use with the same name both the (bi)directed and 
the undirected version of .
\cite{CH89} prove that the bidirected version
of a minimum (cost) spanning tree (MST)
of the input graph  has power at most twice the optimum,
and therefore the MST algorithm has approximation ratio at most 2.
This is known to be tight (see Section \ref{s_prel}).

We improve this to  
by combining techniques from Robins-Zelikovsky 
\cite{RZ00,RobinsZ05} for Steiner Tree, 
Caragiannis, Flammini, and Moscardelli \cite{CFM07}
for the broadcast version of symmetric Power Assignment
(assuming a bidirected  and a ``root"  is given,
 must contain a directed path from  to every vertex of ),
together with a simple idea inspired by
Byrka, Grandoni, Rothvo\ss, and Sanit\`{a} (2010).

Very restricted versions of Min-Power Strong Connectivity
have been proven NP-Hard \cite{KKKP00,CPS00,CK07}. Other than \cite{C10},
we are not aware
of better than a factor of 2 approximation except for \cite{CK07},
(where , for ; see also \cite{C10}),
\cite{CCPRS05} (where  is assumed to be a metric), and
the exact (dynamic programming) algorithms of
\cite{KKKP00} for the specific case
where each vertex of  maps to a point on a line, and  is an
increasing function of the Euclidean distance 
between the images of  and .
A related version, also NP-Hard, asks for  to be bidirected 
(also called ``undirected" or ``symmetric" in previous papers).
This problem is called Min-Power Symmetric Connectivity, 
and the best known ratio of  \cite{ACMPTZ06} is obtained with
techniques first applied to Steiner Tree;
when  one gets  with the same method
\cite{NY09}. In fact, many but not all power assignment 
algorithms use techniques from Steiner Tree variants
(or direct reduction to Steiner Tree variants; these connections 
to Steiner Tree are not obvious and cannot be easily explained), 
and in particular
Caragiannis et al \cite{CFM07} uses the relative greedy
heuristic of Zelikovsky \cite{Z96}. New interesting techniques were also
developed for power assignment problems, as in
\cite{KortsarzMNT08}, an improvement over \cite{HajiaghayiKMN07}.

The existing lower bound of the optimum, which we use,
is the cost of the minimum spanning tree of . Indeed 
(argument from \cite{CH89}), 
the optimum solution  contains an 
in-arborescence rooted at , for some , and then, 
for all ,  is at least
the cost of the directed edge connecting 
 to its parent in this in-arborescence, whose total cost 
is at least the cost of the minimum spanning tree of .

We also use a relative greedy method as in \cite{Z96,RZ00};
Robins-Zelikovsky \cite{RZ00} is rarely used as a technique, 
and not by only citing the ratio (improved by now in \cite{BGRS10}).
We use the natural structures  of \cite{CFM07} to improve over the
minimum spanning tree: these are {\em stars}, directed trees of height 1. 
Our second lower bound (improved over \cite{C10})
comes from ``covering"
the edges of the a spanning tree by the stars of the optimum solution.
With precise definitions later, we just mention that
an edge of a tree is covered by a star if it is on a path of the tree between
two vertices of the star.
A ``cheap" fractional covering can be easily obtained from either optimum
or the linear programming relaxation.
In our earlier work \cite{C10}, we used an integer cover which was
extremely hard to obtain. 
Using fractional covers (inspired by \cite{BGRS10})
is the only significant difference of this version versus \cite{C10}.
Also, interestingly, the 
submodularity of the covering function is only used implictly.

\section{Preliminaries}
\label{s_prel}

In directed graphs, we use {\em arc} to denote a directed edge.
In a directed graph ,
an {\em incoming arborescence}  rooted at  is a 
spanning subgraph  of 
such that the underlying undirected graph of  is a tree 
and every vertex of  other than  has exactly one outgoing arc in . 


Given an arc  , its {\em undirected version} is 
the undirected edge with endpoints  and .
Arcs  and  are {\em antiparallel}, and the antiparallel arcs
resulting from undirected edge  are  and ; if undirected edge
  has cost, then each of the two antiparallel arcs
resulting from undirected edge   have this cost.
We sometimes identify a spanning tree  with its set of edges.

An alternative definition of our problem (how it was originally posed) is:
we are given a simple undirected graph 
and a cost function .
A power assignment is a function , and it
induces a simple directed graph  on vertex set  given by
 being an arc of  if and only if
  and .
The problem is to minimize  subject to 
being strongly connected.
To see the equivalence of the definition, given directed spanning subgraph 
, define for each  the power assignment .

The following known example (see Figure \ref{f_2mst})
 shows that the ratio of 2 for the MST algorithm is tight.
Consider  points located on a single line such that the distance
between consecutive points alternates between 1 and  ,
and let the cost function  be the square of the Euclidean distance
Then the minimum spanning
tree MST connects consecutive neighbors and has power
.
On the other hand, the bidirected tree  with arcs connecting
each other node (see Figure \ref{f_2mst}(b)) has power equal
. When
 and , we obtain that
.  On the other hand 
(argument taken from \cite{CH89}), for any input graph,
the power of the bidirected minimum spanning tree  is at most

where  for an optimum solution  
(the last inequality is from the introduction).

\begin{figure}
\centerline{\psfig{figure=2mst.eps,width=4.7in}} 
\caption{\label{f_2mst}
Tight example for the performance ratio of the MST algorithm.
In both cases, the solution is bidirected and the undirected
version of the arcs of the solution is given by solid edges.
For each vertex, its power in the solution is written next to it.
(a) The MST-based power assignment needs total power .
(b) Optimum power assignment has total power
.
}
\end{figure}

The example above may give intuition on how Power Assignment 
(and even more specifically, Min-Power Symmetric Connectivity
- the variant when  must be bidirected mentioned in the introduction)
relates to the -restricted Steiner trees, with stars
(trees of height 1) taking the place of restricted components. 
Another example  from \cite{ACMPTZ06}, (see Figure \ref{asymm}, and the
following paragraph), shows how Min-Power Strong Connectivity 
differs from Min-Power Symmetric Connectivity, and may give intuition how
Min-Power Strong Connectivity relates to Travelling Salesman, and also
Min-Cost Strong Connectivity and Min-Cost Two-Edge Connectivity
(a two-edge-connected graph has an edge orientation that makes it strongly
connected  - see for example Chapter 2,
written by A. Frank, of \cite{GrahamGL95}).
However we cannot think of direct reductions either way, and,
as we mention in Conclusions, the methods we use
only apply to certain instances of 
Min-Cost Strong Connectivity and Min-Cost Two-Edge Connectivity. 


The power of a Min-Power Strong Connectivity optimum solution can be
almost half the power of a Min-Power Symmetric Connectivity optimum solution
for the same instance: we present a series of examples
illustrated in Figure \ref{asymm}.  The  vertices
are embedded in the plane in  groups of  points each.
Each group has two ``terminals"
(represented as thick circles in Figure \ref{asymm}),
and the  terminals are the corners of
a regular -gon with sides of length 1.
Each group has another  equally spaced points
(dashes in Figure \ref{asymm}) on the line segment between the two terminals.
The cost function  is the square of the Euclidean distance.
It is easy to see that a minimum power assignment
ensuring strong connectivity assigns a power of 1 to one thick
terminal in each group and a power of  to all other points
 in the group - the arcs going clockwise.
The total power then equals . For symmetric
connectivity it is necessary to assign power of 1 to all but
two of the thick points, and of  to the remaining points,
which results in 
total power .
Also, keep in mind that
the minimum spanning tree solution is a symmetric solution.

\begin{figure}
\centerline{\psfig{figure=asym1.eps,width=4.7in}}
\caption{\label{asymm}
Total power for Min-Power Strong Connectivity can be
half the total power for Min-Power Symmetric Connectivity.
In both cases, the solution is represented by solid segments
(arrows meaning not bidirected),
and next to each vertex its power is given.
(a) Minimum power assignment ensuring strong connectivity has total
power .
 (b) Minimum power assignment ensuring symmetric connectivity has 
total power ,
the solution is bidirected and the undirected
version of the arcs of the solution is given by solid segments.
}\end{figure}


\section{The Approximation Algorithm}
\label{s_weighted}

This section is dedicated to proving the main result of this paper:


\begin{theorem}
There exists a polynomial time algorithm
for Min-Power Strong Connectivity with approximation ratio .
\label{t_main}
\end{theorem}

The outline of the proof 
of Theorem \ref{t_main}, is as follows:
\begin{enumerate}
\item In the first subsection,
we present the greedy algorithm, preceded by necessary notation.

\item Then we establish (Lemma \ref{l_output_good}) that the algorithm's
output is strongly connected.

\item A second subsection gives the analysis.
It starts with the new lower bound, the fractional cover of tree
edges by stars.

\item Finally, Lemma \ref{l_zel} shows how the Greedy method of
\cite{RZ00} (applied with new parameters) combines the two lower bounds
(the one above, and the cost of the minimum spanning tree) 
to obtain the claimed approximation ratio.
\end{enumerate}

\subsection{The algorithm}

Our algorithm uses a greedy approach similar to \cite{Z96,RZ00,CFM07}.
Let  be the undirected minimum spanning tree of . 
The fact that  has minimum cost will be not further used,
except to note ,
where  for an optimum solution .
Let  be the bidirected version of .

For  and ,
let  be the directed star with center  containing
all the arcs  with ; note that  is the power of ,
also denoted by .
For a directed star , let  be its set of arcs and
 be its set of vertices. 


For given , let  be the set of edges  of 
such that there exist 
with  on the path from  to  in .
Let  be the set of arcs  of   
such that there exist 
with  on the directed path from  to  in ;
it is easy to verify
that the undirected version of  is .

For a collection  of directed stars , 
define  and
. We sometimes write  instead of
.
The function  is known to be monotone and submodular  
(see an example in \cite{Schrijver03}, pages 768-769).
For , recall that ,
where  is defined to be those arcs of  
for which the undirected version is not in .

The algorithm starts with  as the set of arcs, and 
adds directed stars to collection  (initially empty)
replacing arcs from  to greedily reduce the sum of costs  
of the arcs in  plus the sum of the powers of the stars in .
For intuition, we mention
that this sum is our upper bound on the power of the algorithm's output.
To simplify later proofs, the algorithm makes changes 
(adding directed stars and removing arcs from )
even if our sum stays the same. Assume below that . To be precise:


\begin{tabbing}
\ \ \ \= \ \ \= \ \ \= \  \ \= \ \ \  \\
Algorithm {\bf Greedy:} \\
\> ,  \\
\> While ( ) do\\
\> \>  \\
\> \>  \\
\> \>  \\
\> Output 
\end{tabbing}

\bigskip

Each of figures \ref{f_algo} and \ref{f_again} shows
 two iterations of the algorithm.
For intuition, we mention that this algorithm ``covers" undirected 
edges of the minimum spanning tree
by ``stars" and when implemented, it is a variant of
Chvatal's \cite{Chvatal79} greedy algorithm for Set Cover.
 

\begin{figure}[ht]
\psfrag{x}{\tiny{}}
\psfrag{y}{\tiny{}}
\psfrag{u}{\tiny{}}
\psfrag{v}{\tiny{}}
\begin{minipage}[b]{0.3\linewidth}
\centering
\includegraphics[scale=0.4]{algo_a.eps}
\end{minipage}
\begin{minipage}[b]{0.3\linewidth}
\centering
\includegraphics[scale=0.4]{algo_b.eps}
\end{minipage}
\begin{minipage}[b]{0.3\linewidth}
\centering
\includegraphics[scale=0.4]{algo_c.eps}
\end{minipage}
\caption{
Costs are not relevant here.  (a) Initial .  (b) Solid arcs give 
 after adding to 
 the star  (with thick solid arcs) centered at .
The arcs removed from  are dashed.
(c) Solid arcs give 
 after adding 
to  the star  (with thick solid arcs) centered at .
The arcs removed from  are dashed.
Note that the algorithm does not remove arc .
Once an arc from a pair of antiparallel arcs of  
is removed, the algorithm keeps the other arc, since
there are cases (as in Figure \ref{f_again})
 when not all arcs of  can be removed
while keeping strong connectivity, and benefiting from
removing arcs whose antiparallel arc has already been removed from 
(when possible)
destroys the ``submodularity" implictly 
needed in the approximation ratio proof.
}\label{f_algo}
\end{figure}

\begin{figure}[ht]
\psfrag{x}{\tiny{}}
\psfrag{y}{\tiny{}}
\psfrag{u}{\tiny{}}
\psfrag{v}{\tiny{}}
\begin{minipage}[b]{0.3\linewidth}
\centering
\includegraphics[scale=0.4]{again_a.eps}
\end{minipage}
\begin{minipage}[b]{0.3\linewidth}
\centering
\includegraphics[scale=0.4]{again_b.eps}
\end{minipage}
\begin{minipage}[b]{0.3\linewidth}
\centering
\includegraphics[scale=0.4]{again_c.eps}
\end{minipage}
\caption{
Costs are not relevant here.  (a) Initial .  
(b)  after adding to 
 the star  (with thick solid arcs) centered at ,
and removing arcs from .
(c) Solid arcs give 
 after adding 
to  the star  (with thick solid arcs) centered at .
The arcs removed from  are dashed.
Note that the algorithm cannot remove arc  even though
, since  and  become disconnected.
}\label{f_again}
\end{figure}

\begin{fact}
\label{obs}
Note that, unless , a star  always exists for which 
 and .
Indeed, as long as a pair of antiparallel arcs  and  are
in , 
we can pick as next star  the one
given by  being the tail of  and .
\end{fact}

Thus, as written, the algorithm can have iterations that do not change
the output, i.e. above the star  above could
have just the edge  and
be added to  while  is removed from . 

\begin{lemma}
The output of {\bf Greedy} is a spanning strongly connected subgraph of .
\label{l_output_good}
\end{lemma}
\proof
We prove the following invariant:

gives a spanning strongly connected subgraph whenever the {\bf while}
condition is checked by the algorithm. Moreover,
suppose we remove from  all edges for which both
antiparallel arcs appear in , splitting  in components
with vertex sets , for some range of .
We prove that
for every  and every , there exists a
directed  path  from  to   using only vertices of 
and arcs from .

\begin{figure}[bht]
\begin{center}\leavevmode \psfrag{x}{}
\psfrag{y0}{}
\psfrag{y1}{}
\psfrag{y2}{}
\psfrag{z0}{}
\psfrag{z1}{}
\psfrag{z2}{}
\psfrag{z3}{}
\psfrag{z4}{}
\psfrag{u}{}
\psfrag{u1}{}
\scalebox{.7}{
\includegraphics{invariant.eps}
}\end{center}
\caption{
Rounded rectangles show the components , dashed before 
is added to  (the split of  by ) and solid afterward 
(the split of  by ).
Arcs of  crossing from one component to another are given by thin arcs,
 by the four thick arcs ,
and the dashed arcs are those removed from  when  is added.
}\label{f_invariant}
\end{figure}

Proving the invariant holds is done as always 
by induction on the number of iterations.
The invariant is true before the first iteration, when each 
has just one vertex, so consider the moment when
a star  is added to . Figure \ref{f_invariant} 
may provide intuition.
We add arcs , for ,
while removing from  (and from  )
  arcs  if  and there is some  such that
 is on the directed simple path from  to  in .
The same effect is obtained if we do this change for 
each  one after the other,
instead of all such  at the same time.


Let  be the simple path in  from  to , and let
, for ,  
be, in order, the arcs of  on  such that also .
Thus the change to  consists of adding the arc  and
removing all arcs ; note that if  no arc is removed
and our induction step is complete. 
Let  and
.
We need to show that  and  satisfy the conditions
from the induction hypothesis.

Let us split  into components by removing all
the undirected edges  with both antiparallel arcs   and  in ; in
particular all , for ,
resulting in components .

By induction,   contains the following directed paths:
 from  to ,  from  to , \ldots, 
 from  to , and  from  to ,
and each of these paths stays in the same component  of 
the split of  by  - as in the previous paragraph.
Thus none of these paths uses  or .
Putting together these paths, the arcs , for ,
and the arc , we have a directed cycle 
containing none of the arcs  , for .
Any arc removed can be replaced, when discussing connectivity,
with a path around the cycle , and so  is strongly connected,
as required.

We now split  into components by removing all
the undirected edges  with both  and  in ,
obtaining components . Note that none of , , from above, has an arc
with endpoints in two distinct components  (as  is the union
of several ). 
As all the edges on the path from  to  in 
do not have anymore both antiparallel arcs in ,
all the vertices on this path including  and all 
are in the same component  of the split of  by .
Thus all the arcs of  have their endpoints in the
same component of the split of  by .

We prove that
for every  and every , there exists a
directed  path  from  to   using only vertices of 
and arcs from . 
First, let us describe a path  from  to  using only arcs of :
find the path from  to  in ,  and let 
, for ,  
be, in order, the arcs of  on  such that also .
When ,  for some 
(same component of the split of  by ) and by
induction, a path  from  to  exists in  using only vertices
inside . We pick , and indeed  only uses arcs of 
since the arcs of  (same set as )
cross from one component to another of the split of  by .
Assume now . Notice that all unordered pairs 
belong in the set of unordered pairs  on the simple path from
 to  mentioned earlier, or else we cannot have that  and 
belong to the same  of the split of  by .
Also, by induction,   contains the following directed paths:
 from  to ,  from  to , \ldots, 
 from  to , and  from  to ,
and each of these paths stays in the same component of the split of  by .
Thus none of these paths uses  or .
Next, obtain  by replacing in ,
if necessary, arcs of  (same set as )
by arcs of , staying, as shown in the previous paragraph,
in the same component of  split by .
Note that  indeed uses only vertices of .
This completes the induction step.  \qed

\subsection{Approximation ratio analysis}


For a collection  of directed stars , 
define , 
the total power used by the stars in .

\begin{lemma}
Let  be a an arbitrary collection of stars,
and  be an arbitrary spanning tree.
There exist non-negative coefficients  (over the collection of
all possible stars ) such that
 
and .
\label{l_fractional}
\end{lemma}



\proof
We assume that , or else  for all  will do.
Assign  for  every star of , and  otherwise.
Therefore .

Recall that 
 and
let . 
If we remove  from , we create two subtrees
 and , where  and  are the endpoints of .
, being strongly connected,
has at least one star  with the center in  and one of 
its other vertices in . Then .
Similarly,   
has at least one star  with the center in  and one of 
its other vertices in . Then .
Note that  (centers in disjoint vertex sets).

We have:

where the inequation is given by the two distinct stars  and 
described above for every edge .
This completes the proof.
\qed

The example from Figure \ref{f_2mst} shows that a constant
better than  is not possible above:
with  being empty, we have  circa , and for each star 
of optimum,  is circa  and  circa .


Now we need the following lemma, whose proof is obtained from 
Robins-Zelikovsky as presented in \cite{GroplHNP01} by changing 
what quantities represent and some parameters,
together with a ``fractional cover" idea from 
the journal submission version of \cite{BGRS10}.

\begin{lemma}
Assuming that for the minimum spanning tree ,
and for any collection of stars ,
there exist non-negative coefficients  such that
 
and ,
where  is the power of the optimum solution and , 
the algorithms' output has power at most  
where . 
\label{l_zel}
\end{lemma}
\proof
First, if , then before any improvement 
we have a solution of cost at most  and .
Thus in the following we assume 
(the first inequality is due to  being a minimum spanning tree).

Note that at the end of the algorithm, 
 contains exactly one of the two antiparallel
arcs for each edge of . Then,
for the final collection of stars , the output  satisfies

as it follows by summation over  from

which holds for every vertex  
(recall that ).
 


Let  be the stars picked by our algorithm
and  let , for  be the collection of the first
 stars; also let for convenience  be the empty collection.
For , let , 
and let .
Note then that for all , since 
,
we have .

If , then  and Equation \ref{e_p_i} below holds. Otherwise,
the greedy choice of the algorithm and the assumptions of the theorem
for  give:


Define the function  by
 for ,
and  for .
Then from Equation \ref{e_p_i} and Fact \ref{obs}
(that ), we obtain:


\begin{figure}[bht]
\begin{center}\leavevmode \psfrag{f1}{}
\psfrag{f2}{}
\psfrag{fq}{}
\psfrag{fi}{}
\psfrag{fi+}{}
\psfrag{fi++}{}
\psfrag{cT}{}
\psfrag{cT-a}{}
\scalebox{.4}{
\includegraphics{integral.eps}
}\end{center}
\caption{
The function  is given by the solid curve.
 is the shaded area, in rectangles of width 
and height 
.
In this particular picture,  and therefore
the integral is circa .
}\label{f_integral}
\end{figure}

Therefore (see Figure \ref{f_integral}):



Using this and  and
Equation \ref{e_opt} (recall that ), we obtain
that the power of the output is at most

finishing the proof.  
\qed


Based on Lemmas \ref{l_output_good}  and
\ref{l_fractional}, Theorem \ref{t_main} follows immediately from
the fact that  makes .
Note also that  implies , which
follows from , 
which is equivalent to , a fact that holds
for all . 
\cite{C10} had .


\section{Linear Programming Relaxation}

While not improving the approximation ratio of Greedy,
this may allow for further LP-based algorithm.
The following natural Integer Linear Program is called {\em IP2} as in
\cite{CQ12}. We adapted the notation,
and have variables  for every arc .
The idea is that  being 1 ``represents"  that  is 
star of the optimum solution.
We say that star ,
for 
iff  and .





{\bf LP2} is the linear relaxation of IP2; that is,
the linear program given by exactly the same constraints except
the last one. 
LP2 has exponentially many ``cut"
constraints, but when we replace them by ``flows", 
 we have  non-zero entries in the matrix, as shown in \cite{CQ12}.
Thus it can be solved in polynomial-time.

Let  be the optimum of the linear program LP2 for a given instance.
Then clearly  and \cite{CQ12} proves that 
.
We do better here. First, Lemma \ref{l_fractional} has a straightforward
adaptation:

\begin{lemma}
Let  be a an arbitrary collection of stars,
and  be an arbitrary spanning tree.
There exist non-negative coefficients  (over the collection of
all possible stars ) such that
 
and .
\label{l_fractional_lp}
\end{lemma}

\proof
We assume that , or else  for all  will do.
Let  (for all ) be a an optimum solution of LP2.
Assign  for all ; therefore
.

Recall that 
 and
let . 
If we remove  from , we create two subtrees
 and , where  and  are the endpoints of .
Constraint \ref{c_ip2_cross} gives

and

Note that  and  
are disjoint sets since a star in the first set has
its center in , while
a star in the second set has its center in .
Moreover, any star 
has . Thus

 
From now on we coppied from the previous proof:

where the inequation is from (\ref{e_frac_cover}).
This completes the proof.
\qed

The example from Figure \ref{f_2mst} again shows that a constant
smaller than  is not possible in Lemma \ref{l_fractional_lp}:
with  being empty, we that for each star 
with , .
Thus, if , then  
,
where we used that the example can be made to have 
, for any .
As  and  can be made arbitraly small,
we can see that indeed a constant smaller than  is not possible.


It is also proven in Section II of \cite{CQ12} that ;
then Lemma \ref{l_zel} goes through with  instead of .
Therefore we conclude that the output of Greedy is at most .

\section{Conclusions}

This work greatly simplifies and at the same time improves our 
earlier work \cite{C10}.
Instead of Greedy we could have used the 
the iterative randomized rounding of
 Byrka et. al \cite{BGRS10} for Steiner Tree,
with the same approximation ratio.
However, we do not see further improvements coming from
using their full range of techniques,
 since we do not see the equivalent of the concept of ``loss"
used explicitly by \cite{RZ00} and implicitly by  \cite{BGRS10}.

\section{Acknowledgments}

We would like to thank Alexander Zelikovsky for a streamlined explanation
of his algorithms,
and to Gianpiero Monaco and Anna Zych for a group reading of \cite{BGRS10}
at the University of Warsaw.


\begin{thebibliography}{10}

\bibitem{ACMPTZ06}
E.~Althaus, G.~Calinescu, I.~Mandoiu, S.~Prasad, N.Tchervenski, and
  A.~Zelikovsky.
\newblock Power efficient range assignment for symmetric connectivity in static
  ad hoc wireless networks.
\newblock {\em Wireless Networks}, 12(3):287--299, 2006.

\bibitem{CCPRS05}
Christoph Amb\"{u}hl, Andrea E.~F. Clementi, Paolo Penna, Gianluca Rossi, and
  Riccardo Silvestri.
\newblock On the approximability of the range assignment problem on radio
  networks in presence of selfish agents.
\newblock {\em Theor. Comput. Sci.}, 343(1-2):27--41, 2005.

\bibitem{BGRS10}
Jaroslaw Byrka, Fabrizio Grandoni, Thomas Rothvo\ss, and Laura Sanit\`{a}.
\newblock An improved {LP}-based approximation for {Steiner} tree.
\newblock In {\em STOC '10: Proceedings of the 42nd ACM symposium on Theory of
  computing}, pages 583--592, New York, NY, USA, 2010. ACM.

\bibitem{C10}
G.~Calinescu.
\newblock Min-power strong connectivity.
\newblock In M.\ Serna, K.\ Jansen, and J.\ Rolin, editors, {\em Proceedings of
  the International Workshop on Approximation Algorithms for Combinatorial
  Optimization}, number 6302 in Lecture Notes in Computer Science, pages
  67--80. Springer, 2010.

\bibitem{CQ12}
G.~Calinescu and K.~Qiao.
\newblock Asymetric topology control: exact solutions and fast approximations.
\newblock In {\em Infocom}, 2012.

\bibitem{CFM07}
Ioannis Caragiannis, Michele Flammini, and Luca Moscardelli.
\newblock {An Exponential Improvement on the MST Heuristic for Minimum Energy
  Broadcasting in Ad Hoc Wireless Networks}.
\newblock In {\em ICALP}, pages 447--458, 2007.

\bibitem{CK07}
Paz Carmi and Matthew~J. Katz.
\newblock Power assignment in radio networks with two power levels.
\newblock {\em Algorithmica}, 47(2):183--201, 2007.

\bibitem{CH89}
W.T. Chen and N.F. Huang.
\newblock The strongly connecting problem on multihop packet radio networks.
\newblock {\em IEEE Transactions on Communications}, 37(3):293--295, 1989.

\bibitem{Chvatal79}
V.\ Chv\'atal.
\newblock A greedy heuristic for the set-covering problem.
\newblock {\em Mathematics of Operations Research}, 4(3):233--235, 1979.

\bibitem{CPS00}
A.E.F. Clementi, P.~Penna, and R.~Silvestri.
\newblock On the power assignment problem in radio networks.
\newblock {\em Electronic Colloquium on Computational Complexity (ECCC)},
  (054), 2000.

\bibitem{GrahamGL95}
R.L.\ Graham, M.\ Gr{\"o}tschel, and L.\ Lov{\'a}sz, editors.
\newblock {\em Handbook of combinatorics. {V}ol. 1,\ 2}.
\newblock Elsevier Science B.V., Amsterdam, 1995.

\bibitem{GroplHNP01}
C.\ Gr\"opl, S.\ Hougardy, T.\ Nierhoff, and H.J.\ Pr\"omel.
\newblock Approximation algorithms for the {S}teiner tree problem in graphs.
\newblock In D.-Z. Du and X.~Cheng, editors, {\em Steiner Trees in Industries},
  pages 235--279. Kluwer Academic Publishers, Dordrecht, 2001.

\bibitem{HajiaghayiIM03}
Mohammad~Taghi Hajiaghayi, Nicole Immorlica, and Vahab~S. Mirrokni.
\newblock Power optimization in fault-tolerant topology control algorithms for
  wireless multi-hop networks.
\newblock In {\em MOBICOM}, pages 300--312, 2003.

\bibitem{HajiaghayiKMN07}
Mohammad~Taghi Hajiaghayi, Guy Kortsarz, Vahab~S. Mirrokni, and Zeev Nutov.
\newblock Power optimization for connectivity problems.
\newblock {\em Math. Program.}, 110(1):195--208, 2007.

\bibitem{KKKP00}
L.~M. Kirousis, E.~Kranakis, D.~Krizanc, and A.~Pelc.
\newblock Power consumption in packet radio networks.
\newblock {\em Theoretical Computer Science}, 243:289--305, 2000.

\bibitem{KortsarzMNT08}
Guy Kortsarz, Vahab~S. Mirrokni, Zeev Nutov, and Elena Tsanko.
\newblock Approximating minimum-power degree and connectivity problems.
\newblock In {\em LATIN}, pages 423--435, 2008.

\bibitem{NY09}
Zeev Nutov and Ariel Yaroshevitch.
\newblock Wireless network design via 3-decompositions.
\newblock {\em Inf. Process. Lett.}, 109(19):1136--1140, 2009.

\bibitem{ramanathan00topology}
Ram Ramanathan and Regina Rosales-Hain.
\newblock Topology control of multihop wireless networks using transmit power
  adjustment.
\newblock In {\em {INFOCOM} (2)}, pages 404--413, 2000.

\bibitem{RZ00}
G.~Robins and A.~Zelikovsky.
\newblock Improved {Steiner} tree approximation in graphs.
\newblock In {\em ACM-SIAM Symposium on Discrete Algorithms}, pages 770--779,
  2000.

\bibitem{RobinsZ05}
G.\ Robins and A.\ Zelikovsky.
\newblock Tighter bounds for graph {S}teiner tree approximation.
\newblock {\em SIAM Journal of Discrete Mathematics}, 19(1):122--134, 2005.

\bibitem{Schrijver03}
A.~Schrijver.
\newblock {\em Combinatorial Optimization}.
\newblock Springer, 2003.

\bibitem{WLBW01}
Roger Wattenhofer, Li~Li, Paramvir Bahl, and Yi-Min Wang.
\newblock Distributed topology control for wireless multihop ad-hoc networks.
\newblock In {\em IEEE INFOCOM'01}, 2001.

\bibitem{Z96}
A.~Zelikovsky.
\newblock Better approximation bounds for the network and {Euclidean} {Steiner}
  tree problems.
\newblock Technical Report CS-96-06, Department of Computer Science, University
  of Virginia, 1996.

\end{thebibliography}
\end{document}
