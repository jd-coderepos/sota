\subsection{Properties of the Algorithm}
\label{subsec:alg_props}
\emph{AbstractRepair} is \emph{well-defined}~\cite{BGS07}, in the 
sense that the algorithm always proceeds and eventually returns a result  or FAILURE such that , for any input ,  and , with .  Moreover, the algorithm steps are well-ordered, as opposed to existing concrete model repair solutions~\cite{CR11,ZD08} that entail nondeterministic behavior.  

\subsubsection{Soundness}
\label{subsubsec:alg_soundness}

\begin{lem}
\label{theor:sound_help}
Let a KMTS , a CTL formula  with  for some  of , and a set  with  for all 
.  If  returns a KMTS , then  and  for all .
\end{lem}
\begin{proof}
We use structural induction on .  For brevity, we write  
to denote that , 
for all .  

\paragraph{Base Case: }
\begin{itemize}
\item if , the lemma is trivially true, because     
\item if , then  
returns FAILURE at line 2 of Algorithm~\ref{alg:main} and the lemma is also trivially true. 
\item if ,  is called at line 4 of Algorithm~\ref{alg:main} and an  is computed 
at line 1 of Algorithm~\ref{alg:ATOMIC}.  Since  in ,  
from 3-valued semantics of CTL over KMTSs we have .  Algorithm~\ref{alg:ATOMIC} returns  at line 3, if and only if  and the lemma is true.     
\end{itemize}

\paragraph{Induction Hypothesis:} For CTL formulae , the lemma is true.  Thus, for  (resp. ), if  returns a KMTS , then  and .    

\paragraph{Inductive Step:}
\begin{itemize}
\item if , then  calls  at line 8 of Algorithm~\ref{alg:main}.     From the induction hypothesis, if a KMTS  is returned by  at line 1 of Algorithm~\ref{alg:OR} and a KMTS  is returned by  respectively, then ,  and , .   returns at line 8 of Algorithm~\ref{alg:main} the KMTS , which can be either  or .  Therefore,  or  and  in both cases.  From 3-valued semantics of CTL,  and the lemma is true. 

\item if , then 
 calls 
 at line 6 of 
Algorithm~\ref{alg:main}.  From the induction hypothesis, if at line 1 of Algorithm~\ref{alg:AND}  returns a KMTS , then 
 and .  Consequently, , where .  At line 7, if  returns a KMTS , then from the induction hypothesis  and .

In the same manner, if the calls at lines 2 and 12 of Algorithm~\ref{alg:AND} return the KMTSs  and , then from the induction hypothesis ,  and , 
 with .

The KMTS  at line 6 of Algorithm~\ref{alg:main} can be either  or  and therefore, , 
 and .  From 3-valued semantics of CTL it holds that  and the lemma is true.  

\item if ,  calls 
 at line 10 of 
Algorithm~\ref{alg:main}.    

If a KMTS  is returned at line 5 of Algorithm~\ref{alg:EX}, there is a
state  with  such that 
 and .
From 3-valued semantics of CTL, we conclude that .

If a  is returned at line 11, there is  such that 
and  from the induction hypothesis, since .  From 3-valued semantics of CTL, we conclude that .   

If a  is returned at line 18, a must transition  to a new state has been added and .  Then, from the induction hypothesis ,  and from 3-valued semantics of CTL, we also conclude that . 



























\item if ,  calls 
 at line 10 of 
Algorithm~\ref{alg:main}.  If  and  returns 
a KMTS  at line 2 of Algorithm~\ref{alg:AG}, then from the induction 
hypothesis  and 
.  Otherwise,  and  also hold true.    

If Algorithm~\ref{alg:AG} returns a  at line 16, then  and  is the result of successive  calls with  and , for
all may-reachable states  from  
such that .     
From the induction hypothesis,  and 
 for all such  and from 3-valued semantics of CTL we conclude that .   
  


























\end{itemize}

We prove the lemma for all other cases in a similar manner.  
\end{proof}

\begin{thm}[Soundness]
\label{theor:sound}
Let a KMTS , a CTL formula  with 
, for some  of 
.  If  returns a 
KMTS , then .  
\end{thm}
\begin{proof}
We use structural induction on  and Lemma~\ref{theor:sound_help}
in the inductive step for .  

\paragraph{Base Case:}
\begin{itemize}
\item if , Theorem~\ref{theor:sound} is trivially true, because 
.     
\item if , then  
returns FAILURE at line 2 of Algorithm~\ref{alg:main} and the theorem is also 
trivially true.    
\item if ,  
is called at line 4 of Algorithm~\ref{alg:main} and an  is computed at line 1.  Because of the fact that 
 in , from 3-valued semantics 
of CTL over KMTSs we have .    
Algorithm~\ref{alg:ATOMIC} returns  at line 3 because  is empty, 
and the theorem is true.    
\end{itemize}

\paragraph{Induction Hypothesis:}
For CTL formulae , , the theorem is true.  Thus, for  
(resp. ), if  returns a 
KMTS , then .      

\paragraph{Inductive Step:}
\begin{itemize}
\item if , then  
calls  at 
line 8 of Algorithm~\ref{alg:main}.    

From the induction hypothesis, if 
returns a KMTS  at line 1 of Algorithm~\ref{alg:OR} 
and  returns a KMTS  
respectively, then  and 
.  
 
returns at line 8 of Algorithm~\ref{alg:main} the KMTS , 
which can be either  or .  Therefore, 
 or 
.  
From 3-valued semantics of CTL, 
 and the theorem is true. 

\item if , then 
 calls 
 
at line 6 of Algorithm~\ref{alg:main}.  From the induction 
hypothesis, if at line 1 of Algorithm~\ref{alg:AND} 
 returns a 
KMTS , then .  
Consequently, , where 
.  
At line 7, if  
returns a KMTS , then from Lemma~\ref{theor:sound_help} 
 and .

Likewise, if the calls at lines 2 and 12 of Algorithm~\ref{alg:AND} 
return the KMTSs  and , then from the induction hypothesis 
 and from Lemma~\ref{theor:sound_help} 
,  
with .

The KMTS  at line 7 of Algorithm~\ref{alg:main} 
can be either  or  and 
therefore,  and  
.  From 3-valued semantics 
of CTL it holds that  
and the lemma is true. 

\item if ,  calls 
 at line 10 of 
Algorithm~\ref{alg:main}.  

If a KMTS  is returned at line 5 of Algorithm~\ref{alg:EX}, 
there is a state  with  
such that .  
From 3-valued semantics of CTL, we conclude that 
.

If a  is returned at line 11, there is 
 such that 
 from the induction hypothesis, 
since .  
From 3-valued semantics of CTL, we conclude that 
.   

If a  is returned at line 18, a must transition 
 to a new state has been added and .  
Then, from the induction hypothesis 
 and from 3-valued semantics of CTL, we also conclude that 
. 



























\item if ,  
calls  at line 10 of 
Algorithm~\ref{alg:main}.  If  and 
 returns a KMTS 
 at line 2 of Algorithm~\ref{alg:AG}, then from the induction 
hypothesis .  Otherwise, 
 and , 
 also hold true.    

If Algorithm~\ref{alg:AG} returns a  at line 16, 
this KMTS is the result of successive calls of  with  and , for
all may-reachable states  from  
such that .     
From the induction hypothesis,  
for all such  and from 3-valued semantics of CTL we conclude that 
.   
  
We prove the theorem for all other cases in the same way.  



























\end{itemize} 
\end{proof}
Theorem~\ref{theor:sound} shows that \emph{AbstractRepair} 
is \emph{sound} in the sense that if it returns a KMTS 
, then  satisfies 
property .  In this case, from the definitions of the basic 
repair operations, it follows that one or more KSs can be 
obtained for which  holds true.    



\subsubsection{Semi-completeness}
\label{subsubsec:alg_completeness}

\begin{defi}[\emph{mr}-CTL]
Given a set  of atomic propositions, we define the syntax of 
a CTL fragment inductively via a Backus Naur Form:  

where  ranges over .  
\end{defi}

\emph{mr}-CTL includes most of the CTL formulae apart from those with nested 
path quantifiers or conjunction.

\begin{thm}[Completeness]
\label{theor:complete}
Given a KMTS , an \textit{mr}-CTL formula  with 
, for some  
of , if there exists a KMTS 
 over the same set  of atomic propositions with 
, 
 returns a 
KMTS  such that 
.    
\end{thm}
\begin{proof}
We prove the theorem using structural induction on .

\paragraph{Base Case:}
\begin{itemize}
\item if , Theorem~\ref{theor:complete} is trivially true, 
because for any KMTS  it holds that 
.    
\item if , then the theorem is trivially true, because 
there does not exist a KMTS  such that 
.   
\item if , there is a KMTS 
with  and therefore .  
Algorithm~\ref{alg:main} calls  
at line 4 and an  
is computed at line 1 of Algorithm~\ref{alg:ATOMIC}.  Since  is empty,  is returned at line 3 and  from 3-valued semantics of CTL.  Therefore, the theorem is true.  
\end{itemize}

\paragraph{Induction Hypothesis:}
For \emph{mr}-CTL formulae , , the theorem is true.  
Thus, for  (resp. ), 
if there is a KMTS  over the same set  of atomic propositions with 
, 
 returns a KMTS 
 such that .  

\paragraph{Inductive Step:}
\begin{itemize}
\item if , from the 3-valued semantics of CTL a 
KMTS that satisfies  exists if and only if there is a KMTS satisfying any of the , .
From the induction hypothesis,
if there is a KMTS  with , 
at line 1 of Algorithm~\ref{alg:OR} returns a KMTS  such that . Respectively,  at line 2 of Algorithm~\ref{alg:OR} can return a KMTS  with . In any case, if either  or  exists, for the KMTS  that is returned at line 13 of Algorithm~\ref{alg:OR} we have  or  and therefore .   

\item if , from the 3-valued semantics of CTL a KMTS that satisfies  at  exists if and only if there is KMTS satisfying
 at some direct must-successor of .

If in the KMTS  there is a state  with , then the new KMTS  is computed at line 3 of Algorithm~\ref{alg:EX}.  Since  is empty  is returned at line 5 and .    

Otherwise, if there is a direct must-successor  of ,  is called at line 8.  From the induction hypothesis, if there is a KMTS  with , then a KMTS  is computed such that 
 and therefore the theorem is true.     

If there are no must-successors of , a new state  is added 
and subsequently connected with a must-transition from .   is then
called for  and  as previously and the theorem holds also true.  
 
\item if , from the 3-valued semantics of CTL a KMTS that satisfies  at  exists, if and only if there is KMTS satisfying  at  and at each may-reachable state from .  

 is called at line 2 of Algorithm~\ref{alg:AG} and from the induction hypothesis if there is KMTS  with , then a KMTS  is computed such that 
.   is subsequently called for  and for all may-reachable  from  with  one-by-one.  
From the induction hypothesis, if there is KMTS  that satisfies  at each such , then all  satisfy  at  and the theorem holds true.

We prove the theorem for all other cases in the same way.    

\end{itemize} 
\end{proof}
Theorem~\ref{theor:complete} shows that \emph{AbstractRepair} 
is \emph{semi-complete} with respect to full CTL: if there is a KMTS that satisfies a \emph{mr}-CTL formula , then the algorithm finds one such KMTS.  