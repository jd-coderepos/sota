\documentclass[11pt]{article}
\usepackage{fullpage}
\usepackage[utf8]{inputenc}
\usepackage[final,pdfborder={0 0 0}]{hyperref}
\usepackage{upgreek}
\usepackage{cite}
\usepackage{amsmath,amsthm,amssymb,nicefrac,graphicx,lastpage,array}
\usepackage{multirow}
\usepackage{subfigure}

\usepackage[inline,nomargin]{fixme}
\fxsetface{inline}{\bfseries\color{red}}

\usepackage{tikz}
\usetikzlibrary{decorations.pathreplacing}
\usetikzlibrary{decorations.pathmorphing}
\usetikzlibrary{calc}
\usetikzlibrary{shapes.geometric}
\usetikzlibrary{shapes.misc}
\usetikzlibrary{positioning}
\usetikzlibrary{patterns}
\usetikzlibrary{fit}

\usetikzlibrary{intersections}

\pgfdeclarelayer{background}
\pgfdeclarelayer{foreground}
\pgfsetlayers{background,main,foreground}

\tikzstyle{every picture}+=[remember picture]
\tikzstyle{defnode}=[solid,thin,draw,circle,minimum size=4pt,inner sep=0pt]
\tikzstyle{edgelab}=[draw=none,fill=white,font=\scriptsize,inner sep=0.5pt, text height=4pt, text depth=0pt,minimum size=5pt]
\tikzstyle{lightedge}=[draw,thin,dotted,semithick]
\tikzstyle{heavyedge}=[draw,solid,very thick]

\usepackage[noend]{algorithmic}
\renewcommand{\algorithmiccomment}[1]{//#1}
\usepackage{algorithm}

\newtheorem{theorem}{Theorem}
\newtheorem{lemma}{Lemma}
\newtheorem{corollary}{Corollary}
\newtheorem{example}{Example}

\def\algorithmautorefname{Algorithm}\def\equationautorefname{Equation}\def\exampleautorefname{Example}\def\definitionautorefname{Definition}\def\propertyautorefname{Property}\def\lemmaautorefname{Lemma}\def\theoremautorefname{Theorem}\def\corollaryautorefname{Corollary}\def\figureautorefname{Figure}\def\subfigureautorefname{Figure}\def\Itemautorefname{Item}\def\tableautorefname{Table}\def\subtableautorefname{Table}\def\sectionautorefname{Section}\def\subsectionautorefname{Section}\def\subsubsectionautorefname{Section}\def\chapterautorefname{Chapter}\def\partautorefname{Part}

\renewcommand{\epsilon}{\upvarepsilon}

\newcommand{\suffrel}{\sqsubseteq_\text{suff}}
\newcommand{\prefrel}{\sqsubseteq_\text{pref}}
\newcommand{\prefrelprop}{\sqsubset_\text{pref}}

\newcommand{\pref}{\operatorname{pref}}
\newcommand{\suff}{\operatorname{suff}}
\newcommand{\maxpref}{\operatorname{maxpref}}
\newcommand{\weight}{\operatorname{weight}}
\newcommand{\height}{\operatorname{height}}
\newcommand{\str}{\operatorname{string}}
\newcommand{\lab}{\operatorname{label}}
\newcommand{\strdepth}{\operatorname{strdepth}}
\newcommand{\strheight}{\operatorname{strheight}}
\newcommand{\lightdepth}{\operatorname{lightdepth}}
\newcommand{\lightheight}{\operatorname{lightheight}}
\newcommand{\minlightheight}{\operatorname{minlightheight}}
\newcommand{\maxlightheight}{\operatorname{maxlightheight}}
\newcommand{\lightstrings}{\operatorname{lightstrings}}
\newcommand{\lbl}{\operatorname{lab}}
\newcommand{\hos}{\operatorname{hos}}
\newcommand{\ind}{\operatorname{pos}}


\newcommand{\rootvertex}{\operatorname{root}}
\newcommand{\leafvertex}{\operatorname{leaf}}
\newcommand{\nextvertex}{\operatorname{next}}

\newcommand{\rightleaf}{\operatorname{rightleaf}}
\newcommand{\leftleaf}{\operatorname{leftleaf}}

\newcommand{\argmax}{\operatornamewithlimits{argmax}}
\newcommand{\up}{\operatorname{\textsc{up}}}
\newcommand{\lcp}{\operatorname{\textsc{lcp}}}
\newcommand{\nca}{\operatorname{\textsc{nca}}}
\newcommand{\ins}{\operatorname{\textsc{insert}}}
\newcommand{\pred}{\operatorname{\textsc{pred}}}
\renewcommand{\succ}{\operatorname{\textsc{succ}}}

\newcommand{\order}{\operatorname{order}}
\newcommand{\orderset}{\operatorname{orderset}}
\newcommand{\lps}{\operatorname{lps}}
\newcommand{\wildcard}{\ast}
\newcommand{\gap}[2]{\mbox{}}

\title{String Indexing for Patterns with Wildcards\thanks{Preliminary version appeared in \emph{Proceedings of the 13th Scandinavian Symposium and Workshops on Algorithm Theory}. Lecture Notes in Computer Science, vol. 7357, pp. 283--294, Springer 2012.}}
\author{Philip Bille ~~~ Inge Li Gørtz\thanks{Supported by a grant from the Danish Council for Independent Research  Natural Sciences} ~~~ Hjalte Wedel Vildhøj ~~~ Søren Vind\
p_i = t\left [l+i+\sum_{r=0}^{i-1} |p_r|, l+i-1+\sum_{r=0}^{i} |p_r| \right ] \quad \text{for} \quad  i=0,1,\ldots,j \; ,

\pref_i(S) &= \{ \pref_i(x) \mid x \in S\} \qquad &\suff_i(S) &= \{ \suff_i(x) \mid x \in S\} \\
\pref(S) &= \bigcup_{x \in S} \pref(x) \qquad &\suff(S) &= \bigcup_{x \in S} \suff(x)

S_\ell = \left \{ \suff_{|\ell|}(x) \mid x \in S \wedge \pref_{|\ell|}(x) = \ell \right \} \; .

\weight(v) \geq \sum_{w \text{ child of }v} \weight(w) \; .

\lightdepth_\alpha(v) \leq \log_{\alpha+1} \weight(\operatorname{root}(T))

\weight(v) \geq (1+\alpha) \cdot \weight(w) > (1+\alpha) \cdot \frac{1}{\alpha+1} \weight(v) = \weight(v) \; ,

\bigl| T_\beta^{i+1}(S) \bigr| ~=~ \bigl|S \bigr| + \sum_{v \in T(S)} \bigl| T_\beta^i(S_v) \bigr| ~ \stackrel{IH}{\leq } ~ \bigl| S \bigr| + \sum_{v \in T(S)} \bigl| S_v \bigr| \sum_{j=0}^i H^j\; .

\sum_{v \in T(S)} |S_v| ~\leq~ |S| H \; ,

O \bigl( \sum_{i=0}^q |S_i| \log |S_i| \bigr) = O \bigl( \log n \sum_{i=0}^q |S_i| \bigr)= O \left( \log n |T_\beta^{k-1} (C^\prime)| \right) = O \bigl( n\log(n) \log_\beta^{k-1} n \bigr ) \; ,

p = p_0\; \gap{a_1}{b_1}\; p_1\; \gap{a_2}{b_2}\; \ldots\; \gap{a_j}{b_j}\; p_j \; .

t &= \texttt{acbccbacccddabdaabcdccbccdaa} \\
p &= \texttt{b} \gap{0}{4} \texttt{cc} \gap{3}{5} \texttt{d}

\left (\ind(\ell_0), \ind(\ell_0)+|\ell^\prime| \right ), \left (\ind(\ell_1), \ind(\ell_1)+|\ell^\prime| \right ), \ldots, \left ( \ind(\ell_r), \ind(\ell_r)+|\ell^\prime| \right ) \; .

p = p_0\; \gap{a_1}{b_1}\; p_1\; \gap{a_2}{b_2}\; \ldots\; \gap{a_j}{b_j}\; p_j \; .

\sum_{l=0}^{B_i-A_i} \binom{B_i-A_i}{l} \beta^{A_i+l} ~\leq~ 2^{B_i-A_i} \beta^{B_i} .

O \left( 2^{B-A} \beta^B \right)

O \left( \sum_{i=0}^j 2^{B_i - A_i} (|p_i| + b_i) + occ \right) ~=~ O \left( 2^{B-A} (m + B) + occ \right) \; .

O(n + nG^{k+o}) = O(n (\sigma^{k+o} \log\log n)^{k+o}) = O(n \sigma^{(k+o)^2} \log^{k+o}\log n) \; . 

This concludes the proof of \autoref{vlg-optimaltimeindex}.













\section{Conclusion}
We have presented several new indexes supporting patterns containing wildcards and variable length gaps. All previous wildcard indexes have query times which are either exponential in the number of wildcards or gaps in the pattern, or linear in the length of the indexed text. We showed that it is possible to obtain an index with linear query time while avoiding space usage exponential in the length of the indexed string. Moreover, we gave an index with linear space usage and a fast query time. For wildcard indexes having a query time sublinear in the length of the indexed string, an interesting open problem is whether there is an index where neither the size nor the query time is exponential in the number of wildcards or gaps in the pattern.

\bibliographystyle{abbrv}
\bibliography{references}

\end{document}
