

\documentclass[12pt, a4paper, twoside]{article}
\usepackage{amsmath,amsfonts,amssymb,amsthm,amscd}
\usepackage{epsfig}
\usepackage{graphicx}
\usepackage[cp1250]{inputenc}
\usepackage{hyperref} \usepackage{enumerate} \def\chapterautorefname{Chapter}
\def\sectionautorefname{Section}
\def\subsectionautorefname{Subsection}
\def\lemmaautorefname{Lemma}
\newtheorem{theorem}{Theorem}
\newtheorem{proposition}[theorem]{Proposition}
\newtheorem{lemma}[theorem]{Lemma}
\newtheorem{corollary}[theorem]{Corollary}
\newtheorem{observation}[theorem]{Observation}
\newtheorem{conjecture}{Conjecture}
\newtheorem{problem}{Problem}
\newtheorem{definition}{Definition}
\newcommand{\mindsix}{8821}
\newcommand{\smallsix}{1763}
\newcommand{\myatop}[2] {\genfrac{}{}{0pt}{1}{#1}{#2}}


\sloppy 



  


\def\l{\left}
\def\r{\right}
\def\epsi{\varepsilon}

\def\rdim #1{\mathbb R^{#1}}
\def\qdim #1{\mathbb Q^{#1}}
\def\zdim #1{\mathbb Z^{#1}}
\def\diam{\mathrm{diam}}
\def\fix{\mathrm{fix}}
\def\prob{\mathrm{Prob}}
\def\limninf {\lim_{n\rightarrow\infty}}
\def\mycirc{\mathcal S} \def\hb{H_{\mathrm{b}}}
\def\hr{H_{\mathrm{r}}}
\begin{document}


\title{Untangling polygons and graphs \-2mm]
{\small\it Department of Applied Mathematics}    \-2mm]
{\small\it Malostransk\'e n\'am.~25}          \-2mm]
{\small\tt cibulka@kam.mff.cuni.cz}           \
\fix(G,\delta) &=& \max_{\beta \text{ plane drawing of } G} \{|v \in V(G): \beta(v)=\delta(v)|\}, \\
\fix(G) &=& \min_{\delta \text{ mapping of }V(G)\text{ to }\rdim 2} \{\fix(G,\delta)\}.

2ls^2 + 2s\sum_{i=1}^{l}(4(l-i)) = 8l^3 + 4sl(l-1)) \leq 16l^3 = m .

ls = 2l^2 = 2^{-\frac53}m^{\frac23} \geq 2^{-\frac53}n^{\frac23} - O(n^\frac13) .

\text{Prob}[\text{cr}(H) \leq K] \leq \binom{t}{D}^2 \l(\frac{2t}{s}\r)^D \frac{s^{t-D}}{(t-D)!} ,

\fix(T) \leq 300 \sqrt{n} \log n \l(\sqrt{\Delta} + \min\l\{\sqrt[6]{n/\log^2 n},~ \sqrt{\diam} \r\}\r) .

K &:=& \l\lfloor 1.5 t \Delta + \min\{8n,2t \cdot \diam(T)\} \r\rfloor, \\
t &:=& \l\lceil 300 \sqrt{n} \log n
(\sqrt{\Delta} + \min\{\sqrt[6]{n/\log^2 n},~ \sqrt{\diam} \} ) \r\rceil,\\
s &:=& \l\lceil  35^2\frac{K+t}{t} \log^2 n \r\rceil .

D = \l\lfloor 35\sqrt{\frac{t(K+t)}{s}}  \r\rfloor \leq \frac{t}{\log n}.

\prob[\fix(T,\delta) \geq t]   &\leq&  \binom{n}{t}\prob[cr(H) \leq K] \\
&\leq& \binom{n}{t} \binom{t}{D}^2 \l(\frac{2t}{s}\r)^D \frac{s^{t-D}}{(t-D)!} \\
&\leq& \l(\frac{en}{t}\r)^t \l(\frac{et}{D}\r)^{2D} \l(\frac{2t}{s}\r)^D \l(\frac{es}{t(1-\frac{1}{\log n})}\r)^{t-D} \\
&\leq& \l(e^2 \frac{ns}{t^2}\r)^t \l(\frac{2et^4}{D^2s^2}\r)^D  \l(1-\frac{1}{\log n}\r)^{D-t} \\
&\leq& \l(e^2 \frac{ns}{t^2}\r)^t \l(\frac{2e^3 t^4}{D^2s^2}\r)^{\frac{t}{\log n}}  \\
&\leq& \l(4e^2 \frac{ns}{t^2}\r)^t \\
&\leq& \l(8e^2 36^2 \frac{ n \log^2{n} (\min\{4n,t \cdot \diam(T)\} + t\Delta) } {t^3}\r)^t. \\

\prob[\fix(T,\delta) \geq t]   
&\leq& \l(8e^2 36^2 \frac{ n \log^2 n (\Delta + \diam(T)) } {t^2}\r)^t \\
&\leq& \l(\frac{8e^2 36^2}{300^2} \frac{ \Delta + \diam(T) } {(\sqrt{\Delta} + \sqrt{\diam(T)})^2}\r)^t \\
&<& 0.9^t.

\prob[\fix(T,\delta) \geq t]   
&\leq& \l(8e^2 36^2 \frac{ n \log^2{n} (t\Delta + 4n) } {t^3}\r)^t \\
&\leq& \l(\frac{8e^2 36^2}{300^2} 
\frac{ t\Delta + 4n } {t \l( \sqrt{\Delta} + n^{\frac16} (\log n)^{-\frac13} \r)^2}\r)^t \\
&\leq& \l(\frac{8e^2 36^2}{300^2} 
\frac{ t\Delta + 4n } {t\Delta + t n^{\frac13} (\log n)^{-\frac23}}\r)^t \\
&\leq& \l(0.9 \frac{ t\Delta + 4n } {t\Delta + 300n^{\frac23}(\log n)^{\frac23} n^{\frac13} (\log n)^{-\frac23}}\r)^t \\
&<& 0.9^t.

\fix(G) \leq 300 \sqrt{n} \log n \l(\sqrt{\Delta} + \min\l\{\sqrt[6]{n/\log^2 n},~ \sqrt{2\diam} \r\}\r) .

\fix(G) \leq c(n\log n)^{\frac23} .

\Delta \geq  c\frac{n \epsi^2}{\log^2 n} .

\fix(G) \leq c \sqrt{bn\log^3 n}.

\end{enumerate}
\end{corollary}

\begin{proof}
All claims are based on the simple observation, that adding edges to a graph
 never increases  and thus if  is a spanning tree of , then .
Part~\ref{item_largedeg} is then straightforward and part~\ref{item_3conn} follows from a theorem
of Barnette~\cite{Barnette1966} which says that every planar 3-vertex-connected graph has a spanning tree 
with maximum degree three. 

To prove parts~\ref{item_all} and \ref{item_smallfix}, we now show that any graph  has a spanning 
tree  with diameter at most . Fix any vertex  of  and run a breadth-first search 
from it. 
All vertices lie at distance at most  from  and thus the diameter of the breadth-first search
tree is at most .
\end{proof}


{\bf Acknowledgments.}  I am grateful to Alexander Wolff for a careful
reading of an earlier version of the paper and many useful suggestions.

\bibliographystyle{plain}

\bibliography{untangling}


\end{document}
