In this section, we exhibit a reduction from BVASS coverability to
 provability, thereby showing its \textsc{2-ExpTime}-hardness.

Previous reductions from counter machines to substructural logics
in~\citep{lincoln92,urquhart99,lazic14} actually reduce to provability
in the logic extended with a \emph{theory} encoding the rules of the
system, which is then reduced to the basic logic.  This last step
relies in an essential way on the presence of exponential or additive
connectives to `dispose' of unused rules.

Having neither exponential nor additive connectives at our disposal,
we introduce in \autoref{sub-compr} a \emph{comprehensive} variant of
the expansive reachability problem, where every rule should be
employed at least once in the deduction.  We further avoid the use of
a theory and define in \autoref{sub-focus} a \emph{focusing} calculus
for , from which the correctness of the reduction given in
\autoref{sub-hard} will be facilitated.

\subsection{Comprehensive \detitle{Reachability}}\label{sub-compr}
Given a BVASS  with expansive semantics and two
states  and , the \emph{comprehensive reachability}
problem asks whether there exists a deduction tree of  with root
label  and leaves label , such that
every rule in  is used at least once.  In terms of root
judgements, it asks whether .
We show that BVASS coverability can be reduced to comprehensive
expansive reachability, hence by \autoref{fc-bvass}:
\begin{proposition}\label{prop-compr}
  Comprehensive reachability in expansive ordinary BVASS is
  \textsc{2-ExpTime}-hard.
\end{proposition}

\subsubsection{Increasing \detitle{Reachability}.\nopunct}
Let us consider an instance  of the coverability
problem in an ordinary BVASS .  As a first
step, we construct an ordinary BVASS
 with additional
\emph{increases}  for every  in  and
.  We claim that coverability in  is equivalent to
reachability in :
\begin{claim}\label{cl-cov-inc}
  There exists  in  such that  \ifshort iff\else if and only if\fi\ .
\end{claim}
\begin{proof}[Proof Sketch]
Clearly, if  for some  in
, then , and using
increases in  shows .
Conversely, if there is a reachability witness for
, then we can assume that increases
 occur as close to the root as possible.  As
increases occurring right below increments, decrements, or splits can
be permuted locally to occur right above, such a reachability witness
has all its increases at the root.  The deduction tree below those
increases is labelled  for some  in  and is
also a deduction tree of .
\end{proof}

\subsubsection{Comprehensive \detitle{Root Rules}.\nopunct}
The second step of the reduction from BVASS coverability builds an
ordinary BVASS  where .  It features an additional set of unary `root' rules---depicted
in \autoref{fig-compr}---designed to allow any rule in
 to be employed in a reachability witness.

\begin{figure}[tbp]
\centering
\begin{tikzpicture}[auto,node distance=.65cm]
  \node[state](qrdd){};
  \node[state,right=of qrdd](qdd){};
  \node[state,right=.8cm of qdd](qr){};
  \path (qrdd) edge node{} (qdd)
    (qdd) edge node{} (qr);
\node[state,right=1.3cm of qr](qi0){};
  \node[state,right=of qi0,label=above:{:}](qi1){};
  \node[state,right=of qi1](qi2){};
  \node[state,right=of qi2](qi3){};
  \node[state,right=of qi3](qi4){};
  \path (qi0) edge node{} (qi1)
    (qi1) edge node{} (qi2)
    (qi2) edge node{} (qi3)
    (qi3) edge node{} (qi4);
  \node[dotted,draw=black!40,rectangle,rounded
  corners=8pt,anchor=above left,minimum
  width=2.2cm,minimum height=1.2cm,above right=-.7cm and -.7cm of qi1]{};
\node[state,below=.6cm of qrdd](qd0){};
  \node[state,right=of qd0](qd1){};
  \node[state,right=of qd1,label=above:{:}](qd2){};
  \node[state,right=of qd2](qd3){};
  \node[state,right=of qd3](qd4){};
  \path (qd0) edge node{} (qd1)
    (qd1) edge node{} (qd2)
    (qd2) edge node{} (qd3)
    (qd3) edge node{} (qd4);
  \node[dotted,draw=black!40,rectangle,rounded
  corners=8pt,anchor=above left,minimum
  width=2.2cm,minimum height=1.2cm,above right=-.7cm and -.7cm of qd2]{};
\node[state,below=.26cm of qd4](qs0){};
  \node[state,right=of qs0](qs1){};
  \node[state,right=of qs1,label=right:{},label=above:{:}](qs2){};
  \draw[draw=black!40] (qs2) ++(.3,-.3) arc (-45:45:.42);
  \node[state,above right=of qs2](qs3){};
  \node[state,below right=of qs2](qs4){};
  \node[state,below right=of qs3](qs5){};
  \path (qs0) edge node{} (qs1)
    (qs1) edge node {} (qs2)
    (qs2) edge (qs3)
    (qs2) edge (qs4)
    (qs3) edge node {} (qs5)
    (qs4) edge[swap] node {} (qs5);
  \node[dotted,draw=black!40,rectangle,rounded
  corners=8pt,anchor=above left,minimum
  width=1.8cm,minimum height=2.7cm,right=-.8cm of qs2]{};
\end{tikzpicture}
\caption{\label{fig-compr}The rules of  in the proof of
  \autoref{prop-compr}, where  ranges over .}
\end{figure}

The idea is to introduce a new state  and a new coordinate
for each rule  in .  Starting from
,  can simulate any rule  from
 by first incrementing by the corresponding unit
vector , then applying the rule, and finally decrementing by
 to return to .  This ensures that, if
 for some  in
, then  where
.  The additional states and rules
from  and to  then show that:
\begin{claim}\label{cl-compr1}
  If  for some , then .
\end{claim}

Conversely, assume that .  This entails .

First assume that no decrement by  for any  in
 is ever used in the corresponding expansive
reachability witness.\ifshort\pagebreak\fi\ Then also no increment by
 occurs, and the only rule applicable at this point is
.  This yields a node 
labelled , and a deduction subtree rooted by  where
only rules of  and expansions are applied.  Because any
expansion can be simulated in  using an increase, this
yields .

Assume now the opposite: there is at least one occurrence of a
decrement by  in the expansive reachability witness.
Consider the bottommost such occurrence along some branch, necessarily
yielding a node with  as state label.  Then in the same
way, below this bottommost occurrence, no increment by 
occurs, and the only rule applicable at this point is
, which yields a node 
labelled  for some  in .  The deduction
subtree rooted by  only uses rules of  and expansions,
thus as in the previous case .
Using increases in  then shows .  Therefore, in all cases:
\begin{claim}\label{cl-compr2}
  If , then .
\end{claim}
By Claims~\ref{cl-cov-inc}, \ref{cl-compr1}, and~\ref{cl-compr2},

if and only if there exists  in  such that
, thereby showing the correctness of our
reduction.

\subsection{Focusing \detitle{Proofs} in }\label{sub-focus}
We enforce a particular proof policy in our simulation of BVASSs in
, which is inspired by the \emph{focusing proof}
techniques~\citep{andreoli92} employed to reduce non-determinism
during proof search in sequent calculi.  With only implication at our
disposal, we find ourselves in a `negative fragment', where focusing
proofs have a very simple calculus .  This is equivalent to
restricting oneself to long normal forms in the associated
-calculus.

A \emph{focusing sequent} is of one of the two forms
`' or `' where `' is called a
\emph{focused} formula.  We let  denote as before
multisets of implicational formul\ae\ and  implicational formul\ae.
Here are the rules of the focusing calculus :
1em]\tabrulelabelr[left implication]{fl-impl}{\Gamma\fprv A\quad\Delta, [B]\fprv a}{\Gamma,\Delta,[A\imp B]\fprv a}{\imp_L^\mathit{f}}
  \qquad
  \tabrulelabelr[right implication]{fl-impr}{\Gamma, A\fprv B}{\Gamma\fprv A\imp B}{\imp_R^\mathit{f}}

  q_\ell,\Delta_T,\Gamma_{\vec v}\fprv q

  \Delta_T&\eqdef \enc{t_1},\dots,\enc{t_k}\;,\\
  \Gamma_{\vec v}&\eqdef e_1^{c_1},\dots,e_d^{c_d}\;,
  \shortintertext{where `' is the individual encoding of rule  and
    `' stands for  repetitions of the formula .  We use the
    following individual rule encodings:}
  \enc{q\xrightarrow{\vec e_i}q_1}&\eqdef (e_i\imp q_1)\imp q\;,\\
  \enc{q\xrightarrow{-\vec e_i}q_1}&\eqdef q_1\imp(e_i\imp q)\;,\\
  \enc{q\to q_1+q_2}&\eqdef q_1\imp(q_2\imp q)\;.

  \sigma(m)\eqdef\{e\in E\mid m(e)>0\}

    \varphi(\emptyset,q_r)&\eqdef q_r\;,&
    \varphi(T\uplus\{t\},q_r)&\eqdef \enc{t}\imp \varphi(T,q_r)\;.
  
  By~\autoref{th-focus},  if and only if .  The
  latter holds if and only if 
  since we can only apply~\eqref{fl-impr}.  Then this occurs if and
  only if  by
  \autoref{cl-exp-fl1} and \autoref{cl-exp-fl2}.
\end{proof}

