

\documentclass[11pt,a4paper]{article}
\usepackage[hyperref]{acl2019}
\usepackage{times}
\usepackage{latexsym}

\usepackage{url}
\usepackage{graphicx}
\usepackage{subfigure}
\usepackage{hyperref}
\usepackage{url}
\usepackage{graphicx}
\usepackage{subfigure}
\usepackage{amssymb}
\usepackage{epsfig}
\usepackage{blindtext}
\usepackage{enumitem}
\usepackage{float}
\usepackage{amsthm}
\usepackage{wrapfig}
\usepackage{bbm}


\usepackage{amsmath}
\usepackage{booktabs}
\usepackage{multirow}
\aclfinalcopy 



\newcommand\BibTeX{B\textsc{ib}\TeX}

\title{Depth Growing for Neural Machine Translation}


\author{Lijun Wu, Yiren Wang\thanks{\;\,The first two authors contributed equally to this work. This work is conducted at Microsoft Research Asia.}, Yingce Xia\thanks{\;\,Corresponding author.}, Fei Tian, Fei Gao, \\ {\bf Tao Qin, Jianhuang Lai, Tie{-}Yan Liu }\\
  School of Data and Computer Science, Sun Yat-sen University;\\  University of Illinois at Urbana-Champaign;  Microsoft Research Asia;\\
  \texttt{\{wulijun3, stsljh\}@mail2.sysu.edu.cn}, \texttt{yiren@illinois.edu}, \\
  \texttt{\{Yingce.Xia, fetia, feiga, taoqin, tyliu\}@microsoft.com}}


\begin{document}
\maketitle
\begin{abstract}
While very deep neural networks have shown effectiveness for computer vision and text classification applications, how to increase the network depth of neural machine translation (NMT) models for better translation quality remains a challenging problem.
Directly stacking more blocks to the NMT model results in no improvement and even reduces performance. In this work, we propose an effective two-stage approach with three specially designed components to construct deeper NMT models, which results in significant improvements over the strong Transformer baselines on WMT EnglishGerman and EnglishFrench translation tasks\footnote{{Our code is available at \url{https://github.com/apeterswu/Depth_Growing_NMT}}}.
\end{abstract}

\section{Introduction}

Neural machine translation (briefly, NMT), which is built upon deep neural networks, has gained rapid progress in recent years~\citep{bahdanau2015neural,sutskever2014sequence,sennrich2016neural,he2016dual,sennrich2016improving,xia2017deliberation,wang2018multiagent} and achieved significant improvement in translation quality~\citep{hassan2018achieving}. Variants of network structures have been applied in NMT such as LSTM~\citep{wu2016google}, CNN~\citep{gehring2017convolutional} and Transformer~\citep{vaswani2017attention}.

Training deep networks has always been a challenging problem, mainly due to the difficulties in optimization for deep architecture. Breakthroughs have been made in computer vision to enable deeper model construction via advanced initialization schemes~\citep{he2015delving}, multi-stage training strategy~\citep{simonyan2015very}, and novel model architectures~\citep{srivastava2015highway,he2016deep}. While constructing very deep neural networks with tens and even more than a hundred blocks have shown effectiveness in image recognition~\citep{he2016deep}, question answering and text classification~\citep{devlin2018bert,radford2019language}, scaling up model capacity with very deep network remains challenging for NMT. The NMT models are generally constructed with up to  encoder and decoder blocks in both state-of-the-art research work and champion systems of machine translation competition. For example, the LSTM-based models are usually stacked for ~\citep{stahlberg2018university} or ~\citep{chen2018best} blocks, and the state-of-the-art Transformer models are equipped with a -block encoder and decoder~\citep{vaswani2017attention, junczys2018microsoft,edunov2018understanding}. Increasing the NMT model depth by directly stacking more blocks results in no improvement or performance drop (Figure~\ref{fig:baselines}), and even leads to optimization failure~\citep{bapna2018training}. 

\begin{figure}[!t]
\centering
\includegraphics[width=1.0\linewidth]{baselines.png}
\caption{Performances of Transformer models with different number of encoder/decoder blocks (recorded on -axis) on WMT EnDe translation task.  denotes the result reported in \cite{vaswani2017attention}.}
\label{fig:baselines}
\end{figure}

There have been a few attempts in previous works on constructing deeper NMT models. \citet{zhou2016deep} and \citet{wang2017deep} propose increasing the depth of LSTM-based models by introducing linear units between internal hidden states to eliminate the problem of gradient vanishing. However, their methods are specially designed for the recurrent architecture which has been significantly outperformed by the state-of-the-art transformer model. \citet{bapna2018training} propose an enhancement to the attention mechanism to ease the optimization of models with deeper encoders. While gains have been reported over different model architectures including LSTM and Transformer, their improvements are not made over the best performed baseline model configuration. How to construct and train {\em deep} NMT models to push forward the state-of-the-art translation performance with larger model capacity remains a challenging and open problem.

In this work, we explore the potential of leveraging {\em deep} neural networks for NMT and propose a new approach to construct and train deeper NMT models. As aforementioned, constructing deeper models is not as straightforward as directly stacking more blocks, but requires new mechanisms to boost the training and utilize the larger capacity with minimal increase in complexity. Our solution is a new two-stage training strategy, which ``grows'' a well-trained NMT model into a deeper network with three components specially designed to overcome the optimization difficulty and best leverage the capability of both shallow and deep architecture. Our approach can effectively construct a deeper model with significantly better performance, and is generally applicable to any model architecture.

We evaluate our approach on two  large-scale benchmark datasets, WMT EnglishGerman and EnglishFrench translations. Empirical studies show that our approach can significantly improve in translation quality with an increased model depth. Specifically, we achieve  and  BLEU score improvement over the strong Transformer baseline in EnglishGerman and EnglishFrench translations.

\begin{figure}[!t]
\centering
\includegraphics[width=0.9\linewidth]{deep_easy.pdf}
\caption{The overall framework of our proposed deep model architecture.  and  are the numbers of blocks in the bottom module (i.e., grey parts) and top module (i.e., blue and green parts). Parameters of the bottom module are fixed during the top module training. The dashed parts denote the original training/decoding of the bottom module. The weights of the two linear operators before softmax are shared.}
\label{fig:arch}
\end{figure}

\section{Approach}
\label{sec:approach}
We introduce the details of our proposed approach in this section. The overall framework is illustrated in Figure~\ref{fig:arch}. 

Our model consists of a bottom module with  blocks of encoder and decoder (the grey components in Figure~\ref{fig:arch}), and a top module with  blocks (the blue and green components). We denote the encoder and decoder of the bottom module as  and , and the corresponding two parts of the top module as  and . An encoder-decoder attention mechanism is used in the decoder blocks of the NMT models, and here we use  and  to represent such attention in the bottom and top modules respectively. 

The model is constructed via a two-stage training strategy: in Stage 1, the bottom module (i.e.,  and ) is trained and subsequently holds constant; in Stage 2,  only the top module (i.e.,  and ) is optimized. 

Let  and  denote the embedding of source and target sequence. Let  denote the number of words in , and  denote the elements before time step . Our proposed model works in the following way:

which contains three key components specially designed for deeper model construction, including:

\noindent(1) {\bf{\em Cross-module residual connections:}} As shown in Eqn.\eqref{eq:method:encoder}, the encoder  of the bottom module encodes the input  to a hidden representation , then a cross-module residual connection is introduced to the top module and the representation  is eventually produced. The decoders work in a similar way as shown in Eqn.\eqref{eq:method:Dec1} and \eqref{eq:method:Dec2}. This enables the top module to have direct access to both the low-level input signals from the word embedding and high-level information generated by the bottom module. Similar principles can be found in ~\citet{wang2017deep,wu2018word}.

\noindent(2) {\bf{\em Hierarchical encoder-decoder attention:}} We introduce a hierarchical encoder-decoder attention calculated with different contextual representations as shown in Eqn.\eqref{eq:method:Dec1} and \eqref{eq:method:Dec2}, where  is used as key and value for  in the bottom module, and  for  in the top module. Hidden states from the corresponding previous decoder block are used as queries for both  and  (omitted for readability). In this way, the strong capability of the well trained bottom module can be best preserved regardless of the influence from top module, while the newly stacked top module can leverage the higher-level contextual representations. More details can be found from source code in the supplementary materials.

\noindent(3) {\bf{\em Deep-shallow decoding:}} At the decoding phase,  and  work together according to Eqn.\eqref{eq:method:encoder} and Eqn.\eqref{eq:method:Dec1} as a shallow network , integrate both bottom and top module works as a deep network  according to Eqn.\eqref{eq:method:encoder}Eqn.\eqref{eq:method:Dec2}.  and  generate the final translation results through reranking.



\noindent {\bf Discussion}

\noindent  {\em Training complexity:} As aforementioned, the bottom module is trained in Stage 1 and only parameters of the top module are optimized in Stage 2. This significantly eases optimization difficulty and reduces training complexity. Jointly training the two modules with minimal training complexity is left for future work.


\noindent  {\em Ensemble learning:} What we propose in this paper is a {\em single} deeper model with hierarchical contextual information, although the deep-shallow decoding is similar to the ensemble methods in terms of inference complexity~\citep{zhou2012ensemble}. While training multiple diverse models for good ensemble performance introduces high additional complexity, our approach, as discussed above, ``grows" a well-trained model into a deeper one with minimal increase in training complexity. Detailed empirical analysis is presented in Section~\ref{subsec:analysis}.

\section{Experiments}
We evaluate our proposed approach on two large-scale benchmark datasets. We compare our approach with multiple baseline models, and analyze the effectiveness of our deep training strategy.

\subsection{Experiment Design}
\paragraph{Datasets}
We conduct experiments to evaluate the effectiveness of our proposed method on two widely adopted benchmark datasets: the WMT\footnote{\url{http://www.statmt.org/wmt14/translation-task.html}} EnglishGerman translation (EnDe) and the WMT EnglishFrench translation (EnFr). We use  parallel sentence pairs for EnDe and  pairs for EnFr as our training data\footnote{Training data are constructed with filtration rules following \url{https://github.com/pytorch/fairseq/tree/master/examples/translation}}. We use the concatenation of \emph{Newstest2012} and \emph{Newstest2013} as the validation set, and \emph{Newstest2014} as the test set. All words are segmented into sub-word units using byte pair encoding (BPE)\footnote{\url{https://github.com/rsennrich/subword-nmt}}~\citep{sennrich2016neural}, forming a vocabulary shared by the source and target languages with  and  tokens for EnDe and EnFr respectively.

\paragraph{Architecture}
The basic encoder-decoder framework we use is the strong Transformer model. We adopt the \texttt{big} transformer configuration following~\citet{vaswani2017attention}, with the dimension of word embeddings, hidden states and non-linear layer set as ,  and  respectively. The dropout rate is  for EnDe and  for EnFr. We set the number of encoder/decoder blocks for the bottom module as  following the common practice, and set the number of additionally stacked blocks of the top module as . Our models are implemented based on the PyTorch implementation of Transformer\footnote{\url{https://github.com/pytorch/fairseq}} and the code can be found in the supplementary materials.

\paragraph{Training}
We use Adam~\citep{kingma2015adam} optimizer following the optimization settings and default learning rate schedule in~\citet{vaswani2017attention} for model training. All models are trained on  M40 GPUs.

\paragraph{Evaluation}
We evaluate the model performances with tokenized case-sensitive BLEU\footnote{\url{https://github.com/moses-smt/mosesdecoder/blob/master/scripts/generic/multi-bleu.perl}} score~\citep{papineni2002bleu} 
for the two translation tasks. We use beam search with a beam size of  and with no length penalty. 

\subsection{Overall Results}
We compare our method ({\em Ours}) with the Transformer baselines of  blocks () and  blocks (), and a 16-block Transformer with transparent attention ({\em Transparent Attn (16B)})\footnote{We directly use the performance figure from~\citep{bapna2018training}, which uses the \texttt{base} Transformer configuration. We run the method of our own implementation with the widely adopted and state-of-the-art \texttt{big} setting, but no improvement has been observed.}~\citep{bapna2018training}. We also reproduce a -block Transformer baseline, which has better performance than what is reported in~\cite{vaswani2017attention} and we use it to initialize the bottom module in our model.

\begin{table}[!tb]
\centering
\caption{The test set performances of WMT EnDe and EnFr translation tasks. `' denotes the performance figures reported in the previous works.}
\begin{tabular}{lll}
\toprule
Model                               & EnDe    & EnFr    \\ \midrule
Transformer (6B)          &       &       \\
Transformer (6B)                    &       &       \\
Transformer (8B)                    &       &       \\
Transformer (10B)                   &       &       \\
Transparent Attn (16B)    &       &           \\
\bf{Ours (8B)}                      &  &  \\
\bottomrule
\end{tabular}
\label{tab:results-wmt}
\end{table}

From the results in Table~\ref{tab:results-wmt}, we see that our proposed approach enables effective training for deeper network and achieves  significantly better performances compared to baselines. With our method, the performance of a well-optimized -block model can be further boosted by adding two additional blocks, while simply using Transformer (8B) will lead to a performance drop. Specifically, we achieve a  BLEU score on EnDe translation with  BLEU improvement over the strong baselines, and achieve a  BLEU improvement for EnFr. The improvements are statistically significant with  in paired bootstrap sampling~\citep{koehn2004statistical}. 

We further make an attempt to train a deeper model with additional  blocks, which has  blocks in total for EnDe translation. The bottom module is also initialized from our reproduced -block transformer baseline. This model achieves a  BLEU score on EnDe translation and it surpasses the performance of our -block model, which further demonstrates that our approach is effective for training deeper NMT models. 


\subsection{Analysis}
\label{subsec:analysis}
To further study the effectiveness of our proposed framework, we present additional comparisons in EnDe translation with two groups of baseline approaches in Figure~\ref{fig:results-study}:

\begin{figure}[!tb]
    \centering
    \includegraphics[width=1.0\linewidth]{bar_compare.pdf}
    \caption{The test performances of WMT EnDe translation task.}
    \label{fig:results-study}
\end{figure}

\noindent (1) Direct stacking ({\em DS}): we extend the -block baseline to -block by directly stacking  additional blocks. 
We can see that both training from scratch ({\em DS scratch}) and ``growing'' from a well-trained -block model ({\em DS grow}) fails to improve performance in spite of larger model capacity. The comparison with this group of models shows that directly stacking more blocks is not a good strategy for increasing network depth, and demonstrates the effectiveness and necessity of our proposed mechanisms for training deep networks.

\noindent (2) Ensemble learning ({\em Ensemble}): we present the two-model ensemble results for fair comparison with our approach that involves a two-pass deep-shallow decoding. Specifically, we present the ensemble performances of two independently trained -block models ({\em Ensemble B/B}), and ensemble of one -block and one -block model independently trained from scratch ({\em Ensemble B/B}). As expected, the ensemble method improves translation quality over the single model baselines by a large margin (over  BLEU improvement). Regarding training complexity, it takes  GPU days ( days on  GPU) to train a single -block model from scratch,  GPU days for a -block model , and  GPU days to ``grow'' a -block model into -block with our approach. Therefore, our model is better than the two-model ensemble in terms of both translation quality (more than  BLEU improvement over the ensemble baseline) and training complexity. 

\section{Conclusion}
In this paper, we proposed a new training strategy with three specially designed components, including cross-module residual connection, hierarchical encoder-decoder attention and deep-shallow decoding, to construct and train deep NMT models. We showed that our approach can effectively construct deeper model with significantly better performance over the state-of-the-art transformer baseline. Although only empirical studies on the transformer are presented in this paper, our proposed strategy is a general approach that can be universally applicable to other model architectures, including LSTM and CNN. In future work, we will further explore efficient strategies that can jointly train all modules of the deep model with minimal increase in training complexity.

\bibliographystyle{acl_natbib}
\bibliography{acl2019}

\end{document}
