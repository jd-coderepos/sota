\documentclass{article} \usepackage{iclr2020_conference,times}
\iclrfinalcopy 

\usepackage{algorithm}
\usepackage[noend]{algorithmic}

\newcommand{\theHalgorithm}{\arabic{algorithm}}
\usepackage{hyperref}
\usepackage{url}
\usepackage{booktabs}
\usepackage[pdftex]{graphicx}


\usepackage{amsmath,amsfonts,bm}

\newcommand{\figleft}{{\em (Left)}}
\newcommand{\figcenter}{{\em (Center)}}
\newcommand{\figright}{{\em (Right)}}
\newcommand{\figtop}{{\em (Top)}}
\newcommand{\figbottom}{{\em (Bottom)}}
\newcommand{\captiona}{{\em (a)}}
\newcommand{\captionb}{{\em (b)}}
\newcommand{\captionc}{{\em (c)}}
\newcommand{\captiond}{{\em (d)}}

\newcommand{\newterm}[1]{{\bf #1}}


\def\figref#1{figure~\ref{#1}}
\def\Figref#1{Figure~\ref{#1}}
\def\twofigref#1#2{figures \ref{#1} and \ref{#2}}
\def\quadfigref#1#2#3#4{figures \ref{#1}, \ref{#2}, \ref{#3} and \ref{#4}}
\def\secref#1{section~\ref{#1}}
\def\Secref#1{Section~\ref{#1}}
\def\twosecrefs#1#2{sections \ref{#1} and \ref{#2}}
\def\secrefs#1#2#3{sections \ref{#1}, \ref{#2} and \ref{#3}}
\def\eqref#1{equation~\ref{#1}}
\def\Eqref#1{Equation~\ref{#1}}
\def\plaineqref#1{\ref{#1}}
\def\chapref#1{chapter~\ref{#1}}
\def\Chapref#1{Chapter~\ref{#1}}
\def\rangechapref#1#2{chapters\ref{#1}--\ref{#2}}
\def\algref#1{algorithm~\ref{#1}}
\def\Algref#1{Algorithm~\ref{#1}}
\def\twoalgref#1#2{algorithms \ref{#1} and \ref{#2}}
\def\Twoalgref#1#2{Algorithms \ref{#1} and \ref{#2}}
\def\partref#1{part~\ref{#1}}
\def\Partref#1{Part~\ref{#1}}
\def\twopartref#1#2{parts \ref{#1} and \ref{#2}}

\def\ceil#1{\lceil #1 \rceil}
\def\floor#1{\lfloor #1 \rfloor}
\def\1{\bm{1}}
\newcommand{\train}{\mathcal{D}}
\newcommand{\valid}{\mathcal{D_{\mathrm{valid}}}}
\newcommand{\test}{\mathcal{D_{\mathrm{test}}}}

\def\eps{{\epsilon}}


\def\reta{{\textnormal{}}}
\def\ra{{\textnormal{a}}}
\def\rb{{\textnormal{b}}}
\def\rc{{\textnormal{c}}}
\def\rd{{\textnormal{d}}}
\def\re{{\textnormal{e}}}
\def\rf{{\textnormal{f}}}
\def\rg{{\textnormal{g}}}
\def\rh{{\textnormal{h}}}
\def\ri{{\textnormal{i}}}
\def\rj{{\textnormal{j}}}
\def\rk{{\textnormal{k}}}
\def\rl{{\textnormal{l}}}
\def\rn{{\textnormal{n}}}
\def\ro{{\textnormal{o}}}
\def\rp{{\textnormal{p}}}
\def\rq{{\textnormal{q}}}
\def\rr{{\textnormal{r}}}
\def\rs{{\textnormal{s}}}
\def\rt{{\textnormal{t}}}
\def\ru{{\textnormal{u}}}
\def\rv{{\textnormal{v}}}
\def\rw{{\textnormal{w}}}
\def\rx{{\textnormal{x}}}
\def\ry{{\textnormal{y}}}
\def\rz{{\textnormal{z}}}

\def\rvepsilon{{\mathbf{\epsilon}}}
\def\rvtheta{{\mathbf{\theta}}}
\def\rva{{\mathbf{a}}}
\def\rvb{{\mathbf{b}}}
\def\rvc{{\mathbf{c}}}
\def\rvd{{\mathbf{d}}}
\def\rve{{\mathbf{e}}}
\def\rvf{{\mathbf{f}}}
\def\rvg{{\mathbf{g}}}
\def\rvh{{\mathbf{h}}}
\def\rvu{{\mathbf{i}}}
\def\rvj{{\mathbf{j}}}
\def\rvk{{\mathbf{k}}}
\def\rvl{{\mathbf{l}}}
\def\rvm{{\mathbf{m}}}
\def\rvn{{\mathbf{n}}}
\def\rvo{{\mathbf{o}}}
\def\rvp{{\mathbf{p}}}
\def\rvq{{\mathbf{q}}}
\def\rvr{{\mathbf{r}}}
\def\rvs{{\mathbf{s}}}
\def\rvt{{\mathbf{t}}}
\def\rvu{{\mathbf{u}}}
\def\rvv{{\mathbf{v}}}
\def\rvw{{\mathbf{w}}}
\def\rvx{{\mathbf{x}}}
\def\rvy{{\mathbf{y}}}
\def\rvz{{\mathbf{z}}}

\def\erva{{\textnormal{a}}}
\def\ervb{{\textnormal{b}}}
\def\ervc{{\textnormal{c}}}
\def\ervd{{\textnormal{d}}}
\def\erve{{\textnormal{e}}}
\def\ervf{{\textnormal{f}}}
\def\ervg{{\textnormal{g}}}
\def\ervh{{\textnormal{h}}}
\def\ervi{{\textnormal{i}}}
\def\ervj{{\textnormal{j}}}
\def\ervk{{\textnormal{k}}}
\def\ervl{{\textnormal{l}}}
\def\ervm{{\textnormal{m}}}
\def\ervn{{\textnormal{n}}}
\def\ervo{{\textnormal{o}}}
\def\ervp{{\textnormal{p}}}
\def\ervq{{\textnormal{q}}}
\def\ervr{{\textnormal{r}}}
\def\ervs{{\textnormal{s}}}
\def\ervt{{\textnormal{t}}}
\def\ervu{{\textnormal{u}}}
\def\ervv{{\textnormal{v}}}
\def\ervw{{\textnormal{w}}}
\def\ervx{{\textnormal{x}}}
\def\ervy{{\textnormal{y}}}
\def\ervz{{\textnormal{z}}}

\def\rmA{{\mathbf{A}}}
\def\rmB{{\mathbf{B}}}
\def\rmC{{\mathbf{C}}}
\def\rmD{{\mathbf{D}}}
\def\rmE{{\mathbf{E}}}
\def\rmF{{\mathbf{F}}}
\def\rmG{{\mathbf{G}}}
\def\rmH{{\mathbf{H}}}
\def\rmI{{\mathbf{I}}}
\def\rmJ{{\mathbf{J}}}
\def\rmK{{\mathbf{K}}}
\def\rmL{{\mathbf{L}}}
\def\rmM{{\mathbf{M}}}
\def\rmN{{\mathbf{N}}}
\def\rmO{{\mathbf{O}}}
\def\rmP{{\mathbf{P}}}
\def\rmQ{{\mathbf{Q}}}
\def\rmR{{\mathbf{R}}}
\def\rmS{{\mathbf{S}}}
\def\rmT{{\mathbf{T}}}
\def\rmU{{\mathbf{U}}}
\def\rmV{{\mathbf{V}}}
\def\rmW{{\mathbf{W}}}
\def\rmX{{\mathbf{X}}}
\def\rmY{{\mathbf{Y}}}
\def\rmZ{{\mathbf{Z}}}

\def\ermA{{\textnormal{A}}}
\def\ermB{{\textnormal{B}}}
\def\ermC{{\textnormal{C}}}
\def\ermD{{\textnormal{D}}}
\def\ermE{{\textnormal{E}}}
\def\ermF{{\textnormal{F}}}
\def\ermG{{\textnormal{G}}}
\def\ermH{{\textnormal{H}}}
\def\ermI{{\textnormal{I}}}
\def\ermJ{{\textnormal{J}}}
\def\ermK{{\textnormal{K}}}
\def\ermL{{\textnormal{L}}}
\def\ermM{{\textnormal{M}}}
\def\ermN{{\textnormal{N}}}
\def\ermO{{\textnormal{O}}}
\def\ermP{{\textnormal{P}}}
\def\ermQ{{\textnormal{Q}}}
\def\ermR{{\textnormal{R}}}
\def\ermS{{\textnormal{S}}}
\def\ermT{{\textnormal{T}}}
\def\ermU{{\textnormal{U}}}
\def\ermV{{\textnormal{V}}}
\def\ermW{{\textnormal{W}}}
\def\ermX{{\textnormal{X}}}
\def\ermY{{\textnormal{Y}}}
\def\ermZ{{\textnormal{Z}}}

\def\vzero{{\bm{0}}}
\def\vone{{\bm{1}}}
\def\vmu{{\bm{\mu}}}
\def\vtheta{{\bm{\theta}}}
\def\va{{\bm{a}}}
\def\vb{{\bm{b}}}
\def\vc{{\bm{c}}}
\def\vd{{\bm{d}}}
\def\ve{{\bm{e}}}
\def\vf{{\bm{f}}}
\def\vg{{\bm{g}}}
\def\vh{{\bm{h}}}
\def\vi{{\bm{i}}}
\def\vj{{\bm{j}}}
\def\vk{{\bm{k}}}
\def\vl{{\bm{l}}}
\def\vm{{\bm{m}}}
\def\vn{{\bm{n}}}
\def\vo{{\bm{o}}}
\def\vp{{\bm{p}}}
\def\vq{{\bm{q}}}
\def\vr{{\bm{r}}}
\def\vs{{\bm{s}}}
\def\vt{{\bm{t}}}
\def\vu{{\bm{u}}}
\def\vv{{\bm{v}}}
\def\vw{{\bm{w}}}
\def\vx{{\bm{x}}}
\def\vy{{\bm{y}}}
\def\vz{{\bm{z}}}

\def\evalpha{{\alpha}}
\def\evbeta{{\beta}}
\def\evepsilon{{\epsilon}}
\def\evlambda{{\lambda}}
\def\evomega{{\omega}}
\def\evmu{{\mu}}
\def\evpsi{{\psi}}
\def\evsigma{{\sigma}}
\def\evtheta{{\theta}}
\def\eva{{a}}
\def\evb{{b}}
\def\evc{{c}}
\def\evd{{d}}
\def\eve{{e}}
\def\evf{{f}}
\def\evg{{g}}
\def\evh{{h}}
\def\evi{{i}}
\def\evj{{j}}
\def\evk{{k}}
\def\evl{{l}}
\def\evm{{m}}
\def\evn{{n}}
\def\evo{{o}}
\def\evp{{p}}
\def\evq{{q}}
\def\evr{{r}}
\def\evs{{s}}
\def\evt{{t}}
\def\evu{{u}}
\def\evv{{v}}
\def\evw{{w}}
\def\evx{{x}}
\def\evy{{y}}
\def\evz{{z}}

\def\mA{{\bm{A}}}
\def\mB{{\bm{B}}}
\def\mC{{\bm{C}}}
\def\mD{{\bm{D}}}
\def\mE{{\bm{E}}}
\def\mF{{\bm{F}}}
\def\mG{{\bm{G}}}
\def\mH{{\bm{H}}}
\def\mI{{\bm{I}}}
\def\mJ{{\bm{J}}}
\def\mK{{\bm{K}}}
\def\mL{{\bm{L}}}
\def\mM{{\bm{M}}}
\def\mN{{\bm{N}}}
\def\mO{{\bm{O}}}
\def\mP{{\bm{P}}}
\def\mQ{{\bm{Q}}}
\def\mR{{\bm{R}}}
\def\mS{{\bm{S}}}
\def\mT{{\bm{T}}}
\def\mU{{\bm{U}}}
\def\mV{{\bm{V}}}
\def\mW{{\bm{W}}}
\def\mX{{\bm{X}}}
\def\mY{{\bm{Y}}}
\def\mZ{{\bm{Z}}}
\def\mBeta{{\bm{\beta}}}
\def\mPhi{{\bm{\Phi}}}
\def\mLambda{{\bm{\Lambda}}}
\def\mSigma{{\bm{\Sigma}}}

\DeclareMathAlphabet{\mathsfit}{\encodingdefault}{\sfdefault}{m}{sl}
\SetMathAlphabet{\mathsfit}{bold}{\encodingdefault}{\sfdefault}{bx}{n}
\newcommand{\tens}[1]{\bm{\mathsfit{#1}}}
\def\tA{{\tens{A}}}
\def\tB{{\tens{B}}}
\def\tC{{\tens{C}}}
\def\tD{{\tens{D}}}
\def\tE{{\tens{E}}}
\def\tF{{\tens{F}}}
\def\tG{{\tens{G}}}
\def\tH{{\tens{H}}}
\def\tI{{\tens{I}}}
\def\tJ{{\tens{J}}}
\def\tK{{\tens{K}}}
\def\tL{{\tens{L}}}
\def\tM{{\tens{M}}}
\def\tN{{\tens{N}}}
\def\tO{{\tens{O}}}
\def\tP{{\tens{P}}}
\def\tQ{{\tens{Q}}}
\def\tR{{\tens{R}}}
\def\tS{{\tens{S}}}
\def\tT{{\tens{T}}}
\def\tU{{\tens{U}}}
\def\tV{{\tens{V}}}
\def\tW{{\tens{W}}}
\def\tX{{\tens{X}}}
\def\tY{{\tens{Y}}}
\def\tZ{{\tens{Z}}}


\def\gA{{\mathcal{A}}}
\def\gB{{\mathcal{B}}}
\def\gC{{\mathcal{C}}}
\def\gD{{\mathcal{D}}}
\def\gE{{\mathcal{E}}}
\def\gF{{\mathcal{F}}}
\def\gG{{\mathcal{G}}}
\def\gH{{\mathcal{H}}}
\def\gI{{\mathcal{I}}}
\def\gJ{{\mathcal{J}}}
\def\gK{{\mathcal{K}}}
\def\gL{{\mathcal{L}}}
\def\gM{{\mathcal{M}}}
\def\gN{{\mathcal{N}}}
\def\gO{{\mathcal{O}}}
\def\gP{{\mathcal{P}}}
\def\gQ{{\mathcal{Q}}}
\def\gR{{\mathcal{R}}}
\def\gS{{\mathcal{S}}}
\def\gT{{\mathcal{T}}}
\def\gU{{\mathcal{U}}}
\def\gV{{\mathcal{V}}}
\def\gW{{\mathcal{W}}}
\def\gX{{\mathcal{X}}}
\def\gY{{\mathcal{Y}}}
\def\gZ{{\mathcal{Z}}}

\def\sA{{\mathbb{A}}}
\def\sB{{\mathbb{B}}}
\def\sC{{\mathbb{C}}}
\def\sD{{\mathbb{D}}}
\def\sF{{\mathbb{F}}}
\def\sG{{\mathbb{G}}}
\def\sH{{\mathbb{H}}}
\def\sI{{\mathbb{I}}}
\def\sJ{{\mathbb{J}}}
\def\sK{{\mathbb{K}}}
\def\sL{{\mathbb{L}}}
\def\sM{{\mathbb{M}}}
\def\sN{{\mathbb{N}}}
\def\sO{{\mathbb{O}}}
\def\sP{{\mathbb{P}}}
\def\sQ{{\mathbb{Q}}}
\def\sR{{\mathbb{R}}}
\def\sS{{\mathbb{S}}}
\def\sT{{\mathbb{T}}}
\def\sU{{\mathbb{U}}}
\def\sV{{\mathbb{V}}}
\def\sW{{\mathbb{W}}}
\def\sX{{\mathbb{X}}}
\def\sY{{\mathbb{Y}}}
\def\sZ{{\mathbb{Z}}}

\def\emLambda{{\Lambda}}
\def\emA{{A}}
\def\emB{{B}}
\def\emC{{C}}
\def\emD{{D}}
\def\emE{{E}}
\def\emF{{F}}
\def\emG{{G}}
\def\emH{{H}}
\def\emI{{I}}
\def\emJ{{J}}
\def\emK{{K}}
\def\emL{{L}}
\def\emM{{M}}
\def\emN{{N}}
\def\emO{{O}}
\def\emP{{P}}
\def\emQ{{Q}}
\def\emR{{R}}
\def\emS{{S}}
\def\emT{{T}}
\def\emU{{U}}
\def\emV{{V}}
\def\emW{{W}}
\def\emX{{X}}
\def\emY{{Y}}
\def\emZ{{Z}}
\def\emSigma{{\Sigma}}

\newcommand{\etens}[1]{\mathsfit{#1}}
\def\etLambda{{\etens{\Lambda}}}
\def\etA{{\etens{A}}}
\def\etB{{\etens{B}}}
\def\etC{{\etens{C}}}
\def\etD{{\etens{D}}}
\def\etE{{\etens{E}}}
\def\etF{{\etens{F}}}
\def\etG{{\etens{G}}}
\def\etH{{\etens{H}}}
\def\etI{{\etens{I}}}
\def\etJ{{\etens{J}}}
\def\etK{{\etens{K}}}
\def\etL{{\etens{L}}}
\def\etM{{\etens{M}}}
\def\etN{{\etens{N}}}
\def\etO{{\etens{O}}}
\def\etP{{\etens{P}}}
\def\etQ{{\etens{Q}}}
\def\etR{{\etens{R}}}
\def\etS{{\etens{S}}}
\def\etT{{\etens{T}}}
\def\etU{{\etens{U}}}
\def\etV{{\etens{V}}}
\def\etW{{\etens{W}}}
\def\etX{{\etens{X}}}
\def\etY{{\etens{Y}}}
\def\etZ{{\etens{Z}}}

\newcommand{\pdata}{p_{\rm{data}}}
\newcommand{\ptrain}{\hat{p}_{\rm{data}}}
\newcommand{\Ptrain}{\hat{P}_{\rm{data}}}
\newcommand{\pmodel}{p_{\rm{model}}}
\newcommand{\Pmodel}{P_{\rm{model}}}
\newcommand{\ptildemodel}{\tilde{p}_{\rm{model}}}
\newcommand{\pencode}{p_{\rm{encoder}}}
\newcommand{\pdecode}{p_{\rm{decoder}}}
\newcommand{\precons}{p_{\rm{reconstruct}}}

\newcommand{\laplace}{\mathrm{Laplace}} 

\newcommand{\E}{\mathbb{E}}
\newcommand{\Ls}{\mathcal{L}}
\newcommand{\R}{\mathbb{R}}
\newcommand{\emp}{\tilde{p}}
\newcommand{\lr}{\alpha}
\newcommand{\reg}{\lambda}
\newcommand{\rect}{\mathrm{rectifier}}
\newcommand{\softmax}{\mathrm{softmax}}
\newcommand{\sigmoid}{\sigma}
\newcommand{\softplus}{\zeta}
\newcommand{\KL}{D_{\mathrm{KL}}}
\newcommand{\Var}{\mathrm{Var}}
\newcommand{\standarderror}{\mathrm{SE}}
\newcommand{\Cov}{\mathrm{Cov}}
\newcommand{\normlzero}{L^0}
\newcommand{\normlone}{L^1}
\newcommand{\normltwo}{L^2}
\newcommand{\normlp}{L^p}
\newcommand{\normmax}{L^\infty}

\newcommand{\parents}{Pa} 

\DeclareMathOperator*{\argmax}{arg\,max}
\DeclareMathOperator*{\argmin}{arg\,min}

\DeclareMathOperator{\sign}{sign}
\DeclareMathOperator{\Tr}{Tr}
\let\ab\allowbreak
 
\renewcommand{\algorithmiccomment}[1]{\bgroup\hfill//~#1\egroup}
\newcommand{\INDSTATE}[1][1]{\STATE\hspace{#1\algorithmicindent}}

\title{Compressive Transformers for Long-Range Sequence Modelling}

\newcommand{\fix}{\marginpar{FIX}}
\newcommand{\new}{\marginpar{NEW}}
\newcommand{\sid}[1]{{\color{blue} {\footnotesize{Sid: \textbf{#1}}}}}
\newcommand{\ap}[1]{{\color{green} {\footnotesize{AP: \textbf{#1}}}}}

\author{
\centering
Jack W. Rae\thanks{
Authors contributed equally,
 DeepMind, London, UK.
 CoMPLEX, Computer Science, University College London, UK.
Please direct correspondence to \{jwrae, apotapenko\}@google.com.
}
\And
Anna Potapenko*
\And
Siddhant M. Jayakumar
\And
Chloe Hillier
\And
Timothy P. Lillicrap
}


\newcommand{\model}{Compressive Transformer}
\newcommand{\dataset}{PG-19}

\begin{document}

\maketitle


\begin{abstract}
We present the \model, an attentive sequence model which compresses past memories for long-range sequence learning. We find the \model~obtains state-of-the-art language modelling results in the WikiText-103 and Enwik8 benchmarks, achieving  ppl and  bpc respectively. We also find it can model high-frequency speech effectively and can be used as a memory mechanism for RL, demonstrated on an object matching task. To promote the domain of long-range sequence learning, we propose a new open-vocabulary language modelling benchmark derived from books, \dataset.
\end{abstract}
\section{Introduction}
Humans have a remarkable ability to remember information over long time horizons. When reading a book, we build up a compressed representation of the past narrative, such as the characters and events that have built up the story so far. We can do this even if they are separated by thousands of words from the current text, or long stretches of time between readings. During daily life, we make use of memories at varying time-scales: from locating the car keys, placed in the morning, to recalling the name of an old friend from decades ago. These feats of memorisation are not achieved by storing every sensory glimpse throughout one's lifetime, but via lossy compression. We aggressively select, filter, or integrate input stimuli based on factors of surprise, perceived danger, or repetition --- amongst other signals \citep{richards2017persistence}.

Memory systems in artificial neural networks began with very compact representations of the past. Recurrent neural networks (RNNs, \citet{rumelhart1986learning}) learn to represent the history of observations in a compressed state vector. The state is \textit{compressed} because it uses far less space than the history of observations --- the model only preserving information that is pertinent to the optimization of the loss. The LSTM \citep{hochreiter1997long} is perhaps the most ubiquitous RNN variant; it uses learned gates on its state vector to determine what information is stored or forgotten from memory. 

However since the LSTM, there has been great benefit discovered in \textit{not} bottlenecking all historical information in the state, but instead in keeping past activations around in an external memory and \textit{attending} to them. The Transformer \citep{vaswani2017attention} is a sequence model which stores the hidden activation of every time-step, and integrates this information using an attention operator \citep{bahdanau2014neural}. The Transformer will thus represent the past with a tensor (depth  memory size  dimension) of past observations that is, in practice, an order of magnitude larger than an LSTM's hidden state.  With this granular memory, the Transformer has brought about a step-change in state-of-the-art performance, within machine translation \citep{vaswani2017attention}, language modelling \citep{dai2019transformer, shoeybi2019megatronlm}, video captioning \citep{zhou2018end}, and a multitude of language understanding benchmarks \citep{devlin2018bert, yang2019xlnet} amongst others.

One drawback in storing everything is the computational cost of attending to every time-step and the storage cost of preserving this large memory. Several works have focused on reducing the computational cost of attention with sparse access mechanisms \citep{rae2016scaling, child2019generating, sukhbaatar2019adaptive, lample2019large}. However sparse attention does not solve the storage problem, and often requires custom sparse kernels for efficient implementation. Instead we look back to the notion of compactly representing the past. We show this can be built with simple dense linear-algebra components, such as convolutions, and can reduce both the space and compute cost of our models.

We propose the \model, a simple extension to the Transformer which maps past hidden activations (memories) to a smaller set of compressed representations (compressed memories). The \model~uses the same attention mechanism over its set of memories and compressed memories, learning to query both its short-term granular memory and longer-term coarse memory. We observe this improves the modelling of text, achieving state-of-the-art results in character-based language modelling ---  bpc on Enwik8 from the Hutter Prize \citep{hutter2012human} --- and word-level language modelling ---  perplexity on WikiText-103 \citep{merity2016pointer}. Specifically, we see the \model~improves the modelling of rare words.

We show the \model~works not only for language, but can also model the waveform of high-frequency speech with a trend of lower likelihood than the TransformerXL and Wavenet \citep{oord2016wavenet} when trained over 400,000 steps. We also show the \model~can be used as a memory component within an RL agent, IMPALA \citep{espeholt2018impala}, and can successfully compress and make use of past observations.

Furthermore we present a new book-level language-modelling benchmark \dataset, extracted from texts in Project Gutenberg\footnote{\url{https://www.gutenberg.org/}}, to further promote the direction of long-context sequence modelling. This is over double the size of existing LM benchmarks and contains text with much longer contexts.
\section{Related Work}
There have been a variety of recent attempts to extend the range of attention, particularly in the Transformer, or to replace the attention operation with something less expensive. \citet{wu2019pay} show that a convolution-like operator that runs in linear time can actually exceed the performance of the quadratic-time self-attention layer in the Transformer at sentence-to-sentence translation and sentence-level language modelling. However such a mechanism inhibits the flow of information across a large number of time-steps for a given layer, and has not shown to be beneficial for long-range sequence modelling.

\citet{dai2019transformer} propose the TransformerXL, which keeps past activations around in memory. They also propose a novel relative positional embedding scheme which they see outperforms the Transformer's original absolute positional system. Our model incorporates both of these ideas, the use of a memory to preserve prior activations and their relative positional embedding scheme.

The Sparse Transformer \citep{child2019generating} uses fixed sparse attention masks to attend to roughly  locations in memory. This approach still requires keeping all memories around during training, however with careful re-materialization of activations and custom kernels, the authors are able to train the model with a reasonable budget of memory and compute. When run on Enwik8, the much larger attention window of  improves model performance, but overall it does not significantly outperform a simpler TransformerXL with a much smaller attention window.

The use of dynamic attention spans is explored in \citet{sukhbaatar2019adaptive}. Different attention heads can learn to have shorter or longer spans of attention --- and they observe this achieves state-of-the-art in character-based language modelling. This idea could easily be combined with our contribution --- a compressive memory. However an efficient implementation is not possible on current dense-linear-algebra accelerators, such as Google's TPUs, due to the need for dynamic and sparse computation. Our approach builds on simple dense linear algebra components, such as convolutions.
\section{Model}
 \begin{figure}
     \centering
     \includegraphics[width=0.8\textwidth]{figures/model_simple_longer.pdf}
     \caption{The Compressive Transformer keeps a fine-grained memory of past activations, which are then compressed into coarser \textit{compressed} memories. The above model has three layers, a sequence length , memory size , compressed memory size . The highlighted memories are compacted, with a compression function  per layer, to a single compressed memory --- instead of being discarded at the next sequence. In this example, the rate of compression .}
     \label{fig:model}
 \end{figure}
We present the \model, a long-range sequence model which compacts past activations into a compressed memory. The \model~is a variant of the Transformer \citep{vaswani2017attention}, a deep residual network which only uses attention to propagate information over time (namely \textit{multi-head attention}). We build on the ideas of the TransformerXL \citep{dai2019transformer} which maintains a memory of past activations at each layer to preserve a longer history of context. The TransformerXL discards past activations when they become sufficiently old (controlled by the size of the memory). The key principle of the \model~is to compress these old memories, instead of discarding them, and store them in an additional \textit{compressed memory}. 

\subsection{Description}
We define  and  to be the number of respective memory and compressive memory slots in the model per layer. The overall input sequence  represents input observations (e.g. tokens from a book). These are split into fixed-size windows of size  for the model to process in parallel. The model observes  at time , which we refer to as the \textit{sequence} (e.g. in Figure \ref{fig:model}). As the model moves to the next sequence, its  hidden activations are pushed into a fixed-sized FIFO memory (like the TransformerXL). The oldest  activations in memory are evicted, but unlike the TransformerXL we do not discard them. Instead we apply a \textit{compression operation}, , mapping the  oldest memories to  compressed memories which we then store in a secondary FIFO \textit{compressed memory}.  denotes the hidden size of activations and  refers to the compression rate, a higher value indicates more coarse-grained compressed memories. The full architecture is described in Algorithm \ref{alg:model}.

\begin{algorithm}
    \caption{\model}
    \label{alg:model}
    \begin{algorithmic}[1] 
    \hbox{At time zero}
    \STATE  \COMMENT{Initialize memory to zeros } \\
    \STATE  \COMMENT{Initialize compressed memory to zeros } \\
    \hbox{At time t}
    \STATE  \COMMENT{Embed input sequence()} \\
    \FOR{layer }
        \STATE  \hbox{concat}(    \COMMENT{()} \\
        \STATE  \hbox{multihead\_attention}) \COMMENT{MHA over both mem types} ()\\
        \STATE  \hbox{layer\_norm}( \COMMENT{Regular skip + layernorm }\\
        \STATE  \COMMENT{Oldest memories to be forgotten }\\
        \STATE   \COMMENT{Compress oldest memories by factor  ()} \\
        \STATE  \hbox{concat} \COMMENT{Update memory (}\\
        \STATE  \hbox{concat} \COMMENT{Update compressed memory } \\
        \STATE  \hbox{layer\_norm(mlp}) \COMMENT{Mixing MLP ()}\\
    \ENDFOR 
    \end{algorithmic}
\end{algorithm}
\begin{algorithm}
    \caption{Attention-Reconstruction Loss}
    \label{alg:loss}
    \begin{algorithmic}[1] 
    \STATE 
    \FOR{layer }
        \STATE  \COMMENT{Stop compression grads from passing... }\\
        \STATE   \COMMENT{...into transformer network.} \\
        \STATE  \hbox{stop\_gradient(attention params at layer i)} \COMMENT{Re-use attention weight matrices.}\\
        \STATE def  \COMMENT{Use content-based attention (no relative).} \\
        \STATE   \COMMENT{Compression network (to be optimized).} \\
        \STATE 
    \ENDFOR 
    \end{algorithmic}
\end{algorithm}
\subsection{Compression Functions and Losses}
For choices of compression functions  we consider \textbf{(1) max/mean pooling}, where the kernel and stride is set to the compression rate ; \textbf{(2) 1D convolution} also with kernel \& stride set to  \textbf{(3) dilated convolutions}; \textbf{(4) most-used} where the memories are sorted by their average attention (usage) and the most-used are preserved. The pooling is used as a fast and simple baseline. The \textit{most-used} compression scheme is inspired from the garbage collection mechanism in the Differentiable Neural Computer \citep{graves2016hybrid} where low-usage memories are erased. The convolutional compression functions contain parameters which require training.

One can train the compression network using gradients from the loss; however for very old memories this requires backpropagating-through-time (\textbf{BPTT}) over long unrolls. As such we also consider some local auxiliary compression losses. We consider an \textbf{auto-encoding} loss where we reconstruct the original memories from the compressed memories , where  is learned. This is a lossless compression objective --- it attempts to retain all information in memory. We also consider an \textbf{attention-reconstruction} loss described in Algorithm \ref{alg:loss} which reconstructs the content-based attention over memory, with content-based attention over the compressed memories. This is a lossy objective, as information that is no longer attended to can be discarded, and we found this worked best. We stop compression loss gradients from passing into the main network as this prevents learning. Instead the Transformer optimizes the task objective and the compression network optimizes the compression objective conditioned on task-relevant representations; there is no need to mix the losses with a tuning constant.

\subsection{Temporal Range}
The TransformerXL with a memory of size  has a maximum temporal range of  with an attention cost of  (see \cite{dai2019transformer} for a detailed discussion). The \model~now has a maximum temporal range of  with an attention cost of . For example, setting  and  we obtain a maximum temporal range that is two times greater than the TransformerXL with an identical attention cost. Thus if we can learn in the  compressed setting, the temporal range of the model can be significantly increased.  
\section{\dataset~Benchmark}
As models begin to incorporate longer-range memories, it is important to train and benchmark them on data containing larger contexts. Natural language in the form of text provides us with a vast repository of data containing long-range dependencies, that is easily accessible.  We propose a new language modelling benchmark, \textbf{\dataset}, using text from books extracted from Project Gutenberg~\footnote{The authors intend to release the \dataset~dataset along with the split into train, validation and test subsets.}. We select Project Gutenberg books which were published over  years old, i.e. before  (hence the name \dataset ) to avoid complications with international copyright, and remove short texts. The dataset contains  books, or  of text --- which makes it over double the size of BookCorpus and Billion Word Benchmark. 
\subsection{Related Datasets}
The two most benchmarked word-level language modelling datasets either stress the modelling of stand-alone sentences (Billion Word Benchmark from \cite{chelba2013one}) or the modelling of a small selection of short news articles (Penn Treebank processed by \cite{mikolov2010recurrent}). \cite{merity2016pointer} proposed the WikiText-103 dataset, which contains text from a high quality subset of English-language wikipedia articles.  These articles are on average  words long. This dataset has been a popular recent LM benchmark due to the potential to exploit longer-range dependencies \citep{grave2016improving, rae2018fast, bai2018trellis}. However recent Transformer models, such as the TransformerXL \citep{dai2019transformer} appear to be able to exploit temporal dependencies on the order of several thousand words. This motivates a larger dataset with longer contexts.

Books are a natural choice of long-form text, and provide us with stylistically rich and varied natural language. Texts extracted from books have been used for prior NLP benchmarks; such as the Children's Book Test \citep{hill2015goldilocks} and LAMBADA \citep{paperno2016lambada}. These benchmarks use text from Project Gutenberg, an online repository of books with expired US copyright, and BookCorpus~\citep{zhu2015aligning}, a prior dataset of  unpublished (at time of authorship) books. CBT and LAMBADA contain extracts from books, with a specific task of predicting held-out words. In the case of LAMBADA the held-out word is specifically designed to be predictable for humans with access to the full textual context --- but difficult to guess with only a local context.

CBT and LAMBADA are useful for probing the linguistic intelligence of models, but are not ideal for training long-range language models from scratch as they truncate text extracts to at most a couple of paragraphs, and discard a lot of the books' text. There has been prior work on training models on book data using BookCorpus directly (e.g. BERT from \citet{devlin2018bert}) however BookCorpus is no longer distributed due to licensing issues, and the source of data is dynamically changing --- which makes exact benchmarking difficult over time. 

The NarrativeQA Book Comprehension Task \citep{kovcisky2018narrativeqa} uses Project Gutenberg texts paired with Wikipedia articles, which can be used as summaries. Due to the requirement of needing a corresponding summary, NarrativeQA contains a smaller selection of books: 1,527 versus the 28,752 books in \dataset. However it is reasonable that \dataset~may be useful for pre-training book summarisation models.
\begin{table}[]
    \footnotesize
    \centering
    \caption{Comparison to existing popular language modelling benchmarks.}
    \label{tab:compare_datasets}
    \begin{tabular}{rc c c c}
    \toprule
    & \textbf{Avg. length (words)} & \textbf{Train Size} & \textbf{Vocab} & \textbf{Type} \\
    \midrule
    1B Word &   27      & 4.15GB      & 793K & News (sentences) \\
    Penn Treebank           & 355       & 5.1MB   &  10K  & News (articles) \\
    WikiText-103            & 3.6K      & 515MB      & 267K & Wikipedia (articles) \\
    \midrule
    \dataset       & 69K       & 10.9GB     & (open) & Books \\
    \bottomrule
    \end{tabular}
\end{table}
\vspace{-1em}
\subsection{Statistics}
\label{sec:stats}
A brief comparison of \dataset~to other LM datasets can be found in Table~\ref{tab:compare_datasets}. 
We intentionally do not limit the vocabulary by \textit{unk-ing} rare words, and release the dataset as an open-vocabulary benchmark. To compare models we propose to continue measuring the word-level perplexity. This can still be computed for any chosen character-based, byte-based or subword-based scheme. To do this, one calculates the total cross-entropy loss  over the given validation or test subset using a chosen tokenization scheme, and then one normalizes this value by the number of words:  where  is the total number of words in the given subset, taken from Table \ref{tab:gutenberg_subsets}. The word-level perplexity is thus . For sake of model comparisons, it is important to use the exact number of words computed in Table \ref{tab:gutenberg_subsets} as the normalisation constant. 

Alongside quantitative analyses, we build an LDA topic model~\citep{blei2003latent} for a qualitative inspection of the text. We present key words for several topics in the Supplementary Table~\ref{tab:gutenberg_topics}. These topics include art, education, naval exploration, geographical description, war, ancient civilisations, and more poetic topics concerning the human condition --- love, society, religion, virtue etc. This contrasts to the more objective domains of Wikipedia and news corpora. 
\section{Experiments}
We optimised all models with Adam \citep{kingma2014adam}. We used a learning rate schedule with a linear warmup from  to  and a cosine decay back down to . For character-based LM we used  warmup steps with  decay steps, and for word-based LM we used  warmup steps with  decay steps. We found that decreasing the optimisation update frequency helped (see Section \ref{section:optimisation}), namely we only applied parameter updates every  steps after  iterations. However we found the models would optimise well for a range of warmup/warm-down values. We clipped the gradients to have a norm of at most , which was crucial to successful optimisation. 
\vspace{-0.5em}
\subsection{PG-19}
\begin{table}
    \centering
    \footnotesize
    \begin{minipage}[t]{0.52\linewidth}
        \caption{\dataset~statistics split by subsets.}
        \begin{tabular}{l c c c}
            \toprule
            & \textbf{Train} & \textbf{Valid.} & \textbf{Test} \\
            \midrule
            \# books & 28,602 &	50	& 100  \\
            \# words & 1,973,136,207	& 3,007,061 &	6,966,499    \\
        \bottomrule
        \end{tabular}
        \label{tab:gutenberg_subsets}
    \end{minipage}
    \begin{minipage}[t]{0.45\linewidth}
        \footnotesize
        \caption{Eval. perplexities on~\dataset.}
        \begin{tabular}{lc c c}
        \toprule
        & Valid. & Test \\
        \midrule
        36L TransformerXL & 45.5 & 36.3 \\
        \textbf{36L Compressive Transf.} & 43.4 & 33.6 \\
        \bottomrule
        \end{tabular}
        \label{tab:gutenberg}
    \end{minipage}
\end{table}
We benchmark the \model~against the TransformerXL on the newly proposed PG-19 books dataset. Because it is open-vocabulary, we train a subword vocabulary of size  with SubwordTextEncoder from the tfds package in TensorFlow and use the dataset statistics to compute word-level perplexity, as described in Section \ref{sec:stats}.
We train a 36 layer \model~with a window size of , both memory and compressed memory size of , and compression rate~. We compare this to a 36 layer TransformerXL trained with window size  and attention window .
The model was trained on  TPUv3 cores with a total batch size of~ and converged after processing around  billion subword tokens. We display the results in Table \ref{tab:gutenberg} where we see the \model~obtains a test perplexity of  versus the TransformerXL's . Despite the dataset size, it is clearly a challenging domain. This can suit as a first baseline on the proposed long-range language modelling benchmark. We show samples from this model in Supplementary Section \ref{app:samples}. The model is able to generate long-form narrative of varying styles: from character dialogue, first person diary entries, to descriptive third-person text. 
\begin{table}[t]
    \footnotesize
    \centering
    \begin{minipage}[t]{0.48\linewidth}
        \centering
            \caption{\footnotesize
                State-of-the-art results on Enwik8.
            }
            \begin{tabular}{l|cc}
                \toprule
                \bf Model & \bf BPC \\
                \midrule
                7L LSTM \citep{graves2013generating}                 & 1.67 \\
                LN HyperNetworks \citet{ha2016hypernetworks}         & 1.34 \\
                LN HM-LSTM \citet{chung2016hierarchical}             & 1.32 \\
                ByteNet \citep{kalchbrenner2016neural}               & 1.31 \\
                RHN \citet{zilly2017recurrent}                       & 1.27 \\
                mLSTM \citet{krause2016multiplicative}               & 1.24 \\
                64L Transf. \citet{al2019character}               & 1.06 \\
                24L TXL \citep{dai2019transformer}          & 0.99 \\
                Sparse Transf. \citep{child2019generating}       & 0.991 \\
                Adaptive Transf. \citep{sukhbaatar2019adaptive}   & 0.98 \\
                \midrule
                \textit{24L TXL (ours)}                   & 0.98 \\
                24L \model                          & \textbf{0.97} \\
                \bottomrule
            \end{tabular}
            \label{table:enwik8}
    \end{minipage}
    \hspace{1em}
    \begin{minipage}[t]{0.48\linewidth}
        \footnotesize
        \centering
        \caption{\footnotesize
            Compression approaches on Enwik8.
        }
        \begin{tabular}{l|cc}
            \toprule
            \bf Compression fn & \bf Compression loss & \bf BPC \\
            \midrule
            Conv & BPTT & 0.996 \\
            Max Pooling & N/A & 0.986 \\
            Conv & Auto-encoding & 0.984 \\
            Mean Pooling & N/A & 0.982 \\
            Most-used & N/A & 0.980 \\
            Dilated conv & Attention & 0.977 \\
            Conv & Attention & \bf 0.973 \\
            \bottomrule
        \end{tabular}
        \label{table:compressions}
    \end{minipage}
\end{table}
\subsection{Enwik8}
We compare the TransformerXL and the \model~on the standard character-level language modelling benchmark Enwiki8 taken from the Hutter Prize \citep{hutter2012human}, which contains 100M bytes of unprocessed Wikipedia text. We select the first 90MB for training, 5MB for validation, and the latter 5MB for testing --- as per convention.  We train -layer models with a sequence window size of . During training, we set the TransformerXL's memory size to , and for the \model~we use memory of size  and compressed memory of size  with compression rate~. During evaluation, we increased the TransformerXL memory size to  and the compressed memory in our model to  (after sweeping over the validation set), obtaining the numbers reported in Table~\ref{table:enwik8}. We show the effect of scaling the compressed memory size and evaluation performance in Supplementary Section \ref{app:cm_ablation}.
The proposed model achieves the new state-of-the-art on this dataset with  bits-per-character.

We compare compression functions and the use of auxiliary losses in Table \ref{table:compressions}. We sweep over compression rates of , , and  and report results with the best performing value for each row.
BPTT signifies that no auxiliary compression loss was used to train the network other than the overall training loss. To feed gradients into the compression function we unrolled the model over double the sequence length and halved the batch size to fit the larger unroll into memory.
\subsection{Wikitext-103}
We train an eighteen-layered \model~on the closed-vocabulary word-level language modelling benchmark WikiText-103, which contains articles from Wikipedia. We train the model with a compressed memory size, memory size, and a sequence window size all equal to . We trained the model over  Tensor Processing Units (TPU) v3 with a batch size of  per core --- making for a total batch size of . The model converged in a little over  hours. We found the single-layer convolution worked best, with a compression rate of . This model obtained  perplexity on the test set. By tuning the memory size over the validation set --- setting the memory size to , and compressed memory size to  --- we obtain  perplexity. This is  perplexity points over prior state of the art, and means the model places a  higher probability on the correct word over the prior SotA TransformerXL.

It is worth noting that in Table \ref{tab:wiki} we do not list methods that use additional training data, or that make use of test-time labels to continue training the model on the test set (known as dynamic evaluation \citep{graves2013generating}). If we incorporate a very naive dynamic evaluation approach of loading a model checkpoint and continuing training over one epoch of the test set, then we obtain a test perplexity of \textbf{16.1}. This is slightly better than the published 16.4 from \cite{krause2019dynamic} --- which uses a more sophisticated dynamic evaluation approach on top of the TransformerXL. However in most settings, one does not have access to test-time labels --- and thus we do not focus on this setting. Furthermore there has been great progress in showing that more data equates to much better language modelling; \cite{shoeybi2019megatronlm} find a large transformer 8B-parameter transformer trained on 170GB of text obtains 10.7 word-level perplexity on WikiText-103. However it is not clear to what extent the WikiText-103 test set may be leaked inside these larger training corpora. For clarity of model comparisons, we compare to published results trained on the WikiText-103 training set. Certainly the direction of larger scale and more data appear to bring immediate gains to the quality of existing language models. Both data scale and quality alongside intelligent model design are complementary lines of research towards better sequence modelling.

We break perplexity down by word frequency in Table \ref{tab:lm_breakdown} and see the \model~makes only a small modelling improvement for frequent words ( over the TransformerXL baseline) but obtains a much larger improvement of  for infrequent words. Furthermore, we see  improvement in modelling rare words over the prior state-of-the-art LSTM language model published in 2018 --- which demonstrates the rate of progress in this area.
\begin{table}[]
    \footnotesize
    \centering
    \caption{\footnotesize Validation and test perplexities on WikiText-103.}
    \begin{tabular}{lc c}
    \toprule
    & Valid. & Test \\
    \midrule
    LSTM \citep{graves2014neural} & - & 48.7 \\
    Temporal CNN \citep{bai2018convolutional} & - & 45.2 \\
    GCNN-14 \citep{dauphin2016language} & - & 37.2 \\ 
    Quasi-RNN \cite{bradbury2016quasi} & 32 & 33 \\
    RMC \citep{santoro2018relational} & 30.8 & 31.9 \\
    LSTM+Hebb. \citep{rae2018fast} & 29.0 & 29.2 \\
    Transformer \citep{baevski2019adaptive} & - & 18.7 \\
    18L TransformerXL, M=384 \citep{dai2019transformer} & - & 18.3 \\
    \midrule
    \textit{18L TransformerXL, M=1024 (ours)} & - & 18.1 \\ 
    18L \model, M=1024 & \textbf{16.0} & \textbf{17.1} \\
    \bottomrule
    \end{tabular}
    \label{tab:wiki}
\end{table}
\begin{table}
    \small
    \centering
    \caption{WikiText-103 test perplexity broken down by word frequency buckets. The most frequent bucket is words which appear in the training set more than  times, displayed on the left. For reference, a uniform model would have perplexity  for all frequency buckets. *LSTM comparison from \citet{rae2018fast}}
    \begin{tabular}{l c c c c c}
    \toprule
     & K & KK & K &  & All \\ 
     \midrule
    LSTM* & 12.1 & 219 & 1,197 & 9,725 & 36.4 \\
    TransformerXL (ours) & 7.8	& 61.2 & 188 &	1,123 &	18.1 \\
    \model & \textbf{7.6} &	\textbf{55.9} & \textbf{158}	& \textbf{937} & \textbf{17.1} \\
    \midrule 
    Relative gain over TXL & 2.6\% & 9.5\% & 21\% & 19.9\% & 5.8\% \\
    \bottomrule
    \end{tabular}
    \label{tab:lm_breakdown}
\end{table}
\vspace{-1.0em}
\subsection{Compressibility of layers}
We can use compression to better understand the model's mode of operation. We inspect how compressible Transformer's activations are as they progress through higher layers in the network. One may expect representations to become more difficult to compress at higher layers, if more semantic information is represented there. We monitor the compression loss at each layer of our best-performing \model~models trained on Enwik8 and WikiText-103 and display these in Supplementary Section \ref{app:compression_loss_layer} Figure \ref{fig:compression_loss_layer}. We note that the compression loss is about one order of magnitude higher for word-level language modelling (WikiText-103) over character-level langauge modelling (Enwik8). Furthermore the first layer of the Transformer is highly compressible. However there is not a clear trend of compression cost increasing with layer depth.
\vspace{-0.5em}
\subsection{Attention}
\vspace{-0.5em}
We inspect where the network is attending to on average, to determine whether it is using its compressed memory. We average the attention weight over a sample of  sequences from a trained model on Enwik8. We aggregate the attention into eighteen buckets, six for each of the compressed memory, memory, and sequence respectively. We set the size of the sequence, memory and compressed memory all to be . We plot this average attention weight per bucket in Figure \ref{fig:attention_weight} with a 1 standard error. We see most of the attention is placed on the current sequence; with a greater weight placed on earlier elements of the sequence due to the causal self-attention mechanism which masks future attention weights. We also observe there is an \textit{increase} in attention from the oldest activations stored in the regular memory, to the activations stored in the compressed memory. \textbf{This goes against the trend of older memories being accessed less frequently --- and gives evidence that the network is learning to preserve salient information}.
\begin{figure}[]
    \centering
    \begin{minipage}[t]{0.48\linewidth}
        \centering
    \includegraphics[width=0.9\linewidth]{figures/average_attention_weight.pdf}
    \caption{\textbf{Attention weight on Enwik8}. Average attention weight from the sequence over the compressed memory (oldest), memory, and sequence (newest) respectively. The sequence self-attention is causally masked, so more attention is placed on earlier elements in the sequence. There is an increase in attention at the transition from memory to compressed memory. }
    \label{fig:attention_weight}
    \end{minipage}
    \hspace{1em}
    \begin{minipage}[t]{0.48\linewidth}
    \includegraphics[width=0.9\linewidth]{figures/enwik8_lr_restart_curves.pdf}
    \caption{\textbf{Learning rate analysis}. Reducing the learning rate (e.g. to zero) during training (on Enwik8) harms training performance. Reducing the frequency of optimisation updates (effectively increasing the batch size) is preferable.}
    \label{fig:restart_curves}
    \end{minipage}
\end{figure}
\subsubsection{Optimisation Schedule}
\label{section:optimisation}
We make an observation about an interesting but undesirable meta-learning phenomenon during long-context training. When the learning rate is tuned to be much smaller (or set to zero) during training, performance degrades drastically both for the TransformerXL and the \model. This is displayed in Figure \ref{fig:restart_curves}.  

Usually we consider distributional shift from the training data to the test data, but we can also observe a shift in the model when transferring from a training to evaluation mode (even when the model is evaluated on the training data). In this case, this is due to the online updating of parameters whilst processing long contiguous articles. We would like the model to generalise well to scenarios where it is not continuously optimised. Updating the parameters only at article boundaries (and then resetting the state) could be one solution for long-range memory models, but this would slow down learning significantly. 

Instead, we propose reducing the frequency of optimisation updates during training. We find this allows for the best of both worlds --- fast initial learning with frequent updates, and better generalisation near the end of training with less frequent updates (e.g. every 4 steps). Reducing the optimisation frequency increases the effective batch size, which has also been shown to be preferable to learning rate decay in image modelling \citep{smithdecay2018}. We observed a final performance improvement in our TransformerXL baseline on Enwik8, from  --- which approximately replicates the published result --- to  --- which matches the most recent SotA architecture. 
We note, the additional space and compute cost of accumulating gradients is negligible across iterations, so there was no performance regression in using this scheme.
\subsection{Speech}
We train the \model~on the waveform of speech to assess its performance on different modalities. Speech is interesting because it is sampled at an incredibly high frequency, but we know it contains a lot of information on the level of phonemes and entire phrases.

To encourage long-term reasoning, we refrain from conditioning the model on speaker identity or text features, but focus on unconditional speech modelling. We train the model on 24.6 hours of 24kHz North American speech data. We chunk the sequences into windows of size , roughly ms of audio, and compare a -layer \model~to a -layer TransformerXL and a -layer WaveNet model \citep{oord2016wavenet} --- a state-of-the-art audio generative model used to serve production speech synthesis applications at Google~\citep{oord2018parallel}. All networks have approximately 40M parameters, as WaveNet is more parameter-efficient per layer. We train each network with  V100 GPUs, and a batch size of  per core (total batch size of ) using synchronous training.

WaveNet processes an entire chunk in parallel, however the TransformerXL and \model~are trained with a window size of  and a total memory size of  (for the \model~we use  memory +  compressed). We thus unroll the model over the sequence. Despite this sequential unroll, the attention-based models train at only half the speed of WaveNet. We see the test-set  negative-log-likelihood in Figure \ref{fig:speech_curve}, and observe that a \model~with a compression rate of  is able to outperform the TransformerXL and maintain a slim advantage over WaveNet. However we only trained models for at most one week (with 32GPUs) and it would be advantageous to continue training until full convergence --- before definitive conclusions are made. 
\subsection{Reinforcement Learning}
Compression is a good fit for video input sequences because subsequent frames have high mutual information. Here we do not test out the \model~on video, but progress straight to a reinforcement learning agent task that receives a video stream of visual observations --- but must ultimately learn to use its memory to reason over a policy.

We test the \model~as a drop-in replacement to an LSTM in the IMPALA setup \citep{espeholt2018impala}. Otherwise, we use the same training framework and agent architecture as described in the original work with a fixed learning rate of  and entropy cost coefficient of . 
We test the \model~on a challenging memory task within the DMLab-30 \citep{beattiedm2016} domain, \textit{rooms\_select\_nonmatching\_object}. This requires the agent to explore a room in a visually rich 3D environment and remember the object present. The agent can then advance to a second room where it must select the object \emph{not present} in the original room. This necessitates that the agent both remember events far in the past, and also learn to efficiently reason about them.

We fix both the memory and compressed memory sizes to . In Figure \ref{fig:rl}, we present results for a range of compression rates, averaged over  seeds. We see that the best performing agents endowed with the \model~are able to solve the task to human-level. We note that the model with compression rate  is unable to learn the task to the same proficiency. The speed of learning and stability seem to increase proportionally with higher rates of compression (up to a limit) -- i.e. the effective memory window of the agent -- and we find compression rate  to once again be the best performing. We see this as a promising sign that the architecture is able to efficiently learn, and suitably use,  compressed representations of its visual input and hope to test this more widely in future work. 
\begin{figure}[t]
\centering
\begin{minipage}{0.48\linewidth}
    \centering
    \includegraphics[width=0.98\textwidth]{figures/speech_learning_curve.pdf}
    \caption{\textbf{Speech Modelling.} We see the \model~is able to obtain competitive results against the state-of-the-art WaveNet in the modelling of raw speech sampled at 24kHz.}
    \label{fig:speech_curve}
\end{minipage}
\hspace{1em}
\begin{minipage}{0.48\linewidth}
    \centering
    \includegraphics[width=0.89\textwidth]{figures/rl.pdf}
    \caption{\textbf{Vision and RL}. We see the \model~integrates visual information across time within an IMPALA RL agent, trained on an object matching task.}
    \label{fig:rl}
\end{minipage}
\end{figure}
\section{Conclusion}
In this paper we explore the notion of compression as a means of extending the temporal receptive field of Transformer-based sequence models. We see a benefit to this approach in the domain of text, with the \model~outperforming existing architectures at long-range language modelling. To continue innovation in this area, we also propose a new book-level LM benchmark, \dataset. This may be used to compare long-range language models, or to pre-train on other long-range reasoning language tasks, such as NarrativeQA \citep{kovcisky2018narrativeqa}.

We see the idea of compressive memories is applicable not only to the modality of text, but also audio, in the form of modelling the waveform of speech, and vision, within a reinforcement-learning agent trained on a maze-like memory task. In both cases, we compare to very strong baselines (Wavenet \citep{oord2016wavenet} and IMPALA \citep{espeholt2018impala}).

The main limitation of this work is additional complexity, if the task one wishes to solve does not contain long-range reasoning then the \model~is unlikely to provide additional benefit. However as a means of scaling memory and attention, we do think compression is a simpler approach to dynamic or sparse attention --- which often requires custom kernels to make efficient. One can build effective compression modules from simple neural network components, such as convolutions. The compression components are immediately efficient to run on GPUs and TPUs.

Memory systems for neural networks began as compressed state representations within RNNs. The recent wave of progress using attention-based models with deep and granular memories shows us that it is beneficial to refrain from immediately compressing the past. However we hypothesise that more powerful models will contain a mixture of granular recent memories and coarser compressed memories. Future directions could include the investigation of adaptive compression rates by layer, the use of long-range shallow memory layers together with deep short-range memory, and even the use of RNNs as compressors. Compressive memories should not be forgotten about just yet.

\ificlrfinal
\section*{Acknowledgements}
We thank Chris Dyer, Felix Gimeno, and Koray Kavukcuoglu for reviewing the manuscript. We thank Peter Dayan, Adam Santoro, Jacob Menick, Emilio Parisotto, Hyunjik Kim, Simon Osindero, Sergey Bartunov, David Raposo, and Daan Wierstra for ideas regarding model design. We thank Yazhe Li and Aaron Van de Oord for their help and advice in instrumenting speech modelling experiments. Finally, we thank our wider DeepMind colleagues for supporting this project with stimulating discussions, engineering infrastructure, and positive reinforcement signals.  

\section*{Author Contributions}
\begin{itemize}
\item[] Model and Experiment design: JR, TL, AP, SJ
\item[] Dataset creation: AP, JR, CH
\item[] Text experiments: JR, AP
\item[] RL experiments: SJ
\item[] Speech experiments: JR
\end{itemize}

\section*{Funding}
This research was funded by DeepMind.

\section*{Competing interests}
The authors declare no competing financial interests.
\else
\fi

\newpage 

\bibliographystyle{abbrvnat}
\setlength{\bibsep}{5pt} 
\setlength{\bibhang}{0pt}
\bibliography{template_refs}



\newpage 
\appendix
\section*{Supplementary Materials}

\section{Compression Across Layers}
\label{app:compression_loss_layer}
We inspect the compression loss broken down by the layer index, to investigate whether there is a trend in network depth with how compressible the representations are. The compression loss here refers to the attention-reconstruction attention loss. We plot this for a 24 layer trained model on Enwik8, and an 18 layer model trained on WikiText-103. The compression loss for character-based language modelling is about one order of magnitude lower than that of word-level language modelling. The first layer's representations are highly compressible, however from then on there is no fixed trend. Some non-contiguous layers have a very similar compression loss (e.g. 4 \& 6, 5 \& 7) which suggests information is being routed from these layer pairs via the skip connection.

\begin{figure}[h!]
    \centering
    \includegraphics[width=\linewidth]{figures/compression_by_layer.pdf}
    \caption{\textbf{Model analysis.} Compression loss broken down by layer.}
    \label{fig:compression_loss_layer}
\end{figure}


\section{Comparison of Compressed Memory Sizes}
\label{app:cm_ablation}
We compare the best test perplexity obtained for the \model~trained on WikiText-103 and Enwik8 across a range of compressed memory sizes. For both models, the best model used a 1D convolution compression network with a compression rate of . The Enwik8 model was trained with an embedding size of ,  attention heads,  layers, an mlp hidden size of , a sequence window size of , and a memory size of . We see the best compressed memory size is  in this sweep, facilitating a total attention window of . The WikiText-103 model was trained with an embedding size of , adaptive inputs using the same parameters as \citep{sukhbaatar2019adaptive},  attention heads,  layers, an mlp hidden size of , a sequence window of size  and a memory of size . The best compressed memory size is  resulting in a total attention window of c. .  

\begin{table}[h!]
    \centering
    \begin{tabular}{c|c c c c c}
    \toprule
         Compressed Memory Size        & 512 & 1024 & 2048 & 3072 & 4096 \\
         Enwik8 BPC                    & 1.01	& 0.99 & 0.98 & 0.97 & 1.00 \\
    \bottomrule
    \end{tabular}
    \caption{Compressed memory size vs test performance for Enwik8}
    \label{tab:memory_sizes_enwik8}
\end{table}

\begin{table}[h!]
    \centering
    \begin{tabular}{c|c c c c c}
    \toprule
         Compressed Memory Size     & 256 & 512 & 1024 & 1536 & 2048 \\
         WikiText-103 Perplexity    & 18.2 & 17.9 & 17.6 & 17.1 & 17.7 \\
    \bottomrule
    \end{tabular}
    \caption{Compressed memory size vs test performance for WikiText-103}
    \label{tab:memory_sizes_wikitext}
\end{table}

\section{\dataset~Preprocessing}

The raw texts from the Gutenberg project were minimally pre-processed by removing boilerplate license text. We then also replaced discriminatory words with a unique  token using the Ofcom list of discriminatory words~\footnote{\url{https://www.ofcom.org.uk/__data/assets/pdf_file/0023/91625/OfcomQRG-AOC.pdf}}. 


\section{\dataset~Topics}

We present top-words for some of the topics on the \dataset~corpus. These were generated with LDA topic model~\citep{blei2003latent}.
\label{app:pg}
\begin{table}[h!]
\centering
\caption{Examples of top topics on \dataset~corpus.}
\label{tab:gutenberg_topics}
\medskip
\begin{tabular}{ccccccc}
\toprule
\textbf{Geography} & \textbf{War} & \textbf{Civilisations} & \textbf{Human Condition} & \textbf{Naval} & \textbf{Education} & \textbf{Art} \\
\midrule
water & people & roman & love & island & work & poet \\ 
river & emperor & rome & religion & ship & school & music \\
feet & war & greek & religious & sea & life & one \\
miles & army & city & life & men & children & poetry \\
north & death & gods & moral & captain & may & work \\
south & battle & king & human & coast & social & literature \\
mountains & city & first & society & land & child & art \\
sea & soldiers & caesar & man & great & education & great \\
lake & power & great & virtue & found & conditions & poem \\
rock & thousand & romans & nature & islands & well & written \\
mountain & arms & athens & marriage & shore & study & english \\
country & empire & greece & women & voyage & best & author \\
valley & upon & temple & christian & vessels & years & play \\
ice & country & son & age & time & possible & genius \\
west & time & egypt & law & english & class & style \\
\bottomrule
\end{tabular}
\end{table}

\section{\dataset~Samples}
\label{app:samples}
We show a few different samples from the \model~trained on \dataset. We use Nucleus Sampling with  \citep{holtzman2019curious}. We choose extracts of books from the test set as prefixes. We see the model is able to continue in the style of the text, creating artificial dialogue or descriptive text, and remembering the names of characters over hundreds of words.

As the \model~is trained without state resetting, it is actually slightly out of sample when provided with the (relatively) short contexts. This is because its memory and compressed memory may be still empty (whereas they are always full during training). However we see a trend of the samples usually improving towards the end.
\newpage
\fbox{\parbox{0.95\textwidth}{\rule{0.95\textwidth}{0.4pt}
{\bf Context from Baby Mine by Maragaret Mayo}:
\1\baselineskip]
Even in college Alfred Hardy was a young man of fixed ideas and high
ideals and proud of it.
\1\baselineskip]
He did not like doing them in the most elegant of all possible marble
walks--where freshmen and sophomores stand and talk and laugh and
chatter till evening falls and look cool and still and cool and
beautifully white in spite of the hot sun, but had to go there for
a month before going home. It happened one day that he and Jimmy were
strolling together in the green shade of St. Martin's, and Jimmy
naturally asked why he was so silent. He was very talkative as most boys
are who have once got the habit of it on shore, so the need for answering
was soon supplied.
\1\baselineskip]
"Oh, yes," said Alfred, "it was plain enough then that he liked to live by
himself and got none of his present pleasures or entertainments because
of the other fellow's giving it to him."
\1\baselineskip]
"And it is pretty true, too," said Alfred, "yet I believe it is the best
thing that could happen even now."
\1\baselineskip]
"And so do I," said Alfred, "and so do I, and so do I."
\1\baselineskip]
"I don't know," said Alfred, "but if someone will come down to the valley
and go through a regular mill with the old forces at work, doing the work
of the two, and doing it better than any other Miller he ever saw or ever
hoped to see, I think there will be a good deal of change in the things
and things to be cared for for and for driven and counter-driven."
\1\baselineskip]
June 1st.  Having taken our leaves of Sir W. Batten and my Lady, who are
gone this morning to keep their Whitsuntide, Sir W. Pen and I and Mr.
Gauden by water to Woolwich, and there went from ship to ship to give
order for and take notice of their forwardness to go forth, and then to
Deptford and did the like, having dined at Woolwich with Captain Poole at
the tavern there.  From Deptford we walked to Redriffe, calling at the
half-way house, and there come into a room where there was infinite of new
cakes placed that are made against Whitsuntide, and there we were very
merry.  By water home, and there did businesses of the office. Among
others got my Lord's imprest of L1000 and Mr. Creed's of L10,000 against
this voyage their bills signed.  Having wrote letters into the country and
read some things I went to bed.
\rule{0.95\textwidth}{0.4pt}
{\bf \model}: 
\1\baselineskip]
So I took leave of my great friend, and the next day being Sunday, my
wife and I and a great number of friends went by water to Deptford, and
there took the oaths of allegiance and supremacy and promised such
obedience to the Lord as should hereafter turn to the advantage of our
State. Being come back we found the shops and warehouses in a pitiable
condition, so my wife and I and all the rest of the friends went home and
there did the like and by Saturday day I being alone in my office I found
myself in a very bad situation, for my Lord, and Mr. Creed, and Pett were
all standing ready to receive and receive the oaths of allegiance and right
to make use of the ten days' licence I had allowed for this purpose by
virtue of the law that was such a help to the servant. Having looked at the
ships, I told my wife what I had seen and that I was greatly troubled at my
situation and, I hope that will mend it.
\1\baselineskip]
High up on the hillside in the midst of a rugged group of jack pines the
Union Jack shook out its folds gallantly in the breeze that swept down
the Kicking Horse Pass. That gallant flag marked the headquarters of
Superintendent Strong, of the North West Mounted Police, whose special
duty it was to preserve law and order along the construction line of the
Canadian Pacific Railway Company, now pushed west some scores of miles.
\1\baselineskip]
As though by mutual consent, all the three members of the
Superintendent Force were standing on the veranda watching his
progress, a man came strolling slowly down the path, stopping often
to gaze curiously about him, then stopping once more when he saw the four
standing on the veranda. His face was dark, his eyes prominent, and the
lines that furrowed his forehead showed marks of many woes, yet his
step was alert and he began to mount the steps with steady strides.
\1\baselineskip]
Had a chariot been made for the good of the three horses, and had the driver been given
orders that he should speed them that he might win, they would have
been heartening things in the sight of the veteran and the victor. To
you they would have been unintelligible to the root of your
understanding. When you gaze up in the faces of those four gray
horses, you can see clearly through the clouds of dust that rise from
their hoofs, and discern plainly where the banker is and where the hobo.
Then you will understand why you shall not press the bitter grapes
and why you shall not spurn the generous doctrines. You will understand
why you shall not praise the lash or the spur, for you will know where
the true would be and where the false would be. Then you will understand
why you, a man with reason and heart, need not tear your hair
over-bitter and why you need not laugh over the blunders of an ignorant
man.
\1\baselineskip]
About nine o'clock that morning, two buggies, drawn by powerful
horses, crossed the Rubicon and turned the railroad from
Sandhurst into the Hollow of the Mountains. And though the
charioteers stood at their horses' heads, and their drivers cried at
their loudest, there was not a man in the four teams who did not feel
that his day was worth all the toil and all the peril that he had
undergone. And if there were a man in them who did not know
that--who did not feel that the road through the Hollow of the
Mountains is made easy by the arrival of travelers and by the coming
of government, there was one who did not at that moment care
whether his day's work were worth all the toil and all the danger
that he had had to endure or whether it were not worth more
than all.
}
}


\end{document}
