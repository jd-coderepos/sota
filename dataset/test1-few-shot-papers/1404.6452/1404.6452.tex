\newcommand{\alphIi}{{\Lambda}}
\newcommand{\alphOo}{{\Delta}}
\newcommand{\dIi}{d_{\alphIi}}
\newcommand{\dOo}{d_{\alphOo}}
\newcommand{\run}{\gamma}

In this section, we review definitions of finite-state transducers and weighted
automata, and present similarity functions. We use the following notation.  We
denote input letters by ,  \etc, input words by ,  \etc, output
letters by ,  \etc and output words by ,  \etc We denote
the concatenation of words  and  by , the 
letter of word  by , the subword  by , the length of the word  by , and the empty
word and empty letter by . Note that for an -word , .
\medskip 

\noindent {\bf Finite-state Transducers.}\label{subsec:fst} A finite-state
transducer (\fst)  is given by a tuple  where
 is the input alphabet,  is the output alphabet,  is a
finite nonempty set of states,  is a set of initial states,  is a set of
transitions\footnote{Note that we disallow -transitions where the
transducer can change state without moving the reading head.}, and  is a 
set of accepting states.

A run  of  on an input word  is defined in
terms of the sequence: , ,  where  and for each , . 
Let  denote the set of states that appear infinitely often along .
For an \fst 
processing -words, a run is accepting if  (B{\"u}chi acceptance condition). 
For an \fst  processing finite words, a run
: ,  ,  on 
input word  is accepting if  (final state acceptance 
condition). 
The output of  along a run is the word 
 if the run is accepting, and is undefined otherwise. The transduction computed
by an \fst  processing infinite words (resp., finite words) 
is the relation  (resp., ), where  iff there is an
accepting run of  on  with  as the output along that run. With
some abuse of notation, we denote by  the set
. The input language,
, of  is the set .

An \fst  is called {\em functional} if the
relation  is a function. In this case, we use
 to denote the unique output word generated  along
any accepting run of  on input word .  Checking if an arbitrary \fst
is functional can be done in polynomial time \cite{gurari_note_1983}. An 
\fst  is {\em deterministic} if for each  and each , 
. An \fst
 is a {\em letter-to-letter} transducer if for every transition of the
form , . A {\em Mealy machine} is 
a deterministic, letter-to-letter transducer, with every state being an accepting state. In what follows, we use transducers and finite-state transducers interchangeably.

\noindent{\em Composition of transducers.} Consider transducers  and   such that for every , .  We define
, the \emph{composition} of  and ,
as the transducer , where  is defined as:
   iff  and upon reading ,  generates  and 
changes state from  to , \ie, iff  and there exist
, , , 
 such that  and 
.  Observe that if  are functional,
 is functional and , where  
 denotes function composition.
\medskip

\noindent{\bf Weighted automata.} Recall that a finite automaton (with B{\"u}chi or 
final state acceptance) can be expressed as a 
tuple , where  is the alphabet,  is a finite
set of states,  is a set of initial states,  is a transition relation, and  is a set
of accepting states.  A weighted automaton (\wa) is a finite automaton whose transitions
are labeled by rational numbers.  Formally, a \wa  is a tuple  such that  is a finite automaton
and  is a function labeling the transitions of . 
The transition labels are called \emph{weights}.

Recall that a run  of a finite automaton on a word  is defined as a sequence of states:  
where  and for each  , .  A run  in a finite automaton processing 
-words (resp., finite words) is accepting if it satisfies 
the B{\"u}chi (resp., final state) acceptance condition. 
The set of accepting
runs of an automaton on a word  is denoted . Given a word ,
every run  of a \wa  on  defines a sequence
 of weights of
successive transitions of ; such a sequence is also referred to as a
weighted run.  To define the semantics of weighted automata we need to define
the value of a run (that combines the sequence of weights of the run into a
single value) and the value across runs (that combines values of different runs
into a single value).  To define values of runs, we consider \emph{value
functions}  that assign real numbers to sequences of rational numbers, and
refer to a \wa with a particular value function  as an -\wa.  Thus, the
value  of a run  of an -\wa  on a word  equals .
The value of a word  assigned by an -\wa , denoted
, is the {\em infimum} of the set of values of all accepting
runs, i.e.,  (the
infimum of an empty set is infinite).  


In this paper, we consider the following value
functions: (1)~the sum function , (2)~the discounted sum function  with  
and (3)~the limit-average
function . Note that the limit-average value function cannot be 
used with finite sequences. We define .

A \wa  is \emph{functional} iff for every word ,
all accepting runs of  on  have the same value. 



\noindent{\em Decision questions. }Given an -\wa  and a threshold , the \emph{emptiness} question
asks whether {\em there exists} a word  such that   and the \emph{universality} question asks whether {\em for all} words
 we have .  The following results are known.

\begin{lemma}
(1)~For every ,
the emptiness problem is decidable 
in polynomial time for nondeterministic -automata
~\cite{Filar:1996:CMD:248676,Droste:2009:HWA:1667106}.
(2)~The universality problem is undecidable for -automata with 
weights drawn from ~\cite{DBLP:journals/ijac/Krob94,DBLP:conf/atva/AlmagorBK11}.
\label{lem:oldResults}
\end{lemma}



\noindent{\em Remark}.
Weighted automata have been defined over semirings~\cite{Droste:2009:HWA:1667106} as
well as using value functions (along with infimum or supremum) as above~\cite{DBLP:conf/fct/ChatterjeeDH09,DBLP:journals/tocl/ChatterjeeDH10}.                 
These variants of weighted automata have incomparable expression power.
We use the latter definition as it enables us to express long-run average and discounted sum,
which are inexpressible using weighted automata over semirings.
Long-run average and discounted sum are
widely used in quantitative verification and define natural distances (\exref{picewise-linear}).
Moreover, unlike the semiring-based definition, the value-function-based definition 
extends easily from finite to infinite words.\\


\noindent{\bf Similarity Functions.} In our work, we use similarity functions
to measure the similarity between words. Let  denote the set . 
A similarity function  is a function with the properties:  (1)  and (2)
 . A similarity function  is also a
distance (function or metric) if it satisfies the additional
properties:  (3)  iff  and (4)  . We emphasize that in our work we do not need to restrict similarity functions to be distances.  

An example of a similarity function
is the \emph{generalized Manhattan distance} defined as: 
 
for infinite words  (resp.,  
 
for finite ), where 
 is the mismatch penalty for substituting letters.
The mismatch penalty is required to be a distance function 
on the alphabet (extended with a special end-of-string letter  for finite words). When  is defined to be  for all  with , and  otherwise, 
 is called the \emph{Manhattan distance}. 
 
\noindent {\em Notation}: We use  to denote {\em
convolution} of words , for . The convolution of 
words merges the arguments into a single word over a -tuple alphabet 
(accommodating arguments of different lengths using  letters at the ends
of shorter words).  Let  be words over alphabets .  Let  denote the
-tuple alphabet .  The convolution  is an 
infinite word (resp., a finite word of length ), 
over , such that: for each ,  (with  if ).  For example,
the convolution  is the 3 letter word .

\begin{definition}[Automatic Similarity Function]
A similarity function  is called
automatic if there exists a \wa  over  such that : . We say that  is computed by .
\end{definition}

One can similarly define automatic similarity functions over finite words. 








 


