\pdfoutput=1
\newif\ifFull
\Fulltrue
\ifFull
\documentclass[11pt]{article}
\else
\documentclass{sig-alternate}
\fi
\usepackage{color}
\ifFull
\newenvironment{proof}{\begin{normalsize}\noindent{\bf Proof:}}{  \end{normalsize} \\}
\newcommand{\qed}{}
\fi

\usepackage{graphicx}
\usepackage{cite}
\usepackage{times}
\usepackage{latexsym}
\usepackage{amssymb}
\usepackage{amsmath}
\usepackage{hyperref}

\ifFull
\setlength{\pdfpagewidth}{8.5in}
\setlength{\pdfpageheight}{11in}
\topmargin 0pt
\advance \topmargin by -\headheight
\advance \topmargin by -\headsep
\textheight 9in
\oddsidemargin 0pt
\evensidemargin \oddsidemargin
\marginparwidth 0.5in
\textwidth 6.5in
\else \pagestyle{empty}
\fi
     
\makeatletter
\def\@begintheorem#1#2{\sl \trivlist \item[\hskip \labelsep{\bf #1\ #2:}]}
\def\@opargbegintheorem#1#2#3{\sl \trivlist
      \item[\hskip \labelsep{\bf #1\ #2\ #3:}]}
\makeatother

\newcommand{\R}{{\bf R}}

\newtheorem{theorem}{Theorem}
\newtheorem{lemma}[theorem]{Lemma}
\newtheorem{proposition}[theorem]{Proposition}
\newtheorem{corollary}[theorem]{Corollary}
\newtheorem{observation}[theorem]{Observation}
\newtheorem{property}[theorem]{Property}
\newtheorem{example}[theorem]{Example}
\newtheorem{counterexample}[theorem]{Counterexample}
\newtheorem{problem}[theorem]{Problem}
\newtheorem{openproblem}[theorem]{Open Problem}
\newtheorem{assumption}[theorem]{Assumption}

\newcommand{\UH}{\mbox{\it UH}}



\ifFull \else 

\conferenceinfo{ACM GIS'10,}{November 2--5, 2010, San Jose, CA, USA}
\crdata{978-1-4503-0428-3/10/11}
\CopyrightYear{2010}

\fi


\begin{document}

\title{Privacy-Preserving Data-Oblivious Geometric \\
     Algorithms for Geographic Data}

\date{}

\ifFull
\author{
David Eppstein\3pt]
Dept.~of Computer Science \\
Univ.~of California, Irvine \\
goodrich(at)ics.uci.edu
\and
Roberto Tamassia\3pt]
Dept.~of Computer Science \\
Univ.~of California, Irvine \\
goodrich(at)ics.uci.edu
\and
Roberto Tamassia\0.5in]
\hbox{\quad}

\end{minipage}
\begin{minipage}[b]{0.22\linewidth}
\centering\includegraphics[width=\linewidth]{case1}\0.15in]
\hbox{\quad}
\end{minipage}\0.1in]

\centering\begin{minipage}[b]{0.22\linewidth}
\centering\includegraphics[width=\linewidth]{case3}
\end{minipage}
\begin{minipage}[b]{0.22\linewidth}
\scriptsize
Case (iii): .\\


\medskip
: Case f\\
\hbox{\qquad}, .\\
: Case g\\
\hbox{\qquad}, .
\end{minipage}
\qquad
\centering\begin{minipage}[b]{0.22\linewidth}
\centering\includegraphics[width=\linewidth]{case4}
\end{minipage}
\begin{minipage}[b]{0.22\linewidth}
\scriptsize
Case (iv): .\\


\medskip
: Case i1\\
\hbox{\qquad}, .\\
: Case h\\
\hbox{\qquad}, .
\end{minipage}
 \caption{The possible cases for the configurations of , ,
and . In some of the cases there are two possible locations for ,
, or  relative to the lines containing these edges, in which case we
use subscripts  or  to distinguish the two relative locations. For each
scenario, we list a set of comparisons and their results according to the
OvL classification, together with the resulting
classification of , , and/or~. Note that in Case~(ii)
of particular note is the 
comparison in Case~(iv), for this is an example of an OvL Case~h where we can
classify  as , since it is impossible for any edges of  to
be above  in this scenario.}
\label{fig:cases}
\end{figure}


Let , and let  and  respectively denote
the subarrays of  and  consisting of the points with indices at
multiples of~.
Let  be the subsequence of the recursively constructed upper hull
 consisting of the edges that have at least one
endpoint vertex in~. Define  similarly with respect to
 and~.
Thus,  and   have size~.
We perform a \emph{round} in our computation as follows:
\begin{enumerate}
\item
We perform an (oblivious)
brute-force computation to compare every pair of edges in
, so as to determine for each non-vertical edge  
in  (resp., ), 
the edge, , in  (resp., ) with smallest slope
greater than  and , the edge in  (resp., )
with largest slope less than .
\item
For each
edge , use Lemma~\ref{lem:test} to label  with , , or , as a
blackbox computation applied to each edge, with its associated edges  and
.
\item
Perform another brute-force comparison of every pair of edges 
in  to label edges  and , as  and , 
for some edge  whose blackbox computation determined these 
labels for  and .
\item
Perform a forward scan and reverse scan on  and  using 
a blackbox computation that
labels any edge to the left of an -labeled edge as  and any edge to the
right of an -labeled edge as~.
\end{enumerate}
Note that all of the above steps can be performed obliviously in  time.

\begin{lemma}
After the above round computation completes,
at most one of the subsequences  or  can contain edges labeled .
\end{lemma}
\begin{proof}
After each application of Lemma~\ref{lem:test} to an edge  in  or
, we either label 
with an  or  or we label  with an  and 
its associated edges  and  as  and .
Note that in the latter case the forward and reverse scans will then
completely label all the edges of the other list (not containing )
as  or ; hence, this list will contain no edges labeled .
That is, if we label any edge  as , then we label all the edges in the
other list as  or .
If, on the other hand, we don't label any edge  in  (resp., ) as
, then there are clearly no edges in  (resp., ) that are labeled
as~.
\qed
\end{proof}

Although at most one of  or  can 
have an edge labeled , the edges in this list may almost all be
labeled~.
Thus the above round computation will reduce the candidate 
tangent vertices in the representation of
one of  or  (but not necessarily both) to a subregion of size 
).
If it reduces , call it a \emph{red-1} reduction and
if it reduces , call it a \emph{blue-1} reduction.
A second application of the round computation will either reduce the other list
to a subregion of size  (i.e., it will be a red-1 or blue-1
reduction) or it will reduce the first subregion
to a single vertex of tangency, which we call a \emph{red-2} or \emph{blue-2}
operation, depending on whether it occurs to  or .
In either case, two more applications of the round computation will determine
the tangent edge between  and .

As described above this sequence of applications of the round computation is
non\-ob\-liv\-i\-ous, but it can be made oblivious by 
considering all valid sequences of red-1, blue-1, red-2, and blue-2, in turn.
One of these constant number of operation sequences 
will be the correct sequence to find the upper tangent. By trying all
these possibilities obliviously (with conditional no-ops for
paths not taken)
we will perform the one 
the leads to the determination of the tangent between  and
.

An important implementation detail is the oblivious
method for doing the reduction in a red-1 or
blue-1 operation. A red-2 or blue-2 operation, which reduces a set of
size  to an object of size  can be done obliviously
in a single scan using a constant-size register.
In the red-1 or blue-1 operation, we have an array  of size  
for which we want to isolate a subregion of size  based on the labels of its boundary elements, and copy it into a
buffer, , of size .
For each subregion, we read the boundary elements and use them to set a register flag, ,
that determines whether this is the region that should be copied.
We then read the -th element, , from our buffer, 
perform a conditional swap (based on ) with the -th element 
in this region of , and write the result back to .
Thus, in an oblivious way, we can copy a subregion of interest into the buffer
, with the total computation taking  time.

Summarizing, we can perform the determination of the upper tangent of 
and  obliviously in  time, including the scan of 
concatenated with  to relabel any vertices under this tangent with
an identifier for this tangent.
A similar construction applies to the
lower hull of . Thus, we have the following result.

\begin{theorem} \label{thm:convex-hull}
Given a set  of  points in the plane, we can 
obliviously construct a representation of
the convex hull of  in  time.
\end{theorem}

Using Theorem~\ref{thm:convex-hull}, we can then apply standard cryptographic circuit simulation
methods to derive a secure multiparty computation involving private data 
(e.g., see~\cite{bnp-fssmp-08,clos-uctms-02,da-smpcp-01,dz-passm-02,m-smpcm-06,mnps-fstpc-04}).
Hence, we obtain a secure two-party protocol for Alice and Bob to
determine which of their respective points belong to the convex hull
of the union of their  points with a communication complexity of
.

\begin{corollary} \label{cor:convex-hull} There is a secure two-party
  protocol that computes the convex hull of the union of two private
  sets of points of total size  with  communication
  complexity.
\end{corollary}



\section{Some Combinatorial Problems}
We next turn to oblivious algorithms for some combinatorial problems that
crop up in our methods for our other geometric algorithms.

\subsection{All Nearest Larger Values}
In the \emph{All Nearest Larger Values} (ANLV) 
problem~\cite{bsv-odlpa-93}, we are given an array
 of  numbers, such that, for each value, ,
we want to determine the values  and , where 
 is the largest index less than  with  and
 is the smallest index greater than  with .
As observed by Berkman {\it et al.}~\cite{bsv-odlpa-93},
this problem is actually a generalization of the problem of merging two
sorted lists,  and , since these lists can be merged by
solving an ANLV problem for an array that consists of 
a reversal of  followed by .
Our oblivious method for solving the ANLV problem,
where we assume without
loss of generality that the values are distinct, 
is different from that of Berkman {\it et al.}~and is as follows.
\begin{enumerate}
\item
Build a complete binary tree, , ``on top'' of the items in 
and perform a bottom-up tournament computation to compute, for each
 in , the value, , which is the maximum value stored in a
descendent of  in .
This is a straightforward oblivious computation that takes  time.
\item
For each leaf  in ,
let  denote the lowest node in  
such that  is a left sibling of an ancestor of  in  and
, where  is the value in  associated with .
Likewise, let  be denote the lowest node in  
such that  is a right sibling of an ancestor of  in  and
, where  is the value in  associated with .
We compute  and , which are initially null for each ,
in a divide-and-conquer computation, with respect to a node  in .
In this computation,
we recursively compute the  and  labels for nodes in the subtrees
rooted at 's left and right children,  and , producing lists, 
and , of labeled descendents of  and .
Then, we scan  to assign
 for each  such that  and  was previously null. 
Also, we scan  list to assign
 for each  such that  and  was previously null. 
We then concatenate these two lists of labeled nodes (some of which are still
null) to create the list  for , which is passed up to 's parent.
This step runs in  time.
\item
For each node  in  with left child  and right child ,
we perform a scan of  to find the smallest 
value  such that , if it exists, and we scan back through
 to label the value  in 's list such that 
to show that  (if it exists)
is the nearest larger value to the left of .
Likewise,
we perform a scan of  to find the smallest 
value  such that , if it exists, and we scan back through
 to label the value  in 's list such that 
to show that  (if it exists)
is the nearest larger value to the right of .
(These scans are used to take care of the boundary values for each node.)
Finally, when we are done with all the scans, we perform a sorting 
step to report back to each node the labels that have been found for it.
This step runs in  time.
\item
For each element  stored in a leaf of the binary tree , 
let us create two tuples, 
 
and
,
where  is the index of the value  in , and  and  are
the left and right ANLV's for  (most of which are
probably null at this point).
Perform an oblivious sort of all these tuples, using a lexicographic
ordering rule, to produce the sorted list, , of such tuples.
This step takes  time.
\item
Scan the list  in reverse order. During this scan we 
maintain three registers,
,  and . The register  is a label of the current node, , for
which we are computing ANLV's for, that is, the first coordinate of the
tuples we are scanning.
The scan for each  is essentially a merge of its left and right children's
lists of nodes whose ANLV is determined by this merge at .
The register  is maintained to be
the smallest ``left'' value in a tuple with this  as
its first () coordinate.
The register  is maintained to be
the smallest ``right'' value in a tuple with this  as
its first () coordinate.
Whenever we encounter a tuple, if it is a ``right'' tuple, we identify 
its left ANLV as , and if it is a ``left'' tuple, we identify
its right ANLV as , assuming we have not already determined this value
previously (which coincides with the point where we reset the register ).
This scan can be done obliviously in  time.
\item
Perform one more sort to bring together the computed 
left and right ANLV's for each node  in .
This step can be done obliviously in  time, and it completes the
algorithm.
\end{enumerate}
Thus, we have the following.

\begin{theorem}
Given an array  of  values, we can obliviously 
solve the ANLV problem for  in  time.
\end{theorem}

\subsection{List Ranking}
In the \emph{list ranking}
problem~\cite{am-dplr-88,cv-dctao-86},
we are given a linked list, , stored in the
records of an array of size , for which we want to compute, for each node
 the number of nodes from  to the end of the list, using pointer
hopping as the distance metric.


\begin{theorem}
\label{thm:list-rank}
Given a linked list  of  nodes, we can obliviously perform a list
ranking of the nodes in , using a computation that always runs in 
 time and succeeds with high probability.
\end{theorem}

\begin{proof}
To solve the list ranking problem on a list  of  nodes
obliviously, we perform the following actions:
\begin{enumerate}
\item
Create for each node  in  a field, , which stores an indication
of the distance from  to the end of .
Initially, each .
\item
Generate a random bit, , for every node  in .
\item
Perform two oblivious sorts so as to ``link out'' each node  
with  that follows a node  with , storing with  a
reference to name of the node, , that currently follows  in .
In addition, with this link out step, we update .
\item
Repeat the previous two steps a constant, , number of times, until it is
likely, with high probability, that the connected part of 
has at most half as many nodes.
\item
If ,
perform an oblivious sorting
step and compression to reduce the working storage used for  to be half
its previous size.  Then repeat the above computation starting with Step~2.
\item
If , then perform  link-out steps, where we apply
a link-out operation, like the one above, for every node in parallel, using
oblivious sorts to perform the actions in an oblivious fashion.
The total running time for all these 
actions is .
\item
Reverse the above link-out steps, in reverse order, so that,
for each node  that was linked out in step , 
we perform two oblivious sorting steps
to communicate the information needed so that we can update ,
where  was the node that followed  when it was linked out.
\end{enumerate}

Since we reduce the number of nodes, and the working storage for , 
by half every
 steps, with high probability, and we then reverse these actions 
to finally solve the
list ranking problem, we get that the running time of this method
is a geometric sum that is .
Moreover, since we terminate the halving process and switch to a parallel
link-out process when , we get that this
method succeeds in computing a list ranking for  with high probability.
\qed
\end{proof}

\subsection{Tree Contraction}
In a
\emph{tree contraction}~\cite{adkp-sptca-89,mr-ptcia-85,rmm-lrptc-93}
computation, we are given a proper binary tree  
such that each leaf node
is associated with a value and each internal node is associated with an
arithmetic
operation to be computed on its two children.
The goal is to efficiently 
compute the value of each node in  in an oblivious fashion, 
even if the height of  is .

\begin{theorem}
\label{thm:tree}
Given a binary arithmetic tree, , with  nodes,
we can obliviously compute the value of each internal node of  in
 time, in a computation that succeeds with high probability.
\end{theorem}

\begin{proof}
Adapting a parallel algorithm of Abrahamson {\it et al.}~\cite{adkp-sptca-89},
we can solve the tree contraction problem obliviously as follows.
\begin{enumerate}
\item
Perform a list ranking operation to number the leaves of  from  to .
Using the algorithm described below, this step takes  time and succeeds
with high probability.
\item
For each node  in  that is an odd-numbered leaf 
and a left child of its parent, link
out  and its parent, making 's sibling to be the new child of 's
grandparent. In doing this operation, record for  and its parent
the iteration it is removed and the names of the grandparent and sibling
nodes at this point.
In addition, in the link-out operation, we label the child-parent edge with
an -sized algebraic operation to apply in going from the child value to
the parent (which is composable when we combine previously-computed edges in
a link-out).
This step can performed obliviously using  oblivious sorting steps.
\item
For each node  in  that is an odd-numbered leaf 
and a right child of its parent, link
out  and its parent, making 's sibling to be the new child of 's
grandparent. In doing this operation, record for  and its parent
the iteration it is removed and the names of the grandparent and sibling
nodes at this point, along with any edge updates as in the previous step.
This step can performed obliviously using  oblivious sorting steps.
\item
If ,
divide the leaf number of each leaf node by  and repeat the above two
steps.
\item
Reverse the above actions to compute the value of each internal node.
\end{enumerate}
This completes the proof.
\qed
\end{proof}



\section{Quadtrees and Well-Separated Pair Decompositions}
Having described our methods for some fundamental combinatorial problems,
we describe in this section our oblivious algorithms for 
constructing a compressed quadtree and for forming a well-separated pair
decomposition.

\subsection{Constructing a Compressed Quadtree}
A \emph{compressed quadtree}~(e.g., 
see~\cite{bet-pcqqt-99,c-wspdl-08,s-sdsqo-89}), 
for a set  of  points in the
plane, normalized to have the unit square, ,
as a bounding box, is
defined as follows.
A \emph{quadtree} (e.g., see~\cite{s-sdsqo-89}) 
for 
is defined recursively, where we create a node  for the 
current bounding box and, if this bounding box has more than a given
threshold number of points of , 
then we divide this box into four equal-sized boxes 
as quadrants, and we recursively construct subtrees for each non-empty quadrant,
with the nodes for these non-empty quadrants having  as their parent.
(See Figure~\ref{fig:quadtree}.)
If we then compress any chains of nodes in this quadtree
that have only one child, then we get the compressed quadtree.
This definition is clearly not something that leads to an oblivious
construction algorithm, of course, but we can in fact
construct a compressed quadtree for  obliviously in  time.

\begin{figure}[hbt]
\centering\includegraphics[width=3.3in]{quadtree.pdf}
\caption{\label{fig:quadtree} A (region) quadtree for a set of points.
(Public-domain image by David Eppstein.)}
\end{figure}

An alternative method for constructing a compressed quadtree, as
observed by several researchers (e.g.,
see~\cite{bet-pcqqt-99,c-wspdl-08,s-sdsqo-89}), 
is based on a sorting of the points of  according to the 
\emph{interleaving order}.
In the interleaving order, we take each point  and interleave the bits
for  with the bits for  is a standard shuffling, and we compare points
according to this order. This order can also be interpreted
geometrically~\cite{c-wspdl-08} for the sake of a comparison-based sorting
algorithm.
Once we have the points of  stored in an array  according to the
interleaving order, we note, as shown by 
Bern {\it et al.}~\cite{bet-pcqqt-99}, that the nodes
contained in any compressed quadtree box 
form a contiguous subsequence in .
Moreover, we can label each transition between two adjacent points in 
with the box that is formed along that transition, and we can then identify
the compressed quadtree box that contains each point  in  by performing
an ANLV computation, where we use box size to determine values in
this ANLV computation.
Given this information, we can perform a postprocessing 
step consisting of two oblivious sorting
steps to determine the adjacency information between the parent and child
nodes in the compressed quadtree.
Thus, we have the following.

\begin{theorem}
\label{thm:quadtree}
Given a set  of distinct points in the plane, we can obliviously construct a
compressed quadtree for  in  time.
\end{theorem}

\subsection{Well-Separated Pair Decomposition}
Another important geometric computation is
the construction of a well-separated pair decomposition (WSPD) for a set 
of  points in the plane.
In a WSPD~\cite{ck-dmpsa-95}, 
we are given a parameter  for
which we want to construct a set of pairs,
, 
such that every pair of points  and  are represented by a pair
 such that  and , or vice versa,
and such that there are balls of radius  containing  and 
respectively so that these balls are of distance at least  apart.
In our applications we choose  to be a constant, e.g.,  will do.

We should note that some authors also insist 
that each pair of points  and 
be represented exactly once in some  pair in WSPD.
But duplicate
representation is actually not a problem for most WSPD applications,
including the ones we consider, so we don't make this additional
requirement. What is essential in our definition 
is that the total number of pairs in a WSPD be linear.

Given a compressed quadtree for a set  of  points
in the plane, Chan~\cite{c-wspdl-08}, shows that a WSPD
can be constructed in  time by a simple
(non-oblivious) recursive search algorithm defined on the nodes of the
compressed quadtree. 
Using a technique of Callahan and Kosaraju,
Chan shows that the time and combinatorial complexity for his algorithm is 
 by using a packing argument, which shows that the number of compressed
quadtree boxes that are no smaller than a box  but are too close to be
candidates for a well-separated pair with  is bounded by a constant
depending on .

We define an alternative, oblivious construction algorithm for a
well-separated pair decomposition by turning this construction and argument
``on its head.''
That is, we use the packing argument itself to construct the WSPD.
In particular, for each box  in the compressed quadtree, , there are
 boxes in the uncompressed quadtree, ,
that are the same size as  and are not well-separated from . And for
each such box, , there are  immediate (children and grandchildren)
descendents of the edge in  corresponding to where  is located in
.
These immediate descendents and the children of  in 
together form candidates for well-separated pairs. And the collection 
of all such sets of candidate pairs form a superset of the pairs that are
considered by the WSPD algorithm of
Chan~\cite{c-wspdl-08}.
Thus, if we can consider all such pairs and only keep the ones that form
well-separated pairs, then we can construct a WSPD
of size~.

The challenge is to collect all such pairs.
We do this as follows.
\begin{enumerate}
\item
For each box  in the compressed quadtree, ,
form the set,  of  boxes 
that are the same size as  in the uncompressed quadtree and are not
well-separated from .
\item
In parallel,
for each  in , pick a box  in  that was previously
unconsidered, 
and create two points  and  inside  that do not fall
inside the same child box of  in the uncompressed quadtree, .
Call these points ``dummy points.''
\item
Create a compressed quadtree, , for all the points in  together
with the dummy points created in the previous step. Note that the
box  is in , even if it is not in .
\item
Label each point in  with a  and each dummy point with a  and
perform an tree compression on , where each internal-node operation
is a binary OR, to determine the binary value of
each internal node in . Note that the nodes of  that also
exist in  will have at least two children that have binary values equal to
. 
\item
Remove all the nodes with binary values equal to  from  and
construct an Euler tour of its edges, perform a list ranking in that Euler
tour, and then an ANLV computation on the nodes in this list using node
degree as the item values.
This computation gives us, for each box  in a  set, the
highest nodes in  whose boxes are contained in .
These nodes and their children, together with the children of , form
candidates for well-separated pairs. Identify which ones are indeed
well-separated and compress them into a list of answers produced in this
round.
\item
Repeat Steps~2 through 5 above until we have considered each 
box in a set , for its box .
\end{enumerate}

Each of the above steps runs in  time, with the list ranking and
tree evaluation steps succeeding with high probability. Likewise, there
are only  iterations to this algorithm. So we get the following.

\begin{theorem}
\label{thm:wspd}
Given a set  of  points in the plane, and a compressed quadtree 
for , we can construct a
well-separated pair decomposition for , with each set being
associated with a node in , in  time with an
oblivious computation that succeeds with high probability.
\end{theorem}



\section{Closest Pairs and All Nearest Neighbors}
Having presented all the above algorithmic techniques, we are now
ready to describe our oblivious algorithm for solving the all nearest
neighbors problem.

So, let us assume we are given a set  of  points in the plane
for which we want to solve the all nearest neighbors problem.
At a high level it is an oblivious adaptation of a parallel all
nearest neighbors algorithm of 
Callahan and Kosaraju~\cite{ck-dmpsa-95}.
\begin{enumerate}
\item
Construct a compressed quadtree  for
the points of , using the method of Theorem~\ref{thm:quadtree}.
\item
Construct a well-separated pairs decomposition (WSPD), based on ,
using the method of Theorem~\ref{thm:wspd}.
\item
Discard each pair  in the WSPD if neither  nor  is a
singleton set.
(Note: if all we want is a closest pair, then we can skip the remaining steps
and find the closest of all the singleton-singleton pairs in the WSPD.)
\item
For each box  in the WSPD for which there is at least one remaining pair,
, construct the set  of all such points, . We represent this information obliviously as a collection of pairs  where  is a point and  is a box, padded with null items.
\item
For each box , partition the plane into a set of  wedges having the center  of  as their apex, and prune  to contain only the closest point to  within each wedge (by replacing the pairs representing other points by null items), with the number of wedges chosen according to the parameters of the WSPD so that for each pruned point there is another point of  that is closer to it than every point in  is. The set of remaining points
in each set  will have size .
\item
Using a tree contraction algorithm of 
Callahan and Kosaraju~\cite{ck-dmpsa-95}, 
construct for each leaf  of , which is
associated with a point , the set , which is the set of 
all points  such that  is in the WSPD for an ancestor of
 in  and such that 's distance to  is no larger than the minimum
distance from  to other points in .
In other words,  consists of all those points of  that
could have  as a nearest neighbor.
This step takes  time to implement obliviously, by
Theorem~\ref{thm:tree}.
\item
For each point  in a set , construct the pair . Sort
all these pairs so as to bring together, for each point , those
points that could be a nearest neighbor to . Then perform a scan
of this list to determine, for each , its nearest neighbor.
This step takes  time.
\end{enumerate}
Each of the above steps can be implemented in  time,
either because of the specific results from the referenced theorems,
or because the step is easily performed obliviously by making a
constant number of calls to an oblivious sorting routine.

\begin{theorem} \label{thm:neighbors}
Given a set  of  points in the plane, we can compute the
nearest neighbor in  for each point in  and find a closest
pair of points in  with an oblivious
computation running in  time.
\end{theorem}

Starting from the data-oblivious algorithms of Theorem~\ref{thm:neighbors},
we can then apply standard cryptographic circuit simulation
methods to derive a secure multiparty computation involving private data 
(e.g., see~\cite{bnp-fssmp-08,clos-uctms-02,da-smpcp-01,dz-passm-02,m-smpcm-06,mnps-fstpc-04}).
Hence, we obtain secure two-party protocol for Alice and Bob to
compute either the closest pair or 
the nearest neighbor in the union of their  points for
each of their respective points, but otherwise learn nothing about
the other person's points. 

\begin{corollary} \label{cor:neighbors}
  There is a secure two-party protocol that computes the all nearest
  neighbors and a closest pair in the union of two private sets of
  points of total size  with  communication complexity.
\end{corollary}

The result of Corollary~\ref{cor:neighbors} is perhaps counter-intuitive,
in that one might, at
first, believe that such a computation reveals
all of the points in question. 
However, if Alice and Bob's respective sets of points are relatively
well-separated, then each of them would learn almost nothing from a
two-party all nearest neighbors computation, for, in this case, each
of their respective points has a nearest neighbor 
in its same original set.



\section{Conclusion}
We have given efficient oblivious algorithms for a number of
geometric problems, which are natural problems to arise in
privacy-preserving protocols for computing functions of points that
are derived from the coordinates of actors in various location-based
services.
We have also given oblivious algorithms for several fundamental combinatorial
problems.
There are a host of open problems, however, that might be
of interest in privacy preserving computations, including the following:
\begin{itemize}
\item
Given a set  of  vertical and horizontal
line segments, can one obliviously compute in  time
the number of pairs of segments in  that intersect?
\item
Given a set  of  points in the plane, can one construct a
representation of the Delaunay triangulation of  obliviously in
 time?
\item
Given a set  of  points in , can one construct a
representation of the convex hull of  obliviously in
 time?
\item
Given a simple polygon  of size , can one construct a
representation of a triangulation of  obliviously in 
time? If so, is this the fastest time possible for an oblivious
algorithm?
\end{itemize}



\subsection*{Acknowledgments}
We would like to thank Wenliang (Kevin) Du for several stimulating
discussions regarding the connections between oblivious algorithms
and privacy-preserving protocols.
This research was supported in part by NSF grants 
0830149, 0830403, 1011840, and 1012060
and by ONR under MURI grant N00014-08-1-1015.



{\raggedright
\bibliographystyle{abbrv}
\bibliography{goodrich,k_anonymity,geom,extra}
}

\end{document}
