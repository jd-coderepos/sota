\subsection{Availability sets}
\label{sec:availSets}

The models we have considered so far have considered the availability of a
\emph{single} event at a time.  If we consider the availability of a
\emph{set} of events, can we distinguish more processes?  The answer turns out
to be yes, but only with processes that can either diverge or that exhibit
unbounded nondeterminism (a result which was surprising to me). 

We will consider actions of the form , where  is
a set of events, representing that all the events in~ are simultaneously
available.  We can adapt the derived operational semantics appropriately:



For convenience, we define

Traces will then be from .  We can extract traces
of this form from the derived operational semantics as in
Definition~\ref{def:traces} (writing  and
 for the corresponding relations).

We call this model the Availability Sets Traces Model, and will sometimes refer
to the previous model as the Singleton Availability Traces Model, in order to
emphasise the difference.
\begin{definition}
The \emph{Availability Sets Traces Model}  contains
those sets  that satisfy the
following conditions.
\begin{enumerate}
\item
 is non-empty and prefix-closed.

\item
\label{healthy:offer_duplicate-sets}
  actions can always be removed from or duplicated
within a trace:


\item
\label{healthy:offer_implies_event-sets}
If a process can offer an event it can perform it:


\item
\label{healthy:event_implies_offer-sets}
If a process can perform an event it can first offer it:


\item
The offers of a process are subset-closed


\item
Processes can always offer the empty set

\end{enumerate}
\end{definition}
\begin{lemma}
\label{lem:opsem-healthy-sets}
For all processes~,\,   is an
element of the Availability Sets Traces Model.
\end{lemma}





\paragraph{Compositional semantics}
We give below semantic equations for the Availability Sets Traces Model.  Most
of the clauses are straightforward adaptations of the corresponding clauses in
the Singleton Availability Traces Model.

For the parallel operators and external choice, we define an operator
 such that  gives all
traces resulting from traces~ and~, synchronising on events and
offers of events from~.  The definition is omitted due to space
restrictions. 


For relational renaming, we lift the renaming to apply to  actions, by
forming the subset-closure of the relational image:

We again lift it to traces pointwise.  

The semantic clauses are as follows.
\begin{eqns}
\Tracesap[STOP]  = \Tracesap[\Div] = (\offer \set{})^* ,
\nl
\Tracesap[a \then P] = 
  
\nl
\Tracesap[P \timeout Q] = 
  
\nl
\Tracesap[P \intchoice Q] = 
  \Tracesap[P] \union \Tracesap[Q], 
\nl
\Tracesap[P \extchoice Q] = \\*
\gap
  
\nl
\Tracesap[{P \parallel[A][B] Q}] = 
  
\nl
\Tracesap[P \interleave Q] = 
  \set{tr | \exists tr_P \in \Tracesap[P],  tr_Q \in \Tracesap[Q] \spot 
     tr \in tr_P \parallel[\set{}]^{\power} tr_Q },
\nl
\Tracesap[P \hide A] = 
  \set{ tr_P \hide A | 
    tr_P \in \Tracesap[P] 
    \land \forall X \spot \offer X \IN tr_P \implies X \inter A = \set{}},
\nl
\Tracesap[P\rrn{R}] =
  \set{tr | \exists tr_P \in \Tracesap[P] \spot tr_P~R~tr },
\nl
\Tracesap[\mu X \spot F(X)] = 
  \mbox{the -least fixed point of the
  semantic mapping corresponding to~.}
\end{eqns}

\begin{theorem}
The semantics is congruent to the operational semantics:

\end{theorem}



\paragraph{Full abstraction}
In order to prove a full abstraction result, we extend our class of tests to
include a test of the form , which tests whether all the
events in~ are available, and if so acts like the test~.  Formally, this
test is captured by the following rule.
\begin{infrule}
P \transd{\offer A} P
\derive
\Ready A \amp T \parallel P \trans{\tau} T \parallel P
\end{infrule}
Given , we can construct a test 
that detects the trace~ as follows.

The full abstraction proof then proceeds precisely as in
Section~\ref{sec:offer}.

We can prove a no-junk result as in Section~\ref{sec:offer}.  Given
trace~, we can  construct a process~ as follows:

Then the traces of  are just~ and those traces implied from~
by the healthiness conditions.  Again, given an element  from the
Availability Sets Traces Model, we can define ; then . 




\paragraph{Distinguishing power}
We now consider the extent to which the Availability Sets Model can
distinguish processes that the Singleton Availability Model can't. 
\begin{example}
\label{example:sets_vs_singles}
The Availability Sets Traces Model distinguishes the processes

since just  has the trace .  However, these are
equivalent in the Singleton Availability Traces Model; in particular, both can
perform 
arbitrary sequences of  and  actions initially.\end{example}
The process~ above can diverge (i.e., perform an infinite number of
internal~ events corresponding to
timeouts).  We can obtain a similar effect without divergence, but using
unbounded nondeterminism.  
\begin{example}
\label{example:sets_vs_singles_2}
Consider 

Then  (from the previous example) and   are distinguished in the
Availability Sets Traces Model but not the Singleton Availability Traces
Model.
\end{example}

For finitely nondeterministic, non-divergent processes, it is enough to
consider the availability of only \emph{finite} sets, since such a process can
offer an infinite set~ iff and only if it can offer all its finite subsets.
However, for infinitely nondeterministic processes, one can make more
distinctions by considering infinite sets.
\begin{example}
\label{example:finite_vs_infinite}
Let  be an infinite set of events.  Consider the processes

Then these have the same finite availability sets, but just the former has
all of~ available.
\end{example}

\begin{prop}
\label{prop:singleton_vs_sets}
If  and~ are non-divergent, finitely nondeterministic processes, that
are equivalent in the Singleton Availability Model, then they are equivalent
in the Availability Sets Model.
\end{prop}
\begin{proof}
Suppose, for a contradiction, that  and  are non-divergent and finitely
deterministic, are equivalent in the Singleton Availability Model, but are
distinguished in the Availability Set Model.  Then, without loss of
generality, there are traces~ and , and set of events~ such that
 is a trace of~ but not of~.  By the
discussion in the previous paragraph, we may assume, without loss of
generality, that  is finite, say .  Since 
is non-divergent and finitely-nondeterministic, there is some bound, ~say,
on the number of consecutive  events that it can perform after~.
Since  can offer all of~ after , it can also offer any individual
events from~, sequentially, in an arbitrary order.  In particular, it has
the singleton availability trace

Since  and  are, by assumption, equivalent in the Singleton Availability
model,  also has this trace.  ~must perform at most   events
within the sub-trace .  This
tells us that there is a sub-trace within that, of length~, containing no
~events. Within this sub-trace there are no state changes (i.e., there
are only self-loops corresponding to the  actions), and so all the
 are offered in the same state.  Hence  is an availability set trace of~, giving a
contradiction.
\end{proof}



\paragraph{Bounded sets}
We can consider some variants on the Availability Sets Traces Model.

First, let us consider the model  that places a limit of size~ upon
availability sets.  It is reasonable straightforward to produce 
compositional semantics for such models, and to adapt the full abstraction and
no-junk results.   It is perhaps surprising that such a semantics is
compositional, since a similar result does not hold for stable
failures~\cite{BL04} (although it is conjectured in~\cite{awr:revivals} that
this does hold for acceptances). 


Clearly, , and
 (the standard traces model).
Examples \ref{example:sets_vs_singles} and \ref{example:sets_vs_singles_2}
show that  is finer that .
We can generalise those examples to show that each model  is
finer than . 
\begin{example}
\label{example:monotonic-k}
Let  be a set of size~.  Consider

Then  and~ are distinguished in  since only  has
the trace .  However they are equivalent in
: in particular, both can initially perform any trace of
offers of size~. \end{example}

The limit of the models~ considers arbitrary \emph{finite}
availability sets; we term this .  The model
 distinguishes the processes  and  from
Example~\ref{example:monotonic-k}, for all~, so is finer than each of the
models with bounded availability sets.  As shown by
Example~\ref{example:finite_vs_infinite},  is coarser
than .


In fact, for an arbitrary infinite cardinal , we can consider the
model~ that places a limit of size~ upon availability sets.
Example~\ref{example:finite_vs_infinite} showed that considering finite
availability sets distinguishes fewer processes than allowing infinite
availability sets, i.e.~.  The following
example shows that the models become finer as  increases.
\begin{example}
\label{example:monotonic-infinite-kappa}
Pick an infinite cardinal , and pick alphabet  such that
.  Then the processes

are distinguished by the model , since only  can offer
sets of size~.  However, for , they are not
distinguished by the model ; for example, if  has the
trace  in~, then
, for each~; but also  has , so  can perform this trace
by picking~ in the nondeterministic choice.
\end{example}
(In fact, this example shows that these models ---like the cardinals--- form a
proper class, rather than a set!)  In most applications, the alphabet~
is countable; these models then coincide for processes with such an alphabet.
The model  distinguishes the processes  and
 from Example~\ref{example:monotonic-infinite-kappa}, for
all~, so is finer than each of the models .

Summarising:










\subsection{Combining the variations}


We can combine the ideas from Sections~\ref{sec:bounded-number}
and~\ref{sec:availSets} to produce a family of models , where:
\begin{itemize}
\item 
 is either a natural number~ or infinite cardinal~, indicating
  an upper bound on the size of availability sets, or the symbol 
  indicating arbitrary finite availability sets are allowed, or the symbol
   indicating arbitrary availability sets are allowed;

\item
 is either a natural number~, indicating an upper bound on the number
  of availability sets between successive standard events, or the symbol
   indicating any finite number is allowed.
\end{itemize}



\begin{figure}[htbp]
\begin{center}
\ 
\xymatrix @R=8.5mm @C=14.85mm @!0{
& & & & & & & & & *[c]{\A^{\power} \equiv \A_\finite^{\power}} \\
\\
& & & & & & & *[c]{\A^{\aleph_2} \equiv \A_\finite^{\aleph_2}} \ar@{.}[uurr] & & 
  \A_3^{\power} \ar@{.}[uu] 
\\
& & & & & & *[c]{\A^{\aleph_1} \equiv \A_\finite^{\aleph_1}} \ar@{-}[ur] & & & 
  \A_2^{\power} \ar@{-}[u] 
\\
& & & & & *[c]{\A^{\aleph_0} \equiv \A_\finite^{\aleph_0}} \ar@{-}[ur] & & 
  \A_3^{\aleph_2} \ar@{.}[uu] \ar@{.}[uurr] & & \A_1^{\power} \ar@{-}[u]
\\
& & & & *[c]{\A^\finite \equiv \A_\finite^\finite} \ar@{-}[ur] & & 
  \A_3^{\aleph_1} \ar@{.}[uu] \ar@{-}[ur] &
  \A_2^{\aleph_2} \ar@{-}[u] \ar@{.}[uurr] 
\\
& & & & & \A_3^{\aleph_0} \ar@{.}[uu] \ar@{-}[ur] & 
  \A_2^{\aleph_1} \ar@{-}[u] \ar@{-}[ur] & \A_1^{\aleph_2} \ar@{-}[u]
  \ar@{.}[uurr]
\\
& & *[c]{\A^3 \equiv \A_\finite^3} \ar@{.}[uurr] & & 
  \A_3^\finite \ar@{.}[uu] \ar@{-}[ur] & 
  \A_2^{\aleph_0} \ar@{-}[u] \ar@{-}[ur] & \A_1^{\aleph_1} \ar@{-}[u]
  \ar@{-}[ur] 
\\
& *[c]{\A^2 \equiv \A_\finite^2} \ar@{-}[ur] & & & 
  \A_2^{\finite} \ar@{-}[u] \ar@{-}[ur] &  \A_1^{\aleph_0} \ar@{-}[u] \ar@{-}[ur] 
\\
{\A \equiv \A^1 \equiv \A_\finite^1\hspace{5mm}} \ar@{-}[ur] & & 
  \A_3^3 \ar@{.}[uu] \ar@{.}[uurr] & & 
  \A_1^{\finite} \ar@{-}[u] \ar@{-}[ur] \\
& \A_3^2 \ar@{.}[uu] \ar@{-}[ur] & \A_2^3 \ar@{-}[u] \ar@{.}[uurr] & 
\\
{\A_3 \equiv \A_3^1} \ar@{.}[uu] \ar@{-}[ur] & \A_2^2 \ar@{-}[u] \ar@{-}[ur] & 
   \A_1^3 \ar@{-}[u] \ar@{.}[uurr] 
\\
\A_2 \equiv \A_2^1 \ar@{-}[u] \ar@{-}[ur] & \A_1^2 \ar@{-}[u] \ar@{-}[ur] 
\\
\A_1 \equiv \A_1^1 \ar@{-}[u] \ar@{-}[ur] 
\\
{\A_0 \equiv \A_\n^0 \equiv \A_0^\k \equiv \mathcal{T}} \ar@{-}[u]
}\ 
\end{center}
\caption{The hierarchy of models\label{fig:hierarchy}}
\end{figure}

If  or  then  is just the standard traces model.  Further,
 and .

We can show that this family is ordered as the natural extension of the
earlier (one-parameter) families; the relationship between the models is
illustrated in Figure~\ref{fig:hierarchy}.  In particular, these models are
distinct for .  We can re-use several of the earlier examples to
this end.
Example~\ref{example:models-monotonic-n} shows that for each~

The following example generalises Example~\ref{example:monotonic-k}.
\begin{example}
Let  and  be positive natural numbers.  Let  be a set of size
.    Consider

Let  be a partition of~, where each  is of
size~.  Then  is a trace of~
but not of~, so these processes are distinguished by~.
However, the two processes are equivalent in~.  Hence
.  Further,
 distinguishes these processes, for all
(finite)~ and~, so .
\end{example}
Example~\ref{example:finite_vs_infinite} shows that
 for each infinite
cardinal~ and for each~.  Further,
Example~\ref{example:monotonic-infinite-kappa} shows that if  are two infinite cardinals, then .








