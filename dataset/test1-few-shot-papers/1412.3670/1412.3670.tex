
\subsection*{Proof of Theorem~\ref{reach-dec-thm}}


We prove the undecidability by constructing a structured BMS  with 2 clocks and one variable 
that simulates the 2 counter machine. We prove that the scheduler has a winning strategy 
to reach   iff the two counter machine halts. 
Our construction of  is such that we have a gadget corresponding to each instruction 
in the two counter machine.  We consider , and 
 as given.  Modes in the target set  
are denoted by a double circle. 

Let the single variable be denoted , and let  be the clocks. 
On entry into any gadget, the value of the variable  is  where  are the 
current values of the two counters, and the clocks  are zero. 
\begin{enumerate}
\item Simulation of an increment instruction ;  goto  . 

The gadget simulating the increment  instruction can be seen in Figure \ref{inc-c1}.
The gadget is entered with . The 
locations in the gadget contain the name of the location as well as 
the rate (possibly a set of rates, or an interval of rates)  of the variable , as the case may be. 
Let us denote by  the value . 
A non-deterministic amount of time is spent at location . 
The ideal time to be spent here is , so that 
  is updated from  to , reflecting 
 the correct new counter values.  is reset on 
 going to location . A time of one unit is spent at location .
 The value of  is unchanged during this process due to the self loop 
 on . There are three possible rates that the environment can 
 give to the scheduler, namely 100, -100 or 0 at location . 
 The scheduler can go to any of the gadgets  or to the location  . 
   

\begin{figure}[h]
\begin{center}
\scalebox{0.7}{
\begin{tikzpicture}[->,>=stealth',shorten >=1pt,auto,node distance=1.8cm,
  semithick]
  \tikzstyle{every state}=[minimum size=3em,rounded rectangle]
  
  \node[initial,initial text={},state] (A) {} ;
   \node[state] at (3.5,0) (B) {} ;
\node[state, player2] at (1,3) (chk1) {};
\node[state, player2] at (1,-3) (chk2) {};
\node[state] at (5,2) (cont) {};
      \path (A) edge node[above] {} (B);
      \path (A) edge node[below] {} (B);
        \path (B) edge[loop below]  node[above] {}node[below]{} (B);
                
  \path (B) edge  node[left]{} node[right]{} (chk1);
  \path (B) edge  node[left]{} node[right]{} (chk2);
  \path (B) edge  node[left]{} node[right]{} (cont);
  
  \end{tikzpicture}}
\caption{Simulation of increment  instruction }
\label{inc-c1}
\end{center}
\end{figure}

Assume that the time spent at  is  for some .
In this case,  and . The environment
can force a check of the scheduler and catch his mistake, by choosing a rate of 100 at location .
 This would make . If the scheduler wants to win, 
 he must reach a mode in , with the value of  
 in . The best thing for the scheduler to do at this point
 is to choose   as his next location, since 
 it allows the value of  to come back to . 
 If the scheduler chooses to go to , he will be worse off, making  even bigger, 
 and if he chooses , the environment can make sure that 
 the scheduler never wins by choosing the rate 100 in all future gadgets.  

Lets thus assume that the scheduler chooses to goto the gadget  in Figure \ref{inc-c1-more}. 
On entry, we have ,  and . 
At location , the value of  remains unchanged,  grows to 1 and is reset, 
and  becomes . At location , a 
time  is spent, obtaining  
and .
If  and , then the scheduler has already lost, since 
adding 3 more to  at location  does not help.  
Consider now the case that  and . 
At location , a time  of one unit is spent, and the environment can choose a 
rate as close to  as he wants : in particular, he can choose a rate 
that is larger than , making the value of , for some 
. This means the scheduler 
can never reach a point in the ball , 
even after adding 3 to  at location . 

If , then irrespective of the rate  chosen by the environment, 
the value of  is , after adding 3 to  at location . 
Thus, if the scheduler made no mistake, 
he reaches a point inside the chosen ball.  





\begin{figure}[h]
\begin{center}
\scalebox{0.7}{
\begin{tikzpicture}[->,>=stealth',shorten >=1pt,auto,node distance=1.8cm,
  semithick]
  \tikzstyle{every state}=[minimum size=3em,rounded rectangle]
  
  \node[initial,initial text={},state] (A) {} ;
        \path (A) edge[loop above]  node[above] {}node[below]{} (A);
   \node[state] at (3.5,0) (B) {} ;
\path (A) edge  node[above] {}node[below]{} (B);
\node[state] at (1,-3) (C) {} ;
\node[state] at (3.5,-3) (D) {} ;
\node[accepting,state] at (-2,-3) (T) {} ;

\path (B) edge  node[above] {}node[below]{} (D);
\path (D) edge  node[above] {}node[below]{} (C);
\path (C) edge  node[above] {}node[below]{} (T);
  \end{tikzpicture}}
\caption{The gadget }
\label{inc-c1-more}
\end{center}
\end{figure}

Now consider the case when the scheduler spends an amount of time , 
for some  at location  in Figure \ref{inc-c1}. Then we have 
 and . At location  in Figure \ref{inc-c1}, as 
seen above, the environment can assign any of the rates 100, -100 or 0 to the scheduler.
If the environment wishes to catch the scheduler's mistake, a rate of -100 will be assigned. The scheduler, 
if he chooses to goto  or , will surely lose, since 
the value of  will decrease further, and will never reach a value in ; 
likewise, if the scheduler chooses , the environment can forever give a rate of -100. 
The best choice for scheduler is therefore, to pick . The gadget  is 
given in Figure \ref{inc-c1-less}.  
 
 On entry to , we have  and . 
 At location , the value of  remains unchanged,  grows to 1 and is reset, 
and  becomes . At location , a 
time  is spent, obtaining  
and . A time 
is spent at location , obtaining . 


At location , a time of one unit is spent, and the environment can choose a rate in .
Consider the case when  and . In this case, scheduler 
has already lost the game, since spending one unit at location 
will only give . 
  However, if , 
let . Environment can then choose a rate , for .
Then . This would 
result in scheduler losing. However, if , then 
for any , the value of  is . 
    

\begin{figure}[h]
\begin{center}
\scalebox{0.7}{
\begin{tikzpicture}[->,>=stealth',shorten >=1pt,auto,node distance=1.8cm,
  semithick]
  \tikzstyle{every state}=[minimum size=3em,rounded rectangle]
  
  \node[initial,initial text={},state] at (-2,0) (A) {} ;
        \path (A) edge[loop above]  node[above] {}node[below]{} (A);
   \node[state] at (0.3,0) (B) {} ;
\path (A) edge  node[above] {}node[below]{} (B);
\node[state] at (2.5,0) (C) {} ;
\node[state] at (0,-3) (D) {} ;
\node[state] at (2.5,-3) (E) {} ;
\node[accepting,state] at (-2.5,-3) (T) {} ;

\path (B) edge  node[above] {}node[below]{} (C);
\path (C) edge  node[left] {}node[right]{} (E);
\path (E) edge  node[above] {}node[below]{} (D);
\path (D) edge  node[above] {}node[below]{} (T);
  \end{tikzpicture}}
\caption{The gadget }
\label{inc-c1-less}
\end{center}
\end{figure}

The only remaining case is when the scheduler indeed picks the correct delay of  
at location  in Figure \ref{inc-c1}. In this case, as seen above, 
the rates 100, -100 chosen by the environment does not affect the scheduler. 
In both these cases, scheduler has a winning strategy of choosing to go to 
one of  and reach a value of  in the chosen ball.  
If , and the environment picks the rate 0 at location , then 
the best strategy for the scheduler is to select  , which 
marks the continuation of the simulation of the two counter machine. 
As expected, we will indeed have on entry into ,  and ,
marking the correct simulation of the increment  instruction.
  
\item Simulation of a decrement instruction ;  goto  . 

  The construction of gadgets for the decrement instruction is similar to that of the increment 
  instruction. 
         
\begin{figure}[h]
\begin{center}
\scalebox{0.7}{
\begin{tikzpicture}[->,>=stealth',shorten >=1pt,auto,node distance=1.8cm,
  semithick]
  \tikzstyle{every state}=[minimum size=3em,rounded rectangle]
  
  \node[initial,initial text={},state] (A) {} ;
   \node[state] at (3.6,0) (B) {} ;
\node[state, player2] at (1,3) (chk1) {};
\node[state, player2] at (1,-3) (chk2) {};
\node[state] at (5,2) (cont) {};
      \path (A) edge node[above] {} (B);
      \path (A) edge node[below] {} (B);
        \path (B) edge[loop below]  node[above] {}node[below]{} (B);
                
  \path (B) edge  node[left]{} node[right]{} (chk1);
  \path (B) edge  node[left]{} node[right]{} (chk2);
  \path (B) edge  node[left]{} node[right]{} (cont);
  
  \end{tikzpicture}}
\caption{Simulation of decrement  instruction }
\label{dec-c1}
\end{center}
\end{figure}

The ideal amount of time to be spent by scheduler at  is .  In this case, 
. Assume that the time spent at  is 
, for some . Then we have . 
At location , the environment picks one of the 3 rates 100, -100, 0. 
If he wants to force a check on the environment, he picks the rate 100, 
making , . 
As seen in the case of the increment gadget, the best strategy for the scheduler is to pick the gadget . 

\begin{figure}[h]
\begin{center}
\scalebox{0.7}{
\begin{tikzpicture}[->,>=stealth',shorten >=1pt,auto,node distance=1.8cm,
  semithick]
  \tikzstyle{every state}=[minimum size=3em,rounded rectangle]
  
  \node[initial,initial text={},state] (A) {} ;
        \path (A) edge[loop above]  node[above] {}node[below]{} (A);
   \node[state] at (3.5,0) (B) {} ;
\path (A) edge  node[above] {}node[below]{} (B);
\node[state] at (3.5,-3) (C) {} ;
\node[state] at (1,-3) (D) {} ;
\node[accepting,state] at (-2,-3) (T) {} ;

\path (B) edge  node[above] {}node[below]{} (C);
\path (C) edge  node[above] {}node[below]{} (D);
\path (D) edge  node[above] {}node[below]{} (T);
  \end{tikzpicture}}
\caption{The gadget }
\label{dec-c1-more}
\end{center}
\end{figure}

Entry into  is made with , . One unit 
of time is spent at , obtaining , . 
At  a time  is spent, obtaining .
A time of one unit is spent at  obtaining . 
Likewise, a time of one unit is spent at , obtaining , where 
. If , then clearly, scheduler has already lost the game.
If , then  can be chosen such that  
such that the value of . Note that if , this is not possible, and 
scheduler can indeed reach  .  

        
Now consider the case when scheduler spends a time  , for some  at  in Figure \ref{dec-c1}. 
 Then we have . Again, the environment can choose the rate -100 at location , 
 and the scheduler's best strategy is to enter gadget . Entry into  happens with 
          . 
     One unit of time is spent at  obtaining . 
     A time   is spent at location  obtaining 
     . 
     One unit of time is spent at  obtaining . 
    Spending one unit at location   with a rate  
  gives . 
 If , then the scheduler has already lost. If ,
 then the environment can always choose   such that 
 . Clearly, if , this 
 is not possible. and scheduler wins. 
    
          
                      
\begin{figure}[h]
\begin{center}
\scalebox{0.7}{
\begin{tikzpicture}[->,>=stealth',shorten >=1pt,auto,node distance=1.8cm,
  semithick]
  \tikzstyle{every state}=[minimum size=3em,rounded rectangle]
  
  \node[initial,initial text={},state] at (-2,0) (A) {} ;
        \path (A) edge[loop above]  node[above] {}node[below]{} (A);
   \node[state] at (0.3,0) (B) {} ;
\path (A) edge  node[above] {}node[below]{} (B);
\node[state] at (2.5,0) (C) {} ;
\node[state] at (2.5,-3) (E) {} ;
\node[accepting,state] at (-0.5,-3) (T) {} ;

\path (B) edge  node[above] {}node[below]{} (C);
\path (C) edge  node[left] {}node[right]{} (E);
\path (E) edge  node[above] {}node[below]{} (T);
  \end{tikzpicture}}
\caption{The gadget }
\label{dec-c1-less}
\end{center}
\end{figure}


\item Zero Check Instruction.  : if  goto  else goto .

Figure \ref{zero-check} describes the gadget for zero check 
of counter . No time is spent 
at location , and the scheduler makes  a guess about the 
value of . If he guesses that  is zero, then he will choose 
the location . The environment can either allow the scheduler to go ahead 
with the simulation by choosing a rate 0, or could verify the correctness of scheduler's guess
by choosing a rate 100.  One unit of time has to be spent at the location . 
Thus, if the scheduler decides to verify 
and chooses the rate 100, the value of  will be . 
The environment will check if , for some . 
If the environment chooses 100, the best strategy for the scheduler 
is to choose the gadget . Going to  does not help the scheduler to win, since 
the environment can pick the rate 100 in all future choice locations, ensuring that 
the scheduler cannot win. 

\begin{figure}[h]
\begin{center}
\scalebox{0.7}{
\begin{tikzpicture}[->,>=stealth',shorten >=1pt,auto,node distance=1.8cm,
  semithick]
  \tikzstyle{every state}=[minimum size=3em,rounded rectangle]
    \node[initial,initial text={},state] at (-2,0)(A) {} ;
   \node[state] at (0,2) (B) {} ;
\node[state] at (0,-2) (C) {} ;
   \node[state] at (3,3) (B1) {} ;
   \node[state,diamond] at (3,1) (B2) {} ;

\node[state] at (3,-1) (C1) {} ;
   \node[state,diamond] at (3,-3) (C2) {} ;


\path (A) edge node[left] {} (B);
      \path (A) edge node[left] {} (C);
      
      \path (B) edge node[above] {} node[below]{}(B1);
      \path (B) edge node[above] {} node[below]{}(B2);
      
      
      
      \path (C) edge node[above] {} node[below]{}(C1);
      \path (C) edge node[above] {} node[below]{}(C2);
      
      




  \end{tikzpicture}}
\caption{Zero Check }
\label{zero-check}
\end{center}
\end{figure}


 The gadget  given in figure \ref{z} is a check gadget which checks if , for some .  The gadget  is entered with , . A time of unit is spent at location , obtaining 
 and . If indeed , and if in addition, , then  and . In this case, 
 the scheduler can go to the location , spend a unit of time at  obtaining . 
 This leads to the location , where the environment can pick any rate in . One unit of time 
 is spent in , and in this case, we reach the mode  with , for 
 . Clearly, the scheduler wins here since his guess about  being zero was correct. 


In case  for , then from location , 
the scheduler cannot win by choosing location  as the next location, 
since the value of  on entry into  will be 
, where  for . 
If , then the scheduler has already lost. If 
, let , for some .
Then , since .
Thus, the environment can pick a rate  
such that . 
  

\begin{figure}[h]
\begin{center}
\scalebox{0.7}{
\begin{tikzpicture}[->,>=stealth',shorten >=1pt,auto,node distance=1.8cm,
  semithick]
  \tikzstyle{every state}=[minimum size=3em,rounded rectangle]
    \node[initial,initial text={},state] at (-2,0)(A) {} ;
                 \node[state] at (0.8,0) (B) {} ;
\node[state] at (3,0) (C) {} ;
\node[state] at (3,2) (F) {} ;
\node[state] at (6,2) (G) {} ;
\node[state,accepting] at (6,0) (T) {} ;


      \node[state] at (3,-4) (D) {} ;
     \node[state,diamond] at (-2,-2) (E1) {} ;
      \node[state,diamond] at (-2,-6) (E2) {} ;
      


\path (A) edge node[above] {} node[below]{} (B);
\path (B) edge node[above] {} (C);
            \path (B) edge node[left] {} (F);
            \path (C) edge node[right] {} (F);
      \path (F) edge node[above] {} node[below]{}(G);
      \path (G) edge node[left] {} node[right]{}(T);
            
            
      \path (C) edge[bend left=80] node[left] {} node[right]{}(D);
      \path (D) edge node[left] {} node[right]{}(E1);
      \path (D) edge node[left] {} node[right]{}(E2);
      \path (D) edge [loop below] node[above]{} node[below]{}(D);
      \path (D) edge[bend left=50]  node[left]{} node[right]{}(C);
  \end{tikzpicture}}
\caption{The gadget }
\label{z}
\end{center}
\end{figure}

Thus, if , the best strategy for scheduler is to goto
location . The subgraph consisting of locations  and gadgets  and 
 simulates the decrement  instruction. The ideal time to be spent 
at  is  so that the value of  is decremented by one. At location ,
the environment can choose a rate 0 (in which case, scheduler will go back to location )
or a rate 100 (in which case scheduler will go to ) or a rate -100 (in which case, 
scheduler will go to ). In the case scheduler goes back to , 
the new value of  is . The ideal time to be spent at  now is 
, and so on. At some point of time when , we will obtain . 
At this point, the scheduler can take the transition to  from , and as
seen above can reach  with . If the scheduler 
goes to  from  when  for some , then 
as seen above, on entry into , , and the environment has a 
choice of rate in  such that scheduler loses. 


\begin{figure}[h]
\begin{center}
\scalebox{0.7}{
\begin{tikzpicture}[->,>=stealth',shorten >=1pt,auto,node distance=1.8cm,
  semithick]
  \tikzstyle{every state}=[minimum size=3em,rounded rectangle]
    \node[initial,initial text={},state] at (-4,2)(A) {} ;
                 \node[state] at (-4,0) (B) {} ;
\node[state] at (-4,-2) (C) {} ;
\node[state] at (-1.5,-8) (F) {} ;
\node[state] at (1,-8) (G) {} ;
\node[state,accepting] at (1,-10) (T) {} ;
      \node[state] at (-4,-6) (D) {} ;
      \node[state] at (-4,-8) (J) {} ;
      \node[state] at (-4,-12) (K) {}; 
\node[state,diamond] at (-8,-10) (D1) {} ;
      \node[state,diamond] at (-8,-14) (D2) {} ;
     
     
\node[state,diamond] at (-8,-4) (E1) {} ;
      \node[state,diamond] at (-8,-8) (E2) {} ;
      \path (A) edge node[left] {} node[right]{} (B);
            \path (B) edge node[left] {} (C);
\path (J) edge node[above] {} (F);
      \path (F) edge node[above] {} node[below]{}(G);
      \path (G) edge node[left] {} node[right]{}(T);
            
            
      \path (C) edge[bend left=70] node[left] {} node[right]{}(D);
      \path (D) edge node[left] {} node[right]{}(E1);
      \path (D) edge node[left] {} node[right]{}(E2);
      \path (D) edge [loop right] node[left]{} node[right]{}(D);
      \path (K) edge [loop right] node[left]{} node[right]{}(K);
       \path (D) edge[bend left=50]  node[left]{} node[right]{}(C);
      \path (D) edge  node[left]{} node[right]{}(J);
      \path (J) edge[bend left=70] node[left] {} node[right]{}(K);
      \path (K) edge[bend left=50]  node[left]{} node[right]{}(J);
 \path (K) edge node[left] {} node[right]{}(D1);
      \path (K) edge node[left] {} node[right]{}(D2);
    
   \end{tikzpicture}}
\caption{The gadget }
\label{nz}
\end{center}
\end{figure}

The gadget  is given in Figure \ref{nz}. This is entered into when the environment 
chooses a rate of  100 at location  in Figure \ref{zero-check}. The idea is to verify that 
indeed  is non-zero. Scheduler has to go through the locations 
 atleast once so that  is decremented atleast once (hence, ). 
The time elapse at  must be , so that , decrementing .
The gadgets  and  can be designed similar to 
the gadgets  and  to catch the errors of the scheduler when the time elapse 
is  and , .  The scheduler must visit the  loop
 times (provided the environment gives rate 0 at location  everytime). 
When the rate 0 is given at location , 
scheduler can move to location  when  becomes 0. 
If  is zero, then we get  at the end of the  loop. Then from , scheduler can go to location  
spending no time at , and reach  with . However, if , 
then scheduler visits the  loop until  (provided the environment gives a rate 0 at location ).
When , the scheduler can move from  to , and reach the target with . 
 \item The Halt location : The location labeled  has rate 1. The scheduler will reach here 
 iff the two counter machine halts, and when the scheduler has simulated all the instructions correctly. The value of  will be , where , for . A non-deterministic amount of time 
 can be spent by the scheduler here so that  will lie in . 
\end{enumerate}    
It can be proved that the scheduler has a winning strategy to reach   iff 
the two counter machine halts. 


























      




 
















  




 




 


            








































































        


          
                      




















      
      


      






















            







     


            






















































































































    
        



  




  











