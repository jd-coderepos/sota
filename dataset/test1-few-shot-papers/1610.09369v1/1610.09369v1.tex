\documentclass{article}

\PassOptionsToPackage{numbers, compress}{natbib}


\usepackage[final]{nips_2016}



\usepackage[utf8]{inputenc} \usepackage[T1]{fontenc}    \usepackage{hyperref}       \usepackage{url}            \usepackage{booktabs}       \usepackage{amsfonts}       \usepackage{nicefrac}       \usepackage{microtype}      

\usepackage{wrapfig}

\usepackage{graphicx}

\usepackage{amsmath}
\usepackage{amsfonts}
\usepackage{amsthm}

\usepackage{caption}
\usepackage{subfig}

\usepackage{algorithm}
\usepackage[noend]{algorithmic}

\newtheorem{theorem}{Theorem}
\newtheorem{problem}{Problem}
\newtheorem{definition}{Definition}
\newtheorem{corollary}{Corollary}

\usepackage{multirow}

\title{Discriminative Gaifman Models}



\author{
  Mathias Niepert\\
  NEC Labs Europe\\
  Heidelberg, Germany \\
  \texttt{mathias.niepert@neclabs.eu} \\
}

\begin{document}


\maketitle

\begin{abstract}
We present discriminative Gaifman models, a novel family of relational machine learning models. Gaifman models learn feature representations bottom up from representations of locally connected and bounded-size regions of knowledge bases (KBs).  Considering local and bounded-size neighborhoods of knowledge bases renders logical inference and learning tractable, mitigates the problem of overfitting, and facilitates weight sharing. Gaifman models sample neighborhoods of knowledge bases so as to make the learned relational models more robust to missing objects and relations which is a common situation in open-world KBs. We present the core ideas of Gaifman models and apply them to large-scale relational learning problems. We also discuss the ways in which Gaifman models relate to some existing relational machine learning approaches.
\end{abstract}

\section{Introduction}

Knowledge bases are attracting considerable interest both from industry and academia~\cite{Bollacker:2008,carlson:2010,Lao:2011,Gardner:2015}. Instances of knowledge bases are the web graph, social and citation networks, and multi-relational knowledge graphs such as Freebase~\cite{Bollacker:2008} and YAGO~\cite{Hoffart:2013}. Large knowledge bases motivate the development of scalable  machine learning models that can reason about objects as well as their properties and relationships. Research in statistical relational learning (SRL) has focused on particular formalisms such as Markov logic~\cite{richardson:2006} and \textsc{ProbLog}~\cite{dries:2015} and is often concerned with improving the  efficiency of inference and learning~\cite{Kersting:2012,VdB:2013}. The  scalability problems of these statistical relational languages, however, remain an obstacle and have prevented a wider adoption. Another line of work focuses on efficient relational machine learning models that perform well on a particular task such as knowledge base completion and relation extraction. Examples are knowledge base factorization and embedding approaches~\cite{Bordes:2011,nickel:2011,riedel13relation,Socher:2013} and random-walk based ML models~\cite{Lao:2011,Gardner:2015}. 
We aim to advance the state of the art in relational machine learning by developing efficient models that learn knowledge base  embeddings that are effective for probabilistic query answering on the one hand, and interpretable and widely applicable on the other.

Gaifman's locality theorem~\cite{gaifman:1982} is a result in the area of finite model theory~\cite{Libkin:2004}. The Gaifman graph of a knowledge base is the undirected graph whose nodes correspond to objects and in which two nodes are connected if the corresponding objects co-occur as arguments of some relation. Gaifman's locality theorem states that every first-order sentence is equivalent to a Boolean combination of sentences whose quantifiers range over local  neighborhoods of the Gaifman graph.
With this paper, we aim to explore Gaifman locality from a machine learning perspective. If every first-order sentence is equivalent to a Boolean combination of sentences whose quantifiers range over local neighborhoods only, we ought to be able to develop models that learn effective representations from these local neighborhoods. There is increasing evidence that learning representations that are built up from local structures can be highly successful. Convolutional neural networks, for instance, learn features over locally connected regions of images. The aim of this work is to investigate the effectiveness and efficiency of machine learning models that perform learning and inference within and across locally connected regions of knowledge bases. This is achieved by combining relational features that are often used in statistical relatinal learning with novel ideas from the area of deep learning. The following problem motivates Gaifman models.

\begin{problem} Given a knowledge base (relational structure, mega-example, knowledge graph) or a collection of knowledge bases, learn a relational machine learning model that supports complex relational queries. The model learns a probability for each tuple in the query answer.
\end{problem}

Note that this is a more general problem than knowledge base completion since it includes the learning of a probability distribution for a complex relational query. The query corresponding to knowledge base completion is  for logical variables  and , and relation . The problem also touches on the problem of open-world probabilistic KBs~\cite{CeylanKR16} since tuples whose prior probability is zero will often have a non-zero probability in the query answer.  





\section{Background}

We first review some important concepts and notation in first-order logic.


\subsection{Relational First-order Logic}

An atom  consists of predicate  of arity  followed by  arguments, which are either elements from a finite domain  or logical variables . We us the terms domain element and object synonymously. A ground atom is an atom without logical variables. Formulas are built from atoms using the usual Boolean connectives and existential and universal quantification. A free variable in a first-order formula is a variable  not in the scope of a quantifier. We write  to denote that  are free in , and  to refer to the free variables of . A substitution replaces all occurrences of logical variable  by  in some formula  and is denoted by . 

A vocabulary consists of a finite set of predicates  and a domain . Every predicate  is associated with a positive integer called the arity of . 
A -structure (or knowledge base)  consists of the domain , a set of predicates , and an interpretation. The Herbrand base of  is the set of all ground atoms that can be constructed from  and . The interpretation assigns a truth value to every atom in the Herbrand base by specifying   for each -ary predicate . For a formula  and a structure , we write  to say that  satisfies  if the variables  are substituted with the domain elements .  We define  .
For the -structure  and  ,  denotes the substructure induced by  on , that is, the -structure  with domain  and  for every -ary .

\subsection{Gaifman's Locality Theorem}



The Gaifman graph of a -structure  is the graph  with vertex set  and an edge between two vertices  if and only if there exists an  and a tuple  such that . Figure~\ref{fig-KB} depicts a fragment of a knowledge base and the corresponding Gaifman graph. 
The distance  between two elements  of a structure  is the length of the shortest path in  connecting  and . For  and , we define the -neighborhood of  to be . We refer to  also as the \emph{depth} of the neighborhood. 
Let . The -neighborhood of  is defined as 


For the Gaifman graph in Figure~\ref{fig-KB}, we have that  and .
 is the formula obtained from  by \emph{relativizing} all quantifiers to , that is, by replacing every subformula of the form 
 by  and every subformula of the form  by .
A formula  of the form , for some , is called -local. Whether an -local formula  holds depends only on the -neighborhood of , that is, for every structure  and every  we have  if and only if .
For  and  being -local, a local sentence is of the form


We can now state Gaifman's locality theorem.
\begin{theorem}\cite{gaifman:1982}
Every first-order sentence is equivalent to a Boolean combination of local sentences.
\end{theorem}



Gaifman's locality theorem states that any first-order sentence can be expressed as a Boolean combination of -local sentences defined for neighborhoods of objects that are mutually far apart (have distance at least ). Now, a novel approach to (statistical) relational learning would be to consider a large set of objects (or tuples of objects) and learn models from their local neighborhoods in the Gaifman graphs. It is this observation that motivates Gaifman models.







\begin{figure}[t!]
\begin{minipage}[b]{.48\textwidth}
\centering
\subfloat{{\label{fig-KB}\raisebox{0.7cm}{\includegraphics[width = 0.96\textwidth]{KB_new.pdf}} }}\caption{A knowledge base fragment for the pair  and the corresponding Gaifman graph.}\label{fig-KB}
\end{minipage}
\hspace{3mm}
\begin{minipage}[b]{.49\textwidth}
\centering
\subfloat{{\label{fig-node-distr} \includegraphics[width = 0.98\textwidth]{FB15-deg-dist.pdf} }}
\caption{The degree distribution of the Gaifman graph for the Freebase fragment \textsc{Fb15k}.}
\label{fig-node-distr}
\end{minipage}
\end{figure}






\section{Learning Gaifman Models}

Instead of taking the costly approach of applying relational learning and inference directly to entire knowledge bases, the representations of Gaifman models are learned bottom up, by performing inference and learning within bounded-size, locally connected regions of Gaifman graphs.
Each Gaifman model specifies the data generating process from a given knowledge base (or collection of knowledge bases), a set of relational features, and a ML model class used for learning.

\begin{definition}
Given a -structure , a discriminative Gaifman model for  is a tuple  as follows:
\begin{itemize}
\item  is a first-order formula called the \emph{target query} with at least one free variable; 

\item  is the \emph{depth} of the Gaifman neighborhoods;

\item  is the \emph{size-bound} of the Gaifman neighborhoods;



\item  is a set of first-order formulas (the relational features); 

\item  is the base model class (loss, hyper-parameters, etc.).
\end{itemize}
\end{definition}




Throughout the rest of the paper, we will provide detailed explanations of the different parameters of Gaifman models and their interaction with data generation,  learning, and inference. 

During the training of Gaifman models, neighborhoods are generated for \emph{tuples of objects}  based on the parameters  and . We first describe the procedure for arbitrary tuples  of objects and will later explain where these tuples come from. For a given tuple  the -neighborhood of  within the Gaifman graph is computed. This results in the set of objects . Now, from this neighborhood we sample  neighborhoods consisting of at most  objects. Sampling bounded-size sub-neighborhoods from  is motivated as follows:

\begin{enumerate}
\item The degree distribution of Gaifman graphs is often skewed (see Figure~\ref{fig-node-distr}), that is, the number of other objects a domain element is related to varies heavily. Generating smaller, bounded-size neighborhoods allows the transfer of learned representations between more and less connected objects. Moreover, the sampling strategy makes Gaifman models more robust to object uncertainty~\cite{Milch:2006}. We show empirically that larger values for  reduce the effectiveness of the learned models for some knowledge bases. 
\item Relational learning and inference is performed within the generated neighborhoods.  can be very large, even for  (see Figure~\ref{fig-node-distr}), and we want full control over the complexity of the computational problems. 
\item Even for a single object tuple  we can generate a large number of training examples if . This mitigates the risk of overfitting. The number of training examples per tuple strongly influences the models' accuracy. 
\end{enumerate}

We can now define the set of -neighborhoods generated from a -neighborhood. 







For a given tuple of objects , Algorithm~\ref{alg:generate} returns a set of  neighborhoods drawn from  such that the number of objects for each  is the same in expectation. 


The formulas in the set  are indexed and of the form  with  and . 
For every tuple , generated neighborhood , and , we perform the substitution  and relativize 's quantifiers to , resulting in  which we  write as .
Let  be the substructure induced by  on . For every formula  and every , we now have that  if and only if . In other words, satisfaction is now checked locally within the neighborhoods , by deciding whether .
The relational semantics of Gaifman models is based on the set of formulas . The feature vector  for tuple , and neighborhood , written as ,  is constructed as follows


That is, if   has free variables,  is equal to the \emph{number of groundings} of  that are satisfied within the neighborhood substructure ; if  has no free variables,  if and only if  is satisfied within the neighborhod substructure ; and  otherwise. The neighborhood representations  capture -local formulas and help the model learn formula combinations that are associated with negative and positive examples. For the right choices of the parameters  and , the neighborhood representations of Gaifman models capture the relational structure associated with positive and negative examples. 

Deciding  for a structure  and a first-order formula  is referred to as \emph{model checking} and computing  is called \emph{-counting}.  The combined complexity of model checking is PSPACE-complete ~\cite{vardi:1982} and there exists a  algorithm for \emph{both} problems where  is the size of an encoding. Clearly, for most real-world KBs this is not feasible. For Gaifman models, however, where the neighborhoods are bounded-size, typically , the above representation can be computed very efficiently for a large class of relational features.  We can now state the following complexity result. 

\begin{theorem}
Let  be a relational structure (knowledge base), let  be the size of the largest -neighborhood of 's Gaifman graph, and let  be the greatest encoding size of any formula in . 
For a Gaifman model with parameters  and , the worst-case complexity for computing the feature representations of  neighborhoods is .
\end{theorem}

Existing SRL approaches could be applied to the generated neighborhoods, treating each as a possible world for structure and parameter learning. However, our goal is to learn relational models that utilize embeddings computed by multi-layered neural networks. 

\small
\begin{figure}[t!]
\begin{minipage}[b]{.50\textwidth}
\begin{algorithm}[H]
\caption{\label{alg:generate}\textsc{GenNeighs}: Computes a list of  neighborhoods of size  for an input tuple .}
\begin{algorithmic}[1]
	\STATE {\bfseries input:} tuple , parameters , , and 
   \STATE 
   \WHILE{}
   \STATE 
   \STATE 
   \FORALL  {}
   \STATE  elements sampled uniformly from 
   \STATE 
   \STATE 
   \ENDFOR
   \STATE  elements sampled uniformly from 
   \STATE 
   \STATE 
   \ENDWHILE
   \STATE {\bfseries return} 
\end{algorithmic}
\end{algorithm}
\end{minipage}\hspace{5mm}
\begin{minipage}[b]{.46\textwidth}
  \centering
\includegraphics[width = 1.0\textwidth]{generate-ill.pdf}\caption{\label{fig-learning-example}Learning of a  Gaifman model.}
\vspace{5mm}
\includegraphics[width = 1.0\textwidth]{generate-ill-inference.pdf}\caption{\label{fig-inference-example}Inference with a Gaifman model.}\end{minipage}
\end{figure}
\normalsize


\subsection{Learning Distributions for Relational Queries}

Let  be a first-order formula (the relational query) and  the result set of the  query, that is, all groundings that render the formula satisfied in the knowledge base. The feature representations generated for tuples of objects  serve as \emph{positive} training examples. The Gaifman models' aim is to learn neighborhood embeddings that capture local structure of tuples for which we know that the target query evaluates to true.  Similar to previous work, we generate \emph{negative} examples by corrupting tuples that correspond to positive examples. The corruption mechanism takes a positive input tuple  and substitutes, for each , the domain element  with objects sampled from  while keeping the rest of the tuple fixed. 

The discriminative Gaifman model performs the following steps. 
\begin{enumerate}
\item Evaluate the target query  and compute the result set 
\item For each tuple  in the result set :
\begin{itemize} 
\item Compute , a multiset of  neighborhoods  with Algorithm~\ref{alg:generate}; each such neighborhood  serves as a \emph{positive} training example
\item Compute , a multiset of  neighborhoods  for corrupted versions of  with Algorithm~\ref{alg:generate}; each such neighborhood serves as a \emph{negative} training example
\item Perform model checking and counting within the neighborhoods to compute the feature representations  and  for each  and , respectively
\end{itemize}
\item Learn a ML model with the generated positive and negative training examples.
\end{enumerate}
Learning the final Gaifman model depends on the base ML model class  and its loss function. We obtained state of the art results with neural networks, gradient-based learning, and  categorical cross-entropy as loss function


where  is the probability the model returns on input . However, other loss functions are possible. The probability of a particular substitution of the target query to be true is now


The expected probability of a representation of a neighborhood drawn uniformly at random from . It is now possible to generate several neighborhoods  and their representations  to estimate , simply by averaging the neighborhoods' probabilities. We have found experimentally that a single neighborhood already leads to highly accurate results but also that more neighborhood samples further improve the accurracy. 

Let us emphasize again the novel semantics of Gaifman models. Gaifman models generate a large number of small, bounded-size structures from a large structure, learn a representation for these bounded-size structures, and use the resulting representation to answer queries concerning the original structure as a whole. The advantages are model weight sharing across a large number of neighborhoods and efficiency of the computational problems. Figure~\ref{fig-learning-example} and Figure~\ref{fig-inference-example} illustrate learning from bounded-size neighborhood structures and inference in Gaifman models. 







\subsection{Structure Learning}


Structure learning is the problem of determining the set of relational features . We provide some directions and leave the problem to future work. Given a collection of  bounded-size neighborhoods of the Gaifman graph, the goal is to determine suitable relational features for the problem at hand. There is a set of features which we found to be highly effective. For example, formulas of the form , , and  for all relations. The latter formulas capture fixed-length paths between  and  in the neighborhoods. Hence, Path Ranking type features~\cite{Lao:2011} can be used in Gaifman models as a particular relational feature class. For path formulas with several different relations we cannot include all  combinations and, hence, we have to determine a subset   occurring in the training data. Fortunately, since the neighborhood size is bounded, it is computationally feasible to compute \emph{frequent paths} in the neighborhoods and to use these as features. The complexity of this learning problem is in the number of elements in the neighborhood and not in the number of all objects in the knowledge base. Relation paths that do not occur in the data can be discarded. 
Gaifman models can also use features of the form , , and , to name but a few. Moreover, features with free variables, such as  are \emph{counting features} (here: the  out-degree of ). It is even computationally feasible to include specific second-order features (for instance, quantifiers ranging over ) and aggregations of feature values.

\subsection{Prior Confidence Values, Types, and Numerical Attributes}

Numerous existing knowledge bases assign confidence values (probabilities, weights, etc.) to their statements. Gaifman models can incorporate confidence values during the sampling  and learning process. Instead of adding random noise to the representations, which we have found to be beneficial,  noise can be added inversely proportional to the confidence values. Statements for which the prior confidence values are lower are more likely to be dropped out during training than statements with higher confidence values. 
Furthermore, Gaifman models can directly incorporate object types such as \textsl{Actor} and \textsl{Action Movie} as well as numerical features such as \textsl{location} and \textsl{elevation}. One simply has to specify a fixed position in the neighborhood representation  for each object position within the input tuples . 


\section{Related Work}

Recent work on relational machine learning for knowledge graphs is surveyed in \cite{Nickel:2016}. We focus on a select few methods we deem most related to Gaifman models and refer the interested reader to the above article. 
A large body of work exists on learning inference rules from knowledge bases. 
Examples include \cite{yates:2007} and \cite{berant:2011} where inference rules of length one are learned; and \cite{Schoenmackers:2010} where general inference rules are learned by applying a support threshold. Their method
does not scale to large KBs and depends on predetermined thresholds. 
Lao et al.~\cite{Lao:2011} train a logistic regression classifier with path features to perform KB completion.  The idea is to perform a random walk between objects and to exploit the discovered paths as features. SFE~\cite{Gardner:2015} improves PRA by making the generation of random walks more efficient. More recent embedding methods have combined paths in KBs with KB embedding methods~\cite{DBLP:conf/emnlp/LinLLSRL15}.  Gaifman models support a much broader class of relational features subsuming path features. For instance, Gaifman models incorporate counting features that have shown to be beneficial for relational models.  

Latent feature models learn features for objects and relations that are not directly observed in the data. Examples of latent feature models are tensor factorization~\cite{nickel:2011,riedel13relation,Socher:2013} and embedding models~\cite{Bordes:2011,bordes:2012,Bordes:2013,lin:2015,DBLP:conf/aaai/JiLH016,DBLP:conf/icml/TrouillonWRGB16}. The majority of these models can be understood as more or less complex neural networks operating on object and relation representations. Gaifman models can also be used to learn knowledge base embeddings. Indeed, one can show that it generalizes or complements existing approaches. For instance, the universal schema~\cite{riedel13relation} considers pairs of objects where relation membership variables comprise the model's features. We have the following interesting relationship between universal schemas~\cite{riedel13relation} and Gaifman models. Given a knowledge base . The Gaifman model for  with , , ,  and  is equivalent to the Universal Schema~\cite{riedel13relation} for  up to the base model class .  
More recent methods combine embedding methods and inference-based logical approaches for relation extraction~\cite{logicmf:naacl15}. Contrary to most existing multi-relational ML models~\cite{Nickel:2016}, Gaifman models natively support higher-arity relations, functional and type constraints, numerical features, and complex target queries. 




\begin{wraptable}[5]{R}{6.6cm}
\vspace{-5mm}
\small
\caption{\label{statistic-table} The statistics of the data sets.}
\begin{tabular}{|c|c|c|c|c|c|}
\hline 
Dataset &  &  & \# train & \# test \\ 
\hline 
WN18 & 40,943 & 18 & 141,442 &  5,000 \\ 
FB15k & 14,951 & 1,345 & 483,142 &  59,071 \\ 
\hline 
\end{tabular}
\end{wraptable}
\normalsize





\section{Experiments}

The aim of the experiments is to understand the efficiency and effectiveness of Gaifman models for typical knowledge base inference problems. 
We evaluate the proposed class of models with two data sets derived from the knowledge bases \textsc{WordNet} and \textsc{Freebase}~\cite{Bollacker:2008}. Both data sets consist of a list of statements  that are known to be true.  For a detailed description of the data sets, whose statistics are listed in Table~\ref{statistic-table},  we refer the reader to previous work~\cite{Bordes:2013}.  

After training the models, we perform entity prediction as follows. For each statement  in the test set,  is replaced by each of the KB's objects in turn. The probabilities of the resulting statements are predicted and sorted in descending order. Finally, the rank of the correct statement within this ordered list is determined. The same process is repeated now with replacements of . 
We compare Gaifman models with  to state of the art knowledge base completion approaches which are listed in Table~\ref{exp-table}.  We trained Gaifman models with  and different values for , , and . We use a neural network architecture with two hidden layers, each having  units and sigmoid activations, dropout of  on the input layer, and a softmax layer. Dropout makes the model more robust to missing relations between objects. We trained one model per relation and left the hyper-parameters fixed across models. We did not perform structure learning and instead used the following set of relational features


To compute the probabilities, we averaged the probabilities of  or  generated -neighborhoods. 

\begin{wrapfigure}[15]{r}{6.6cm}
\vspace{-6mm}
\includegraphics[width = \linewidth]{running-times.pdf}\caption{\label{fig-runtimes} Query answers per second rates for different values of the parameter .}
\end{wrapfigure}


We performed runtime experiments to evaluate the models' efficiency. Embedding models have the advantage that one dot product for every candidate object is sufficient to compute the score for the corresponding statement and we need to assess the performance of Gaifman models in this context. All experiments were run on commodity hardware with 64G RAM and a single 2.8 GHz CPU.

\begin{table}
\caption{\label{exp-table} Results of the entity prediction experiments.}
\centering
\small
\begin{tabular}{|c|ccc|ccc|}
\hline 
Data Set & \multicolumn{3}{c|}{WN18} & \multicolumn{3}{c|}{FB15K} \\ 
\hline 
Metric & Mean rank & Hits@10 & Hits@1 & Mean rank & Hits@10 & Hits@1\\ 
\hline
RESCAL\cite{nickel:2011} & 1,163 & 52.8 & - & 683 & 44.1 & - \\ 
SE\cite{Bordes:2011}  & 985 & 80.5 & - & 162 & 39.8 & - \\ 
LFM\cite{jenatton:2012}  & 456 &  81.6 & - & 164 & 33.1 & - \\ 
TransE\cite{Bordes:2013}  & 251  & 89.2 &  8.9 & 51  &  71.5 & 28.1 \\ 
TransR\cite{lin:2015}  & 219  & 91.7 & - & 78 &  65.5 & - \\ 
DistMult\cite{Yang:2015} & 902 &  93.7 & 76.1 & 97 & 82.8 & 44.3 \\
\hline
\hline
Gaifman [1, , 1, \ 5]  & 298  & 93.9 & 75.8 & 124 &  78.1 & 59.8 \\
Gaifman [1, 20, 1, \ 2]  & 357 & 88.1 & 66.8 & 114 & 79.2 & 60.1 \\
\hline
Gaifman [1, 20, 5, 25]  & 392 & 93.6 & 76.4 & 97 & 82.1 & 65.6 \\
Gaifman [2, 20, 5, 25]  & 378 &  93.9 & 76.7 & 84 & 83.4 & 68.5 \\
Gaifman [3, 20, 5, 25]  & 352 &  93.9 &  76.1 & 75 & 84.2 & 69.2 \\
\hline
\end{tabular}
\end{table}
\normalsize

Table~\ref{exp-table} lists the experimental results for different parameter settings . The Gaifman models achieve the highest hits@10 and hits@1 values for both data sets. As expected, the more neighborhood samples are used to compute the probability estimate  the better the result. When the entire -neighborhood is considered (), the performance for WN18 does not deteriorate as it does for FB15k. This is due to the fact that objects in WN18 have on average few neighbors. FB15k has more variance in the Gaifman graph's degree distribution (see Figure~\ref{fig-node-distr}) which is reflected in the better performance for smaller  values. The experiments also show that it is beneficial to generate a large number of representations (both positive and negative ones). The performance improves with larger number of training examples.


The runtime experiments demonstrate that Gaifman models perform inference very efficiently for . Figure~\ref{fig-runtimes} depicts the number of query answers the Gaifman models are able to serve per second, averaged over relation types. A query answer returns the probability for one object pair. These numbers include neighborhood generation and network inference. The results are promising with about  query answers per second (averaged across relation types) as long as  remains  small. Since most object pairs of WN18 have a -neighborhood whose size is smaller than ,  the answers per second rates for  is not reduced as drastically as for FB15k. 

\section{Conclusion and Future Work}

Gaifman models are a novel family of relational machine learning models that perform learning and inference within and across locally connected regions of relational structures. Future directions of research include structure learning, more sophisticated base model classes, and application of Gaifman models to additional relational ML problems. 

\subsection*{Acknowledgements}

Many thanks to Alberto Garc{\'{\i}}a{-}Dur{\'{a}}n, Mohamed Ahmed, and Kristian Kersting for their helpful feedback.

\footnotesize

\begin{thebibliography}{10}

\bibitem{berant:2011}
J.~Berant, I.~Dagan, and J.~Goldberger.
\newblock Global learning of typed entailment rules.
\newblock In {\em Annual Meeting of the Association for Computational
  Linguistics}, pages 610--619, 2011.

\bibitem{Bollacker:2008}
K.~Bollacker, C.~Evans, P.~Paritosh, T.~Sturge, and J.~Taylor.
\newblock Freebase: A collaboratively created graph database for structuring
  human knowledge.
\newblock In {\em SIGMOD}, pages 1247--1250, 2008.

\bibitem{bordes:2012}
A.~Bordes, X.~Glorot, J.~Weston, and Y.~Bengio.
\newblock Joint learning of words and meaning representations for open-text
  semantic parsing.
\newblock In {\em Conference on Artificial Intelligence and Statistics}, pages
  127--135, 2012.

\bibitem{Bordes:2013}
A.~Bordes, N.~Usunier, A.~Garcia-Duran, J.~Weston, and O.~Yakhnenko.
\newblock Translating embeddings for modeling multi-relational data.
\newblock In {\em Neural Information Processing Systems}, pages 2787--2795.
  2013.

\bibitem{Bordes:2011}
A.~Bordes, J.~Weston, R.~Collobert, and Y.~Bengio.
\newblock Learning structured embeddings of knowledge bases.
\newblock In {\em AAAI Conference on Artificial Intelligence}, 2011.

\bibitem{carlson:2010}
A.~Carlson, J.~Betteridge, B.~Kisiel, B.~Settles, E.~R. Hruschka, and T.~M.
  Mitchell.
\newblock Toward an architecture for never-ending language learning.
\newblock In {\em Twenty-Fourth AAAI Conference on Artificial Intelligence},
  2010.

\bibitem{CeylanKR16}
I.~I. Ceylan, A.~Darwiche, and G.~{Van den Broeck}.
\newblock Open-world probabilistic databases.
\newblock In {\em Proceedings of the 15th International Conference on
  Principles of Knowledge Representation and Reasoning (KR)}, 2016.

\bibitem{dries:2015}
A.~Dries, A.~Kimmig, W.~Meert, J.~Renkens, G.~Van~den Broeck, J.~Vlasselaer,
  and L.~De~Raedt.
\newblock Prob{L}og2: {P}robabilistic logic programming.
\newblock {\em Lecture Notes in Computer Science}, 9286:312--315, 2015.

\bibitem{gaifman:1982}
H.~Gaifman.
\newblock On local and non-local properties.
\newblock In {\em Proceedings of the herbrand symposium, logic colloquium},
  volume~81, pages 105--135, 1982.

\bibitem{Gardner:2015}
M.~Gardner and T.~M. Mitchell.
\newblock Efficient and expressive knowledge base completion using subgraph
  feature extraction.
\newblock In {\em Proceedings of the 2015 Conference on Empirical Methods in
  Natural Language Processing}, pages 1488--1498, 2015.

\bibitem{Hoffart:2013}
J.~Hoffart, F.~M. Suchanek, K.~Berberich, and G.~Weikum.
\newblock Yago2: A spatially and temporally enhanced knowledge base from
  wikipedia.
\newblock {\em Artif. Intell.}, 194:28--61, 2013.

\bibitem{jenatton:2012}
R.~Jenatton, N.~L. Roux, A.~Bordes, and G.~R. Obozinski.
\newblock A latent factor model for highly multi-relational data.
\newblock In {\em Neural Information Processing Systems}, pages 3167--3175,
  2012.

\bibitem{DBLP:conf/aaai/JiLH016}
G.~Ji, K.~Liu, S.~He, and J.~Zhao.
\newblock Knowledge graph completion with adaptive sparse transfer matrix.
\newblock In D.~Schuurmans and M.~P. Wellman, editors, {\em Proceedings of the
  Thirtieth {AAAI} Conference on Artificial Intelligence}, pages 985--991,
  2016.

\bibitem{Kersting:2012}
K.~Kersting.
\newblock Lifted probabilistic inference.
\newblock In {\em European Conference on Artificial Intelligence}, pages
  33--38, 2012.

\bibitem{Lao:2011}
N.~Lao, T.~Mitchell, and W.~W. Cohen.
\newblock Random walk inference and learning in a large scale knowledge base.
\newblock In {\em Empirical Methods in Natural Language Processing}, pages
  529--539, 2011.

\bibitem{Libkin:2004}
L.~Libkin.
\newblock {\em Elements Of Finite Model Theory}.
\newblock SpringerVerlag, 2004.

\bibitem{DBLP:conf/emnlp/LinLLSRL15}
Y.~Lin, Z.~Liu, H.~Luan, M.~Sun, S.~Rao, and S.~Liu.
\newblock Modeling relation paths for representation learning of knowledge
  bases.
\newblock In {\em Proceedings of the 2015 Conference on Empirical Methods in
  Natural Language Processing, {EMNLP} 2015, Lisbon, Portugal, September 17-21,
  2015}, pages 705--714, 2015.

\bibitem{lin:2015}
Y.~Lin, Z.~Liu, M.~Sun, Y.~Liu, and X.~Zhu.
\newblock Learning entity and relation embeddings for knowledge graph
  completion.
\newblock In {\em AAAI Conference on Artificial Intelligence}, pages
  2181--2187, 2015.

\bibitem{Milch:2006}
B.~C. Milch.
\newblock {\em Probabilistic Models with Unknown Objects}.
\newblock PhD thesis, 2006.

\bibitem{Nickel:2016}
M.~Nickel, K.~Murphy, V.~Tresp, and E.~Gabrilovich.
\newblock A review of relational machine learning for knowledge graphs.
\newblock {\em Proceedings of the {IEEE}}, 104(1):11--33, 2016.

\bibitem{nickel:2011}
M.~Nickel, V.~Tresp, and H.-P. Kriegel.
\newblock A three-way model for collective learning on multi-relational data.
\newblock In {\em International conference on machine learning (ICML)}, pages
  809--816, 2011.

\bibitem{richardson:2006}
M.~Richardson and P.~Domingos.
\newblock Markov logic networks.
\newblock {\em Machine learning}, 62(1-2):107--136, 2006.

\bibitem{riedel13relation}
S.~Riedel, L.~Yao, B.~M. Marlin, and A.~McCallum.
\newblock Relation extraction with matrix factorization and universal schemas.
\newblock In {\em HLT-NAACL}, 2013.

\bibitem{logicmf:naacl15}
T.~Rockt{\"{a}}schel, S.~Singh, and S.~Riedel.
\newblock Injecting logical background knowledge into embeddings for relation
  extraction.
\newblock In {\em Conference of the North American Chapter of the ACL (NAACL)},
  2015.

\bibitem{Schoenmackers:2010}
S.~Schoenmackers, O.~Etzioni, D.~S. Weld, and J.~Davis.
\newblock Learning first-order horn clauses from web text.
\newblock In {\em Conference on Empirical Methods in Natural Language
  Processing}, pages 1088--1098, 2010.

\bibitem{Socher:2013}
R.~Socher, D.~Chen, C.~D. Manning, and A.~Ng.
\newblock Reasoning with neural tensor networks for knowledge base completion.
\newblock In {\em Neural Information Processing Systems}, pages 926--934. 2013.

\bibitem{DBLP:conf/icml/TrouillonWRGB16}
T.~Trouillon, J.~Welbl, S.~Riedel, {\'{E}}.~Gaussier, and G.~Bouchard.
\newblock Complex embeddings for simple link prediction.
\newblock In {\em Proceedings of the 33nd International Conference on Machine
  Learning}, volume~48, pages 2071--2080, 2016.

\bibitem{VdB:2013}
G.~Van~den Broeck.
\newblock Lifted inference and learning in statistical relational models.
\newblock 2013.

\bibitem{vardi:1982}
M.~Y. Vardi.
\newblock The complexity of relational query languages.
\newblock In {\em ACM symposium on Theory of computing}, pages 137--146, 1982.

\bibitem{Yang:2015}
B.~Yang, W.-t. Yih, X.~He, J.~Gao, and L.~Deng.
\newblock Embedding entities and relations for learning and inference in
  knowledge bases.
\newblock In {\em International Conference on Learning Representations}, 2015.

\bibitem{yates:2007}
A.~Yates and O.~Etzioni.
\newblock Unsupervised resolution of objects and relations on the web.
\newblock In {\em Conference of the North American Chapter of the Association
  for Computational Linguistics}, 2007.

\end{thebibliography}



\end{document}
