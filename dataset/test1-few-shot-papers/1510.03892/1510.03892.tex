As pointed out in the previous section, the \textsc{HERESy} architecture, depicted in   the diagram of Fig.~\ref{fig:harch}, is composed by several modules,
each of which  is in charge of  performing a specific hijacking task on the isolated running containers~\cite{docker}.
The \textit{proxy} module plays the role of  container factory: it builds up on-demand a specialized container on the basis of the protocol over which the potentially malicious remote entity tries to access the system and then it redirects all the subsequent incoming traffic to it. At this point, \textsc{HERESy} begins the hijack: file system modifications are tracked by the \textit{filesystem  module} which exploits versioning control technologies; traffic analysis is performed
by the \textit{network sniffing} module; extraneous binaries,  in case they are downloaded by malwares or attackers, are detected by the \textit{process execution} tracing module. Fig.~\ref{fig:wf} depicts an activity diagram that models the \textsc{HERESy} execution workflow which handles connections attempts.

\begin{figure}[tbp]

\includegraphics[width=1.0\columnwidth]{heresyArch}
\caption{\textsc{HERESy} architecture}
\label{fig:harch}
\end{figure}

\begin{figure}[tb]

	\centering
	\includegraphics[width=0.98\columnwidth]{heresyBehavior}
	\caption{\textsc{HERESy} process}
		\label{fig:wf}
\end{figure}

\subsection{Proxy module}
\label{sub:proxy}
Once a connection attempt reaches the system, if the proxy has been configured to handle the requested service, a fresh new container is built on the fly and all the attacker's traffic is redirected to it.
Each attacker will run and operate inside its own isolated environment. When the attacker leaves, the environment is destroyed after all the relevant data have been suitably stored. To speed up the build operation, and hence to improve the \textsc{HERESy} responsiveness, an environment for each service is always up and maintained ready to be used.



\subsection{Filesystem module}
\label{sub:fs}
One of the main problem with malicious code is that more than often it modifies the content of file system resources. It is possible to track these modifications by writing custom file systems or by hijacking function library calls. However, the former is an hard task to accomplish because of the low level machinery around the host operating system, the latter can be ineffective because, if the malware calls into the kernel bypassing the libraries,  interception does not occur at all. 
The  solution devised by \textsc{HERESy} consists in exploiting versioning control software on the existing virtual file system layer. File system resources are shared between the host \textsc{HERESy} system and the current attacked environment. By
 using file system notification mechanism \cite{fsnotify}, each time a modification,  creation or deletion happens, the system triggers the versioning control software which tracks the updates.
This technique allows  to  explore the file system history going backward and forward in time. This indeed is crucial for tracing back the malware behavior from the viewpoint of the action made on the file system.  
The file system module of the current version of \textsc{HERESy}  uses of GIT as a versioning control software and the Linux \texttt{inotify} API to handle events \cite{fsnotify} (see Fig.~\ref{fig:fsmod}). Previous experiments used FUSE \cite{fuse} as a file system layer, however the integration with docker was hard to achieve. Another project relaying on GIT in a similar way is \texttt{gitfs}, a FUSE file system which automatically converts all the changes made into commits~\cite{gitfs}.

\begin{figure}[bt]

\centering
	  \includegraphics[width=1.0\columnwidth]{fsmod}
\caption{ File system module}
\label{fig:fsmod}
\end{figure}


\subsection{Network module}
\label{sub:net}

Traffic generated by attackers or malicious software is of course fundamental to track where the malicious activities starts and where they try to propagate. For each environment a complete network traffic dump is performed, so a clean data representation of the network flow is given to the threat analyst. The system spawns a daemon for each sandbox environment which is built up. The module relays on the \texttt{libcap} library and saves the attacker generated traffic into the classical \texttt{pcap} format. So using the berkeley packet filtering~\cite{McCanne1993} we are able to select just the traffic generated by the target container.

\subsection{Process execution module}
\label{sub:proc}
For each kind of environment all the binaries are hashed and the signatures are saved into the core system memory. Each time a malicious software or the attacker executes a binary, the command is logged. Also, the module performs the binary hash and if the calculated signature is not present into the white list, the module starts tracking the process using the dynamic binary instrumentation \cite{Hazelwood2011}, so it dumps instructions and memory accesses saving the entire hostile program execution flow.


\begin{figure}[btp]

\includegraphics[width=1.0\columnwidth]{procExec}
\caption{Process execution tracing module}
\label{fig:procexec}
\end{figure}
Figure~\ref{fig:procexec}, shows the process execution tracing step. We use for the hashing process SHA512 which is considered strong enough to avoid collisions, so it becomes difficult for an attacker to generate a binary which has the same hash to trick the system. 

In practice, we used both CRIU~\cite{criu2014} and DynamoRIO~\cite{Bruening2003a} in order to trace the process execution and to dump its resources. The following strategy has been employed to alternate our \texttt{dynamoRIO} based tool with \texttt{CRIU}:  the process is first run inside the \texttt{dynamoRIO} based tool, which detects whether the process is hostile or not on the basis of the signatures of the executable images used by the process; in the case of a not trusted process, the tool begins to dump all the executed instructions and if detects a memory write it asks \texttt{CRIU} to dump all the process resources and waits for its termination; \texttt{CRIU}
receives dump requests by means of a daemon which listen on a socket for messages coming from the \texttt{dynamoRIO} based tool; once \texttt{CRIU} completes its task, it notifies the \texttt{dynamoRIO} tool which then restarts its process instrumentation activities. 
Each time the hostile process accesses the memory, the tool switching, above described, occurs again. 

Currently, we are trying to improve this technique in order to disarm \textit{debug-aware} malwares which once detect they are into a honeypot try to evade the system, for example by killing themselves.


 






