In this section we evaluate multiscale gossip in simulation and study its behaviour in practical scenarios. First we investigate the effect of using few versus many levels. Then we show that multiscale gossip performs very well against path averaging \cite{benezit07}, the current state-of-the-art gossip algorithm that requires linear number messages in the size of the network to converge to the average with  accuracy. Finally, we investigate scenarios where transmissions do not always succeed and messages are either retransmitted or lost.

\subsection{Varying levels of Hierarchy}

In the analysis we concluded that we can select the number of levels  i.e. we don't need too many levels. This can be verified in practice. Figure \ref{fig:manylevels} investigates the effect of increasing the levels of hierarchy. The figure shows the number of messages until convergence within  error, averaged over ten graphs of   nodes. More levels yield a diminishing reward and we do not need more than  or  levels. As discussed in the next subsection this observation led us to try a scheme with only two levels of hierarchy which still produces an efficient algorithm. 

\begin{figure} 
\begin{center}
\includegraphics[width=8cm]{./figures/EffectOfHierarchy.pdf}
\caption{\label{fig:manylevels} Increasing the levels of hierarchy yields a diminishing reward. Results are averages over  random geometric graphs with  nodes and final accuracy . All graphs are created with radius .} 
\end{center}
\end{figure}



\subsection{Mutliscale Gossip vs Path Averaging}

We compare multiscale gossip against path averaging \cite{benezit07} which is in theory the fastest algorithm for gossiping on random geometric  graphs. Is it worth emphasizing that both algorithms operate under the same two assumptions. First, each nodes needs to know the coordinates of itself and its neighbours on the unit square. Second, each node needs to know the size of the network . In path averaging this is implicit since each message needs to be routed back to the source through the same path. It is thus necessary that nodes have global unique ids which is equivalent to knowing the maximum id and thus the size of the network. In multi scale gossip, the network size is used for each node to determine its role in the logical hierarchy and also decide the number of hierarchy levels.

Figure \ref{fig:hgvspa} shows the number of messages needed to converge within  error for graphs of sizes  to . The bottom curve tagged \textit{MultiscaleGossip} shows the ideal case where computation inside each cell stops automatically when the desired accuracy is reached. The curve labeled \textit{MultiscaleGossipFI} was generated using fixed number of iterations per level based on worst case graph sizes and the curve labeled \textit{MultiscaleGossip2level} was generated using only two levels of hierarchy and an  instead of . Both of these variants are explained below. For path averaging we also simulate the ideal scenario where nodes stop transmitting automatically when achieving the targeted accuracy. As we see all variants of  multiscale gossip use noticeably fewer transmissions than path averaging. One reason why path averaging seems to be slower than in \cite{benezit07} is because we use  a smaller connectivity radius for our graphs ( instead of ). 


\begin{figure} 
\begin{center}
\includegraphics[width=8cm]{./figures/HGvsPA2.pdf}
\caption{\label{fig:hgvspa} Comparison of multiscale gossip to path averaging on random geometric graphs of increasing sizes. MultiscaleGossip used with  levels of hierarchy. MultiscaleGossipFI is the version using a fixed number of iterations for gossiping at a specific level. MultiscaleGossip2level is a version using only two levels of Hierarchy and is explained in Section \ref{sec:consider}. Results are averages over  random geometric graphs with  nodes and final accuracy  reached on all runs. All graphs are created with radius .} 
\end{center}
\end{figure}


Figure \ref{fig:cdf} depicts the cumulative density functions of transmissions for multiscale gossip and path averaging. Specifically, we plot the fraction of nodes that transmitted  times or less for a random geometric graph with  nodes. Both path averaging and multiscale gossip were stopped as soon as the desired error level is reached: . As we see, the node with most transmissions in multiscale gossip still sends fewer messages than about  of the nodes in path averaging.  

\begin{figure}
\begin{center}
\includegraphics[width=8cm]{./figures/comparecdf.pdf}
\caption{\label{fig:cdf} Cumulative density function of the probability that a node sends less of equal to  messages as a function of  on a random geometric graph with  nodes. The node with maximum number of transmissions for multiscale gossip has less transmissions that about  of the nodes in path averaging.} 
\end{center}
\end{figure}


Multiscale gossip has several advantages over Path Averaging. All the information relies on pairwise messages. In contrast, averaging over paths of length more than two has two main disadvantages. First, if a message is lost, a large number of nodes (potentially ) are affected by the information loss. Second, when messages are sent to a remote location over many hops, they increase in size as the message body accumulates the information of all the intermediate nodes. Besides being variable, the message size now depends on the length of the path and ultimately on the network size. Our messages are always of constant size and independent of the hop distance or network size. Moreover, the maximum number of hops any message has to travel is  at worst\footnote{At level  the distance in hops between leaders is at worst  for .}. This should be compared to distance  which is necessary for path averaging to achieve linear scaling. Finally, multiscale gossip is relatively easy to analyze and implement using standard randomized gossip as a building block for the averaging computations.

\subsubsection*{Fixed Number of Iterations per Level} The ideal scenario for multiscale gossip is if computation inside each cell stops automatically when the desired accuracy is reached. This way no messages are wasted. However in practice for cells at the same level may need to gossip on graphs of different sizes that take different numbers of messages to converge. This creates a need for node synchronization so that all computation in one level is finished before the next level can begin. To alleviate the synchronization issue, we can fix the number of randomized gossip iterations per level so that all computation between different subgraphs at the same level takes practically the same amount of time. However, we need to be careful not to perform fewer iterations than needed for the desired accuracy. Given that nodes are deployed uniformly at random in the unit square, we can make a worst case estimate of how many nodes are expected to be in a cell of a certain area. Since by construction all cells at the same level have equal area, we gossip on all graphs at that level for a fixed number of iterations. Moreover, as seen in the previous section, we can use enough levels of hierarchy to only have  nodes at the last level. This can ensure that we will not do less iterations that necessary. In practice, usually at level , we have less nodes than expected so we end up wasting messages running gossip for longer than necessary. 

\subsubsection*{Two-level Gossip} multiscale gossip is a synchronized algorithm where computation in one level begins after the previous level has converged. Synchronization can be complicated or inefficient if we have too many levels. This motivates trying an algorithm with only two levels. In this case, for graphs of size a few thousand nodes, splitting the unit square into  cells with  is not a good choice as it produces a very small grid of representatives and quite large level- cells. To achieve better load balancing between the two levels, we use . This choice has the advantage that the maximum number of hops any message has to travel is . To see this is true, observe that each  cell  has area . Thus the maximum distance between representatives is . If we divide by the connecting radius  we get the result. Another interesting finding is that for moderate sized graphs, using cells of area  produces subgraphs which are very well connected. Since nodes are deployed uniformly at random, an area  is expected to contain  nodes. A subgraph inside a   cell  is still a random geometric graph with  nodes, but for which the radius used to connect nodes is not . It is . This is equivalent to creating a random geometric graph of  nodes in the unit square but with a scaled up radius of . From \cite{cover_mixing_ran_geom_ercal_07} we know that a random geometric graph of  nodes is rapidly mixing (i.e. linear number of messages for convergence) if the connecting radius is . Now, e.g. for  and , we get   for . Consequently, the  cells are rapidly mixing for networks of less than a few millions of nodes. Figure \ref{fig:hgvspa} verifies this analysis. For graphs from  to  nodes and final error , we see that \textit{MultiscaleGossip2level} performs very close to multiscale gossip with more levels of hierarchy and better than path averaging.




\subsection{Operating under Transmission Failures}

As explained in the previous section, multiscale gossip needs to send messages to shorter distances across the network than Path Averaging. It is important to see what effect this has on the robustness of the algorithms against transmission failures. Two different but general scenarios are considered. In the first scenario, no message is truly lost. There is a non-zero probability that a message over a network edge is not sent successfully, but the nodes communicate via a hand shake mechanism so messages are eventually delivered after a number of attempts. If the probability of successful transmission is , then the cost for a single message transmission over an edge is geometrically distributed : . The second scenario is more extreme. Each message is delivered with probability  over each edge, otherwise it is lost. This model has  severe consequences. Depending on where in its path the message is lost, a number of nodes will not update their values properly so besides the overall delay in convergence, part of the signal energy is lost and the final estimate of the average is no longer guaranteed to be close to the true average. 

\subsubsection{Hand Shake Model}

In this scenario we don not have to worry about convergence. All messages are eventually delivered and it is just a matter of time. Figure \ref{fig:handshake} shows the results of multiscale gossip against path averaging on networks of different sizes. The probability of successful transmission also varies from  to . As we see multiscale gossip is significantly less affected by such failures. This example clearly illustrates the importance of not having to send messages in long distances of the network. Since each individual link introduces some delay, the fact that messages in multiscale gossip usually don't need to travel far and need to go  hops at most, allow the algorithm to converge using much fewer messages than path averaging.

\begin{figure} [t]
\begin{center}
\includegraphics[width=8cm]{./figures/HandShake.pdf}
\caption{\label{fig:handshake} Multiscale gossip and Path averaging on graphs of varius sizes using hand shakes to overcome transmission failures. The probability of successful transmission  varies from  to . All results are averaged over 25 graphs of each size. Final accuracy is  and the random geometric graphs use .} 
\end{center}
\end{figure}

\subsubsection{Message Loss Model}

In this scenario, a message is delivered with probability  or lost forever. This severely impedes the algorithms from converging fast. Moreover, information is lost permanently distorting the final result and making it impossible to meet the desired final accuracy. The amount of distortion depends on where along a multi-hop path the failure occurs. For example in path averaging, if the message is lost at the first transmission on its way back, all the nodes along the path except the last will have distorted information. Similar situations occur in multiscale gossip between leader communications where messages need to travel across multiple hops. Since in this scenario we have no guarantee that the desired final accuracy criterion will be met, it is hard to draw conclusions whether multiscale gossip is better than path averaging. Our observations showed that both algorithms can only reach an accuracy in the order of  when targeting at . Specifically, multiscale gossip would only reach up to  accuracy while path averaging could not improve beyond .  At the same time the total number of messages still increases linearly for multiscale gossip while it seem to blow up exponentially for path averaging. 




%
