




\documentclass[10pt]{article}

\setcounter{secnumdepth}{5}

\usepackage[letterpaper,margin=1.0in]{geometry}
\usepackage{latexsym}   \usepackage{verbatim}
\usepackage{color}
\usepackage{enumerate}
\usepackage{amsmath}
\usepackage{amsthm}
\usepackage{amsfonts}
\usepackage{amssymb}
\usepackage{url}
\usepackage{geometry}
\usepackage{graphicx}
\usepackage{textcomp}
\usepackage{xspace}    

\usepackage{textcomp}
\usepackage{listings}
\lstdefinestyle{numbers}
{numbers=none, stepnumber=1, numberblanklines=false,
numberstyle=\tiny, numbersep=8pt}
\lstdefinestyle{nonumbers}
{numbers=none}
\lstset{
captionpos=b
language=C++,         basicstyle=\small,    style=numbers,        emptylines=*1,        breaklines=true,      escapeinside=<>,      framesep=2.5mm,       basicstyle=\ttfamily,
upquote=true,
frame=single,
}

\usepackage{longtable}
\newenvironment{ttlongtable}[1]{\ttfamily \begin{longtable}{#1}}{\end{longtable}}

\usepackage{hyperref}
\hypersetup{
  pdfauthor={Sou-Cheng T. Choi},
  pdftitle={GAIL---Guaranteed Automatic Integration Library in MATLAB: Documentation for Version 2.1},
  pdfsubject={GAIL---Guaranteed Automatic Integration Library in MATLAB: Documentation for Version 2.1},
  pdfkeywords={automatic, adaptive, fixed-cost, guaranteed, Monte Carlo, Quasi Monte Carlo, linear splines, reliable, reproducible},
  pdfpagemode=UseOutlines,
  pdfstartview=FitH,
  bookmarks,
  bookmarksopen,
  colorlinks,
  linkcolor=blue,
  citecolor=blue,
  urlcolor=blue,
  hypertexnames=false
}
\newcommand{\todo}[1]{{\color{blue}~\textsf{[\footnotesize TODO: #1]}}}

\sloppy
\definecolor{lightgray}{gray}{0.2}
\setlength{\parindent}{0pt}

\newcommand{\partref}[1]{Part~\ref{part:#1}}
\newcommand{\secref}[1]{Section~\ref{sec:#1}}
\newcommand{\tabref}[1]{Table~\ref{tab:#1}}
\newcommand{\figref}[1]{Figure~\ref{fig:#1}}

\newcommand{\stripe}{{\hrule height 0.25pt\hfil}}

\newcounter{example}[section]
\renewcommand{\theexample}{\arabic{section}.\arabic{example}}
\newenvironment{example}[1][\relax]{\addtocounter{example}{1}\par\vspace{1\baselineskip}\expandafter\ifx#1\relax
{\noindent\scshape\bfseries Example~\theexample.}\else
{\noindent\scshape\bfseries Example~\theexample.}{\nondent\bfseries#1.}\fi\stripe\\
}
{\stripe\vspace{1\baselineskip}\par}

\newcommand{\ntt}{\normalfont\ttfamily}
\newcommand{\cn}[1]{{\protect\ntt\bslash#1}}
\newcommand{\pkg}[1]{{\protect\ntt#1}}
\newcommand{\fn}[1]{{\protect\ntt#1}}
\newcommand{\env}[1]{{\protect\ntt#1}}
\hfuzz1pc 





\newtheorem{thm}{Theorem}[section]
\newtheorem{cor}[thm]{Corollary}
\newtheorem{lem}[thm]{Lemma}
\newtheorem{prop}[thm]{Proposition}
\newtheorem{ax}{Axiom}
\newtheorem{assumption}{Assumption}[section]








\newcommand{\thmref}[1]{Theorem~\ref{#1}}
\newcommand{\lemref}[1]{Lemma~\ref{#1}}

\newcommand{\bysame}{\mbox{\rule{3em}{.4pt}}\,}



\newcommand{\A}{\mathcal{A}}
\newcommand{\B}{\mathcal{B}}
\newcommand{\st}{\sigma}
\newcommand{\XcY}{{(X,Y)}}
\newcommand{\SX}{{S_X}}
\newcommand{\SY}{{S_Y}}
\newcommand{\SXY}{{S_{X,Y}}}
\newcommand{\SXgYy}{{S_{X|Y}(y)}}
\newcommand{\Cw}[1]{{\hat C_#1(X|Y)}}
\newcommand{\G}{{G(X|Y)}}
\newcommand{\PY}{{P_{\mathcal{Y}}}}
\newcommand{\X}{\mathcal{X}}
\newcommand{\wt}{\widetilde}
\newcommand{\wh}{\widehat}

\DeclareMathOperator{\per}{per}
\DeclareMathOperator{\cov}{cov}
\DeclareMathOperator{\non}{non}
\DeclareMathOperator{\cf}{cf}
\DeclareMathOperator{\add}{add}
\DeclareMathOperator{\Cham}{Cham}
\DeclareMathOperator{\IM}{Im}
\DeclareMathOperator{\esssup}{ess\,sup}
\DeclareMathOperator{\meas}{meas}
\DeclareMathOperator{\seg}{seg}

\newcommand{\interval}[1]{\mathinner{#1}}

\newcommand{\eval}[2][\right]{\relax
  \ifx#1\right\relax \left.\fi#2#1\rvert}

\newcommand{\envert}[1]{\left\lvert#1\right\rvert}
\let\abs=\envert

\newcommand{\enVert}[1]{\left\lVert#1\right\rVert}
\let\norm=\enVert

\newcommand{\T}{^T\!}
\newcommand{\submax}{_{\max}}
\newcommand{\submin}{_{\min}}
\newcommand{\pmat}[1]{{\renewcommand{\arraystretch}{1.1}\begin{pmatrix}#1\end{pmatrix}}}  \newcommand{\bmat}[1]{{\renewcommand{\arraystretch}{1.1}\begin{bmatrix}#1\end{bmatrix}}}
\newcommand{\smat}[1]{{\renewcommand{\arraystretch}{1.1}\left[\begin{smallmatrix}#1\end{smallmatrix}\right]}}


\newenvironment{api}
{\minipage{0.8\textwidth}\center\lstset{framexleftmargin=-10mm,frame=single}\lstlisting}
{\endlstlisting\endcenter\endminipage} 



\def\bi{\begin{itemize}}
\def\be{\begin{enumerate}}
\def\bc{\begin{center}}
\def\ii{\item}
\def\ei{\end{itemize}}
\def\ee{\end{enumerate}}
\def\ec{\end{center}}
 
 

\begin{document}
\pagenumbering{roman}

\begin{comment}
\subsection{Contents}

\begin{itemize}
\setlength{\itemsep}{-1ex}
   \item Guaranteed Automatic Integration Library (GAIL) 2.1 User Guide
   \item Functions
   \item Installation
   \item Tests
   \item Website
   \item Functions
   \item 1-D approximation
   \item 1-D integration
   \item High dimension integration
   \item 1-D minimization
   \item funappx\_g
   \item Syntax
   \item Description
   \item Guarantee
   \item Examples
   \item See Also
   \item References
   \item funmin\_g
   \item Syntax
   \item Description
   \item Guarantee
   \item Examples
   \item See Also
   \item References
   \item integral\_g
   \item Syntax
   \item Description
   \item Guarantee
   \item Examples
   \item See Also
   \item References
   \item meanMC\_g
   \item Syntax
   \item Description
   \item Guarantee
   \item Examples
   \item See Also
   \item References
   \item meanMCBer\_g
   \item Syntax
   \item Description
   \item Guarantee
   \item Examples
   \item See Also
   \item References
   \item cubMC\_g
   \item Syntax
   \item Description
   \item Guarantee
   \item Examples
   \item See Also
   \item References
   \item cubLattice\_g
   \item Syntax
   \item Description
   \item Guarantee
   \item Examples
   \item See Also
   \item References
   \item cubSobol\_g
   \item Syntax
   \item Description
   \item Guarantee
   \item Examples
   \item See Also
   \item References
   \item Installation Instructions
\end{itemize}
\end{comment}

\title{GAIL---Guaranteed Automatic Integration Library in MATLAB: Documentation for Version 2.1\thanks{Our work was supported in part by grants from the National Science Foundation under grant NSF-DMS-1115392, and the Office of Advanced Scientific Computing Research, Office of Science, U.S. Department of Energy, under contract DE-AC02-06CH11357.
  }}
\author{Sou-Cheng T. Choi\thanks{NORC at the University of
    Chicago, Chicago, IL 60603 and Illinois Institute of Technology, Chicago 606016, IL; e-mail: \url{sctchoi@uchicago.edu}.} \and
    Yuhan Ding\thanks{Illinois Institute of Technology; e-mail: \url{yding2@hawk.iit.edu}.}  \and
    Fred J.~Hickernell\thanks{Illinois Institute of Technology; e-mail: \url{hickernell@iit.edu}.}  \and
    Lan Jiang\thanks{ Illinois Institute of Technology; e-mail: \url{ljiang14@hawk.iit.edu}.}  \and
    Llu\'is Antoni Jim\'enez Rugama\thanks{Illinois Institute of Technology; e-mail: \url{ lluisantoni@gmail.com}.}  \and
    Xin Tong  \thanks{University of Illinois at Chicago; e-mail: \url{xtong5@hawk.iit.edu}.}  \and
    Yizhi Zhang\thanks{Illinois Institute of Technology; e-mail: \url{yzhang97@hawk.iit.edu}.}  \and
    Xuan Zhou\thanks{Illinois Institute of Technology; e-mail: \url{xuanjzhou@gmail.com}.} 
}\date{Mar 14, 2015}
 

\maketitle






\newpage
\section*{Open Source License}


 
\bigskip
\noindent
Copyright \textcopyright\ 2015, Illinois Institute of Technology. All rights reserved.
 
Redistribution and use in source and binary forms, with or without 
modification, are permitted provided that the following conditions are 
met:

\begin{itemize}
\item  Redistributions of source code must retain the above copyright 
    notice, this list of conditions and the following disclaimer.
    
\item Redistributions in binary form must reproduce the above copyright 
    notice, this list of conditions and the following disclaimer in the 
    documentation and/or other materials provided with the distribution.
    
\item Neither the name of Illinois Institute of Technology nor the names of
    its contributors may be used to endorse or promote products derived 
    from this software without specific prior written permission.
\end{itemize}
 
THIS SOFTWARE IS PROVIDED BY THE COPYRIGHT HOLDER AND CONTRIBUTORS 
``AS IS'' AND WITHOUT ANY WARRANTY OF ANY KIND, WHETHER EXPRESS, IMPLIED, 
STATUTORY OR OTHERWISE, INCLUDING WITHOUT LIMITATION WARRANTIES OF 
MERCHANTABILITY, FITNESS FOR A PARTICULAR USE AND NON-INFRINGEMENT, ALL 
OF WHICH ARE HEREBY EXPRESSLY DISCLAIMED. MOREOVER, THE USER OF THE 
SOFTWARE UNDERSTANDS AND AGREES THAT THE SOFTWARE MAY CONTAIN BUGS, 
DEFECTS, ERRORS AND OTHER PROBLEMS THAT COULD CAUSE SYSTEM FAILURES, AND 
ANY USE OF THE SOFTWARE SHALL BE AT USER?S OWN RISK. THE COPYRIGHT 
HOLDERS AND CONTRIBUTORS MAKES NO REPRESENTATION THAT THEY WILL ISSUE 
UPDATES OR ENHANCEMENTS TO THE SOFTWARE.  
 
\bigskip
 
IN NO EVENT WILL THE COPYRIGHT HOLDER OR CONTRIBUTORS BE LIABLE FOR ANY 
DIRECT, INDIRECT, SPECIAL, INCIDENTAL, CONSEQUENTIAL, EXEMPLARY OR 
PUNITIVE DAMAGES, INCLUDING, BUT NOT LIMITED TO, DAMAGES FOR INTERRUPTION 
OF USE OR FOR LOSS OR INACCURACY OR CORRUPTION OF DATA, LOST PROFITS, OR 
COSTS OF PROCUREMENT OF SUBSTITUTE GOODS OR SERVICES, HOWEVER CAUSED 
(INCLUDING BUT NOT LIMITED TO USE, MISUSE, INABILITY TO USE, OR 
INTERRUPTED USE) AND UNDER ANY THEORY OF LIABILITY, INCLUDING BUT NOT 
LIMITED TO CONTRACT, STRICT LIABILITY, OR TORT (INCLUDING NEGLIGENCE OR 
OTHERWISE) ARISING IN ANY WAY OUT OF THE USE OF THIS SOFTWARE, EVEN IF 
ADVISED OF THE POSSIBILITY OF SUCH DAMAGE AND WHETHER OR NOT THE 
COPYRIGHT HOLDER AND CONTRIBUTORS WAS OR SHOULD HAVE BEEN AWARE OR 
ADVISED OF THE POSSIBILITY OF SUCH DAMAGE OR FOR ANY CLAIM ALLEGING 
INJURY RESULTING FROM ERRORS, OMISSIONS, OR OTHER INACCURACIES IN THE 
SOFTWARE OR DESTRUCTIVE PROPERTIES OF THE SOFTWARE.  TO THE EXTENT THAT 
THE LAWS OF ANY JURISDICTIONS DO NOT ALLOW THE FOREGOING EXCLUSIONS AND 
LIMITATION, THE USER OF THE SOFTWARE AGREES THAT DAMAGES MAY BE 
DIFFICULT, IF NOT IMPOSSIBLE TO CALCULATE, AND AS A RESULT, SAID USER HAS 
AGREED THAT THE MAXIMUM LIABILITY OF THE COPYRITGHT HOLDER AND 
CONTRIBUTORS SHALL NOT EXCEED US\[a,b]f[t_{l-1},t_l]1 \le l \le Lfappxf\mathrm{fmin}\mathrm{exitflag} = 0.fq\mathrm{exceedbudget} = 0.$



\subsection{Examples}

\begin{par}
\textbf{Example 1}
\end{par} \vspace{1em}
\begin{verbatim}
f = @(x) x.^2; [q, out_param] = integral_g(f)

\end{verbatim}

        \color{lightgray} \begin{verbatim}
q =

    0.3333


out_param = 

               f: @(x)x.^2
               a: 0
               b: 1
          abstol: 1.0000e-06
             nlo: 10
             nhi: 1000
            nmax: 10000000
         maxiter: 1000
           ninit: 100
             tau: 197
    exceedbudget: 0
       tauchange: 0
            iter: 2
               q: 0.3333
         npoints: 3565
          errest: 9.9688e-07
\end{verbatim} \color{black}
\begin{par}
\textbf{Example 2}
\end{par} \vspace{1em}

\begin{verbatim}
[q, out_param] = integral_g(@(x) exp(-x.^2),'a',1,'b',2,...
   'nlo',100,'nhi',10000,'abstol',1e-5,'nmax',1e7)

\end{verbatim}

        \color{lightgray} \begin{verbatim}
q =

    0.1353


out_param = 

               a: 1
          abstol: 1.0000e-05
               b: 2
               f: @(x)exp(-x.^2)
         maxiter: 1000
             nhi: 10000
             nlo: 100
            nmax: 10000000
           ninit: 1000
             tau: 1997
    exceedbudget: 0
       tauchange: 0
            iter: 2
               q: 0.1353
         npoints: 2998
          errest: 7.3718e-06

\end{verbatim} \color{black}
    
\subsection{See Also}
\begin{par}
integral, quad, funappx\_g, meanMC\_g, cubMC\_g, funmin\_g
\end{par} \vspace{1em}
\begin{comment}
\begin{par}

\end{par} \vspace{1em}

\subsection{References}

\begin{par}
[1]  Nick Clancy, Yuhan Ding, Caleb Hamilton, Fred J. Hickernell, and Yizhi Zhang, The Cost of Deterministic, Adaptive, Automatic Algorithms: Cones, Not Balls, Journal of Complexity 30 (2014), pp. 21-45.
\end{par} \vspace{1em}
\begin{par}
[2]  Sou-Cheng T. Choi, Yuhan Ding, Fred J. Hickernell, Lan Jiang, Lluis Antoni Jimenez Rugama, Xin Tong, Yizhi Zhang and Xuan Zhou, "GAIL: Guaranteed Automatic Integration Library (Version 2.1)" [MATLAB Software], 2015. Available from \begin{verbatim}http://code.google.com/p/gail/\end{verbatim}
\end{par} \vspace{1em}
\begin{par}
[3] Sou-Cheng T. Choi, "MINRES-QLP Pack and Reliable Reproducible Research via Supportable Scientific Software", Journal of Open Research Software, Volume 2, Number 1, e22, pp. 1-7, 2014.
\end{par} \vspace{1em}
\begin{par}
[4] Sou-Cheng T. Choi and Fred J. Hickernell, "IIT MATH-573 Reliable Mathematical Software" [Course Slides], Illinois Institute of Technology, Chicago, IL, 2013. Available from \begin{verbatim}http://code.google.com/p/gail/\end{verbatim}
\end{par} \vspace{1em}
\begin{par}
[5] Daniel S. Katz, Sou-Cheng T. Choi, Hilmar Lapp, Ketan Maheshwari, Frank Loffler, Matthew Turk, Marcus D. Hanwell, Nancy Wilkins-Diehr, James Hetherington, James Howison, Shel Swenson, Gabrielle D. Allen, Anne C. Elster, Bruce Berriman, Colin Venters, "Summary of the First Workshop On Sustainable Software for Science: Practice And Experiences (WSSSPE1)", Journal of Open Research Software, Volume 2, Number 1, e6, pp. 1-21, 2014.
\end{par} \vspace{1em}
\begin{par}
If you find GAIL helpful in your work, please support us by citing the above papers, software, and materials.
\end{par} \vspace{1em}
\end{comment}

\newpage
\section{meanMC\_g}

\begin{par}
Monte Carlo method to estimate the mean of a random variable
\end{par} \vspace{1em}


\subsection{Syntax}

\begin{par}
tmu = \textbf{meanMC\_g}(Yrand)
\end{par} \vspace{1em}
\begin{par}
tmu = \textbf{meanMC\_g}(Yrand,abstol,reltol,alpha)
\end{par} \vspace{1em}
\begin{par}
tmu = \textbf{meanMC\_g}(Yrand,'abstol',abstol,'reltol',reltol,'alpha',alpha)
\end{par} \vspace{1em}
\begin{par}
[tmu, out\_param] = \textbf{meanMC\_g}(Yrand,in\_param)
\end{par} \vspace{1em}


\subsection{Description}

\begin{par}
tmu = \textbf{meanMC\_g}(Yrand) estimates the mean, mu, of a random variable Y to  within a specified generalized error tolerance,  tolfun:=max(abstol,reltol*\ensuremath{|} mu \ensuremath{|}), i.e., \ensuremath{|} mu - tmu \ensuremath{|} \ensuremath{<}= tolfun with  probability at least 1-alpha, where abstol is the absolute error  tolerance, and reltol is the relative error tolerance. Usually the  reltol determines the accuracy of the estimation, however, if the \ensuremath{|} mu \ensuremath{|}  is rather small, the abstol determines the accuracy of the estimation.  The default values are abstol=1e-2, reltol=1e-1, and alpha=1\%. Input  Yrand is a function handle that accepts a positive integer input n and  returns an n x 1 vector of IID instances of the random variable Y.
\end{par} \vspace{1em}
\begin{par}
tmu = \textbf{meanMC\_g}(Yrand,abstol,reltol,alpha) estimates the mean of a  random variable Y to within a specified generalized error tolerance  tolfun with guaranteed confidence level 1-alpha using all ordered  parsing inputs abstol, reltol, alpha.
\end{par} \vspace{1em}
\begin{par}
tmu = \textbf{meanMC\_g}(Yrand,'abstol',abstol,'reltol',reltol,'alpha',alpha)  estimates the mean of a random variable Y to within a specified  generalized error tolerance tolfun with guaranteed confidence level  1-alpha. All the field-value pairs are optional and can be supplied in  different order, if a field is not supplied, the default value is used.
\end{par} \vspace{1em}
\begin{par}
[tmu, out\_param] = \textbf{meanMC\_g}(Yrand,in\_param) estimates the mean of a  random variable Y to within a specified generalized error tolerance  tolfun with the given parameters in\_param and produce the estimated  mean tmu and output parameters out\_param. If a field is not specified,  the default value is used.
\end{par} \vspace{1em}
\begin{par}
\textbf{Input Arguments}
\end{par} \vspace{1em}
\begin{itemize}
\setlength{\itemsep}{-1ex}
   \item Yrand --- the function for generating n IID instances of a random  variable Y whose mean we want to estimate. Y is often defined as a  function of some random variable X with a simple distribution. The  input of Yrand should be the number of random variables n, the output  of Yrand should be n function values. For example, if Y = X.\^{}2 where X  is a standard uniform random variable, then one may define Yrand =  @(n) rand(n,1).\^{}2.
\end{itemize}
\begin{itemize}
\setlength{\itemsep}{-1ex}
   \item in\_param.abstol --- the absolute error tolerance, which should be  positive, default value is 1e-2.
\end{itemize}
\begin{itemize}
\setlength{\itemsep}{-1ex}
   \item in\_param.reltol --- the relative error tolerance, which should be  between 0 and 1, default value is 1e-1.
\end{itemize}
\begin{itemize}
\setlength{\itemsep}{-1ex}
   \item in\_param.alpha --- the uncertainty, which should be a small positive  percentage. default value is 1\%.
\end{itemize}
\begin{par}
\textbf{Optional Input Arguments}
\end{par} \vspace{1em}
\begin{itemize}
\setlength{\itemsep}{-1ex}
   \item in\_param.fudge --- standard deviation inflation factor, which should  be larger than 1, default value is 1.2.
\end{itemize}
\begin{itemize}
\setlength{\itemsep}{-1ex}
   \item in\_param.nSig --- initial sample size for estimating the sample  variance, which should be a moderate large integer at least 30, the  default value is 1e4.
\end{itemize}
\begin{itemize}
\setlength{\itemsep}{-1ex}
   \item in\_param.n1 --- initial sample size for estimating the sample mean,  which should be a moderate large positive integer at least 30, the  default value is 1e4.
\end{itemize}
\begin{itemize}
\setlength{\itemsep}{-1ex}
   \item in\_param.tbudget --- the time budget in seconds to do the two-stage  estimation, which should be positive, the default value is 100 seconds.
\end{itemize}
\begin{itemize}
\setlength{\itemsep}{-1ex}
   \item in\_param.nbudget --- the sample budget to do the two-stage  estimation, which should be a large positive integer, the default  value is 1e9.
\end{itemize}
\begin{par}
\textbf{Output Arguments}
\end{par} \vspace{1em}
\begin{itemize}
\setlength{\itemsep}{-1ex}
   \item tmu --- the estimated mean of Y.
\end{itemize}
\begin{itemize}
\setlength{\itemsep}{-1ex}
   \item out\_param.tau --- the iteration step.
\end{itemize}
\begin{itemize}
\setlength{\itemsep}{-1ex}
   \item out\_param.n --- the sample size used in each iteration.
\end{itemize}
\begin{itemize}
\setlength{\itemsep}{-1ex}
   \item out\_param.nremain --- the remaining sample budget to estimate mu. It was  calculated by the sample left and time left.
\end{itemize}
\begin{itemize}
\setlength{\itemsep}{-1ex}
   \item out\_param.ntot --- total sample used.
\end{itemize}
\begin{itemize}
\setlength{\itemsep}{-1ex}
   \item out\_param.hmu --- estimated mean in each iteration.
\end{itemize}
\begin{itemize}
\setlength{\itemsep}{-1ex}
   \item out\_param.tol --- the reliable upper bound on error for each iteration.
\end{itemize}
\begin{itemize}
\setlength{\itemsep}{-1ex}
   \item out\_param.var --- the sample variance.
\end{itemize}
\begin{itemize}
\setlength{\itemsep}{-1ex}
   \item out\_param.exit --- the state of program when exiting.

         0 \quad Success
    
         1 \quad Not enough samples to estimate the mean
\end{itemize}
    \begin{itemize}
\setlength{\itemsep}{-1ex}
   \item out\_param.kurtmax --- the upper bound on modified kurtosis.
\end{itemize}
\begin{itemize}
\setlength{\itemsep}{-1ex}
   \item out\_param.time --- the time elapsed in seconds.
\end{itemize}
\begin{itemize}
\setlength{\itemsep}{-1ex}
   \item out\_param.flag --- parameter checking status
   
          1  \quad checked by meanMC\_g
\end{itemize}



\subsection{Guarantee}

\begin{par}
This algorithm attempts to calculate the mean, mu, of a random variable to a prescribed error tolerance, tolfun:= max(abstol,reltol*\ensuremath{|} mu \ensuremath{|}), with guaranteed confidence level 1-alpha. If the algorithm terminated without showing any warning messages and provide an answer tmu, then the follow inequality would be satisfied:
\end{par} \vspace{1em}
Pr(\ensuremath{|} mu - tmu \ensuremath{|} \ensuremath{<}= tolfun) \ensuremath{>}= 1-alpha
The cost of the algorithm, N\_tot, is also bounded above by N\_up, which is defined in terms of abstol, reltol, nSig, n1, fudge, kurtmax, beta. And the following inequality holds:
Pr (N\_tot \ensuremath{<}= N\_up) \ensuremath{>}= 1-beta
Please refer to our paper for detailed arguments and proofs.



\subsection{Examples}

\begin{par}
\textbf{Example 1}
\end{par} \vspace{1em}
\begin{verbatim}


  in_param.reltol=0; in_param.abstol = 1e-3; in_param.reltol = 0;
  in_param.alpha = 0.05; Yrand=@(n) rand(n,1).^2;
  tmu = meanMC_g(Yrand,in_param)
\end{verbatim}

        \color{lightgray} \begin{verbatim}
tmu =

    0.3331

\end{verbatim} \color{black}
    \begin{par}
\textbf{Example 2}
\end{par} \vspace{1em}
\begin{verbatim}


  tmu = meanMC_g(@(n)exp(rand(n,1)),1e-3,0)
\end{verbatim}

        \color{lightgray} \begin{verbatim}
tmu =

    1.7185

\end{verbatim} \color{black}
    \begin{par}
\textbf{Example 3}
\end{par} \vspace{1em}
\begin{verbatim}


  tmu = meanMC_g(@(n)cos(rand(n,1)),'reltol',1e-2,'abstol',0,...
      'alpha',0.05)
\end{verbatim}

        \color{lightgray} \begin{verbatim}
tmu =

    0.8415

\end{verbatim} \color{black}
    

\subsection{See Also}

\begin{par}
funappx\_g, integral\_g, cubMC\_g, meanMCBer\_g, cubSobol\_g, cubLattice\_g
\end{par} \vspace{1em}
\begin{comment}
\begin{par}

\end{par} \vspace{1em}
\begin{par}

\end{par} \vspace{1em}
\begin{par}

\end{par} \vspace{1em}
\begin{par}

\end{par} \vspace{1em}
\begin{par}

\end{par} \vspace{1em}

\subsection{References}

\begin{par}
[1]  F. J. Hickernell, L. Jiang, Y. Liu, and A. B. Owen, Guaranteed conservative fixed width confidence intervals via Monte Carlo sampling, Monte Carlo and Quasi-Monte Carlo Methods 2012 (J. Dick, F. Y. Kuo, G. W. Peters, and I. H. Sloan, eds.), Springer-Verlag, Berlin, 2014. arXiv:1208.4318 [math.ST]
\end{par} \vspace{1em}
\begin{par}
[2] Sou-Cheng T. Choi, Yuhan Ding, Fred J. Hickernell, Lan Jiang, Lluis Antoni Jimenez Rugama, Xin Tong, Yizhi Zhang and Xuan Zhou, "GAIL: Guaranteed Automatic Integration Library (Version 2.1)" [MATLAB Software], 2015. Available from \begin{verbatim}http://code.google.com/p/gail/\end{verbatim}
\end{par} \vspace{1em}
\begin{par}
[3] Sou-Cheng T. Choi, "MINRES-QLP Pack and Reliable Reproducible Research via Supportable Scientific Software", Journal of Open Research Software, Volume 2, Number 1, e22, pp. 1-7, 2014.
\end{par} \vspace{1em}
\begin{par}
[4] Sou-Cheng T. Choi and Fred J. Hickernell, "IIT MATH-573 Reliable Mathematical Software" [Course Slides], Illinois Institute of Technology, Chicago, IL, 2013. Available from \begin{verbatim}http://code.google.com/p/gail/\end{verbatim}
\end{par} \vspace{1em}
\begin{par}
[5] Daniel S. Katz, Sou-Cheng T. Choi, Hilmar Lapp, Ketan Maheshwari, Frank Loffler, Matthew Turk, Marcus D. Hanwell, Nancy Wilkins-Diehr, James Hetherington, James Howison, Shel Swenson, Gabrielle D. Allen, Anne C. Elster, Bruce Berriman, Colin Venters, "Summary of the First Workshop On Sustainable Software for Science: Practice And Experiences (WSSSPE1)", Journal of Open Research Software, Volume 2, Number 1, e6, pp. 1-21, 2014.
\end{par} \vspace{1em}
\begin{par}
If you find GAIL helpful in your work, please support us by citing the above papers, software, and materials.
\end{par} \vspace{1em}
\end{comment}

\newpage
\section{meanMCBer\_g}

\begin{par}
Monte Carlo method to estimate the mean of a Bernoulli random variable to within a specified absolute error tolerance with guaranteed confidence level 1-alpha.
\end{par} \vspace{1em}


\subsection{Syntax}

\begin{par}
pHat = \textbf{meanMCBer\_g}(Yrand)
\end{par} \vspace{1em}
\begin{par}
pHat = \textbf{meanMCBer\_g}(Yrand,abstol,alpha,nmax)
\end{par} \vspace{1em}
\begin{par}
pHat = \textbf{meanMCBer\_g}(Yrand,'abstol',abstol,'alpha',alpha,'nmax',nmax)
\end{par} \vspace{1em}
\begin{par}
[pHat, out\_param] = \textbf{meanMCBer\_g}(Yrand,in\_param)
\end{par} \vspace{1em}


\subsection{Description}

\begin{par}
pHat = \textbf{meanMCBer\_g}(Yrand) estimates the mean of a Bernoulli random  variable Y to within a specified absolute error tolerance with  guaranteed confidence level 99\%. Input Yrand is a function handle that  accepts a positive integer input n and returns a n x 1 vector of IID  instances of the Bernoulli random variable Y.
\end{par} \vspace{1em}
\begin{par}
pHat = \textbf{meanMCBer\_g}(Yrand,abstol,alpha,nmax) estimates the mean  of a Bernoulli random variable Y to within a specified absolute error  tolerance with guaranteed confidence level 1-alpha using all ordered  parsing inputs abstol, alpha and nmax.
\end{par} \vspace{1em}
\begin{par}
pHat = \textbf{meanMCBer\_g}(Yrand,'abstol',abstol,'alpha',alpha,'nmax',nmax)  estimates the mean of a Bernoulli random variable Y to within a  specified absolute error tolerance with guaranteed confidence level  1-alpha. All the field-value pairs are optional and can be supplied in  different order.
\end{par} \vspace{1em}
\begin{par}
[pHat, out\_param] = \textbf{meanMCBer\_g}(Yrand,in\_param) estimates the mean  of a Bernoulli random variable Y to within a specified absolute error  tolerance with the given parameters in\_param and produce the estimated  mean pHat and output parameters out\_param.
\end{par} \vspace{1em}
\begin{par}
\textbf{Input Arguments}
\end{par} \vspace{1em}
\begin{itemize}
\setlength{\itemsep}{-1ex}
   \item Yrand --- the function for generating IID instances of a Bernoulli            random variable Y whose mean we want to estimate.
\end{itemize}
\begin{itemize}
\setlength{\itemsep}{-1ex}
   \item pHat --- the estimated mean of Y.
\end{itemize}
\begin{itemize}
\setlength{\itemsep}{-1ex}
   \item in\_param.abstol --- the absolute error tolerance, the default value is 1e-2.
\end{itemize}
\begin{itemize}
\setlength{\itemsep}{-1ex}
   \item in\_param.alpha --- the uncertainty, the default value is 1\%.
\end{itemize}
\begin{itemize}
\setlength{\itemsep}{-1ex}
   \item in\_param.nmax --- the sample budget, the default value is 1e9.
\end{itemize}
\begin{par}
\textbf{Output Arguments}
\end{par} \vspace{1em}
\begin{itemize}
\setlength{\itemsep}{-1ex}
   \item out\_param.n --- the total sample used.
\end{itemize}
\begin{itemize}
\setlength{\itemsep}{-1ex}
   \item out\_param.time --- the time elapsed in seconds.
\end{itemize}
\begin{itemize}
\setlength{\itemsep}{-1ex}
   \item out\_param.exit --- the state of program when exiting.

         0 \quad Success

         1 \quad Not enough samples to estimate p with guarantee
\end{itemize}


\subsection{Guarantee}

If the sample size is calculated according Hoeffding's inequality, which equals to ceil(log(2/out\_param.alpha)/(2*out\_param.abstol\^{}2)), then the following inequality must be satisfied:
\begin{par}
Pr(\ensuremath{|} p - pHat \ensuremath{|} \ensuremath{<}= abstol) \ensuremath{>}= 1-alpha.
\end{par} \vspace{1em}
Here p is the true mean of Yrand, and pHat is the output of MEANMCBER\_G.
Also, the cost is deterministic.



\subsection{Examples}

\begin{par}
\textbf{Example 1}
\end{par} \vspace{1em}

\begin{verbatim}


    in_param.abstol = 1e-3; in_param.alpha = 0.01; in_param.nmax = 1e9;
    p=1/9; Yrand=@(n) rand(n,1)<p;
    pHat = meanMCBer_g(Yrand,in_param)
\end{verbatim}

        \color{lightgray} \begin{verbatim}
pHat =

    0.1113

\end{verbatim} \color{black}
    \begin{par}
\textbf{Example 2}
\end{par} \vspace{1em}
\begin{verbatim}


    pHat = meanMCBer_g(Yrand,1e-4)
\end{verbatim}

        \color{lightgray} \begin{verbatim}
pHat =

    0.1111

\end{verbatim} \color{black}
    \begin{par}
\textbf{Example 3}
\end{par} \vspace{1em}
\begin{verbatim}


    pHat = meanMCBer_g(Yrand,'abstol',1e-2,'alpha',0.05)
\end{verbatim}

        \color{lightgray} \begin{verbatim}
pHat =

    0.1118

\end{verbatim} \color{black}
    

\subsection{See Also}

\begin{par}
funappx\_g, integral\_g, cubMC\_g, meanMC\_g, cubLattice\_g, cubSobol\_g
\end{par} \vspace{1em}
\begin{comment}
\begin{par}

\end{par} \vspace{1em}
\begin{par}

\end{par} \vspace{1em}
\begin{par}

\end{par} \vspace{1em}
\begin{par}

\end{par} \vspace{1em}
\begin{par}

\end{par} \vspace{1em}

\subsection{References}

\begin{par}
[1]  F. J. Hickernell, L. Jiang, Y. Liu, and A. B. Owen, Guaranteed conservative fixed width confidence intervals via Monte Carlo sampling, Monte Carlo and Quasi-Monte Carlo Methods 2012 (J. Dick, F. Y. Kuo, G. W. Peters, and I. H. Sloan, eds.), Springer-Verlag, Berlin, 2014. arXiv:1208.4318 [math.ST]
\end{par} \vspace{1em}
\begin{par}
[2] Sou-Cheng T.  Choi, Yuhan Ding, Fred J. Hickernell, Lan Jiang, Lluis Antoni Jimenez Rugama, Xin Tong, Yizhi Zhang and Xuan Zhou, "GAIL: Guaranteed Automatic Integration Library (Version 2.1)" [MATLAB Software], 2015. Available from \begin{verbatim}http://code.google.com/p/gail/\end{verbatim}
\end{par} \vspace{1em}
\begin{par}
[3] Sou-Cheng T. Choi, "MINRES-QLP Pack and Reliable Reproducible Research via Supportable Scientific Software", Journal of Open Research Software, Volume 2, Number 1, e22, pp. 1-7, 2014.
\end{par} \vspace{1em}
\begin{par}
[4] Sou-Cheng T. Choi and Fred J. Hickernell, "IIT MATH-573 Reliable Mathematical Software" [Course Slides], Illinois Institute of Technology, Chicago, IL, 2013. Available from \begin{verbatim}http://code.google.com/p/gail/\end{verbatim}
\end{par} \vspace{1em}
\begin{par}
[5] Daniel S. Katz, Sou-Cheng T. Choi, Hilmar Lapp, Ketan Maheshwari, Frank Loffler, Matthew Turk, Marcus D. Hanwell, Nancy Wilkins-Diehr, James Hetherington, James Howison, Shel Swenson, Gabrielle D. Allen, Anne C. Elster, Bruce Berriman, Colin Venters, "Summary of the First Workshop On Sustainable Software for Science: Practice And Experiences (WSSSPE1)", Journal of Open Research Software, Volume 2, Number 1, e6, pp. 1-21, 2014.
\end{par} \vspace{1em}
\begin{par}
If you find GAIL helpful in your work, please support us by citing the above papers, software, and materials.
\end{par} \vspace{1em}
\end{comment}

\newpage
\section{cubMC\_g}

\begin{par}
Monte Carlo method to evaluate a multidimensional integral
\end{par} \vspace{1em}


\subsection{Syntax}

\begin{par}
[Q,out\_param] = \textbf{cubMC\_g}(f,hyperbox)
\end{par} \vspace{1em}
\begin{par}
Q = \textbf{cubMC\_g}(f,hyperbox,measure,abstol,reltol,alpha)
\end{par} \vspace{1em}
\begin{par}
Q = \textbf{cubMC\_g}(f,hyperbox,'measure',measure,'abstol',abstol,'reltol',reltol,'alpha',alpha)
\end{par} \vspace{1em}
\begin{par}
[Q out\_param] = \textbf{cubMC\_g}(f,hyperbox,in\_param)
\end{par} \vspace{1em}


\subsection{Description}

\begin{par}
[Q,out\_param] = \textbf{cubMC\_g}(f,hyperbox) estimates the integral of f over  hyperbox to within a specified generalized error tolerance, tolfun =  max(abstol, reltol*\ensuremath{|} I \ensuremath{|}), i.e., \ensuremath{|} I - Q \ensuremath{|} \ensuremath{<}= tolfun with probability at  least 1-alpha, where abstol is the absolute error tolerance, and reltol  is the relative error tolerance. Usually the reltol determines the  accuracy of the estimation, however, if the \ensuremath{|} I \ensuremath{|} is rather small, the  abstol determines the accuracy of the estimation. The default values  are abstol=1e-2, reltol=1e-1, and alpha=1\%. Input f is a function  handle that accepts an n x d matrix input, where d is the dimension of  the hyperbox, and n is the number of points being evaluated  simultaneously. The input hyperbox is a 2 x d matrix, where the first  row corresponds to the lower limits and the second row corresponds to  the upper limits.
\end{par} \vspace{1em}
\begin{par}
Q = \textbf{cubMC\_g}(f,hyperbox,measure,abstol,reltol,alpha)  estimates the integral of function f over hyperbox to within a  specified generalized error tolerance tolfun with guaranteed confidence  level 1-alpha using all ordered parsing inputs f, hyperbox, measure,  abstol, reltol, alpha, fudge, nSig, n1, tbudget, nbudget, flag. The  input f and hyperbox are required and others are optional.
\end{par} \vspace{1em}
\begin{par}
Q = \textbf{cubMC\_g}(f,hyperbox,'measure',measure,'abstol',abstol,'reltol',reltol,'alpha',alpha)  estimates the integral of f over hyperbox to within a specified  generalized error tolerance tolfun with guaranteed confidence level  1-alpha. All the field-value pairs are optional and can be supplied in  different order. If an input is not specified, the default value is used.
\end{par} \vspace{1em}
\begin{par}
[Q out\_param] = \textbf{cubMC\_g}(f,hyperbox,in\_param) estimates the integral of  f over hyperbox to within a specified generalized error tolerance  tolfun with the given parameters in\_param and produce output parameters  out\_param and the integral Q.
\end{par} \vspace{1em}
\begin{par}
\textbf{Input Arguments}
\end{par} \vspace{1em}
\begin{itemize}
\setlength{\itemsep}{-1ex}
   \item f --- the integrand.
\end{itemize}
\begin{itemize}
\setlength{\itemsep}{-1ex}
   \item hyperbox --- the integration hyperbox. The default value is  [zeros(1,d); ones(1,d)], the default d is 1.
\end{itemize}
\begin{itemize}
\setlength{\itemsep}{-1ex}
   \item in\_param.measure --- the measure for generating the random variable,  the default is 'uniform'. The other measure could be handled is  'normal'/'Gaussian'. The input should be a string type, hence with  quotes.
\end{itemize}
\begin{itemize}
\setlength{\itemsep}{-1ex}
   \item in\_param.abstol --- the absolute error tolerance, the default value  is 1e-2.
\end{itemize}
\begin{itemize}
\setlength{\itemsep}{-1ex}
   \item in\_param.reltol --- the relative error tolerance, the default value  is 1e-1.
\end{itemize}
\begin{itemize}
\setlength{\itemsep}{-1ex}
   \item in\_param.alpha --- the uncertainty, the default value is 1\%.
\end{itemize}
\begin{par}
\textbf{Optional Input Arguments}
\end{par} \vspace{1em}
\begin{itemize}
\setlength{\itemsep}{-1ex}
   \item in\_param.fudge --- the standard deviation inflation factor, the  default value is 1.2.
\end{itemize}
\begin{itemize}
\setlength{\itemsep}{-1ex}
   \item in\_param.nSig --- initial sample size for estimating the sample  variance, which should be a moderate large integer at least 30, the  default value is 1e4.
\end{itemize}
\begin{itemize}
\setlength{\itemsep}{-1ex}
   \item in\_param.n1 --- initial sample size for estimating the sample mean,  which should be a moderate large positive integer at least 30, the  default value is 1e4.
\end{itemize}
\begin{itemize}
\setlength{\itemsep}{-1ex}
   \item in\_param.tbudget --- the time budget to do the estimation, the  default value is 100 seconds.
\end{itemize}
\begin{itemize}
\setlength{\itemsep}{-1ex}
   \item in\_param.nbudget --- the sample budget to do the estimation, the  default value is 1e9.
\end{itemize}
\begin{itemize}
\setlength{\itemsep}{-1ex}
   \item in\_param.flag --- the value corresponds to parameter checking status.

               0 \quad not checked
    
               1 \quad checked by meanMC\_g
    
               2 \quad checked by cubMC\_g

\end{itemize}
    \begin{par}
\textbf{Output Arguments}
\end{par} \vspace{1em}
\begin{itemize}
\setlength{\itemsep}{-1ex}
   \item Q --- the estimated value of the integral.
\end{itemize}
\begin{itemize}
\setlength{\itemsep}{-1ex}
   \item out\_param.n --- the sample size used in each iteration.
\end{itemize}
\begin{itemize}
\setlength{\itemsep}{-1ex}
   \item out\_param.ntot --- total sample used.
\end{itemize}
\begin{itemize}
\setlength{\itemsep}{-1ex}
   \item out\_param.nremain --- the remaining sample budget to estimate I. It was  calculated by the sample left and time left.
\end{itemize}
\begin{itemize}
\setlength{\itemsep}{-1ex}
   \item out\_param.tau --- the iteration step.
\end{itemize}
\begin{itemize}
\setlength{\itemsep}{-1ex}
   \item out\_param.hmu --- estimated integral in each iteration.
\end{itemize}
\begin{itemize}
\setlength{\itemsep}{-1ex}
   \item out\_param.tol --- the reliable upper bound on error for each iteration.
\end{itemize}
\begin{itemize}
\setlength{\itemsep}{-1ex}
   \item out\_param.kurtmax --- the upper bound on modified kurtosis.
\end{itemize}
\begin{itemize}
\setlength{\itemsep}{-1ex}
   \item out\_param.time --- the time elapsed in seconds.
\end{itemize}
\begin{itemize}
\setlength{\itemsep}{-1ex}
   \item out\_param.var --- the sample variance.
\end{itemize}
\begin{itemize}
\setlength{\itemsep}{-1ex}
   \item out\_param.exit --- the state of program when exiting.

               0 \quad success
    
               1 \quad Not enough samples to estimate the mean
    
              10 \quad hyperbox does not contain numbers
   
              11 \quad hyperbox is not 2 x d
    
              12 \quad hyperbox is only a point in one direction
    
              13 \quad hyperbox is infinite when measure is 'uniform'

              14 \quad hyperbox is not doubly infinite when measure is 'normal'
\end{itemize}


\subsection{Guarantee}

\begin{par}
This algorithm attempts to calculate the integral of function f over a hyperbox to a prescribed error tolerance tolfun:= max(abstol,reltol*\ensuremath{|} I \ensuremath{|}) with guaranteed confidence level 1-alpha. If the algorithm terminated without showing any warning messages and provide an answer Q, then the follow inequality would be satisfied:
\end{par} \vspace{1em}
\begin{par}
Pr(\ensuremath{|} Q - I \ensuremath{|} \ensuremath{<}= tolfun) \ensuremath{>}= 1-alpha
\end{par} \vspace{1em}
\begin{par}
The cost of the algorithm, N\_tot, is also bounded above by N\_up, which is a function in terms of abstol, reltol, nSig, n1, fudge, kurtmax, beta. And the following inequality holds:
\end{par} \vspace{1em}
\begin{par}
Pr (N\_tot \ensuremath{<}= N\_up) \ensuremath{>}= 1-beta
\end{par} \vspace{1em}
\begin{par}
Please refer to our paper for detailed arguments and proofs.
\end{par} \vspace{1em}


\subsection{Examples}

\begin{par}
\textbf{Example 1}
\end{par} \vspace{1em}
\begin{verbatim}


 f = @(x) sin(x); interval = [1;2];
 Q = cubMC_g(f,interval,'uniform',1e-3,1e-2)
\end{verbatim}

        \color{lightgray} \begin{verbatim}
Q =

    0.9564

\end{verbatim} \color{black}
    \begin{par}
\textbf{Example 2}
\end{par} \vspace{1em}
\begin{verbatim}


 f = @(x) exp(-x(:,1).^2-x(:,2).^2); hyperbox = [0 0;1 1];
 Q = cubMC_g(f,hyperbox,'measure','uniform','abstol',1e-3,...
     'reltol',1e-13)
\end{verbatim}

        \color{lightgray} \begin{verbatim}
Q =

    0.5574

\end{verbatim} \color{black}
    \begin{par}
\textbf{Example 3}
\end{par} \vspace{1em}
\begin{verbatim}



  d = 3;f = @(x) 2^d*prod(x,2)+0.555; hyperbox = [zeros(1,d);ones(1,d)];
  in_param.abstol = 1e-3; in_param.reltol=1e-3;
  Q = cubMC_g(f,hyperbox,in_param)
\end{verbatim}

        \color{lightgray} \begin{verbatim}
Q =

    1.5549

\end{verbatim} \color{black}
    \begin{par}
\textbf{Example 4}
\end{par} \vspace{1em}
\begin{verbatim}



 f = @(x) exp(-x(:,1).^2-x(:,2).^2); hyperbox = [-inf -inf;inf inf];
 Q = cubMC_g(f,hyperbox,'normal',0,1e-2)
\end{verbatim}

        \color{lightgray} \begin{verbatim}

Q =

    0.3328

\end{verbatim} \color{black}
    

\subsection{See Also}

\begin{par}
funappx\_g, integral\_g, meanMC\_g, meanMCBer\_g, cubLattice\_g, cubSobol\_g
\end{par} \vspace{1em}
\begin{comment}
\begin{par}

\end{par} \vspace{1em}
\begin{par}

\end{par} \vspace{1em}
\begin{par}

\end{par} \vspace{1em}
\begin{par}

\end{par} \vspace{1em}
\begin{par}

\end{par} \vspace{1em}

\subsection{References}

\begin{par}
[1]  F. J. Hickernell, L. Jiang, Y. Liu, and A. B. Owen, Guaranteed conservative fixed width confidence intervals via Monte Carlo sampling, Monte Carlo and Quasi-Monte Carlo Methods 2012 (J. Dick, F. Y. Kuo, G. W. Peters, and I. H. Sloan, eds.), pp. 105-128, Springer-Verlag, Berlin, 2014 DOI: 10.1007/978-3-642-41095-6\_5
\end{par} \vspace{1em}
\begin{par}
[2] Sou-Cheng T. Choi, Yuhan Ding, Fred J. Hickernell, Lan Jiang, Lluis Antoni Jimenez Rugama, Xin Tong, Yizhi Zhang and Xuan Zhou, "GAIL: Guaranteed Automatic Integration Library (Version 2.1)" [MATLAB Software], 2015. Available from \begin{verbatim}http://code.google.com/p/gail/\end{verbatim}
\end{par} \vspace{1em}
\begin{par}
[3] Sou-Cheng T. Choi, "MINRES-QLP Pack and Reliable Reproducible Research via Supportable Scientific Software", Journal of Open Research Software, Volume 2, Number 1, e22, pp. 1-7, DOI: \begin{verbatim}http://dx.doi.org/10.5334/jors.bb\end{verbatim}, 2014.
\end{par} \vspace{1em}
\begin{par}
[4] Sou-Cheng T. Choi and Fred J. Hickernell, "IIT MATH-573 Reliable Mathematical Software" [Course Slides], Illinois Institute of Technology, Chicago, IL, 2013. Available from \begin{verbatim}http://code.google.com/p/gail/\end{verbatim}
\end{par} \vspace{1em}
\begin{par}
[5] Daniel S. Katz, Sou-Cheng T. Choi, Hilmar Lapp, Ketan Maheshwari, Frank Loffler, Matthew Turk, Marcus D. Hanwell, Nancy Wilkins-Diehr, James Hetherington, James Howison, Shel Swenson, Gabrielle D. Allen, Anne C. Elster, Bruce Berriman, Colin Venters, "Summary of the First Workshop On Sustainable Software for Science: Practice And Experiences (WSSSPE1)", Journal of Open Research Software, Volume 2, Number 1, e6, pp. 1-21, 2014.
\end{par} \vspace{1em}
\begin{par}
If you find GAIL helpful in your work, please support us by citing the above papers, software, and materials.
\end{par} \vspace{1em}

\end{comment}

\newpage
\section{cubLattice\_g}

\begin{par}
Quasi-Monte Carlo method using rank-1 Lattices cubature over a d-dimensional region to integrate within a specified generalized error tolerance with guarantees under Fourier coefficients cone decay assumptions.
\end{par} \vspace{1em}


\subsection{Syntax}

\begin{par}
[q,out\_param] = \textbf{cubLattice\_g}(f,hyperbox)
\end{par} \vspace{1em}
\begin{par}
q = \textbf{cubLattice\_g}(f,hyperbox,measure,abstol,reltol)
\end{par} \vspace{1em}
\begin{par}
q = \textbf{cubLattice\_g}(f,hyperbox,'measure',measure,'abstol',abstol,'reltol',reltol)
\end{par} \vspace{1em}
\begin{par}
q = \textbf{cubLattice\_g}(f,hyperbox,in\_param)
\end{par} \vspace{1em}


\subsection{Description}


\begin{par}
[q,out\_param] = \textbf{cubLattice\_g}(f,hyperbox) estimates the integral of f  over the d-dimensional region described by hyperbox, and with an error  guaranteed not to be greater than a specific generalized error tolerance,  tolfun:=max(abstol,reltol*\ensuremath{|} integral(f) \ensuremath{|}). Input f is a function handle. f should  accept an n x d matrix input, where d is the dimension and n is the  number of points being evaluated simultaneously. The input hyperbox is  a 2 x d matrix, where the first row corresponds to the lower limits  and the second row corresponds to the upper limits of the integral.  Given the construction of our Lattices, d must be a positive integer  with 1\ensuremath{<}=d\ensuremath{<}=250.
\end{par} \vspace{1em}
\begin{par}
q = \textbf{cubLattice\_g}(f,hyperbox,measure,abstol,reltol)  estimates the integral of f over the hyperbox. The answer  is given within the generalized error tolerance tolfun. All parameters  should be input in the order specified above. If an input is not specified,  the default value is used. Note that if an input is not specified,  the remaining tail cannot be specified either. Inputs f and hyperbox  are required. The other optional inputs are in the correct order:  measure,abstol,reltol,shift,mmin,mmax,fudge,transform,toltype and  theta.
\end{par} \vspace{1em}
\begin{par}
q = \textbf{cubLattice\_g}(f,hyperbox,'measure',measure,'abstol',abstol,'reltol',reltol)  estimates the integral of f over the hyperbox. The answer  is given within the generalized error tolerance tolfun. All the field-value  pairs are optional and can be supplied in any order. If an input is not  specified, the default value is used.
\end{par} \vspace{1em}
\begin{par}
q = \textbf{cubLattice\_g}(f,hyperbox,in\_param) estimates the integral of f over the  hyperbox. The answer is given within the generalized error tolerance tolfun.
\end{par} \vspace{1em}
\begin{par}
\textbf{Input Arguments}
\end{par} \vspace{1em}
\begin{itemize}
\setlength{\itemsep}{-1ex}
   \item f --- the integrand whose input should be a matrix n x d where n is  the number of data points and d the dimension, which cannot be  greater than 250. By default f is f=@ x.\^{}2.
\end{itemize}
\begin{itemize}
\setlength{\itemsep}{-1ex}
   \item hyperbox --- the integration region defined by its bounds. It must be  a 2 x d matrix, where the first row corresponds to the lower limits  and the second row corresponds to the upper limits of the integral.  The default value is [0;1].
\end{itemize}
\begin{itemize}
\setlength{\itemsep}{-1ex}
   \item in\_param.measure --- for f(x)*mu(dx), we can define mu(dx) to be the  measure of a uniformly distributed random variable in they hyperbox  or normally distributed with covariance matrix I\_d. The only possible  values are 'uniform' or 'normal'. For 'uniform', the hyperbox must be  a finite volume while for 'normal', the hyperbox can only be defined as  (-Inf,Inf)\^{}d. By default it is 'uniform'.
\end{itemize}
\begin{itemize}
\setlength{\itemsep}{-1ex}
   \item in\_param.abstol --- the absolute error tolerance, abstol\ensuremath{>}=0. By  default it is 1e-4.
\end{itemize}
\begin{itemize}
\setlength{\itemsep}{-1ex}
   \item in\_param.reltol --- the relative error tolerance, which should be  in [0,1]. Default value is 1e-2.
\end{itemize}
\begin{par}
\textbf{Optional Input Arguments}
\end{par} \vspace{1em}
\begin{itemize}
\setlength{\itemsep}{-1ex}
   \item in\_param.shift --- the Rank-1 lattices can be shifted to avoid the  origin or other particular points. By default we consider a uniformly  [0,1) random shift.
\end{itemize}
\begin{itemize}
\setlength{\itemsep}{-1ex}
   \item in\_param.mmin --- the minimum number of points to start is 2\^{}mmin.  The cone condition on the Fourier coefficients decay requires a  minimum number of points to start. The advice is to consider at least  mmin=10. mmin needs to be a positive integer with mmin\ensuremath{<}=mmax. By  default it is 10.
\end{itemize}
\begin{itemize}
\setlength{\itemsep}{-1ex}
   \item in\_param.mmax --- the maximum budget is 2\^{}mmax. By construction of  our Lattices generator, mmax is a positive integer such that  mmin\ensuremath{<}=mmax\ensuremath{<}=26. The default value is 24.
\end{itemize}
\begin{itemize}
\setlength{\itemsep}{-1ex}
   \item in\_param.fudge --- the positive function multiplying the finite  sum of Fast Fourier coefficients specified in the cone of functions.  This input is a function handle. The fudge should accept an array of  nonnegative integers being evaluated simultaneously. For more  technical information about this parameter, refer to the references.  By default it is @(m) 5*2.\^{}-m.
\end{itemize}
\begin{itemize}
\setlength{\itemsep}{-1ex}
   \item in\_param.transform --- the algorithm is defined for continuous  periodic functions. If the input function f is not, there are 5  types of transform to periodize it without modifying the result.  By default it is the Baker's transform. The options are:   
   
     'id' : no transformation.  
     
    'Baker' : Baker's transform or tent map in each coordinate. Preserving              only continuity but simple to compute. Chosen by default.   
    
    'C0' : polynomial transformation only preserving continuity.    
    
    'C1' : polynomial transformation preserving the first derivative.    
    
    'C1sin' : Sidi's transform with sine, preserving the first derivative.              This is in general a better option than 'C1'.
\end{itemize}
\begin{itemize}
\setlength{\itemsep}{-1ex}
   \item in\_param.toltype --- this is the generalized tolerance function.  There are two choices, 'max' which takes  max(abstol,reltol*\ensuremath{|} integral(f) \ensuremath{|} ) and 'comb' which is the linear combination  theta*abstol+(1-theta)*reltol*\ensuremath{|} integral(f) \ensuremath{|} . Theta is another  parameter to be specified with 'comb'(see below). For pure absolute  error, either choose 'max' and set reltol = 0 or choose 'comb' and set  theta = 1. For pure relative error, either choose 'max' and set  abstol = 0 or choose 'comb' and set theta = 0. Note that with 'max',  the user can not input abstol = reltol = 0 and with 'comb', if theta = 1  abstol con not be 0 while if theta = 0, reltol can not be 0.  By default toltype is 'max'.
\end{itemize}
\begin{itemize}
\setlength{\itemsep}{-1ex}
   \item in\_param.theta --- this input is parametrizing the toltype  'comb'. Thus, it is only active when the toltype  chosen is 'comb'. It establishes the linear combination weight  between the absolute and relative tolerances  theta*abstol+(1-theta)*reltol*\ensuremath{|} integral(f) \ensuremath{|}. Note that for theta = 1,  we have pure absolute tolerance while for theta = 0, we have pure  relative tolerance. By default, theta=1.
\end{itemize}
\begin{par}
\textbf{Output Arguments}
\end{par} \vspace{1em}
\begin{itemize}
\setlength{\itemsep}{-1ex}
   \item q --- the estimated value of the integral.
\end{itemize}
\begin{itemize}
\setlength{\itemsep}{-1ex}
   \item out\_param.d --- dimension over which the algorithm integrated.
\end{itemize}
\begin{itemize}
\setlength{\itemsep}{-1ex}
   \item out\_param.n --- number of Rank-1 lattice points used for computing  the integral of f.
\end{itemize}
\begin{itemize}
\setlength{\itemsep}{-1ex}
   \item out\_param.bound\_err --- predicted bound on the error based on the cone  condition. If the function lies in the cone, the real error will be  smaller than generalized tolerance.
\end{itemize}
\begin{itemize}
\setlength{\itemsep}{-1ex}
   \item out\_param.time --- time elapsed in seconds when calling cubLattice\_g.
\end{itemize}
\begin{itemize}
\setlength{\itemsep}{-1ex}
   \item out\_param.exitflag --- this is a binary vector stating whether  warning flags arise. These flags tell about which conditions make the  final result certainly not guaranteed. One flag is considered arisen  when its value is 1. The following list explains the flags in the  respective vector order:

                    1 \quad   If reaching overbudget. It states whether
                    the max budget is attained without reaching the
                    guaranteed error tolerance.

                    2 \quad  If the function lies outside the cone. In
                    this case, results are not guaranteed. Note that
                    this parameter is computed on the transformed
                    function, not the input function. For more
                    information on the transforms, check the input
                    parameter in\_param.transform; for information about
                    the cone definition, check the article mentioned
                    below.
                    
\end{itemize}

\subsection{Guarantee}

\begin{par}
This algorithm computes the integral of real valued functions in dimension d with a prescribed generalized error tolerance. The Fourier coefficients of the integrand are assumed to be absolutely convergent. If the algorithm terminates without warning messages, the output is given with guarantees under the assumption that the integrand lies inside a cone of functions. The guarantee is based on the decay rate of the Fourier coefficients. For more details on how the cone is defined, please refer to the references below.
\end{par} \vspace{1em}


\subsection{Examples}

\begin{par}
\textbf{Example 1}
\end{par} \vspace{1em}
\begin{verbatim}


  f = @(x) prod(x,2); hyperbox = [zeros(1,2);ones(1,2)];
  q = cubLattice_g(f,hyperbox,'uniform',1e-5,0,'transform','C1sin')
\end{verbatim}

        \color{lightgray} \begin{verbatim}
q =

    0.2500

\end{verbatim} \color{black}
    \begin{par}
\textbf{Example 2}
\end{par} \vspace{1em}
\begin{verbatim}


  f = @(x) x(:,1).^2.*x(:,2).^2.*x(:,3).^2; hyperbox = [-inf(1,3);inf(1,3)];
  q = cubLattice_g(f,hyperbox,'normal',1e-3,1e-3,'transform','C1sin')
\end{verbatim}

        \color{lightgray} \begin{verbatim}
q =

    1.0000

\end{verbatim} \color{black}
    \begin{par}
\textbf{Example 3}
\end{par} \vspace{1em}
\begin{verbatim}


  f = @(x) exp(-x(:,1).^2-x(:,2).^2); hyperbox = [-ones(1,2);2*ones(1,2)];
  q = cubLattice_g(f,hyperbox,'uniform',1e-3,1e-2,'transform','C1')
\end{verbatim}

        \color{lightgray} \begin{verbatim}
q =

    2.6532

\end{verbatim} \color{black}
    \begin{par}
\textbf{Example 4}
\end{par} \vspace{1em}
\begin{verbatim}


  f = @(x) exp(-0.05^2/2)*max(100*exp(0.05*x)-100,0); hyperbox = [-inf(1,1);inf(1,1)];
  q = cubLattice_g(f,hyperbox,'normal',1e-4,1e-2,'transform','C1sin')
\end{verbatim}

        \color{lightgray} \begin{verbatim}
q =

    2.0563

\end{verbatim} \color{black}
    \begin{par}
\textbf{Example 5}
\end{par} \vspace{1em}
\begin{verbatim}


  f = @(x) 8*prod(x,2); hyperbox = [zeros(1,5);ones(1,5)];
  q = cubLattice_g(f,hyperbox,'uniform',1e-5,0)
\end{verbatim}

        \color{lightgray} \begin{verbatim}
q =

    0.2500
\end{verbatim} \color{black}
    \begin{par}
\textbf{Example 6}
\end{par} \vspace{1em}
\begin{verbatim}


  f = @(x) 3./(5-4*(cos(2*pi*x))); hyperbox = [0;1];
  q = cubLattice_g(f,hyperbox,'uniform',1e-5,0,'transform','id')
\end{verbatim}

        \color{lightgray} \begin{verbatim}
q =

    1.0000
\end{verbatim} \color{black}

\subsection{See Also}
\begin{par}
cubSobol\_g, cubMC\_g, meanMC\_g,  meanMCBer\_g, integral\_g
\end{par} \vspace{1em}
\begin{comment}
\begin{par}

\end{par} \vspace{1em}
\begin{par}

\end{par} \vspace{1em}
\begin{par}

\end{par} \vspace{1em}

\subsection{References}

\begin{par}
[1] Lluis Antoni Jimenez Rugama and Fred J. Hickernell: Adaptive Multidimensional Integration Based on Rank-1 Lattices (2014). Submitted for publication: arXiv:1411.1966.
\end{par} \vspace{1em}
\begin{par}
[2] Sou-Cheng T. Choi, Fred J. Hickernell, Yuhan Ding, Lan Jiang, Lluis Antoni Jimenez Rugama, Xin Tong, Yizhi Zhang and Xuan Zhou, "GAIL: Guaranteed Automatic Integration Library (Version 2.1)" [MATLAB Software], 2015. Available from \begin{verbatim}http://code.google.com/p/gail/\end{verbatim}
\end{par} \vspace{1em}
\begin{par}
[3] Sou-Cheng T. Choi, "MINRES-QLP Pack and Reliable Reproducible Research via Supportable Scientific Software", Journal of Open Research Software, Volume 2, Number 1, e22, pp. 1-7, DOI: \begin{verbatim}http://dx.doi.org/10.5334/jors.bb\end{verbatim}, 2014.
\end{par} \vspace{1em}
\begin{par}
[4] Sou-Cheng T. Choi and Fred J. Hickernell, "IIT MATH-573 Reliable Mathematical Software" [Course Slides], Illinois Institute of Technology, Chicago, IL, 2013. Available from \begin{verbatim}http://code.google.com/p/gail/\end{verbatim}
\end{par} \vspace{1em}
\begin{par}
[5] Daniel S. Katz, Sou-Cheng T. Choi, Hilmar Lapp, Ketan Maheshwari, Frank Loffler, Matthew Turk, Marcus D. Hanwell, Nancy Wilkins-Diehr, James Hetherington, James Howison, Shel Swenson, Gabrielle D. Allen, Anne C. Elster, Bruce Berriman, Colin Venters, "Summary of the First Workshop On Sustainable Software for Science: Practice And Experiences (WSSSPE1)", Journal of Open Research Software, Volume 2, Number 1, e6, pp. 1-21, 2014.
\end{par} \vspace{1em}
\begin{par}
If you find GAIL helpful in your work, please support us by citing the above papers, software, and materials.
\end{par} \vspace{1em}

\end{comment}

\newpage
\section{cubSobol\_g}

\begin{par}
Quasi-Monte Carlo method using Sobol' cubature over the d-dimensional region to integrate within a specified generalized error tolerance with guarantees under Walsh-Fourier coefficients cone decay assumptions
\end{par} \vspace{1em}


\subsection{Syntax}

\begin{par}
[q,out\_param] = \textbf{cubSobol\_g}(f,hyperbox)
\end{par} \vspace{1em}
\begin{par}
q = \textbf{cubSobol\_g}(f,hyperbox,measure,abstol,reltol)
\end{par} \vspace{1em}
\begin{par}
q = \textbf{cubSobol\_g}(f,hyperbox,'measure',measure,'abstol',abstol,'reltol',reltol)
\end{par} \vspace{1em}
\begin{par}
q = \textbf{cubSobol\_g}(f,hyperbox,in\_param)
\end{par} \vspace{1em}


\subsection{Description}

\begin{par}
[q,out\_param] = \textbf{cubSobol\_g}(f,hyperbox) estimates the integral of f  over the d-dimensional region described by hyperbox, and with an error  guaranteed not to be greater than a specific generalized error tolerance,  tolfun:=max(abstol,reltol*\ensuremath{|} integral(f) \ensuremath{|}). Input f is a function handle. f should  accept an n x d matrix input, where d is the dimension and n is the  number of points being evaluated simultaneously. The input hyperbox is  a 2 x d matrix, where the first row corresponds to the lower limits  and the second row corresponds to the upper limits of the integral.  Given the construction of Sobol' sequences, d must be a positive  integer with 1\ensuremath{<}=d\ensuremath{<}=1111.
\end{par} \vspace{1em}
\begin{par}
q = \textbf{cubSobol\_g}(f,hyperbox,measure,abstol,reltol)  estimates the integral of f over the hyperbox. The answer  is given within the generalized error tolerance tolfun. All parameters  should be input in the order specified above. If an input is not specified,  the default value is used. Note that if an input is not specified,  the remaining tail cannot be specified either. Inputs f and hyperbox  are required. The other optional inputs are in the correct order:  measure,abstol,reltol,mmin,mmax,fudge,toltype and  theta.
\end{par} \vspace{1em}
\begin{par}
q = \textbf{cubSobol\_g}(f,hyperbox,'measure',measure,'abstol',abstol,'reltol',reltol)  estimates the integral of f over the hyperbox. The answer  is given within the generalized error tolerance tolfun. All the field-value  pairs are optional and can be supplied in any order. If an input is not  specified, the default value is used.
\end{par} \vspace{1em}
\begin{par}
q = \textbf{cubSobol\_g}(f,hyperbox,in\_param) estimates the integral of f over the  hyperbox. The answer is given within the generalized error tolerance tolfun.
\end{par} \vspace{1em}
\begin{par}
\textbf{Input Arguments}
\end{par} \vspace{1em}
\begin{itemize}
\setlength{\itemsep}{-1ex}
   \item f --- the integrand whose input should be a matrix n x d where n is  the number of data points and d the dimension, which cannot be  greater than 1111. By default f is f=@ x.\^{}2.
\end{itemize}
\begin{itemize}
\setlength{\itemsep}{-1ex}
   \item hyperbox --- the integration region defined by its bounds. It must be  a 2 x d matrix, where the first row corresponds to the lower limits  and the second row corresponds to the upper limits of the integral.  The default value is [0;1].
\end{itemize}
\begin{itemize}
\setlength{\itemsep}{-1ex}
   \item in\_param.measure --- for f(x)*mu(dx), we can define mu(dx) to be the  measure of a uniformly distributed random variable in the hyperbox  or normally distributed with covariance matrix I\_d. The only possible  values are 'uniform' or 'normal'. For 'uniform', the hyperbox must be  a finite volume while for 'normal', the hyperbox can only be defined as  (-Inf,Inf)\^{}d. By default it is 'uniform'.
\end{itemize}
\begin{itemize}
\setlength{\itemsep}{-1ex}
   \item in\_param.abstol --- the absolute error tolerance, abstol\ensuremath{>}=0. By  default it is 1e-4.
\end{itemize}
\begin{itemize}
\setlength{\itemsep}{-1ex}
   \item in\_param.reltol --- the relative error tolerance, which should be  in [0,1]. Default value is 1e-2.
\end{itemize}
\begin{par}
\textbf{Optional Input Arguments}
\end{par} \vspace{1em}
\begin{itemize}
\setlength{\itemsep}{-1ex}
   \item in\_param.mmin --- the minimum number of points to start is 2\^{}mmin.  The cone condition on the Fourier coefficients decay requires a  minimum number of points to start. The advice is to consider at least  mmin=10. mmin needs to be a positive integer with mmin\ensuremath{<}=mmax. By  default it is 10.
\end{itemize}
\begin{itemize}
\setlength{\itemsep}{-1ex}
   \item in\_param.mmax --- the maximum budget is 2\^{}mmax. By construction of  the Sobol' generator, mmax is a positive integer such that  mmin\ensuremath{<}=mmax\ensuremath{<}=53. The default value is 24.
\end{itemize}
\begin{itemize}
\setlength{\itemsep}{-1ex}
   \item in\_param.fudge --- the positive function multiplying the finite  sum of Fast Walsh Fourier coefficients specified in the cone of functions.  This input is a function handle. The fudge should accept an array of  nonnegative integers being evaluated simultaneously. For more  technical information about this parameter, refer to the references.  By default it is @(m) 5*2.\^{}-m.
\end{itemize}
\begin{itemize}
\setlength{\itemsep}{-1ex}
   \item in\_param.toltype --- this is the generalized tolerance function.  There are two choices, 'max' which takes  max(abstol,reltol*\ensuremath{|} integral(f) \ensuremath{|} ) and 'comb' which is the linear combination  theta*abstol+(1-theta)*reltol*\ensuremath{|} integral(f) \ensuremath{|} . Theta is another  parameter to be specified with 'comb'(see below). For pure absolute  error, either choose 'max' and set reltol = 0 or choose 'comb' and set  theta = 1. For pure relative error, either choose 'max' and set  abstol = 0 or choose 'comb' and set theta = 0. Note that with 'max',  the user can not input abstol = reltol = 0 and with 'comb', if theta = 1  abstol con not be 0 while if theta = 0, reltol can not be 0.  By default toltype is 'max'.
\end{itemize}
\begin{itemize}
\setlength{\itemsep}{-1ex}
   \item in\_param.theta --- this input is parametrizing the toltype  'comb'. Thus, it is only active when the toltype  chosen is 'comb'. It establishes the linear combination weight  between the absolute and relative tolerances  theta*abstol+(1-theta)*reltol*\ensuremath{|} integral(f) \ensuremath{|}. Note that for theta = 1,  we have pure absolute tolerance while for theta = 0, we have pure  relative tolerance. By default, theta=1.
\end{itemize}
\begin{par}
\textbf{Output Arguments}
\end{par} \vspace{1em}
\begin{itemize}
\setlength{\itemsep}{-1ex}
   \item q --- the estimated value of the integral.
\end{itemize}
\begin{itemize}
\setlength{\itemsep}{-1ex}
   \item out\_param.d --- dimension over which the algorithm integrated.
\end{itemize}
\begin{itemize}
\setlength{\itemsep}{-1ex}
   \item out\_param.n --- number of Sobol' points used for computing the  integral of f.
\end{itemize}
\begin{itemize}
\setlength{\itemsep}{-1ex}
   \item out\_param.bound\_err --- predicted bound on the error based on the cone  condition. If the function lies in the cone, the real error will be  smaller than generalized tolerance.
\end{itemize}
\begin{itemize}
\setlength{\itemsep}{-1ex}
   \item out\_param.time --- time elapsed in seconds when calling cubSobol\_g.
\end{itemize}
\begin{itemize}
\setlength{\itemsep}{-1ex}
   \item out\_param.exitflag --- this is a binary vector stating whether  warning flags arise. These flags tell about which conditions make the  final result certainly not guaranteed. One flag is considered arisen  when its value is 1. The following list explains the flags in the  respective vector order:
   
                    1 \quad  If reaching overbudget. It states whether
                    the max budget is attained without reaching the
                    guaranteed error tolerance.

                    2 \quad  If the function lies outside the cone. In
                    this case, results are not guaranteed. For more
                    information about the cone definition, check the
                    article mentioned below.
\end{itemize} 


\subsection{Guarantee}

\begin{par}
This algorithm computes the integral of real valued functions in dimension d with a prescribed generalized error tolerance. The Walsh-Fourier coefficients of the integrand are assumed to be absolutely convergent. If the algorithm terminates without warning messages, the output is given with guarantees under the assumption that the integrand lies inside a cone of functions. The guarantee is based on the decay rate of the Walsh-Fourier coefficients. For more details on how the cone is defined, please refer to the references below.
\end{par} \vspace{1em}


\subsection{Examples}

\begin{par}
\textbf{Example 1}
\end{par} \vspace{1em}
\begin{verbatim}


  f = @(x) prod(x,2); hyperbox = [zeros(1,2);ones(1,2)];
  q = cubSobol_g(f,hyperbox,'uniform',1e-5,0)
\end{verbatim}

        \color{lightgray} \begin{verbatim}
q =

    0.2500

\end{verbatim} \color{black}
    \begin{par}
\textbf{Example 2}
\end{par} \vspace{1em}
\begin{verbatim}


  f = @(x) x(:,1).^2.*x(:,2).^2.*x(:,3).^2; hyperbox = [-inf(1,3);inf(1,3)];
  q = cubSobol_g(f,hyperbox,'normal',1e-3,1e-3)
\end{verbatim}

        \color{lightgray} \begin{verbatim}
q =

    1.0004

\end{verbatim} \color{black}
    \begin{par}
\textbf{Example 3}
\end{par} \vspace{1em}
\begin{verbatim}


  f = @(x) exp(-x(:,1).^2-x(:,2).^2); hyperbox = [-ones(1,2);2*ones(1,2)];
  q = cubSobol_g(f,hyperbox,'uniform',1e-3,1e-2)
\end{verbatim}

        \color{lightgray} \begin{verbatim}
q =

    2.6532

\end{verbatim} \color{black}
    \begin{par}
\textbf{Example 4}
\end{par} \vspace{1em}
\begin{verbatim}


  f = @(x) exp(-0.05^2/2)*max(100*exp(0.05*x)-100,0); hyperbox = [-inf(1,1);inf(1,1)];
  q = cubSobol_g(f,hyperbox,'normal',1e-4,1e-2)
\end{verbatim}

        \color{lightgray} \begin{verbatim}
q =

    2.0552

\end{verbatim} \color{black}
\begin{par}
\textbf{Example 5}
\end{par} \vspace{1em}
\begin{verbatim}


  f = @(x) 8*prod(x,2); hyperbox = [zeros(1,5);ones(1,5)];
  q = cubSobol_g(f,hyperbox,'uniform',1e-5,0)
\end{verbatim}

        \color{lightgray} \begin{verbatim}
q =

    0.2500

\end{verbatim} \color{black}
    

\subsection{See Also}

\begin{par}
cubLattice\_g, cubMC\_g, meanMC\_g, meanMCBer\_g, integral\_g
\end{par} \begin{comment}
\begin{par}

\end{par} \vspace{1em}
\begin{par}

\end{par} \vspace{1em}
\begin{par}

\end{par} \vspace{1em}
\end{comment}

\newpage
 
\frenchspacing
 
\bibliographystyle{plain} \bibliography{rrr-sss-refs}


\begin{comment}
\begin{par}
[1]  Nick Clancy, Yuhan Ding, Caleb Hamilton, Fred J. Hickernell, and Yizhi Zhang,  The Cost of Deterministic, Adaptive, Automatic Algorithms: Cones, Not Ball, Journal of Complexity 30 (2014), pp. 21-45.
\end{par} \vspace{1em}
\begin{par}
[2]  Yuhan Ding, Fred J. Hickernell, and Sou-Cheng T. Choi,  Locally Adaptive Method for Approximating Univariate Functions in Cones with a Guarantee for Accuracy, working, 2015.
\end{par} \vspace{1em}
\begin{par}
[3]  Xin Tong. A Guaranteed, Adaptive, Automatic Algorithm for Univariate Function Minimization. MS thesis, Illinois Institute of Technology, 2014.
\end{par} \vspace{1em}
\begin{par}
[4]  F. J. Hickernell, L. Jiang, Y. Liu, and A. B. Owen, Guaranteed conservative fixed width confidence intervals via Monte Carlo sampling, Monte Carlo and Quasi-Monte Carlo Methods 2012 (J. Dick, F. Y. Kuo, G. W. Peters, and I. H. Sloan, eds.), Springer-Verlag, Berlin, 2014. arXiv:1208.4318 [math.ST]
\end{par} \vspace{1em}
\begin{par}
[5] Lluis Antoni Jimenez Rugama and Fred J. Hickernell: Adaptive Multidimensional Integration Based on Rank-1 Lattices (2014). Submitted for publication: arXiv:1411.1966.
\end{par} \vspace{1em}
\begin{par}
[6] Fred J. Hickernell and Lluis Antoni Jimenez Rugama: Reliable adaptive cubature using digital sequences (2014). Submitted for publication: arXiv:1410.8615.
\end{par} \vspace{1em}
\begin{par}
[7] Sou-Cheng T. Choi, Fred J. Hickernell, Yuhan Ding, Lan Jiang, Lluis Antoni Jimenez Rugama, Xin Tong, Yizhi Zhang and Xuan Zhou, GAIL: Guaranteed Automatic Integration Library (Version 2.1) [MATLAB Software], 2015. Available from \url{http://code.google.com/p/gail/}
\end{par} \vspace{1em}
\begin{par}
[8] Sou-Cheng T. Choi, MINRES-QLP Pack and Reliable Reproducible Research via Supportable Scientific Software, Journal of Open Research Software, Volume 2, Number 1, e22, pp. 1-7, DOI: http://dx.doi.org/10.5334/jors.bb, 2014.
\end{par} \vspace{1em}
\begin{par}
[9] Sou-Cheng T. Choi and Fred J. Hickernell, IIT MATH-573 Reliable Mathematical Software [Course Slides], Illinois Institute of Technology, Chicago, IL, 2013. Available from \url{http://code.google.com/p/gail/}
\end{par} \vspace{1em}
\begin{par}
[10] Daniel S. Katz, Sou-Cheng T. Choi, Hilmar Lapp, Ketan Maheshwari, Frank Loffler, Matthew Turk, Marcus D. Hanwell, Nancy Wilkins-Diehr, James Hetherington, James Howison, Shel Swenson, Gabrielle D. Allen, Anne C. Elster, Bruce Berriman, Colin Venters, Summary of the First Workshop On Sustainable Software for Science: Practice And Experiences (WSSSPE1), Journal of Open Research Software, Volume 2, Number 1, e6, pp. 1-21, 2014.
\end{par} \vspace{1em}
\begin{par}
If you find GAIL helpful in your work, please support us by citing the above papers, software, and materials.
\end{par} \vspace{1em}
\end{comment}



\end{document}
