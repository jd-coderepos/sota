\documentclass{llncs}

\pagestyle{plain}
\usepackage{amsmath}
\usepackage{multirow}
\usepackage{bbding}
\usepackage{tabularx}
\usepackage{array}      

\usepackage{wrapfig}

\usepackage{algorithm}
\usepackage{algpseudocode}
\algnotext{EndFor}
\algnotext{EndIf}

\usepackage{tikz} 
\usetikzlibrary{positioning}

\usepackage{subfigure}
\usepackage{makeidx}

\newcommand{\eqspace}{\quad \quad} 
\newtheorem{observation}{Observation}


\def\lengtharrow{40mm}
\newcommand{\boitegauche}[1]{\begin{minipage}[t][][t]{55mm}\flushright#1~\end{minipage}}
\newcommand{\boitedroite}[1]{\begin{minipage}[t][][t]{55mm}\flushleft#1~\end{minipage}}
\newcommand{\boitecentre}[1]{\begin{minipage}[t][][t]{\lengtharrow}{\hbox to 
\lengtharrow{\hfill#1\hfill}}\end{minipage}}
\newcommand{\flechegauche}[1]{}
\newcommand{\flechedroite}[1]{}
\newcolumntype{C}{>{\scriptsize}{c}}

\newcommand\blfootnote[1]{\begingroup
  \renewcommand\thefootnote{}\footnote{#1}\addtocounter{footnote}{-1}\endgroup
}


\begin{document}



\title{Distance-bounding facing both mafia and distance frauds: Technical 
report }

\author{Rolando Trujillo-Rasua\inst{1}\and Benjamin Martin\inst{2}\and Gildas 
Avoine\inst{2,3}}
\institute{ Interdisciplinary Centre for Security, Reliability and Trust \\
University of Luxembourg \\  \and 
Universit\'e catholique de Louvain\\Information Security Group, Belgium \\ \and 
INSA 
Rennes, IRISA UMR 6074, 
France\\}




\maketitle
\thispagestyle{plain}
\begin{abstract}
Contactless technologies such as RFID, NFC, and sensor networks are vulnerable to mafia and distance frauds. Both frauds aim at passing an authentication protocol by cheating on the actual distance between the prover and the verifier. To cope these security issues, distance-bounding protocols have been designed. However, none of the current proposals simultaneously resists to these two frauds without requiring additional memory and computation. The situation is even worse considering that just a few distance-bounding protocols are able to deal with the inherent background noise on the communication channels. This article introduces a noise-resilient distance-bounding protocol that resists to both mafia and distance frauds. The security of the protocol is analyzed with respect to these two frauds in both scenarios, namely noisy and noiseless channels. Analytical expressions for the adversary's success probabilities are provided, and are illustrated by experimental results. The analysis, performed in an already existing framework for fairness reasons, demonstrates the undeniable advantage of the introduced lightweight design over the previous proposals.

\keywords{Authentication, distance-bounding, relay attack, mafia fraud, distance fraud, noise.}
\end{abstract}

\section{Introduction}
\blfootnote{This document contains content accepted for publication at the IEEE 
Transactions on Wireless Communications~\cite{TMA2014}.}



A \emph{mafia fraud} is a man-in-the-middle attack applied against an authentication protocol where the adversary simply relays the exchanges without neither manipulating nor understanding them~\cite{AvoineBKLM-2011-jcs}. The earliest version of this attack was introduced by Conway in 1976 and is known as the \emph{chess grandmaster problem}~\cite{citeulike:1223195}. In this problem, a little girl is able to compete with two chess grandmasters during a postal chess game, where she transparently relays the moves between the two grandmasters. She eventually wins a game or draws both. 



In modern cryptography, mafia frauds can typically be used against authentication protocols. The adversary relays the messages between the prover and the verifier, who think they communicate together, while there is an adversary in the middle. This so-called mafia fraud was actually suggested by Desmedt, Bengio and Goutier in 1987~\cite{Desmedt:1987:SUS:646752.704723} to defeat the Fiat-Shamir protocol~\cite{FiatS-1986-crypto}. 


One of the most promising field to apply the mafia fraud is the contactless technology, especially Radio Frequency IDentification (RFID) and Near Field Communication (NFC) where the devices answer to any solicitation without explicit agreement of their holder. Some attacks have already been performed against both RFID and NFC systems~\cite{Francis:2010:PNP:1926325.1926331,Hancke:2008:ATD:1352533.1352566}. Nevertheless, mafia fraud is not limited to contactless technologies, it also threats other technologies such as smartcards~\cite{DrimerM-2007-usenix} and e-voting~\cite{OrenW-2009-eprint}.





Two other attacks related to the mafia fraud exist: the \emph{terrorist fraud} and the \emph{distance fraud}. The distance fraud only involves a malicious prover, who cheats on his distance to the verifier. It was first introduced by Brands and Chaum~\cite{Brands:1994:DP:188307.188361}, and comes from the distance measuring process used to defeat the mafia fraud. The terrorist fraud is a variant of the mafia fraud where the prover is malicious and actively helps the adversary to succeed the attack~\cite{BengioBDGQ-1991-crypto}. No solution exists yet to avoid this exotic fraud, which is not addressed in this paper. Additional countermeasure must actually be considered to thwart this fraud. 




As mentioned above, a distance measuring process can mitigate the mafia and distance frauds. To that aim, Brands and Chaum~\cite{Brands:1994:DP:188307.188361} proposed the \emph{distance-bounding protocols} (DB protocols). The distance estimation relies on the measurement of the Round-Trip-Time (RTT) of single bit exchanges. Considering the physical impossibility to travel faster than the speed of light, RTT bounds the distance between the parties. Several distance-bounding protocols have been proposed~\cite{AvoineBKLM-2011-jcs}. However, none of the current DB protocols are lightweight and resistance to both mafia and distance frauds. Furthermore, just a few of them are able
to deal with the inherent background noise of the communication
channel. 




\noindent\textbf{Contribution.} In this paper we introduce a novel DB protocol that significantly reduces the success probability of an adversary capable of mounting both mafia and distance frauds. Our protocol does not rely on computationally expensive primitives, has a very low memory requirement, and is noise-resilient. Therefore, it is efficient and suitable for extremely low resources devices. We provide analytical and experimental results that together show the superiority of our proposal w.r.t. to previous ones. 


\noindent\textbf{Organization.} Further below Section~\ref{section:review} presents a brief background about DB protocols. Section~\ref{sec:rationality} explains the rationality behind our proposal and Section~\ref{section:protocol} introduces and details the proposal. Sections~\ref{sec:mafia} and~\ref{sec:distance} are dedicated to the resistance of the protocol to mafia and distance frauds respectively. Section~\ref{section:noise} describes our noise resiliency mechanism. Section~\ref{section:result} provides comparative results with several DB protocols in both scenarios the free-noise case and the noisy case. Finally, Section~\ref{sec:conclusion} draws the conclusions.


\section{Background on distance-bounding}\label{section:review}

The first lightweight DB protocol was proposed by Hancke and Kuhn's~\cite{Hancke:2005:RDB:1128018.1128472} in 2005. Its simplicity and suitability for resource-constrained devices have promoted the design of other DB protocols based on it~\cite{AvoineT-2009-isc,KimA-2011-ieeetwc,Trujillo-Rasua:2010:PDP:1926325.1926352}. All these protocols share the same design: (a) there is a slow phase\footnote{In DB protocols, a \emph{fast phase}, which generally consists on  rounds, is a phase where the verifier computes RTTs. Otherwise, we say that it is a \emph{slow phase}.} where both prover and verifier generate and exchange nonces, (b) the nonces and a keyed cryptographic hash function are used to compute the answers to be sent (resp. checked) by the prover (resp. verifier). Below, we provide the main characteristics of each of these protocols, especially the technique they use to compute the answers. 

\noindent\textbf{Hancke and Kuhn's protocol~\cite{Hancke:2005:RDB:1128018.1128472}.} The answers are extracted from two -bit registers such that any of the  -bit challenges determines which register should be used to answer. 

\noindent\textbf{Avoine and Tchamkerten's protocol~\cite{AvoineT-2009-isc}.} 
Binary trees are used to compute the prover answers: the verifier challenges 
define the unique path in the tree, and the prover answers are the vertex value 
on this path. There are several parameters impacting the memory consumption and 
the resistance to distance and mafia frauds:  the number of trees and  
the depth of these trees. It holds , where  is the number of 
rounds in the fast phase. The larger , the better the frauds resistance and 
the larger the memory consumption. 

\noindent\textbf{Trujillo-Rasua, Martin and Avoine's protocol~\cite{Trujillo-Rasua:2010:PDP:1926325.1926352}.} This protocol is similar to the previous one, except that it uses particular graphs instead of trees to compute the prover answers.

\noindent\textbf{Kim and Avoine's protocol~\cite{KimA-2011-ieeetwc}.} This 
protocol, closer to the Hancke and Kuhn's 
protocol~\cite{Hancke:2005:RDB:1128018.1128472} than~\cite{AvoineT-2009-isc} 
and~\cite{Trujillo-Rasua:2010:PDP:1926325.1926352}, uses two registers to 
define the prover answers. An important additional feature is that the prover 
is able to detect a mafia fraud thanks to \emph{predefined challenges}, that 
is, challenges known by both prover and verifier. The number of predefined 
challenges impacts the frauds resistance: the larger, the better the mafia 
fraud resistance, but the lower the resistance to distance fraud.



There exist other DB protocols with different designs and computational 
complexities (\emph{e.g.,} protocols based on signatures and/or a final extra 
slow 
phase~\cite{Brands:1994:DP:188307.188361,Singelee:2007:DBN:1784404.1784415}). 
However, they are beyond the scope of this article that focuses on lightweight 
protocols only. The interested reader could refer to~\cite{AvoineBKLM-2011-jcs} 
for more details.


\section{Rationality of our proposal}\label{sec:rationality}

Being resistant to both mafia and distance frauds is the primary goal of a DB protocol. An important lower-bound for both frauds is , which is the probability of a naive adversary who answers randomly to the  verifier challenges during the fast phase. However, this resistance is hard to attain for lightweight DB protocols. Therefore, our aim is to design a protocol that is close to this bound for both mafia and distance frauds, without requiring costly operations or an extra final slow phase. We also aim to reach the additional property of being noise-resilient. Below, the intuitions that lead to our design are explained for each of the three considered properties.

\noindent\textbf{Mafia fraud.} Among the DB protocols without final slow phase, those achieving the best mafia fraud resistance are round-dependent~\cite{AvoineT-2009-isc,KimA-2011-ieeetwc,Trujillo-Rasua:2010:PDP:1926325.1926352}. The idea is that the correct answer at the th round should depend on the th challenge and also on the () previous challenges. Our proposal also uses a round-dependent technique, the proposed construction is significantly simpler than those proposed in~\cite{AvoineT-2009-isc,KimA-2011-ieeetwc,Trujillo-Rasua:2010:PDP:1926325.1926352}, though.

\noindent\textbf{Distance fraud.} As in mafia fraud, the best protocols in term of distance fraud are round-dependent. However, round-dependency by means of predefined challenges as in the Kim and Avoine's construction~\cite{KimA-2011-ieeetwc} fails to properly resist to distance fraud. Intuitively, as more control over the challenges the prover has, the lower the resistance to distance fraud is. For this reason, our proposal allows the verifier to have full and exclusive control over the challenges.

\noindent\textbf{Noise-resiliency.} Round-dependent protocols can hardly work in noisy environments. A noise in a given round might affect all the subsequent rounds and thus, these rounds becomes useless from the security point of view. Therefore, in order to deal with noise, round-dependent protocols should be able to detect the noisy-rounds so that the prover responses can be checked considering these noise occurrences. To the best of our knowledge, our protocol is the first round-dependent DB protocol able to detect such a noisy-rounds with a high level of accuracy thanks to the simplicity of its design.

\section{Proposal}\label{section:protocol}

This section describes the DB protocol introduced in the paper. Initialization, 
execution, and decision steps are presented below and a general view is 
provided in Figure~\ref{fig:2b}.

\subsubsection{Initialization.} The prover () and the verifier () agree 
on 
(a) a security parameter , (b) a timing bound , (c) 
a pseudo random function   that outputs  bits, (d) a secret key .

\subsubsection{Execution.} The protocol consists of a slow phase and a fast 
phase.

\subsubsection{Slow Phase.}  (respectively  ) randomly picks a nonce 
 (respectively ) and sends it to  (respectively  ). 
Afterwards,  and  compute  and divide the 
result into three -bit registers , , and . Both  and  
create the function  where  
is the set of all the bit-sequences of size at most  including the empty 
sequence. The function  is parameterized with the bit-sequence , and it outputs  when the input is the empty sequence. For 
every non-empty bit-sequence  where , the 
function is defined as .

\subsubsection{Fast Phase.} In each of the  rounds,  picks a random 
challenge , starts a timer, and sends  to . 
Upon reception of ,  replies with  
where . Once  receives , he stops the timer and 
computes the round-trip-time .

\subsubsection{Decision.} If  and    then the protocol succeeds\footnote{ 
is a system parameter that implicitly 
represents the maximum allowed distance between the prover and the verifier.}.
\begin{figure*}
\centering
 {\renewcommand\normalsize{\footnotesize}\normalsize

\begin{tabular}{|ccc|}
\hline
\normalsize{\textbf{Prover}}&&\normalsize{\textbf{Verifier}}\\
secret &&secret \\
&\textbf{\underline{slow phase}}&\\
&&\\
picks  &&picks \\
&&\\
&&\\


&&\\
&\textbf{\underline{fast phase}}&\\
&&\\
&\textbf{for :}&\\
&&picks \\
&& start timer\\
&&stop 
timer\\
 && computes \\
 &&\\\hline
 \end{tabular}}

\caption{Protocol description\label{fig:2b}}
\end{figure*}

\section{Resistance to mafia fraud}\label{sec:mafia}

Analyses of DB protocols usually consider two strategies to evaluate the resistance against a mafia fraud: the pre-ask and the post-ask strategies~\cite{AvoineBKLM-2011-jcs}. Although considering these two strategies only do not provide a formal security proof, this evaluates the resistance of the protocol, at least against these well-known attack strategies, and is the only way known so far to evaluate DB protocols. Providing a formal security proof of DB protocols would be interesting but is clearly out of the scope of this paper.

This section reminds the concept of pre-ask strategy\footnote{Note that the 
post-ask strategy is not relevant in protocols without an extra final slow 
phase~\cite{AvoineBKLM-2011-jcs}}, then identifies the adversarial behavior 
that maximizes the success probability when considering the pre-ask strategy, 
and the section finally computes this probability.

\subsection{Best behavior}

The \emph{pre-ask strategy} consists first, for the adversary, to relays the 
initial slow phase. Then, she runs the fast phase with the prover. With the 
answers she obtains, she finally executes the fast phase with the verifier. In 
our case, we consider that the adversary first sends a sequence of challenges 
 to the legitimate prover and receives 
 where  and  for every . Next, she executes the fast phase with the verifier receiving 
the challenges . Given , the adversarial 
behavior that maximizes the success probability is provided in 
Theorem~\ref{theo_best_strategy}.


\begin{theorem}\label{theo_best_strategy}
The adversary's behavior that maximizes her mafia fraud success probability 
with a pre-ask strategy is: (a) For every round  where , answer randomly. (b) For every round  where , guess the value  and answer 
with the value  where  and .
\end{theorem}

\begin{proof}

First, let us prove the following lemma.

\begin{lemma}\label{lemma_best_strategy}
Given that , the adversary's behavior maximizing her success 
probability at the th round is equivalent to the best behavior for guessing 
the value .
\end{lemma}

\begin{proof}
Given that , then , which 
means that . 
Therefore, either the adversary guesses  or 
the adversary losses this round.
\end{proof}

In the case where , the prover's response  
does not help the adversary since  and  are 
independent values. Therefore, there does not exist any best behavior, 
\emph{i.e.,} whatever the adversary behavior, her success probability at this 
round is . This result and Lemma~\ref{lemma_best_strategy} 
conclude the proof. \qed
\end{proof}

\subsection{Adversary's success probability}


Given the adversary's behavior provided by Theorem~\ref{theo_best_strategy}, Theorem~\ref{theo_mafia} provides a recursive way to compute her success probability.

\begin{theorem}\label{theo_mafia}
Let  be the event that the adversary has won the first  rounds by 
following her best behavior with the pre-ask strategy. Let  be the event 
that the adversary guesses  at the th 
round. The probability  can be recursively computed as follows:









\noindent \text{Where }  
\text{ and }  \text{ are 
the stopping} \text{ conditions}.
\end{theorem}

\begin{proof}

If , the adversary knows that  and thus, her success probability until the th round 
is , which means that . Considering 
that  and , then  can be expressed as 
follows:


\noindent Equation~\ref{reationality_eq_1} states that the computation of 
 requires . Note that  
holds if  holds, so:


\noindent Given that  depends on whether  
or not, and considering that , then  can be transformed as follows:


\noindent Assuming that  can be 
computed for every , then according to Equations~\ref{reationality_eq_1.5} 
and~\ref{reationality_eq_2},  can be 
recursively computed as follows:



\noindent Note that, the result  can be used as the stopping condition for the recursion defined in 
Equation~\ref{reationality_eq_3}. This recursion simplifies the analysis of 
: instead of analyzing the probability to win all the  rounds, 
only the probability to win the th round is needed. Since it depends on the 
adversary's behavior, and the latter depends on whether  or 
not, we compute   as follows:


\noindent When  the adversary answers randomly and thus 
. 
Considering this result and that  ,  Equation~\ref{reationality_eq_2.5} yields 
to:




\noindent From Equation~\ref{reationality_eq_4}, we deduce that computing 
 requires . Theorem~\ref{lemma_mafia_1} states that the adversary's behavior 
in this case is to guess . Hence:


\noindent We now aim at computing . Since , then  . Therefore, the 
adversary's strategy  maximizing  consists in holding her previous guess for the 
()th round. So:


\noindent As pointed out by Equation~\ref{eq_2_corollary_mafia_1} and 
Equation~\ref{eq_1_corollary_mafia_1}, computing  requires . Since it is indexed by , we assume that 
 is already computed for every  and as shown by 
Lemma~\ref{lemma_mafia_1},  can be recursively computed.

\begin{lemma}\label{lemma_mafia_1}
Given that  can be computed for every , then  can be recursively 
computed as follows:


where  is the 
stopping condition.
\end{lemma}

\begin{proof}

\noindent By definition of , and because , . Moreover, Theorem~\ref{theo_best_strategy} states that  and 
 are equivalent when , hence . Considering both results we 
obtain:







\noindent Finally, considering that  and , 
the probability  can be 
expressed as follows:

\noindent Equations~~\ref{eq_1_mafia_1} and~\ref{eq_2_mafia_1} yield the 
expected result. 

\end{proof}


\noindent Lemma~\ref{lemma_mafia_1} together with  
Equations~\ref{reationality_eq_1},~\ref{reationality_eq_3},~\ref{reationality_eq_4},~\ref{eq_1_corollary_mafia_1},
 and~\ref{eq_2_corollary_mafia_1}, conclude the proof of this theorem.  


\end{proof}




\section{Resistance to distance fraud} \label{sec:distance}

This section analyzes the adversary success probability when mounting a 
distance fraud. As stated in~\cite{AvoineBKLM-2011-jcs}, the common way to 
analyze the resistance to distance fraud is by considering the early-reply 
strategy instead of the pre-ask strategy. This strategy consists on sending the 
responses in advance, \emph{i.e.,} before receiving the challenges. Doing so, 
the adversary gains some time and might pass the timing constraint. In this 
section, the behavior that maximizes the success probability using the 
early-reply strategy is identified, and then a recursive way to compute the 
resistance w.r.t. a distance fraud is provided.

\subsection{Best behavior}

With the early-reply strategy, in order to send a response in advance in the 
th round with probability of being correct greater that , the 
adversary must send either  or  where  is the sequence of challenges sent 
by the verifier until the th round. Theorem~\ref{theo_strategy_distance} 
shows that, guessing the values  for every  is needed to maximize the adversary success probability.


\begin{theorem}\label{theo_strategy_distance}
Let  be the sequence of challenges  sent by the verifier 
until the th round (). The adversary's behavior that maximizes her 
distance 
fraud success probability is equivalent to the best behavior for guessing the 
values \\.
\end{theorem}

\begin{proof}

\noindent In order to send a response in advance at the th round with 
probability of being correct greater that , the adversary must send 
either  or . By definition,  and 
. Therefore,  and . Since the 
adversary knows the values , and , guessing the correct 
value at this round is equivalent to guessing the correct value of 
. 

\end{proof}

As stated in Theorem~\ref{theo_strategy_distance}, computing the adversary 
success probability requires the best behavior to guess the outputs sequence 
. Theorem~\ref{theo_distance_f} solves this 
problem.


\begin{theorem}\label{theo_distance_f}
The best adversary's behavior to guess  is to assume that her 
previous guess for  is correct and to compute  as 
follows: (a) if , then consider . (b) if 
, pick a random bit  and consider that .

\end{theorem}

\begin{proof}

\noindent Assuming that , then  and thus, 
the probability to guess  is equal to the probability of guessing 
. In the case of , the adversary does not have a better 
behavior than choosing a random bit of challenge  and considering that 
. Given that  where 
 is the empty sequence, the proof can be straightforwardly concluded 
by induction. 

\end{proof}

\subsection{Adversary's success probability}

Given the best adversary's behavior provided by Theorems~\ref{theo_strategy_distance} and~\ref{theo_distance_f}, Theorem~\ref{theo_distance} shows a recursive way to compute the resistance to distance fraud.

\begin{theorem}\label{theo_distance}
Let  be the event that the distance fraud adversary successfully passes 
the protocol until the th round. Then,  can be computed as follows:


where  is the stopping condition.
\end{theorem}


\begin{proof}

Let  be the event that the adversary correctly guesses the value of 
. Then, the event  depends on the events  and 
, which can be expressed as follows:


\noindent Two cases occur (a)  and (b) . In 
the first case, the adversary wins the th round if and only if she guesses 
the value , so  
and . In the second case, 
the adversary has no better probability to win than  and thus, 
. Therefore, we deduce  and . 
Using these results and  Equation~\ref{theo_distance_eq_1} we have:


\noindent Equation~\ref{theo_distance_eq_2} states that  can be 
computed by recursion if we express  in terms of the 
events  where . Therefore, in the remaining of this proof we 
aim at looking for such result. As above, in order to analyze , the events  and  should be considered:



\noindent Four cases should be analyzed depending on the value of  and the 
events  and . 

\begin{case}[,  and  hold]\label{case1}
This case occurs if the adversary correctly guesses the challenge . 
Therefore, she provides the correct answer at this round . So, .
\end{case}

\begin{case}[,  and  hold]\label{case2}
Given that  and  hold, the adversary computed  using a challenge different from the 
verifier's one, \emph{i.e.,} . Therefore,  because both events 
 and  coexist only if , then .
\end{case}


\begin{case}[,  and  hold]\label{case3}
Given , the event  has no effect on the event . Thus, 
 because it depends on whether . So, 
.
\end{case}

\begin{case}[,  and  hold]\label{case4}
When , then , which means that this case 
cannot occur. Therefore, .
\end{case}



Cases~\ref{case1} and~\ref{case3} yield  the following result: 



And Cases~\ref{case2} and~\ref{case4} yield this other result: 




\noindent Because  and , we have . Similarly,  because  and . Combining these 
results with Equations~\ref{theo_distance_eq_4} and~\ref{theo_distance_eq_5}, 
Equation~\ref{theo_distance_eq_3} becomes:



\noindent Considering that , Equations~\ref{theo_distance_eq_6} 
and~\ref{theo_distance_eq_2} yield the expected result. 

\end{proof}

\section{Noise resilience}\label{section:noise}

Some efforts have been made in order to adapt existing distance-bounding 
protocols to noisy channels. Most of them rely on using a threshold  
representing the maximum number of incorrect responses expected by the 
verifier~\cite{Hancke:2005:RDB:1128018.1128472,KimA-2011-ieeetwc}.
 Intuitively, the larger , the lower the false rejection ratio but also the 
lower the resistance to mafia and distance frauds. Others use an error 
correction code during an extra slow 
phase~\cite{Singelee:2007:DBN:1784404.1784415}. However, the latter cannot be 
applied to our protocol given that it does not contain any final slow phase. 
Consequently, 
we focus on the threshold technique. 

\subsection{Understanding the noise effect in our protocol}

We consider in the analysis that a 1-bit challenge (on the \emph{forward} 
channel) can be flipped due to noise with probability  and a 1-bit answer 
(on the \emph{backward} channel) can be flipped with probability .  
Further, we denote as  the bit-challenge received by the prover at 
the th round, which might be obviously different to the challenge . 
Similarly,  denotes the response received by the verifier at the 
th round. As in previous works~\cite{Hancke:2005:RDB:1128018.1128472}, the 
considered forward and backward channels are assumed to be memoryless. 
Table~\ref{table_noise} shows the three different scenarios when considering a 
noisy communication at the -th round in our protocol. 


\begin{table*}
\centering
 {\renewcommand\normalsize{\footnotesize}\normalsize
\begin{tabular}{|l|l|l|l|} \hline
\textbf{Forward Noise} & \textbf{Backward Noise} & \textbf{Forward and}\\
&&\textbf{Backward Noise} \\
\hline
 receives  &  receives   & 
 receives  \\
 updates & updates &  updates \\
 sends & sends 
&  sends \\
 receives & receives &  
receives \\\hline
\end{tabular}}
\vspace{10pt}
\caption{The three possible scenarios when some noise occurs at the th 
round.}
\label{table_noise}
\end{table*}


According to the protocol, in a noise-free th round 
executed with a legitimate prover it holds that . We thus say that prover and 
verifier are \emph{synchronized} at the th round if , otherwise they are said to be \emph{desynchronized}. 
Intuitively, in a noise-free th round the answer  can be 
considered correct by the verifier either if  and they are 
synchronized or if  and they are desynchronized. 

The challenge is therefore to identify whether the prover and the verifier are 
synchronized or not. To that aim, we rise the following observation.


\begin{observation}\label{obs_sync}
Several consecutive rounds where all, or almost all, the answers hold that  (resp. ), might indicate that the 
legitimate prover and the verifier have been synchronized (resp. 
desynchronized). 
\end{observation}



\subsection{Our noise resilient mechanism}\label{sec:noise}

Based on Observation~\ref{obs_sync}, we propose an heuristic aimed at 
identifying those rounds where prover and verifier \emph{switch} from being 
synchronized to desynchronized or vice versa. The heuristic is named 
\emph{SwitchedRounds} and its pseudocode description is provided by 
Algorithm~\ref{alg_des}.

\emph{SwitchedRounds} creates first the sequence  where  
if , otherwise . Following 
Observation~\ref{obs_sync}, it searches for the longest subsequence 
 that matches any of the following patterns\footnote{The patterns 
have been written 
following the POSIX Extended Regular Expressions standard. The symbols  
and \wedge(1+)0\wedge(1+)\ (f) 01d_1 \dots d_n\wedge(0+)1\wedge(0+)\ on .
\If{ or no matching exists}
	\Return the empty set;
\EndIf
\State Let  be  if the matching is with 1ri+1q_r = 
1Ad_1...d_{i-1}Bd_{j+1}...d_{n}EA \cup \{(r, s)\} \cup BExd_1...d_nx\Delta lEc_1...c_n\tilde{r}_1...\tilde{r}_nR^0R^1Q\Delta ld_1...d_nd_i = \tilde{r}_i \oplus 
R_i^{c_i}\oplus f(C_i)s0errors01 \leq i \leq n(i, s') \in Es \leftarrow s'errors++d_i = 0s = 1d_i = 1s = 0errors++errors > x64xy(x,y)(x,y)n2^{n+1}-1nn/33n5n1\left(\frac{1}{2}\right)^{64}xs\Delta lpz = (x, s, p)\Pinp_fp_b\Pi_{p_f, p_b, n}^{security}(z)\Pi_{p_f, 
p_b, n}^{availability}(z)z 
= (x, s, p)\Pi\Pi\Delta\Pi \in 
\{\text{HK, 
KA2, Our}\}\Pi_{p_f, p_b, n}^{security}(z)\Pi_{p_f, p_b, n}^{availability}(z)\Pi \in \{\text{HK, KA2, Our}\}\Pi = 
\text{HK}\text{HK}_{p_f, p_b, n}^{security}(z)\text{HK}_{p_f, p_b, 
n}^{availability}(z)\Pi = \text{KA2}\text{KA2}_{p_f, p_b, n}^{security}(z)\text{KA2}_{p_f, p_b, 
n}^{availability}(z)\text{Our}_{p_f, p_b, n}^{security}(z)\text{Our}_{p_f, 
p_b, 
n}^{availability}(z)\Pi10^6485\%402080646448p_f = p_b \leq 0.05p_f + p_b = 0.05p_fp_b0.1p_f = p_bp_f = p_b = 0p_f + p_b = 0.05p_fp_fp_f = p_bp_f + p_b = 0.05485\%p_fp_bp_f = p_b \in 
  \{0, 0.005, ..., 0.045, 0.05\}p_f + p_b = 
  0.05p_f \in \{0, 0.005, ..., 0.045, 0.05\}xHK_{p_f, p_b, n}^{availability}(x)WHK_{p_f, p_b, n}^{availability}(x)W\Pr(W) = 
\frac{1}{2}p_f(1-p_b) + p_fp_b$

Equations~\ref{theo_1_eq1} and~\ref{theo_1_eq2} yield the expected result.

\end{proof}

\bibliographystyle{abbrv} \begin{thebibliography}{10}

\bibitem{AvoineBKLM-2011-jcs}
G.~Avoine, M.~A. Bing\"ol, S.~Karda\c{s}, C.~Lauradoux, and B.~Martin.
\newblock {A Framework for Analyzing {RFID} Distance Bounding Protocols}.
\newblock {\em Journal of Computer Security}, 19(2):289--317, March 2011.

\bibitem{AvoineT-2009-isc}
G.~Avoine and A.~Tchamkerten.
\newblock {An efficient distance bounding {RFID} authentication protocol:
  balancing false-acceptance rate and memory requirement}.
\newblock In {\em Information Security Conference -- ISC'09}, volume 5735, 
pp. 250--261, 2009.

\bibitem{BengioBDGQ-1991-crypto}
S.~Bengio, G.~Brassard, Y.~Desmedt, C.~Goutier, and J.-J. Quisquater.
\newblock Secure implementation of identification systems.
\newblock {\em Journal of Cryptology}, 4(3):175--183, 1991.

\bibitem{Brands:1994:DP:188307.188361}
S.~Brands and D.~Chaum.
\newblock Distance-bounding protocols.
\newblock In {\em Workshop on the theory and app.lication of cryptographic
  techniques on Advances in cryptology}, EUROCRYPT '93, pp. 344--359.

\bibitem{citeulike:1223195}
J.~H. Conway.
\newblock {\em {On Numbers and Games}}.
\newblock AK Peters, Ltd., 2nd edition, Dec. 2000.

\bibitem{Desmedt:1987:SUS:646752.704723}
Y.~Desmedt, C.~Goutier, and S.~Bengio.
\newblock Special uses and abuses of the {F}iat-{S}hamir passport protocol.
\newblock In {\em Theory and App.lications of Cryptographic
  Techniques on Advances in Cryptology}, CRYPTO '87, pp. 21--39.

\bibitem{DrimerM-2007-usenix}
S.~Drimer and S.~J. Murdoch.
\newblock Keep your enemies close: distance bounding against smartcard relay
  attacks.
\newblock In {\em USENIX Security Symposium on USENIX Security Symposium},
  pp. 1--16, 2007.

\bibitem{FiatS-1986-crypto}
A.~Fiat and A.~Shamir.
\newblock How to prove yourself: Practical solutions to identification and
  signature problems.
\newblock In {\em Advances in Cryptology -- CRYPTO'86}, volume 263, pp.
186--194.

  	\bibitem{T2013}
  	R.~Trujillo-Rasua, Complexity of distance fraud attacks in graph-based 
  	distance  bounding, 10th International Conference on Mobile and Ubiquitous 
  	Systems: Computing, Networking and Services, MobiQuitous'13.


\bibitem{Francis:2010:PNP:1926325.1926331}
L.~Francis, G.~Hancke, K.~Mayes, and K.~Markantonakis.
\newblock Practical {NFC} peer-to-peer relay attack using mobile phones.
\newblock In {\em Radio
  frequency identification: security and privacy issues}, RFIDSec'10, pp.
  35--49.

\bibitem{Hancke:2005:RDB:1128018.1128472}
G.~P. Hancke and M.~G. Kuhn.
\newblock An {RFID} distance bounding protocol.
\newblock In {\em Security
  and Privacy for Emerging Areas in Communications Networks}, pp. 67--73, 2005.

\bibitem{Hancke:2008:ATD:1352533.1352566}
G.~P. Hancke and M.~G. Kuhn.
\newblock Attacks on time-of-flight distance bounding channels.
\newblock In {\em Proceedings of the first ACM conference on Wireless network
  security}, WiSec '08, pp. 194--202.

\bibitem{KimA-2011-ieeetwc}
C.~H. Kim and G.~Avoine.
\newblock {{RFID} Distance Bounding Protocols with Mixed Challenges}.
\newblock {\em IEEE Transactions on Wireless Communications}, 10(5):1618--1626,
  2011.

\bibitem{OrenW-2009-eprint}
Y.~Oren and A.~Wool.
\newblock Relay attacks on {RFID}-based electronic voting systems.
\newblock Cryptology ePrint Archive, Report 2009/422, 2009.

\bibitem{Singelee:2007:DBN:1784404.1784415}
D.~Singel{\'e}e and B.~Preneel.
\newblock Distance bounding in noisy environments.
\newblock In {\em Security and
  privacy in ad-hoc and sensor networks}, ESAS'07, pp. 101--115.

\bibitem{Trujillo-Rasua:2010:PDP:1926325.1926352}
R.~Trujillo-Rasua, B.~Martin, and G.~Avoine.
\newblock The poulidor distance-bounding protocol.
\newblock In {\em Radio
  frequency identification: security and privacy issues}, RFIDSec'10, pages
  239--257. 

\bibitem{Singelee:2007:DBN:1784404.1784415}
	D.~Singel{\'e}e, B.~Preneel, Distance bounding in noisy environments, in:
	Proceedings of the 4th European conference on Security and privacy in ad-hoc
	and sensor networks, ESAS'07, pp. 101--115.
	
\bibitem{TMA2014}
R.~Trujillo-Rasua, B.~Martin and G.~Avoine.
\newblock {Distance-bounding facing both mafia and distance frauds}.
\newblock {\em IEEE Transactions on Wireless Communications}, 2014.

\end{thebibliography}




\end{document} 
