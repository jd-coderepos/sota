\documentclass[12pt]{article}

\usepackage{graphics}
\usepackage[dvips]{graphicx}
\usepackage[left]{lineno}


\usepackage[dvipsnames]{xcolor}
\usepackage{ulem}


\definecolor{Boxbackground}{rgb}{0.91, 0.839, 1}       

\newif\ifTrackChange

 


\usepackage{amsmath} 
\usepackage{setspace} 
\usepackage{hyphenat}
\usepackage{geometry}
\geometry{papersize={210mm,297mm}, layoutsize={210mm,297mm} }
\usepackage{enumitem}

\usepackage{hyperref}
\hypersetup{
  colorlinks   = true,  urlcolor     = blue,  linkcolor    = Brown, citecolor    = Brown  }
\usepackage[framemethod=TikZ]{mdframed}

\title{Architecture of Environmental Risk Modelling: for a faster and more robust response to natural disasters}

\usepackage{ifthen}
\ifthenelse{1=1}{
\usepackage{fancyhdr}
\pagestyle{fancy}
\fancyhf{} 
\renewcommand{\headrulewidth}{0pt}
\fancyfoot[L]{\ifthenelse{\value{page}=1}{{\footnotesize \vspace{2mm}Cite as:\\
Rodriguez-Aseretto, D., Schaerer, C., de Rigo, D., 2014. \textbf{Architecture of Environmental Risk Modelling: for a faster and more robust response to natural disasters}. \textit{~Conference of Computational Interdisciplinary Sciences}, Asunci\'on, Paraguay
}}{}}
\fancyfoot[C]{\ifthenelse{\value{page}=1}{}{\thepage}}
\fancyfoot[R]{\ifthenelse{\value{page}=1}{\thepage}{}}
\fancyhead[R]{}
\fancyhead[L]{\ifthenelse{\value{page}=1}{}{\footnotesize Rodriguez-Aseretto, D., Schaerer, C., de Rigo, D., 2014. \textit{Architecture of Environmental Risk Modelling: for a faster and more robust response to natural disasters}. CCIS 2014}}
}


\begin{document}


\begin{center}


{\large{\bf Architecture of Environmental Risk Modelling: for a faster\
\label{eq:multicriteria_policy}
u^t(\cdot) = {\operatorname{arg\,min}}_{u\, \in \,{ U }^{ \, u }_{ t\, ,\, t_{ { end } } }} \left[ \mathcal{C}^{\,1,\,t} \mathcal{C}^{\,2,\,t} \cdots \mathcal{C}^{\,k,\,t} \cdots \mathcal{C}^{\,n,\,t} \right]


\noindent where {the -th} cost  is linked to the corresponding impact assessment criterion. {This POLFC schema within the SemAP paradigm may be considered a semantically-enhanced dynamic data-driven application system (DDDAS) \cite{de_Rigo_etal_IFIP2013,DiLeo_etal_2013,RodriguezAseretto_etal_2013}.} Finally, the Emergency Manager may communicate the updated scenarios of the emergency evolution (by means of geospatial maps and other executive summary information) in order for decision-makers and stakeholders to be able to assess the updated multi-criteria pattern of costs and the preferred control options. 
This critical communication constitutes the science-policy interface and must be as supportive as possible. It is designed to exploit web map services (WMS) \cite{McInerney2012,Bastin2012} (on top of the underpinning free software for WSTMe, e.g. \cite{RodriguezAseretto_EFDAC2013}) which may be accessed in a normal browser or with specific Apps for smart-phones~\cite{Aseretto_submitted}. 


\medskip
\subsubsection*{3. CONCLU{DING} REMARKS}
\smallskip

\noindent NSF Cyberinfrastructure Council report reads: \textit{While hardware performance has been growing exponentially - with gate density doubling every 18 months, storage capacity every 12 months, and network capability every 9 months - it has become clear that increasingly capable hardware is not the only requirement for computation-enabled discovery. Sophisticated software, visualization tools, middleware and scientific applications created and used by interdisciplinary teams are critical to turning flops, bytes and bits into scientific breakthroughs} \cite{NSFC2007}. Transdisciplinary environmental problems such as the ones dealing with complexity and deep-uncertainty in supporting natural-hazard emergency might appear as seemingly intractable \cite{Altay_Green_2006}. Nevertheless, approximate rapid-assessment based on computationally intensive modelling may offer a new perspective at least able to support emergency operations and decision-making with qualitative or semi-quantitative scenarios. Even a partial approximate but timely investigation on the potential interactions of the many sources of uncertainty might help emergency managers and decision-makers to base control strategies on the best available -- although typically incomplete -- sound scientific information. In this context, a key aspect of soundness relies on explicitly considering the multiple dimensions of the problem and the array of uncertainties involved. As no silver bullet seems to be available for reliably attacking this amount of uncertainty and complexity, an integration of methods is proposed, inspired by their promising synergy. Array programming is perfectly suited for easily managing a multiplicity of arrays of hazard models, dynamic input information, static parametrisation and the distribute array of social contributions (Citizen Sensor). The transdisciplinary nature of complex natural hazards -- their need for an unpredictably broad and multifaceted readiness to robust scalability -- may benefit (1) from a disciplined \textit{abstract modularisation} of the data-transformations which compose the models (D-TM), and (2) from a \textit{semantically-enhanced} design of the D-TM structure and interactions. These two aspects define the Semantic Array Programming (SemAP, \cite{deRigo2012,deRigo2012b}) paradigm whose application -- extended to geospatial aspects \cite{GeoSemAP_2013} -- is proposed to consider also the array of uncertainties (data, modelling, geoparsing, software uncertainty) and the array of criteria to assess the potential impacts associated with the hazard scenarios. The unevenly available information during an emergency event may be efficiently exploited by means of a partial open loop feedback control (POLFC, \cite{Castelletti2008}) schema, already successfully tested in this integrated approach \cite{de_Rigo_etal_IFIP2013,DiLeo_etal_2013,RodriguezAseretto_etal_2013} as a promising evolution of adaptive data-driven strategies \cite{RodriguezAseretto_etal_2008}. Its demanding computations may become affordable during an emergency event with an appropriate array-based parallelisation strategy within Urgent-HPC. 


\begin{spacing}{0.85}

\begin{nohyphens}

\begin{thebibliography}{43}

\footnotesize\setlength{\itemsep}{-0.5mm}
\newcommand{\TitleURL}[2]{\href{#1}{#2}}

\bibitem{Sippel_Otto_2014}
SIPPEL, S., OTTO, F.E.L., 2014. \TitleURL{http://dx.doi.org/10.1007/s10584-014-1153-9}{Beyond climatological extremes - assessing how the odds of hydrometeorological extreme events in South-East Europe change in a warming climate}. Climatic Change: 1-18.


\bibitem{Cirella_etal_2014}
CIRELLA, G.T., et al, 2014. \TitleURL{http://dx.doi.org/10.1007/978-94-007-7161-1\_16}{Natural hazard risk assessment and management methodologies review: Europe}. In: Linkov, I. (Ed.), Sustainable Cities and Military Installations. NATO Science for Peace and Security Series C: Environmental Security. Springer Netherlands, pp. 329-358.


\bibitem{Ciscar_etal_2013}
CISCAR, J. C., et al, 2013. \TitleURL{http://www.climate-impacts-2013.org/files/cwi\_ciscar.pdf}{Climate impacts in Europe: an integrated economic assessment}. In: Impacts World 2013 - International Conference on Climate Change Effects. Potsdam Institute for Climate Impact Research (PIK) e. V., pp. 87-96. 

\bibitem{Ciscar_etal_2014}
CISCAR, J. C., et al, 2014. \TitleURL{http://dx.doi.org/10.2791/7409}{Climate Impacts in Europe - The JRC PESETA II project}. Vol. 26586 of EUR - Scientific and Technical Research. Publ. Off. Eur. Union, 155 pp. 


\bibitem{Dankers_Feyen_2008}
DANKERS, R., FEYEN, L., 2008. \TitleURL{http://dx.doi.org/10.1029/2007jd009719}{Climate change impact on flood hazard in Europe:~An assessment based on high-resolution climate simulations}. J.~\!Geophys.~\!Res.~\!113(D19):~\!D19105+.


\bibitem{Gaume_etal_2009}
GAUME, E., et al, 2009. \TitleURL{http://dx.doi.org/10.1016/j.jhydrol.2008.12.028}{A compilation of data on European flash floods}. Journal of Hydrology 367(1-2): 70-78.


\bibitem{Marchi_etal_2010}
MARCHI, L., et al, 2010. \TitleURL{http://dx.doi.org/10.1016/j.jhydrol.2010.07.017}{Characterisation of selected extreme flash floods in Europe and implications for flood risk management}. Journal of Hydrology 394(1-2): 118-133.


\bibitem{Feser_etal_2014}
FESER, F., et al, 2014. \TitleURL{http://dx.doi.org/10.1002/qj.2364}{Storminess over the North Atlantic and Northwestern Europe - a review}. Quarterly Journal of the Royal Meteorological Society.


\bibitem{Jongman_etal_2014}
JONGMAN, B., et al, 2014. \TitleURL{http://dx.doi.org/10.1038/nclimate2124}{Increasing stress on disaster-risk finance due to large floods}. Nature Climate Change 4 (4): 264-268.


\bibitem{Self_2006}
SELF, S., 2006. \TitleURL{http://dx.doi.org/10.1098/rsta.2006.1814}{The effects and consequences of very large explosive volcanic eruptions}. Philosophical Transactions of the Royal Society A: Mathematical, Physical and Engineering Sciences 364(1845): 2073-2097.


\bibitem{Swindles_2011}
SWINDLES, G. T., et al,  2011. \TitleURL{http://dx.doi.org/10.1130/g32146.1}{A 7000 yr perspective on volcanic ash clouds affecting northern Europe}. Geology 39(9): 887-890.


\bibitem{Gramling_2014}
GRAMLING, C., 2014. \TitleURL{http://dx.doi.org/10.1126/science.345.6200.990}{As volcano rumbles, scientists plan for aviation alerts}. Science 345(6200): 990.


\bibitem{Allard_etal_2013}
ALLARD, G., et al, 2013. \TitleURL{http://www.fao.org/docrep/017/i3226e/i3226e.pdf}{State of Mediterranean forests 2013}. FAO, 177 pp. 


\bibitem{Schmuck_etal_2014}
SCHMUCK, G., et al, 2014. \TitleURL{http://dx.doi.org/10.2788/99870}{Forest Fires in Europe, Middle East and North Africa 2013}. Publications Office of the European Union. 107 pp.


\bibitem{Nijhuis_2012}
NIJHUIS, M., 2012. \TitleURL{http://dx.doi.org/10.1038/489352a}{Forest fires: Burn out}. Nature 489 (7416), 352-354.


\bibitem{Boyd_etal_2013}
BOYD, I.L., et al, 2013. \TitleURL{http://dx.doi.org/10.1126/science.1235773}{The consequence of tree pests and diseases for ecosystem services}. Science 342 (6160): 1235773+.


\bibitem{Venette_etal_2012}
VENETTE, R.C., et al, 2012.
\TitleURL{http://www.webcitation.org/6BOcoB2eZ}{Summary of the international pest risk mapping workgroup meeting sponsored by the cooperative research program on biological resource management for sustainable agricultural systems}. In: 6th International Pest Risk Mapping Workgroup Meeting: "Advancing risk assessment models for invasive alien species in the food chain: contending with climate change, economics and uncertainty". Organisation for Economic Co-operation and Development (OECD), pp. 1-2. 

\bibitem{Maes_etal_2013}
MAES, J., et al, 2013. \TitleURL{http://dx.doi.org/10.2779/12398}{Mapping and Assessment of Ecosystems and their Services - An analytical framework for ecosystem assessments under action 5 of the EU biodiversity strategy to 2020}. Publications office of the European Union, 57 pp.


\bibitem{EC_2013}
EUROPEAN COMMISSION, 2013. \TitleURL{http://eur-lex.europa.eu/LexUriServ/LexUriServ.do?uri=COM:2013:0659:FIN:EN:PDF}{Communication from the commission to the European Parliament, the Council, the European Economic and Social Committee and the Committee of the Regions - A new EU forest strategy: for forests and the forest-based sector}. No. COM(2013) 659 final. Communication from the Commission to the Council and the European Parliament.


\bibitem{EC_2013b}
EUROPEAN COMMISSION, 2013. \TitleURL{http://eur-lex.europa.eu/LexUriServ/LexUriServ.do?uri=SWD:2013:0342:FIN:EN:PDF}{Commission staff working document accompanying the document: Communication from the commission to the European Parliament, the Council, the European Economic and Social Committee and the Committee of the Regions - a new EU forest strategy: for forests and the forest-based sector}. Commission Staff Working Document 2013 (SWD/2013/0342 final), 98pp.


\bibitem{Barredo_2007}
BARREDO, J. I., 2007. \TitleURL{http://dx.doi.org/10.1007/s11069-006-9065-2}{Major flood disasters in Europe: 1950-2005}. Natural Hazards 42(1): 125-148.


\bibitem{Barredo_2010}
BARREDO, J. I., 2010. \TitleURL{http://dx.doi.org/10.5194/nhess-10-97-2010}{No upward trend in normalised windstorm losses in Europe: 1970-2008}. Natural Hazards and Earth System Science 10(1): 97-104.


\bibitem{Evans_etal_2012}
EVANS, M. R., et al, 2012. \TitleURL{http://dx.doi.org/10.1098/rstb.2011.0191}{Predictive ecology: systems approaches}. Philosophical transactions of the Royal Society of London. Series B, Biological sciences 367(1586): 163-169.


\bibitem{Phillis_Kouikoglou}
PHILLIS, Y. A., KOUIKOGLOU, V. S.,  2012. \TitleURL{http://dx.doi.org/10.1016/j.ecolmodel.2012.03.032}{System-of-Systems hierarchy of biodiversity conservation problems}. Ecological Modelling 235-236: 36-48.


\bibitem{Langmann_2014}
LANGMANN, B., 2014. \TitleURL{http://dx.doi.org/10.1155/2014/340123}{On the role of climate forcing by volcanic sulphate and volcanic ash}. Advances in Meteorology 2014: 1-17.


\bibitem{Gottret_2001}
GOTTRET, M. V., WHITE, D., 2001. \TitleURL{http://www.ecologyandsociety.org/vol5/iss2/art17/}{Assessing the impact of integrated natural resource management: Challenges and experiences}. Ecology and Society 5 (2): 17+


\bibitem{Hagmann_2001}
HAGMANN, J., et al, 2001. \TitleURL{http://www.ecologyandsociety.org/vol5/iss2/art29/}{Success factors in integrated natural resource management R\&D: Lessons from practice}. Ecology and Society 5 (2), 29+.


\bibitem{Zhang_2004}
ZHANG, X., et al, 2004. \TitleURL{http://www.citeulike.org/group/15400/article/11477904}{Scaling issues in environmental modelling}. In: Wainwright, J., Mulligan, M. (Eds.), Environmental modelling : finding simplicity in complexity. Wiley. 

\bibitem{Estreguil_etal_2013}
ESTREGUIL, C., et al, 2013. \TitleURL{http://dx.doi.org/10.2788/77842}{Forest landscape in Europe: Pattern, fragmentation and connectivity}. EUR - Scientific and Technical Research 25717 (JRC 77295), 18 pp.


\bibitem{Turner2010}
TURNER, R., 2010, \TitleURL{http://dx.doi.org/10.1890/10-0097.1}{Disturbance and landscape dynamics in a changing world}. Ecology, 91 10: 2833-2849. 


\bibitem{Van_Westen_2013}
Van WESTEN, C.J., 2013. \TitleURL{http://dx.doi.org/10.1016/b978-0-12-374739-6.00051-8}{Remote sensing and GIS for natural hazards assessment and disaster risk management}. In: Bishop, M. P. (Ed.), Remote Sensing and GIScience in Geomorphology. Vol. 3 of Treatise on Geomorphology. Elsevier, pp. 259-298.


\bibitem{Urban_etal_2012}
URBAN, M.C., et al, 2012. \TitleURL{http://dx.doi.org/10.1111/j.1752-4571.2011.00208.x}{A crucial step toward realism: responses to climate change from an evolving metacommunity perspective}. Evolutionary Applications 5 (2): 154-167.


\bibitem{Baklanov_2007}
BAKLANOV, A., 2007. \TitleURL{http://dx.doi.org/10.1007/978-1-4020-5877-6\_3}{Environmental risk and assessment modelling - scientific needs and expected advancements}. In: Ebel, A., Davitashvili, T. (Eds.), Air, Water and Soil Quality Modelling for Risk and Impact Assessment. NATO Security Through Science Series. Springer Netherlands, pp. 29-44. 


\bibitem{Steffen_2011}
STEFFEN, W., et al, 2011. \TitleURL{http://dx.doi.org/10.1007/s13280-011-0185-x}{The anthropocene: From global change to planetary stewardship}. AMBIO 40 (7): 739-761. 


\bibitem{White_etal_2012}
WHITE, C., et al, 2012. \TitleURL{http://dx.doi.org/10.1111/j.1461-0248.2012.01773.x}{The value of coordinated management of interacting ecosystem services}. Ecology Letters 15 (6): 509-519.


\bibitem{deRigo2013}
de RIGO, D., 2013, \TitleURL{http://arxiv.org/abs/1311.4762}{Software Uncertainty in Integrated Environmental Modelling: the role of Semantics and Open Science}. Geophys. Res. Abstr. 15: 13292+.


\bibitem{deRigoSubm}
de RIGO, D., (exp.) 2014. Behind the horizon of reproducible integrated environmental modelling at European scale: ethics and practice of scientific knowledge freedom. F1000 Research, submitted. 

\bibitem{Lempert_2002}
LEMPERT, R. J., May 2002. \TitleURL{http://dx.doi.org/10.1073/pnas.082081699}{A new decision sciences for complex systems}. Proceedings of the National Academy of Sciences 99 (suppl 3): 7309-7313.


\bibitem{Rammel_2007}
RAMMEL, C., et al, 2007. \TitleURL{http://dx.doi.org/10.1016/j.ecolecon.2006.12.014}{Managing complex adaptive systems - a co-evolutionary perspective on natural resource management}. Ecological Economics 63 (1): 9-21.


\bibitem{van_der_Sluijs_2012}
van der SLUIJS 2012, J. P., 2012. \TitleURL{http://dx.doi.org/10.3167/nc.2012.070204}{Uncertainty and dissent in climate risk assessment: A Post-Normal perspective}. Nature and Culture 7 (2): 174-195.


\bibitem{de_Rigo_etal_IFIP2013}
de RIGO, D., et al, 2013. \TitleURL{http://dx.doi.org/10.1007/978-3-642-41151-9\_35}{An architecture for adaptive robust modelling of wildfire behaviour under deep uncertainty}. IFIP Adv. Inf. Commun. Technol. 413: 367-380.


\bibitem{Guariso}
GUARISO, G., et al, 2000. A map-based web server for the collection and distribution of environmental data. In: Kosmatin Fras, M., Mussio, L., Crosilla, F., Podobnikar, T. (Eds.), Bridging the gap: ISPRS WG VI/3 and IV/3 Workshop, Ljubljana, February 2-5, 2000 : Collection of Abstracts. Ljubljana. 

\bibitem{GuarisoEDSS}
GUARISO, G., WERTHNER, H., 1989. Environmental decision support systems. E. Horwood; Halsted Press. 

\bibitem{Guariso_1985}
GUARISO, G., et al, 1985. \TitleURL{http://dx.doi.org/10.1016/0377-2217(85)90150-X}{Decision support systems for water management: The lake Como case study}. European Journal of Operational Research 21(3): 295-306.


\bibitem{Soncini_Sessa_2007}
SONCINI-SESSA, R., et al, 2007.
Integrated and participatory water resources \mbox{management} theory. Elsevier. 


\bibitem{Guariso_1994}
GUARISO, G., PAGE, B. (Eds), 1994. Computer support for environmental impact \mbox{assessment}: proceedings of the IFIP TC5/WG5.11 Working Conference on Computer \mbox{Support} for Environmental Impact Assessment, CSEIA 93, Como, Italy, 6-8 October, 1993. North-Holland. 

\bibitem{Casagrandi}
CASAGRANDI, R., GUARISO, G., 2009. \TitleURL{http://dx.doi.org/10.1016/j.envsoft.2008.11.013}{Impact of ICT in environmental sciences: A citation analysis 1990-2007}. Environmental Modelling \& Software 24 (7): 865-871.


\bibitem{Cole_2010}
COLE, J., 2010. \TitleURL{http://www.rusi.org/publications/whitehallreports/ref:O4C2CC38D725EE/}{Interoperability in a crisis 2: Human factors and organisational processes}. Tech. rep., Royal United Services Institute.


\bibitem{Sterman_2002}
STERMAN, J.D., 2002. \TitleURL{http://dx.doi.org/10.1002/sdr.261}{All models are wrong: reflections on becoming a systems scientist}. System Dynamics Review 18 (4): 501-531.


\bibitem{Weichselgartner_2010}
WEICHSELGARTNER, J., KASPERSON, R., 2010. \TitleURL{http://dx.doi.org/10.1016/j.gloenvcha.2009.11.006}{Barriers in the science-policy-practice interface: Toward a knowledge-action-system in global environmental change research}. Global Environmental Change 20 (2): 266-277.


\bibitem{Bainbridge_1983}
BAINBRIDGE, L., 1983. \TitleURL{http://dx.doi.org/10.1016/0005-1098(83)90046-8}{Ironies of automation}. Automatica 19(6): 775-779.


\bibitem{GeoSemAP_2013}
de RIGO, D., et al, 2013.
\TitleURL{http://dx.doi.org/10.6084/m9.figshare.155703}{Toward open science at the European scale: \mbox{Geospatial} \mbox{Semantic} Array Programming for integrated environmental modelling}. Geophysical \mbox{Research} \mbox{Abstracts} 15: 13245+. 


\bibitem{Stensson_2013}
STENSSON, P., JANSSON, A., 2013. \TitleURL{http://dx.doi.org/10.1080/00140139.2013.858777}{Autonomous technology - sources of confusion: a model for explanation and prediction of conceptual shifts}. Ergonomics 57 (3): 455-470. 


\bibitem{Russell_2003}
RUSSELL, D. M., et al, 2003. \TitleURL{http://dx.doi.org/10.1147/sj.421.0177}{Dealing with ghosts: Managing the user experience of autonomic computing}. IBM Systems Journal 42 (1): 177-188.


\bibitem{Anderson_2003}
ANDERSON, S., et al, 2003. \TitleURL{http://dx.doi.org/10.1109/dexa.2003.1232106}{Making autonomic computing systems accountable: the problem of human computer interaction}. In: Database and Expert Systems Applications, 2003. Proceedings. 14th International Workshop on. IEEE, pp. 718-724. 


\bibitem{Kephart_2003}
KEPHART, J. O., CHESS, D. M. 2003. \TitleURL{http://dx.doi.org/10.1109/mc.2003.1160055}{The vision of autonomic computing}. Computer 36 (1): 41-50.


\bibitem{van_der_Sluijs_2005}
van der SLUIJS, J.P., 2005. \TitleURL{http://www.iwaponline.com/wst/05206/wst052060087.htm}{Uncertainty as a monster in the science-policy interface: four coping strategies}. Water Science \& Technology 52 (6): 87-92.



\bibitem{Frame_2008}
FRAME, B., 2008. \TitleURL{http://dx.doi.org/10.1068/c0790s}{'Wicked', 'messy', and 'clumsy': long-term frameworks for sustainability}. Environment and Planning C: Government and Policy 26 (6): 1113-1128.


\bibitem{McGuire_2010}
MCGUIRE, M., SILVIA, C., 2010. \TitleURL{http://dx.doi.org/10.1111/j.1540-6210.2010.02134.x}{The effect of problem severity, managerial and organizational capacity, and agency structure on intergovernmental collaboration: Evidence from local emergency management}. Public Administration Review 70(2): 279-288.


\bibitem{Bea_etal_2009}
BEA, R., et al, 2009. \TitleURL{http://dx.doi.org/10.1057/rm.2008.12}{A new approach to risk: The implications of e3}. Risk Management 11 (1): 30-43.


\bibitem{Adams_Hester_2012}
ADAMS, K.M., HESTER, P.T., 2012. \TitleURL{http://dx.doi.org/10.1504/ijsse.2012.052683}{Errors in systems approaches}. International Journal of System of Systems Engineering 3 (3/4): 233+.


\bibitem{Larsson_etal_2010}
LARSSON, A., et al, 2010. \TitleURL{http://dx.doi.org/10.2202/1547-7355.1646}{Decision evaluation of response strategies in emergency management using imprecise assessments}. Journal of Homeland Security and Emergency Management 7 (1).


\bibitem{Innocenti_Albrito_2011}
INNOCENTI, D., ALBRITO, P., 2011. \TitleURL{http://dx.doi.org/10.1016/j.envsci.2010.12.010}{Reducing the risks posed by natural hazards and climate change: the need for a participatory dialogue between the scientific community and policy makers}. Environmental Science \& Policy 14 (7): 730-733.


\bibitem{Ravetz_2004}
RAVETZ, J., 2004. \TitleURL{http://dx.doi.org/10.1016/S0016-3287(03)00160-5}{The post-normal science of precaution}. Futures 36 (3): 347-357.


\bibitem{INRMM}
de RIGO, D., 2012. Integrated Natural Resources Modelling and Management: minimal redefinition of a known challenge for environmental modelling. Excerpt from the Call for a shared research agenda toward scientific knowledge freedom, Maieutike Research Initiative 

\bibitem{deRigo2012}
de RIGO, D., 2012, \TitleURL{http://www.iemss.org/iemss2012/proceedings/D3\_1\_0715\_deRigo.pdf}{Semantic Array Programming for Environmental Modelling: \mbox{Application} of the Mastrave Library}. Int. Congress on Environmental Modelling and \mbox{Software}. Managing Resources of a Limited Plant, 1167-1176.


\bibitem{deRigo2012b}
de RIGO, D., 2012. \TitleURL{http://mastrave.org/doc/MTV-1.012-1/}{Semantic Array Programming with Mastrave - Introduction to Semantic Computational Modelling}.


\bibitem{Corti_etal_2012}
CORTI, P., et al, 2012. \TitleURL{http://dx.doi.org/10.6084/m9.figshare.101918}{Fire news management in the context of the european forest fire information system (EFFIS)}. In: proceedings of "Quinta conferenza italiana sul software geografico e sui dati geografici liberi" (GFOSS DAY 2012).


\bibitem{Sheth_2009}
SHETH, A., 2009. \TitleURL{http://dx.doi.org/10.1109/mic.2009.77}{Citizen sensing, social signals, and enriching human experience}. Internet Computing, IEEE 13 (4): 87-92.


\bibitem{Zhang_etal_2011}
ZHANG, D., et al, 2011. \TitleURL{http://dx.doi.org/10.1109/mc.2011.65}{The emergence of social and community intelligence}. Computer 44 (7): 21-28.


\bibitem{Adam_etal_2012}
ADAM, N.R., et al, 2012. \TitleURL{http://dx.doi.org/10.1109/mis.2012.113}{Spatial computing and social media in the context of disaster management}. Intelligent Systems, IEEE 27 (6): 90-96.


\bibitem{Fraternali_2012}
FRATERNALI, P., et al, 2012. \TitleURL{http://dx.doi.org/10.1016/j.envsoft.2012.03.002}{Putting humans in the loop: Social computing for water resources management}. Environmental Modelling \& Software 37: 68-77.


\bibitem{Aseretto_submitted}
RODRIGUEZ-ASERETTO, D., et al, (exp.) 2014. Image geometry correction of daily forest fire progression map using MODIS active fire observation and Citizens sensor. IEEE Earthzine 7(2). Submitted.


\bibitem{Bosco_etal_2013}
BOSCO, C., et al, 2013. \TitleURL{http://dx.doi.org/10.1007/978-3-642-41151-9\_31}{Multi-Scale Robust Modelling of Landslide Susceptibility: \mbox{Regional} Rapid Assessment and Catchment Robust Fuzzy Ensemble}. IFIP Adv. Inf. Commun. \mbox{Technol}. 413: 321-335.

\bibitem{DiLeo_etal_2013}
DI LEO, M., et al, 2013. \TitleURL{http://dx.doi.org/10.1007/978-3-642-41151-9\_2}{Dynamic data driven ensemble for wildfire behaviour assessment: a case study}. IFIP Adv. Inf. Commun. Technol. 413: 11-22.


\bibitem{Ackerman_Heinzerling_2002}
ACKERMAN, F., HEINZERLING, L., 2002. \TitleURL{http://dx.doi.org/10.2307/3312947}{Pricing the priceless: Cost-Benefit analysis of environmental protection}. University of Pennsylvania Law Review 150 (5): 1553-1584.


\bibitem{Gasparatos_2010}
GASPARATOS, A., 2010. \TitleURL{http://dx.doi.org/10.1016/j.jenvman.2010.03.014}{Embedded value systems in sustainability assessment tools and their implications}. Journal of Environmental Management 91 (8): 1613-1622.


\bibitem{de_Rigo_2001}
de RIGO, D., et al, 2001. \TitleURL{http://dx.doi.org/10.5281/zenodo.7481}{Neuro-dynamic programming for the efficient management of reservoir networks}. In: Proceedings of MODSIM 2001, International Congress on Modelling and Simulation. Vol. 4. Model. Simul. Soc. Australia and New Zealand, pp. 1949-1954. 


\bibitem{RodriguezAseretto_etal_2009}
RODRIGUEZ-ASERETTO, D., et al, 2009. \TitleURL{http://dx.doi.org/10.1007/978-3-642-01973-9\_55}{Injecting dynamic Real-Time data into a DDDAS for forest fire behavior prediction}. Lecture Notes in Computer Science 5545: 489-499.


\bibitem{Cencerrado_etal_2009}
CENCERRADO, A., et al, 2009. \TitleURL{http://dx.doi.org/10.1007/978-3-642-01970-8\_23}{Support for urgent computing based on resource virtualization}. Lecture Notes in Computer Science 5544: 227-236. 


\bibitem{Yoshimoto2012}
JOSHIMOTO, K, K, et al, 2012. \TitleURL{http://dx.doi.org/10.1016\%2Fj.procs.2012.04.186}{Implementations of Urgent Computing on Production HPC Systems}. Procedia Computer Science 9: 1687-1693.


\bibitem{de_Rigo_Bosco_2011}
de RIGO, D., BOSCO, C., 2011. \TitleURL{http://dx.doi.org/10.1007/978-3-642-22285-6\_34}{Architecture of a Pan-European framework for \mbox{integrated} soil water erosion assessment}. IFIP Adv. Inf. Commun. \mbox{Technol}. 359: 310-318.


\bibitem{de_Rigo_etal_EZ2013}
de RIGO, D., et al, 2013. \TitleURL{http://dx.doi.org/10.1007/978-3-642-41151-9\_26}{Continental-Scale Living Forest Biomass and Carbon Stock: A Robust Fuzzy Ensemble of IPCC Tier 1 Maps for Europe}. IFIP Adv. Inf. Commun. \mbox{Technol}. 359: 271-284. 


\bibitem{RodriguezAseretto_etal_2013}
RODRIGUEZ-ASERETTO, D., et al, 2013. \TitleURL{http://dx.doi.org/10.1016/j.procs.2013.05.355}{A data-driven model for large wildfire \mbox{behaviour} prediction in Europe}. Procedia Computer Science 18: 1861-1870.


\bibitem{Castelletti2008}
CASTELLETTI, A., et al, \TitleURL{http://www.nt.ntnu.no/users/skoge/prost/proceedings/ifac2008/data/papers/1685.pdf}{On-Line design of water reservoir policies based on \mbox{inflow} \mbox{prediction}}. IFAC-PapersOnLine 17: 14540-14545.

\bibitem{McInerney2012} 
MCINERNEY, D., et al, 2012. \TitleURL{http://dx.doi.org/10.1109/JSTARS.2012.2194136}{Developing a forest data portal to support Multi-Scale decision making}. IEEE J. Sel. Top. Appl. Earth Obs. Remote Sens. 5(6): 1692-1699. 


\bibitem{Bastin2012} 
BASTIN, et al, 2012. \TitleURL{http://www.earthzine.org/?p=389531}{Web services for forest data, analysis and monitoring: Developments from EuroGEOSS}. IEEE Earthzine 5(2): 389531+.

\bibitem{RodriguezAseretto_EFDAC2013} 
RODRIGUEZ-ASERETTO, D., et al, 2013. \TitleURL{http://dx.doi.org/10.6084/m9.figshare.155700}{Free and open source software underpinning the European Forest Data Centre}. Geophysical Research Abstracts 15: 12101+. 


\bibitem{NSFC2007}
NATIONAL SCIENCE FOUNDATION, CYBERINFRASTRUCTURE COUNCIL, 2007. \TitleURL{http://www.nsf.gov/pubs/2007/nsf0728/index.jsp?org=EF}{Cyberinfrastructure vision for 21st century discovery}. Tech. Rep. NSF 07-28, National Science Foundation.


\bibitem{Altay_Green_2006}
ALTAY, N., GREEN, W.G., 2006. \TitleURL{http://dx.doi.org/10.1016/j.ejor.2005.05.016}{OR/MS research in disaster operations management}. European Journal of Operational Research 175 (1): 475-493.


\bibitem{RodriguezAseretto_etal_2008}
RODRIGUEZ-ASERETTO, D., et al, 2008. \TitleURL{http://dx.doi.org/10.1109/CSE.2008.15}{An adaptive system for forest fire behavior prediction}. In: Computational Science and Engineering, 2008. CSE '08. 11th IEEE International Conference on. IEEE, pp. 275-282.



\end{thebibliography}

\end{nohyphens}

\end{spacing}

\end{document}
