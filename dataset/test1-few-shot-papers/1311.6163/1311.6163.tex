\documentclass[journal]{IEEEtran}
\newcommand{\ee}{\epsilon}			
\usepackage{epsf,psfrag,amssymb,amsfonts,color,cite,fancybox}
\usepackage[mathscr]{eucal}
\usepackage[dvips]{graphicx}
\usepackage{ifpdf}
\usepackage{longtable}	
\usepackage{multicol}
\usepackage{float}

\ifCLASSINFOpdf

\else

\fi

\usepackage[cmex10]{amsmath}
\interdisplaylinepenalty=2500

\usepackage[caption=false,font=footnotesize]{subfig}

\hyphenation{op-tical net-works semi-conduc-tor}


\begin{document}
\renewcommand{\thefigure}{\arabic{figure}}
\renewcommand{\thesubfigure}{\alph{subfigure}}

\title{Analytical Studies of Quasi Steady-State Model in Power System Long-Term Stability Analysis}


\author{Xiaozhe Wang,~\IEEEmembership{Student Member,~IEEE,}
        Hsiao-Dong Chiang,~\IEEEmembership{Fellow,~IEEE.}
\thanks{This work was supported by the Consortium for Electric Reliability
Technology Solutions provided by U.S. Department No. DE-FC26-
09NT43321.}
\thanks{Xiaozhe Wang is with the School of Electrical and Computer Engineering, Cornell University, Ithaca,
NY 14853 USA, email: xw264@cornell.edu.}\thanks{Hsiao-Dong Chiang is with the School of Electrical and Computer Engineering, Cornell University, Ithaca, NY 14853 USA, email:hc63@cornell.edu.}\thanks{Copyright (c) 2013 IEEE. Personal use of this material is
permitted. However, permission to use this material
for any other purposes must be obtained from the IEEE by
sending an email to pubs-permissions@ieee.org.}}


\maketitle


\begin{abstract}
In this paper, a theoretical foundation for the Quasi Steady-State (QSS) model in power system long-term stability analysis is developed. Sufficient conditions under which the QSS model gives accurate approximations of the long-term stability model in terms of trajectory and -limit set are derived. These sufficient conditions provide some physical insights regarding the reason for the failure of the QSS model. Additionally, several numerical examples are presented to illustrate the analytical results derived.
\end{abstract}

\begin{IEEEkeywords}
sufficient conditions, quasi steady-state model, power system long-term stability.
\end{IEEEkeywords}

\IEEEpeerreviewmaketitle



\section{Introduction}
\IEEEPARstart{T}HE ever-increasing loading of transmission networks together with a steady increase in load demands has pushed many power systems ever closer to their stability limit \cite{Chiang:book}-\cite{SauerPai:book}. Long-term stability has become more and more important for secure operation of power systems. However, the long-term stability model is large and involves different time scales. The time domain simulation approach for the long-term stability model is expensive in terms of computational efforts and data processing. These constraints are even more stringent in the context of on-line stability assessment.
The quasi steady-state (QSS) proposed in \cite{Cutsem:book}-\cite{Cutsem:artical} tried to reach a good compromise between accuracy and efficiency for long-term stability analysis. The assumptions behind the QSS model that the post-fault transient stability model is stable and the long-term stability model is singularity-free are not necessarily true. There have been some efforts attending to address these issues \cite{Cutsem:artical2}-\cite{Wang:artical}. However, less attention has been paid to another critical issue that even these assumptions are satisfied, the QSS model may still provide incorrect approximations for the long-term stability model. Some counter examples in which the QSS model were stable while the long-term stability model underwent long-term instabilities were presented in \cite{Wangxz:article}. Since the QSS model can not consistently provide correct stability analysis of the long-term stability model, there is a great need to identify conditions under which the QSS model works.
In this paper, sufficient conditions under which the QSS model can provide correct approximations for the long-term stability model are developed. Briefly speaking, if neither the long-term stability model nor the QSS model meets a singularity, then the QSS model provides correct approximations for the long-term stability model in terms of trajectory if the QSS model moves along the stable component of its constraint manifold and the projection of each point on the trajectory of the long-term stability model lies inside the stability region of the corresponding transient stability model. Moreover, if the QSS model converges to a long-term stable equilibrium point (SEP), then the long-term stability model will converge to the same point. Several numerical examples in which the QSS model succeeded or failed are analyzed by the derived analytical results.

This paper is organized as follows. Section \ref{sectioncompleteqssmodel} recalls basic concepts of power system models, and Section \ref{mathprelim} introduces mathematical preliminaries in nonlinear system theories. Then sufficient conditions of the QSS model are derived in Section \ref{analyticalstudies}, and several numerical examples are analyzed based on the derived theorems in Section \ref{numericalstudies}. Conclusions and perspectives are stated in Section \ref{conclusion}.

\section{Power System Models}\label{sectioncompleteqssmodel}
The long-term stability model, or interchangeably complete dynamic model, for calculating system dynamic response relative to a disturbance can be described as:


Equation (\ref{algebraic eqn}) describes the electrical transmission system and the internal static behaviors of passive devices, and (\ref{fast ode}) describes the internal dynamics of devices such as generators, their associated control systems, certain loads, and other dynamically modeled components.  and  are continuous functions, and vector  and  are the corresponding short-term state variables and algebraic variables. Besides, Equations (\ref{slow ode}) and (\ref{slow dde}) describe long-term dynamics including exponential recovery load, turbine governor, load tap changer (LTC), over excitation limiter (OXL), etc.  and  are the continuous and discrete long-term state variables respectively, and  is the maximum time constant among devices. Since transient dynamics have much smaller time constants compared with those of long-term dynamics,  and  are also termed as slow state variables, and  are termed as fast state variables. Detailed power system models and corresponding variables are given in Appendix.

The transient stability model and the QSS model are regarded as two approximations of the long-term stability model in short-term and long-term time scales respectively, and they are believed to offer a good compromise between accuracy and efficiency. In transient stability model, slow variables are considered as constants. While in the QSS model, the dynamic behavior of fast variables are considered as instantaneously fast and thus replaced by its equilibrium equations in the long-term time scale. If we represent the long-term stability model and the QSS model in  time scale, where , and we denote  as , then power system models can be represented as shown in Table \ref{table1}.

\begin{center}
\begin{table}[h]
\centering
\caption{The mathematical description of models in power system}\label{table1}
\begin{tabular}{|c|c|}
\hline
the long-term stability model&
\\
&\\
&\\
&\\
\hline
the transient stability model&\\
short-term:0-30s&\\
\hline
the QSS model&\\
long-term:30s-a few minutes&\\
&\\
&\\
\hline
\end{tabular}
\end{table}
\end{center}

The QSS model may fail to capture dynamics of the long-term stability model, thus provide incorrect approximations of the long-term stability model leading to incorrect stability assessment.
\section{Mathematical Preliminaries}\label{mathprelim}
In this section, some relevant stability concepts from nonlinear system theories are briefly reviewed. Knowledge of stability region is required in analyzing the QSS model for long-term stability analysis.

\subsection{Stability of Equilibrium Point and Stability Region}
We consider the following autonomous nonlinear dynamical system:

where  satisfies a sufficient condition for the existence and uniqueness of a solution. The solution of (\ref{general autonomous}) starting at initial state  at time  is called the system trajectory and is denoted as .  is said to be an equilibrium point of (\ref{general autonomous}) if . The definition of asymptotic stability is given as below \cite{Chiang:book}:

\noindent\textit{\textbf{Definition 1: Asymptotic Stability}}

An equilibrium point  of (\ref{general autonomous}) is said to be asymptotically stable if, for each open neighborhood  of , the followings are true: (i)  for all ; (ii) .

Without confusion, we use stable equilibrium point (SEP) instead of asymptotically stable equilibrium point in this paper. An equilibrium point is \textit{hyperbolic} if the corresponding Jacobian matrix has no eigenvalues with zero real parts. And a hyperbolic equilibrium point  is a \textit{type-k equilibrium point} if there exist  eigenvalues of  with positive real parts. The stability region of a SEP  is the set of all points  such that . In other words, the \textit{stability region} is defined as:


From a topological point of view, the stability region is an open invariant and connected set. Every trajectory in a stability region lies entirely in the stability region and the dimension of the stability region is .

\noindent\textit{\textbf{Definition 2: -limit Set}}

A point  is said to be the -\textit{limit point} of  if, corresponding to each  and , there is a  with the property that . Equivalently, there is a sequence  in , , with the property that . The set of all -limit points for  is defined as its -\textit{limit set}.

\subsection{Singular Perturbed System}
We next consider the following general singular perturbed model:

where  is a small positive parameter.  is a vector of slow variables while  is a vector of fast variables. Let  denotes the trajectory of model (\ref{singular perturb}) starting at  and  denotes the set of equilibrium points of it, i.e. . If  is a SEP of model (\ref{singular perturb}), then the stability region of  is defined as:

The slow model is obtained by setting  in (\ref{singular perturb}):
The algebraic equation  constraints the slow dynamics to the following set which is termed as \textit{constraint manifold}:

The trajectory of model (\ref{slow}) starting at  is denoted by  and the stability region is


The \textit{singular points} of system (\ref{slow}) or \textit{singularity } is defined as:


Singular points can drastically influence the trajectories of the differential-algebraic equation (DAE) system. Typically, the singular set  is a stratified set of maximal dimension  embedded in  and  is separated by  into open regions \cite{Veukatasubramanian:article}\cite{Alberto:article}.

\noindent \textit{\textbf{Definition 3: Type of Constraint Manifold}}

The connected set  is a \textit{type-k component of } if the matrix , evaluated at every point of , has  eigenvalues that have positive real parts. If all the eigenvalues of  calculated at points of  have a negative real part, then we call  a \textit{stable component of }; otherwise, it's an \textit{unstable component of }.

We next define the fast model associated with the singularly perturbed model, i.e. boundary layer model. Define the fast time scale . In this time scale, model (\ref{singular perturb}) takes the form:

Let  denote the trajectory of model (\ref{singular perturb in sigma}) starting at .

where  is frozen and treated as a parameter. The constraint manifold  is a set of equilibriums of models (\ref{BLS}). For each fixed , a fast dynamical model (\ref{BLS}) is defined.

\noindent\textit{\textbf{Definition 4: Uniformly Asymptotically Stable}}

Assuming , and  is an isolated root of equation:

then  is an equilibrium point of system (\ref{BLS}), if  is a SEP of system (\ref{BLS}) for all , then  is \textit{uniformly asymptotically stable} with respect to .

The next Theorem ensures that, that solutions of the singular perturbed model (\ref{singular perturb}) can be, at least for sufficiently small , approximated by solutions of the slow model (\ref{slow}).


\noindent\textbf{Theorem 1 (Tikhonov's Result on Finite Interval)}\cite{Khalil:book}\cite{Lobry:article2}:

Consider the singular perturbation problem (\ref{singular perturb}) and let  be an isolated root of system (\ref{BLSsep}). Assume that there exist positive constants ,  and , and a compact domain  such that the following conditions are satisfied for all , , , 

(a). The functions ,  and  are continuous;

(b). The slow model (\ref{slow}) has a unique solution  with initial condition , defined on  and  for all ;

(c). The fast model (\ref{BLS}) has the uniqueness of the solutions with prescribed initial conditions. Let  be the solution of system:


(d). The equilibrium point  of fast model is uniformly asymptotically stable in ;

(e). The initial condition  belongs to the stability region  of system (\ref{initialBLS}).

Then for every , there exists a positive constant  such that for all , every solution  of the singular perturbation model (\ref{singular perturb}) exists at least on , and satisfies

for all .

Note that  is required to be a compact set, thus the solution of slow model stays inside a compact set to avoid non essential technicalities with the maximal interval of definition of a solution\cite{Lobry:article2}.

Theorem 1 can be extended to the infinite-time interval under some additional conditions which ensure stability of the solutions of the singular perturbation problem (\ref{singular perturb}) \cite{Lobry:article2}.

\noindent\textbf{Theorem 2 (Tikhonov's Result on Infinite Interval)}\cite{Khalil:book}\cite{Lobry:article2}:

Consider the singular perturbation problem (\ref{singular perturb}) and let  be an isolated root of system (\ref{BLSsep}). Assume that there exist positive constants  and , and a compact domain  such that the following conditions are satisfied for all , , , 

(a). The functions ,  and  are continuous;

(b). The solution  of the slow model starting from  exists for all , and the -limit set of the slow model (\ref{slow}) is a SEP denoted as ;

(c). The fast model (\ref{BLS}) has the uniqueness of the solutions with prescribed initial conditions. Let  be the solution of system (\ref{initialBLS});

(d). The equilibrium point  of fast model is uniformly asymptotically stable in ;

(e). The initial condition  belongs to the stability region  of system (\ref{initialBLS}).

Then for every , there exists a positive constant  such that for all , every solution  of the singular perturbation model (\ref{singular perturb}) exists for all , and satisfies

for all .

Assume the solution of singular perturbation problem (\ref{singular perturb})  is unique, then we have \cite{Lobry:article2}\cite{Lobry:article}:


\section{Analytical Studies of QSS model}\label{analyticalstudies}

The long-term stability model of power system can be represented as:


where . Note that shunt compensation switching and LTC operation are typical discrete events captured by  and  is shunt susceptance and the transformer ratio, respectively. Transitions of  depend on system variables, thus  change values from  to  at distinct times  where , otherwise, these variables remain constants.

Consider the long-term stability model (\ref{complete}), it can be regarded as two decoupled systems (\ref{couple1}) and (\ref{couple2}) shown as below when  jump from  to :

and

discrete variables  are updated first and then system (\ref{couple2}) works with fixed parameters .

Similarly, when  jump from  to , the QSS model

can be decoupled as:

and

\subsection{Models in Nonlinear Framework}
For the study region , where , , , , both the long-term stability model and the QSS model have the same set of equilibrium points . Assuming  is a long-term SEP of both the long-term stability model (\ref{complete}) and the QSS model (\ref{QSS}) starting from  and  respectively, and  denotes trajectory of the long-term stability model (\ref{complete}) and  denotes trajectory of the QSS model (\ref{QSS}). Then, the stability region of the long-term stability model (\ref{complete}) is:


The stability region of the QSS model (\ref{QSS}) is


The singular points of constraint manifold  are:

And type- component of  where  is defined as:


When  and , for each fixed  and , given a point  on , the corresponding transient stability model is defined as:

If , then
 is an equilibrium point of (\ref{transient}), where

If  is a SEP of (\ref{transient}), then the stability region of  is represented as:

where  denotes the trajectory of the transient stability model (\ref{transient}).

Assuming that  is nonsingular, then transient stability model (\ref{transient}) can be linearized near the equilibrium point as:

and we can define a subset of the stable component of constraint manifold :


such that each point on  is a SEP of the corresponding transient stability model (\ref{transient}) for fixed  and . A comprehensive theory of stability regions can be found in \cite{Alberto:article}\cite{Chiang:article1988}\cite{Chiang:article1989}.

We divide the task of establishing a theoretical foundation for the QSS model into two steps as \textbf{Case I} and \textbf{Case II}. Firstly, we analyze the trajectory and -limit set relations of the long-term stability model (\ref{couple2}) and the QSS model (\ref{coupleqss2}), that is we regard discrete variables  as fixed parameters. Next, we move one step further to include discrete dynamics  and deduce the relations of the long-term stability model (\ref{complete}) and the QSS model (\ref{QSS}) in terms of trajectory and -limit set.

Before proceeding, we need some important assumptions:

\noindent\textbf{S1.} Neither the long-term stability model nor the QSS model meets singularity points.

\noindent\textbf{S2.} The trajectories of the long-term stability model, the QSS model and transient stability models with specified initial conditions exist and are unique. Additionally,  is compact.

\noindent\textbf{S3.} Equilibrium point of transient stability model is continuous in  when  are fixed as parameters.

Note that the uniqueness of solutions is generally satisfied in power system models. Besides, since a power system is a real physical system, the domain of each variable is generally compact. As for S3, if S1 is satisfied, we know that equilibrium point of transient stability model  is at least locally continuous by Implicit Function Theorem. Moreover, as  only varies slowly and subtly, S3 is also generally satisfied. As a result, if S1 is satisfied, we can safely assume that S2 and S3 are satisfied in power system models.

\subsection{Case I: Relations of Trajectory and -limit Set}

Assuming the initial point of the long-term stability model is , and the initial point of QSS model is . Then the initial transient stability model can be represented as:

with the equilibrium point .
Equivalently, system (\ref{initialtransientxy}) can be represented as

with equilibrium point .

Additionally, denote the solution of QSS model (\ref{coupleqss2}) as  and the solution of the initial transient stability model as  . Besides, denote  to be a set such that for all , , and let .

\noindent\textbf{Theorem 3: (Trajectory Relation)}:

Assuming there exist positive constants ,  and  such that S1-S3 and the following conditions are satisfied for all :

(a). The trajectory  of the QSS model (\ref{coupleqss2}) moves along ;

(b). The projection of initial point  of the long-term stability model (\ref{couple2}) to the subspace of  and  is inside the stability region  of the initial transient stability model (\ref{initialtransient}).

Then for every  there exists a positive constant  such that for all , the solution  of the long-term stability model (\ref{couple2}) exists at least on , and satisfies:

for all .

Theorem 3 asserts that if the projection of initial point of the long-term stability model lies inside the stability region of the initial transient stability model, and  moves along , then for sufficiently small , trajectory of the long-term stability model (\ref{couple2}) can be approximated by trajectory of the QSS model (\ref{coupleqss2}).

\noindent\textbf{Proof:}
If S1 is satisfied, then  and  are nonsingular, according to the Implicit Function Theorem, ,  can be solved from:

with the solution , Thus the long-term stability model (\ref{couple2}) becomes:


Hence, the long-term stability model is transformed into the standard singular perturbation problem considered in Theorem 1, and the QSS model is the corresponding slow model. Next, from the detailed power system models in Appendix, the following fact follows.

\noindent\textit{Fact 1}: These maps ,  and  that describe slow dynamics, fast dynamics and algebraic constraints respectively are continuous.

With Fact 1 and S3, condition (a) of Theorem 1 is satisfied; with S2, condition (b) and (c) of Theorem 1 are satisfied. Furthermore, if condition (a) of Theorem 3 is satisfied, then  is a subset of stable component of the constraint manifold, thus each point on  is a SEP of the corresponding transient stability model. In other words, if  moves along , then  is  asymptotically stable uniformly in , hence condition (d) of Theorem 1 is satisfied. Note that if  is asymptotically stable,  is necessarily to be isolated by definition. Finally, condition (b) of Theorem 3 ensures the satisfaction of condition (e) in Theorem 1. According to Theorem 1, it follows that for every  there exists a positive constant  such that for all , the solution  of system (\ref{approx_tran}) exists at least on , and satisfies:


This completes the proof of the theorem.

Next we proceed to identify the -limit set relation between the long-term stability model (\ref{couple2}) and the QSS model (\ref{coupleqss2}).

\noindent\textbf{Theorem 4: (-Limit Set Relation)}:

Assuming there exist positive constants  and  such that S1-S3 and the following conditions are satisfied for all :

(a). The trajectory  of the QSS model (\ref{coupleqss2}) moves along ;

(b). The projection of initial point  of the long-term stability model (\ref{couple2}) to the subspace of  and  is inside the stability region  of the initial transient stability model (\ref{initialtransient});

(c). The -limit set of the QSS model (\ref{coupleqss2}) starting from  is a SEP .

Then the solution  of the long-term stability model (\ref{couple2}) exists for all , and satisfies the following limit relations:



Theorem 4 asserts that if all conditions of Theorem 3 are satisfied and the QSS model (\ref{coupleqss2}) converges to a long-term SEP of the QSS model, then for sufficiently small , the long-term stability model (\ref{couple2}) will converge to the same point.

\textbf{Proof}:
If S1 is satisfied, then  and  are nonsingular. Likewise, we can transform the long-term stability model (\ref{couple2}) to system (\ref{approx_tran}) which is the standard singular perturbation problem considered in Theorem 2.

From the proof of Theorem 3, we have that with S2, S3 and Fact 1, condition (a) and (c) of Theorem 2 are satisfied. Besides, condition (a) and (b) of Theorem 4 ensures the satisfaction of condition (d) and (e) in Theorem 2 respectively. Finally, with S2 and condition (c) of Theorem 4, it follows that the solution of the QSS model (\ref{coupleqss2}) exists for all  and the -limit set of the QSS model is a SEP, thus condition (b) of Theorem 2 is satisfied. Therefore all conditions of Theorem 2 are satisfied, it follows that for every , there exists a positive constant  such that for all , the solution  of the long-term stability model (\ref{couple2}) exists for all , and satisfies

for all . Since the solution of the long-term stability model (\ref{couple2}) is unique, we have

This completes the proof of the theorem. And Fig. \ref{QSSctnstable} gives an illustration of Theorem 3 and Theorem 4.

\begin{figure}[!ht]
\centering
\includegraphics[width=3.5in,keepaspectratio=true,angle=0]{drawing.eps}\caption{Illustration of Theorems 3 and 4.  is constrained on  all the time. The projected initial point  of the long-term stability model locates inside , then  always stays close to . Moreover, since  converges to a SEP ,  also converges to .
}\label{QSSctnstable}
\end{figure}

\subsection{Case II: Relations of Trajectory and -limit Set}

Next, we are at the stage to incorporate discrete behaviors of  in the long-term stability model and the QSS model, and explore trajectory and -limit set relations between them.

Assuming  initially at , and jump from  to  at time , where . Similarly, the initial point of the long-term stability model is , and the initial point of QSS model is . Then the initial transient stability model can be represented as:

with equilibrium point .

Denote the solution of QSS model (\ref{coupleqss2}) as , and denote the solution of the initial transient stability model and transient stability models immediately after  jump to  as  for all .


\noindent\textit{\textbf{Definition 5: Consistent Attraction}}

We say that the long-term stability model satisfies the condition of \textit{consistent attraction}, if whenever long-term discrete variables jump from  to , , the point on the trajectory of the long-term stability model immediately after  jump stays inside the stability region of the corresponding transient stability model.

The following two theorems provide a theoretical foundation for the QSS model in which trajectory and -limit set relations of the long-term stability model (\ref{complete}) and the QSS model (\ref{QSS}) are established.

\noindent\textbf{Theorem 5: (Trajectory Relation)}

Assuming there exist positive constants ,  and  such that S1-S3 and the following conditions are satisfied for all :

(a). The trajectory  of the QSS model (\ref{QSS}) moves along ;

(b). The projection of initial point  of the long-term stability model (\ref{complete}) to the subspace of  and  is inside the stability region  of the initial transient stability model (\ref{initialtransient_II}), and the long-term stability model (\ref{complete}) satisfies the condition of consistent attraction.

Then for every  there exists a positive constant  such that for all , every solution  of system (\ref{couple2}) exists at least on , and satisfies:

for all , .

Theorem 5 asserts that if the trajectory of QSS model moves along , and the projection of each point on trajectory of the long-term stability model always lies inside the stability region of the corresponding transient stability model, then for sufficiently small , trajectory of the long-term stability model (\ref{complete}) can be approximated by trajectory of the QSS model (\ref{QSS}).

\noindent\textbf{Proof:}
Conditions of Theorem 5 ensure that conditions of Theorem 3 are satisfied for each fixed , . Thus we can apply the conclusions of Theorem 3 for each . We have that, for every  there exists a positive constant  such that for all , the solution  of system (\ref{couple2}) exists at least on , and satisfies:

for all . Let , then for every  there exists a positive constant  such that for all , the solution  of system (\ref{couple2}) exists at least on , and satisfies:

for all , where .
The proof is complete.

We next show the -limit set relation between the long-term stability model (\ref{complete}) and the QSS model (\ref{QSS}).

\noindent\textbf{Theorem 6: (-Limit Set Relation)}

Assuming there exist positive constants  and  such that S1-S3 and the following conditions are satisfied for all :

(a). The trajectory  of the QSS model (\ref{QSS}) moves along ;

(b). The projection of initial point  of the long-term stability model (\ref{complete}) to the subspace of  and  is inside the stability region  of the initial transient stability model (\ref{initialtransient_II}), and the long-term stability model (\ref{complete}) satisfies the condition of consistent attraction;

(c). The -limit set of the QSS model (\ref{QSS}) starting from  is a SEP .

Then the solution  of the long-term stability model (\ref{complete}) exists for all , and satisfies the following limit relations:


Theorem 6 asserts that if all conditions of Theorem 5 are satisfied and the QSS model (\ref{QSS}) converges to a long-term SEP of the QSS model, then for sufficiently small , the long-term stability model will converge to the same point.

\noindent\textbf{Proof:}
Since , then according to Theorem 4, we have:


Next, since the long-term stability model (\ref{approx_tran}) with each fixed parameter  has a unique solution for all , the whole long-term stability model (\ref{complete}) with initial condition  will also has a unique solution which is denoted as . Hence we have:


The proof is complete.

\begin{figure}[!ht]
\centering
\includegraphics[width=3.3in,keepaspectratio=true,angle=0]{drawingzdstable_anly.eps}\caption{Illustration of Theorems 5 and 6.  is constrained on  all the time. The projected initial point of the long-term stability model locates inside the stability region of the initial transient stability model, and the long-term stability model satisfies the condition of consistent attraction. Finally both  and  converge to the same point . }\label{QSSstable}
\end{figure}

Fig. \ref{QSSstable} gives an illustration of Theorem 5 and Theorem 6. Note that the condition of consistent attraction is crucial. For instance, assuming when  jump from  to , the first point after the jump  on  locates outside the stability region  of the transient stability model:

then  will move away from  as shown in Fig. \ref{QSSzdunstable}.
\begin{figure}[!ht]
\centering
\includegraphics[width=3.5in,keepaspectratio=true,angle=0]{drawingzd.eps}\caption{Illustration of Theorems 5 and 6. When  jump to , the first point on  after the jump gets outside of the stability region of the corresponding transient stability model, thus the long-term stability model doesn't satisfy the condition of consistent attraction, and  moves far away from the QSS model from then on.}\label{QSSzdunstable}
\end{figure}

\section{Numerical Illustration} \label{numericalstudies}
In this section, two examples will be analyzed using the derived theorems. In the first numerical example, the QSS model provided correct approximations of the long-term stability model since all conditions of Theorem 5 and Theorem 6 are satisfied. In the second numerical example, the QSS model failed to give correct approximations due to the violation of condition (b) in Theorem 5. All simulations were done using PSAT 2.1.6 \cite{Milano:article}.
\subsection{Numerical Example I}
The first example was a modified IEEE 14-bus systems\cite{testcasearchive} in which QSS model gave correct approximations of the long-term stability model. In this system, an exponential recovery load was included at Bus 5 and two turbine governors at Bus 1 and Bus 2 were added respectively. The assumption S1 that neither the long-term stability model and the QSS model meets singularity points was satisfied. And we can safely assume that S2 was also satisfied. Besides, from the trajectory of the QSS model, we can see that S3 was also satisfied in this case. In addition, the trajectory  of QSS model moved along . As QSS model was implemented 30s after the contingency, fast dynamics settled down at the time such that the projection of the initial point  of the long-term stability model lied inside . And whenever  jumped to , , the first point  on   after the jump stayed inside the stability region of the corresponding transient stability model such that the long-term stability model satisfied the condition of consistent attraction. Since all conditions of Theorem 5 were satisfied,  always stayed close to . Additionally, as the QSS model converged to a long-term SEP, the long-term stability model converged to the same point. Fig. \ref{my14completeqss} shows the trajectory comparisons of the long-term stability model and the QSS model.
\begin{figure}[!ht]
\centering
\begin{minipage}[t]{0.5\linewidth}
\includegraphics[width=1.8in ,keepaspectratio=true,angle=0]{vr1Exc1.eps}
\end{minipage}\begin{minipage}[t]{0.5\linewidth}
\includegraphics[width=1.8in ,keepaspectratio=true,angle=0]{voltagebus2.eps}
\end{minipage}
\caption{The trajectory comparisons of the long-term stability model and the QSS model. The trajectory of the long-term stability model followed that of the QSS model until both of them converged to the same long-term SEP.}\label{my14completeqss}
\end{figure}

To check the condition of consistent attraction, we did the following simulations. When the QSS model was implemented and the ratio of the LTC firstly jumped at 30s, the intersection of the stability region in the subspace of two fast variables is plotted in Fig. \ref{vr1vf_transient_30}, and the first point  on  when  jumped to  was marked. Additionally, trajectories of two fast variables in the corresponding transient stability model are shown in Fig. \ref{my14transient0.959}. It can be seen that the trajectory starting from  settled down to the SEP of the transient stability model which further confirmed that  did lie inside the stability region  of the corresponding transient stability model.

Fig. \ref{vr1vf_transient_50} shows stability region of the transient stability model in the subspace of the same fast variables when  jumped from  to . Likewise, this procedure can be done successively to verify that the long-term stability model satisfied the condition of consistent attraction.

\begin{figure}
\centering
\subfloat[]{\label{vr1vf_transient_30}\includegraphics[width=1.8in ,keepaspectratio=true,angle=0]{vfvr1_14_zd_2_.eps}}
\subfloat[]{\label{vr1vf_transient_50}\includegraphics[width=1.8in ,keepaspectratio=true,angle=0]{vfvr1_14_zd_4_.eps}}
\caption{Illustration of Theorems 5 and 6. (a). The stability region of the corresponding transient stability models in the subspace of two fast variables when  changed to  at 30s; (b). The same as (a), except that  changed to  at 50s. In both (a) and (b), the first points of the long-term stability model after  changed were inside the stability region of the corresponding transient stability models.}
\label{my14_SR}
\end{figure}


\begin{figure}[!ht]
\begin{minipage}[t]{0.5\linewidth}
\includegraphics[width=1.8in ,keepaspectratio=true,angle=0]{rotoranglesyn2_14transient.eps}
\end{minipage}\begin{minipage}[t]{0.5\linewidth}
\includegraphics[width=1.8in,keepaspectratio=true,angle=0]{vr1Exc1_14transient.eps}
\end{minipage}
\caption{Trajectories of the transient stability model when  jumped to  at 30s. The trajectories starting from the first point of the long-term stability model converged to the SEP of the transient stability model which indicated that the first point of the long-term stability model was inside the stability region of the transient stability model.}\label{my14transient0.959}
\end{figure}

\subsection{Numerical Example II}
This was also a 14-bus system, while the QSS model did not give correct approximations of the long-term stability model due to the violation of condition (b) in Theorem 5. The trajectory comparisons are shown in Fig. \ref{my14completeqss_try}.

\begin{figure}[!ht]
\centering
\begin{minipage}[t]{0.5\linewidth}
\includegraphics[width=1.8in ,keepaspectratio=true,angle=0]{rotorangle_try.eps}
\end{minipage}\begin{minipage}[t]{0.5\linewidth}
\includegraphics[width=1.8in ,keepaspectratio=true,angle=0]{vr1Exc1_try.eps}
\end{minipage}
\caption{The trajectory comparisons of the long-term stability model and the QSS model. The QSS model converged to a long-term SEP while the long-term stability model stopped at 101.2155s s due to instability caused by wild oscillation of transient variables.}\label{my14completeqss_try}
\end{figure}

In this case, S1-S3 was also satisfied, and trajectory  of the QSS model moved along . However, condition (b) of Theorem 5 was violated. When  jumped from  to  at 30s, the long-term stability model fixed at  was stable, the trajectory comparison is plotted in Fig. \ref{my14try_31} in which both the long-term stability model and the QSS model converged to the same long-term SEP. Fig. \ref{my14trytransient_31} shows the trajectory of a fast variable in the transient stability model and Fig. \ref{vr1vf_trytransient_30} shows stability region of the corresponding transient stability model in the subspace of two fast variables.

\begin{figure}\centering
\subfloat[]{\label{my14try_31}\includegraphics[width=1.8in ,keepaspectratio=true,angle=0]{voltagebus2_14try_31.eps}}
\subfloat[]{\label{my14trytransient_31}\includegraphics[width=1.8in ,keepaspectratio=true,angle=0]{vr1Exc2_14trytransient_31.eps}}
\caption{(a). The trajectory comparison of the long-term stability model and the QSS model when  were fixed at , and both the long-term stability model and the QSS model converged to the same long-term SEP. (b). Trajectory of the transient stability model when  changed to  at 30s. The trajectory starting from the first point of the long-term stability model converged to the SEP of the transient stability model.}
\end{figure}


However when  jumped from  to  at 40s, the long-term stability model was no longer stable which can be seen from the trajectory comparison in Fig. \ref{my14try_41}. The transient variables were excited due to the evolution of discrete variables  and the trajectory of the long-term stability model was trapped in a stable limit cycle. From a physical viewpoint, the OXL of the generator at Bus 2 reached its limit while the LTC between Bus 2 and Bus 4 tried to restore the voltage at Bus 4 thus required more power support from the generator at Bus 2. The conflict between the OXL and the LTC resulted in the limit cycle shown in Fig. \ref{my14try_41}.

Similarly, Fig. \ref{my14trytransient_41} shows the trajectory of a fast variable in the transient stability model, and Fig. \ref{vr1vf_trytransient_50} shows the stability region of the corresponding transient stability model in the same subspace as Fig. \ref{vr1vf_trytransient_30}. From these two figures it can be seen that the first point  of  after  jumped to  lied outside the stability region  of the corresponding transient stability model. As a result, the long-term stability model did not satisfy the condition of consistent attraction.

In summary, fast dynamics were excited by the evolution of long-term discrete dynamics  such that the condition of consistent attraction was violated. As a result, the QSS model did not provide correct approximations of the long-term stability model in terms of trajectories and presented incorrect stability assessment in concluding that the long-term stability model was stable while the long-term stability model was long-term unstable.

\begin{figure}\centering
\subfloat[]{\label{vr1vf_trytransient_30}\includegraphics[width=1.8in ,keepaspectratio=true,angle=0]{vfvr1_14try_zd_2_.eps}}
\subfloat[]{\label{vr1vf_trytransient_50}\includegraphics[width=1.8in ,keepaspectratio=true,angle=0]{vfvr1_14try_zd_3_.eps}}
\caption{Illustration of Theorems 5 and 6. (a). The stability region of the corresponding transient stability model in the subspace of two fast variables when  changed to  at 30s. The first point of the long-term stability model was inside the stability region of the corresponding transient stability model; (b). The same as (a) except that  changed to  at 40s. The first point of the long-term stability model was outside the stability region of the corresponding transient stability model.}
\label{my14try_SR}
\end{figure}

\begin{figure}\centering
\subfloat[]{\label{my14try_41}\includegraphics[width=1.8in ,keepaspectratio=true,angle=0]{vr1Exc2_14try_41.eps}}
\subfloat[]{\label{my14trytransient_41}\includegraphics[width=1.8in ,keepaspectratio=true,angle=0]{vr1Exc2_14trytransient_41.eps}}
\caption{(a). The trajectory comparison of the long-term stability model and the QSS model when  were fixed at . The long-term stability model became unstable while the QSS model converged to a long-term SEP. (b). Trajectory of the transient stability model when  changed to  at 40s. The trajectory starting from the first point of the long-term stability model did not converge to the SEP of the corresponding transient stability model.}
\end{figure}

We provide some physical explanation behind sufficient conditions of the QSS model to explain when the QSS model may fail. In long-term time scale, LTCs are to restore the load-side voltages and hence the corresponding load powers, while OXLs restrict the power support from generators\cite{Cutsem:artical}. The counter effects between LTCs and OXLs further introduce large changes on exciters, leading to long-term instabilities. However, the QSS model assumes that variables of exciters are stable and converge instantaneously fast as LTCs and OXLs evolve, therefore, large changes occurring in the variables of exciters are not reflected in the QSS model. As a result, when the described physical mechanism of long-term instability occurs, the QSS model can fail to provide correct approximations of the long-term stability model.

\section{Conclusions and Perspectives}\label{conclusion}
A theoretical foundation for the QSS model intended for power system long-term stability analysis has been developed. Sufficient conditions for the QSS model to approximate the long-term stability model are derived and relations of trajectory as well as -limit point between the long-term stability model and the QSS model are established. Several numerical examples in which the QSS model either succeeds or fails to provide accurate approximations are analyzed using the derived analytical results.

The analytical results derived also point to a research direction for improving the QSS model. It has been shown that the QSS model will provide accurate approximations if the trajectory of QSS model moves along the stable component of the constraint manifold and fast dynamics are not excited by the slow variables. All conditions in Theorem 5 are easy to check except the condition of consistent attraction. If an efficient numerical scheme can be developed to check this condition, then the QSS model can be improved based on the theoretical foundation. It's our intent to develop an improved QSS model to accurately approximate the long-term stability model.

\appendices
\section{Detailed Power System Models\cite{Cutsem:book}\cite{Milano:article}}
\subsection{Generator (GEN):}
Notations are in Table \ref{appedixtable1}.
Dynamic Equations:

 is a function for saturation and


besides  and  are defined as
, ,
and following equations describe the relation between the voltage and current:
,
.


Algebraic Equations:


\begin{table}[!ht]
\centering
\caption{Synchronous Machine Variables}\label{appedixtable1}
\begin{tabular}{|c|c|}
\hline
Variable&Description\\
\hline
&generator rotor angle\\
\hline
&generator rotor speed\\
\hline
&q-axis transient voltage\\
\hline
&d-axis transient voltage\\
\hline
&mechanical power\\
\hline
&initial mechanical power\\
\hline
&field voltage\\
\hline
&initial field voltae\\
\hline
&q-axis open circuit transient time constant\\
\hline
&d-axis open circuit transient time constant\\
\hline
&q-axis synchronous reactance\\
\hline
&q-axis transient reactance\\
\hline
&armature resistance\\
\hline
&mechanical starting time (2inertia constant)\\
\hline
&damping coefficient\\
\hline
&speed feedback gain\\
\hline
&active power feedback gain\\
\hline
&base frequency\\
\hline
&electrical power\\
\hline
\end{tabular}
\end{table}

\subsection{Automatic Voltage Regulator (AVR):}
Notations are in Table \ref{appedixtable2}.

Dynamic Equations:

where


and  is the ceiling function: .

Algebraic Equations:



\begin{table}[h]
\centering
\caption{Exciter Variables}\label{appedixtable2}
\begin{tabular}{|c|c|}
\hline
Variable&Description\\
\hline
&maximum regulator voltage\\
\hline
&minimum regulator voltage\\
\hline
&amplifier gain\\
\hline
&amplifier time constant\\
\hline
&stabilizer gain\\
\hline
&stabilizer time constant\\
\hline
&field circuit integral deviation\\
\hline
&field circuit time constant\\
\hline
&measurement time constant\\
\hline
& ceiling coefficient\\
\hline
& ceiling coefficient\\
\hline
&the reference voltage(or initial)\\
\hline
,,&state variables\\
\hline
\end{tabular}
\end{table}

\subsection{Turbine Governor (TG):}
Notations are in Table \ref{appedixtable4}.

Dynamic Equations:


where


Algebraic Equations:


\begin{table}[h]
\centering
\caption{Turbine Governor Variables}\label{appedixtable4}
\begin{tabular}{|c|c|}
\hline
Variable&Description\\
\hline
&reference speed\\
\hline
&droop\\
\hline
&maximum turbine output\\
\hline
&minimum turbine output\\
\hline
&governor time constant\\
\hline
&servo time constant\\
\hline
&transient gain time constant\\
\hline
&power fraction time constant\\
\hline
&reheat time constant\\
\hline
&state variables (i=1,2,3)\\
\hline
\end{tabular}
\end{table}

\subsection{Over Excitation Limiter (OXL):}
Notations are in table \ref{appedixtable5}.

Dynamic Equations:


Algebraic Equations:

with , . And the over excitation limiter starts to work after a fixed delay  regardless of the field current overload.

\begin{table}
\centering
\caption{Over Excitation Limiter Variables}\label{appedixtable5}
\begin{tabular}{|c|c|}
\hline
Variable&Description\\
\hline
&d-axis estimated generator reactance\\
\hline
&q-axis estimated generator reactance\\
\hline
&synchronous machine field current\\
\hline
&maximum field current\\
\hline
&the reference voltage of automatic voltage regulator\\
\hline
&integrator time constant\\
\hline
&active (or reactive) power of generator\\
\hline
&fixed time delay\\
\hline
&state variabe\\
\hline
\end{tabular}
\end{table}

\subsection{Exponential Recovery Load (ERL):}
Notations are in table \ref{appedixtable6}.

Dynamic Equations:

where  and  are the static and transient real power absorptions, similar definition for  and .  and  are PQ load power from power flow solutions. Besides,
, , , , , .

Algebraic Equations:


\begin{table}[h]
\centering
\caption{Exponential Recovery Load Variables}{\label{appedixtable6}}
\begin{tabular}{|c|c|}
\hline
Variable&Description\\
\hline
&active power percentage\\
\hline
&reactive power percentage\\
\hline
&active power time constant\\
\hline
&reactive power time constant\\
\hline
&static active power exponent\\
\hline
&dynamic active power exponent\\
\hline
&static reactive power exponent\\
\hline
&dynamic reactive power exponent\\
\hline
,&state variabels\\
\hline
\end{tabular}
\end{table}
\subsection{Load Tap Changer (LTC):}

The tapping delay are assumed to be independent of , but larger for first tap change than for the subsequent ones while without the inverse time characteristic. Refer to \cite{Cutsem:book} for more details.

\section{Detailed and Generic long-term stability models}
The detailed and generic long-term stability model are shown in Table \ref{detailed&generic}. Moreover, the detailed variables and their corresponding generic variables , ,  and  are also indicated.

\begin{table*}
\caption{long-term stability model and corresponding generic variables}
\begin{center}\normalsize
\begin{tabular}{|*{7}{l|}}
\hline
Detailed Long-Term Stability Model&Generic Long-Term Stability Model&Detailed Variables\\
\hline
TG: (\ref{Tg_d1})-(\ref{Tg_d2}), OXL: (\ref{OXL_d1}),&&slow continuous variables :\\
ERL: (\ref{ERL_d1})-(\ref{ERL_d2}).&&TG: , OXL: , ERL: ,.\\
\hline
LTC: (\ref{LTC_d1}).&&slow discrete variables : .\\
\hline
GEN: (\ref{Gen_d1})-(\ref{Gen_d2}),&&fast continuous variables :\\
AVR: (\ref{AVR_d1})-(\ref{AVR_d2}).&&GEN: ,,,, AVR: ,,,.\\
\hline
TG:(\ref{Tg_a1}), OXL:(\ref{OXL_a1})-(\ref{OXL_a2}), &&algebraic variables :\\
ERL:(\ref{ERL_a1})-(\ref{ERL_a2}), GEN:(\ref{Gen_a1}),&&TG: , OXL: , AVR: ,\\
AVR:(\ref{AVR_a1})-(\ref{AVR_a2}), power relations.&&GEN: ,,,, Bus:  and .\\
\hline
\end{tabular}
\end{center}
\label{detailed&generic}
\end{table*}

\section*{Acknowledgment}
The authors would like to thank Prof. Luis F. C. Alberto and Dr. Tao Wang for helpful discussions.
\ifCLASSOPTIONcaptionsoff
  \newpage
\fi


\begin{thebibliography}{1}

\bibitem{Chiang:book}
H. D. Chiang, {\em Direct Methods for Stability Analysis of Electric Power Systems-Theoretical Foundation, BCU Methodologies, and Applications}.
\newblock New Jersey: John Wiley \& Sons, Inc, 2011.

\bibitem{Jointaskforce:article}
P. Kundur, J. Paserba, V. Ajjarapu, G. Andersson, A. Bose, C. Canizares, N. Hatziargyriou, D. Hill, A. Stankovic, C. Taylor, T. Van Cutsem, V. Vittal, {\em Definition and Classification of Power System Stability}.
\newblock IEEE Transactions on Power Systems, vol. 19, no. 2, pp. 1387-1401, May 2004.

\bibitem{SauerPai:book}
P. W. Sauer, M. A. Pai, {\em Power System Dynamics and Stability}.
\newblock  Prentice-Hall, New Jersey, U.S.A. 1998.

\bibitem{Cutsem:book}
T. V. Cutsem, C. Vournas, {\em Voltage Stability of Electric Power Systems}.
\newblock Boston/London/Dordrecht: Kluwer Academic Publishers, 1998.

\bibitem{Cutsem:artical4}
L. Loud, P. Rousseaux, D. Lefebvre, T. V. Cutsem, {\em A Time-Scale Decomposition-Based Simulation Tool for Voltage Stability Analysis}.
\newblock In: Proceedings of the IEEE Power Tech Conference, vol. 2, Porto, Portugal, 2001.

\bibitem{Cutsem:artical}
T. V. Cutsem, Y. Jacquemart, J. N. Marquet, {\em A Comprehensive Analysis of Mid-term Voltage Stability}.
\newblock IEEE Transactions on Power Systems, vol. 10, no. 3, pp. 1173-1182, August 1995.

\bibitem{Cutsem:artical2}
T. V. Cutsem, M. E. Grenier, D. Lefebvre, {\em Combined Detailed and Quasi Steady-State Time Simulations for Large-disturbance Analysis}.
\newblock International Journal of Electrical Power and Energy Systems, vol. 28, issue 9, pp. 634-642, November 2006.

\bibitem{Cutsem:artical3}
P. Rousseaux, T. V. Cutsem, {\em Quasi Steady-State Simulation Diagnosis Using Newton Method with Optimal Multiplier}.
\newblock Power Engineering Society General Meeting, 2006.

\bibitem{Wang:artical}
Q. Wang, H. Song, V. Ajjarapu {\em Continuation-Based Quasi-Steady-State Analysis}.
\newblock IEEE Transactions on Power Systems, vol. 21, no. 1, February 2006.

\bibitem{Wangxz:article}
X. Z. Wang, H. D. Chiang, {\em Some issues with Quasi-Steady State Model in Long-term Stability}
\newblock appeared in IEEE PES general meeting 2013, paper number GM0652.

\bibitem{Veukatasubramanian:article}
V.  Veukatasubramanian, H.  Schattler  and  J.  Zaborszky, {\em Dynamics of Large Constrained Nonlinear Systems---A Taxonomy Theory}
\newblock Proc. IEEE, Vol. 83, pp. 1530-1561, November. 1995.

\bibitem{Alberto:article}
Luis F. C. Alberto, H. D. Chiang, {\em Theoretical Foundation of CUEP Method for Two-Time Scale Power System Models}.
\newblock Power and Energy Society General Meeting, 2009.

\bibitem{Khalil:book}
H. Khalil, {\em Nonlinear System, 3rd ed},
\newblock New York: Macmillan Publishing Company, 2002.

\bibitem{Lobry:article2}
C. Lobry, T. Sari, {\em Singular perturbations in control theory}.
\newblock Nonlinear Control and Applications, Hermann : Travaux en Cours 64, 2005b, pp. 151-177.

\bibitem{Lobry:article}
C. Lobry, T. Sari, S. Touhami, {\em On Tykhonov’s theorem for convergence of solutions of slow and fast systems}.
\newblock Electronic Journal of Differential Equations, vol. 1998, no. 19, pp. 1-22, 1998.

\bibitem{Chiang:article1988}
H. D. Chiang, M. W. Hirsch, and F. F. Wu, {\em Stability regions of nonlinear autonomous dynamical systems}.
\newblock IEEE Transactions on Automatic Control, Vol. 33, No. 1, pp. 16-27, Jananuary 1988.

\bibitem{Chiang:article1989}
H. D. Chiang, J. S. Thorp, {\em Stability regions of nonlinear dynamical systems: A constructive methodology}.
\newblock IEEE Transactions on Automatic Control, Vol. 34, No. 12, pp. 1229-1241, December 1989.

\bibitem{Milano:article}
F. Milano, {\em Power System Analysis Toolbox Documentation for PSAT version 2.1.5, November 1, 2009}.

\bibitem{testcasearchive}
R. Christie, "Power Systems Test Case Archive," Electrical Engineering
dept., University of Washington, Apr. 2000. [Online]. Available:
http://www.ee.washington.edu/research/pstca.

\end{thebibliography}

\begin{IEEEbiography}[{\includegraphics[width=1in,height=1.25in,clip,keepaspectratio,angle=0]{myphoto.eps}}]{Xiaozhe Wang}
received the B.S. degree in Information Science \& Electronic Engineering from Zhejiang University, Zhejiang, China, in 2010, M.Eng degree in Electrical and Computer Engineering from Cornell University, Ithaca, NY in 2011. She is currently pursuing the Ph.D. degree in the same school. She was a Fung scholar in 2008-2009, and a Jacobs Scholar in 2011-2012. Her research interests include nonlinear systems, power system stabilities, and optimization in power systems.
\end{IEEEbiography}

\begin{IEEEbiography}[{\includegraphics[width=1in,height=1.25in,clip,keepaspectratio,angle=0]{chiang1.eps}}]{Hsiao-Dong Chiang}
(M'87-SM'91-F'97) received the PhD degree in EECS from University of California, Berkeley in 1986 and is currently Professor of ECE at Cornell University, Ithaca, New York. His research effort is focused on theoretical developments and practical applications in the areas of nonlinear system stability and control theory, nonlinear computation, and nonlinear optimization methods. He and his colleagues have published more than three hundred and fifty referred papers and have been awarded 14 patents arising from their research and development work both in the United States and internationally. He was Associate Editor of the IEEE Trans. CAS from 1990 to 1991, and was Associate Editor for Express Letters of the IEEE Transactions on CAS Part I from 1993 to 1995. He is currently on the Editorial board of two international Journals. He is Author of the book "Direct Methods for Power System Stability Analysis: Theoretical Foundation, BCU Methodology and Applications", John Wiley \& Sons, 2010.
\end{IEEEbiography}

\end{document}
