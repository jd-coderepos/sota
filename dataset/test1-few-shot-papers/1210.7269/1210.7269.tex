\documentclass[a4paper, 11pt, oneside]{article}



\usepackage{fullpage}     
\usepackage{complexity}   
\usepackage{graphicx}     
\usepackage{amsmath,amsthm,amssymb}  
\usepackage{fancybox}     
\usepackage{enumerate}    
\usepackage{calc}         
\usepackage{xspace}       
\usepackage[pdftitle={The star and biclique coloring and choosability problems},pdfauthor={Marina Groshaus, Francisco J.\ Soulignac, Pablo Terlisky},pdfcreator={},pdfsubject={The star and biclique coloring and choosability problems},pdfkeywords={star-coloring, biclique-coloring, star-choosability, biclique-choosability, star-chromatic number, biclique-chromatic number},colorlinks=true,linkcolor=blue,citecolor=blue]{hyperref}  

{\theoremstyle{definition} \newtheorem{defn}{Definition}}
\newtheorem{theorem}{Theorem}
\newtheorem{observation}[theorem]{Observation}
\newtheorem{lemma}[theorem]{Lemma}
\newtheorem{corollary}[theorem]{Corollary}
\newtheorem{openproblem}{Problem}



\let\oldNP\NP\renewcommand{\NP}{\oldNP\xspace}

\newcommand{\bccol}[1]{\textsc{biclique -coloring}}
\newcommand{\stcol}[1]{\textsc{star -coloring}}
\newcommand{\stchose}[1]{\textsc{star -choosability}}
\newcommand{\bcchose}[1]{\textsc{biclique -choosability}}
\newcommand{\ptwop}{\ensuremath{\Pi^p_2}\xspace}
\newcommand{\stp}{\ensuremath{\Sigma^p_2}\xspace}
\newcommand{\ptp}{\ensuremath{\Pi^p_3}\xspace}
\newcommand{\qsat}[1]{\ensuremath{\textsc{qsat}_{#1}}}
\newcommand{\range}[3]{\ensuremath{#1 \in \{#2,\ldots,#3\}}}
\newcommand{\MC}{\ensuremath{\mathcal{C}}}
\newcommand{\MB}{\ensuremath{\mathcal{B}}}
\newcommand{\MS}{\ensuremath{\mathcal{S}}}

\newcommand{\naesat}{\textsc{nae-sat}\xspace}
\newcommand{\naesatt}{\textsc{naesat}\xspace}

\let\VP=\mathbf 
\def\VEC#1{\vec{\VP #1}}



\newenvironment{Problem}{\begin{Sbox}\begin{minipage}{\textwidth-2\parindent}\parskip=.1\baselineskip \vspace{.25\baselineskip}}{\vspace{.25\baselineskip}\end{minipage}\end{Sbox}\vspace{\baselineskip}\begin{center}\doublebox{\TheSbox}\end{center}\vspace{\baselineskip}}

\newcommand{\problemName}[1]{\noindent#1\vspace{.5\baselineskip}}
\newcommand{\InputTag}{\textsf{INPUT:}\ }
\newlength{\InputLength}\settowidth{\InputLength}{\InputTag}
\newcommand{\problemInput}[1]{\hangindent=\InputLength\InputTag #1}     
\newcommand{\QuestionTag}{\textsf{QUESTION:}\ }
\newlength{\QuestionLength}\settowidth{\QuestionLength}{\QuestionTag}
\newcommand{\problemQuestion}[1]{\hangindent=\QuestionLength\QuestionTag #1}

\newcommand{\problem}[3]{\begin{Problem}\problemName{#1}\par\problemInput{#2}\par\problemQuestion{#3}\end{Problem}}

\let\Definition=\emph

\def\imagenes{}



\title{The star and biclique coloring and choosability problems}
\author{Marina Groshaus\thanks{CONICET}~\thanks{Departamento de Computaci\'on, FCEN, Universidad de Buenos Aires, 
Buenos Aires, Argentina.} \and 
  Francisco J.\ Soulignac\footnotemark[1]~\footnotemark[2]~\thanks{Universidad Nacional de Quilmes, Buenos Aires, Argentina.} \and Pablo Terlisky\footnotemark[2]
}

\date{\normalsize\texttt{\{groshaus,fsoulign,terlisky\}@dc.uba.ar}}

\begin{document}
\maketitle
\begin{abstract}
  A biclique of a graph  is an induced complete bipartite graph.  A star of  is a biclique contained in the closed neighborhood of a vertex.  A star (biclique) -coloring of  is a -coloring of  that contains no monochromatic maximal stars (bicliques).  Similarly, for a list assignment  of , a star (biclique) -coloring is an -coloring of  in which no maximal star (biclique) is monochromatic.  If  admits a star (biclique) -coloring for every -list assignment , then  is said to be star (biclique) -choosable.  In this article we study the computational complexity of the star and biclique coloring and choosability problems.  Specifically, we prove that the star (biclique) -coloring and -choosability problems are -complete  and -complete for , respectively, even when the input graph contains no induced  or .   Then, we study all these problems in some related classes of graphs, including -free graphs for every  on three vertices, graphs with restricted diamonds, split graphs, threshold graphs, and net-free block graphs.

 \vspace*{.2\baselineskip} {\bf Keywords:} star coloring, biclique coloring, star choosability, biclique choosability.
\end{abstract}


\section{Introduction}

Coloring problems are among the most studied problems in algorithmic graph theory.  In its classical form, the -coloring problem asks if there is an assignment of  colors to the vertices of a graph in such a way that no edge is monochromatic.  Many generalizations and variations of the classical coloring problem have been defined over the years.  One of such variations is the clique -coloring problem, in which the vertices are colored so that no maximal clique is monochromatic.  In this article we study the star and biclique coloring problems, which are variations of the coloring problem similar to clique colorings.  A biclique is a set of vertices that induce a complete bipartite graph , while a star is a biclique inducing the graph .  In the star (biclique) -coloring problem, the goal is to color the vertices with  colors without generating monochromatic maximal stars (bicliques).

The clique coloring problem has been investigated for a long time, and it is still receiving a lot of attention.  Recently, the clique -coloring problem was proved to be \stp-complete~\cite{MarxTCS2011} for every , and it remains \stp-complete for  even when the input is restricted to graphs with no odd holes~\cite{DefossezJGT2009}.  The problem has been studied on many other classes of input graphs, for which it is was proved to be \NP-complete or to require polynomial time (e.g.~\cite{BacsoGravierGyarfasPreissmannSebHoSJDM2004,CerioliKorenchendler2009,DefossezJGT2006,GravierHoangMaffrayDM2003,KratochvilTuzaJA2002,MacedoMachadoFigueiredo2012}).  Due to the close relation between cliques and bicliques, many problems on cliques have been translated in terms of bicliques (e.g.~\cite{AmilhastreVilaremJanssenDAM1998,PrisnerC2000a,TuzaC1984}).  However, there are some classical problems on cliques whose biclique versions were not studied until recently~\cite{EguiaSoulignacDMTCS2012,GroshausMonteroJoGT2012,GroshausSzwarcfiterGC2007,GroshausSzwarcfiterJGT2010}.  Clique colorings are examples of such problems; research on biclique colorings begun in 2010 in the Master Thesis of one of the authors~\cite{Terlisky2010} whose unpublished results are being extended in the present article.  It is worth mentioning that, despite its youthfulness, at least two articles on biclique colorings were written: \cite{MacedoMachadoFigueiredo2012} develop a polynomial time algorithm for biclique coloring some unichord-free graphs, and \cite{MacedoDantasMachadoFigueiredo2012} determines the minimum number of colors required by biclique colorings of powers of paths and cycles.

The list coloring problem is a generalization of the coloring problem in which every vertex  is associated with a list , and the goal is to color each vertex  with an element of  in such a way that no edge is monochromatic.  Function  is called a list assignment, and it is a -list assignment when  for every vertex .  A graph  is said to be -choosable when it admits an -coloring with no monochromatic edges, for every -list assignment .  The choosability problem asks whether a graph is -choosable. In the same way as the coloring problem is generalized to the clique (star, biclique) coloring problem, the choosability problem is generalized to the clique (star, biclique) choosability problem.  That is, a graph  is clique (star, biclique) -choosable when it admits an -coloring generating no monochromatic maximal cliques (star, bicliques), for every -list assignment .  The choosability problems seem harder than their coloring versions, because a universal quantifier on the list must be checked.  This difficulty is reflected for the -choosability and clique -choosability problems in the facts that the former is \ptwop-complete for every ~\cite{GutnerTarsiDM2009}, whereas the latter is \ptp-complete for every ~\cite{MarxTCS2011}.  In~\cite{MoharvSkrekovskiEJC1999} it is proven that every planar graph is clique -choosable.  However, contrary to what happens with the clique coloring problem, there are not so many results regarding the complexity of the clique coloring problem for restricted classes of graphs.

In this paper we consider the star and biclique coloring and choosability problems, both for general graphs and for some restricted classes of graphs.  The star and biclique coloring and choosability problems are defined in Section~\ref{sec:preliminaries}, where we introduce the terminology that will be used throughout the article.  In Section~\ref{sec:general case}, we prove that the star -coloring problem is \stp-complete for , and that it remains \stp-complete even when its input is restricted to -free graphs.  Clearly, every maximal biclique of a -free graph is a star.  Thus, we obtain as a corollary that the biclique -coloring problem on -free graphs is \stp-complete as well.  The completeness proof follows some of the ideas by Marx~\cite{MarxTCS2011} for the clique coloring problem.  In Section~\ref{sec:choosability} we show that the star -choosability problem is \ptp-complete for , and that it remains \ptp-complete for -free graphs.  Again, the \ptp-completeness of the biclique -coloring problem on -free is obtained as a corollary.  As in~\cite{MarxTCS2011}, we require a structure to force a color on a vertex.  The remaining sections of the article study the star and biclique coloring problems on graphs with restricted inputs.  These graphs are related to the graph  that is generated to prove the \stp-completeness of the star coloring problem in Section~\ref{sec:general case}.  The aim is to understand what structural properties can help make the problem simpler.  In Section~\ref{sec:small forbiddens}, we discuss the star and biclique coloring and choosability problems on -free, -free, -free, and -free graphs.  For ,  and , the star coloring and star choosability problems are almost trivial and can be solved in linear time.  On the other hand, both problems are as hard as they can be for -free graphs, even when the input is a co-bipartite graph.  In Section~\ref{sec:diamond-free} we prove that the star coloring problem is \NP-complete for diamond-free graphs and that the star choosability problem is \ptwop-complete for a superclass of diamond-free graphs.  If no induced  is allowed for a fixed , then the biclique coloring and the biclique choosability problems are also \NP-complete and \ptwop-complete.  In Section~\ref{sec:split}, the star coloring and the star choosability problems on split graphs are proved to be \NP-complete and \ptwop-complete, respectively.  Finally, Sections \ref{sec:threshold}~and~\ref{sec:block} show that the star coloring and the star choosability problems are equivalent for both threshold and net-free block graphs, and both can be solved in linear time.  Table~\ref{tab:results} sums up the results obtained in the article for the star coloring and star choosability problems.

\begin{table}
 \centering
 \begin{tabular}{|l|c|c|}
 \hline
 Graph class                                   & star -coloring  & star -choosability  \\
 \hline
 -free                   & \stp-complete      & \ptp-complete         \\
 -free, -free, -free &            &               \\
 -free                         & \NP-complete       & \ptwop-complete       \\
 co-bipartite                                  &                    & \ptwop-complete       \\
 \{, gem, dart\}-free                     & \NP-complete       & \ptwop-complete       \\
 diamond-free                                  & \NP-complete       & \ptwop                \\
 -free                         & \NP-complete       & \ptwop-complete       \\
 split                                         & \NP-complete       & \ptwop-complete       \\
 threshold                                     &            &               \\
 net-free block                                &            &               \\
 \hline
 \end{tabular}
 \caption{Complexity results obtained in this article.}\label{tab:results}
\end{table}

\section{Preliminaries}
\label{sec:preliminaries}

In this paper we work with simple graphs.  The vertex and edge sets of a graph  are denoted by  and , respectively.  Write  to denote the edge of  formed by vertices .  For the sake of simplicity,  is also considered as the family of subsets of  containing the set  for each .  For , the \Definition{neighborhood} of  is the set  of vertices adjacent to , while the \Definition{closed neighborhood} of  is .  A vertex  \Definition{dominates} , and  is \Definition{dominated} by , when , while  \Definition{false dominates} , and  is \Definition{false dominated} by , when .  If , then  and  are \Definition{twins}, and if , then  and  are \Definition{false twins}.   The \Definition{degree} of  is .  A vertex is an \Definition{isolated vertex}, a \Definition{leaf}, and a \Definition{universal vertex} when  equals , , and , respectively.  We omit the subscripts from  and  when no ambiguities arise.

The \Definition{complement} of  is the graph  where  and .  For a graph , the \Definition{union} of  and  is the graph  where  and .  Write  to indicate that  and  are isomorphic.  The \Definition{-cycle} graph (), denoted by , is the connected graph that has  vertices of degree .  The \Definition{-path} graph, denoted by , is the graph obtained from  by removing an edge.  The \emph{-wheel} graph (), denoted by , is the graph obtained from  by inserting a universal vertex.  The \Definition{diamond}, \Definition{gem}, \Definition{dart}, and \Definition{net} are the graphs shown in Figure~\ref{fig:diam-dart-gem}.  The \Definition{-complete} graph, denoted by , is the graph formed by  pairwise adjacent vertices.  An \Definition{independent set} is a subset of  formed by pairwise non-adjacent vertices.  Graph  is \Definition{bipartite} when  can be partitioned into two independent sets  and . In this case, the unordered pair  is called a \Definition{bipartition} of .  The \Definition{-complete bipartite} graph (, ), denoted by , is the graph isomorphic to .  Note that  is a bipartite graph.  The graph  is also called the \Definition{-star} graph.  The universal vertices of  are referred to as the \Definition{centers} of , while  is said to be \Definition{centered} at .  Note that  has two centers.  

\begin{figure}
 \hfill\includegraphics{\imagenes diamond}\hfill\includegraphics{\imagenes dart}\hfill\includegraphics{\imagenes gem}\hfill{}\includegraphics{\imagenes net}\hfill{}
 \caption{The diamond, dart, gem, and net graphs are shown from left to right.}\label{fig:diam-dart-gem}
\end{figure}


Let .  We write  to denote the subgraph of  induced by , and  to denote .  Set  is said to be a \Definition{clique}, \Definition{biclique}, and \Definition{star} when  is isomorphic to a complete, bipartite complete, and star graph, respectively.  For the sake of simplicity, we use the terms \Definition{clique} and \Definition{biclique} to refer to  as well.  Moreover, we may refer to  as a biclique or star when  is a bipartite complete or star graph with bipartition . The family of maximal cliques, maximal bicliques, and maximal stars are denoted by , , and , respectively. 

A sequence of distinct vertices  is a \Definition{path} of \Definition{length}  when  is adjacent to .  If in addition  is adjacent to , then  is a \Definition{cycle} of \Definition{length} .  A \Definition{tree} is a connected graph that contains no cycles.  A \Definition{rooted tree} is a tree  with a fixed vertex  called the \Definition{root} of .  The \Definition{parent} of  in  is the neighbor of  in its path to .  A path (resp.\ cycle) of  is \Definition{chordless} when  (resp.\ ).  A \Definition{hole} is a chordless cycle of length at least .  A graph  is said to be \Definition{-free}, for some graph , when no induced subgraph of  is isomorphic to .  Similarly,  is -free, for any family of graph , when  is -free for every .  A graph is \Definition{chordal} when it is -free, i.e., chordal graphs have no holes.  

A \Definition{coloring} of  is a function  that maps each vertex  to a color .  When  for every ,  is called a \Definition{-coloring}.  We define  for any .  Set  is said to be \Definition{-monochromatic} when .  When there is no ambiguity, we say that  \Definition{monochromatic} instead of -monochromatic.  For a family  of subsets of , we say that  is a \Definition{proper coloring} of  if no  is monochromatic.  Four kinds of families are considered in this article.  A coloring  is a \Definition{vertex}, \Definition{clique}, \Definition{biclique}, and \Definition{star coloring} when  is a proper coloring of , , , and , respectively.  The problems of finding a proper coloring for these families are defined as follows.

\begin{Problem}
  \problemName{\textsc{Vertex (resp.\ clique, biclique, star) -coloring}}

  \problemInput{A connected graph  and a value .}

  \problemQuestion{Is there a vertex (resp.\ clique, biclique, star) -coloring of ?}
\end{Problem}

List colorings are a generalization of colorings. A \Definition{list assignment} of  is a function that maps each vertex  to a set .  When  for every ,  is called a \Definition{-list assignment}.  An -coloring of  is a coloring  such that  for every .  Define  for any .  Given family  of subset of  and a number , graph  is said to be \Definition{-choosable} with respect to  when there exists a proper -coloring of  for every -list assignment  of .  Graph  is \Definition{vertex}, \Definition{clique}, \Definition{biclique}, and \Definition{star -choosable} when  is -choosable with respect to , , , and , respectively.  The problem of determining if  is -choosable is defined as follows.

\begin{Problem}
  \problemName{\textsc{Vertex (resp.\ clique, biclique, star) -choosability}}

  \problemInput{A connected graph  and a value .}

  \problemQuestion{Is  vertex (resp.\ clique, biclique, star) -choosable?}
\end{Problem}

The vertex (resp.\ clique, biclique, star) \Definition{chromatic number}, denoted by  (resp.\ , , and ), is the minimum  such that  admits a vertex (resp.\ clique, biclique, star coloring) -coloring. Similarly, the vertex (clique, biclique, star) \Definition{choice number}, denoted by  (resp.\ , , ) is the minimum number  such that  is vertex (resp.\ clique, biclique, star coloring) -choosable.  By definition,  and  for .  

For a function  with domain , the \Definition{restriction} of  to  is the function  with domain  where  for .  In such case,  is said to be an \Definition{extension} of  to .  A \Definition{leafed vertex} is a vertex adjacent to a leaf.  For the sake of simplicity, whenever we state that  \emph{contains a leafed vertex} , we mean that  contains  and a leaf adjacent to .  It is well known that  for every induced subgraph  of .  Such a property is false for clique, biclique, and star colorings.  In particular, for any graph , it is possible to build a graph  such that  and  contains  as an induced subgraph.  For , graph  is built from  by iteratively inserting a twin of each vertex of .  Similarly,  is obtained by inserting false twins for , while, by the next observation,  is obtained by inserting a leaf adjacent to each vertex for .  

\begin{observation}\label{obs:leafed vertex}
 Let  be a graph with a list assignment ,  be a leafed vertex of , and  be a leaf adjacent to .  Then, any -coloring of  can be extended into an -coloring of  in such a way that there is no monochromatic maximal star with center in .
\end{observation}

A \Definition{block} is a maximal set of twin vertices.  If  and  are twin vertices, then  is both a maximal star and a maximal biclique, and thus  and  have different colors in any star or biclique -coloring .  Consequently,  for any block  of .  We record this fact in the following observation.

\begin{observation}\label{obs:block coloring}
 Let  be a graph with a list assignment , and  be twin vertices.  Then,  for any star or biclique -coloring  of .
\end{observation}

\section{Complexity of star and biclique coloring}
\label{sec:general case}

In this section we establish the hardness of the star and biclique coloring problems by showing that both problems are \stp-complete.  The main result of this section is that \stcol{k} is \stp-complete for every , even when its input is restricted to -free graphs.  Since all the bicliques of a -free graph are stars, this immediately implies that \bccol{k} is also \stp-complete for -free graphs.  The hardness results is obtained by reducing instances of the \qsat{2} problem.  The \qsat{h} problem is known to be -complete for every ~\cite{Papadimitriou1994}, and is defined as follows.

\problem{\textbf{Quantified -satisfiability with  alternations} (\qsat{h})}
{A formula  that is in -CNF if  is odd, while it is in -DNF if  is even.}
{Is  true? (.)}

Recall that  is in -CNF if it is a conjunction of clauses where each clause is a disjunction with three literals.  Similarly,  is in -DNF when it is a disjunction of clauses, each clause being a conjunction with three literals.

\subsection{Keepers, switchers, and clusters}

In this section we introduce the keeper, switcher, and cluster connections that are required for the reductions.  The keeper connections are used to force the same color on a pair of vertices, in any star coloring.  Conversely, the switcher connections force some vertices to have different colors.  Finally, the cluster connections are used to represent the variables of a DNF formula.  We begin defining the keeper connections.

\begin{defn}[-keeper]\label{def:k-keeper connection}
  Let  be a graph and .  Say that  is a \Definition{-keeper connecting } () when  can be partitioned into a clique  and  cliques  with  vertices each in such a way that  and  are cliques for \range{i}{1}{k-1}, and there are no more edges incident to vertices in .
\end{defn}

Figure~\ref{fig:keeper-switcher}~(a) shows a -keeper connecting two vertices  and . The main properties of -keepers are summarized in the following lemma.

\begin{lemma}\label{lem:keeper properties}
 Let  be a graph and  be a -keeper connecting  (). Then,
\begin{enumerate}[(i)]
  \item no induced hole or  contains a vertex in ,\label{lem:keeper properties:forbidden}
  \item  and  have the same color in any star -coloring of , and\label{lem:keeper properties:colors}
  \item Any -coloring  of  in which  can be extended into a -coloring of  in such a way that no monochromatic maximal star contains a vertex in .\label{lem:keeper properties:extension}
\end{enumerate}
\end{lemma}

\begin{proof}
Let , \ldots,  and  be as in Definition~\ref{def:k-keeper connection}.  (\ref{lem:keeper properties:forbidden}) is trivial.  

(\ref{lem:keeper properties:colors}) Let  be a star -coloring of  and fix .  Since  is a block of size , it contains vertices  with colors , , and , respectively, by Observation~\ref{obs:block coloring}.  Hence,  since otherwise  would be a monochromatic maximal star, and, similarly,  because  is not monochromatic. In other words,  and .  Replacing  and  in the reasoning above, we can conclude that  as well.  Therefore, .

(\ref{lem:keeper properties:extension}) To extend , set , and  for every \range{i}{1}{k-1}.  It is not hard to see that no maximal star with a vertex in  is monochromatic.
\end{proof}

\begin{figure}
 \centering
 \begin{tabular}{ccc}
 \includegraphics{\imagenes k-keeper} & \includegraphics{\imagenes k-switcher} & \includegraphics{\imagenes long-k-switcher}\\
 (a) & (b) & (c)
 \end{tabular}
 \caption{(a) A -keeper connecting .  (b) A -switcher connecting . (c) A long -switcher connecting .  In every figure, gray nodes represent leafed vertices, circular shapes represent cliques, and double lines represent a -keeper connecting two vertices.}\label{fig:keeper-switcher}
\end{figure}


The \Definition{switcher connection}, whose purpose is to force a set of vertices to have at least two colors, is now defined (see Figure~\ref{fig:keeper-switcher}~(b)).

\begin{defn}[-switcher]\label{def:switcher}
  Let  be a graph and  be an independent set of  with .  Say that  is a \Definition{-switcher connecting } () when ,  is a clique for \range{i}{1}{h}, and there are no more edges incident to vertices in .
\end{defn}

The following result is the analogous of Lemma~\ref{lem:keeper properties} for switchers.

\begin{lemma}\label{lem:switcher properties}
 Let  be a graph and  be a -switcher connecting  (). Then,
\begin{enumerate}[(i)]
  \item  for any star -coloring  of , and \label{lem:switcher properties:colors}
  \item Any -coloring  of  in which  can be extended into a -coloring of  in such a way that no monochromatic maximal star has its center in .\label{lem:switcher properties:extension}
\end{enumerate}
\end{lemma}

As defined, switchers are not useful for proving the hardness of the star coloring problem for -free graphs.  The reason is that their connected vertices must have no common neighbors to avoid induced 's, and our proof requires vertices with different colors and common neighbors.  To solve this problem we extend switchers into long switchers by combining them with keepers (see Figure~\ref{fig:keeper-switcher}~(c)).  We emphasize that the set of vertices connected by a long switcher need not be an independent set.

\begin{defn}[long -switcher]\label{def:long switcher}
  Let  be a graph, and  be a set of vertices of  with .  Say that  is a \Definition{long -switcher connecting } () when  can be partitioned into an independent set of leafed vertices , a -switcher , and -keepers  in such a way that  connects ,  connects  for \range{i}{1}{h}, and there are no more edges adjacent to vertices in .
\end{defn}

The analogous of Lemma~\ref{lem:switcher properties} follows; its proof is a direct consequence of Observation~\ref{obs:leafed vertex} and Lemmas \ref{lem:keeper properties}~and~\ref{lem:switcher properties}.

\begin{lemma}\label{lem:long switcher properties}
 Let  be a graph and  be a long -switcher connecting  (). Then,
\begin{enumerate}[(i)]
  \item no induced  or  of  contains a vertex of ,
  \item  for any star -coloring  of , and \label{lem:long switcher properties:colors}
  \item Any -coloring  of  in which  can be extended into a -coloring of  in such a way that no monochromatic maximal star has its center in .\label{lem:long switcher properties:extension}
\end{enumerate}
\end{lemma}

The last type of connection that we require is the cluster, which is used to represent variables of DNF formulas.  As switchers, clusters are used to connect several vertices at the same time.  Specifically, clusters require two sets , , and a vertex .  The purpose of the connection is to encode all the valuations of the variables in a formula  by using monochromatic stars with center in .  Thus, if a variable  is being represented by  and , then each monochromatic maximal star with center in  contains all the vertices in  and none of , when  is true, or it contains all the vertices of  and none of , when  is false.

\begin{defn}[-cluster]\label{def:cluster}
  Let  be a graph with vertices , , and  ().  Say that  is an \Definition{-cluster connecting } when  has  leafed vertices , vertex  is adjacent to all the vertices in , sequence  is a hole, and there are no more edges incident to vertices in .  For \range{i}{1}{\ell}, we write  and  to refer to  and , respectively.  
\end{defn}

Note that if  is an -cluster connecting , then the subgraph induced by  is isomorphic to a -wheel that has  as its universal vertex.  As mentioned, the main property of clusters is that its members can be colored in such a way that monochromatic stars represent valuations.

\begin{lemma}\label{lem:cluster properties}
 Let  be a graph and  be an -cluster connecting  for .  Then, 
 \begin{enumerate}[(i)]
  \item if  for every , then no induced  or  of  contains a vertex in ,
  \item if  is a maximal star of  with , then  equals either  or , and \label{lem:cluster properties:stars}
  \item any -coloring  of  () can be extended into a -coloring of  in such a way that , and no monochromatic maximal star of  has its center in . \label{lem:cluster properties:extension}
 \end{enumerate}
\end{lemma}

\begin{proof}
  Let , , and  be as in Definition~\ref{def:cluster}.
  
  (i) For \range{i}{1}{\ell}, the non-leaf neighbors of  are .  Since  and  are not adjacent and , it follows that  belongs to no induced  or  of .  
  
  (ii) Suppose .  If  for some \range{i}{1}{\ell}, then  because .  Then, since , it follows that  belongs to .  Consequently, by induction on , we obtain that  equals either  or .  In the former case,  because every vertex in  has a neighbor in .  In the latter case,  because  and at least one of  belongs to  for every \range{i}{1}{\ell}.
  
  (iii)  Just extend  so that , and color the leaves according to Observation~\ref{obs:leafed vertex}.
\end{proof}

\subsection{Hardness of the star-coloring problem}
\label{sec:general case:2-stp}

It is well known that a problem  is \stp when the problem of authenticating a positive certificate of  is ~\cite{Papadimitriou1994}.  For the star -coloring problem, a -coloring of  can be taken as a positive certificate.  Since it is  to authenticate that a -coloring of  is indeed a star coloring, we obtain that \stcol{k} is \stp.  The following theorem states the \stp-hardness of the problem.  

\begin{theorem}\label{thm:stcol-2}
 \stcol{k} is \stp-complete and it remains \stp-complete even when its input is restricted to -free graphs.
\end{theorem}

\begin{proof}
  We already know that  belongs to \stp.  To prove its hardness, we show a polynomial time reduction from \qsat{2}.  That is, for any -DNF formula  with  clauses , and  variables , , we build a graph  that admits a star -coloring if and only if  is true.  For the sake of simplicity, in this proof we use , , , and  as indices that refer to values in , , , and .

  Graph  can be divided into connection, inner, and leaf vertices.  \Definition{Connection} vertices are in turn divided into a set of \Definition{clause} vertices , a set of \Definition{-vertices} , two sets of \Definition{-literal} vertices  and  with  vertices each (for each ), two sets of \Definition{-literal} vertices  and  with  vertices each (for each ), a set of \Definition{color} vertices , and two special vertices  and . Let , and .  \Definition{Inner} vertices are those vertices included in switchers, keepers, and clusters of .  Inner vertices and the edges between the connection vertices are given by the next rules.
\begin{description}
  \item [Edges:]   is adjacent to all the vertices in , and if  (resp.\ , , ) is a literal of , then  is adjacent  (resp.\ , , ).  

  \item [Keepers:] there is a -keeper connecting  with .

  \item [Long switchers:] there are long -switchers connecting  and  (for every ),  (for every ),  for every ,  for every  and every connection vertex , , and .

  \item [Variables:] there are -clusters connecting  and .
\end{description}
Finally, each connection vertex other than  is leafed.  This ends up the construction of , which can be easily computed from  in polynomial time. Figure~\ref{fig:coloring2} depicts a schema of the graph.

\begin{figure}
  \centering
  \includegraphics{\imagenes complexity-1-v2a}
  \caption{Schema of the graph obtained from  in Theorem~\ref{thm:stcol-2}.  For the sake of simplicity, we omit  and the edges from  to .  Circular shapes drawn with dashes represent independent sets; marked edges between two vertices represent -switchers connecting them; circular shapes with marks represent sets of vertices pairwise connected by -switchers; and squares represent sets of vertices.}\label{fig:coloring2}
\end{figure}

  Before dealing with the star -coloring problem on , we show that  is -free.  By statement~(\ref{lem:keeper properties:forbidden}) of Lemmas \ref{lem:keeper properties}, \ref{lem:long switcher properties}~and~\ref{lem:cluster properties}, it suffices to prove that the subgraph  induced by the connection vertices is -free.  For this, observe that any induced  or  must contain a vertex of  because  is an induced star of  and , , and  have degree at most  in .   Now,  has at most three neighbors in , namely , , and maybe .  Hence, since  when  is adjacent to , we obtain that  belongs to no induced  nor .  A similar analysis is enough to conclude no vertex of  belongs to an induced  nor , thus  is -free.

  Now we complete the proof by showing that  is true if and only if  admits a star -coloring.  Suppose first that  is true for some valuation , and define  as the -coloring of  that is obtained by the following two steps.  First, set , , , , and .  Next, iteratively set  for the leaves and inner vertices according to Observation~\ref{obs:leafed vertex}, and statement~(\ref{lem:keeper properties:extension}) of Lemmas~\ref{lem:keeper properties}, \ref{lem:long switcher properties},~and~\ref{lem:cluster properties}.  Observe that the second step is well defined because every pair of vertices connected by -keepers have the same color, while every pair of vertices connected by long -switchers have different colors.

  We claim that  is a star -coloring of .  To see why, consider a maximal star  of  and observe that  cannot be a leaf.  If , then  is not monochromatic by Observation~\ref{obs:leafed vertex} and Lemmas~\ref{lem:keeper properties}, \ref{lem:long switcher properties}~and~\ref{lem:cluster properties}.  Suppose, then, that  and, moreover, that  is monochromatic.  Then, no -cluster intersects  by statement (\ref{lem:cluster properties:extension}) of Lemma~\ref{lem:cluster properties}.  Consequently, by statement (\ref{lem:cluster properties:stars}) of Lemma~\ref{lem:cluster properties},  equals either  or , while  equals either  or  for every  and every .  Extend  to include  in its domain, so that  if and only if .  By hypothesis, , thus there is some clause  whose literals are all true according to .  If  has some neighbor in , then , thus  because .  Similarly, if  has some neighbor in  (resp.\ , ), then  (resp.\ , ).  Therefore, since  is an independent set and , we obtain that , thus  is not monochromatic.

  For the converse, let  be a star -coloring of .  Since there is are long -switchers connecting  and  for every  and every connection vertex , we obtain that  and  by statement~(\ref{lem:long switcher properties:colors}) of Lemma~\ref{lem:long switcher properties}. Define  as any valuation in which  if and only if .  Since  can take any value from , it is enough to prove that .  Let  and .

  Let .  By construction,  is an independent set, thus  is a star of .  Recall that there are long -switchers connecting  and .  Hence  and  by statement~(\ref{lem:long switcher properties:colors}) of Lemma~\ref{lem:long switcher properties}.  This implies that .  Similarly, there are long -switchers connecting  for each , thus  as well.  Finally, using Lemma~\ref{lem:keeper properties}, we obtain that  because there is a long -switcher connecting  and a -keeper connecting .  So, by hypothesis,  is not a maximal star, which implies that  is also an independent set for some .

  Since  and  have the same neighbors in the -keeper  connecting them, it follows that .  Similarly, all the vertices in a cluster are adjacent to at least one vertex of .  Finally, each vertex of  either belongs or has some neighbor in .  Consequently, , i.e.,  represents some clause .  If  is a literal of , then  is adjacent to .  Hence, since  is an independent set, it follows that .  By the way  is defined, this means that .  Similar arguments can be used to conclude that if  is a literal of , then .  That is,  is satisfied by , thus  is true.
\end{proof}

\subsection{Graphs with no short holes and small forbidden subgraphs}
\label{sec:chordal}

Note that every hole  of the graph  defined in Theorem~\ref{thm:stcol-2} either 1. contains an edge  for vertices  connected by a -keeper, or 2.\ contains a path  for a vertex  in a cluster  connecting  with  and .  A slight modification of  can be used in the proof of Theorem~\ref{thm:stcol-2} so as to enlarge the hole .  In case 1.,  can be subdivided by inserting a vertex  in such a way that  and  are connected by -keepers.  Similarly, in case 2., dummy vertices not adjacent to any  can be inserted into  and  so as to increase the distance between  and  in .  Neither of these modifications generates a new hole in .  Thus, in Theorem~\ref{thm:stcol-2} we can use a iterative modification of  whose induced holes have length at least .  The following corollary is then obtained.

\begin{corollary}
 For every , \stcol{k} is \stp-complete when the input is restricted to -free graphs whose induced holes have length at least .
\end{corollary}

An interesting open question is what happens when  grows to infinity, i.e., what is the complexity of star -coloring a chordal graph or a chordal -free graph.  In following sections we consider the star coloring and star choosability problems in some subclasses of chordal graphs, namely split, threshold, and block graphs.

Theorem~\ref{thm:stcol-2} also shows that the star -coloring problem is hard for -free graphs, which is a class of graphs defined by forbidding two small subgraphs.  Thus, another interesting question posed by Theorem~\ref{thm:stcol-2} is what happens when other small graphs are forbidden, so as to understand what structural properties can simplify the problem.  Following sections discuss the coloring problems from this perspective as well.  In particular, we study the problem for: every -free graphs where  has three vertices; a superclass of diamond-free graphs; and split and threshold graphs.  Before dealing with this restricted versions, we establish the complexity of the star -choosability problem for -free graphs.

\section{Complexity of the choosability problems}
\label{sec:choosability}
 
In this section we deal with the list version of the star and biclique-coloring problems.  The goal is to show that \stchose{k} and \bcchose{k} are \ptp-complete problems even when their inputs are restricted to -free graphs.  Again, only one proof is required because the star and biclique-choosability problems coincide for -free graphs.  In This opportunity, however, the proof is by induction on .  That is, we first conclude that the \stchose{2} problem is \ptp-complete with a polynomial-time reduction from the \qsat{3}, and next we show that \stchose{k} can be reduced in polynomial time to the \stchose{(k+1)}.  The proof for  is similar to the proof of Theorem~\ref{thm:stcol-2}; however, we did not find an easy way to generalize it for  because long switchers generate graphs that are not star -choosable.

\subsection{Keepers, switchers, clusters and forcers}

For the case  we require the keeper and cluster connections once again, and a new version of the long switcher.  We begin reviewing the main properties of keepers, switchers, and clusters with respect to the star choosability problem.

\begin{lemma}\label{lem:keeper choosability}
 Let  be a graph,  be a -list assignment of , and  be a -keeper connecting  ().  Then, any -coloring  of  can be extended into an -coloring of  in such a way that no monochromatic maximal star contains a vertex in .
\end{lemma}

\begin{proof}
 Let  and  be the vertices of  as in Definition~\ref{def:k-keeper connection}.  Extend  into an -coloring of  such that ,  for every \range{i}{1}{k-1}, and .  Since  is a -list assignment, such an extension can always be obtained.  Furthermore, no monochromatic maximal star has a vertex in .
\end{proof}

\begin{lemma}\label{lem:switcher choosability}
 Let  be a graph,  be a -list assignment of , and  be a -switcher connecting  ().  Then, any -coloring  of  in which  can be extended into an -coloring of  in such a way that no monochromatic maximal star has its center in .
\end{lemma}

By the previous lemma, if  is a -switcher connecting , then  is star -choosable.  This property does not hold for long switchers because it is no longer true that a -keeper connects vertices of the same color.  So, to avoid induced 's, we need a new version of the long switcher.  This new switcher is defined only for -colorings, and it is star -choosable as desired.  We refer to this switcher as the \Definition{list switcher}.  In short, the difference between the long -switcher and the list switcher is that the latter has no leafed vertices and -keepers are replaced by edges (see Figure~\ref{fig:switcher-forcer}~(a)).  Its definition is as follows.

\begin{defn}[list switcher]\label{def:list switcher}
  Let  be a graph, and  be a set of vertices of  with .  Say that  is a \Definition{list switcher connecting } when  can be partitioned into an independent set  and a -switcher  in such a way that  connects ,  for \range{i}{1}{h}, and there are no more edges adjacent to vertices in .
\end{defn}

The following lemma is equivalent to Lemma~\ref{lem:long switcher properties} for list switchers.

\begin{lemma}\label{lem:list switcher properties}
 Let  be a graph and  be a list switcher connecting .  Then,
\begin{enumerate}[(i)]
  \item no induced  or  of  contains a vertex of ,
  \item  for any star -coloring  of , and \label{lem:list switcher properties:colors}
  \item for any -list assignment  of , every -coloring  of  in which  can be extended into an -coloring of  in such a way that no monochromatic maximal star contains a vertex in .\label{lem:list switcher properties:extension}
\end{enumerate}
\end{lemma}

\begin{proof}
 We only prove (\ref{lem:list switcher properties:extension}).  Let , and  be as in Definition~\ref{def:list switcher}, and  be the -switcher connecting .  Suppose, without loss of generality, that , and observe that either  or  or .  In this setting, extend  to include  and  in such a way that , , and .  Following, extend  into an -coloring of  such that , , and  for \range{i}{3}{h}.  It is not hard to see that no monochromatic maximal star contains a vertex in .
\end{proof}

Finally, the proof of statement (\ref{lem:cluster properties:extension}) of Lemma~\ref{lem:cluster properties} implies the following lemma.

\begin{lemma}\label{lem:cluster choosability}
  Let  be a graph,  be a -list assignment of , and  be an -cluster connecting  for .  Then, any -coloring  of  can be extended into an -coloring of  in such a way that , and no monochromatic maximal star of  has its center in .
\end{lemma}


\begin{figure}
 \centering
 \begin{tabular}{c@{\hspace{1cm}}c}
  \includegraphics{\imagenes list-switcher} & \includegraphics{\imagenes 2-forcer} \\
  (a) & (b) 
 \end{tabular}
 \caption{(a) A list switcher connecting . (b) A -forcer connecting  with a -list assignment ; note that  is the unique -admissible color for .}\label{fig:switcher-forcer}
\end{figure}


Besides the -keepers, list switchers, and clusters, we use forth kind of connection that can be used to force the color of a given vertex when an appropriate list assignment is chosen.  This connection is called the \emph{forcer} and, contrary to the other connections, it connects only one vertex.

\begin{defn}[-forcer]\label{def:k-forcer connection}
  Let  be a graph and .  Say that  is a \emph{-forcer connecting } () when  can be partitioned into sets of leafed vertices  and , and cliques  for  and  in such a way that , ,  is a -switcher connecting , and there are no more edges incident to vertices in .
\end{defn}

Let  be a -list assignment of  and  be a -forcer connecting .  We say that  is \emph{-admissible for } when there is an -coloring  of  in which  and no monochromatic maximal star has its center in .  Clearly, if  is -admissible for , then any -coloring  of  in which  can be extended into an -coloring of  in such a way that no monochromatic maximal star has its center in .  The main properties of forcers are summed up in the following lemma (see Figure~\ref{fig:switcher-forcer}~(b)).

\begin{lemma}\label{lem:forcer properties}
  Let  be a graph and  be a -forcer connecting  ().  Then, 
  \begin{enumerate}[(i)]
    \item no induced  or  of  contains a vertex in ,\label{lem:forcer properties:forbidden}
    \item for every -list assignment  of  there is an -admissible color for , and\label{lem:forcer properties:extension}
    \item every -list assignment  of  can be extended into a -list assignment of  in which  has a unique -admissible color.\label{lem:forcer properties:color}
  \end{enumerate}
\end{lemma}

\begin{proof}
 Let , , and  be as in Definition~\ref{def:k-forcer connection}, and define .  Statement (\ref{lem:forcer properties:forbidden}) follows from the fact that no pair of vertices in  have a common neighbor.
 
 (\ref{lem:forcer properties:extension}) Let  be the complete bipartite graph with bipartition , and  be a -list assignment of  where  for every .  In~\cite{MarxTCS2011} it is proven that  admits a vertex -coloring .  Define  as any -coloring of  in which  for , where -switchers and leaves are colored according to Lemma~\ref{lem:switcher choosability} and Observation~\ref{obs:leafed vertex}, respectively.  The coloring of the -switchers is possible because  is a vertex -coloring.  By Observation~\ref{obs:leafed vertex}, no vertex in  is the center of a maximal star, while by Lemma~\ref{lem:switcher choosability}, no vertex in  is the center of a maximal monochromatic star for  and .  That is,  is -admissible for .
 
 (\ref{lem:forcer properties:color})  Extend  into a -list assignment of  such that 1.\  for every pair of vertices , 2.\  is a family of different subsets included in  such that  for every  and , and 3.\  for every , and .  Define  and  as in statement (\ref{lem:forcer properties:extension}).  By statement~(\ref{lem:forcer properties:extension}), there is an -coloring  of  that contains no monochromatic maximal star with center in .  Since  is a block of , it follows that  by Observation~\ref{obs:block coloring}.  Then, since no maximal star with center in  is monochromatic, it follows that  for every , .  Thus, if  is the coloring such that  for every , then  is a vertex -coloring of .  Consequently, as proven in~\cite{MarxTCS2011},  is the unique color of  that belongs to the subset of .
 \end{proof}

\subsection{Hardness of the star choosability problem}

A problem  is \ptp when the problem of authenticating a negative certificate of  is \stp~\cite{Papadimitriou1994}.  For the star -choosability problem, a -list assignment of  can be taken as the negative certificate.  Using arguments similar to those in Section~\ref{sec:general case:2-stp} for star -colorings, it is not hard to see that it is a \stp problem to authenticate whether a graph  admits no -colorings for a given -list assignment .  Therefore, \stchose{k} is \ptp.  In this section we establish the hardness of \stchose{k}.  For  we reduce the complement of an instance of \qsat{3} into an instance of \stchose{2}.  Then, we proceed by induction showing how to reduce an instance of \stchose{k} into an instance of \stchose{(k+1)} for every .  

The proof for the case  is, in some sense, an extension of Theorem~\ref{thm:stcol-2}.  The goal is to force the true literals of  variables to have the same color as , so that a monochromatic maximal star centered at  appears when the formula is false.  

\begin{theorem}\label{thm:stchose-2}
 \stchose{2} is \ptp-hard, and it remains \ptp-hard even when its input is restricted to -free graphs.
\end{theorem}

\begin{proof}
  The hardness of \stchose{2} is obtained by reducing the complement of \qsat{3}.  That is, given a -DNF formula  with  clauses , and  variables , , , we build a graph  that is -list-choosable if and only if  is true.  For the sake of simplicity, in this proof we use , , , and  to refer to values in , , , and , respectively.  
  
  Graph  is similar to the graph in Theorem~\ref{thm:stcol-2}.  Its vertex set is again divided into connection, inner, and leaf vertices. In turn, \emph{connection} vertices are divided into a set , a set , sets , , , , , and  with  vertices each, and two vertices .  Let , , and .
  
  \emph{Inner} vertices are those vertices included in -keepers, list switcher, clusters and -forcers.  The following rules define inner vertices and the edges between connection vertices.  \textbf{Edges:}  is adjacent to all the vertices in , and if  (resp.\ , , , , ) is a literal of , then  is adjacent to  (resp.\ , , , , ).  \textbf{Keepers:}  and  are connected by a -keeper.  \textbf{List switchers:}  there are list switchers connecting \{, \} and \{, \} (for every ),  (for every ), and  (for every ).  \textbf{Clusters:}  there are -clusters connecting , , and . \textbf{Forcers:} there are -forcers connecting each vertex of  (for every ) and each vertex of .

  Finally, every connection vertex other than  is leafed.  This ends up the construction of  (see Figure~\ref{fig:choosing}), which can be easily computed from  in polynomial time.   Arguments similar to those in Theorem~\ref{thm:stcol-2} are enough to conclude that  is -free.  

\begin{figure}
  \centering
  \includegraphics{\imagenes complexity-2}
  \caption{Schema of the graph obtained from  in Theorem~\ref{thm:stchose-2}; for the sake of simplicity, we omit the edges from  to .  Square vertices represent vertices connected to a -forcer.}\label{fig:choosing}
\end{figure}

  We now show that  is true if and only if  is -choosable.  We first show that if  is true, then  admits a star -coloring  for any -list assignment .  The -coloring  is obtained by executing the following algorithm.
  \begin{description}
    \item[Step 1:] For every  connected to a forcer, let  be -admissible for .  Such a color always exists by statement~(\ref{lem:forcer properties:extension}) of Lemma~\ref{lem:forcer properties}.  Suppose, w.l.o.g., that  and let  be a valuation of  such that  if and only if .  
    
    \item[Step 2:] By hypothesis,  can be extended to include  so that  is true.  If  or  for some , then:
    \begin{description}
      \item [Step 2.1:] Let  in such a way that  if and only if  and , while  if and only if  and .
      \item [Step 2.2:] Let .
      \item [Step 2.3:] Let  and  for every .
    \end{description}
    If , then:
    \begin{description}
      \item [Step 2.4:] Let  if and only if  and , and .
    \end{description}
    Note that, whichever case gets executed,  and  for every .
    
    \item[Step 3:] Let  and .  
    
    \item[Step 4:] Let  for , leaves, and inner vertices be as in Observation~\ref{obs:leafed vertex}, and Lemmas~\ref{lem:keeper choosability}, \ref{lem:list switcher properties}, \ref{lem:cluster choosability}~and~\ref{lem:forcer properties}.  Observe that this is always possible.  In particular, observe that every pair of vertices connected by a list switcher have different colors, while every vertex connected to a forcer has an -admissible color.
  \end{description}

  We claim that  is a star -coloring of .  Let  be any maximal star of .  By Observation~\ref{obs:leafed vertex} and Lemmas~\ref{lem:keeper choosability}, \ref{lem:list switcher properties},  \ref{lem:cluster choosability}~and~\ref{lem:forcer properties},  is not monochromatic when .  Suppose, for the rest of the proof, that  and  is monochromatic.  By Lemma~\ref{lem:keeper choosability}, this implies that , thus .  Also, by Lemma~\ref{lem:cluster choosability},  intersects no -cluster, thus .  Moreover, by statement~(\ref{lem:cluster properties:stars}) of Lemma~\ref{lem:cluster properties},  equals either  or ,  equals either  or , and  is either  or , for every , , and .  Extend  to  so that  if and only if .  By hypothesis, , thus there is some clause  whose literals are all true according to .  If  has some neighbor in , then , thus  and .  If  has some neighbor , then  which means, by the way  is defined for  in Step~1, that .  Consequently, by Step~3, , i.e., .  Similarly, if  has some neighbor in , then  which means that .  Thus, there must exist at least one vertex  with .  Then, since , it follows that .   Finally, if  has a neighbor in , then , thus  for some  by either Step~2.1 or Step~2.4.  Hence, , thus .  Analogously,  has no neighbors in .  Summing up, since  is an independent set, it follows that  has no neighbors in , thus  and  is not monochromatic by Step~3.

  For the converse, suppose  is star -choosable, and consider any valuation  of .  Define  to be a -list assignment of  so that  is the unique color admissible for all the vertices in ,  is the unique color admissible for all the vertices in , and  for every vertex not connected to a -forcer.  By statement~(\ref{lem:forcer properties:color}) of Lemma~\ref{lem:forcer properties}, such list assignment  always exists.  Let  be a star -coloring of  and extend  to include  in its domain so that .  Note that  can take any value from , so it is enough to prove that .  Define , , and , and let .  As in Theorem~\ref{thm:stcol-2}, it can be observed that (i)  is a monochromatic star and (ii) every vertex in  is either adjacent or equal to a vertex in .  Thus, since  is a star -coloring of , there must be some vertex  adjacent to no vertex in .  Moreover, such vertex  corresponds to some clause  whose literals are all true by the way  is defined.
\end{proof}

The proof for  is by induction, i.e., we reduce \stchose{k} into \stchose{(k+1)} for every .  Roughly speaking, the idea of the reduction is to insert a vertex  that forbids every vertex of the reduced graph to have the same color as .

\begin{theorem} \label{thm:stchose-k}
  \stchose{k} is \ptp-complete for every , and it remains \ptp-complete when the input is restricted to -free graphs.
\end{theorem}

\begin{proof}
  The proof is by induction on .  The base case  corresponds to Theorem~\ref{thm:stchose-2}.  For the inductive step, we show how to transform a -free graph  into a -free graph  so that  is star -choosable if and only if  is star -choosable.  
  
  The vertices of  are divided into connection and inner vertices.  Connection vertices comprise a set  inducing  and a vertex .  Inner vertices are included in -forcers or -switchers connecting connection vertices.  There is a -forcer connecting , and a -switcher connecting  for every .  Let  be the -switcher connecting , i.e.,  and  are cliques of .  By statement (\ref{lem:forcer properties:forbidden}) of Lemmas \ref{lem:switcher properties}~and~\ref{lem:forcer properties},  is -free.  

  Suppose  is star -choosable.  Let  be a -list assignment of , and  be -admissible for .  Recall that  always exists by statement~(\ref{lem:forcer properties:extension}) of Lemma~\ref{lem:forcer properties}.  Define  as a -list assignment of  such that  for .  By hypothesis, there is a star -coloring  of .  Define  to be the -coloring of  such that  for  and .  Inner vertices are colored according to Lemma~\ref{lem:switcher choosability} and statement~(\ref{lem:forcer properties:extension}) of Lemma~\ref{lem:forcer properties}.  Clearly, if  is a maximal star of  and  is a connection vertex, then either  or  includes a maximal star of .  Whichever the case,  is not monochromatic, i.e.,  is a star coloring of .

  For the converse, let  be an -list assignment of  and take a color .  Define  as any -list assignment of  such that  is the unique -admissible color for , and  for every .  Such list assignment always exists by statement~(\ref{lem:forcer properties:color}) of Lemma~\ref{lem:forcer properties}.  Let  be a star -coloring of .  By construction, , and by Lemma~\ref{lem:switcher properties}, .  Hence, the restriction  of  to  is an -coloring of .  Moreover, if  is a maximal star of , then  is a maximal star of , for every .  Since  is a block of  and , it follows that .  Hence,  is not monochromatic.
\end{proof}


\section{Forbidding graphs of order 3} 
\label{sec:small forbiddens}

The previous sections dealt with time complexity of the star and biclique coloring and choosability problems.  The remaining of the article is devoted to these problems in other restricted classes of graphs.  As discussed in Section~\ref{sec:chordal}, we are interested in classes of graphs that are related to chordal graphs or can be defined by forbidding small induced subgraphs.  In this section, we study the classes of -free graphs, for every graph  on three vertices.

There are four graphs with exactly three vertices, namely , , , and .  The following theorem shows that -free graphs are star -choosable.

\begin{theorem}\label{thm:stcol triangle-free}
  Every -free graph is star -choosable.  Furthermore, for any -list assignment, a star -coloring can be obtained in linear time.
\end{theorem}

\begin{proof}
  Let  be a -list assignment of a -free graph ,  be a rooted tree subgraph of  with ,  be the root of , and  be the parent of  in  for each .  Define  to be an -coloring of  where  and  for every .  Since  is -free,  is a maximal star of  for  only if , hence  is not monochromatic.  Observe that a BFS traversal of  is enough to compute , thus  is computed in linear time from .
\end{proof}

As a corollary, we obtain that -free graphs are biclique -choosable also.  However, this corollary can be easily strengthened so as to include those -free graphs that are \emph{biclique-dominated}.  A graph  is \Definition{biclique-dominated} when every maximal biclique is either a star or has a false dominated vertex.  Some interesting classes of graphs are -free and biclique-dominated, including hereditary biclique-Helly graphs~\cite{EguiaSoulignacDMTCS2012}.

\begin{theorem}
  Every -free graph that is biclique-dominated is biclique -choosable.  Furthermore, for any -list assignment, a biclique -coloring can be computed in polynomial time.
\end{theorem}

\begin{proof}
  Let  be a -list assignment of a -free graph  that is biclique-dominated.  The algorithm for biclique -coloring  has two steps.  First, apply Theorem~\ref{thm:stcol triangle-free} on  so as to obtain a star -coloring  of .  Second, traverse each vertex  and, for each  that is false dominated by , change  with any color in .  (It is not important if  or  are later changed when other vertices are examined.) The coloring thus generated is a biclique -coloring.  Indeed, if a maximal biclique contains a false dominated vertex , then it also contains the vertex  such that  was last changed in the second step while traversing .  Since false domination is a transitive relation, it follows that  when the second step is completed.  On the other hand, if  is a maximal biclique with no false dominated vertices, then  is a star.  Since the colors of the vertices of  are not affected by the second step, we obtain that  is not monochromatic.  It is not hard to see that the algorithm requires polynomial time.
\end{proof}

Coloring a connected -free graph is trivial because the unique connected -free graph  with  vertices is .  Thus, .

\begin{theorem}
 If  is a connected -free graph with  vertices, then .
\end{theorem}

The case of -free graphs, examined in the next theorem, is not much harder.

\begin{theorem}
 If  is a -free graph with  universal vertices, then .
\end{theorem}

\begin{proof}
 Let  be the set of universal vertices of .  Clearly,  is a block of , thus  and  by Observation~\ref{obs:block coloring}.  For the other bound, let  be a -list assignment of , and  be the sets of vertices that induce components of .  Define  as an -coloring of  such that  and  for \range{i}{1}{n}.  Note that  is a set of false twin vertices (\range{i}{1}{j}) because  is a clique of .  Thus, every maximal star or biclique  is formed by two vertices of  or it contains a set  for some \range{i}{1}{n}.  Whichever the case,  is not monochromatic, thus  is a star and biclique -coloring.
\end{proof}

The remaining class is the class of -free graphs.  By definition, if  is -free, then every maximal star and every maximal biclique of  has  vertices.  Thus, it takes polynomial time to determine if an -coloring of  is a star or biclique coloring, for any -list assignment .  Hence, when restricted to -free graphs, the star and biclique -coloring problems belong to \NP, while the star and biclique -choosability problems belong to \ptwop.  The next theorem shows that, when , the choosability problems are \ptwop-complete even when the input is further restricted to co-bipartite graphs.

\begin{theorem}\label{thm:co-bipartite choosability}
 \stchose{k} and \bcchose{k} are \ptwop-complete for every  when the input is restricted to co-bipartite graphs.
\end{theorem}

\begin{proof}
 The proof is obtained by reducing the problem of determining if a connected bipartite graph with no false twins is vertex -choosable, which is known to be \ptwop-complete~\cite{GutnerTarsiDM2009}.  Let  be a connected bipartite graph with no false twins,  be a bipartition of , and .  Define  to be the bipartite graph obtained from  by inserting, for every , the stars  (\range{i}{1}{4}) with  and the edges , , , , , and  (see Figure~\ref{fig:co-bip choose}).  We claim that  is vertex -choosable if and only if  is star (resp.\ biclique) -choosable.  
 
\begin{figure}
  \centering\includegraphics{\imagenes demo-teo-20-v2}\caption{Transformation applied to  in Theorem~\ref{thm:co-bipartite choosability}; each independent set has  vertices.}\label{fig:co-bip choose}
\end{figure}


 Suppose first that  is vertex -choosable, and let  be a -list assignment of  and  be the restriction of  to .  By hypothesis,  admits a vertex -coloring .  Define  to be any vertex -coloring of  so that  for , and  for every  and every \range{i}{1}{4}.  It is not hard to see that such a coloring always exists.  Clearly, every maximal star (resp.\ biclique)  of  is formed by two twins of  or it contains two vertices that are adjacent in .  In the latter case  is not -monochromatic because  is a vertex coloring of , while in the former case  is not -monochromatic because both of its vertices must belong to , as  has no false twins, for some  and some \range{i}{1}{4}.  
  
 For the converse, suppose  is star (resp.\ biclique) -choosable, and let  be a -list assignment of .  Define  to be a star (resp.\ biclique) -coloring of , for the -list assignment  of  where  for every  with , and every  with \range{i}{1}{4}.  Suppose, to obtain a contradiction, that  for some  with  and .  Then, for every  (\range{i}{1}{4}), we obtain that  because  is a maximal star (resp.\ biclique) of .  Hence, since  is a block of , we obtain by Observation~\ref{obs:block coloring} that  for every .  Consequently, since  is a maximal star (resp.\ biclique) for every  and every , it follows that .  Analogously,  for every \range{i}{1}{4}.  But then,  is a monochromatic maximal biclique that contains a maximal star, a contradiction.  Therefore,  for every edge  of , which implies that the restriction of  to  is a vertex -coloring of .
\end{proof}

Let  be a -free graph with no false twins, and define  as the -free graph that is obtained from  as in Theorem~\ref{thm:co-bipartite choosability}.  By fixing the list assignment that maps each vertex to  in the proof of Theorem~\ref{thm:co-bipartite choosability}, it can be observed that  admits a vertex -coloring if and only if  admits a star (resp.\ biclique) -coloring, for every .  The problem of determining if a connected -free graph with no false twins admits a vertex -coloring is known to be \NP-complete~\cite{Lovasz1973,MaffrayPreissmannDM1996}.  Hence, the star and biclique -coloring problems are -complete when restricted to -free graphs, for every .
 

\begin{theorem}\label{thm:co-k3 coloring}
 \stcol{k} and \bccol{k} are \NP-complete for every  when the input is restricted to -free graphs.
\end{theorem}


\section{Graphs with restricted diamonds}
\label{sec:diamond-free}

The graph  defined in Theorem~\ref{thm:stcol-2} contains a large number of induced diamonds.  For instance, to force different colors on a pair of vertices  and , a -switcher  connecting  is used.  Such switcher contains  diamonds, one for each edge of .  An interesting question is, then, whether induced diamonds can be excluded from Theorem~\ref{thm:stcol-2}.  The answer is no, as we prove in this section that the star coloring problem is \NP-complete for diamond-free graphs.  By taking a deeper look at , it can be noted that every diamond of  has a pair of twin vertices.  In order to prove that the star coloring problem is \NP-complete for diamond-free graphs, we show that the problem is \NP even for the larger class of graphs in which every diamond has two twin vertices.  This class corresponds to the class of \{, dart, gem\}-free graphs (cf.\ below), and it is worth to note that its graphs may admit an exponential number of maximal stars.  We also study the biclique coloring problem on this class, for which we prove that the problem is \NP when there are no induced  for .  At the end of the section, we study the star and biclique choosability problems, which turn to be \ptwop-hard for \{, dart, gem\}-free graph.

Let  be a graph.  Say that  is \Definition{block separable} if every pair of adjacent vertices  not dominating  are twins in .  The following lemma shows that \{, dart, gem\}-free graphs are precisely those graphs in which every induced diamond has twin vertices, and they also correspond to those graphs is which every vertex is block separable.  This last condition is crucial in the \NP coloring algorithms.

\begin{theorem}\label{thm:w4dartgemequivalence}
The following statements are equivalent for a graph .
  \begin{enumerate}[(i)]
    \item  is \{, dart, gem\}-free.
    \item Every induced diamond of  contains a pair of twin vertices.
    \item Every  is block separable. 
  \end{enumerate}
\end{theorem}
\begin{proof} 
 (i)  (ii) If  induces a diamond with universal vertices  and there exists , then  induces a , a dart, or a gem in  depending on the remaining adjacencies between  and the vertices of .

 (ii)  (iii) Suppose  is not block separable, thus  contains two adjacent vertices  and  not dominating  that are not twins in ; say . Then,  and  are the universal vertices of a diamond containing  and a vertex in , i.e.,  contains an induced diamond with no twin vertices.

 (iii)  (i) The , dart, and gem graphs have a vertex of degree  that is not block separable.
\end{proof}


Note that if  is block separable, then  can be partitioned into sets  where  and each  is a block of .  Moreover, no vertex in  is adjacent to a vertex in , for .  We refer to  as the \Definition{block separation} of .  By definition,  is a maximal star of  with  if and only if ,  and  for \range{i}{1}{\ell}.  By Theorem~\ref{thm:w4dartgemequivalence}, every vertex of a \{, dart, gem\}-free graph admits a block separation, hence the next result follows.

\begin{lemma}
 Let  be a \{, dart, gem\}-free graph with a coloring .  Then,  is a star coloring of  if and only if
 \begin{itemize}
  \item  for every block  of , and
  \item for every  with block separation , there exists  such that .
 \end{itemize}
\end{lemma}

It is well known that the blocks of a graph  can be computed in  time.  Hence, it takes  time obtain the block separation of a block separable vertex , and, consequently, the star -coloring and the star -choosability problems are in \NP and \ptwop for \{, dart, gem\}-free graphs, respectively.

\begin{theorem}\label{thm:np w4-dart-gem}
 \stcol{k} is \NP when the input is restricted to \{, dart, gem\}-free graphs.
\end{theorem}

\begin{theorem}\label{thm:ptp w4-dart-gem}
 \stchose{k} is \ptwop when the input is restricted to \{, dart, gem\}-free graphs.
\end{theorem}

We now consider the biclique coloring problem.  The algorithm for determining if a coloring  is a biclique coloring of  is divided in two steps.  First, it checks that no monochromatic maximal star is a maximal biclique.  Then, it checks that  contains no monochromatic maximal biclique  with . 

For the first step, suppose  is a coloring of  where  for every block  of .  Let  be a vertex with a block separation .  As discussed above,  is a maximal star if and only if , , and  for every \range{i}{1}{\ell}.  If  is not a maximal biclique, then there exists   adjacent to all the vertices in .  Observe that  has at most one neighbor in  with color , for each color .  Otherwise, taking into account that twin vertices have different colors,  would induce a diamond with no twin vertices, for .  Therefore, at most one monochromatic maximal star with center  is included in a biclique containing , for each .  Thus, to check if there is a monochromatic maximal biclique containing  we first check whether .  If negative, then  is not a biclique coloring of .  Otherwise, all the monochromatic maximal stars with center in  are generated in polynomial time, and for each such star  it is tested if there exists  adjacent to all the vertices in .

\begin{lemma}\label{lem:w4-dart-gem stars}
 If a \{, dart, gem\}-free graph  and a coloring  are given as input, then it takes polynomial time to determine if there exists a monochromatic maximal biclique  with .
\end{lemma}

For the second step, suppose  is an independent set with at least two vertices, and let .  Note that if  are adjacent, then they are twins in  because  are the universal vertices of any induced diamond formed by taking a pair of vertices in .  Hence,  can be partitioned into a collection  of blocks of  where no vertex in  is adjacent to a vertex in , for .  Thus,  is a maximal biclique of  if and only if no vertex of  is complete to  and  for every \range{i}{1}{\ell}.  That is,  has a monochromatic maximal biclique  if and only if , each block of  has a vertex of color , and .

\begin{lemma}\label{lem:w4-dart-gem bicliques}
 Let  be a \{, dart, gem\}-free graph.  If an independent set  and a coloring  of  are given as input, then it takes polynomial time to determine if  has a monochromatic maximal biclique  with .
\end{lemma}

If  is -free for some constant , then every biclique  of  with  has .  Thus, to determine if  is a biclique coloring of , it is enough traverse every independent set  of  with  vertices and to check that there exists no  such that  is a monochromatic maximal biclique.  By Lemmas \ref{lem:w4-dart-gem stars}~and~\ref{lem:w4-dart-gem bicliques}, it takes polynomial time to determine if there exists  such that  is a monochromatic maximal star.  Since there are  independent sets with at most  vertices, the algorithm requires polynomial time.  We thus conclude that \bccol{k} and \bcchose{k} are respectively \NP and \ptwop when the input is restricted to \{, , dart, gem\}-free graphs.

\begin{theorem}\label{thm:np w4-dart-gem-kii}
 \bccol{k} is \NP when the input is restricted to \{, , dart, gem\}-free graphs, for .
\end{theorem}

\begin{theorem}
 \bcchose{k} is \ptwop when the input is restricted to \{, , dart, gem\}-free graphs, for .
\end{theorem}

In the rest of this section, we discuss the completeness of the star and biclique coloring and choosability problems.  As in Sections \ref{sec:general case}~and~\ref{sec:choosability}, only one proof is used for each problem because -free graphs are considered.  For the reductions, two restricted satisfiability problems are required, namely \naesat and \naesatt.  A valuation  of a CNF formula  is a \Definition{nae-valuation} when all the clauses of  have a true and a false literal.  The formula  is \Definition{nae-true} when  admits a nae-valuation, while  is \Definition{nae-true} when every valuation of  is a \Definition{nae-valuation}.   \naesat is the \NP-complete problem (see~\cite{Garey1979}) in which a CNF formula  is given, and the goal is to determine if  admits a nae-valuation.  Analogously, \naesatt is the \ptwop-complete problem (see~\cite{EiterGottlob1995}) in which a CNF formula  is given, and the purpose is to determine if  is nae-true.  We begin discussing the completeness of the star coloring problem.  In order to avoid induced diamonds, we define a replacement of long switchers.

\begin{defn}[diamond -switcher]\label{def:diamond k-switcher}
  Let  be a graph and  be an independent set of  with .  Say that  is a \Definition{diamond -switcher connecting } () when  can be partitioned into a vertex , a set of leafed vertices , a family  of -keepers, and a clique  with  vertices in such a way that  is a clique,  is a star,  connects  for , and there are no more edges adjacent to vertices in .
\end{defn}

A diamond -switcher is depicted in Figure~\ref{fig:diamond switcher}.  The main properties of diamond switchers are given in the next lemma.

\begin{figure}[htb]
 \centering
 \includegraphics{\imagenes diam-switcher}
 \caption{A diamond -switcher connecting .}\label{fig:diamond switcher}
\end{figure}

\begin{lemma}\label{lem:diamond switcher properties}
 Let  be a graph and  be a diamond -switcher connecting  (). Then,
\begin{enumerate}[(i)]
  \item no induced , diamond, or  of  contains a vertex of ,\label{lem:diamond switcher properties:forbidden}
  \item  for any star -coloring  of , and \label{lem:diamond switcher properties:colors}
  \item Any -coloring  of  in which  can be extended into a -coloring of  in such a way that no monochromatic maximal star has its center in .\label{lem:diamond switcher properties:extension}
\end{enumerate}
\end{lemma}

\begin{proof}
Let , , , \ldots, , and  be as in Definition~\ref{def:diamond k-switcher}.  Statement (\ref{lem:diamond switcher properties:forbidden}) follows by statement~(\ref{lem:keeper properties:forbidden}) of Lemma~\ref{lem:keeper properties}, observing that  is an independent set and that no induced diamond can contain a vertex in a -keeper.  

(\ref{lem:keeper properties:colors}) Let  be a star -coloring of .  Since  is a block of size , it contains a vertex  with color  by Observation~\ref{obs:block coloring}.  Then, taking into account that  is a maximal star and  by statement~(\ref{lem:keeper properties:colors}) of Lemma~\ref{lem:keeper properties} for , it follows that  for some .

(\ref{lem:keeper properties:extension}) To extend , first set  and  (), and then iteratively extend  to color the leaves and the -keepers according to Observation~\ref{obs:leafed vertex} and statement~(\ref{lem:keeper properties:extension}) of Lemma~\ref{lem:keeper properties}.
\end{proof}

We are now ready to prove the \NP-completeness of the star-coloring problem.

\begin{theorem}\label{thm:w4dartgemnpc}
\stcol{k} is -complete when the input is restricted to \{, diamond, \}-free graphs for every .
\end{theorem}

\begin{proof}
 By Theorem~\ref{thm:np w4-dart-gem}, \stcol{k} is \NP for \{, diamond, \}-free graphs.  For the hardness part, we show a polynomial time reduction from \naesat.  That is, given a CNF formula  with  clauses  and  variables , we define a \{, diamond, \}-free graph  such that  admits a nae-valuation if and only if  admits a star -coloring.  

 The vertices of  are divided into \Definition{connection} and \Definition{inner} vertices.  For each \range{i}{1}{n} there are two connection vertices  \Definition{representing} the literals  and , respectively.  Also, there are  connection vertices .  Let , , and  for \range{h}{1}{\ell}.  Inner vertices are the vertices included in diamond -switchers connecting connection vertices.  For each  and each  there is a \Definition{color} diamond -switcher connecting .  Also, for each  there is a \Definition{valuation} diamond -switcher connecting .  Finally, there is a \Definition{clause} diamond -switcher connecting  for every \range{h}{1}{\ell}.  Observe that  is an independent set of .  Thus, by statement~(\ref{lem:diamond switcher properties:forbidden}) of Lemma~\ref{lem:diamond switcher properties},  is \{, diamond, \}-free.  

 Suppose  has a nae-valuation , and let  be a -coloring of the connection vertices such that  and  for \range{i}{1}{n} and \range{j}{3}{k}.  Clearly, every color or valuation -switcher connects a pair of vertices that have different colors.  Also, since  is a nae-valuation, every set  (\range{h}{1}{\ell}) has two vertices representing literals  and  of  with .  Hence,  is not monochromatic, thus every clause diamond -switcher connects a non-monochromatic set of vertices.  Therefore, by statement~(\ref{lem:diamond switcher properties:extension}) of Lemma~\ref{lem:diamond switcher properties},  can be iteratively extended into a star -coloring of .

 For the converse, suppose  admits a star -coloring .  By applying statement~(\ref{lem:diamond switcher properties:colors}) of Lemma~\ref{lem:diamond switcher properties} while considering the different kinds of diamond -switchers, we observe the following facts.  First, by the color diamond -switchers,  and .  Then, we can assume that  and .  Hence, by the valuation diamond -switcher connections, we obtain that  for every .  Thus, the mapping  such that  is a valuation.  Moreover, by the clause diamond -switcher connections,  is not monochromatic for \range{h}{1}{\ell}.  Consequently,  is a nae-valuation of .
\end{proof}

Observe that the graph  defined in Theorem~\ref{thm:w4dartgemnpc} is not chordal.  However, as discussed in Section~\ref{sec:chordal}, every edge  such that  are connected by a -keeper (inside the diamond -switchers) can be subdivided so as to eliminate all the induced holes of length at most , for every .

We now deal with the star choosability problem.  Recall that long switchers are not well suited for the star choosability problem because they contain keepers, and vertices connected by keepers need not have the same colors in every list coloring.  Keepers are also present inside diamond switchers, thus it is not a surprise that diamond -switchers are not star -choosable.   For this reason, as in Section~\ref{sec:choosability}, the proof is by induction, using list switchers for .  Since list switchers contain induced diamonds, the \stp-hardness will be obtained for \{, gem, dart\}-free graphs, and not for diamond-free graphs.  Unfortunately, we did not find a way to avoid these diamonds.  Moreover, some kind of forcers are required as well; our forcers have induced diamonds that we were not able to remove either.  The hardness proof for  is, in some sense, a combination of the proofs of Theorems~\ref{thm:stchose-2}~and~\ref{thm:w4dartgemnpc}.  Roughly speaking, the idea is to force the colors of the universal variables of the input formula as in Theorem~\ref{thm:stchose-2}, while a nae-valuation is encoded with colors as in Theorem~\ref{thm:w4dartgemnpc}.

\begin{theorem}\label{thm:w4-dart-gem ptp complete}
  \stchose{2} is \ptwop-hard when its input is restricted to \{, dart, gem, \}-free graphs.
\end{theorem}

\begin{proof}
 The hardness of \stchose{2} is obtained by reducing \naesatt.  That is, given a CNF formula  with  clauses , and  variables , , we build a \{, dart, gem, \}-free graph  that is star -choosable if and only if  is nae-true.  For the sake of simplicity, in this proof we use , , and  as indices that refer to values in , , and .
  
 Graph  is an extension of the graph in Theorem~\ref{thm:w4dartgemnpc} for  (replacing diamond switcher with list switchers).  It has a \Definition{connection} vertex  (resp.\ , , ) \Definition{representing}  (resp.\ , , ) for each  (and each ), and two \Definition{connection} vertices , .  Let , , , and .  Graph  also has \Definition{inner vertices} which are the vertices in list switchers and -forcers connecting connection vertices.  There are -forcers connecting each vertex of , and list switchers connecting:  and  for each ;  for each ; and  and  for each .  
  
 Let  be a -list assignment of , and suppose  is nae-true.  Define  as an -coloring of the connection vertices satisfying the following conditions.
 \begin{enumerate}[(i)]
  \item  is any color -admissible for .  Such a color always exists by statement~(\ref{lem:forcer properties:extension}) of Lemma~\ref{lem:forcer properties}.  Suppose, w.l.o.g., that  and , and define  as a valuation of  such that  if and only if .  

  \item By hypothesis,  can be extended into a nae-valuation of .  If , then .  Otherwise, .  Similarly,  if , while  otherwise.
    
  \item .  
 \end{enumerate}
 It is not hard to see that  can always be obtained.  Observe that ,  only if , and  is monochromatic only if .  Therefore,  can be extended into an -coloring of  by statement~(\ref{lem:list switcher properties:extension}) of Lemma~\ref{lem:list switcher properties} and statement~(\ref{lem:forcer properties:extension}) of Lemma~\ref{lem:forcer properties}.
 
 For the converse, suppose  is star -choosable, and consider any valuation  of .  Define  to be a -list assignment of  such that  is the unique color admissible for ,  is the unique color admissible for ,  is the unique color admissible for , and  for .  By statement~(\ref{lem:forcer properties:color}) of Lemma~\ref{lem:forcer properties}, such a list assignment always exists.  Let  be a star -coloring of .  By repeatedly applying statement~(\ref{lem:list switcher properties:colors}) of Lemma~\ref{lem:list switcher properties}, it can be observed that none of , , , and  are monochromatic.  Therefore,  is a nae-valuation of .
\end{proof}

Note that if  is a -switcher, then every induced diamond that contains a vertex  also contains a twin of .  Hence, by Theorem~\ref{thm:w4dartgemequivalence}, if  is a -forcer connecting , then no vertex in  belongs to an induced dart or gem.  Consequently, if  is a \{, dart, gem, \}-free graph, then the graph  defined in the proof of Theorem~\ref{thm:stchose-k} is \{, dart, gem, \}-free.  That is, a verbatim copy of the proof of Theorem~\ref{thm:stchose-k} can be used to conclude the following.

\begin{theorem}
  \stchose{k} is \ptwop-complete when its input is restricted to \{, dart, gem, \}-free graphs for every .
\end{theorem}

\section{Split graphs}
\label{sec:split}

In this section we consider the star coloring and star choosability problems restricted to split graphs.  The reason for studying split graphs is that they form an important subclass of chordal graphs, and also correspond to the class of \{, , \}-free graphs~\cite{Golumbic2004}.  A graph  is \Definition{split} when its vertex set can be partitioned into an independent set  and a clique .  There are  maximal stars centered at , namely  and  for .  Thus, the star coloring and star choosability problems on split graphs are \NP and \ptwop, respectively.  In this section we prove the completeness of both problems.

We begin observing that  admits a star coloring with  colors, where  is the size of the maximum block.  Indeed, each block  of  is colored with colors , while each vertex of  is colored with color .  We record this fact in the following observation.
\begin{observation}
 If  is a split graph whose blocks have size at most , then  admits a star coloring using  colors.  Furthermore, such a coloring can be obtained in linear time.
\end{observation}
Computing a star coloring of a split graph using  colors is easy, but determining if  colors suffice is an -complete problem.  The proof of hardness is almost identical to the one in Section~\ref{sec:diamond-free} for \{, diamond, \}-free graphs.  That is, given a CNF formula  we build a graph  using \emph{split} switchers in such a way that  admits a nae-valuation if and only if  admits a star -coloring.  As switchers, \Definition{split} switchers force a set of vertices to have at least two colors in a star -coloring.  The difference is that split switchers do so in a split graph (see Figure~\ref{fig:split-switcher}).

\begin{figure}
 \centering\includegraphics{\imagenes 2-split-switcher}
 \caption{A split -switcher connecting a set .}\label{fig:split-switcher}
\end{figure}


\begin{defn}[split -switcher]\label{def:nae connection}
  Let  be a split graph and  with .  Say that  is a \Definition{split -switcher connecting } when  can be partitioned into two sets  and two vertices  in such a way that ,  is a clique for every ,  and  are cliques, there are no more edges between vertices in  and vertices in , and there are no more edges incident to  and .
\end{defn}

The properties of split -switchers are summarized in the following lemma.

\begin{lemma}\label{lem:split switcher properties}
 Let  be a split graph and  be a split -switcher connecting . Then,
\begin{enumerate}[(i)]
  \item  for every star -coloring  of , and\label{lem:split switcher properties:colors}
  \item For every -list assignment of , any -coloring  of  in which  can be extended into an -coloring of  in such a way that no monochromatic maximal star has its center in .\label{lem:split switcher properties:extension}
\end{enumerate}
\end{lemma}

\begin{proof}
 Let , , , and  be as in Definition~\ref{def:nae connection}.
 
 (\ref{lem:split switcher properties:colors}) Let  be a star -coloring of .  By definition,  and  are blocks of , thus  by Observation~\ref{obs:block coloring}.  Let  and  be the vertices with color  in  and , respectively.  Since the maximal star  is not monochromatic, it follows that .  Therefore, .
 
 (\ref{lem:split switcher properties:extension}) To extend  to , define , and .  Let  be a maximal star with .  If either  or  and , then , thus  is not monochromatic.  Otherwise, if  and , then , thus  is not monochromatic as well.
\end{proof}

By replacing diamond -switchers with split -switchers in the proof of Theorem~\ref{thm:w4dartgemnpc}, the -completeness of the star -coloring problem for split graphs is obtained.  For the sake of completeness, we sketch the proof, showing how to build the graph  from the CNF formula.

\begin{theorem}\label{thm:splitnpc}
\stcol{k} restricted to split graphs is -complete for every . 
\end{theorem}

\begin{proof}
 Split graphs have  maximal stars, hence \stcol{k} is  for split graphs.  For the hardness part, let  be a CNF formula with  clauses  and  variables .  Define  as the split graphs with \Definition{connection} and \Definition{inner} vertices, as follows.  For each \range{i}{1}{n} there are two connection vertices  \Definition{representing} the literals  and , respectively.  Also, there are  connection vertices .  Let , , and  for \range{h}{1}{\ell}.  Inner vertices form the split -switchers connecting connection vertices.  For each  and each  there is a \Definition{color} split -switcher connecting .  Also, for each  there is a \Definition{valuation} split -switcher connecting .  Finally, there is a \Definition{clause} split -switcher connecting  for every \range{h}{1}{\ell}. Clearly,  is a split graph, and, as in the proof of Theorem~\ref{thm:w4dartgemnpc},  admits star -star coloring if and only if  admits a nae-valuation.
\end{proof}


For the star choosability problem of split graphs, the idea is to adapt the proof of Theorem~\ref{thm:w4-dart-gem ptp complete}, providing a new kind of forcer.  This new forcer is called the \emph{split -forcer} and it is just a -forcer where its -switchers form a clique.  For the sake of completeness, we include its definition.

\begin{defn}[split -forcer]\label{def:k-split-forcer connection}
  Let  be a split graph and .  Say that  is a \emph{split -forcer connecting } () when  can be partitioned into sets  and  for  and  in such a way that , ,  and  are cliques, there are no more edges between vertices in  and vertices in , and there are no more edges incident to vertices in .
\end{defn}

Let  be a -list assignment of a split graph  and  be a split -forcer connecting .  As in Section~\ref{sec:choosability}, we say that  is \emph{-admissible for } when there is an -coloring  of  such that  and no monochromatic maximal star has its center in .  The following lemma resembles Lemma~\ref{lem:forcer properties}.

\begin{lemma}\label{lem:split forcer properties}
  Let  be a split graph and  be a split -forcer connecting .  Then, 
  \begin{enumerate}[(i)]
    \item for every -list assignment  of  there is an -admissible color for , and\label{lem:split forcer properties:extension}
    \item every -list assignment  of  can be extended into a -list assignment of  in which  has a unique -admissible color.
  \end{enumerate}
\end{lemma}

\begin{proof}
  The lemma can be proven with a verbatim copy of the proof of Lemma~\ref{lem:forcer properties}.  In particular, observe that a split -forcer can be obtained from a -forcer by inserting the edges between  and , for every  and .  
\end{proof}

The hardness of the star choosability problem is also obtained by adapting the proof of Theorem~\ref{thm:w4-dart-gem ptp complete}.  We remark, however, that in this case no induction is required, because split -switchers are -choosable by statement~(\ref{lem:split switcher properties:extension}) of Lemma~\ref{lem:split switcher properties}.  The proof is sketched in the following theorem.


\begin{theorem}\label{thm:split stchose}
 \stchose{k} restricted to split graphs is \ptwop-complete for every .
\end{theorem}

\begin{proof}
 \stchose{k} is \ptwop for split graphs because split graphs have a polynomial amount of maximal stars.  Let  be a CNF formula with  clauses , and  variables , .  Use , , , and  to denote indices in , , , and , respectively.  Define  as the split graph that has a \Definition{connection} vertex  (resp.\ , , ) \Definition{representing}  (resp.\ , , ), and  \Definition{connection} vertices , , .  Let , , , , and .  Graph  also has \Definition{inner vertices} that are the vertices of split -switchers and split -forcers connecting connection vertices.  There are split -forcers connecting , , , and split -switchers connecting: , , , , , and  for every .  

 Following the proof of Theorem~\ref{thm:w4-dart-gem ptp complete}, it can be observed that, for ,  admits a star -coloring, for a -list assignment  of , if and only if  is nae-true.  For , observe that if  is nae-true, then a star -coloring  is obtained if  is taken so that  and .  Conversely, if  admits a star -coloring for the -list assignment in which  for every connecting vertex , then , thus a nae-valuation of  is obtained.
\end{proof}

To end this section, consider the more general class of -free graphs.  By definition,  is a star of a graph  if and only if  is a maximal independent set of .  In~\cite{FarberDM1989}, it is proved that  has  maximal independent sets when  is -free.  Thus, -free graphs have  maximal stars, which implies that the star coloring and star choosability problems on this class are \NP and \ptwop, respectively.  

\begin{theorem}\label{thm:2k2 coloring}
 \stcol{k} and \stchose{k} are respectively \NP-complete and \ptwop-complete for every  when the input is restricted to -free graphs.
\end{theorem}

\section{Threshold graphs}
\label{sec:threshold}

\newcommand{\AsVec}[1]{[#1]}

Threshold graphs form a well studied class of graphs which posses many definitions and characterizations~\cite{Golumbic2004,MahadevPeled1995}.  The reason for studying them in this article is that threshold graphs are those split graphs with no induced 's.  Equivalently, a graph is a threshold graph if an only if it is -free.

In this section we develop a linear time algorithm for deciding if a threshold graph  admits a star -coloring.  If affirmative, then a star -coloring of  can be obtained in linear time. If negative, then a certificate indicating why  admits no coloring is obtained.  We prove also that  is star -choosable if and only if  admits a star -coloring.  Thus, deciding whether  is star -choosable takes linear time as well.  It is worth noting that threshold graphs can be encoded with  bits using two sequences of natural numbers (cf.\ below).  We begin this section with some definitions on such sequences.

Let  be a sequence of natural numbers.  Each \range{i}{1}{r} is called an \Definition{index} of .  For , we write  and  to respectively indicate that  and  for every index .  Similarly, we write  when , i.e., when  for some index .  Note that  could be empty; in such case,  and  for every .  For indices , we use  to denote the sequence .  If , then .  Similarly, we define , , and .  

A \Definition{threshold representation} is a pair  of sequences of natural numbers such that . Let  and .  Each threshold representation defines a graph  whose vertex set can be partitioned into  blocks  with  and  independent sets  with  such that, for , the vertices in  are adjacent to all the vertices in .  It is well known that  is a connected threshold graph if and only if it is isomorphic to  for some threshold representation ~\cite{Golumbic2004,MahadevPeled1995}.  The following observation describes all the maximal stars of .

\begin{observation}\label{obs:stars threshold}
  Let  be a threshold representation and  be a vertex of .  Then,  is a maximal star of  if and only if there are indices  of  such that , and  for some vertex .
\end{observation}

For , we say that index  of  is \Definition{-forbidden} for the threshold representation  when either  or  and there exists some index  such that ,  and .  The next theorem shows how to obtain a star -coloring of  when a threshold representation is provided.

\begin{theorem}\label{thm:threshold coloring}
  The following statements are equivalent for a threshold representation .
  \begin{enumerate}
    \item  is star -choosable.
    \item  admits a star -coloring.
    \item No index of  is not -forbidden for .
  \end{enumerate}
\end{theorem}

\begin{proof}
  (i)  (ii) is trivial.  

  (ii)  (iii).  Suppose  admits a star -coloring  and yet  contains some -forbidden index .  Since  is a block of , then  by Observation~\ref{obs:block coloring}.  Hence,  and there exists an index  such that , , and .  Let  be the unique vertex in  for \range{h}{i}{j-1}.  By Observation~\ref{obs:block coloring}, both  and  have at least one vertex of each color \range{c}{1}{k}, while for each index \range{h}{i+1}{j-1} there exists a color  such that .  Consequently, there are indices  in  such that  and  for every index \range{h}{a+1}{b-1}.  Indeed, it is enough to take \range{a}{i}{j-1} as the maximum index with  and \range{b}{a}{j} as the minimum index such that .  Therefore, if  and  are the vertices of  and  with color , respectively, then  is a monochromatic maximal star by Observation~\ref{obs:stars threshold}, a contradiction.

  (iii)  (i).  Let  be a -list assignment of  and define  as any vertex of  for each index  of .  For each index  of , define \range{p(i)}{1}{i} as the minimum index such that .  Let  be an -coloring of  that satisfies all the following conditions for every index  of :
  \begin{enumerate}[(1)]
    \item ,
    \item if  and , then  and  do not belong to ,
    \item if  and , then , and 
    \item if  and , then .
  \end{enumerate}
  A coloring satisfying all the above conditions can obtained iteratively, by coloring the vertices in  before coloring the vertices in  for every pair of indices .  We claim that  is a star -coloring of .  To see why, let  be a maximal star of .  By Observation~\ref{obs:stars threshold}, there are two indices  of  such that  and  for some .  If , then  by~(1).  If , then  is not monochromatic by~(4).  If , then  is not monochromatic by~(2).  If  and , then \range{p(i)}{i}{j-1}, thus  is not monochromatic by~(2).  Finally, if  and , then there exists index  such that ; otherwise  would be a -forbidden index of .  Then, by~(3),  which implies that  is not monochromatic.  Summing up, we conclude that  has no monochromatic maximal star.
\end{proof}

Theorem~\ref{thm:threshold coloring} has several algorithmic consequences for a threshold representation  of a graph .  As mentioned,  is a split graph where  and , thus  is either  or , for .  While deciding if  colors suffice for a general split graph is an {\NP}-complete problem, only  time is needed to decide whether  when  is given as input; it is enough to find a -forbidden index of .  Furthermore, if , then a -forbidden index can be obtained in  time as well.  Also, if , then a star -coloring  of  can be obtained in linear time by observing rules (1)--(4) of implication (iii)  (i).  To obtain , begin traversing  to find  for every index  of .  Then, color the vertices of each block  with colors .  Following, color the vertices  in that order, taking the value of  into account for each index  of .  If , then ; otherwise,  is any value in .  Finally, color the vertices in  with color  for each index  of .  To encode  only two values are required for each index  of , namely,  and .  Thus,  can be obtained in  time as well.  Finally,  can be obtained in  time from  when  is encoded with adjacency lists~\cite{Golumbic2004,MahadevPeled1995}.  Thus, all the above algorithms take  when applied to the adjacency list representation of .  We record all these observation in the theorem below.

\begin{theorem}
  Let  be a threshold representation of a graph .  The problems of computing , , a -forbidden index of  for , and a -coloring of  can be solved in  time when  is given as input.  In turn, all these problems take  time when an adjacency list representation of  is given.
\end{theorem}



\section{Net-free block graphs}
\label{sec:block}

In this section we study the star coloring and star choosability problems on block graphs.  A graph is a \Definition{block graph} if it is chordal and diamond-free.  We develop a linear time algorithm for deciding if a net-free block graph  admits a star -coloring.  The algorithm is certified; a star -coloring of  is obtained in the affirmative case, while a forbidden induced subgraph of  is obtained in the negative case.  As threshold graphs, net-free block graphs admit an  space representation using weighted trees (cf.\ below).  The certificates provided by the algorithms are encoded in such trees, and can be easily transformed to certificates encoded in terms of .  We begin describing the weighted tree representation of net-free block graphs.

A \Definition{netblock representation} is a pair  where  is a tree and  is a \Definition{weight function} from  to .  The graph  \Definition{represented} by  is obtained by inserting a clique  with  vertices adjacent to  and , for every .  It is well known that a connected graph is a netblock graph if and only if it is isomorphic to  for some netblock representation ~\cite{BrandstadtLeSpinrad1999}. By definition, two vertices  of  are twins if and only if 1.\  for some , or 2.\  is a leaf of  adjacent to  and , or 3.\ both  and  are leafs of  (in which case  is a complete graph).  The following observation describes the remaining maximal stars of .

\begin{observation}
\label{obs:blockbicliques}
Let  be a netblock representation and  be a vertex of .  Then,  is a maximal star of  with  if and only if  is an internal vertex of  and  contains exactly one vertex of  for every .
\end{observation}

Let  be a netblock representation.  For , a \Definition{-subtree} of  is a subtree of  formed by edges of weight ; a \Definition{maximal} -subtree is a -subtree whose edge set is not included in the edge set of another -subtree.  A \Definition{-exit} vertex of  is a vertex  that has some neighbor  such that .  The following theorem characterize those net-free block graphs that admit a star -coloring.

\begin{theorem}\label{thm:block graph coloring}
The following statements are equivalent for a netblock representation .

\begin{enumerate}[i.]
  \item  is star -choosable.
  \item  admits a star -coloring.
  \item  for every  and every maximal -subtree of  contains a -exit vertex of .
\end{enumerate}
\end{theorem}

\begin{proof}
   (i)  (ii) is trivial.  

   (ii)  (iii).  Suppose  admits a star -coloring .  By Observation~\ref{obs:block coloring},  for every  and, moreover,  when  incides in a leaf.  Let  be a maximal -subtree of , and consider a maximal path  of  such that  for .   Note that if , then , thus (1) .   We claim that  is a -exit vertex.  Indeed, if  is a leaf of  and  is its unique neighbor in , then  by (1) and the fact that  is a block.  Consequently, by the maximality of , it follows that , i.e.,  is a -exit vertex.  On the other case, if  is an internal vertex of , then, by Observation~\ref{obs:blockbicliques},  is a maximal star of  for every  that contains exactly one vertex of  for each .  Therefore, there exists  such that (2) .  By (1), , thus  by the maximality of  and .  Moreover,  by (2), thus  is a -exit vertex.
   
   (iii)  (i).  For every maximal -subtree  of , let  be a -exit vertex of .  For every , define .  Observe this definition is correct, because maximal -subtrees are vertex-disjoint.  Let  be a -list assignment of , and define  as an -coloring of  satisfying the following properties for every .  
  \begin{enumerate}[(1)]
    \item ,
    \item  and , and
    \item if  and  belongs to the unique path from  to  in , then .
  \end{enumerate}
  A coloring satisfying the above conditions can obtained by first coloring the vertices of , and then coloring the vertices in  for every .  Observe, in particular, that if , then , thus (3) is always possible.  We claim that  is a star -coloring of .  Let  be a maximal star of .  Suppose first that , thus either  for  or  is a leaf of  and  for some .  In the former case  by (2).  In the latter case, by (2), either  or .  If , then  and  is not a leaf of ; otherwise one of  and  would induce be a maximal -subtree without -exit vertices.  Hence,  by (3).  Suppose now that , thus  is an internal vertex of  by Observation~\ref{obs:blockbicliques}.  By hypothesis,  has some neighbor  such that either  (when ) or  belongs to the path of  between  and  (when ).  By Observation~\ref{obs:blockbicliques},  contains a vertex  of .  If , then , thus   by (1)~and~(2).  On the other case, if , then , thus  by (3).
\end{proof}
 
The algorithmic consequences of Theorem~\ref{thm:block graph coloring} are analogous as those observed for threshold graph in Section~\ref{sec:threshold}.  If  is a netblock representation and  is maximum among the sizes of the blocks of , then  equals either  or .  When  is given, it takes  time to find all the -subtrees and its -exit vertices, if they exist.  Thus, deciding if  takes  time when  is given as input.  Furthermore, if , then it takes  time to compute a maximal -subtree of  with no -exit vertices.  Such a subtree can be transformed into an induced subgraph of  in  time if required.  Also, a -star-coloring  of  can be obtained in linear with rules (1)--(3) of implication (iii)  (i).  First apply a BFS traversal of  to color the vertices of  with rule (1), and then color the remaining vertices of  following rules (2) and (3).  Finally, observe that  can be obtained in  time from  when  is encoded by adjacency lists.  Thus, all the discussed algorithms take linear time when applied to the adjacency list representation of .  We record all these observation in the theorem below.

\begin{theorem}\label{thm:block graph complexity}
  Let  be a netblock representation of a graph .  The problems of computing , , a maximal -subtree of  with no -exit vertices for , and a -coloring of  can be solved in  time when  is given as input.  In turn, all these problems take  time when an adjacency list representation of  is given.
\end{theorem}



\section{Further remarks and open problems}

In this paper we investigated the time complexity of the star and biclique coloring and choosability problems.  In this section we discuss some open problems that follow from our research.

Theorem~\ref{thm:stcol-2} states that the star -coloring problem is \stp-complete even when the input is restricted to \{, \}-free graphs.  In Section~\ref{sec:chordal} we discussed how to generalize this theorem to include \{, \}-free graphs, for every .  An interesting question is what happens when  grows to infinity, i.e., what happens when chordal graphs are considered.  By Theorem~\ref{thm:splitnpc}, we know that the star -coloring problem is at least \NP-hard on chordal graphs.  Is it \NP-complete or not?  Similarly, by Theorem~\ref{thm:split stchose}, the star -choosability problem on chordal graphs is at least \ptwop-hard; is it \ptwop-complete?

\begin{openproblem}
 Determine the time complexity of the star -coloring (-choosability) problem on chordal graphs and chordal -free graphs.
\end{openproblem}

To prove the \stp-completeness of the star -coloring problem we showed how to transform a formula  into a graph .  Graph  has many connection vertices that are joined together by -keepers and -switchers.  The purpose of the -keepers is to force two vertices to have the same colors, while -switchers are used to force different colors on a pair of vertices.  Both -keepers and -switchers contain blocks of size .  By taking a close examination at Lemmas \ref{lem:keeper properties}~and~\ref{lem:switcher properties}, it can be seen that these blocks play an important role when colors need to be forced.  An interesting question is whether these blocks can be avoided.  

\begin{openproblem}
 Determine the time complexity of the star -coloring (-choosability) problem on graph where every block has size at most , for .
\end{openproblem}

Keepers and switchers not only have blocks of size ; when a -keeper or a -switcher connects two vertices  and , a clique of size  containing  and  is generated.  We know that the star -coloring problem is \stp-complete for -free graphs, but what happens when -free graphs are considered?  The answer for the case  is given by Theorem~\ref{thm:stcol triangle-free}, i.e., the star -coloring problem is easy on -free graphs.  And for larger values of ?

\begin{openproblem}
 Determine the time complexity of the star -coloring (-choosability) problem on -free graphs, for  and .
\end{openproblem}

In Section~\ref{sec:choosability}, the \ptp-completeness of the star -choosability problem on -free graphs is proved by induction.  For the case , a graph  is built from a DNF formula using the same ideas that we used to prove the hardness of the star -coloring problem.  Then, for the case , the graph  is transformed into a graph .  The reason for splitting the proof in two cases is that long -switchers can no longer be included into .  So, to avoid the inclusion of induced 's, list switchers are used to build , while -switchers are used to transform  into  for .  This way, each generated graph  is -free.  We did not find a way to extend the holes in  as it is done in Section~\ref{sec:chordal} for the star -coloring problem.  Thus,  contains  as an induced subgraph for every , while  contains  as an induced subgraph.  Is the problem simpler when such holes are avoided?

\begin{openproblem}
 Determine the time complexity of the star -choosability problem when all the holes in the input graph have length at least , for .
\end{openproblem}

The star coloring problem is easier when any of the graphs on three vertices does not appear as an induced subgraph, as discussed in Section~\ref{sec:small forbiddens}.  Similarly, the biclique coloring problem is simpler when it contains no induced , , or , or when it is -free and biclique-dominated.  Also, by Theorem~\ref{thm:np w4-dart-gem-kii}, the biclique coloring problem on -free graphs is ``only'' \NP when it contains no induced  for .  

\begin{openproblem}
 Determine the time complexity of the biclique -coloring (-choosability) problem on -free graphs.
\end{openproblem}

The star and biclique coloring problems are also simplified when diamonds are forbidden, as seen in Section~\ref{sec:diamond-free}.  By Theorem~\ref{thm:np w4-dart-gem}, the star -coloring problem is \NP for diamond-free graphs, while the biclique -coloring problem is \NP for \{diamond, \}-free graphs () by Theorem~\ref{thm:np w4-dart-gem-kii}.  Additionally, if holes and nets are forbidden, then the star -coloring problem can be solved easily by Theorem~\ref{thm:block graph complexity}.  This leaves at least two interesting questions.

\begin{openproblem}
 Determine the time complexity of the biclique -coloring problems on diamond-free graphs.
\end{openproblem}

\begin{openproblem}
 Determine the time complexity of the star -choosability problems on diamond-free graphs.
\end{openproblem}



\begin{thebibliography}{10}

\bibitem{AmilhastreVilaremJanssenDAM1998}
J.~Amilhastre, M.~C. Vilarem, and P.~Janssen.
\newblock Complexity of minimum biclique cover and minimum biclique
  decomposition for bipartite domino-free graphs.
\newblock {\em Discrete Appl. Math.}, 86(2-3):125--144, 1998.

\bibitem{BacsoGravierGyarfasPreissmannSebHoSJDM2004}
G{\'a}bor Bacs{\'o}, Sylvain Gravier, Andr{\'a}s Gy{\'a}rf{\'a}s, Myriam
  Preissmann, and Andr{\'a}s Seb{\H{o}}.
\newblock Coloring the maximal cliques of graphs.
\newblock {\em SIAM J. Discrete Math.}, 17(3):361--376 (electronic), 2004.

\bibitem{BrandstadtLeSpinrad1999}
Andreas Brandst{\"a}dt, Van~Bang Le, and Jeremy~P. Spinrad.
\newblock {\em Graph classes: a survey}.
\newblock SIAM Monographs on Discrete Mathematics and Applications. Society for
  Industrial and Applied Mathematics (SIAM), Philadelphia, PA, 1999.

\bibitem{CerioliKorenchendler2009}
M{\'a}rcia~R. Cerioli and Andr{\'e}~L. Korenchendler.
\newblock Clique-coloring circular-arc graphs.
\newblock In {\em L{AGOS}'09---{V} {L}atin-{A}merican {A}lgorithms, {G}raphs
  and {O}ptimization {S}ymposium}, volume~35 of {\em Electron. Notes Discrete
  Math.}, pages 287--292. Elsevier Sci. B. V., Amsterdam, 2009.

\bibitem{DefossezJGT2006}
David D{\'e}fossez.
\newblock Clique-coloring some classes of odd-hole-free graphs.
\newblock {\em J. Graph Theory}, 53(3):233--249, 2006.

\bibitem{DefossezJGT2009}
David D{\'e}fossez.
\newblock Complexity of clique-coloring odd-hole-free graphs.
\newblock {\em J. Graph Theory}, 62(2):139--156, 2009.

\bibitem{EguiaSoulignacDMTCS2012}
Martiniano Egu\'{\i}a and Francisco~J. Soulignac.
\newblock Hereditary biclique-{H}elly graphs: recognition and maximal biclique
  enumeration.
\newblock arXiv: 1103.1917.

\bibitem{EiterGottlob1995}
Thomas Eiter and Georg Gottlob.
\newblock Note on the complexity of some eigenvector problems.
\newblock Technical Report CD-TR 95/89, Christian Doppler Labor f\"ur
  Expertensyteme, TU Vienna, 1995.

\bibitem{FarberDM1989}
Martin Farber.
\newblock On diameters and radii of bridged graphs.
\newblock {\em Discrete Math.}, 73(3):249--260, 1989.

\bibitem{Garey1979}
Michael~R. Garey and David~S. Johnson.
\newblock {\em Computers and intractability}.
\newblock W. H. Freeman and Co., San Francisco, Calif., 1979.
\newblock A guide to the theory of NP-completeness, A Series of Books in the
  Mathematical Sciences.

\bibitem{Golumbic2004}
Martin~Charles Golumbic.
\newblock {\em Algorithmic graph theory and perfect graphs}, volume~57 of {\em
  Annals of Discrete Mathematics}.
\newblock Elsevier Science B.V., Amsterdam, second edition, 2004.
\newblock With a foreword by Claude Berge.

\bibitem{GravierHoangMaffrayDM2003}
Sylvain Gravier, Ch{\'{\i}}nh~T. Ho{\`a}ng, and Fr{\'e}d{\'e}ric Maffray.
\newblock Coloring the hypergraph of maximal cliques of a graph with no long
  path.
\newblock {\em Discrete Math.}, 272(2-3):285--290, 2003.

\bibitem{GroshausMonteroJoGT2012}
Marina Groshaus and Leandro~P. Montero.
\newblock On the iterated biclique operator.
\newblock {\em Journal of Graph Theory}, 2012.
\newblock Available online.

\bibitem{GroshausSzwarcfiterGC2007}
Marina Groshaus and Jayme~L. Szwarcfiter.
\newblock Biclique-{H}elly graphs.
\newblock {\em Graphs Combin.}, 23(6):633--645, 2007.

\bibitem{GroshausSzwarcfiterJGT2010}
Marina Groshaus and Jayme~L. Szwarcfiter.
\newblock Biclique graphs and biclique matrices.
\newblock {\em J. Graph Theory}, 63(1):1--16, 2010.

\bibitem{GutnerTarsiDM2009}
Shai Gutner and Michael Tarsi.
\newblock Some results on {}-choosability.
\newblock {\em Discrete Math.}, 309(8):2260--2270, 2009.

\bibitem{KratochvilTuzaJA2002}
Jan Kratochv{\'{\i}}l and Zsolt Tuza.
\newblock On the complexity of bicoloring clique hypergraphs of graphs.
\newblock {\em J. Algorithms}, 45(1):40--54, 2002.

\bibitem{Lovasz1973}
L.~Lov{\'a}sz.
\newblock Coverings and coloring of hypergraphs.
\newblock In {\em Proceedings of the {F}ourth {S}outheastern {C}onference on
  {C}ombinatorics, {G}raph {T}heory, and {C}omputing ({F}lorida {A}tlantic
  {U}niv., {B}oca {R}aton, {F}la., 1973)}, pages 3--12, Winnipeg, Man., 1973.
  Utilitas Math.

\bibitem{MacedoDantasMachadoFigueiredo2012}
H\'elio~B. Mac\^edo~Filho, Simone Dantas, Raphael C.~S. Machado, and Celina
  M.~H. de~Figueiredo.
\newblock Biclique-colouring powers of paths and powers of cycles.
\newblock In {\em Proc. of the 11th Cologne-Twente Workshop on Graphs and
  Combinatorial Optimization, CTW 2012}, pages 134--138. 2012.
\newblock arXiv:1203.2543.

\bibitem{MacedoMachadoFigueiredo2012}
H\'elio~B. Mac\^edo~Filho, Raphael C.~S. Machado, and Celina M.~H.
  de~Figueiredo.
\newblock Clique-colouring and biclique-colouring unichord-free graphs.
\newblock In David Fern\'andez-Baca, editor, {\em LATIN 2012: Theoretical
  Informatics}, volume 7256 of {\em Lecture Notes in Computer Science}, pages
  530--541. Springer Berlin Heidelberg, 2012.

\bibitem{MaffrayPreissmannDM1996}
Fr{\'e}d{\'e}ric Maffray and Myriam Preissmann.
\newblock On the {NP}-completeness of the {}-colorability problem for
  triangle-free graphs.
\newblock {\em Discrete Math.}, 162(1-3):313--317, 1996.

\bibitem{MahadevPeled1995}
N.~V.~R. Mahadev and U.~N. Peled.
\newblock {\em Threshold graphs and related topics}, volume~56 of {\em Annals
  of Discrete Mathematics}.
\newblock North-Holland Publishing Co., Amsterdam, 1995.

\bibitem{MarxTCS2011}
D{\'a}niel Marx.
\newblock Complexity of clique coloring and related problems.
\newblock {\em Theoret. Comput. Sci.}, 412(29):3487--3500, 2011.

\bibitem{MoharvSkrekovskiEJC1999}
Bojan Mohar and Riste {\v{S}}krekovski.
\newblock The {G}r\"otzsch theorem for the hypergraph of maximal cliques.
\newblock {\em Electron. J. Combin.}, 6:Research Paper 26, 13 pp.\
  (electronic), 1999.

\bibitem{Papadimitriou1994}
Christos~H. Papadimitriou.
\newblock {\em Computational complexity}.
\newblock Addison-Wesley Publishing Company, Reading, MA, 1994.

\bibitem{PrisnerC2000a}
Erich Prisner.
\newblock Bicliques in graphs. {I}. {B}ounds on their number.
\newblock {\em Combinatorica}, 20(1):109--117, 2000.

\bibitem{Terlisky2010}
Pablo Terlisky.
\newblock Biclique-coloreo de grafos.
\newblock Master's thesis, Universidad de Buenos Aires, 2010.

\bibitem{TuzaC1984}
Zsolt Tuza.
\newblock Covering of graphs by complete bipartite subgraphs: complexity of
  {}-{} matrices.
\newblock {\em Combinatorica}, 4(1):111--116, 1984.

\end{thebibliography}

\end{document}
