\documentclass{article} \usepackage{iclr2018_conference,times}
\usepackage{hyperref}
\usepackage{url}
\usepackage{amsmath}\usepackage{amssymb}
\usepackage{graphicx}


\title{Few-Shot Learning with Graph Neural Networks}



\author{Victor Garcia\thanks{Work done while Victor Garcia was a visiting scholar at New York University} 
\\
Amsterdam Machine Learning Lab \\University of Amsterdam \\
Amsterdam, 1098 XH, NL\\
\texttt{v.garciasatorras@uva.nl} \\
\And
Joan Bruna \\
Courant Institute of Mathematical Sciences \\New York University \\
New York City, NY, 10010, USA\\
\texttt{bruna@cims.nyu.edu } \\
}



\newcommand{\fix}{\marginpar{FIX}}
\newcommand{\new}{\marginpar{NEW}}

\iclrfinalcopy 

\begin{document}


\maketitle

\begin{abstract}
We propose to study the problem of few-shot learning with the prism of 
inference on a partially observed graphical model, constructed from a collection 
of input images whose label can be either observed or not. By assimilating 
generic message-passing inference algorithms with their neural-network counterparts, 
we define a graph neural network architecture that generalizes several of the 
recently proposed few-shot learning models. 
Besides providing improved numerical performance, our framework is 
easily extended to variants of few-shot learning, such as semi-supervised 
or active learning, demonstrating the ability of graph-based models to 
operate well on `relational' tasks. 
\end{abstract}


\section{Introduction}



Supervised end-to-end learning has been extremely successful in computer vision, speech, or machine translation tasks, 
thanks to improvements in optimization technology, larger datasets and streamlined designs of deep convolutional or recurrent architectures. Despite these successes, this learning setup does not cover many aspects where learning is nonetheless possible and desirable. 

One such instance is the ability to learn from few examples, in the so-called few-shot learning tasks. 
Rather than relying on regularization to compensate for the lack of data, researchers have explored 
ways to leverage a distribution of similar tasks, inspired by human learning \cite{lake2015human}. 
This defines a new supervised learning setup (also called `meta-learning') in which the input-output 
pairs are no longer given by iid samples of images and their associated labels, but by iid samples of 
collections of images and their associated label similarity. 

A recent and highly-successful research program has exploited this meta-learning paradigm 
on the few-shot image classification task \cite{lake2015human, koch2015siamese, vinyals2016matching, mishra2017meta, snell2017prototypical}. In essence, these works learn a contextual, task-specific similarity measure, 
that first embeds input images using a CNN, and then learns how to combine the embedded images in the 
collection to propagate the label information towards the target image. 

In particular, \cite{vinyals2016matching} cast the few-shot learning problem 
as a supervised classification task mapping a support set of images 
into the desired label, and developed an end-to-end architecture 
accepting those support sets as input via attention mechanisms. 
In this work, we build upon this line of work, and 
 argue that this task is naturally expressed as a 
supervised interpolation problem on a graph, 
where nodes are associated 
with the images in the collection, and edges are given by a trainable similarity 
kernels. Leveraging recent progress on representation learning for 
graph-structured data \cite{bronstein2017geometric,gilmer2017neural}, we thus propose a simple 
graph-based few-shot learning model
that implements a task-driven message passing algorithm.
The resulting architecture is trained end-to-end, captures the 
invariances of the task, such as permutations within the input collections, 
and offers a good tradeoff between simplicity, generality, 
performance and sample complexity. 

Besides few-shot learning, a related task is the ability to learn from a mixture of
labeled and unlabeled examples --- semi-supervised learning, as well 
as \emph{active learning}, in which the learner has the option to request those missing labels
that will be most helpful for the prediction task. 
Our graph-based architecture is naturally extended to these setups with 
minimal changes in the training design. 
We validate experimentally the model on few-shot image classification, matching 
state-of-the-art performance with considerably fewer parameters,
and demonstrate applications to semi-supervised and active learning setups. 



Our contributions are summarized as follows:
\begin{itemize}
    \item We cast few-shot learning as a supervised message passing task which is trained end-to-end using graph neural networks.
    \item We match state-of-the-art performance on Omniglot and Mini-Imagenet tasks with fewer parameters. 
    \item We extend the model in the semi-supervised and active learning regimes. 
\end{itemize}

The rest of the paper is structured as follows. Section \ref{relatedsec} describes related work, 
Sections \ref{setupsec}, \ref{modelsec} and \ref{trainingsec} present the problem setup, our graph neural network model and 
the training, and Section \ref{experimentssec} reports numerical experiments. 










 \section{Related Work}
\label{relatedsec}
One-shot learning was first introduced by \cite{fei2006one}, they assumed that currently learned classes can help to make predictions on new ones when just one or few labels are available. More recently, \cite{lake2015human} presented a Hierarchical Bayesian model that reached human level error on few-shot learning alphabet recongition tasks.

Since then, great progress has been done in one-shot learning. \cite{koch2015siamese} presented a deep-learning model based on computing the pair-wise distance between samples using Siamese Networks, then, this learned distance can be used to solve one-shot problems by k-nearest neighbors classification.  \cite{vinyals2016matching} Presented an end-to-end trainable k-nearest neighbors using the cosine distance, they also introduced a contextual mechanism using an attention LSTM model that takes into account all the samples of the subset  when computing the pair-wise distance between samples. \cite{snell2017prototypical} extended the work from \cite{vinyals2016matching}, by using euclidean distance instead of cosine which provided significant improvements, they also build a prototype representation of each class for the few-shot learning scenario. \cite{mehrotra2017generative} trained a deep residual network together with a generative model to approximate the pair-wise distance between samples. 

A new line of meta-learners for one-shot learning is rising lately:  \cite{ravi2016optimization} introduced a meta-learning method where an LSTM updates the weights of a classifier for a given episode. \cite{munkhdalai2017meta} also presented a meta-learning architecture that learns meta-level knowledge across tasks, and it changes its inductive bias via fast parametrization. \cite{finn2017model} is using a model agnostic meta-learner based on gradient descent, the goal is to train a classification model such that given a new task, a small amount of gradient steps with few data will be enough to generalize. Lately,  \cite{mishra2017meta} used Temporal Convolutions which are deep recurrent networks based on dilated convolutions, this method also exploits contextual information from the subset  providing very good results.

Another related area of research concerns deep learning architectures on graph-structured data. 
The GNN was first proposed in \cite{gori2005new, scarselli09}, as a trainable recurrent message-passing 
whose fixed points could be adjusted discriminatively. 
Subsequent works \cite{li2015gated, sukhbaatar2016learning} have relaxed the model by untying the recurrent layer weights and proposed several nonlinear updates through gating mechanisms. 
Graph neural networks are in fact natural generalizations of convolutional networks to non-Euclidean graphs. \cite{bruna2013spectral, henaff2015deep} proposed to learn smooth spectral multipliers of the graph Laplacian, albeit with high computational cost, and \cite{defferrard2016convolutional, kipf2016semi} resolved the computational bottleneck by learning polynomials of the graph Laplacian, thus avoiding the computation of eigenvectors and completing the connection with GNNs. In particular, \cite{kipf2016semi} 
was the first to propose the use of GNNs on semi-supervised classification problems. 
We refer the reader to \cite{bronstein2017geometric} for an exhaustive literature review on the topic. 
GNNs and the analogous Neural Message Passing Models are finding application in many different domains. \cite{battaglia2016interaction, torralba16} develop graph interaction networks that learn pairwise particle interactions and apply them to discrete particle physical dynamics. \cite{duvenaud2015convolutional, kearnes2016molecular} study molecular fingerprints using variants of the GNN architecture, and \cite{gilmer2017neural} further develop the model by combining it with set representations \cite{vinyals2015order}, showing state-of-the-art results on molecular prediction.










 \section{Problem Set-up}
\label{setupsec}

We describe first the general setup 
and notations, and then particularize it 
to the case of few-shot learning, semi-supervised 
learning and active learning. 

We consider input-output 
pairs  drawn iid from a distribution  of 
partially-labeled image collections 

 for arbitrary values of  and . Where  is the number of labeled samples,  is the number of unlabeled samples ( for the semi-supervised and active learning scenarios) and  is the number of samples to classify.  is the number of classes.  We will focus in the case  where we just classify one sample per task . 
   denotes a class-specific image distribution over .
In our context, the targets  are associated with image categories of designated images  with no observed label. 
Given a training set , we consider the standard 
supervised learning objective

using the model  specified in Section \ref{modelsec} and
 is a standard regularization objective. 

 
\paragraph{Few-Shot Learning}
When ,  and , there is a single image in the collection with unknown label. 
If moreover each label appears exactly  times, this setting is referred as 
the -shot, -way learning. 

\paragraph{Semi-Supervised Learning}
When  and , the input collection contains auxiliary images  that the model can use to improve the 
prediction accuracy, by leveraging the fact that these samples are drawn from common distributions 
as those determining the output. 

\paragraph{Active Learning}
In the active learning setting, the learner has the ability to request labels from the 
sub-collection . 
We are interested in studying to what extent this active learning can 
improve the performance with respect to the previous semi-supervised setup, 
and match the performance of 
the one-shot learning setting with  known labels when , . 

\begin{figure}
    
    \centering
    \includegraphics[width=0.85\textwidth]{Figures/setting.png}

    \caption{Visual representation of One-Shot Learning setting.}
    \label{fig:setup}
\end{figure} 



\section{Model}
\label{modelsec}



This section presents our approach, based 
on a simple end-to-end graph neural network architecture. 
We first explain how the input context is mapped into a 
graphical representation, then detail the architecture, 
and next show how this model generalizes a number of previously 
published few-shot learning architectures. 

\subsection{Set and Graph Input Representations}

The input  contains a collection of images, both labeled and unlabeled. 
The goal of few-shot learning is to propagate label information from labeled samples towards 
the unlabeled query image. This propagation of information can be formalized 
as a posterior inference over a graphical model determined by the input images and 
 labels. 

Following several recent works that 
cast posterior inference using message passing with neural networks defined over graphs \cite{scarselli09, duvenaud2015convolutional, gilmer2017neural}, 
we associate  with a fully-connected graph  
where nodes  correspond to the images present in  (both labeled and unlabeled). 
In this context, the setup does not specify a fixed similarity  between 
images  and , suggesting an approach where this similarity measure 
is learnt in a discriminative fashion with a parametric model similarly as in \cite{gilmer2017neural}, such as a siamese neural architecture. 
This framework is closely related to the set representation from \cite{vinyals2016matching}, 
but extends the inference mechanism using the graph neural network formalism that we detail next.





\subsection{Graph Neural Networks}

Graph Neural Networks, introduced in \cite{gori2005new,scarselli09} and further simplified in \cite{li2015gated, duvenaud2015convolutional, sukhbaatar2016learning} are neural networks based 
on local operators of a graph , offering a powerful balance 
between expressivity and sample complexity; see \cite{bronstein2017geometric} for a recent survey on models and applications of deep learning on graphs.

In its simplest incarnation, given an input signal  on the vertices of 
a weighted graph , we consider a family  of graph intrinsic linear operators that act locally on this signal.
The simplest is the \textit{adjacency operator}  where 
 with  iff  and
 its associated weight.
A GNN layer  receives as input a signal  and produces  as \\ {\small 
}
where , 
, are trainable parameters and  is a point-wise non-linearity, chosen in this work to be a `leaky' ReLU \cite{xu2015empirical}.

Authors have explored several modeling variants from this basic formulation, by replacing the point-wise nonlinearity 
with gating operations \cite{duvenaud2015convolutional}, or by generalizing the generator family to 
Laplacian polynomials \cite{defferrard2016convolutional, kipf2016semi, bruna2013spectral}, 
or including -th powers of  to ,  to
 encode -hop neighborhoods of each node \cite{bruna2017community}.
Cascaded operations in the form (\ref{gnneq}) are able to approximate 
a wide range of graph inference tasks. In particular, inspired by message-passing algorithms, 
 \cite{kearnes2016molecular,gilmer2017neural} generalized the GNN to also learn edge features  from the 
current node hidden representation:

where  is a symmetric function parametrized with e.g. a neural network. 
In this work, we consider a Multilayer Perceptron stacked after the absolute difference between two vector nodes. See eq. \ref{metric_distance}:


Then  is a metric, which is learned by doing a non-linear combination of the absolute difference between the individual features of two nodes. Using this architecture the distance property \textit{Symmetry}   is fulfilled by construction and the distance property \textit{Identity}   is easily learned.

The trainable adjacency is then normalized to a stochastic kernel by using a softmax along each row. 
The resulting update rules for node features are obtained by adding the edge feature kernel  
into the generator family  and applying (\ref{gnneq}).
Adjacency learning is particularly important in applications where 
the input set is believed to have some geometric structure, but the metric 
is not known a priori, such as is our case.

In general graphs, the network depth is chosen to be of the order of the graph diameter, so that all nodes obtain information from the entire graph. In our context, however, since the graph is densely connected, the depth is interpreted simply as giving the model more expressive power.

\paragraph{Construction of Initial Node Features}

The input collection  is mapped into node features as follows. 
For images  with known label , the one-hot encoding of the label is concatenated with the embedding features of the image at the input of the GNN.

where  is a Convolutional neural network and  is a one-hot encoding of the label. Architectural
details for  are detailed in Section \ref{architecture_omniglot} and \ref{architecture_mini_imagenet}. 
For images  with unknown label , we modify the previous construction 
to account for full uncertainty about the label variable by replacing  with the uniform 
distribution over the -simplex: , and analogously 
for .

\begin{figure}
    
    \centering
    \includegraphics[width=0.85\textwidth]{Figures/gnn3.png}

    \caption{Graph Neural Network ilustration. The Adjacency matrix is computed before every Convolutional Layer.}
    \label{fig:gnn2}
\end{figure} 






















\subsection{Relationship with Existing Models}


The graph neural network formulation of few-shot learning generalizes 
a number of recent models proposed in the literature.

\paragraph{Siamese Networks}
Siamese Networks \cite{koch2015siamese} can be interpreted as 
a single layer message-passing iteration of our model, and using the same initial
node embedding (\ref{nodeinit})  , using a non-trainable 
edge feature 

and resulting label estimation

with  selecting the label field from . In this model, 
the learning is reduced to learning image embeddings  whose 
euclidean metric is consistent with the label similarities. 



\paragraph{Prototypical Networks}


Prototypical networks \cite{snell2017prototypical} evolve Siamese networks
by aggregating information within each cluster determined by nodes with the same
label. This operation can also be accomplished with a gnn as follows. 
we consider 

where  is the number of examples per class, and 

where  is defined as in the Siamese Networks.
We finally apply the previous kernel  
applied to  to yield class prototypes:


\paragraph{Matching Networks}

Matching networks \cite{vinyals2016matching} use a set representation for the ensemble of images in , 
similarly as our proposed graph neural network model, but with two important differences. 
First, the attention mechanism considered in this set representation
is akin to the edge feature learning, with the difference that the mechanism attends always to the same node embeddings, as opposed to our stacked adjacency learning, which is closer to \cite{vaswani2017attention}. In other words, instead of the attention kernel in (\ref{edgefeats}), 
 matching networks consider attention mechanisms of the form , 
 where  is the encoding function for the elements of the support set, obtained with bidirectional LSTMs. In that case, the support set encoding is thus computed independently of the target image.
Second, the label and image fields are treated separately throughout the model, with a final step that aggregates linearly the labels using a trained kernel. This may prevent the model to leverage complex dependencies between labels and images at intermediate stages.
















\section{Training}
\label{trainingsec}

We describe next how to train the parameters of the GNN in the 
different setups we consider: few-shot learning, semi-supervised learning 
and active learning. 

\subsection{Few-Shot and Semi-Supervised Learning}
\label{few_shot_semi_superv}


In this setup, the model is asked only to predict the label  corresponding to the 
image to classify , associated with node  in the graph. 
The final layer of the GNN is thus a softmax mapping
the node features to the -simplex. We then consider the Cross-entropy loss
evaluated at node :


The semi-supervised setting is trained identically --- the only difference is that 
the initial label fields of the node will be filled with the uniform distribution on nodes 
corresponding to . 

\subsection{Active Learning}
In the Active Learning setup, the model has the intrinsic ability to query for one of the labels from . The network will learn to ask for the most informative label in order to classify the sample . The querying is done after the first layer of the GNN by using a Softmax attention over the unlabeled nodes of the graph. For this we apply a function  that maps each unlabeled vector node to a scalar value. Function  is parametrized by a two layers neural network. A Softmax is applied over the  scalar values obtained after applying :

 
In order to query only one sample, we set all elements from the  vector to 0 except for one. At test time we keep the maximum value, at train time we randomly sample one value based on its multinomial probability. Then we multiply this sampled attention by the label vectors: 

The label of the queried vector  is obtained, scaled by the weight . This value is then summed to the current representation , since we are using dense connections in our GNN model we can sum this  value directly to where the uniform label distribution was concatenated 

After the label has been summed to the current node, the information is forward propagated. This attention part is trained end-to-end with the rest of the network by backpropagating the loss from the output of the GNN. 


 \section{Experiments}
\label{experimentssec}
For the few-shot, semi-supervised and active learning experiments we used the Omniglot dataset presented by \cite{lake2015human} and Mini-Imagenet dataset introduced by \cite{vinyals2016matching} which is a small version of ILSVRC-12 \cite{krizhevsky2012imagenet}. All experiments are based on the -shot, -way setting. For all experiments we used the same values -shot and -way for both training and testing.
\newline
{\scriptsize Code available at: \url{https://github.com/vgsatorras/few-shot-gnn}}
\subsection{Datasets and Implementation}
\subsubsection{Omniglot}

\paragraph{Dataset:}
Omniglot is a dataset of 1623 characters from 50 different alphabets, each character/class has been drawn by 20 different people. Following \cite{vinyals2016matching} implementation we split the dataset into 1200 classes for training and the remaining 423 for testing. We augmented the dataset by multiples of 90 degrees as proposed by \cite{santoro2016meta}.

\paragraph{Architectures:}
\label{architecture_omniglot}
Inspired by the embedding architecture from \cite{vinyals2016matching}, following \cite{mishra2017meta}, a CNN was used as an embedding  function consisting of four stacked blocks of 33-convolutional layer with 64 filters, batch-normalization, 22 max-pooling, leaky-relu\} the output is passed through a fully connected layer resulting in a 64-dimensional embedding. For the GNN we used 3 blocks each of them composed by 1) a module that computes the adjacency matrix and 2) a graph convolutional layer. A more detailed description of each block can be found at Figure \ref{fig:GNN_model}.


\subsubsection{Mini-Imagenet}
\paragraph{Dataset:}  Mini-Imagenet is a more challenging dataset for one-shot learning proposed by \cite{vinyals2016matching} derived from the original ILSVRC-12 dataset \cite{krizhevsky2012imagenet}. It consists of 8484 RGB images from 100 different classes with 600 samples per class. It was created with the purpose of increasing the complexity for one-shot tasks while keeping the simplicity of a light size dataset, that makes it suitable for fast prototyping. We used the splits proposed by \cite{ravi2016optimization} of 64 classes for training, 16 for validation and 20 for testing. Using 64 classes for training, and the 16 validation classes only for early stopping and parameter tuning.

\paragraph{Architecture:}
\label{architecture_mini_imagenet}
The embedding architecture used for Mini-Imagenet is formed by 4 convolutional layers followed by a fully-connected layer resulting in a 128 dimensional embedding. This light architecture is useful for fast prototyping: 
\newline
133-conv. layer (64 filters), batch normalization, max pool, leaky relu,
\newline
133-conv. layer (96 filters), batch normalization, max pool, leaky relu,
\newline
133-conv. layer (128 filters), batch normalization, max pool, leaky relu, dropout, 
\newline
133-conv. layer (256 filters), batch normalization, max pool, leaky relu, dropout,
\newline
1 fc-layer (128 filters), batch normalization. 
\newline
The two dropout layers are useful to avoid overfitting the GNN in Mini-Imagenet dataset.
The GNN architecture is similar than for Omniglot, it is formed by 3 blocks, each block is described at Figure \ref{fig:GNN_model}.

\subsection{Few-Shot}
Few-shot learning experiments for Omniglot and Mini-Imagenet are presented at Table \ref{table:few_shot_omniglot} and Table \ref{table:few_shot_mini_imagenet} respectively. 

We evaluate our model by performing different q-shot, K-way experiments on both datasets. For every few-shot task , we sample \textit{K} random classes from the dataset, and from each class we sample \textit{q} random samples. An extra sample to classify is chosen from one of that \textit{K} classes.



\textit{Omniglot}: The GNN method is providing competitive results while still remaining simpler than other methods. State of the art results are reached in the 5-Way and 20-way 1-shot experiments. In the 20-Way 1-shot setting the GNN is providing slightly better results than \cite{munkhdalai2017meta} while still being a more simple approach. The TCML approach from \cite{mishra2017meta} is in the same confidence interval for 3 out of 4 experiments, but it is slightly better for the 20-Way 5-shot, although the number of parameters is reduced from 5M (TCML) to 300K (3 layers GNN).


At \textit{Mini-Imagenet} table we are also presenting a baseline "\textit{Our metric learning + KNN}" where no information has been aggregated among nodes, it is a K-nearest neighbors applied on top of the pair-wise learnable metric  and trained end-to-end, this learnable metric is competitive by itself compared to other state of the art methods. Even so, a significant improvement (from 64.02\% to 66.41\%) can be seen for the 5-shot 5-Way Mini-Imagenet setting when aggregating information among nodes by using the full GNN architecture. A variety of embedding functions  are used among the different papers for Mini-Imagenet experiments, in our case we are using a simple network of 4 conv. layers followed by a fully connected layer (Section \ref{architecture_mini_imagenet}) which served us to compare between \textit{Our GNN} and \textit{Our metric learning + KNN} and it is useful for fast prototyping. More complex embeddings have proven to produce better results, at \cite{mishra2017meta} a deep residual network is used as embedding network  increasing the accuracy considerably. Regarding the TCML architecture in Mini-Imagenet, the number of parameters is reduced from 11M (TCML) to 400K (3 layers GNN).






\begin{table*}[!htb]
\centering
\tabcolsep=0.08cm
\footnotesize
\makebox[0pt]{
\begin{tabular}{l*{5}{c}} 

\hline & \multicolumn{2}{c}{\textbf{5-Way}} &   \multicolumn{2}{c}{\textbf{20-Way}} \\
\textbf{Model} & \textbf{1-shot} & \textbf{5-shot}  & \textbf{1-shot} & \textbf{5-shot} & \\ 
\hline

\textbf{Pixels} {\scriptsize\cite{vinyals2016matching}}
    & 41.7\%    &   63.2\%    &  26.7\%    &   42.6\%  \\  
\hline
\textbf{Siamese Net} {\scriptsize\cite{koch2015siamese}}
    & 97.3\%    &   98.4\%    & 88.2\%    &   97.0\%  \\  
\textbf{Matching Networks} {\scriptsize\cite{vinyals2016matching}}
    & 98.1\%    &   98.9\%    & 93.8\%    &   98.5\%  \\  
\textbf{N. Statistician} {\scriptsize\cite{edwards2016towards}}
    & 98.1\%    &   99.5\%    & 93.2\%    &   98.1\%  \\  
\textbf{Res. Pair-Wise} {\scriptsize\cite{mehrotra2017generative}}
    & -    &   -    & 94.8\%    &   -  \\  
\textbf{Prototypical Networks} {\scriptsize\cite{snell2017prototypical}}
& 97.4\%    &   99.3\%    & 95.4\%    &   98.8\% \\

\textbf{ConvNet with Memory} {\scriptsize\cite{kaiser2017learning}}
& 98.4\%    &   99.6\%    & 95.0\%    &   98.6\%  \\

\textbf{Agnostic Meta-learner} {\scriptsize\cite{finn2017model}}
& 98.7 0.4\%    &   99.9 0.3\%    & 95.8 0.3\%    &   98.9 0.2\%  \\   

\textbf{Meta Networks} {\scriptsize\cite{munkhdalai2017meta}}
& 98.9\%    &   -    & 97.0\%    &   -  \\   
\textbf{TCML} {\scriptsize\cite{mishra2017meta}}
& 98.96\% 0.20\%     &   99.75\% 0.11\%     & 97.64\% 0.30\%    &   99.36\% 0.18\% \\   
\hline
\textbf{Our GNN} 
    & 99.2\%     &  99.7\%  & 97.4\%  & 99.0\% \\
\hline
\end{tabular}
}
   \caption{Few-Shot Learning | Omniglot accuracies. Siamese Net results are extracted from \cite{vinyals2016matching} reimplementation.}
\label{table:few_shot_omniglot}
\end{table*}



\begin{table*}[t!]
\centering
\begin{tabular}{l*{4}{c}} 

\hline &  \multicolumn{2}{c}{\textbf{5-Way}} \\
\textbf{Model} & \textbf{1-shot} & \textbf{5-shot}  \\ 
\hline

\textbf{Matching Networks} {\scriptsize\cite{vinyals2016matching}}
    & 43.6\%    &   55.3\%    \\  
\textbf{Prototypical Networks} {\scriptsize\cite{snell2017prototypical}} 
& 46.61\% 0.78\%    &   65.77\% 0.70\% \\   
\textbf{Model Agnostic Meta-learner} {\scriptsize\cite{finn2017model}}
& 48.70\% 1.84\%    &   63.1\% 0.92\%    \\   
\textbf{Meta Networks} {\scriptsize\cite{munkhdalai2017meta}}
& 49.21\% 0.96    &   -    \\   
\textbf{\citeauthor{ravi2016optimization}} {\scriptsize\cite{ravi2016optimization}}
& 43.4\% 0.77\% & 60.2\% 0.71\% \\
\textbf{TCML} {\scriptsize\cite{mishra2017meta}} 
& 55.71\% 0.99\%   &   68.88\% \%    \\   
\hline
\textbf{Our metric learning + KNN}
        & 49.44\% 0.28\%     &  64.02\% 0.51\% \\
\textbf{Our GNN} 
        & 50.33\% 0.36\%     &  66.41\% 0.63\% \\
\hline
\end{tabular}
   \caption{Few-shot learning | Mini-Imagenet  average accuracies with 95\% confidence intervals.}
\label{table:few_shot_mini_imagenet}
\end{table*}




\subsection{Semi-Supervised}
Semi-supervised experiments are performed on the 5-way 5-shot setting. Different results are presented when 20\% and 40\% of the samples are labeled. The labeled samples are balanced among classes in all experiments, in other words, all the classes have the same amount of labeled and unlabeled samples.

Two strategies can be seen at Tables \ref{table:semi_supervised_omniglot} and \ref{table:semi_supervised_mini_imagenet}. \textit{"GNN - Trained only with labeled"} is equivalent to the supervised few-shot setting, for example, in the 5-Way 5-shot 20\%-labeled setting, this method is equivalent to the 5-way 1-shot learning setting since it is ignoring the unlabeled samples. \textit{"GNN - Semi supervised"} is the actual semi-supervised method, for example, in the 5-Way 5-shot 20\%-labeled setting, the GNN receives as input 1 labeled sample per class and 4 unlabeled samples per class.


Omniglot results are presented at Table \ref{table:semi_supervised_omniglot}, for this scenario we observe that the accuracy improvement is similar when adding images than when adding labels. The GNN is able to extract information from the input distribution of unlabeled samples such that only using 20\% of the labels in a 5-shot semi-supervised environment we get same results as in the 40\% supervised setting.


In Mini-Imagenet experiments, Table \ref{table:semi_supervised_mini_imagenet}, we also notice an improvement when using semi-supervised data although it is not as significant as in Omniglot. The distribution of Mini-Imagenet images is more complex than for Omniglot. In spite of it, the GNN manages to improve by  2\% in the 20\% and 40\% settings.


\begin{table*}[!htb]
\centering
\small
\begin{tabular}{l*{4}{c}} 

\hline & \multicolumn{3}{c}{\textbf{5-Way 5-shot}} &   \\
\textbf{Model} & \textbf{20\%-labeled} & \textbf{40\%-labeled}  &  \textbf{100\%-labeled} & \\ 
\hline

\textbf{GNN - Trained only with labeled}
    & 99.18\%    &   99.59\%    &  99.71\%    \\  
\textbf{GNN - Semi supervised}
    & 99.59\%    &   99.63\%    &  99.71\%   \\  

\hline

\end{tabular}
   \caption{Semi-Supervised Learning | Omniglot accuracies.}
\label{table:semi_supervised_omniglot}
\end{table*}


\begin{table*}[!htb]
\centering
\small
\begin{tabular}{l*{4}{c}} 

\hline & \multicolumn{3}{c}{\textbf{5-Way 5-shot}} &   \\
\textbf{Model} & \textbf{20\%-labeled} & \textbf{40\%-labeled}  &  \textbf{100\%-labeled} & \\ 
\hline

\textbf{GNN - Trained only with labeled}
    & 50.33\% 0.36\%    &   56.91\% 0.42\%    &  66.41\% 0.63\%    \\  
\textbf{GNN - Semi supervised}
    & 52.45\% 0.88\%    &   58.76\% 0.86\%    &  66.41\% 0.63\%   \\  

\hline

\end{tabular}
   \caption{Semi-Supervised Learning | Mini-Imagenet average accuracies with 95\% confidence intervals.}
\label{table:semi_supervised_mini_imagenet}
\end{table*}


\subsection{Active Learning}
We performed Active Learning experiments on the 5-Way 5-shot set-up when 20\% of the samples are labeled. In this scenario our network will query for the label of one sample from the unlabeled ones. The results are compared with the \textit{Random} baseline where the network chooses a random sample to be labeled instead of one that maximally reduces the loss of the classification task .

Results are shown at Table \ref{fig:dummy_gaussian_warm}. The results of the GNN-Random criterion are close to the Semi-supervised results for 20\%-labeled samples from Tables \ref{table:semi_supervised_omniglot} and \ref{table:semi_supervised_mini_imagenet}. It means that selecting one random label practically does not improve the accuracy at all. When using the GNN-AL learned criterion, we notice an improvement of  for Mini-Imagenet, it means that the GNN manages to correctly choose a more informative sample than a random one. In Omniglot the improvement is smaller since the accuracy is almost saturated and the improving margin is less.


\begin{table}[!htb]
\minipage{0.5\textwidth}
    \centering
    \tabcolsep=0.11cm
    \begin{tabular}{l*{2}{c}r}
    
    \textbf{Method}             & \textbf{5-Way 5-shot 20\%-labeled}   \\
    \hline
     \textbf{GNN - AL} & 99.62\%    \\
     \textbf{GNN - Random} & 99.59\%   \\

    \end{tabular}
\endminipage\hfill
\minipage{0.5\textwidth}
    \centering
    \tabcolsep=0.11cm
    \begin{tabular}{l*{6}{c}r}
    \textbf{Method}             & \textbf{5-Way 5-shot 20\%-labeled}   \\
    \hline
     \textbf{GNN - AL} & 55.99\% 1.35\%   \\
    \textbf{GNN - Random} & 52.56\% 1.18\%  \\

    \end{tabular}
\endminipage\hfill
\caption{Omniglot (left) and Mini-Imagneet (right), average accuracies are shown at both tables, the GNN-AL is the learned criterion that performs Active Learning by selecting the sample that will maximally reduce the loss of the current classification. The GNN - Random is also selecting one sample, but in this case a random one. Mini-Imagenet results are presented with 95\% confidence intervals.}
\label{fig:dummy_gaussian_warm}
\end{table} 
 
\section{Conclusions}


This paper explored graph neural representations for 
few-shot, semi-supervised and active learning. 
From the meta-learning perspective, these tasks become 
supervised learning problems where the input is given by 
a collection or set of elements, whose relational structure 
can be leveraged with neural message passing models. In particular,
stacked node and edge features generalize 
the contextual similarity learning underpinning previous 
few-shot learning models. 

The graph formulation is helpful to unify several training 
setups (few-shot, active, semi-supervised) under the same 
framework, a necessary step towards the goal of having a 
single learner which is able to operate simultaneously in 
different regimes (stream of labels with few examples per class, 
or stream of examples with few labels). This general 
goal requires scaling up graph models to millions of nodes, 
motivating graph hierarchical and coarsening approaches \cite{defferrard2016convolutional}.

Another future direction is to generalize the scope of 
 Active Learning, to include e.g. the ability to ask questions \cite{brenden17}, 
or in reinforcement learning setups, where few-shot learning is critical to 
adapt to non-stationary environments. 

 

\subsubsection*{Acknowledgments}
This work was partly supported by Samsung Electronics (Improving Deep Learning using Latent Structure).


\bibliography{iclr2018_conference}
\bibliographystyle{iclr2018_conference}

\newpage
\section*{Appendix}

\begin{figure}[h]
    
    \centering
    \includegraphics[width=0.85\textwidth]{Figures/gnn_specifications.jpg}

    \caption{GNN model. Three blue blocks are used for Omniglot and Mini-Imagenet. (nf=96).}
    \label{fig:GNN_model}
\end{figure} 


\end{document}
