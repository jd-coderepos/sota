\documentclass{llncs}

\usepackage[margin=1.1in]{geometry}

\usepackage{times}
\usepackage{color}
\usepackage{multirow}


\usepackage{amsmath}
\usepackage{amssymb}

\usepackage{cleveref}

\newcommand{\hide}[1]{}

\newcommand{\set}[1]{\left\{#1\right\}}
\newcommand{\ceil}[1]{\left\lceil #1 \right\rceil}
\renewcommand{\Pr}{\mathbb{P}}

\newcommand{\eps}{\varepsilon}
\newcommand{\calT}{\mathcal{T}}
\newcommand{\calS}{\mathcal{S}}

\newcommand{\TODO}[1]{ {\color{blue} TODO: #1 }}


\renewcommand{\paragraph}[1]{\medskip\noindent{\bf #1\ }}

\author{Bernhard Haeupler\\MIT CSAIL\\Cambridge, MA, USA\\
\texttt{haeupler@mit.edu}\and Fabian Kuhn\\ Dept.\ of Computer Science\\
University of Freiburg, Germany
\\\texttt{kuhn@informatik.uni-freiburg.de}}

\date{}

\begin{document}

\title{Lower Bounds on Information Dissemination in Dynamic Networks}
\author{Bernhard Haeupler\inst{1} \and Fabian Kuhn\inst{2}}
\institute{Computer Science and Artificial Intelligence Lab, MIT, USA\\
\email{haeupler@mit.edu} 
\and
Dept.\ of Computer Science, University of Freiburg, Germany
\email{kuhn@cs.uni-freiburg.de}}
\date{}

\maketitle

\begin{abstract}
    We study lower bounds on information dissemination in adversarial
    dynamic networks.
Initially,  pieces of information (henceforth called tokens)
    are distributed among  nodes. The tokens need to be broadcast
    to all nodes through a synchronous network in which the topology
    can change arbitrarily from round to round provided that some
    connectivity requirements are satisfied.

    If the network is guaranteed to be connected in every round and
    each node can broadcast a single token per round to its neighbors,
    there is a simple token dissemination algorithm that manages to
    deliver all  tokens to all the nodes in 
    rounds. Interestingly, in a recent paper, Dutta et al. proved an
    almost matching  lower bound for
    deterministic token-forwarding algorithms that are not allowed to
    combine, split, or change tokens in any way. In the present paper,
    we extend this bound in different ways.

    If nodes are allowed to forward  tokens instead of only
    one token in every round, a straight-forward extension of the
     algorithm disseminates all  tokens in time
    . We show that for any randomized token-forwarding
    algorithm,  rounds are
    necessary. If nodes can only send a single token per round, but we
    are guaranteed that the network graph is -vertex connected in
    every round, we show a lower bound of , which
    almost matches the currently best  upper bound. Further,
    if the network is -interval connected, a notion that captures
    connection stability over time, we prove that  rounds are needed. The best known upper bound in
    this case manages to solve the problem in 
    rounds. Finally, we show that even if each node only needs to
    obtain a -fraction of all the tokens for some ,  are still required.
\end{abstract}

\section{Introduction}

The growing abundance of (mobile) computation and communication
devices creates a rich potential for novel distributed systems and
applications. Unlike classical networks, often the resulting networks
and applications are characterized by a high level of churn and,
especially in the case of mobile devices, a potentially constantly
changing topology. Traditionally, changes in a network have been
studied as faults or as exceptional events that have to be tolerated
and possibly repaired. However, particularly in mobile applications,
dynamic networks are a typical case and distributed algorithms
have to properly work even under the assumption that the topology is constantly changing.

Consequently, in the last few years, there has been an increasing
interest in distributed algorithms that run in dynamic
systems. Specifically, a number of recent papers investigate the
complexity of solving fundamental distributed computations and
information dissemination tasks in dynamic networks, e.g.,
\cite{avin08,baumann09,clementi08,clementi09,cornejo12,LBarxiv,odell05,HK,KLO,KMO,KOSurvey}.
Particularly important in the context of this paper is the
synchronous, adversarial dynamic network model defined in
\cite{KLO}. While the network consists of a fixed set of participants
, the topology can change arbitrarily from round to round, subject
to the restriction that the network of each round needs to be
connected or satisfy some stronger connectivity requirement.

We study lower bounds on the problem of disseminating a
bunch of tokens (messages) to all the nodes in a dynamic network as
defined in \cite{KLO}.\footnote{To be in line with \cite{KLO} and
    other previous work, we refer to the information pieces to be
    disseminated in the network as tokens.} Initially  tokens are
placed at some nodes in the network. 
Time is divided into
synchronous rounds, the network graph of every round is connected, and
in every round, each node can broadcast one token to all its
neighbors. If in addition, all nodes know the size of the network ,
we can use the following basic protocol to broadcast all  tokens to
all the nodes. The tokens are broadcast one after the other such that
for each token during  rounds, every node that knows about the
token forwards it. Because in each round, there has to be an edge
between the nodes knowing the token and the nodes not knowing it, at
least one new node receives the token in every round and thus, after
 rounds, all nodes know the token. Assuming that only one token
can be broadcast in a single message, the algorithm requires 
rounds to disseminate all  tokens to all the nodes.

Even though the described approach seems almost trivial, as long as we
do not consider protocols based on network coding,  is the best
upper bound known.\footnote{In fact, if tokens and thus also messages
    are restricted to a polylogarithmic number of bits, even network
    coding does not seem to yield more than a polylog.\
    improvement \cite{AnalyzingNC,HK}.} In \cite{KLO}, a
\emph{token-forwarding algorithm} is defined as an algorithm that
needs to forward tokens as they are and is not allowed to combine or
change tokens in any way. Note that the algorithm above is a
token-forwarding algorithm. In a recent paper, Dutta et al.\ show that
for deterministic token-forwarding algorithms, the described simple
strategy indeed cannot be significantly improved by showing a lower
bound of  rounds \cite{LBarxiv}. Their lower bound
is based on the following observation. Assume that initially, every
node receives every token for free with probability 
(independently for all nodes and tokens). Now, with high probability,
whatever tokens the nodes decide to broadcast in the next round, the
adversary can always find a graph in which new tokens are learned
across at most  edges. Hence, in each round, at most
 tokens are learned.  Because also after randomly assigning
tokens with probability , overall still roughly  tokens are
missing, the lower bound follows. We extend the
lower bound from \cite{LBarxiv} in various natural
directions. Specifically, we make the contributions listed in the
following. All our lower bounds hold for deterministic algorithms and
for randomized algorithms assuming a strongly adaptive adversary (cf.\
Section \ref{sec:model}). Our results are also summarized in
\Cref{table:bounds} which is discussed in Section \ref{sec:related}.

\paragraph{Multiple Tokens per Round:} Assume that instead of
forwarding a single token per round, each node is allowed to forward
up to  tokens in each round. In the simple token-forwarding
algorithm that we described above, we can then forward a block of 
tokens to every node in  rounds and we therefore get an
 round upper bound.  We show that every
(randomized) token-forwarding algorithm needs at least  rounds. 

\paragraph{Interval Connectivity:} It is natural to assume that a
dynamic network cannot change arbitrarily from round to round and that
some paths remain stable for a while. This is formally captured by the
notion of interval connectivity as defined in \cite{KLO}. A network is
called -interval connected for an integer parameter  if
for any  consecutive rounds, there is a stable connected
subgraph. It is shown in \cite{KLO} that in a -interval connected
dynamic network, -token dissemination can be solved in  rounds. In this paper, we show that every
(randomized) token-forwarding algorithm needs at least  rounds. 

\paragraph{Vertex Connectivity:} If instead of merely requiring that
the network is connected in every round, we assume that the network is
-vertex connected in every round for some , we can also obtain
a speed-up. Because in a -vertex connected graph, every vertex cut
has size at least , if in a round all nodes that know a token 
broadcast it, at least  new nodes are reached. The basic
token-forwarding algorithm thus leads to an 
upper bound. We prove this upper bound tight up to a small factor by
showing an  lower bound.

\paragraph{\boldmath-Partial Token Dissemination:} 
Finally we consider the basic model, but relax the requirement on the problem
by requiring that every node needs to obtain only a -fraction
of all the  tokens for some parameter . We show
that even then, at least 
rounds are needed. This also has implications for algorithms that use
forward error correcting codes (FEC) to forward coded packets instead
of tokens.  We show that such algorithms still need at least
 rounds until
every node has received enough coded packets to decode all 
tokens.  


\section{Related Work}
\label{sec:related}

\begin{table}[t]
\centering
\begin{tabular}{ | l || c | c | c | } \hline
    & \ TF Alg. \cite{KLO} \  & \ NC Alg. \cite{AnalyzingNC,HK,HM}\  & \ TF Lower Bound \ \\ \hline \hline 
\multirow{2}{*}{always connected} & \multirow{2}{*}{ } 	&  \
\ \ \ \ \ \ \ \   \ \ \ \ \ \ \small (1)  &  \
\ \ \ \ \ \ \ \    \ \ \ \ \ \ \ \ \small\cite{KLO}\\ 
 & &  \ \ \ \ \ \ \ \ \   \ \ \ \ \small (2) &  \ \ \ \ \ \ \ \ \    \ \ \ \ \ \ \ \ \ \ \ \ \small \cite{LBarxiv} \\ \hline


-interval conn.  	& \multirow{2}{*}{\ \ \ \ \ \ \ \ 
      \ \small (+,*)  } 	& \ \ \ \ \ \
 \, \small (1,*)& \multirow{2}{*}{\ \ \ \
    {\boldmath} \ \ \small (+) } \\ 

-stability	&  	& \ \ \ \ \ \ \ \ \ \ 	 \ \ \ \ \small (2) &   \\ \hline


\multirow{2}{*}{always -connected}  			& \multirow{2}{*}{ } 	&  \ \ \ \ \ \ \ \ \ \ 	 \ \ \,\, \small  (1) &  {\boldmath} \\ 

			& &	 \ \ \ \ \ \ \ \ \ \  \ \ \ \ \ \ \ \small (2)& {
                            {\boldmath}
                            }  \\ \hline



\multirow{2}{*}{-token packets}  			&
\multirow{2}{*}{ } 	& \ \ \ \ \ \  	 \ \ \ \small (1)&  \multirow{2}{*}{\boldmath}  \\ 

&  	&\ \ \ \ \ \ \ \ \ \ 	 \ \ \ \ \small (2)&   \\ \hline


\multirow{2}{*}{-partial token diss.} & \multirow{2}{*}{} 	&\ \ \ \ \ \ \ \ \ \ 	  \ \ \ \ \ \ \ \small (1)  &  \multirow{2}{*}{ {\boldmath} }\\ 
 & & \ \ \ \ \ \ \ \ \ \  \ \,\ \small (2) & \\ \hline
\end{tabular}\
    \label{eq:potential}
    \Phi(t) := \sum_{u\in V} \left|K_u(t) \cup K_u'(t)\right|.
\label{eq:freeedge}
    \big(i_u(r)\in K'_v(r-1) \land i_v(r)\in K'_u(r-1)\big) \lor \big(i_u(r)=i_v(r)\big).

    \label{eq:edgeweight}
    w(e) := \left|I_v(r)\setminus (K_u(r\!-\!1)\cup K_u'(r\!-\!1))\right| + 
    \left|I_u(r)\setminus (K_v(r\!-\!1)\cup K_v'(r\!-\!1))\right|.

    \exp\left((\ell+1)\cdot\left(
            \ln n + b\ln k + \ell+1 - 
            \frac{\ell w}{12}\ln\left(\frac{w}{\eps}\right)
        \right)\right).
    
    \Pr[\mathcal{E}_i] \leq {\ell b\choose \ell w/4} \cdot
    \left(\frac{\eps}{4e b}\right)^{\ell w/4} \leq
    \left(\frac{4e\ell b}{\ell w}\right)^{\ell
        w/4}\cdot\left(\frac{\eps}{4e b}\right)^{\ell w /4} =
    \left(\frac{\eps}{w}\right)^{\ell w /4}.
    
    \Pr\left[X\geq \frac{\ell+1}{3}\right] \leq 
    {\ell+1\choose (\ell+1)/3}\cdot
    \left(\frac{\eps}{w}\right)^{\frac{\ell w}{4}\cdot \frac{\ell+1}{3}}
    \leq
    2^{\ell+1}\cdot
    \left(\frac{\eps}{w}\right)^{\frac{\ell w}{4}\cdot \frac{\ell+1}{3}}.
    
    {n\choose \ell+1}\cdot {k\choose b}^{\ell+1} \leq 
    \left(n k^b\right)^{\ell+1}.
    
    \Omega\left( \frac{nk}{(\log n + b\log k)b\log\log b} \right) \geq 
    \Omega\left( \frac{nk}{b^2 \log n \log \log n} \right)
    
    \sum_{i=0}^{\log b} 2w_i\cdot\ell_i = \sum_{i=0}^{\log b}2^{i+1}\cdot\ell_i.
    
    \ell_i = O\left(\frac{\log n + b\log k}{w_i\log w_i}\right) =
    O\left(\frac{\log n + b\log k}{2^i\cdot i}\right)
    
    \sum_{i=0}^{\log b}O\left(\frac{\log n + b\log k}{i}\right) = 
    O\big((\log n + b\log k)\log\log b\big).
    
    \Omega\left( \frac{nk}{T(T\log k + \log n)} \right) \geq 
    \Omega\left( \frac{nk}{T^2 \log n} \right) 
    
     \sum_i c - |\Gamma(X_i)| \leq 
     \sum_i c - (2c-2) - |X_i| + 1 = \sum_i |X_i| - \sum_i (c-1) \leq
     n. \vspace{-0.5cm}
     
    \Pr\left[ \min \set{|L|\ :\ L\subseteq[\tau]\land 
            \sum_{i \in \{j | X_j = 1\} \cup L} \ell_i \geq
        c} > x \right] < 2^{-\Theta(\frac{x\ell}{c})}.
    
    {s\choose \beta\log n} \leq 
    \left(\frac{es}{\beta\log n}\right)^{\beta\log n} =
    \left(\frac{e\alpha c}{\beta}\right)^{\beta\log n} =
    2^{\Theta(\log c \log n)},
    
    which is less than  for sufficiently large
    . Hence, with probability at least , for
    at most  tokens  on the left side there are less
    than  nodes  on the right that have .  For these  tokens , we add to  to
     for at most  nodes  on the right side,
    such that afterwards, for every token  sent by a node on the
    left side, there are at least  nodes  on the
    right for which . Note that this increases the
    potential function by at most .

    We next show that by adding another  tokens to
    the -sets of the nodes on the left side, we manage to get that
    every node  on the left side has at least  free neighbors
    on the right side. For a token  sent by some node  on
    the left side and a token  sent by some node  on the right
    side, let  be the number of nodes  for which
    . Note that if token  is in  for some ,  has  neighbors in .

    Using the augmentation of the -sets for nodes on the right,
    we have that for every , . For every , with probability , we
    have . In addition, we add tokens additional  to
     for which  such that in the end,
    . By Lemma
    \ref{lem:increase}, the probability that we need to add  tokens
    is upper bounded by . As the number of tokens we need to add to
     is independent for different , in total we need to add
    at most  tokens with probability at least
    . Note that this is still true after a
    union bound over all the possible ways to distributed the
     tokens among the  nodes. Using Lemma
    \ref{lem:kconnectmindegree}, we then have to add at most
     additional non-free edges to make the graph
    induced by  -vertex connected.
        
    There are at most  ways  to choose the set 
    and  ways to assign tokens to the nodes in
    . Hence, if we choose  sufficiently large, the
    probability that we need to increase the potential by at most
     for all sets  and all token assignments is
    positive. The theorem now follows as in the previous lower bounds
    (e.g., as in the proof of Theorem \ref{thm:original}).
\hspace*{\fill}\qed\end{proof}





\section*{Acknowledgments}

We would like to thank Chinmoy Dutta, Gopal Pandurangan, Rajmohan
Rajaraman, and Zhifeng Sun for helpful discussions and for sharing
their work at an early stage.

\clearpage

\bibliographystyle{abbrv}


\begin{thebibliography}{10}

\bibitem{anta10}
A.~F. Anta, A.~Milani, M.~A. Mosteiro, and S.~Zaks.
\newblock Opportunistic information dissemination in mobile ad-hoc networks:
  The profit of global synchrony.
\newblock In {\em Proc.\ 24th Int.\ Symp.\ on Distributed Computing (DISC)},
  pages 374--388, 2010.

\bibitem{avin08}
C.~Avin, M.~Kouck\'{y}, and Z.~Lotker.
\newblock How to explore a fast-changing world (cover time of a simple random
  walk on evolving graphs).
\newblock In {\em Proc. of 35th Coll. on Automata, Languages and Programming
  (ICALP)}, pages 121--132, 2008.

\bibitem{baumann09}
H.~Baumann, P.~Crescenzi, and P.~Fraigniaud.
\newblock Parsimonious flooding in dynamic graphs.
\newblock In {\em Proc.\ 28th ACM Symp. on Principles of Distributed Computing
  (PODC)}, pages 260--269, 2009.

\bibitem{clementi08}
A.~Clementi, C.~Macci, A.~Monti, F.~Pasquale, and R.~Silvestri.
\newblock Flooding time in edge-markovian dynamic graphs.
\newblock In {\em Proc. of 27th ACM Symp. on Principles of Distributed
  Computing (PODC)}, pages 213--222, 2008.

\bibitem{clementi09}
A.~Clementi, A.~Monti, F.~Pasquale, and R.~Silvestri.
\newblock Broadcasting in dynamic radio networks.
\newblock {\em Journal of Computer and System Sciences}, 75(4):213--230, 2009.

\bibitem{clementi09b}
A.~Clementi, F.~Pasquale, A.~Monti, and R.~Silvestri.
\newblock Information spreading in stationary markovian evolving graphs.
\newblock In {\em Proc. of IEEE Symp. on Parallel \& Distributed Processing
  (IPDPS)}, 2009.

\bibitem{clementi12}
A.~Clementi, R.~Silvestri, and L.~Trevisan.
\newblock Information spreading in dynamic graphs.
\newblock In {\em Proc.\ 31st Symp.\ on Principles of Distributed Computing
  (PODC)}, 2012.

\bibitem{cornejo12}
A.~Cornejo, S.~Gilbert, and C.~Newport.
\newblock Aggregation in dynamic networks.
\newblock In {\em Proc.\ 31st Symp.\ on Principles of Distributed Computing
  (PODC)}, 2012.

\bibitem{LBarxiv}
C.~Dutta, G.~Pandurangan, R.~Rajaraman, and Z.~Sun.
\newblock Information spreading in dynamic networks.
\newblock {\em CoRR}, abs/1112.0384, 2011.

\bibitem{AnalyzingNC}
B.~Haeupler.
\newblock Analyzing network coding gossip made easy.
\newblock In {\em Proc.\ 43nd Symp. on Theory of Computing (STOC)}, pages
  293--302, 2011.

\bibitem{HK}
B.~Haeupler and D.~Karger.
\newblock Faster information dissemination in dynamic networks via network
  coding.
\newblock In {\em Proc.\ 30th Symp.\ on Principles of Distributed Computing
  (PODC)}, pages 381--390, 2011.

\bibitem{HM}
B.~Haeupler and M.~M{\'e}dard.
\newblock One packet suffices - highly efficient packetized network coding with
  finite memory.
\newblock In {\em Information Theory Proceedings (ISIT), 2011 IEEE
  International Symposium on}, pages 1151 --1155, 2011.

\bibitem{jackson2005independence}
B.~Jackson and T.~Jord{\'a}n.
\newblock Independence free graphs and vertex connectivity augmentation.
\newblock {\em Journal of Combinatorial Theory, Series B}, 94(1):31--77, 2005.

\bibitem{clocksync}
F.~Kuhn, C.~Lenzen, T.~Locher, and R.~Oshman.
\newblock Optimal gradient clock synchronization in dynamic networks.
\newblock In {\em Proc. of 29th ACM Symp. on Principles of Distributed
  Computing (PODC)}, pages 430--439, 2010.

\bibitem{dualgraphs}
F.~Kuhn, N.~Lynch, C.~Newport, R.~Oshman, and A.~Richa.
\newblock Broadcasting in unreliable radio networks.
\newblock In {\em Proc. of 29th ACM Symp. on Principles of Distributed
  Computing (PODC)}, pages 336--345, 2010.

\bibitem{KLO}
F.~Kuhn, N.~Lynch, and R.~Oshman.
\newblock Distributed computation in dynamic networks.
\newblock In {\em Proc.\ 42nd Symp.\ on Theory of Computing (STOC)}, pages
  557--570, 2010.

\bibitem{KMO}
F.~Kuhn, Y.~Moses, and R.~Oshman.
\newblock Coordinated consensus in dynamic networks.
\newblock In {\em Proc.\ 30th Symp.\ on Principles of Distributed Computing
  (PODC)}, pages 1--10, 2011.

\bibitem{KOSurvey}
F.~Kuhn and R.~Oshman.
\newblock Dynamic networks: Models and algorithms.
\newblock {\em SIGACT News}, 42(1):82--96, 2011.

\bibitem{odell05}
R.~O'Dell and R.~Wattenhofer.
\newblock {Information dissemination in highly dynamic graphs}.
\newblock In {\em Proc. of Workshop on Foundations of Mobile Computing
  (DIALM-POMC)}, pages 104--110, 2005.

\end{thebibliography}

\end{document}