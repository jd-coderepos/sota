\begin{figure}
 \centering
 \vspace{-0.3cm}
 \includegraphics[trim={10px 10px 10px 10px}, clip, width=\linewidth]{gfx/dag_singleton.pdf}
 \caption{A partition graph of the Python application in
   Fig. \ref{fig:python_code}. For illustrative proposes, the graph
   does not include ignored weight edges \ifdefined\LongVersion
   (cf. Fig. \ref{algo:legal})\fi.}
\label{fig:dag_singleton}
\end{figure}

\newcommand{\AlgoLegal}{
\begin{algorithmic}[1]
\Function{Legal}{}
    \State 
    \State  length of longest path between  and  in  \label{algo:legal:path}
    \If{} \label{algo:legal:transitive}
        \State \Return 
    \Else
        \State \Return 
    \EndIf
\EndFunction
\end{algorithmic}
}

\newcommand{\AlgoGreedy}{
\begin{algorithmic}[1]
\Function{Greedy}{}
    \While{} \label{algo:greedy:while}
        \State  \label{algo:greedy:heaviest}
        \If{}
            \State  \Call{Merge}{}\label{algo:greedy:merge}
        \Else
            \State Remove edge  from  \label{algo:greedy:remove}
        \EndIf
    \EndWhile
    \State \Return 
\EndFunction
\end{algorithmic}
}

\newcommand{\AlgoGentlyA}{
\begin{algorithmic}[1]
\Function{FindCandidate}{}
    \For{}
        \If{\textbf{not} }
            \State Remove edge  from 
        \EndIf
    \EndFor
    \For{}
    \If{degree is less than  for either  or  when only counting edges in }
            \If{}
                \State \Return 
            \EndIf
        \EndIf
    \EndFor
    \State \Return 
\EndFunction
\end{algorithmic}
}

\newcommand{\AlgoGentlyB}{
\begin{algorithmic}[1]
\Function{Unintrusive}{}
    \While{}
        \State  \Call{Merge}{}
    \EndWhile
    \State \Return 
\EndFunction
\end{algorithmic}
}

\newcommand{\AlgoOptimalA}{
\begin{algorithmic}[1]
\Function{MergeByMask}{}
    \State  \textbf{true}\Comment{Flag that indicates fusibility}
    \For{ \textbf{to} }
        \If{}
        \State  the 'th edge in 
            \If{\textbf{not} \Call{Fusible}{}}
                \State  \textbf{false}
            \EndIf
            \State  \Call{Merge}{}
        \EndIf
    \EndFor
    \State \Return 
\EndFunction
\end{algorithmic}
}

\newcommand{\AlgoOptimalB}{
\begin{algorithmic}[1]
\Function{Optimal}{}
    \State  \Call{Unintrusive}{}
    \For{}
        \If{\textbf{not} }
            \State Remove edge  from 
        \EndIf
    \EndFor
    \State  \Call{Greedy}{}\Comment{Good guess}
    \State \Comment{Fill array }
    \State \Comment{The mask offset}
    \State 
    \State \Call{Enqueue}{}
    \While{}
        \State  \Call{Dequeue}{}
        \State  \Call{MergeByMask}{}
        \If{}
            \If{ \textbf{and}  is acyclic}
                \State  \Comment{New best partitioning}
            \EndIf
        \EndIf
        \For{ \textbf{to} }
            \State 
            \State 
            \State \Call{Enqueue}{}
        \EndFor
    \EndWhile
    \State \Return 
\EndFunction
\end{algorithmic}
}

\ifdefined\ShortVersion
\begin{figure*}
\footnotesize
\setlength{\columnsep}{5pt}
\begin{multicols}{3}
    \subfloat[][Help function for Unintrusive]{\label{algo:gently_help}\begin{tcolorbox}[width=1\linewidth]\AlgoGentlyA\end{tcolorbox}}\par
    \subfloat[][Unintrusive]{\label{algo:gently}\begin{tcolorbox}[width=1\linewidth]\AlgoGentlyB\end{tcolorbox}}\newpage \vskip -10cm
    \subfloat[][Greedy]{\label{algo:greedy}\begin{tcolorbox}[width=1\linewidth]\AlgoGreedy\end{tcolorbox}}\par
    \subfloat[][Help function for Optimal]{\label{algo:optimal_help}\begin{tcolorbox}[width=1\linewidth]\AlgoOptimalA\end{tcolorbox}}\newpage\vskip -10cm
    \subfloat[][Optimal]{\label{algo:optimal}  \begin{tcolorbox}[width=1\linewidth]\AlgoOptimalB\end{tcolorbox}}\par
    \end{multicols}
\caption{The partition algorithms where the function, , returns the partition cost of the partition graph .}
\end{figure*}
\fi


\section{Algorithms}

In this section, we present an exact algorithm for finding an
optimal solution to WSP (with exponential worst-case execution time),
and two fast algorithms that find approximate solutions. We
use the Python application shown in Fig.~\ref{fig:python_code} to
demonstrate the results of each partition algorithm.

\subsection{Partition graphs and chains of block merges}
All three algorithms work on data structures called {\em partition
  graphs}, defined as follows:
\begin{definition}[Partition graph]
  Given a graph  and a partition  of , the
  corresponding {\em partition graph} is the graph  that has an edge  if there is an
  edge  with  and . That is, the vertices
  are the blocks, connected by the edges that cross block boundaries.
\end{definition}
\noindent
From this we build the state needed in WSP computations:
\begin{definition}[WSP state]
  Given a WSP-instance  and a partition ,
  the {\em WSP state} is the partition graph  together with a complete weighted
  graph  with weights .
\end{definition}
Notice that  for the Bohrium cost
function, as shown in Prop.~\ref{prop:cost_saving}, and does not
require a full cost calculation.
\begin{definition}[Merge operator on partition graphs]
  We extend the merge operator of Def.~\ref{def:block_merge} to
  partition graphs as .  This acts exactly as a vertex contraction
  on the partition graph.
\end{definition}
The merge operator is commutative in the sense that the order in a
sequence of successive vertex contractions doesn't affect the result \cite{wolle2004note}.
An auxiliary function, \Call{Merge}{}, is used in each algorithm to update the state.
\begin{definition}
  Let  be a WSP state. We define
  
  where  is the updated weight graph on the edges
  incident to the new vertex .
\end{definition}
The complexity of \Call{Merge}{} is dominated by the weight update, which requires
a  computations per edge incident to the merged vertex,
and is bounded by \bigO{V^2}. We next need a local condition for when
a merge is allowed:
\begin{lemma}[Legal merge]
  \label{lem:legal_partition_merge}
  Let  be the successor to a legal
  partition , derived by merging blocks  and
  .  Then  if and only if
  \begin{enumerate}
  \item , and
  \item there is no path of length  from  to  in the
        partition graph .
  \end{enumerate}
\end{lemma}
\begin{proof}
  Recall that  is the subset of partitions in 
  that satisfy Def.~\ref{def:wsp_legal_partition}.  Because  is
  legal, no block contains an edge in . Hence  obeys
  Def.~\ref{def:wsp_legal_partition}(1) if and only if no two vertices
   and  are connected in , or equivalently,
  . 

  Similarly, by assumption, there are no cycles in .
  Thus,  violates Def.~\ref{def:wsp_legal_partition}(2) if and
  only if  contains a path , forming the cycle  in
   (where ).
\end{proof}

\begin{prop}[Reachability through legal merges]
  Given two legal partitions , there exists a successor
  chain 
  entirely contained in , i.e.  corresponding only to
  legal block merges.
\end{prop}
\begin{proof}
  A successor chain  always exists in the total set of partitions ,
  and all such chains are of the same length .  Any such chain
  contains no partition that violates
  Def.~\ref{def:wsp_legal_partition}(1): each step is a merge,
  so once a fuse-preventing edge is placed inside a block, it would
  be included also in a block from .
  Hence we only need to worry about Def.~\ref{def:wsp_legal_partition}(2).

  We now show by induction that a successor chain consisting of only
  legal partitions exists.  First, if , it is trivially
  so. Assume now that the statement is true for all , and
  consider  of distance .

Pick any successor chain from  to .  If any step violates
  Def.~\ref{def:wsp_legal_partition}(2), then let  be the first partition in the chain that does
  so. Then there is a path  in the
  transitive reduction of . Because  satisfies
  Def.~\ref{def:wsp_legal_partition}(2),  is
  contained in a block from , whereby . This merge introduces no cycles, because the
  path is in the transitive reduction.  Now let  be a legal successor chain of
  length , known to exist by hypothesis. Then  is a length- successor chain consisting of only
  legal partitions, concluding the proof by induction.
\end{proof}
In particular, the optimal solutions can be reached in this way from
the bottom partition ,
which we will use in the design of the algorithms.


\ifdefined\LongVersion
\begin{figure}
\footnotesize
\begin{tcolorbox}
    \AlgoLegal
\end{tcolorbox}
\caption{A help function thet determines whether the weight edge, }\label{algo:legal}
\end{figure}
\fi


\ifdefined\ShortVersion
\begin{figure}
    \vspace{-0.5cm}
    \centering
    \subfloat[][Greedy]{\label{fig:dag_greedy}         \includegraphics[trim={15px 10px 12px 10px}, clip, scale=0.5, valign=t]{gfx/dag_greedy.pdf}}
    \subfloat[][Unintrusive]{\label{fig:dag_gently}    \includegraphics[trim={12px 10px 15px 10px}, clip, scale=0.5, valign=t]{gfx/dag_gently.pdf}}\\
    \vspace{-0.3cm}
    \subfloat[][Optimal]{\label{fig:dag_optimal}       \includegraphics[trim={10px 10px 10px 10px}, clip, scale=0.5, valign=t]{gfx/dag_optimal.pdf}}
    \subfloat[][Linear]{\label{fig:dag_topological}\includegraphics[trim={-10px 10px -10px 10px}, clip, scale=0.5, valign=t]{gfx/dag_topological.pdf}}
    \caption{Four partition graphs that are the result of running the
      four partition algorithms with Fig. \ref{fig:dag_singleton} as
      input. Lowercase variable names indicate that they are
      \emph{array contracted}.}
\end{figure}
\fi

\subsection{Unintrusive Partition Algorithm}

In order to reduce the size of the partition graph to be analyzed, we apply an
unintrusive strategy where we merge vertices that are guaranteed to be
part of an optimal solution. Consider the two vertices, , in
Fig. \ref{fig:dag_singleton}. The only beneficial merge possibility
 has is with , so if  is merged in the optimal solution, it
is with . Now, since fusing  will not impose any restriction
to future possible vertex merges in the graph, the two vertices are
said to be \emph{unintrusively fusible}. We formalize this property
using the {\em non-fusible sets}:
\begin{definition}[, non-fusible set]
  The {\em non-fusible} set,   for a block 
  is the set of blocks connected with  in  through
  a path containing a non-fusible edge.
\end{definition}
\begin{theorem}
\label{tho:gentle_fusible}
Given a partition graph , let  be the merged
vertex in . The vertices
 and  are {\em unintrusively fusible} whenever:
\begin{enumerate}
    \item , i.e.~the non-fusibles are unchanged. \item Either  or  is a pendant vertex in , i.e. the
         degree of either  or  must be .
\end{enumerate}
\end{theorem}
\begin{proof}
  If Condition 1 is satisfied, any merge that is disallowed at a further
  stage due to Def.~\ref{def:wsp_legal_partition}(1) would be disallowed also without the merge.
  Similarly, merging a pendant vertex with its parent does not affect
  the possiblity of introducing cycles through future merges (Def.~\ref{def:wsp_legal_partition}(1)).
  Finally, since the cost function is monotonic, the merge cannot adversely affect a future cost.
\end{proof}

Fig.~\ref{algo:gently} shows the unintrusive partitioning algorithm. It uses
a helper function, \Call{FindCandidate}{}, to find two vertices that are
unintrusively fusible. The complexity of \Call{FindCandidate}{}
is , which dominates the while-loop in \Call{Unintrusive}{},
whereby the overall complexity of the unintrusive merge algorithm is
. Note that there is little need to further optimize
\Call{Unintrusive}{} since we will only use it as a preconditioner for the
optimal solution, which will dominate the computation time.

\ifdefined\LongVersion
\begin{figure}
\footnotesize
\begin{tcolorbox}
    \AlgoGentlyA
\end{tcolorbox}
\begin{tcolorbox}
    \AlgoGentlyB
\end{tcolorbox}
\caption{The unintrusive merge algorithm that only merge \emph{unintrusively fusible} vertices.}
\label{algo:gently}
\end{figure}
\fi

Fig. \ref{fig:dag_gently} shows an unintrusive partition of the Python
example with a partition cost of 70. However, the significant
improvement is the reduction of the number of weight edges in the
graph. As we shall see next, in order to find an optimal graph
partition in practical time, the number of weight edges in the graph
must be modest.


\subsection{Greedy Partition Algorithm}

Fig. \ref{algo:greedy} shows a greedy merge algorithm. It uses the
function  to find the edge in  with the
greatest weight and either remove it or merge over it. Note that
 must search through  in each iteration
since  might change the weights.

The number of iterations in the while loop (line
\ref{algo:greedy:while}) is  since at least one weight edge is
removed in each iteration either explicitly (line
\ref{algo:greedy:remove}) or implicitly by \Call{Merge}{} (line
\ref{algo:greedy:merge}). The complexity of finding the heaviest (line
\ref{algo:greedy:heaviest}) is , calling \Call{Legal}{} is
, and calling \Call{Merge}{} is  thus the overall
complexity is .

Fig. \ref{fig:dag_greedy} shows a greedy partition of the Python
example.  The partition cost is 58, which is a significant improvement
over no merge. However, it is not the optimal partitioning, as we
shall see later.

\ifdefined\LongVersion

\begin{figure}
\footnotesize
\begin{tcolorbox}
    \AlgoGreedy
\end{tcolorbox}
\caption{The greedy merge algorithm that greedily merges the vertices connected with the heaviest weight edge in .}
\label{algo:greedy}
\end{figure}

\begin{figure}
 \centering
 \includegraphics[scale=0.5]{gfx/dag_greedy.pdf}
 \caption{A partition graph of the greedy merge of the graph in Fig. \ref{fig:dag_singleton}.}
\label{fig:dag_greedy}
\end{figure}

\begin{figure}
 \centering
 \includegraphics[scale=0.5]{gfx/dag_gently.pdf}
 \caption{A partition graph of the unintrusive merge of the graph in Fig. \ref{fig:dag_singleton}.}
\label{fig:dag_gently}
\end{figure}

\fi


\subsection{Optimal Partition Algorithm}
Because the WSP problem is NP-hard, we cannot in general hope to solve
it exactly in polynomial time. However, we may be able to solve the
problems within reasonable time in common cases given a carefully
chosen search strategy through the  possible partitions. For this
purpose, we have implemented a branch-and-bound algorithm,
exploiting the monotonicity of the partition cost (Def.~\ref{def:wsp_cost}(2)).
It is shown in Fig.~\ref{algo:optimal}, and proceeds as follows:

Before starting, the largest unintrusive partition is found. This is
the largest partition that we can ensure is included in an optimal
partition.  The blocks of the unintrusive partition will be the
vertices in our initial partition graph.  Second, a good suboptimal
solution is computed. We use the greedy algorithm for this purpose,
but any scheme will do.  We now start a search rooted in the
-partition where everything is one block. This has the lowest
cost, but will in general be illegal.  Each recursion step cuts a
weight edge that has not been considered before, if it yields a cost
that is strictly lower than the currently best partition 
(if the cost is higher than for , no further splitting will
yield a better partition, and its search subtree can be ignored). If
we reach a legal partition, this will be the new best candidate, and
no further splitting will yield a better one. When the work queue is
empty,  holds an optimal solution to WSP.

Fig.~\ref{algo:optimal} shows the implementation, Fig.~\ref{fig:search_tree} shows an example of a branch-and-bound search tree, and Fig.~\ref{fig:dag_optimal} shows an optimal partition of the Python example with a partition cost of 38.

\begin{figure}
 \centering
 \vspace{-10px}
 \includegraphics[width=\linewidth]{gfx/search_tree.pdf}
 \caption{A branch-and-bound search tree of the unintrusively merged
   partition graph (Fig. \ref{fig:dag_gently}). Each vertex lists a
   sequences of vertex merges that build a specific graph
   partition. The grayed out area indicates the part of the search
   tree that a depth-first-search can skip because of the cost
   bound. }
 \label{fig:search_tree}
\end{figure}

\ifdefined\LongVersion

\begin{figure}
\footnotesize
\begin{tcolorbox}
    \AlgoOptimalA
\end{tcolorbox}
\begin{tcolorbox}
    \AlgoOptimalB
\end{tcolorbox}
\caption{The optimal merge algorithm that optimally merges the
  vertices in . The function, , returns the partition cost
  of the partition graph .}
\label{algo:optimal}
\end{figure}

\begin{figure}
 \centering
 \includegraphics[scale=0.5]{gfx/dag_optimal.pdf}
 \caption{A partition graph of the optimal merge of the graph in
   Fig. \ref{fig:dag_singleton}.}
\label{fig:dag_optimal}
\end{figure}

\begin{figure}
 \centering
 \includegraphics[scale=0.5]{gfx/dag_topological.pdf}
 \caption{A partition graph of a Linear partition of the Python
   example (Fig. \ref{fig:python_code}).}
\label{fig:dag_topological}
\end{figure}

\fi


\subsection{Linear Merge}
For completeness, we also implement a partition algorithm that does
not use a graph representation. In this na\"{\i}ve approach, we simply
go through the array operation list and add each array operation to
the \emph{current} partition block unless the array operations makes
the current block illegal, in which case we add the array operation to
a new partition block, which then becomes the current one. The
asymptotic complexity of this algorithm is  where  is the
number of array operations.

Fig.~\ref{fig:dag_topological} show that result of partitioning the
Python example with a cost of 58.


\subsection{Merge Cache}
In order to amortize the execution time of applying the merge algorithms,
Bohrium implements a merge cache of previously found partitions of
array operation lists. It is often the case that scientific
applications use large calculation loops such that an iteration in the
loop corresponds to a list of array operations. Since the loop
contains many iterations, the cache can amortize the overall execution time
time.

