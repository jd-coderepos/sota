\documentclass{sig-alternate}
\usepackage[paper=letterpaper,width=7in,height=9.25in,centering]{geometry}
\let\algorithm\relax
\let\endalgorithm\relax
\usepackage[ruled,vlined]{algorithm2e}
\usepackage{tikz}
\usepackage{url}
\usepackage{lhelp}

\newcommand{\RRC}{{\small RRC}}
\newcommand{\CAD}{{\small CAD}}
\newcommand{\QE}{{\small QE}}
\def\C {\ensuremath{\mathbb{C}}}
\def\I {\ensuremath{\mathcal{I}}}
\def\J {\ensuremath{\mathcal{J}}}
\def\K {\ensuremath{\mathbf{k}}}
\def\L {\ensuremath{\mathbb{L}}}
\def\N {\ensuremath{\mathcal{N}}}
\def\Q {\ensuremath{\mathbb{Q}}}
\def\R {\ensuremath{\mathbb{R}}}
\def\T {\ensuremath{\mathfrak{T}}}
\def\S {\ensuremath{\mathfrak{S}}}
\newcommand{\NN}{\mbox{}}
\newcommand{\PP}{\mbox{}}
\newcommand{\HH}{\mbox{}}
\def\QQ {\ensuremath{\mathcal{Q}}}
\newcommand{\uu}{\mathbf{u}}
\newcommand{\xx}{\mathbf{x}}
\newcommand{\yy}{\mathbf{y}}
\newcommand{\F}{\mathcal{F}}
\newcommand{\discrim}[1]{\mbox{{\rm discrim}}}
\newcommand{\alg}[1]{\mbox{{\rm alg}}}
\newcommand{\init}[1]{\mbox{{\rm init}}}
\newcommand{\iter}[1]{\mbox{{\rm iter}}}
\newcommand{\mdeg}[1]{\mbox{{\rm mdeg}}}
\newcommand{\mvar}[1]{\mbox{{\rm mvar}}}
\newcommand{\prem}[1]{\mbox{{\rm prem}}}
\newcommand{\pquo}[1]{\mbox{{\rm pquo}}}
\newcommand{\rank}[1]{\mbox{{\rm rank}}}
\newcommand{\ires}[1]{\mbox{{\rm ires}}}
\newcommand{\src}[1]{\mbox{{\rm src}}}
\newcommand{\sat}[1]{\mbox{{\rm sat}}}
\newcommand{\sep}[1]{\mbox{{\rm sep}}}
\newcommand{\tail}[1]{\mbox{{\rm tail}}}
\newcommand{\head}[1]{\mbox{{\rm head}}}
\newcommand{\oproj}[1]{\mbox{{\rm oproj}}}
\newcommand{\der}[1]{\mbox{{\rm der}}}
\newcommand{\oaproj}[1]{\mbox{{\rm oaproj}}}
\newcommand{\Regularize}[1]{\mbox{{\sf Regularize}}}
\newcommand{\Triangularize}[1]{\mbox{{\sf Triangularize}}}
\newcommand{\RealTriangularize}[1]{\mbox{{\sf RealTriangularize}}}
\newcommand{\BorderPolynomial}[1]{\mbox{{\sf BorderPolynomialSet}}}
\newcommand{\BP}[1]{\mbox{{\sf BorderPolynomial}}}
\newcommand{\GenerateRegularSas}[1]{\mbox{{\sf GenerateRegularSas}}}
\newcommand{\GeneratePreRegularSas}[1]{\mbox{{\sf GeneratePreRegularSas}}}
\newcommand{\DeterminantPolynomial}[1]{\mbox{{\sf DeterminantPolynomialSet}}}
\newcommand{\OpenCAD}[1]{\mbox{{\sf OpenCAD}}}
\newcommand{\OpenAugCAD}[1]{\mbox{{\sf OpenCAD}}}
\newcommand{\LazyRealTriangularize}[1]{\mbox{{\sf LazyRealTriangularize}}}
\newcommand{\GenerateFormula}[1]{\mbox{{\sf GenerateFormula}}}
\newcommand{\RealRootCounting}[1]{\mbox{{\sf RealRootCounting}}}
\newcommand{\ReviseFormula}[1]{\mbox{{\sf Disjunction}}}
\newcommand{\SamplePoints}[1]{\mbox{{\sf SamplePoints}}}
\newcommand{\factor}[1]{\mbox{{\rm factor}}}
\newcommand{\ProjectionFactorSet}[1]{\mbox{{\sf ProjectionFactorSet}}}
\newcommand{\notdone}{{\tt NotDone}}
\newcommand{\RegularChains}{{\tt RegularChains}}
\newcommand{\Maple}{{\tt Maple}}
\def\KK {\ensuremath{\mathbf{K}}}
\def\u {\ensuremath{\mathbf{u}}}
\def\x {\ensuremath{\mathbf{x}}}
\def\y {\ensuremath{\mathbf{y}}}

\newtheorem{Theorem}{Theorem}
\newtheorem{Lemma}{Lemma}
\newtheorem{Example}{Example}
\newtheorem{Definition}{Definition}
\newtheorem{Corollary}{Corollary}
\newtheorem{Remark}{Remark}
\newtheorem{Notation}{Notation}
\newtheorem{Proposition}{Proposition}

\newcommand{\mc}[1]{\mathcal{#1}}
\newcommand{\mr}[1]{\mathrm{#1}}
\newcommand{\bb}[1]{\mathbb{#1}}
\newcommand{\mb}[1]{{#1}}
\newcommand{\mf}[1]{\mathfrak{#1}}
\newcommand{\set}[1]{\left\{\!#1\!\right\}}
\newcommand{\PU}{\Pi_{\mr{U}}}
\newcommand{\ideal}[1]{<\!\!#1\!\!>}
\newcommand{\pd}[2]{\frac{\partial\,#1}{\partial\,#2}}
\newcommand{\Wif}{W_{\infty}}
\newcommand{\Oif}{O_{\infty}}
\newcommand{\Wii}{W_{ini}}
\newcommand{\Bi}{B_{ini}}
\newcommand{\Bs}{B_{sep}}
\newcommand{\Bie}{B_{ineqs}}
\newcommand{\Hii}{\mb{H}_{ini}}
\newcommand{\Hs}{\mb{H}_{sep}}
\newcommand{\Hie}{\mb{H}_{ie}}
\newcommand{\Wf}{W_{F}}	\newcommand{\Of}{O_{F}}	\newcommand{\Wie}{W_{ineqs}}
\newcommand{\Wsd}{W_{sd}}
\newcommand{\Osd}{O_{sd}}
\newcommand{\Wsep}{W_{sep}}
\newcommand{\Wsing}{W_{sing}}
\newcommand{\Osing}{O_{sing}}
\newcommand{\Wc}{W_{c}}
\newcommand{\Oc}{O_{c}}
\newcommand{\DV}{\mathsf{DV}}
\def\R {\ensuremath{\mathbb{R}}}
\def\Q {\ensuremath{\mathbb{Q}}}
\def\C {\ensuremath{\mathbb{C}}}

\DeclareMathOperator{\res}{res}
\DeclareMathOperator{\sres}{ires}
\DeclareMathOperator{\ssres}{ires}
\DeclareMathOperator{\sprem}{iprem}
\DeclareMathOperator{\sis}{sis}
\DeclareMathOperator{\mset}{mset}
\DeclareMathOperator{\ssprem}{ssprem}
\DeclareMathOperator{\iprem}{iprem}
\DeclareMathOperator{\Zero}{Zero}
\DeclareMathOperator{\Z}{Zero}
\DeclareMathOperator{\RegZero}{RegZero}
\DeclareMathOperator{\mv}{mv}
\DeclareMathOperator{\LM}{LM}
\DeclareMathOperator{\LC}{LC}
\DeclareMathOperator{\ldd}{ldd}
\DeclareMathOperator{\ini}{ini}
\DeclareMathOperator{\lc}{lc}
\DeclareMathOperator{\lcf}{lc}
\DeclareMathOperator{\diff}{diff}
\DeclareMathOperator{\DM}{DisMatrix}
\DeclareMathOperator{\GDM}{GDM}
\DeclareMathOperator{\DS}{DisSeq}
\DeclareMathOperator{\GDS}{gds}
\DeclareMathOperator{\dis}{discrim}
\DeclareMathOperator{\rsd}{rsd}
\DeclareMathOperator{\Jac}{Jac}
\DeclareMathOperator{\OpenProj}{OpenProj}
\DeclareMathOperator{\Proj}{Proj}
\DeclareMathOperator{\OpenAugProj}{OpenAugProj}
\DeclareMathOperator{\AugProj}{AugProj}
\DeclareMathOperator{\DFC}{DFC}
\DeclareMathOperator{\Sing}{Sing}
\DeclareMathOperator{\Reg}{Reg}
\DeclareMathOperator{\sign}{sign}
\DeclareMathOperator{\sPrem}{sPrem}
\DeclareMathOperator{\Rem}{rem}
\DeclareMathOperator{\sRes}{sRes}
\DeclareMathOperator{\SyHa}{SyHa}
\DeclareMathOperator{\TaQ}{TaQ}
\DeclareMathOperator{\SPRemS}{SPRemS}
\DeclareMathOperator{\oaf}{oaf}
\newcommand{\true}{\mathbf{true}}
\newcommand{\false}{\mathbf{false}}

\newif\ifcomment
\commentfalse
\newif\ifdraft
\draftfalse


\begin{document}


\title{Triangular Decomposition of Semi-algebraic Systems}
\numberofauthors{6} 
\author{
\alignauthor
Changbo Chen\\
       \affaddr{University of Western Ontario}\\
       \email{cchen252@csd.uwo.ca}
\alignauthor
James H. Davenport\\
       \affaddr{University of Bath}\\
       \email{J.H.Davenport@bath.ac.uk}
\alignauthor 
John P. May\\
       \affaddr{Maplesoft}\\
       \email{jmay@maplesoft.com}
\and
\alignauthor 
Marc Moreno Maza\\
       \affaddr{University of Western Ontario}\\
       \email{moreno@csd.uwo.ca}
\alignauthor
Bican Xia\\
       \affaddr{Peking University}\\
       \email{xbc@math.pku.edu.cn}
\alignauthor
Rong Xiao\\
       \affaddr{University of Western Ontario}\\
       \email{rong@csd.uwo.ca}
}


\conferenceinfo{ISSAC 2010,}{25--28 July 2010, Munich, Germany.}
\CopyrightYear{2010}
\crdata{978-1-4503-0150-3/10/0007}


\maketitle
\begin{abstract}
Regular chains and triangular decompositions are
fundamental and well-developed tools for describing the complex solutions 
of polynomial systems. 
This paper proposes adaptations of these tools focusing
on solutions of the real analogue:
semi-algebraic systems.

We show that any such system can be decomposed into finitely many 
{\em regular semi-algebraic systems}.
We propose two specifications of such a decomposition
and present corresponding algorithms.
Under some assumptions, one type of decomposition can be computed in singly 
exponential time w.r.t.\ the number of variables.  
We implement our algorithms and 
the experimental results illustrate their effectiveness.
\end{abstract}

\category{I.1.2}{Symbolic and Algebraic Manipulation}{Algorithms}
[Algebraic algorithms, Analysis of algorithms]
\terms{Algorithms, Experimentation, Theory}
\keywords{regular semi-algebraic system, regular chain, 
triangular decomposition, border polynomial, fingerprint polynomial set}

\section{Introduction}
Regular chains, the output of triangular decompositions of systems 
of polynomial equations, enjoy remarkable properties.
Size estimates play in their favor~\cite{DaSKac09}
and permit the design of modular~\cite{DMSWX05a} and
fast~\cite{LiMorenoPan09} methods for computing
triangular decompositions.
These features stimulate the development of algorithms
and software for solving polynomial systems
via triangular decompositions.

For the fundamental case of 
semi-algebraic systems
with rational number coefficients, to which this paper is devoted,
we observe that several algorithms for studying the real solutions
of such systems take advantage of the structure of a regular chain.
Some are specialized to isolating the real solutions of
systems with finitely many complex 
solutions~\cite{xz06,CGY07,BoulierChenLemaireMorenoMaza09}.
Other algorithms deal with parametric
polynomial systems via real root classification ({\small RRC})~\cite{yhx01}
or with arbitrary systems via cylindrical
algebraic decompositions ({\small CAD})~\cite{CMXY09}.

In this paper,
we introduce the notion of a {\em regular semi-algebraic system}, 
which in broad terms  is the ``real''
counterpart of the notion of a regular chain.
Then we define two notions of a 
{\em decomposition of a semi-algebraic system}:
one that we call {\em lazy triangular decomposition}, 
where the ana\-lysis of components of strictly smaller dimension 
is deferred, and one that we call 
 {\em full triangular decomposition} 
where all cases are worked out.
These decompositions are 
obtained by combining triangular decompositions
of algebraic sets over the complex field with a special 
Quantifier Elimination ({\small QE}) method
based on {\small RRC} techniques.

\smallskip\noindent{\small \bf Regular semi-algebraic system.}
Let  be a regular chain of  for some
ordering of the variables .
Let  and  
designate respectively the
variables of  that are free and algebraic w.r.t.\ .
Let  be finite such that
each polynomial in  is regular w.r.t.\ the saturated ideal of .
Define .
Let  be a quantifier-free formula of 
involving only the variables of .
We say that  is a {\em regular semi-algebraic system} if:
\begin{itemizeshort}
\item[]  defines a non-empty open 
             semi-algebraic set  in ,
\item[] the regular system  specializes well at every point  of 
    (see Section~\ref{sect:preliminary} for this notion),
\item[] at each point  of , 
the specialized system  
has at least one real zero.
\end{itemizeshort}
The zero set of , denoted by ,
is defined as
the set of points  such that 
 is true and , , for all 
and all .

\smallskip\noindent{\small \bf Triangular decomposition 
of a semi-algebraic system.}
In Section~\ref{sect:specification} we show
that the zero set of any semi-algebraic system  can be decomposed 
as a finite union (possibly empty) 
of zero sets of regular semi-algebraic systems.
We call such a decomposition a {\em full triangular decomposition} 
(or simply {\em triangular decomposition} when clear from context)
of  , 
and denote by {\sf RealTriangularize} an algorithm to compute it.
The proof of our statement relies on triangular decompositions
in the sense of Lazard 
(see Section~\ref{sect:preliminary} for this notion)
for which it is not known whether or not they  
can be computed in singly exponential time w.r.t.\ the number of variables.
Meanwhile, we are hoping to obtain an algorithm
for decomposing semi-algebraic systems 
(certainly under some genericity assumptions)
that would fit in that complexity class.
Moreover, we observe that, in practice, 
full triangular decompositions
are not always necessary and that 
providing information about the components of maximum
dimension is often sufficient.
These theoretical and practical motivations 
lead us to a 
weaker notion of a decomposition
of a semi-algebraic system.

\newcommand{\BB}{\mbox{}}

\smallskip\noindent{\small \bf Lazy triangular decomposition 
of a semi-algebraic system.}
Let  (see Section~\ref{sect:specification}
for this notation) be a semi-algebraic system
of   and 
be  its zero set.
Denote by  the dimension of the constructible  set
 
A finite set of regular semi-algebraic systems , 
is called a \emph{lazy triangular decomposition} of  if
\begin{itemizeshort}
\item   holds, and
\item there exists  such that 
     the real-zero set 
     contains 

    and the complex-zero set 
    either is empty or
   has dimension less than .
\end{itemizeshort}
We denote by {\sf LazyRealTriangularize} an algorithm 
computing such a decomposition. 
In the implementation presented hereafter,
{\sf LazyRealTriangularize} 
outputs additional information in order to 
continue the computations and obtain
a full triangular decomposition, if needed.
This additional information appears in the form
of {\em unevaluated function calls}, explaining
the usage of the adjective {\em lazy}
in this type of decompositions.


\smallskip\noindent{\small {\bf Complexity results for lazy 
triangular decomposition.}}
In Section~\ref{sec:BPRTD}, we provide a running time 
estimate for computing a lazy triangular decomposition
of the semi-algebraic system  
when   has no inequations nor inequalities,
(that is, when  holds)
and when  generates a strongly equidimensional ideal of dimension .
We show that one can compute such a decomposition
in time singly exponential w.r.t.\ .
Our estimates are not sharp and are just meant to reach
a singly exponential bound.
We rely on the work of J. Renagar~\cite{Ren92} for {\small QE}. 
In Sections~\ref{sec:FPS} and \ref{sect:Algorithm}
we turn our attention to algorithms
that are more suitable for implementation even though
they rely on sub-algorithms with a doubly exponential 
running time w.r.t.\ .


\smallskip\noindent{\small \bf A special case of quantifier elimination.}
By means of triangular decomposition of algebraic sets over {\C},
triangular decomposition
of semi-algebraic systems  (both full and lazy)  reduces
to a special case of {\small QE}.
In Section~\ref{sec:FPS}, we implement this latter step 
via the concept of a {\em fingerprint polynomial set}, 
which is inspired by that of a {\em discrimination polynomial set}
used for {\small RRC} in~\cite{yhx01,Xiao09}.


\smallskip\noindent{\small \bf Implementation and experimental results.}
In Section~\ref{sect:Algorithm} we describe the algorithms
that we have implemented for computing triangular decompositions
(both full and lazy) 
of semi-algebraic systems.
Our {\sc Maple} code is written
on top of the {\tt RegularChains} library.
We provide experimental data for two groups of well-known problems. 
In the first group, each input semi-algebraic 
system consists of equations only
while the second group is a collection of {\small QE} problems.
To illustrate the difficulty of our test problems,
and only for this purpose, 
we provide timings obtained with other well-known 
polynomial system solvers which are based on algorithms
whose running time estimates are comparable to ours.
For this first group we use the {\sc Maple}
command {\tt Groebner:-Basis} for computing
lexicographical Gr\"obner bases. 
For the second group we use 
a general purpose {\small QE} software:
{\sc qepcad b} (in its non-interactive mode)~\cite{Bro03}.
Our experimental results show that our {\sf LazyRealTriangularize}
code can solve most of our test problems 
and that it can solve more problems than the 
package it is compared to.

We conclude this introduction by 
 computing a 
triangular decomposition of a particular semi-algebraic system
taken from ~\cite{Brown05}.
Consider the following question: when does 
 have a non-real root 
satisfying ~?
This problem can be expressed as
, where

and
.


We call our {\sf LazyRealTriangularize} command
on the semi-algebraic system 

with the variable order .
Its first step is to call the {\sf Triangularize} 
command of the {\tt RegularChains}
library on the algebraic system .
We obtain one squarefree regular chain , 
where  and , satisfying .
The second step of {\sf LazyRealTriangularize}
is to check whether the polynomials
defining inequalities and inequations are regular w.r.t.\
the saturated ideal of , which is the case here.
The third step is to compute the so called
{\em border polynomial set} (see Section~\ref{sect:preliminary})
which is 
 with 

and
.
One can check that the regular system  
specializes well outside of the hypersurface .
The fourth step is to compute the fingerprint polynomial set
which yields the quantifier-free formula 
telling us that  is a regular semi-algebraic system.
After performing these four steps,
(based on 
Algorithm~\ref{Algo:LazyRealTriangularize}, 
Section~\ref{sect:Algorithm}) 
the function call
 
in our implementation returns the following:
{\small

}
The above output shows that  
forms a lazy triangular decomposition of the input semi-algebraic system.
Moreover, together with the output of the recursive calls,
one obtains  a full triangular decomposition.
Note that the cases of the two recursive calls
correspond to  and  .
Since our {\sf LazyRealTriangularize} uses the 
{\sc Maple} piecewise structure for formatting its output,
one simply needs to evaluate 
the recursive calls with the \verb+value+ command, yielding 
the same result as directly calling {\sf RealTriangularize}

{\small

}
where , ,
.

From this output, after some simplification, one could obtain the
equivalent quantifier-free formula, , of the original {\small QE} problem.


\section{Triangular decomposition  of \\ Algebraic Sets}
\label{sect:preliminary}
We review in the section the basic notions related
to regular chains and triangular decompositions
of algebraic sets.
Throughout this paper, let  be a field of 
characteristic 0 and
 be its algebraic closure.
Let  be the polynomial ring over 
 and with ordered variables . 
Let  be polynomials. 
Assume that .
Then denote by
, , and  
respectively
the greatest variable appearing in  
(called the {\em main variable} of ),
the leading coefficient of  w.r.t.\ 
(called the {\em initial} of ), and
the degree of  w.r.t.\ 
(called the {\em main degree} of );
denote by  the derivative of  w.r.t.\ ;
denote by  the discriminant of 
w.r.t.\ .


\smallskip\noindent{\small \bf Triangular sets}.
Let  be a {\em triangular set},
that is, a set of non-constant polynomials with pairwise
distinct main variables. 
Denote by  the set of
main variables of the polynomials in . 
A variable  in  is called
{\em algebraic} w.r.t.\  if , 
otherwise it is said {\em free} w.r.t.\ . 
If no confusion is possible, we shall always denote by
 and  
respectively the free and the main variables of .
\ifcomment{
For , denote by  the polynomial in 
with main variable .
For ,
we denote by  the set of 
polynomials  such that .
}\fi
Let  be the product
of the initials of the polynomials in .
We denote by \sat{T} the {\em saturated ideal} of : 
if  is the empty triangular set, then \sat{T} is defined as the trivial
ideal , otherwise it is the ideal
.
The {\em quasi-component}  of 
is defined as .
Denote  as the Zariski closure of .

\smallskip\noindent{\small \bf Iterated resultant}.
Let  and  be two polynomials of . 
Assume  is non-constant and let .
We define  as follows:
if  does not appear in , 
then ; otherwise 
is the resultant of  and  w.r.t.\ .
Let  be a triangular set of .
We define  by induction:
if  is empty, then ;
otherwise let  be the greatest variable appearing in , then 
, 
where  and  denote respectively the polynomials
of  with main variables equal to and less than .

\smallskip\noindent{\small \bf Regular chain}.
A triangular set  is a {\em regular chain} 
if: either  is empty; 
or (letting  be  the polynomial in  with maximum main variable),
  is a regular chain,
and the initial of  is regular w.r.t.\ 
\sat{T \setminus \{ t \}}.
The empty regular chain is denoted by .
Let .
The pair  is a {\em regular system} if each polynomial in 
is regular modulo . 
A regular chain  or a regular system , is {\em squarefree} 
if for all , the  is regular w.r.t.\ .


\smallskip\noindent{\small \bf Triangular decomposition}.
Let . Regular chains  of 
form a {\em triangular decomposition} of  if either: 
 (Kalkbrener's sense) or 
 (Lazard's sense).
In this paper, we denote by {\sf Triangularize} an algorithm,
such as the one of~\cite{MMM99}, 
computing a triangular decomposition of the former kind.


\smallskip\noindent{\small \bf Regularization.}
Let .
Let  be a regular chain of .
Denote by  an operation which computes  
a set of regular chains  such that 
 for each , either  or  is
regular w.r.t.\ ; 
 
we have ,
 and  for .


\smallskip\noindent{\small \bf Good specialization~\cite{CGLMP07}}. 
Consider a squarefree regular system  of .
Recall that  and  
stand respectively for  and .
Let  be a point of .
We say that  {\em specializes well} at  if:
 none of the initials of the polynomials in  vanishes
modulo the ideal ;
 the image of  modulo 
              
              is a squarefree regular system.


\smallskip\noindent{\small \bf Border polynomial~\cite{yhx01}}. 
Let  be a squarefree regular system of .
Let  be the primitive and square free part of 
the product of all
 and all  for  and .
We call  the {\em border polynomial} of 
and denote by  an algorithm to compute it. 
We call the set of irreducible factors of  the
{\em border  polynomial set} of .
Denote by  an algorithm
to compute it.
Proposition~\ref{Proposition:borderpolynomial}
follows from 
the specialization property of subresultants
and states a fundamental property of border polynomials.


\begin{Proposition}
\label{Proposition:borderpolynomial}
The system  specializes well at  if and only if 
the border polynomial .
\end{Proposition}



\section{Triangular decomposition of \\
semi-algebraic systems}
\label{sect:specification}
In this section, we  prove that any semi-algebraic system can be decomposed 
into finitely many regular semi-algebraic systems.
This latter notion was defined in the introduction.

\smallskip\noindent{\small \bf Semi-algebraic system}. 
Consider four finite polynomial subsets
,
,
,
and  
of .
Let  denote the set of non-negative inequalities 
.
Let  denote the set of positive inequalities
.
Let  denote the set of inequations .
We will denote by  the {\em basic semi-algebraic system}
.
We denote by 
the semi-algebraic system ({\small SAS}) 
which is the conjunction of the following conditions:
,
,
 and
.


\smallskip\noindent{\small \bf Notations on zero sets}. 
In this paper, we use ``'' to denote the zero set of 
a polynomial system, involving equations and inequations, in  
and ``'' to denote the zero set
of a semi-algebraic system in .

\smallskip\noindent{\small \bf Pre-regular semi-algebraic system}.
Let  be a squarefree regular system  of .
Let  be the border polynomial of .
Let  be a polynomial set such that 
 divides the product of polynomials in .
We call the triple  a 
{\em pre-regular semi-algebraic system} of .
Its zero set, written as , is 
the set  such that , , ,
for all , , .
Lemma~\ref{Lemma:decompose-prsas} and Lemma~\ref{Lemma:prsas}
are fundamental properties of pre-regular semi-algebraic systems.


\begin{Lemma}
\label{Lemma:decompose-prsas}
Let  be a semi-algebraic system of .
Then there exists finitely many pre-regular semi-algebraic systems
, , 
s.t. .
\end{Lemma}
\begin{proof}
The semi-algebraic system  decomposes
into basic semi-algebraic systems,
by rewriting  inequality of type  as: .
Let  be one of those basic semi-algebraic systems.
If  is empty, then the triple , 
is a pre-regular semi-algebraic system.
If  is not empty, by Proposition~\ref{Proposition:borderpolynomial}
and the specifications of {\sf Triangularize} and {\sf Regularize}, 
one can compute finitely many
squarefree regular systems  
such that  holds and 
where  is the border polynomial set of the regular system .
Hence, we have

where each 
 is a pre-regular semi-algebraic system. 
\end{proof}

\begin{Lemma}
\label{Lemma:prsas}
Let  be a pre-regular semi-algebraic system 
of .
Let  be the product of polynomials in .
The complement of the hypersurface  in  consists of 
finitely many open cells 
of dimension .
Let  be one of them.
Then for all , 
the number of real zeros of  is the same.
\end{Lemma}
\begin{proof}
From 
Proposition~\ref{Proposition:borderpolynomial}
and recursive use of Theorem 1 in~\cite{col75}
on the delineability of a polynomial.
\end{proof}

\begin{Lemma}
\label{Lemma:pre2regular}
Let  be a pre-regular semi-algebraic system 
 of .
One can decide whether its zero set is empty or not.
If it is not empty, then one can compute 
a regular semi-algebraic system 
whose zero set in  is the same as that of .
\end{Lemma}
\begin{proof}
If ,
we can always test whether 
the zero set of  is empty or not,
for instance using {\small CAD}.
If it is empty, we are done.
Otherwise, defining , 
the triple  is a regular semi-algebraic system.
If  is not empty, 
we solve the quantifier elimination problem 
 and let
 be the resulting formula.
If  is false, we are done.
Otherwise, by Lemma~\ref{Lemma:prsas},
above each connected component of , 
the number of real zeros of the system 
is constant.
Then, the zero set defined by  is the union
of the connected  components of 
 above which 
possesses at least one solution.
Thus, 
defines a nonempty open set of 
and 
 is a regular semi-algebraic system.
\end{proof}


\begin{Theorem}
\label{Theorem:decompose}
Let  be a semi-algebraic system of .
Then one can compute a (full) triangular decomposition
of , that is, as defined  in the introduction, 
finitely many regular semi-algebraic 
systems such that the union of their zero sets is the 
zero set of .
\end{Theorem}
\begin{proof}
It follows from Lemma~\ref{Lemma:decompose-prsas}
and~\ref{Lemma:pre2regular}.
\end{proof}


\section{Complexity Results}
\label{sec:BPRTD}

We start this section by stating complexity 
estimates for basic operations on multivariate polynomials.

\smallskip\noindent{\small \bf Complexity of basic polynomial operations.} 
Let  be polynomials with
respective total degrees ,
and let .
Let  and 
be the {\em height}
(that is, the bit size of the maximum absolute value of 
the numerator or denominator of a coefficient) of ,
the product  and the resultant ,  respectively.
In~\cite{DST88}, it is proved that 
 can be computed within  bit operations
where 
and .
It is easy to establish that 
 and  are respectively
upper bounded by 

and 
.
Finally, let 
 be a  matrix over .
Let  (resp. )
be the maximum total degree (resp. height)
of a polynomial coefficient of .
Then  can be computed within 
 
bit operations, see~\cite{HS98}.


We turn now to the main subject of this section,
that is, complexity estimates for 
a lazy triangular decomposition of a polynomial system
under some genericity assumptions.
Let . 
A lazy triangular decomposition (as defined in the Introduction) of 
the semi-algebraic system , 
which only involves equations,
is obtained by the above
algorithm.

\begin{algorithm}
\dontprintsemicolon
\linesnumbered
\caption{\LazyRealTriangularize{\S}
\label{Algo::3-step}}
\KwIn{a semi-algebraic system }
\KwOut{a lazy triangular decomposition of }
\; \For{}{
 \; 
solve , 
and let  be the resulting quantifier-free formula\;
\lIf{}{
output 
}
}
\end{algorithm}

\smallskip\noindent{\small \bf Proof of Algorithm~\ref{Algo::3-step}}.
The termination of the algorithm is obvious.
Let us prove its correctness.
Let , for  be 
the output of Algorithm~\ref{Algo::3-step} 
and let  for  be the
regular chains such that . 
By Lemma~\ref{Lemma:pre2regular}, each  is a regular semi-algebraic system.
For , define .
Then we have
, 
where each  is equidimensional. 
For each , by Proposition~\ref{Proposition:borderpolynomial}, 
we have 
Moreover, we have 

Hence,
 holds.
In addition, since  is regular modulo , we have
 and . 
So the , for , 
form a lazy triangular decomposition of . 


In this section, under some genericity assumptions for ,
we establish running time estimates for Algorithm~\ref{Algo::3-step},
see Proposition~\ref{prop:lazy-rtd}.
This is achieved through:
\begin{itemizeshort}
\item[   ] Proposition~\ref{Prop:foobar}  
giving running time and output size estimates
for a Kalkbrener triangular decomposition of an algebraic set, and
\item[   ] Theorem~\ref{thm: complexity-bp}
 giving running time and output size estimates
for a border polynomial computation.
\end{itemizeshort}
Our assumptions for these results are the following:
\begin{itemizeshort}
\item[ ]  is equidimensional of dimension ,
\item[ ]  are algebraically independent 
   modulo each associated prime ideal of 
   the ideal generated by  in ,
\item[ ]  consists of  polynomials,
      .
\end{itemizeshort}
Hypotheses  and 
are equivalent to the existence of  
regular chains  of 
such that  are free
w.r.t.\ each of  and such that we have


Denote by ,  respectively 
the maximum total degree and height 
of .
In her PhD Thesis~\cite{Szanto99}, {\'A}.~Sz{\'a}nt{\'o}
describes an algorithm
which computes a Kalkbrener 
triangular decomposition, ,  of .
Under Hypotheses 
to , this algorithm runs in time
 counting operations in {\Q},
while the total degrees of the 
polynomials in the output are bounded by .
In addition,  are square free, 
{\em strongly normalized}~\cite{MMM99} and {\em reduced}~\cite{ALM99}.

From , we obtain 
regular chains  forming another
Kalkbrener triangular decomposition of , as follows.
Let  and .
Let  be the polynomial of  with  as main variable.
Let  be the primitive part of  
regarded as a polynomial in .
Define .
According to the complexity results for polynomial
operations stated at the beginning of this section, 
this transformation can be done 
within  operations in {\Q}.

Dividing  by its initial
we obtain a monic polynomial  
of .
Denote by  the regular chain .
Observe that  is the 
reduced lexicographic Gr\"obner basis of the radical ideal 
it generates in 
.
So Theorem 1 in~\cite{DaSKac09} applies to each regular chain .
For each polynomial , this theorem provides 
height and total degree estimates
expressed as functions of the {\em degree}~\cite{BCSL97}  and
the {\em height}~\cite{pph3,KrPaSo01} of the algebraic set
.
Note that the degree and height of
 are upper bounded by those of .
Write 
where each  is a monomial
and  are in 
 such that 
 holds.
Let  be the lcm of the 's.
Then for  and each :
\begin{itemizeshort}
	\item the total degree is bounded by  and,
	\item  the height 
           by .
\end{itemizeshort}
Multiplying  by 
brings  back. We deduce
the height and total degree 
estimates for each  below.

\begin{Proposition}
\label{Prop:foobar}
The Kalkbrener triangular decomposition  
of  can be computed in
 operations in {\Q}.
In addition,  every polynomial 
has total degree upper bounded by ,
and has height upper bounded by
.
\end{Proposition}

Next we estimate the running time
and output size for computing the border polynomial
of a regular system.
\begin{Theorem}
\label{thm: complexity-bp}
Let  be a squarefree regular system of ,
with  and .
Let  be the border polynomial of .
Denote by ,  
respectively the maximum total degree and height
of a polynomial in .
Then the total degree of  is upper bounded by 
, 
and  can be computed within
 
bit operations.
\end{Theorem}
\begin{proof}
Define .
We need to compute the  iterated resultants ,
for all .
Let .
Observe that the total degree and height of 
are  bounded by 
 and  respectively.
Define 
, \ldots, 
, \ldots,
.
Let .
Denote by  and   the total degree and
height of , respectively.
Using the complexity estimates stated at the beginning 
of this section, we have
 
and
.
Therefore, we have 
.
From these size estimates, one can deduce that each resultant  
(thus the iterated resultants) can be computed within 
 bit operations, 
by the complexity of computing a determinant 
stated at the beginning of this section.

Hence, the product of all iterated resultants has total
degree and height bounded
by  
and
, respectively. 
Thus, the primitive and squarefree part of this product 
can be computed within 
 bit operations, 
based on the complexity of  a polynomial
gcd computation stated at the beginning of this section.
\end{proof}


\begin{Proposition}
		\label{prop:lazy-rtd}
From the Kalkbrener triangular decomposition  of  
Proposition~\ref{Prop:foobar}, 
a lazy triangular decomposition of 
can be computed
in   bit operations.
Thus, a lazy triangular decomposition of this system 
is computed from the input polynomials in singly exponential time
w.r.t.\ , counting operations in {\Q}.
\end{Proposition}
\begin{proof}
For each , 
let  be the border polynomial of  and
let  (resp. ) be the height 
(resp. the total degree) bound
of the polynomials in the pre-regular 
semi-algebraic system .
According to Algorithm~\ref{Algo::3-step}, the remaining task is to solve 
the {\small QE} problem  for
each , 
which can be solved within 
 bit operations,
based on the results of~\cite{Ren92}.
The conclusion 
 follows from the size estimates in Proposition \ref{Prop:foobar} and
Theorem \ref{thm: complexity-bp}. 
\end{proof}


\section{Quantifier Elimination by RRC}
\label{sec:FPS}

In the last two sections, we saw that in order 
to compute a triangular decomposition of 
a semi-algebraic system, 
a key step is to solve the following quantifier 
elimination problem: 

where  is a pre-regular semi-algebraic 
system of . 
This problem is an instance of the so-called
{\em real root classification} ({\small RRC})~\cite{YX05}.
In this section, we show how to solve this problem
when  is what we call a {\em fingerprint polynomial set}.

\smallskip\noindent{\small \bf Fingerprint polynomial set.}
Let  be a pre-regular semi-algebraic system of .
Let .
Let  be
the product of all polynomials in .
We call    a {\em fingerprint polynomial set} ({\small FPS}) of  if:
\begin{itemizeshort}
\item[] for all , for all  we have:\\
  ,
\item[] for all  with 
and , , 
if the signs of   and  are 
the same for all , then 
 has real solutions if and only if  does.
\end{itemizeshort}

Hereafter,
we present a method to construct an {\small FPS}
based on projection operators of {\small CAD}.

\smallskip\noindent{\small \bf Open projection operator~\cite{adam00, brown01}.}
Hereafter in this section, let  be ordered variables.
Let  be non-constant.
Denote by  the set
of the non-constant irreducible factors of .
For , define .
Let  (resp. ) be the set of the
polynomials in  with
main variable equal to (resp. less than) . 
The {\em open projection operator} () w.r.t.\ variable 
maps  to
a set of polynomials of  defined below:

Then, we define .

\smallskip\noindent{\small \bf Augmentation.}
Let  and .
Denote by   the {\em derivative closure} of  w.r.t.\ , 
that is,

The \emph{open augmented projected factors} of  is denoted by 
and defined as follows. 
Let  be the smallest positive integer such that
 holds.
Denote by  the set
; we have
\begin{itemizeshort}
\item if , then ;
\item if ,  then 
		
\end{itemizeshort}

\begin{Theorem}
\label{thm:oap}
Let  be finite and let  be a map from
 to the set of signs .
Then the set 

is either empty or a connected open set in .
\end{Theorem}
\begin{proof}
By induction on .
When , the conclusion follows from Thom's Lemma
~\cite{BPR06}.
Assume . 
If  is not the smallest positive integer  such that
 holds, then
  can be written 
 and the conclusion follows by induction.
Otherwise, write   as , 
where  and .
We have: . 
Denote by  the set 
.
If  is empty then so is  and the conclusion is clear.
From now on assume  not empty. Then, 
by induction hypothesis,  is 
a connected open set in .
By the definition of the operator ,
the product of the polynomials in  is
delineable over  w.r.t.\ .
Moreover,   is derivative closed (may be empty) w.r.t.\ .
Therefore 
is either empty or a 
connected open set by Thom's Lemma.
\end{proof}

\begin{Theorem}
\label{thm:dpoap}
Let  be a pre-regular semi-algebraic system of .
The polynomial set  is a fingerprint polynomial set of .
\end{Theorem}
\begin{proof}
Recall that the border polynomial  of 
divides the product of the polynomials in . 
We have . 
So  satisfies  in the definition of {\small FPS}.
Let us prove .
Let  be the product of the polynomials in .
Let  
such that both ,  hold 
and the signs of  and  are equal for
all .
Then, by Theorem~\ref{thm:oap},   and  
belong to the same connected component of ,
and thus to the same connected component of .
Therefore the number of real solutions of  and that of 
are the same by Lemma~\ref{Lemma:prsas}.
\end{proof}

From now on, 
let us assume that the set  in 
the pre-regular semi-algebraic system 
is an {\small FPS} of .
We solve the quantifier elimination problem (\ref{eq:RRC0})
in three steps:
 compute at least one sample point in each connected component of 
the semi-algebraic set defined by ;
 for each sample point  such that 
the specialized system  possesses 
real solutions, compute the sign
of  for each ;
 generate the corresponding quantifier-free formulas.

In practice, when the set  is not an {\small FPS}, 
one adds some polynomials from ,
using a heuristic procedure (for instance one by one)  
until Property  of the definition of an {\small FPS} is
satisfied.
This strategy is implemented in Algorithm~\ref{Algo:GenerateRegularSas}
of Section~\ref{sect:Algorithm}.


\section{Implementation}
\label{sect:Algorithm}
In this section, we present algorithms for {\sf LazyRealTriangularize}
and  {\sf RealTriangularize} that we have implemented 
on top of the {\tt RegularChains} library in {\sc Maple}.
We also provide experimental results
for test problems which are available at
\url{www.orcca.on.ca/~cchen/issac10.txt}.

\begin{algorithm}
\dontprintsemicolon
\linesnumbered
\caption{\GeneratePreRegularSas{\S}\label{Algo:GeneratePreRegularSas}}
\KwIn{a semi-algebraic system }
\KwOut{
a set of pre-regular semi-algebraic 
systems\\
, , such that

}
; \; 
    \For{}{
        \For{}{
            \For{}{
                {\bf if}  {\bf then} 
            }
        }
        ; \;
    }
    ; \;
    \For{}{
        \For{}{
            \For{}{
                \eIf{}{
                     
                   }{
                     
                }
            }
        }
        ; \;
    }
    \;
    \For{}{
         \;
         output \;
    }
\end{algorithm}

\begin{algorithm}
\dontprintsemicolon
\linesnumbered
\caption{\GenerateRegularSas{B, T, P}\label{Algo:GenerateRegularSas}}
\KwIn{, a pre-regular semi-algebraic system of , where  and . }
\KwOut{A pair  satisfying: \\
  such that ;\\
  is a finite set of regular semi-algebraic systems, s.t. 
.
}
\; 
      \If{}{
         \eIf{}{
              return \;     
         }{
              return \;
         }
      }
\While{true}{
            ; ; \;
            \For{}{
                 \eIf{}{
                      
                    }{
                      
                 }
            }
            \eIf{}{
                 \;
				 {\bf if}  {\bf then}  return \;
				 {\bf else} return \;
              }{
                select a subset 
                 by some heuristic method\;
                 \;
            }
      }

\end{algorithm}


\smallskip\noindent{\small \bf Basic subroutines.}
For a zero-dimensional squarefree regular system 
, \RealRootCounting{T, P}~\cite{xz06} 
returns the number of real zeros of .
For  and a point  of 
such that  for all , \GenerateFormula{A, s}
computes a formula , where  is defined 
as  if  and  otherwise.
For a  set of formulas ,
\ReviseFormula{G} computes a logic formula  
equivalent to
the disjunction of the formulas in .


\smallskip\noindent{\small \bf Proof of
Algorithm~\ref{Algo:GeneratePreRegularSas}}.
Its termination is obvious.
Let us prove its correctness.
By the specification of {\sf Triangularize} and {\sf Regularize},
at line , we have

Write  as . Then we deduce that

For each ,
at line , we generate
a pre-regular semi-algebraic system .
By Proposition~\ref{Proposition:borderpolynomial}, we have

which implies that

So Algorithm~\ref{Algo:GeneratePreRegularSas}
satisfies its specification.


\begin{algorithm}
\dontprintsemicolon
\linesnumbered
\caption{\SamplePoints{A, k}\label{Algo:SamplePoints}}
\KwIn{ is a finite set of non-zero polynomials}
\KwOut{ A finite subset of  contained in \\
       
        and having a non-empty intersection with 
        each connected component  of . }
\uIf{}{
         return one rational point from each 
        connected component  of  \;
    }
    \Else{
; \;
         \For{}{              
              Collect in a set  one rational point from each 
        connected component  of ;\;
              {\bf for}  {\bf do} output 
         }
    }
\end{algorithm}

\begin{algorithm}
\dontprintsemicolon
\linesnumbered
\caption{\LazyRealTriangularize{\S}\label{Algo:LazyRealTriangularize}}
\KwIn{a semi-algebraic system }
\KwOut{a lazy triangular decomposition of 
}
\;
    \For{}{
(\;
                 \lIf{}{ output }\;
}
\end{algorithm}


\begin{algorithm}
\dontprintsemicolon
\linesnumbered
\caption{\RealTriangularize{\S}\label{Algo:RealTriangularize}}
\KwIn{ a semi-algebraic system }
\KwOut{a triangular decomposition of }
\;
    \For{}{
          (\;
\lIf{}{ output }\;
         \For{}{
output \;
         }
    }
\end{algorithm}



\smallskip\noindent{\small \bf Proof of
Algorithms~\ref{Algo:GenerateRegularSas} and~\ref{Algo:SamplePoints}}.
By the definition of , 
Algorithm~\ref{Algo:SamplePoints} terminates and satisfies its specification.
By Theorem~\ref{thm:dpoap}, 
 is an {\small FPS}.
Thus, by the definition of an {\small FPS},
Algorithm~\ref{Algo:GenerateRegularSas}
terminates and satisfies its specification.

\smallskip\noindent{\small \bf Proof of Algorithm~\ref{Algo:LazyRealTriangularize}}.
Its termination is obvious.
Let us prove the algorithm is correct.
Let ,  be the output. 
By the specification of each sub-algorithm, 
each  is a regular semi-algebraic system and we have:

Next we show that 
there exists an ideal ,
whose dimension is less than 
and 
such that 

holds.

At line , 
by the specification of Algorithm~\ref{Algo:GeneratePreRegularSas}, 
we have

At line , 
by the specification of Algorithm~\ref{Algo:GenerateRegularSas},
for each , we compute a set  such that 
and 

both hold.
Combining the two relations together, we have

Therefore, the following relations hold

Define 

Since each  is regular modulo , 
we have

So all  form a lazy triangular decomposition of .



\smallskip\noindent{\small \bf Proof of Algorithm~\ref{Algo:RealTriangularize}}.
For its termination, it is sufficient to prove that
there are only finitely many recursive calls to {\sf RealTriangularize}. 
Indeed, if  is the input of a call to {\sf RealTriangularize}
then each of the immediate recursive calls takes 
as input, where  belongs to the 
set  of some pre-regular semi-algebraic system .
Since  is regular (and non-zero) modulo  we have:

Therefore, the algorithm terminates 
by the ascending chain condition on ideals of .
The correctness of Algorithm~\ref{Algo:RealTriangularize}
follows from the specifications of the 
sub-algorithms.




\smallskip\noindent{\small \bf Table 1}.
Table 1 summarizes the notations used in Tables 2 and 3. Tables 2 and 3
demonstrate benchmarks running in {\sc Maple} 14 , 
using an Intel Core 2 Quad {\small CPU} (2.40{\small GHz}) 
with 3.0{\small GB} memory.
The timings are in seconds and 
the time-out is 1 hour.

\begin{figure}
\centering
{\textbf{Table 1} Notations for Tables 2 and 3}
\medskip
\newline
{\small
\begin{tabular}{l|l}
\hline
symbol & meaning\\\hline
\#e  & number of equations in the input system\\
\#v & number of variables in the input equations\\
d & maximum total degree of an input equation\\
G & {\sf Groebner:-Basis} ({\sf plex} order) in {\sc Maple}\\
T & {\sf Triangularize} in {\RegularChains} library of {\sc Maple}\\
LR & {\sf LazyRealTriangularize} implemented in {\sc Maple}\\
R &{\sf RealTriangularize} implemented in {\sc Maple}\\
Q & {\sc Qepcad~b}\\
 & computation does not complete within 1 hour\\
FAIL & {\sc Qepcad~b} failed due to prime list exhausted\\\hline
\end{tabular}
}
\end{figure}

\smallskip\noindent{\small \bf Table 2}.
The systems in this group  involve equations only.
We report the running times for a 
triangular decomposition of the input algebraic variety
and a lazy triangular decomposition of the corresponding real variety.
These illustrate the good performance of 
our tool.


\begin{figure}
\centering
{\textbf{Table 2} Timings for varieties}
\medskip
\newline
{\small
\begin{tabular}{c|c|c|c|c}
\hline
system                  &\#v/\#e/d& G            & T     & LR\\\hline
Hairer-2-BGK            & 13/ 11/ 4 & 25         &1.924  & 2.396\\Collins-jsc02           & 5/ 4/ 3   & 876        &0.296  & 0.820\\Leykin-1                & 8/ 6/ 4   & 103        &3.684  & 3.924\\8-3-config-Li           & 12/ 7/ 2  & 109        &5.440  & 6.360\\Lichtblau               & 3/ 2/ 11  & 126        &1.548  & 11\\Cinquin-3-3             & 4/ 3/ 4   & 64         &0.744  & 2.016\\Cinquin-3-4             & 4/ 3/ 5   &     &10     & 22\\DonatiTraverso-rev      & 4/ 3/ 8   & 154        &7.100  & 7.548\\Cheaters-homotopy-1     & 7/ 3/ 7   & 3527       &174    &  \\hereman-8.8             & 8/ 6/ 6   &     &33     & 62\\L                       &12/ 4/ 3   &      & 0.468 & 0.676\\
dgp6                    &17/19/ 2   & 27         & 60    & 63\\
dgp29                   & 5/ 4/ 15  & 84         & 0.008 & 0.016\\
\hline
\end{tabular}
}
\end{figure}

\smallskip\noindent{\small \bf Table 3}.
The examples in this table 
are quantifier elimination problems
and most of them
involve both equations and inequalities.
We provide
the timings for computing a lazy and a full
triangular decomposition of the 
corresponding semi-algebraic system
and the timings for solving the quantifier
elimination problem via {\sc Qepcad b}~\cite{Bro03} 
(in non-interactive mode).
Computations complete with our tool on 
more examples than with {\sc Qepcad b}.

\begin{figure}
\centering
{\textbf{Table 3} Timings for semi-algebraic systems}
\medskip
\newline
{\small
\begin{tabular}{c|c|c|c|c|c}
\hline
system        &\#v/\#e/d&    T    & LR      & R    & Q\\\hline
BM05-1        &4/ 2/ 3  & 0.008   & 0.208  & 0.568   &86\\
BM05-2        &4/ 2/ 4  & 0.040   & 2.284  &  &FAIL\\
Solotareff-4b &5/ 4/ 3  & 0.640   & 2.248  &924      &\\
Solotareff-4a &5/ 4/ 3  & 0.424   & 1.228  &8.216    &FAIL\\
putnam        &6/ 4/ 2  & 0.044   & 0.108  &0.948    &\\
MPV89         &6/ 3/ 4  & 0.016   & 0.496  & 2.544   &\\
IBVP          &8/ 5/ 2  & 0.272   & 0.560  &12   &\\
Lafferriere37 &3/ 3/ 4  & 0.056   & 0.184  &0.180    &10\\
Xia           &6/ 3/ 4  & 0.164   & 191    & 739     &\\
SEIT          &11/ 4/3  & 0.400   & &  &\\
p3p-isosceles &7/ 3/ 3  & 1.348   & &  &\\
p3p           &8/ 3/ 3  & 210     & &  &FAIL\\
Ellipse       &6/ 1/ 3  & 0.012   & &  &\\
\hline
\end{tabular}
}
\end{figure}


\smallskip\noindent{\small \bf Remark}. 
The output of our tools 
is a set of regular semi-algebraic systems,
which is different than that of {\sc Qepcad b}.
We note also that our tool is more effective for
systems with more equations than inequalities.



\bigskip\noindent{\bf Acknowledgments.}
The authors would like to thank the referees for their valuable
remarks that helped to improve the presentation of the work.

\newpage

\small
\begin{thebibliography}{10}
\vspace{0.5em}
\bibitem{ALM99}
P.~Aubry, D.~Lazard, and M.~{{Moreno Maza}}.
\newblock On the theories of triangular sets.
\newblock {\em J. Symb. Comput.}, 28(1-2):105--124, 1999.

\bibitem{BPR06}
S.~Basu, R.~Pollack, and M-F. Roy.
\newblock {\em Algorithms in real algebraic geometry}.
\newblock Springer-Verlag, 2006.

\bibitem{BoulierChenLemaireMorenoMaza09}
F.~Boulier, C.~Chen, F.~Lemaire, and M.~{Moreno Maza}.
\newblock Real root isolation of regular chains.
\newblock In {\em Proc. {ASCM'09}}.

\bibitem{brown01}
C.~W. Brown.
\newblock Improved projection for cylindrical algebraic decomposition.
\newblock {\em J. Symb. Comput.}, 32(5):447--465, 2001.

\bibitem{Bro03}
C.~W. Brown.
\newblock {\sc qepcad b}: a program for computing with semi-algebraic sets
  using cads.
\newblock {\em SIGSAM Bull.}, 37(4):97--108, 2003.

\bibitem{Brown05}
C.~W. Brown and S.~McCallum.
\newblock On using bi-equational constraints in cad construction.
\newblock In {\em ISSAC'05}, pages 76--83, 2005.

\bibitem{BCSL97}
P.~B\"{u}rgisser, M.~Clausen, and M.~A. Shokrollahi.
\newblock {\em Algebraic Complexity Theory}.
\newblock Springer, 1997.





\bibitem{CGLMP07}
C.~Chen, O.~Golubitsky, F.~Lemaire, M.~{Moreno Maza}, and W.~Pan.
\newblock Comprehensive triangular decomposition.
\newblock In {\em {CASC'07}}, pages 73--101, 2007.

\bibitem{CMXY09}
C.~Chen, M.~{Moreno Maza}, B.~{Xia}, and L.~Yang.
\newblock Computing cylindrical algebraic decomposition via triangular
  decomposition.
\newblock In {\em ISSAC'09}, pages 95--102.

\bibitem{CGY07}
J.S. Cheng, X.S. Gao, and C.K. Yap.
\newblock Complete numerical isolation of real zeros in zero-dimensional
  triangular systems.
\newblock In {\em ISSAC '07}, pages 92--99, 2007.

\bibitem{col75}
G.~E. Collins.
\newblock Quantifier elimination for real closed fields by cylindrical
  algebraic decomposition.
\newblock {\em Springer Lecture Notes in Computer Science}, 33:515--532, 1975.

\bibitem{DaSKac09}
X.~Dahan, A.~Kadri, and {\'E}.~Schost.
\newblock Bit-size estimates for triangular sets in positive dimension.
\newblock Technical report, University of Western Ontario, 2009.

\bibitem{DMSWX05a}
X.~{Dahan}, M.~{{Moreno Maza}}, {\'E}.~{Schost}, W.~{Wu}, and Y.~{Xie}.
\newblock Lifting techniques for triangular decompositions.
\newblock In {\em ISSAC'05}, pages 108--115, 2005.

\bibitem{DST88}
J.H. Davenport, Y.~Siret, and E.~Tournier.
\newblock {\em {Computer Algebra}}.
\newblock Academic Press, 1988.

\bibitem{HS98}
H.~Hong and J.~R. Sendra.
\newblock Computation of variant results,
\newblock B.~Caviness and J.~Johnson, eds, {\em Quantifier Elimination
  and Cylindrical Algebraic Decomposition}, 
1998.


\bibitem{KrPaSo01}
T.~Krick, L.~M. Pardo, and M.~Sombra.
\newblock Sharp estimates for the arithmetic {N}ullstellensatz.
\newblock {\em Duke Math. J.}, 109(3):521--598, 2001.



\bibitem{LiMorenoPan09}
X.~Li, M.~{Moreno Maza}, and W.~Pan.
\newblock Computations modulo regular chains.
\newblock In {\em ISSAC'09}, pages 239--246, 2009.

\bibitem{MMM99}
M.~{{Moreno Maza}}.
\newblock On triangular decompositions of algebraic varieties.
\newblock MEGA-2000, Bath, UK.
\newblock {\tt http://www.csd.uwo.ca/}{moreno/books-papers.html}

\bibitem{pph3}
P.~Philippon.
\newblock Sur des hauteurs alternatives {III}.
\newblock {\em J. Math. Pures Appl.}, 74(4):345--365, 1995.

\bibitem{Ren92}
J.~Renegar.
\newblock On the computational complexity and geometry of the first-order
  theory of the reals. parts {I}--{III}.
\newblock {\em J. Symb. Comput.}, 13(3):255--352, 1992.

\bibitem{adam00}
A.~Strzebo\'{n}ski.
\newblock Solving systems of strict polynomial inequalities.
\newblock {\em J. Symb. Comput.}, 29(3):471--480, 2000.

\bibitem{Szanto99}
{\'A}.~Sz{\'a}nt{\'o}.
\newblock {\em Computation with polynomial systems}.
\newblock PhD thesis, Cornell University, 1999.

\bibitem{xz06}
B.~Xia and T.~Zhang.
\newblock Real solution isolation using interval arithmetic.
\newblock {\em Comput. Math. Appl.}, 52(6-7):853--860, 2006.

\bibitem{Xiao09}
R.~Xiao.
\newblock {\em Parametric Polynomial System Solving}.
\newblock PhD thesis, Peking University, Beijing, 2009.


\bibitem{yhx01}
L.~Yang, X.~Hou, and B.~Xia.
\newblock A complete algorithm for automated discovering of a class of
  inequality-type theorems.
\newblock {\em Science in China, Series \bf{F}}, 44(6):33--49, 2001.

\bibitem{YX08}
L.~Yang and B.~Xia.
\newblock {\em Automated proving and discovering inequalities}.
\newblock Science Press, Beijing, 2008.

\bibitem{YX05}
L.~Yang and B.~Xia.
\newblock Real solution classifications of a class of parametric semi-algebraic
  systems.
\newblock In {\em A3L'05}, pages 281--289, 2005.

\end{thebibliography}
\end{document}
