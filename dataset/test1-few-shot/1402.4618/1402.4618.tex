
\documentclass[11pt,twocolumn]{IEEEtran}




\pdfoutput=1  

\IEEEoverridecommandlockouts                              




\usepackage{graphicx,subfigure}
\usepackage{cite,url}

\usepackage[usenames,dvipsnames]{color}
 


\usepackage[cmex10]{amsmath}
\usepackage{amsfonts,amssymb,mathrsfs}

\usepackage[colorlinks,bookmarksopen,bookmarksnumbered,citecolor=red,urlcolor=red]{hyperref}




\def\adjsym{\star}
\def\ynom{\bar{y}^0}


\def\slPi{{\mit\Pi}}

\def\contains{\supset}


 





\newcommand*{\qed}{\nobreak\hfill\ensuremath{\square}}






\long\def\defbox#1{\framebox[.9\hsize][c]{\parbox{.85\hsize}{\parindent=0pt
\baselineskip=12pt plus .1pt      \parskip=6pt plus 1.5pt minus 1pt #1}}}




\long\def\beginbox#1\endbox{\subsection*{}\hbox{\hspace{.05\hsize}\defbox{\medskip#1\bigskip}}\subsection*{}}

\def\endbox{}



\def\transpose{{\hbox{\it\tiny T}}}


\def\mm{\phantom{-}}

\def\wkarrow{\buildrel{\rm w}\over\longrightarrow}
\def\vgarrow{\buildrel{\rm v}\over\longrightarrow}
\def\tvlimit#1{\buildrel{\scriptscriptstyle #1}\over\Longrightarrow}
\def\darrow{\buildrel{\rm d}\over\longrightarrow}

\def\goes#1{\buildrel{#1}\over\longrightarrow}
\def\wiggles#1{\buildrel{#1}\over\leadsto}


\def\tr{{\rm tr\, }}

\def\rank{{\rm rank\,}}
\def\deg{\hbox{\rm deg\thinspace}}
\def\sign{{\rm sign}}
\def\supp{{\rm supp\,}}


\newsavebox{\junk}
\savebox{\junk}[1.6mm]{\hbox{}}
\def\lll{{\usebox{\junk}}}


\def\det{{\mathop{\rm det}}}
\def\limsup{\mathop{\rm lim\ sup}}
\def\liminf{\mathop{\rm lim\ inf}}
\def\argmin{\mathop{\rm arg\, min}}

\def\argmax{\mathop{\rm arg\, max}}







\def\A{{\sf A}}
\def\K{{\sf K}}
\def\U{{\sf U}}
\def\V{{\sf V}}
\def\W{{\sf W}}

\def\state{{\sf X}}
\def\ystate{{\sf Y}}

\def\zstate{{\sf Z}}







\def\by{\clB(\ystate)}

\def\tinyplus{\vbox{\hrule width 3pt depth -1.5pt height 1.9pt
      \vskip -1.5pt
      \hbox{\hskip 1.3pt\vrule height 3pt depth 0pt width .4pt}}}


\def\xz{{\sf X^{\rm z}}}
\def\bxz{{\cal B}(\state^\hbox{\rm\tiny z})}
\def\xzplus{{\sf X^{{\rm z}_{ \tinyplus}}}}
\def\xbxm{({\sf X},{\cal B}({\bf X}),{\bf m})}
\def\bx{{{\cal B}(\state)}}
\def\bxzplus{{\cal B}(\state^{{\rm z}_{\tinyplus}} )} \def\bxplus{{\clB^+(\state)}}
 

\newcommand{\field}[1]{\mathbb{#1}}



\def\posRe{\field{R}_+}
\def\Re{\field{R}}

\def\A{\field{A}}


\def\One{\mbox{\rm{\large{1}}}}
\def\Zero{\mbox{\rm{\large{0}}}}


\def\ind{\field{I}}


\def\TT{\field{T}}
\def\ZZ{\field{Z}}


\def\intgr{\field{Z}}

\def\nat{\field{Z}_+}


\def\rat{\field{Q}}

\def\Co{\field{C}}




\def\cclF{{\check{\clF}}}
\def\catom{{\check{\atom}}}
\def\cExpect{\check{\Expect}}
\def\cPhi{\check{\Phi}}
\def\cbfPhi{\check{\bfPhi}}
\def\cProb{\check{\Prob}}
\def\cProbsub{\cProb\!}
\def\cPhi{\check{\Phi}}

\def\cpi{\check{\pi}}
\def\cA{\check{A}}

\def\cG{{\check{G}}}

\def\cP{{\check{P}}}
\def\cbxplus{{\cal B}^+(\cstate)}
\def\cstate{{\check{\state}}}
\def\bcx{\clB(\cstate)}

\def\LPhi{\bfPhi^{\Lambda}}






\def\bfA{{\bf A}}
\def\bfB{{\bf B}}
\def\bfC{{\bf C}}
\def\bfD{{\bf D}}
\def\bfE{{\bf E}}
\def\bfF{{\bf F}}
\def\bfG{{\bf G}}
\def\bfH{{\bf H}}
\def\bfI{{\bf I}}
\def\bfJ{{\bf J}}
\def\bfK{{\bf K}}
\def\bfL{{\bf L}}
\def\bfM{{\bf M}}
\def\bfN{{\bf N}}
\def\bfO{{\bf O}}
\def\bfP{{\bf P}}
\def\bfQ{{\bf Q}}
\def\bfR{{\bf R}}
\def\bfS{{\bf S}}
\def\bfT{{\bf T}}
\def\bfU{{\bf U}}
\def\bfV{{\bf V}}
\def\bfW{{\bf W}}
\def\bfX{{\bf X}}
\def\bfY{{\bf Y}}
\def\bfZ{{\bf Z}}

\def\bfa{{\bf a}}
\def\bfb{{\bf b}}
\def\bfc{{\bf c}}
\def\bfd{{\bf d}}
\def\bfe{{\bf e}}
\def\bff{{\bf f}}
\def\bfg{{\bf g}}
\def\bfh{{\bf h}}
\def\bfi{{\bf i}}
\def\bfj{{\bf j}}
\def\bfk{{\bf k}}
\def\bfl{{\bf l}}
\def\bfm{{\bf m}}
\def\bfn{{\bf n}}
\def\bfo{{\bf o}}
\def\bfp{{\bf p}}
\def\bfq{{\bf q}}
\def\bfr{{\bf r}}
\def\bfs{{\bf s}}
\def\bft{{\bf t}}
\def\bfu{{\bf u}}
\def\bfv{{\bf v}}
\def\bfw{{\bf w}}
\def\bfx{{\bf x}}
\def\bfy{{\bf y}}
\def\bfz{{\bf z}}






\def\bfmath#1{{\mathchoice{\mbox{\boldmath}}{\mbox{\boldmath}}{\mbox{\boldmath}}{\mbox{\boldmath}}}}
 







\def\bfma{\bfmath{a}}
\def\bfmb{\bfmath{b}}
\def\bfmd{\bfmath{d}}
\def\bfme{\bfmath{e}}
\def\bfmm{\bfmath{m}}
\def\bfmq{\bfmath{q}}
\def\bfmr{\bfmath{r}}
\def\bfmv{\bfmath{v}} 
\def\bfmx{\bfmath{x}}
\def\bfmy{\bfmath{y}}
\def\bfmu{\bfmath{u}}
\def\bfmw{\bfmath{w}}
 

\def\bfmhaw{\bfmath{\haw}}  

\def\bfmz{\bfmath{z}}  

\def\bfmA{\bfmath{A}}
\def\bfmB{\bfmath{B}}  

\def\bfmC{\bfmath{C}} 
\def\bfmD{\bfmath{D}} 
\def\bfmE{\bfmath{E}}   
\def\bfmF{\bfmath{F}}     
\def\bfmG{\bfmath{G}}     
\def\bfmH{\bfmath{H}}     
\def\bfmI{\bfmath{I}}      
\def\bfmhaI{\bfmath{\haI}}   
\def\bfmhaL{\bfmath{\haL}}     
\def\bfmL{\bfmath{L}}
\def\bfmM{\bfmath{M}}
\def\bfmN{\bfmath{N}}  
\def\bfmQ{\bfmath{Q}}  
\def\bfmR{\bfmath{R}}  
\def\bfmbarR{\bfmath{\bar{R}}}  
 

 
 

\def\bfmS{\bfmath{S}}  
\def\bfmT{\bfmath{T}}  
\def\bfmU{\bfmath{U}}  
 

\def\bfmhaN{\bfmath{\widehat N}}
\def\bfmhaQ{\bfmath{\widehat Q}}
 
\def\bfmX{\bfmath{X}}
\def\bfmtilX{\bfmath{\widetilde{X}}}

\def\bfmtilQ{\bfmath{\widetilde{Q}}}

\def\bfmtilq{\tilde{\bfmath{q}}}





\def\bfmY{\bfmath{Y}}   
\def\bfmhaY{\bfmath{\widehat Y}}


\def\bfmhhaY{\bfmath{\hhaY}} \def\bfmhhaY{\hbox to 0pt{\hss}\widehat{\phantom{\raise 1.25pt\hbox{}}}}
 

\def\bfmV{\bfmath{V}}  
\def\bfmW{\bfmath{W}}  
\def\bfmhaW{\bfmath{\haW}}  
\def\bfmZ{\bfmath{Z}}  


\def\bfmhaZ{{\bfmath{\widehat Z}}}


\def\bfmhaw{\bfmath{\widehat w}}
\def\bfmhay{\bfmath{\widehat y}}
\def\bfmhaz{\bfmath{\widehat z}}
\def\bfmhaQ{\bfmath{\widehat Q}}
\def\bfmhaS{\bfmath{\widehat S}}
\def\bfmhaU{\bfmath{\widehat U}}
\def\bfmhaW{\bfmath{\widehat W}}






\def\bfatom{\bfmath{\theta}}
\def\bfPhi{\bfmath{\Phi}}


\def\bfeta{\bfmath{\eta}}

\def\bfpi{\bfmath{\pi}} 

 

\def\bfphi{\bfmath{\phi}} 
\def\bfpsi{\bfmath{\psi}}


\def\bfrho{\bfmath{\rho}}
\def\bfzeta{\bfmath{\zeta}}

\def\bfUpsilon{\bfmath{\Upsilon}} 



  





\def\haPsi{{\widehat{\Psi}}}
\def\haGamma{{\widehat{\Gamma}}}
\def\haSigma{{\widehat{\Sigma}}}
\def\habfPhi{{\widehat{\bfPhi}}}
\def\haP{{\widehat P}}






\def\hac{{\hat c}}
\def\haz{\hat{z}}
\def\hai{{\hat {\imath}}}
\def\haf{{\hat f}}
\def\hag{{\hat g}}
\def\ham{{\hat m}}
\def\hah{{\hat h}}
\def\hap{{\hat p}}
\def\haq{{\hat q}}



\def\bfmhaQ{{\bfmath{}}}

\def\hav{\hat v}
\def\haw{{\hat w}}
\def\hay{{\hat y}}


\def\haF{{\widehat F}}
\def\haI{{\hat I}}
\def\haJ{{\hat J}}

\def\haN{\hat{N}}

\def\haQ{{\widehat Q}}
\def\haT{{\widehat T}}
\def\haW{{\widehat W}}

\def\haeta{{\hat\eta}}

\def\hatheta{{\hat\theta}}
\def\hapsi{{\hat\psi}}
\def\hapi{{\hat\pi}}
\def\hanu{{\hat\nu}}
\def\hamu{{\hat\mu}}
\def\halpha{{\hat\alpha}}
\def\habeta{{\hat\beta}}
\def\hagamma{{\hat\gamma}}

\def\halambda{{\hat\lambda}}
\def\haLambda{{\hat\Lambda}}



\def\rmA{{\rm A}}
\def\rmB{{\rm B}}
\def\rmC{{\rm C}}
\def\rmD{{\rm C}}
\def\rmE{{\rm E}}
\def\rmH{{\rm H}}
\def\rmI{{\rm I}}
\def\rmM{{\rm M}}
\def\rmP{{\rm P}}
\def\rmQ{{\rm Q}}

\def\rmd{{\rm d}}
\def\rmg{{\rm g}}
\def\rmk{{\rm k}}
\def\rmu{{\rm u}}
\def\rmx{{\rm x}}

\def\rmP{{\rm P}}
\def\rmE{{\rm E}}








\def\til={{\widetilde =}}
\def\tilPhi{{\widetilde \Phi}}
\def\tilN{{\widetilde N}}
\def\tilP{{\widetilde P}}
\def\tilQ{{\widetilde Q}}
\def\tilT{\widetilde T}
\def\tilW{\widetilde W}






\def\tilmu{{\tilde \mu}}
\def\tilpi{{\tilde \pi}}
\def\tilbfX{{\bf\tilde X}}
\def\tilbfm{{\bf \tilde  m}}
\def\tiltheta{{\tilde \theta}}
\def\tile{\tilde e}
\def\tilsigma{\tilde \sigma}
\def\tiltau{\tilde \tau}






\def\clA{{\cal A}}
\def\clB{{\cal B}}
\def\clC{{\cal C}}
\def\clD{{\cal D}}
\def\clE{{\cal E}}
\def\clF{{\cal F}}
\def\clG{{\cal G}}
\def\clH{{\cal H}}
\def\clJ{{\cal J}}
\def\clI{{\cal I}}
\def\clK{{\cal K}}
\def\clL{{\cal L}}
\def\clM{{\cal M}}
\def\clN{{\cal N}}
\def\clO{{\cal O}}
\def\clP{{\cal P}}
\def\clQ{{\cal Q}}
\def\clR{{\cal R}}
\def\clS{{\cal S}}
\def\clT{{\cal T}}
\def\clU{{\cal U}}
\def\clV{{\cal V}}
\def\clW{{\cal W}}
\def\clX{{\cal X}}
\def\clY{{\cal Y}}
\def\clZ{{\cal Z}}







\def\bfatom{{\mbox{\boldmath}}}
\def\atom{{\mathchoice{\bfatom}{\bfatom}{\alpha}{\alpha}}}

\def\Gexp{\clG^+(\gamma)}
\def\Gexps{\clG^+(s)}



\def\Var{\hbox{\sf Var}\,}


\def\At{A_+}
\def\barAt#1{\overline{A_+(#1)}}

\def\head#1{\subsubsection*{#1}}





\def\atopss#1#2{\genfrac{}{}{0cm}{2}{#1}{#2}} 

\def\atop#1#2{\genfrac{}{}{0pt}{}{#1}{#2}}

   
 \def\FRAC#1#2#3{\genfrac{}{}{}{#1}{#2}{#3}}
 


\def\fraction#1#2{{\mathchoice{\FRAC{0}{#1}{#2}}{\FRAC{1}{#1}{#2}}{\FRAC{3}{#1}{#2}}{\FRAC{3}{#1}{#2}}}}

\def\ddt{{\mathchoice{\FRAC{1}{d}{dt}}{\FRAC{1}{d}{dt}}{\FRAC{3}{d}{dt}}{\FRAC{3}{d}{dt}}}}


\def\ddr{{\mathchoice{\FRAC{1}{d}{dr}}{\FRAC{1}{d}{dr}}{\FRAC{3}{d}{dr}}{\FRAC{3}{d}{dr}}}}


\def\ddr{{\mathchoice{\FRAC{1}{d}{dy}}{\FRAC{1}{d}{dy}}{\FRAC{3}{d}{dy}}{\FRAC{3}{d}{dy}}}}


\def\ddtp{{\mathchoice{\FRAC{1}{d^{\hbox to 2pt{\rm\tiny +\hss}}}{dt}}{\FRAC{1}{d^{\hbox to 2pt{\rm\tiny +\hss}}}{dt}}{\FRAC{3}{d^{\hbox to 2pt{\rm\tiny +\hss}}}{dt}}{\FRAC{3}{d^{\hbox to 2pt{\rm\tiny +\hss}}}{dt}}}}


\def\ddalpha{{\mathchoice{\FRAC{1}{d}{d\alpha}}{\FRAC{1}{d}{d\alpha}}{\FRAC{3}{d}{d\alpha}}{\FRAC{3}{d}{d\alpha}}}}



\def\half{{\mathchoice{\FRAC{1}{1}{2}}{\FRAC{1}{1}{2}}{\FRAC{3}{1}{2}}{\FRAC{3}{1}{2}}}}

\def\third{{\mathchoice{\FRAC{1}{1}{3}}{\FRAC{1}{1}{3}}{\FRAC{3}{1}{3}}{\FRAC{3}{1}{3}}}}

\def\fourth{{\mathchoice{\FRAC{1}{1}{4}}{\FRAC{1}{1}{4}}{\FRAC{3}{1}{4}}{\FRAC{3}{1}{4}}}}
 
 \def\threefourth{{\mathchoice{\FRAC{1}{3}{4}}{\FRAC{1}{3}{4}}{\FRAC{3}{3}{4}}{\FRAC{3}{3}{4}}}}

\def\sixth{{\mathchoice{\FRAC{1}{1}{6}}{\FRAC{1}{1}{6}}{\FRAC{3}{1}{6}}{\FRAC{3}{1}{6}}}}

\def\eighth{{\mathchoice{\FRAC{1}{1}{8}}{\FRAC{1}{1}{8}}{\FRAC{3}{1}{8}}{\FRAC{3}{1}{8}}}}


\def\ddzeta{{\mathchoice{\FRAC{1}{d}{d\zeta}}{\FRAC{1}{d}{d\zeta}}{\FRAC{3}{d}{d\zeta}}{\FRAC{3}{d}{d\zeta}}}}



\def\ddt{\frac{d}{dt}}

\def\ddzeta{\frac{d}{d\zeta}}

\def\eqdist{\buildrel{\rm dist}\over =}

\def\eqdef{\mathbin{:=}}

\def\lebmeas{\mu^{\hbox{\rm \tiny Leb}}}
\def\Prbty{{\cal  M}}



\def\Prob{{\sf P}}
\def\Probsub{{\sf P\! }}
\def\Expect{{\sf E}}

\def\taboo#1{{{}_{#1}P}}

\def\io{{\rm i.o.}}


\def\as{{\rm a.s.}}
\def\Vec{{\rm vec\, }}



\def\oneN{{\displaystyle\frac{1}{ N}}}

\def\average#1,#2,{{1\over #2} \sum_{#1}^{#2}}

\def\enters{\hbox{ \rm \  enters\  }}

\def\eye(#1){{\bf(#1)}\quad}
\def\epsy{\varepsilon}

\def\varble{\,\cdot\,}




\newtheorem{theorem}{Theorem}[section]
\newtheorem{corollary}[theorem]{Corollary}
\newtheorem{proposition}[theorem]{Proposition}
\newtheorem{lemma}[theorem]{Lemma}



\def\Lemma#1{Lemma~\ref{#1}}
\def\Proposition#1{Proposition~\ref{#1}}
\def\Theorem#1{Theorem~\ref{#1}}
\def\Corollary#1{Corollary~\ref{#1}}


\def\Section#1{Sec.~\ref{#1}}

\def\Figure#1{Fig.~\ref{#1}} 

\def\Appendix#1{Appendix~\ref{c:#1}}

\def\eq#1/{(\ref{e:#1})}


 

\newcommand{\eeqn}{}

\def\bdes{\begin{description}}
\def\edes{\end{description}}

\def\bara{{\overline {a}}}

\def\barc{{\overline {c}}}

\def\barf{{\overline {f}}}
\def\barg{{\overline {g}}}
\def\barh{{\overline {h}}}
\def\barI{{\overline {l}}}
\def\barm{{\overline {m}}}
\def\barn{{\overline {n}}}

\def\barp{{\overline {p}}}
\newcommand{\barx}{{\bar{x}}}
\def\bary{{\overline {y}}}
\def\barA{{\overline {A}}}
\def\barB{{\overline {B}}}
\def\barC{{\overline {C}}}
\def\barE{{\overline {E}}}
\def\barM{{\overline {M}}}
\def\barP{{\overline {P}}}
\def\barQ{{\overline {Q}}}
\def\barT{{\overline {T}}}


\def\undern{{\underline{n}}}
\def\ubarho{{\underline{\rho}}}

\def\baratom{{\overline{\atom}}}
\def\barho{{\overline{\rho}}}
\def\barmu{{\overline{\mu}}}
\def\barnu{{\overline{\nu}}}
\def\baralpha{{\overline{\alpha}}}
\def\barbeta{{\overline{\beta}}}

\def\baralpha{{\overline{\alpha}}}
\def\bareta{{\overline{\eta}}}
 


\def\barbP{\overline{\bfPhi}}
\def\barPhi{\overline{\Phi}}
\def\barF{\overline{\clF}}
\def\barsigma{\overline{\sigma}}

\def\barSigma{\overline{\Sigma}}


\def\bartau{\overline{\tau}}



\def\Sampsigma{\hat{\sigma}}
\def\Samptau{\hat{\tau}}
\def\SampP{\hat{P}}
\def\SampPhi{\hat{\Phi}}
\def\SampbfPhi{\hat{\bfPhi}}
\def\SampF{\hat{\cal F}}







     
\newcounter{rmnum}
\newenvironment{romannum}{\begin{list}{{\upshape (\roman{rmnum})}}{\usecounter{rmnum}
\setlength{\leftmargin}{14pt}
\setlength{\rightmargin}{8pt}
\setlength{\itemsep}{2pt}
\setlength{\itemindent}{-1pt}
}}{\end{list}}

\newcounter{anum}
\newenvironment{alphanum}{\begin{list}{{\upshape (\alph{anum})}}{\usecounter{anum}
\setlength{\leftmargin}{14pt}
\setlength{\rightmargin}{8pt}
\setlength{\itemsep}{2pt}
\setlength{\itemindent}{-1pt}
}}{\end{list}}






\newenvironment{assumptions}[1]{\setcounter{keep_counter}{\value{#1}}
\def\aref{#1}
\begin{list}{\sf\, (#1\arabic{#1})\mindex{Assumption (#1\arabic{#1})}\, }{\usecounter{#1}\setlength{\rightmargin}{\leftmargin}}
\setcounter{#1}{\value{keep_counter}}}{\end{list}}



\def\ass(#1:#2){(#1\ref{#1:#2})}




\def\ritem#1{
\item[{\sf \ass(\current_model:#1)}]
}

\newenvironment{recall-ass}[1]{\begin{description}
\def\current_model{#1}}{
\end{description}
}


\def\Ebox#1#2{\begin{center}
 \parbox{#1\hsize}{\epsfxsize=\hsize \epsfbox{#2}}
\end{center}}




\newcommand{\bd}{\begin{description}}
\newcommand{\ed}{\end{description}}
\newcommand{\bt}{\begin{theorem}}
\newcommand{\et}{\end{theorem}}
\newcommand{\ba}{\begin{array}{rcl}}
\newcommand{\ea}{\end{array}} 


\def\MT{\cite{meytwe92d}}



\def\head#1{\paragraph{#1}}


 
\newlength{\noteWidth}
\setlength{\noteWidth}{.5in}
\long\def\notes#1{\ifinner
           {\tiny #1}
           \else
           \marginpar{\parbox[t]{\noteWidth}{\raggedright\tiny #1}}
       \fi\typeout{#1}}

\def\notes#1{}  

\def\spm#1{\typeout{#1}}

\setcounter{secnumdepth}{3} 
\setcounter{tocdepth}{1}

 
 \def\rd#1{{\color{red}#1}}


\makeindex
 
\def\Ebox#1#2{\begin{center}    
\includegraphics[width=#1\hsize]{#2}
\end{center}} 


\def\util{\mathchoice{\mbox{\small}}{\mbox{\small}}{\mbox{}}{\mbox{}}}


\def\tilutil{\mathchoice{\mbox{\small}}{\mbox{\small}}{\mbox{}}{\mbox{}}}


\def\welf{\mathchoice{\mbox{\small}}{\mbox{\small}}{\mbox{}}{\mbox{}}}
 
\def\cp{{\check{p}}}
\def\cL{{\check{L}}}
\def\cmu{{\check{\mu}}}
\def\cgenerate{{\check{\generate}}}
\def\hagenerate{{\widehat{\generate}}}

\def\Fig#1{Fig.~\ref{#1}}     


\def\Prop#1{Prop.~\ref{#1}}     
\def\Thm#1{Thm.~\ref{#1}}     
 

 


 









 
 
\def\haX{\widehat X} 
 

\def\NN{}


\def\st{\text{s.t.}}
  
\def\cust{{\hbox{\sf\scriptsize\raisebox{.03cm}{c}}}}
\def\serve{{\hbox{\sf\scriptsize\raisebox{.03cm}{s}}}}
  


 \def\rd#1{{\color{red}#1}}
 \def\Real{\text{Re}\,}

 \def\diag{\text{diag}\,}

\def\generate{{\cal D}}

\def\tilkappa{\tilde\kappa}


\title{\LARGE \bf
Passive Dynamics in Mean Field Control
}




\author{Ana Bu\v{s}i\'c and Sean Meyn\thanks{This research is supported by the French National Research Agency grant ANR-12-MONU-0019,  NSF grants CPS-0931416 and  CPS-1259040,   and  US-Israel BSF Grant 2011506.}\thanks{A.B.\ is with Inria and the Computer Science Dept. of \'Ecole Normale Sup\'erieure, Paris, France.}\thanks{
S.M. is with the Department of Electrical and Computer
Engg.\ at the University of Florida, Gainesville.}}


  


\begin{document}
 

\maketitle
\thispagestyle{empty}
\pagestyle{empty}




 
\begin{abstract} 

Mean-field models are a popular tool in a variety of fields.  They provide an understanding of  the
impact of interactions among a large number of particles or people or other ``self-interested agents'', and are an increasingly popular tool in distributed control.

This paper considers a particular randomized distributed control architecture introduced in our own recent work. In numerical results it was found that  the associated mean-field model had attractive properties for purposes of control.  In particular, when viewed as an input-output system, its linearization was found to be minimum phase.

In this paper we take a closer look at the control model.  The   results are summarized as follows:
\begin{romannum}
\item 
The Markov Decision Process framework of Todorov is extended to continuous time models,  in which the ``control cost'' is based on relative entropy.  This is the basis of the construction of a family of Markovian generators,  parameterized by a scalar .

\item 
A decentralized control architecture is proposed in which each agent evolves as a controlled Markov process.   A central authority broadcasts a common control signal  to each agent.    The central authority chooses   based on an  aggregate scalar output of the Markovian agents.

\textit{This is the basis of the mean field model.  }


 
\item
Provided the control-free system (with ) is a reversible Markov process, the following identity holds for the transfer function  obtained from the linearization,  

where the right hand side denotes the power spectral density for the output of any one of the 
individual Markov processes  (with ).  
\end{romannum}


\end{abstract} 
 
  
 


 

\section{Introduction}


Mean field models are a standard  tool in physics when analyzing a large number of particles, where an individual particle has negligible impact upon the ensemble.  Similar models are the foundation of competitive equilibrium theory in economics, and mean field models are increasingly popular in   control theory  \cite{huacaimal07,borsun12,gasgauleb12,guaragwil12}.

The present work considers application to distributed control, inspired by numerical results  in our prior work \cite{meybarbusyueehr14} on automated demand response for a large collection of loads.  The goal was to obtain \textit{ancillary service} to help regulate the power grid, as in many prior works \cite{coupertemdeb12,macalhis10}.   


The paper \cite{meybarbusyueehr14}  focused on a large population of ``on-off'' loads, with special attention to residential pool pumps.   The normal operation of a pool pump was modeled as a Markov decision process, which included as an exogenous input a regulation signal from a balancing authority.   This resulted in an input-output system with input equal to the regulation signal, and output equal to the number of pools in operation.  In the numerical example considered, the control system had some very attractive properties:  Its linearization was stable, and simulations of one million pools resulted in behavior very closely matched to the deterministic linear model obtained from linearization of the Markovian dynamics.  Most important was the finding that the linearization was \textit{minimum phase}.  This is a valuable property in any control system.  

In this paper we set out to see why these conclusions might be expected in greater generality.

To explain the goals of the paper we take a high-level look at the prior work \cite{meybarbusyueehr14}.
 Shown in \Fig{fig:pppDynamics} is a state transition diagram for the discrete-time Markovian model considered in \cite{meybarbusyueehr14}.  
 The variables   and 
indicate the probability of turning a pool pump on (respectively, off),  which depends upon how long the pool has been off (respectively, on).
 \begin{figure}[h]
\Ebox{.9}{pppDynamicsCSS.pdf} 
\vspace{-.25cm}
\caption{State transition diagram for the pool-pump model.
}
\label{fig:pppDynamics} 
\end{figure} 


A continuous time counterpart is described by 
a model on a continuous state space, 

If ,  this means that the pool pump has been operating for exactly  seconds.
If , this implies that the pump  was turned on at time .
The differential generator for this Markovian model is defined  for functions  that are differentiable in their second variable.  There are functions  and  such that for any such ,
.2cm]
q^\oplus(\tau) [ f(\oplus,0) - f(\ominus,\tau) ] + \frac{\partial}{\partial \tau} f\, (\ominus,\tau) , \\
\hfill x= (\ominus,\tau)
\end{cases}

\lim_{N\to\infty}
\frac{1}{N}\sum_{i=1}^N  \ind\{ X^i_t \in A \} = \mu_t(A)\,, \quad A\subset \state.  
\label{e:mfgPool}

\ddt \mu_t = \mu_t \generate_{\zeta_t}
\label{e:muState}

\ddt \int f(x)\mu_t(dx) =   \int  \bigl( \generate_{\zeta_t}
f\, (x)\bigr)\mu_t(dx) \, .

y_t =   \int \util(x)\mu_t(dx)  
\label{e:muOutput}

\pi \generate\, (x^j) = \sum_{i=1}^d \pi(x^i) \generate(x^i, x^j) = 0,\qquad x^j\in\state.

\ddt \Phi_t = A \Phi_t + B \zeta_t,\qquad \gamma_t = C \Phi_t 
\label{e:LSSmfg}

C_i = \tilutil(x^i) =  \util(x^i) -\ynom,
\label{e:Cdefn}

B_j = \sum_{i=1}^d \pi(x^i) \generate'_0(x^i, x^j) ,\quad 1\le j\le d.
\label{e:B}

\Real (G(j\omega)) \ge 0,\qquad \omega\in\Re.   
\label{e:spr}

\Real (G(j\omega)) = \text{PSD}_Y(\omega) \qquad \omega\in\Re\,,
\label{e:sprPSD}

\begin{aligned}
\generate f\, (x)  &= \sum_{x'} \generate(x,x') f(x') \\
& = \lim_{t\downarrow 0} \frac{1}{t} \Expect[f(X_t) - f(X_0) \mid X_0=x]\,, \quad x\in\state.
\label{e:gendef}
\end{aligned}

\label{e:Twelfare}
\welf_T(p) =    \zeta \Expect_p\Bigl[\int_0^T \util(X_t)\, dt\Bigr] -D(p \| p^0)

\Expect_p[F] = \Expect[e^L F]
\label{e:likely}

D(p \| p^0) = \Expect_p[L]

L^* =  {- \Lambda^*_T}  + \int_0^T  \util(X_t) \, dt 
\label{e:Lstar}

 \Lambda^*_T 
 = \log \Bigl(\Expect \Bigl[ \exp\Bigl(\int_0^T  \util(X_t) \, dt\Bigr) \Bigr]\Bigr)
\label{e:LambdaNorm1}
 
D( p^* \| p) =    \Expect_{p^*}[L^*] = -\Lambda^*_T  +\Expect_{p^*}\Bigl[\int_0^T \util(X_t)\, dt\Bigr]

\max_{p} \welf_T(p) =\welf_T( p^*) = \Lambda^*_T

1= \Expect [e^{L^*}]=
e^{-\Lambda^*_T}\Expect \Bigl[ \exp\Bigl(\int_0^T  \util(X_t) \, dt\Bigr) \Bigr]
\label{e:LambdaNorm}

\lim_{T\to\infty} \frac{1}{T} \welf_T( \cp_T)  
=
\welf_\infty^*
\eqdef 
\lim_{T\to\infty} \frac{1}{T} \welf_T( p^*_T)   
\label{e:Cinfty1}

[\generate + I_{\util}] v = \Lambda v
\label{e:evector}

\cgenerate =   \ind_{v}^{-1} [\generate + I_{\util} - \Lambda  I]\ind_{v}
\label{e:cgenerate}

\welf_T(\cp_T)   =  T\welf_\infty^* - \Expect\Bigl[\log\Bigl( \frac{v(X_T)}{v(X_0)} \Bigr)  \Bigr]   
 
\welf_\infty^* =
\lim_{T\to\infty} \frac{1}{T} \log\Bigl( \Expect \Bigl[ \exp\Bigl(\int_0^T  \util(X_t) \, dt\Bigr) \Bigr]  \Bigr)
\label{e:CinftyLambda}

\Expect_{\cp}[F] = \Expect\bigl[ e^{\cL_T}F\bigr]

\cL_T = \log\Bigl( \frac{v(X_T)}{v(X_0)} \Bigr) +  \int_0^T[  \util(X_t) -\Lambda ]\, dt 

1= \Expect [e^{\cL_T}]=
e^{-\Lambda T}\Expect \Bigl[  \frac{v(X_T)}{v(X_0)}  \exp\Bigl(\int_0^T  \util(X_t) \, dt\Bigr) \Bigr]

\lim_{T\to\infty} \frac{1}{T} \welf_T(\cp)  = \Lambda(\util) = 
\lim_{T\to\infty} \frac{1}{T} \Lambda^*_T  =
\welf_\infty^* 

 \begin{aligned}
\welf_T(\cp)  
&=     \Expect_{\cp}\Bigl[\int_0^T \util(X_t)\, dt\Bigr] -D(\cp_T \| p^0_T)
\\
  &=  - \Expect\Bigl[\log\Bigl( \frac{v(X_T)}{v(X_0)} \Bigr)  \Bigr]  + \Lambda T
\end{aligned}

[\generate + \zeta I_{\util}] v_\zeta = \Lambda_\zeta v_\zeta
\label{e:evectorz}
\Lambda_\eta=
\Lambda(\zeta\util)
\eqdef
\lim_{T\to\infty} \frac{1}{T} \log \Expect\Bigl[ \exp\Bigl(  \zeta\int_0^T  \util(X_t) \, dt\Bigr) \Bigr] 
   \frac{v_\zeta(x^j)}{v_\zeta(x^i)} \bigl[\generate(x^i,x^j) +  (\zeta\util (x^i) - \Lambda_\zeta )\ind\{x^j=x^i\} \bigr]
\label{e:cgenerate-zeta}

\ddt \mu_t\, (x) = \sum_{x^i\in\state} \mu_t(x^i) \generate_{\zeta_t}(x^i,x)
\label{e:muStateFinite}

\pi(x^i) \generate_0(x^i,x^j) = \pi(x^j) \generate_0(x^j,x^i) ,\qquad x^i,x^j\in\state 

\Real G(j\omega) =  \text{PSD}_Y(\omega)  ,\qquad \omega\in\Re\,,
\label{e:sweet}

\text{
 for .}
\label{e:G}

R_s = \int_0^\infty e^{-s t} P^t \, dt

R_s =[s I - \generate]^{-1} 
\label{e:RsGen}

G(s) = C[Is-A]^{-1}B = B^\transpose [Is-A^\transpose]^{-1}C^\transpose = B^\transpose R_s C^\transpose 
\label{e:GR}

G(j\omega) 
=
\int_0^\infty e^{-j\omega t} \Expect_\pi[f(X_0) g(X_t)]\, dt  

\begin{aligned}
G(s) & =  \int_0^\infty e^{-s t} \Bigl[\sum_{i,j} P^t(x^i,x^j)B_i C_j  \Bigr]\, dt   \\
& =
\int_0^\infty e^{-s t} \Expect_\pi[f(X_0) g(X_t)]\, dt  
\end{aligned}

\generate_0 h_0 = -\tilutil
\label{e:Fish}

h(x)  = R_0 \tilutil \, (x) = \int_0^\infty  \Expect\bigl[ \tilutil(X_t) \mid X_0 =x\bigr]\, dt ,\quad x\in\state
\label{e:Fishpi0}

\generate^\adjsym (x^i,x^j) =  \pi(x^j)  \generate (x^j,x^i) \frac{1}{\pi(x^i)},\qquad x^i,x^j\in\state

B_i =  
-\pi(x^i)
 [   \generate^\adjsym h_0\, (x^i)   + \generate h_0\, (x^i)   ]

\ddzeta \Lambda_\zeta \Big|_{\zeta=0} = \ynom = \sum_i \pi(x^i) \util(x^i)

\ddzeta v_\zeta(x^i) \Big|_{\zeta=0}  = h_0(x^i),\qquad x^i\in\state.

\begin{aligned}
\generate_\zeta (x^i,x^j) 
&=  [1- \zeta h_0(x^i)] \generate (x^i,x^j) [1+ \zeta h_0(x^j)] \\
&  \quad +  [\zeta \util(x^i)   - \Lambda_\zeta ]\ind\{x^i=x^j\} +o(\zeta)
  \\
  &=  [1- \zeta h_0(x^i)] \generate (x^i,x^j) [1+ \zeta h_0(x^j)] \\
  & \quad +  \zeta [\util(x^i) - \ynom] \ind\{x^i=x^j\}     +o(\zeta)
\end{aligned}

\begin{aligned}
\ddzeta \generate_\zeta (x^i,x^j)  \Big|_{\zeta=0} & =  -   h_0(x^i)  \generate (x^i,x^j) +   \generate (x^i,x^j) h_0(x^j) \\
& \; \quad  +    \tilutil(x^i) \ind\{x^i=x^j\}    
\end{aligned}

\begin{aligned}
B_j &= \sum_{x^i} \pi(x^i) \Bigl(-   h_0(x^i)  \generate (x^i,x^j) +   \generate (x^i,x^j) h_0(x^j) \\
& \; \quad +    \tilutil(x^i) \ind\{x^i=x^j\}    
\Bigr)
\\
  &=   - \sum_{x^i}    h_0(x^i) \pi(x^i)  \generate (x^i,x^j)   + \pi(x^j) \tilutil(x^j) 
\end{aligned}

 f(x^i) = B_i/\pi(x^i) = 2 \tilutil(x^i),\qquad g(x^i)=C_i=\tilutil(x^i)
 
 \begin{aligned}
G(j\omega) 
&=
\int_0^\infty e^{-j\omega t} \Expect_\pi[f(X_0) g(X_t)]\, dt  
\\
&=
2\int_0^\infty e^{-j\omega t} \Expect_\pi[\tilutil(X_0) \tilutil(X_t)]\, dt  
\end{aligned}

 \begin{aligned}
 \Real
G(j\omega)  
&=
2\Real\int_0^\infty e^{-j\omega t} \Expect_\pi[\tilutil(X_0) \tilutil(X_t)]\, dt  
\\
&=
 \int_{-\infty}^\infty e^{-j\omega t} \Expect_\pi[\tilutil(X_0) \tilutil(X_t)]\, dt  
\end{aligned}

\qed




\section{Conclusions}
\label{s:conc}

This paper gives a general condition under which the linearization of a mean field model is positive-real. 

The linearization around  is a natural choice, but the main result of the paper can be extended to any constant value:  If  is reversible, then so is  for each fixed .   Based on this observation, it is possible to show that the linearization about any fixed value of  is positive real under the assumptions of \Thm{t:reversiblePassive}.
This suggests an open question:  Is the nonlinear model with state equation \eqref{e:muState}
 passive?  
Passivity would be a valuable property for the purposes of control.
 
There are many open questions in the context of design.  Can we obtain more general sufficient conditions for the positive real condition, the weaker minimum phase condition, or the stronger  passivity condition for the nonlinear model?

To relax the assumptions of \Thm{t:reversiblePassive}, it is likely that we will require application of the Kalman-Yakubovich-Popov Lemma, which provides an algebraic characterization of the passive real condition \cite{ran96}.
 
We are currently considering these theoretical questions, and applications to problems in decentralized control, especially in power systems settings.

\notes{say somewhere:  The minimum phase condition observed in the numerical examples in \cite{meybarbusyueehr14} cannot be explained by any of the theory in this paper
since a Markov model with state transition diagram shown in \Fig{fig:pppDynamics} 
 cannot be reversible.}



 \newpage
 
 \def\cprime{}\def\cprime{}

\bigskip

\begin{thebibliography}{10}

\bibitem{borsun12}
V.S. Borkar and R.~Sundaresan.
\newblock Asympotics of the invariant measure in mean field models with jumps.
\newblock {\em Stochastic Systems}, 2(2):322--380, 2012.

\bibitem{coupertemdeb12}
R.~Couillet, S.M. Perlaza, H.~Tembine, and M.~Debbah.
\newblock Electrical vehicles in the smart grid: A mean field game analysis.
\newblock {\em {IEEE} Journal on Selected Areas in Communications},
  30(6):1086--1096, 2012.

\bibitem{demzei98a}
A.~Dembo and O.~Zeitouni.
\newblock {\em Large Deviations Techniques And Applications}.
\newblock Springer-Verlag, New York, second edition, 1998.

\bibitem{gasgauleb12}
N.~Gast, B.~Gaujal, and J.-Y. Le~Boudec.
\newblock Mean field for {Markov} decision processes: From discrete to
  continuous optimization.
\newblock {\em IEEE Trans. Automat. Control}, 57(9):2266--2280, 2012.

 
\bibitem{guaragwil12}
P.~Guan, M.~Raginsky, and R.~Willett.
\newblock Online {Markov} decision processes with {Kullback-Leibler} control
  cost.
\newblock In {\em American Control Conference (ACC)}, pages 1388--1393, 2012.



\bibitem{huacaimal07}
M.~Huang, P.~E. Caines, and R.~P. Malhame.
\newblock Large-population cost-coupled {LQG} problems with nonuniform agents:
  Individual-mass behavior and decentralized {{}-Nash} equilibria.
\newblock {\em IEEE Trans. Automat. Control}, 52(9):1560--1571, 2007.

\bibitem{kal64}
R.~Kalman.
\newblock When is a linear control system optimal?
\newblock {\em Journal of Basic Engineering}, 86:51, 1964.

\bibitem{konmey03a}
I.~Kontoyiannis and S.~P. Meyn.
\newblock Spectral theory and limit theorems for geometrically ergodic {Markov}
  processes.
\newblock {\em Ann. Appl. Probab.}, 13:304--362, 2003.


\bibitem{konmey05a}
I.~Kontoyiannis and S.~P. Meyn.
\newblock Large deviations asymptotics and the spectral theory of
  multiplicatively regular {M}arkov processes.
\newblock {\em Electron. J. Probab.}, 10(3):61--123 (electronic), 2005.

\bibitem{kul54}
S.~Kullback.
\newblock Certain inequalities in information theory and the {Cramer-Rao}
  inequality.
\newblock {\em Ann. Math. Statist.}, 25(4):745--751, 1954.



\bibitem{macalhis10}
Z.~Ma, D.~Callaway, and I.~Hiskens.
\newblock Decentralized charging control for large populations of plug-in
  electric vehicles: Application of the {Nash} certainty equivalence principle.
\newblock In {\em Control Applications (CCA), 2010 IEEE International
  Conference on}, pages 191--195,  2010.


\bibitem{meybarbusyueehr14}
S.~Meyn, P.~Barooah, A.~Bu\v{s}i\'{c}, Y.~Chen, and J.~Ehren.
\newblock Ancillary service to the grid using intelligent deferrable loads.
\newblock {\em {ArXiv e-prints: arXiv:1402.4600 and to appear, IEEE Trans. Automat. Control}}, 2014.
Based on the invited paper published in the {\em {Proceedings of the 52nd IEEE Conf. on Decision and
  Control}}, pages 6946--6953, 2013.


\bibitem{pramenrun96}
P.~Pra, L.~Meneghini, and W.~Runggaldier.
\newblock Connections between stochastic control and dynamic games.
\newblock {\em Mathematics of Control, Signals and Systems}, 9(4):303--326,
  1996.



 

\bibitem{ran96}
A.~Rantzer.
\newblock On the {Kalman-Yakubovich-Popov} lemma.
\newblock {\em Systems \&\ Control Letters}, 28(1):7--10, 1996.

\bibitem{safath77}
M.~Safonov and M.~Athans.
\newblock Gain and phase margin for multiloop {LQG} regulators.
\newblock {\em IEEE Trans. Automat. Control}, 22(2):173--179, 1977.

\bibitem{huaunnmeyveesur11}
J.~Unnikrishnan, D.~Huang, S.~P. Meyn, A.~Surana, and V.~V. Veeravalli.
\newblock Universal and composite hypothesis testing via mismatched divergence.
\newblock {\em IEEE Trans. Inform. Theory}, 57(3):1587--1603, 2011.

\end{thebibliography}


\end{document}
