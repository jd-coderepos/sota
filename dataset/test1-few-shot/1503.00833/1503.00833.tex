\documentclass{llncs}

\pagestyle{plain}
\thispagestyle{plain}

\usepackage{amsmath}
\usepackage{amssymb}
\usepackage[dvipdfmx]{graphicx}


\newcommand{\IS}{{\sc IS}}
\newcommand{\ISFULL}{{\sc Independent Set}}
\newcommand{\ISR}{{\sc ISR}}
\newcommand{\ISRFULL}{{\sc Independent Set Reconfiguration}}
\newcommand{\VC}{{\sc VC}}
\newcommand{\VCFULL}{{\sc Vertex Cover}}
\newcommand{\VCR}{{\sc VCR}}
\newcommand{\VCRFULL}{{\sc Vertex Cover Reconfiguration}}
\newcommand{\DS}{{\sc DS}}
\newcommand{\DSFULL}{{\sc Dominating Set}}
\newcommand{\DSR}{{\sc DSR}}
\newcommand{\DSRFULL}{{\sc Dominating Set Reconfiguration}}

\newcommand{\YES}{\mathsf{yes}}
\newcommand{\NO}{\mathsf{no}}

\newcommand{\anyds}{D}
\newcommand{\canods}{C}
\newcommand{\vercano}[1]{w_{#1}}

\newcommand{\parta}{V_1}
\newcommand{\partb}{V_2}
\newcommand{\partc}{V_3}

\newcommand{\calI}{\mathcal{I}}
\newcommand{\interval}[1]{h(#1)}
\newcommand{\intervala}[1]{l(#1)}
\newcommand{\intervalb}[1]{r(#1)}

\newcommand{\sevstep}[1]{\overset{#1}{\leftrightsquigarrow}}

\newcommand{\cocompg}[1]{G_{#1}}
\newcommand{\cocompnum}{p}
\newcommand{\coga}{G_a}
\newcommand{\cogb}{G_b}
\newcommand{\cographa}{w_a}
\newcommand{\cographb}{w_b}
\newcommand{\cocanods}{C}

\newenvironment{listing}[1]{\begin{list}{*}{\settowidth{\labelwidth}{#1}\setlength{\leftmargin}{\labelwidth}\advance \leftmargin by 12pt
\setlength{\itemsep}{0pt}\setlength{\parsep}{0pt}\setlength{\topsep}{0pt}\setlength{\parskip}{0pt}}}{\end{list}}



\begin{document}
	\title{The complexity of\\ dominating set reconfiguration}

\author{Arash Haddadan\inst{1} \and
	Takehiro Ito\inst{2} \and
	Amer E. Mouawad\inst{1} \and\\
	Naomi Nishimura\inst{1} \and
	Hirotaka Ono\inst{3} \and
	Akira Suzuki\inst{2} \and
	Youcef Tebbal\inst{1}
}

\institute{University of Waterloo,\\
	200 University Ave. West, Waterloo, Ontario N2L 3G1, Canada.\\
	\email{\{ahaddada, aabdomou, nishi, ytebbal\}@uwaterloo.ca}
\and
	Graduate School of Information Sciences, Tohoku University, \\
	Aoba-yama 6-6-05, Sendai, 980-8579, Japan.\\
	\email{\{takehiro, a.suzuki\}@ecei.tohoku.ac.jp}
\and
	Faculty of Economics, Kyushu University, \\
	Hakozaki 6-19-1, Higashi-ku, Fukuoka, 812-8581, Japan.\\
	\email{hirotaka@econ.kyushu-u.ac.jp}
}

\maketitle

\begin{abstract}
Suppose that we are given two dominating sets  and  of a graph 
whose cardinalities are at most a given threshold .
Then, we are asked whether there exists a sequence of dominating sets of  between
 and  such that each dominating set in the sequence is of cardinality
at most  and can be obtained from the previous one by either adding or deleting exactly one vertex.
This problem is known to be PSPACE-complete in general.
In this paper, we study the complexity of this decision problem from the viewpoint of graph classes.
We first prove that the problem remains PSPACE-complete even for planar graphs, bounded bandwidth graphs, split graphs, and bipartite graphs.
We then give a general scheme to construct linear-time algorithms and show
that the problem can be solved in linear time for cographs, trees, and interval graphs.
Furthermore, for these tractable cases, we can obtain a desired sequence such
that the number of additions and deletions is bounded by , where  is the number of vertices in the input graph.
\end{abstract}

\section{Introduction}

Consider the art gallery problem modeled on graphs:
Each vertex corresponds to a room which has a monitoring camera
and each edge represents the adjacency of two rooms.
Assume that each camera in a room can monitor the room itself and its adjacent rooms.
Then, we wish to find a subset of cameras that can monitor all rooms;
the corresponding vertex subset  of the graph  is called a {\em dominating set},
that is, every vertex in  is either in  or adjacent to a vertex in .
For example, \figurename~\ref{fig:example} shows six different dominating sets of the same graph.
Given a graph  and a positive integer , the problem of determining
whether  has a dominating set of cardinality at most
 is a classical NP-complete problem~\cite{GJ79}.

\subsection{Our problem}

However, the art gallery problem could be considered in more ``dynamic'' situations:
In order to maintain the cameras, we sometimes need to change the current dominating set into another one.
This transformation needs to be done by switching the cameras individually and we certainly need to
keep monitoring all rooms, even during the transformation.

In this paper, we thus study the following problem:
Suppose that we are given two dominating sets of a graph  whose cardinalities are at most
a given threshold  (e.g., the leftmost and rightmost ones in \figurename~\ref{fig:example}, where ), and
we are asked whether we can transform one into the other via dominating sets of  such that each
intermediate dominating set is of cardinality at most  and can be obtained from the
previous one by either adding or deleting a single vertex.
We call this decision problem the {\sc dominating set reconfiguration (DSR)} problem.
For the particular instance of \figurename~\ref{fig:example}, the answer is  as illustrated in \figurename~\ref{fig:example}.

\begin{figure}[t]
\centering
	\includegraphics[width=0.9\linewidth]{fig/example.eps}
	\vspace{-1em}
	\caption{A sequence  of dominating sets in the
    same graph, where  and the vertices in dominating sets are depicted by large (blue) circles.}
	\vspace{-1em}
	\label{fig:example}
\end{figure}
	
\subsection{Known and related results}

Recently, similar problems have been extensively studied under the
reconfiguration framework~\cite{IDHPSUU}, which arises when we
wish to find a step-by-step transformation between two feasible solutions of
a combinatorial problem such that all intermediate solutions are also feasible.
The reconfiguration framework has been applied to several well-studied problems, including
{\sc satisfiability}~\cite{Kolaitis},
{\sc independent set}~\cite{HearnDemaine2005,IDHPSUU,KaminskiMM12,MNRSS13,Wro14},
{\sc vertex cover}~\cite{IDHPSUU,INZ14,MNR14,MNRSS13},
{\sc clique}, {\sc matching}~\cite{IDHPSUU},
{\sc vertex-coloring}~\cite{BC09},
and so on. (See also a survey~\cite{van13}.)

Mouawad et al.~\cite{MNRSS13} proved that {\sc dominating set reconfiguration} is -hard
when parameterized by , where  is the cardinality threshold
of dominating sets and  is the length of a sequence of dominating sets.
	
Haas and Seyffarth~\cite{HS14} gave sufficient conditions for the
cardinality threshold  for which any two dominating sets can be transformed into one another.
They proved that the answer to {\sc dominating set reconfiguration} is  for
a graph  with  vertices if  and  has a matching of cardinality at least two;
they also gave a better sufficient condition when restricted to bipartite or chordal graphs.
Recently, Suzuki et al.~\cite{SMN14} improved the former condition and showed
that the answer is  if  and  has a matching of
cardinality at least , for any nonnegative integer .

\subsection{Our contribution}
	
To the best of our knowledge, no algorithmic results are known
for the {\sc dominating set reconfiguration} problem and it is therefore desirable
to obtain a better understanding of what separates ``hard'' from ``easy'' instances.
To that end, we study the problem from the viewpoint of
graph classes and paint an interesting picture of the boundary
between intractability and polynomial-time solvability.
(See also \figurename~\ref{fig:results}.)

We first prove that the problem is PSPACE-complete even on
planar graphs, bounded bandwidth graphs, split graphs, and bipartite graphs.
Our reductions for PSPACE-hardness follow from the classical reductions
for proving the NP-hardness of {\sc dominating set}.
However, the reductions should be constructed carefully so that they preserve
not only the existence of dominating sets but also the reconfigurability.
	
We then give a general scheme to construct linear-time algorithms for the problem.
As examples of its application, we demonstrate that the problem can
be solved in linear time on cographs (also known as -free graphs), trees, and interval graphs.
Furthermore, for these tractable cases, we can obtain a desired sequence such
that the number of additions and deletions (i.e., the length of a reconfiguration sequence) can
be bounded by , where  is the number of vertices in the input graph.
	
Proofs of lemmas and theorems marked with a star can be found in the appendix.

\begin{figure}[t]
    \centering
	\includegraphics[width=\linewidth]{fig/results.eps}
	\vspace{-1em}
	\caption{Our results, where each arrow represents the inclusion relationship between graph classes:
			 represents that  is properly included in ~\cite{BLS99}.
We also show PSPACE-completeness on graphs of bounded bandwidth (Theorem~\ref{the:hardness1}).}
	\vspace{-1em}
    \label{fig:results}
\end{figure}

\section{Preliminaries}


\subsubsection{Graph notation and dominating set.}

We assume that each input graph  is a simple undirected graph with vertex set 
and edge set , where  and .
For a set  of vertices, the subgraph of  {\em induced} by  is
denoted by , where  has vertex set  and edge set .

For a vertex  in a graph , we let  and .
For a set  of vertices, we define  and .
We sometimes drop the subscript  if it is clear from the context.

For a graph , a set  is a {\em dominating set} of  if .
Note that  always forms a dominating set of .
For a vertex  and a dominating set  of , we say that  is {\em dominated} by  if  and .
A vertex  in a dominating set  is {\em deletable} if  is also a dominating set of .
A dominating set  of  is {\em minimal} if there is no deletable vertex in .

\subsubsection{Dominating set reconfiguration.}
	
We say that two dominating sets  and  of the same graph 
are {\em adjacent} if there exists a vertex  such that
,
i.e.  is the only vertex in the {\em symmetric difference} of  and .
For two dominating sets  and  of , a sequence  of
dominating sets of  is called a {\em reconfiguration sequence} between  and  if it has the following properties:
\begin{listing}{aaa}
\item[(a)]  and ; and
\item[(b)]  and  are adjacent for each .
\end{listing}
Note that any reconfiguration sequence is {\em reversible}, that is,  is
also a reconfiguration sequence between  and .
We say a vertex  is {\em touched} in a reconfiguration
sequence  if  is either added or deleted at least once in .

For two dominating sets  and  of a graph  and an integer , we
write  if there exists a reconfiguration sequence 
between  and  in  such that  holds for every , for some .
Note that  clearly holds if .
Then, the {\sc dominating set reconfiguration (DSR)} problem is defined as follows:	
\begin{center}
	\parbox{0.85\hsize}{
\begin{listing}{{\bf Question:}}
	\item[{\bf Input:}] A graph , two dominating sets  and  of , and an integer threshold 
	\item[{\bf Question:}] Determine whether  or not.
	\end{listing}}
\end{center}
We denote by a -tuple  an instance of {\sc dominating set reconfiguration}.
Note that {\sc DSR} is a decision problem and hence it does not ask for an actual reconfiguration sequence.
We always denote by  and  the {\em source} and {\em target} dominating sets of , respectively.

\section{PSPACE-completeness}\label{sec:hardness}

In this section, we prove that {\sc dominating set reconfiguration} remains PSPACE-complete even for restricted classes of graphs;
some of these classes show nice contrasts to our algorithmic results in Section~\ref{sec:algo}.
(See also \figurename~\ref{fig:results}.)

\begin{theorem}\label{the:hardness1}
{\sc DSR} is PSPACE-complete on planar graphs of maximum degree six and on graphs of bounded bandwidth.
\end{theorem}

\begin{proof}
One can observe that the problem is in PSPACE~\cite[Theorem~1]{IDHPSUU}.
We thus show that it is PSPACE-hard for those graph classes by a polynomial-time
reduction from {\sc vertex cover reconfiguration}~\cite{IDHPSUU,INZ14,MNR14}.
In {\sc vertex cover reconfiguration}, we are given two vertex covers  and  of a graph
 such that  and , for some integer , and asked
whether there exists a reconfiguration sequence of vertex covers
 of  such that , , ,
and  for each .
	
Our reduction follows from the classical reduction from {\sc vertex cover} to {\sc dominating set}~\cite{GJ79}.
Specifically, for every edge  in , we add a new vertex 
and join it with each of  and  by two new edges  and ;
let  be the resulting graph.
Then, let  be the corresponding instance of {\sc dominating set reconfiguration}.
Clearly, this instance can be constructed in polynomial time.

We now prove that  holds if and only if there
is a reconfiguration sequence of vertex covers in  between  and .
However, the if direction is trivial, because any vertex cover of
 forms a dominating set of  and both problems employ the same
reconfiguration rule (i.e., the symmetric difference is of size one).
Therefore, suppose that  holds, and hence there exists
a reconfiguration sequence of dominating sets in  between  and .
Recall that neither  nor  contain a newly added vertex in .
Thus, if a vertex  in  is touched, then  must be added first.
By the construction of , both  and  hold.
Therefore, we can replace the addition of  by that of either
 or  and obtain a (possibly shorter) reconfiguration sequence
of dominating sets in  between  and  which touches vertices only in .
Then, it is a reconfiguration sequence of vertex covers in 
between  and , as needed.
	
{\sc Vertex cover reconfiguration} is known to be PSPACE-complete
on planar graphs of maximum degree three~\cite{INZ14,MNR14} and on graphs of bounded bandwidth~\cite{Wro14}.
Thus, the reduction above implies PSPACE-hardness on planar
graphs of maximum degree six and on graphs of bounded bandwidth;
note that, since the number of edges in  is only the
triple of that in , the bandwidth increases only by a constant multiplicative factor.
\qed
\end{proof}

We note that both pathwidth and treewidth
of a graph  are bounded by the bandwidth of .
Thus, Theorem~\ref{the:hardness1} yields that
{\sc dominating set reconfiguration} is PSPACE-complete on
graphs of bounded pathwidth and treewidth.

Adapting known techniques from NP-hardness proofs for the {\sc dominating set} problem~\cite{Ber84}, 
we also show PSPACE-completeness of {\sc dominating set reconfiguration}
on split graphs and on bipartite graphs; a graph is {\em split} if its vertex set can be
partitioned into a clique and an independent set~\cite{BLS99}.

\begin{theorem}[*]\label{the:split}
{\sc DSR} is PSPACE-complete on split graphs.
\end{theorem}

\begin{theorem}[*]\label{the:bipartite}
{\sc DSR} is PSPACE-complete on bipartite graphs.
\end{theorem}

\section{General scheme for linear-time algorithms} \label{sec:algo}

In this section, we show that {\sc dominating set reconfiguration} is
solvable in linear time on cographs, trees, and interval graphs.
Interestingly, these results can be obtained by the application
of the same strategy; we first describe the general scheme in Section~\ref{dsr:genestra}.
We then show in Sections~\ref{dsr:cograph}--\ref{dsr:alinterval} that the
problem can be solved in linear time on those graph classes.

\subsection{General scheme}\label{dsr:genestra}
The general idea is to introduce the concept of a ``canonical'' dominating set for a graph .
We say that a minimum dominating set  of  is {\em canonical}
if  holds for every dominating set  of  and .
Then, we have the following theorem.

\begin{theorem} \label{the:canonical}
If a graph  has a canonical dominating set, then {\sc dominating set reconfiguration} can be solved in linear time on  .
\end{theorem}

We note that proving the existence of a canonical dominating set is sufficient for solving the decision problem.
Therefore, we do not need to find an actual canonical dominating set in linear time.
In Sections~\ref{dsr:cograph}--\ref{dsr:alinterval}, we will show that cographs, trees, and interval graphs admit
canonical dominating sets, and hence the problem can be solved in linear time on those graph classes.
Note that, however, Theorem~\ref{the:canonical} can be applied to any graph which has a canonical dominating set.
In the remainder of this subsection, we prove Theorem~\ref{the:canonical} starting 
with the following lemma.	

\begin{lemma} \label{lem:plusone}
Suppose that a graph  has a canonical dominating set.
Then, an instance  of {\sc dominating set reconfiguration} is a -instance if .
\end{lemma}

\begin{proof}
Let  be a canonical dominating set of .
Then,  holds for .
Suppose that .
Since , we clearly have .
Similarly, we have .
Since any reconfiguration sequence is reversible, we have , as needed.
\qed
\end{proof}
	
Lemma~\ref{lem:plusone} implies that if a graph  has a canonical
dominating set , then it suffices to consider the case where .
Note that there exist -instances of {\sc dominating set reconfiguration} in
such a case but we show that they can be easily identified in linear time, as implied by the following lemma.

\begin{lemma} \label{lem:pluszero}
Let  be an instance of {\sc dominating set reconfiguration},
where  is a graph admitting a canonical dominating set and .
Then,  is a -instance if and only if
 is not minimal for every  such that .
\end{lemma}

Lemma~\ref{lem:pluszero} can be immediately obtained from the following lemma.

\begin{lemma} \label{lem:minimal}
Suppose that a graph  has a canonical dominating set .
Let  be an arbitrary dominating set of  and let .
Then,  holds if and only if  is not a minimal dominating set.
\end{lemma}

\begin{proof}
{\em Necessity.}
Suppose that  is not minimal.
Then,  contains at least one vertex  which is deletable
from , that is,  forms a dominating set of .
Since , we have .
Therefore,  holds.

\noindent	
{\em Sufficiency.}
We prove the contrapositive.
Suppose that  is minimal.
Then, no vertex in  is deletable and hence any dominating
set  which is adjacent to  must be obtained by adding a vertex to .
Therefore,  for any dominating
set  which is adjacent to . Hence,  does not hold.
\qed
\end{proof}
	
We note again that Lemmas~\ref{lem:plusone} and \ref{lem:pluszero} imply that
an actual canonical dominating set is not required to solve the problem.
Furthermore, it can be easily determined in linear time whether a dominating set of a graph  is minimal or not.
Thus, Theorem~\ref{the:canonical} follows from Lemmas~\ref{lem:plusone} and \ref{lem:pluszero}.
\medskip

Before constructing canonical dominating sets in
Sections~\ref{dsr:cograph}--\ref{dsr:alinterval}, we give the following lemma showing that
it suffices to construct a canonical dominating set for a connected graph.

\begin{lemma}[*]\label{lem:conncted}
Let  be a graph consisting of  connected components .
For each , suppose that  is a canonical dominating set for .
Then,  is a canonical dominating set for .
\end{lemma}

\subsection{Cographs} \label{dsr:cograph}
	
We first define the class of cographs (also known as -free graphs)~\cite{BLS99}.
For two graphs  and , their {\em union } is the graph such
that  and , while
their {\em join } is the graph such that 
and .
Then, a {\em cograph} can be recursively defined as follows:
\begin{listing}{aaa}
	\item[{\rm (1)}] a graph consisting of a single vertex is a cograph;
	\item[{\rm (2)}] if  and  are cographs, then the union  is a cograph; and
	\item[{\rm (3)}] if  and  are cographs, then the join  is a cograph.
\end{listing}

In this subsection, we show that {\sc dominating set reconfiguration} is solvable in linear time on cographs.
By Theorem~\ref{the:canonical}, it suffices to prove the following lemma.
\begin{lemma} \label{lem:cographcanonical}
Any cograph admits a canonical dominating set.
\end{lemma}

As a proof of Lemma~\ref{lem:cographcanonical}, we will construct a canonical dominating set for any cograph .
By Lemma~\ref{lem:conncted}, it suffices to consider the case where  is
connected and we may assume that  has at least two vertices, because otherwise the problem is trivial.
Then, from the definition of cographs,  must be obtained by the join
operation applied to two cographs  and , that is, .
Notice that any pair  of vertices  and  forms a dominating set of .
Let  be a dominating set of , defined as follows:

\begin{listing}{a}
	\item[-] If there exists a vertex  such that , then let .
	\item[-] Otherwise choose an arbitrary pair of vertices 
    and  and let .
\end{listing}

\noindent
Clearly,  is a minimum dominating set of .
We thus prove the following lemma, which completes the proof of Lemma~\ref{lem:cographcanonical}.

\begin{lemma}[*]\label{lem:cograph03}
For every dominating set  of ,  holds, where .
\end{lemma}

We have thus proved that any cograph has a canonical dominating set.
Then, Theorem~\ref{the:canonical} gives the following corollary.

\begin{corollary}
{\sc DSR} can be solved in linear time on cographs.
\end{corollary}

\subsection{Trees}\label{dsr:altree}

In this subsection, we show that {\sc dominating set reconfiguration} is solvable in linear time on trees.
As for cographs, it suffices to prove the following lemma.

\begin{lemma} \label{lem:treecanonical}
Any tree admits a canonical dominating set.
\end{lemma}

As a proof of Lemma~\ref{lem:treecanonical}, we will construct a canonical dominating set for a tree .
We choose an arbitrary vertex  of degree one in  and regard  as a rooted tree with root .

We first label each vertex in  either , , or , starting from the
leaves of  up to the root  of , as in the following steps (1)--(3);
intuitively, the vertices labeled  will form a dominating set of ,
each vertex labeled  will be dominated by its parent, and
each vertex labeled  will be dominated by at least one of its children
(see also \figurename~\ref{fig:tree}(a)):

\begin{listing}{aaa}
	\item[(1)] All leaves in  are labeled .
	\item[(2)] Pick an internal vertex  of , which is not the root, such that all children of  have already been labeled.
					Then,
					\begin{listing}{a}
					\item[-] assign  label  if all children of  are labeled ;
					\item[-] assign  label  if at least one child of  is labeled ; and
					\item[-] otherwise assign  label .
					\end{listing}
	\item[(3)] Assign the root  (of degree one) label  if its child is labeled , otherwise assign  label .
\end{listing}
For each , we denote by  the set of all vertices in  that are assigned label .
Then,  forms a partition of .

\begin{figure}[t]
	\centering
		\includegraphics[width=0.85\linewidth]{fig/tree.eps}
		\vspace{-1em}
		\caption{(a) The labeling of a tree , and (b) the partition of  into .}
		\vspace{-1em}
	\label{fig:tree}
\end{figure}

We will prove that  forms a canonical dominating set of .
We first prove, in Lemmas~\ref{lem:treeds} and \ref{lem:treemin}, that  is a minimum dominating set of 
and then prove, in Lemma~\ref{lem:treereach}, that  holds
for every dominating set  of  and .

\begin{lemma} \label{lem:treeds}
 is a dominating set of .
\end{lemma}

\begin{proof}
It suffices to show that both  and  hold.

Let  be any vertex in , and hence  is labeled .
Then, by the construction above,  is not the root of  and the parent of  must be labeled .
Therefore,  holds, as claimed.

Let  be any vertex in , and hence  is labeled .
Then,  is not a leaf of . Notice that label  is assigned to a
vertex only when at least one of its children is labeled .
Thus,  holds.
\qed
\end{proof}

We now prove that  is a minimum dominating set of .
To do so, we introduce some notation.
Suppose that the vertices in  are ordered as  by
a post-order depth-first traversal of the tree starting from the root  of .
For each , we denote by  the subtree
of  which is induced by  and all its descendants in .
Then, for each , we define a vertex subset  of  as follows
(see also \figurename~\ref{fig:tree}(b)):



\noindent
Note that  forms a partition of .
Furthermore, notice that

holds for every .
Then, Eq.~(\ref{eq:exactlyone}) and the following lemma imply that  is a minimum dominating set of .

\begin{lemma}[*]\label{lem:treemin}
Let  be an arbitrary dominating set of .
Then,  holds for every .
\end{lemma}

We finally prove the following lemma, which completes the proof of Lemma~\ref{lem:treecanonical}.
\begin{lemma}[*]\label{lem:treereach}
For every dominating set  of ,  holds, where .
\end{lemma}

We have thus proved that  forms a canonical dominating set for any tree .
Then, Theorem~\ref{the:canonical} gives the following corollary.
\begin{corollary}
{\sc DSR} can be solved in linear time on trees.
\end{corollary}

\subsection{Interval graphs}\label{dsr:alinterval}
A graph  with  is an {\em interval graph} if
there exists a set  of (closed) intervals  such
that  if and only if  for each .
We call the set  of intervals an {\em interval representation} of the graph.
For a given graph , it can be determined in linear time whether  is
an interval graph, and if so obtain an interval representation of ~\cite{KM89}.
In this subsection, we show that {\sc dominating set reconfiguration} is solvable in linear time on interval graphs.
As for cographs, it suffices to prove the following lemma.
\begin{lemma}\label{lemma:interval01}
Any interval graph admits a canonical dominating set.
\end{lemma}

As a proof of Lemma~\ref{lemma:interval01}, we will construct a canonical dominating set for any interval graph .
By Lemma~\ref{lem:conncted} it suffices to consider the case where  is connected.
Let  be an interval representation of .
For an interval , we denote by  and  the left and right endpoints of , respectively;
we sometimes call the values  and  the {\em -value} and {\em -value} of , respectively.
As for trees, we first label each vertex in  either , , or , from left to right;
the vertices labeled  will form a dominating set of  (see \figurename~\ref{fig:interval} as an example):
\begin{listing}{aaa}
\item[(1)] Pick the unlabeled vertex  which has the minimum -value among all unlabeled vertices and assign  label .
\item[(2)] Let  be the vertex in  which has the maximum -value among all vertices in .
            Note that  may have been already labeled and  may hold.
			We (re)label  to .
\item[(3)] For each unlabeled vertex in , we assign it label .
\end{listing}

\noindent
We execute steps (1)--(3) above until all vertices are labeled.
For each , we denote by  the set of all vertices in  that are assigned label .
Then,  forms a partition of .

By the construction above, it is easy to see that  forms a dominating set of .
We thus prove that  is canonical in Lemmas~\ref{lemma:interval03}
and \ref{lemma:interval04}, that is,  is a minimum dominating
set of  (in Lemma~\ref{lemma:interval03}) and  holds for
every dominating set  of  and  (in Lemma~\ref{lemma:interval04}).

\begin{figure}[t]
	\centering
		\includegraphics[width=0.85\linewidth]{fig/interval.eps}
		\vspace{-1em}
		\caption{The labeling of an interval graph in the interval representation.}
		\vspace{-1em}
	\label{fig:interval}
\end{figure}





We now prove that the dominating set  of  is minimum.
To do so, we introduce some notation.
Assume that the vertices in  are ordered as  such
that .
For each , we define the vertex subset
 of  as follows (see \figurename~\ref{fig:interval} as an example):
	


\noindent
Note that  forms a partition of  such that

holds for every .
Then, Eq.~(\ref{eq:interval:minimum01}) and the following lemma imply that  is a minimum dominating set of .
	
\begin{lemma}[*]\label{lemma:interval03}
Let  be an arbitrary dominating set of .
Then,  holds for every .
\end{lemma}

We finally prove the following lemma, which completes the proof of Lemma~\ref{lemma:interval01}.

\begin{lemma}[*]\label{lemma:interval04}
For every dominating set  of ,  holds, where .
\end{lemma}

Combining Lemma~\ref{lemma:interval01} and Theorem~\ref{the:canonical} yields the following corollary.
\begin{corollary} \label{cor:interval}
{\sc DSR} can be solved in linear time on interval graphs.
\end{corollary}


\section{Concluding remarks} \label{dsr:conclusion}

In this paper, we delineated the complexity of the
{\sc dominating set reconfiguration} problem restricted to various graph classes.
As shown in \figurename~\ref{fig:results}, our results clarify some
interesting boundaries on the graph classes lying between tractability and PSPACE-completeness:
For example, the structure of interval graphs can be seen as a path-like structure of cliques.
As a super-class of interval graphs, the well-known class of chordal graphs has a tree-like structure of cliques.
We have proved that {\sc dominating set reconfiguration} is solvable in
linear time on interval graphs, while it is PSPACE-complete on chordal graphs.

We note again that our linear-time algorithms for cographs, trees, and interval
graphs employ the same strategy. We also emphasize that this general
scheme can be applied to any graph which admits a canonical dominating set.
It is easy to modify our algorithms so that they actually
find a reconfiguration sequence for a -instance  on cographs, trees, or interval graphs.
Observe that each vertex is touched at most once in the
reconfiguration sequence from  (or ) to the canonical dominating set.
Therefore, for a -instance on an -vertex graph belonging to one of those classes, there exists a reconfiguration
sequence between  and  which touches vertices only  times.
In other words, the length of a shortest reconfiguration sequence between  and  can be bounded by .

\subsubsection*{Acknowledgments.}
This work is partially supported by the Natural Science and
Engineering Research Council of Canada (A.~Mouawad, N.~Nishimura and Y.~Tebbal) and
by MEXT/JSPS KAKENHI 25106504 and 25330003 (T.~Ito), 25104521 and 26540005 (H.~Ono), and 26730001 (A.~Suzuki).

\bibliographystyle{abbrv}
\begin{thebibliography}{99}

\bibitem{Ber84}
Bertossi, A.A.:
Dominating sets for split and bipartite graphs.
Information Processing Letters 19, pp.~37--40 (1984)



\bibitem{BC09}
Bonsma, P., Cereceda, L.:
Finding paths between graph colourings: PSPACE-completeness and superpolynomial distances.
Theoretical Computer Science
410, pp.~5215--5226 (2009)


\bibitem{BLS99}
Brandst\"adt, A., Le, V.B., Spinrad, J.P.:
Graph Classes: A Survey, SIAM (1999)




\bibitem{GJ79}
Garey, M.R., Johnson, D.S.:
Computers and Intractability: A Guide to the Theory of {NP}-Completeness.
Freeman, San Francisco (1979)



\bibitem{Kolaitis}
Gopalan, P., Kolaitis, P.G., Maneva, E.N., Papadimitriou, C.H.:
The connectivity of Boolean satisfiability: computational and structural dichotomies.
SIAM J.~Computing 38, pp.~2330--2355 (2009)

\bibitem{HS14}
Haas, R., Seyffarth, K.:
The -dominating graph.
Graphs and Combinatorics 30, pp.~609--617 (2014)

\bibitem{HearnDemaine2005}
Hearn, R.A., Demaine, E.D.:
PSPACE-completeness of sliding-block puzzles and other problems through the nondeterministic constraint logic model of computation.
Theoretical Computer Science
343, pp.~72--96 (2005)


\bibitem{IDHPSUU}
Ito, T., Demaine, E.D., Harvey, N.J.A., Papadimitriou, C.H., Sideri, M., Uehara, R., Uno, Y.:
On the complexity of reconfiguration problems.
Theoretical Computer Science
412, pp.~1054--1065 (2011)

\bibitem{INZ14}
Ito, T., Nooka, H., Zhou, X.:
Reconfiguration of vertex covers in a graph.
To appear in Proc. of IWOCA 2014.

\bibitem{KaminskiMM12}
Kami\'nski, M., Medvedev, P., Milani, M.:
Complexity of independent set reconfigurability problems.
Theoretical Computer Science 439, pp.~9--15 (2012)

\bibitem{KM89}
Korte, N., M{\"{o}}hring, R.:
An incremental linear-time algorithm for recognizing interval graphs.
SIAM J.~Computing 18, pp.~68--81 (1989)



\bibitem{MNR14}
Mouawad, A.E., Nishimura, N., Raman, V.:
Vertex cover reconfiguration and beyond.
Proc.~of ISAAC~2014, LNCS 8889, pp.~452--463 (2014)

\bibitem{MNRSS13}
Mouawad, A.E., Nishimura, N., Raman, V., Simjour, N., Suzuki, A.:
On the parameterized complexity of reconfiguration problems.
Proc.~of IPEC 2013, LNCS 8246, pp.~281--294 (2013)



\bibitem{SMN14}
Suzuki, A., Mouawad, A.E., Nishimura, N.:
Reconfiguration of dominating sets.
Proc.~of COCOON~2014, LNCS 8591, pp.~405--416 (2014)



\bibitem{van13}
van den Heuvel, J.:
The complexity of change.
Surveys in Combinatorics 2013,
London Mathematical Society Lecture Notes Series 409 (2013).

\bibitem{Wro14}
Wrochna, M.:
Reconfiguration in bounded bandwidth and treedepth.
{\tt  arXiv:1405.0847} (2014)

\end{thebibliography}

\newpage
\appendix
\section*{Appendix}
\renewcommand{\thesubsection}{\Alph{subsection}}

\subsection{Details omitted from Section~\ref{sec:hardness}}
\subsubsection{Proof of Theorem~\ref{the:split}}
\begin{proof}
We again give a polynomial-time reduction from {\sc vertex cover reconfiguration}.
We extend the idea developed for the NP-hardness proof of {\sc dominating set} on split graphs~\cite{Ber84}.
	
Let  be an instance of {\sc vertex cover reconfiguration}, where
 and .
We construct the corresponding split graph , as follows.
(See also \figurename~\ref{fig:reduction}(a) and (b).)
Let , where  and ;
each vertex  corresponds to the edge  in .
We join all pairs of vertices in  so that  forms a clique in .
In addition, for each edge  in , we join  with each of  and  in .
Let  be the resulting graph, and let  be the
corresponding instance of {\sc dominating set reconfiguration}.
Clearly, this instance can be constructed in polynomial time.
Thus, we will prove that  holds if and only if there is a
reconfiguration sequence of vertex covers in  between  and .

We first prove the if direction.
Because both problems employ the same reconfiguration rule, it suffices to prove
that any vertex cover  of  forms a dominating set of .
Since  and  is a clique, all vertices in  are dominated by the vertices in .
Thus, consider a vertex  in , which corresponds to the edge  in .
Then, since  is a vertex cover of , at least one of  and  must be contained in .
This means that  is dominated by the endpoint  or  in .
Therefore,  is a dominating set of .
	
\begin{figure}
    \centering
	\includegraphics[width=0.9\linewidth]{fig/reduction.eps}
	\vspace{-1em}
	\caption{(a) Vertex cover  of a graph, (b) dominating set
     of the corresponding split graph, and (c) dominating set  of the corresponding bipartite graph.}
	\vspace{-1em}
	\label{fig:reduction}
\end{figure}

We now prove the only-if direction.
Notice that, for each vertex  corresponding to
the edge  in , we have  and .
Therefore, if  holds, then we can obtain a reconfiguration
sequence of dominating sets in  between  and  which touches vertices only in ;
recall the arguments in the proof of Theorem~\ref{the:hardness1}.
Observe that any dominating set  of  such that 
forms a vertex cover of , because each vertex  is dominated by at least one vertex in .
We have thus verified the only-if direction.
\qed
\end{proof}

\subsubsection{Proof of Theorem~\ref{the:bipartite}}
\begin{proof}
We give a polynomial-time reduction from {\sc dominating set reconfiguration}
on split graphs to the same problem restricted to bipartite graphs.
The same idea is used in the NP-hardness proof of {\sc dominating set} for bipartite graphs~\cite{Ber84}.
	
Let  be an instance
of {\sc dominating set reconfiguration}, where  is a split graph.
Then,  can be partitioned into two subsets  and  which
form a clique and an independent set in , respectively.
Furthermore, by the reduction given in the proof of Theorem~\ref{the:split}, the problem
for split graphs remains PSPACE-complete even if both  and  hold. 	
	
We now construct the corresponding bipartite graph , as follows.
(See also \figurename~\ref{fig:reduction}(b) and (c).)
First, we delete any edge joining two vertices in , and make  an independent set.
Then, we add a new edge consisting of two new vertices  and  and join  with each vertex in .
The resulting graph  is bipartite.
Let , , , and we
obtain the corresponding {\sc dominating set reconfiguration} instance , where  is bipartite.
Clearly, this instance can be constructed in polynomial time.
Thus, we will prove that  holds if and only if  holds.

We first prove the if direction.
Suppose that  holds. Hence, there exists
a reconfiguration sequence in  between  and .
Consider any dominating set  of  in this sequence.
Then,  holds because  and we
have deleted only the edges such that both endpoints are in .
Since , we can conclude that  is a dominating set of .
Furthermore, .
Thus,  holds.

We then prove the only-if direction.
Suppose that  holds, and hence there exists a
reconfiguration sequence in  between  and .
Notice that any dominating set of  contains at least one of  and .
Since  and , we can assume that  is
contained in all dominating sets in the reconfiguration sequence.
Recall that both  and  hold.
Thus, if a vertex  is touched, then it must be added first.
Since , we have , where .
Therefore, we can replace the addition of  by that of either  or  and
obtain a reconfiguration sequence in  between  and  which touches vertices only in .
Consider any dominating set  of  in such a reconfiguration sequence.
Since , we have .
Furthermore, since  and  forms a clique
in , we have .
Since there is no edge joining  and a vertex in , each vertex in
 is dominated by some vertex in .
Therefore,  is a dominating set of  of
cardinality at most  and  holds.
\qed
\end{proof}

\subsection{Details omitted from Section~\ref{sec:algo}}
\subsubsection{Proof of Lemma~\ref{lem:conncted}}
\begin{proof}
Let  be any dominating set of .
For each , since  is
canonical for , we have  for .
 Therefore, we can independently transform  into  for each .
Clearly, this is a reconfiguration sequence from  to .
Furthermore, since  is a minimum dominating set of , we have
 for each .
Thus, any dominating set appearing in the sequence is of cardinality at most .
\qed
\end{proof}

\subsection{Details omitted from Section~\ref{dsr:cograph}}
\subsubsection{Proof of Lemma~\ref{lem:cograph03}}
\begin{proof}
We construct a reconfiguration sequence from  to  such that
each intermediate dominating set is of cardinality at most .
\medskip

\noindent
{\bf Case (i):} .
	
In this case,  consists of a universal vertex , that is, .
Therefore, we first add  to  if , and then delete the vertices in  one by one.
Since , all intermediate vertex subsets are dominating sets of .
Since the addition is applied only to , we have  for .
\medskip

\noindent
{\bf Case (ii):} .

In this case,  consists of two vertices  and .
Since  is a minimum dominating set of , we have .
Note that, however,  or  may hold.
We assume without loss of generality that .
Then, we construct a sequence of vertex subsets of , as follows:
\begin{listing}{aaa}
	\item[(1)] Add  to  if ; let .
	\item[(2)] If , then delete
    one vertex in ; otherwise delete a vertex
    in  if it exists.
	Let  be the resulting vertex subset of .
	\item[(3)] Add  to  if ; let .
	\item[(4)] Delete from  all vertices in  one by one.
\end{listing}
We will prove that each vertex subset appearing above is a dominating set of  with cardinality at most .
Indeed, it suffices to show that  is a dominating set of  such that ;
note that  contains both  and  and hence
any vertex subset appearing in Steps~(3) and (4) above is a dominating set of  with cardinality at most .

We first consider the case where .
In this case, , and hence we
can delete one vertex   from .
We thus have , as required.
Since ,   contains at least one vertex in .
Furthermore,  and hence  is a dominating set of .
	
We then consider the case where .
Note that, since  and , we
have  in this case.
Let .
If  (and hence ) then 
and .
Therefore,  and .
Furthermore, since  and ,  is a dominating set of .
On the other hand, if , then we have .
Consequently,  and hence  is a dominating set of  of cardinality .
\qed
\end{proof}

\subsection{Details omitted from Section~\ref{dsr:altree}}
\subsubsection{Proof of Lemma~\ref{lem:treemin}}
\begin{proof}
Suppose for a contradiction that  holds for some index .
We will prove that  contains at least one vertex  such that .
Then, since , the vertex  is not dominated by any vertex in ;
this contradicts the assumption that  is a dominating set of .
Recall that all leaves in  are labeled , and hence  is an internal vertex.
	
First, consider the case where  has a child  which is a leaf of .
Then,  holds for the leaf ; a contradiction.

Second, consider the case where , that is,   contains the root  of .
Recall that  is of degree one and is labeled either  or ;
we will prove that  holds.
If  is labeled , then its (unique) child  is labeled  and hence .
Therefore,  contains both  and  and hence  holds; a contradiction.
On the other hand, if  is labeled  and hence , then its child  is labeled either  or .
Therefore,  contains both  and , and hence  holds; a contradiction.

Finally, consider the case where  and  is an
internal vertex such that all children of  are also internal vertices in .
Since  is labeled , there exists at least one child  of  which is labeled .
Then, since  is an internal vertex, all children of  (and hence all ``grandchildren'' of ) are labeled .
Therefore,  holds for the child  of ; a contradiction.
\qed
\end{proof}

\subsubsection{Proof of Lemma~\ref{lem:treereach}}
\begin{proof}
We construct a reconfiguration sequence from  to 
such that each intermediate dominating set is of cardinality at most .
	
Let .
For each  from  to , we focus on the vertices in 
and transform  into  as follows:

\begin{listing}{aaa}
	\item[(1)] add the vertex  to  if ;
	\item[(2)] delete the vertices in  one by one; and
	\item[(3)] let  be the resulting vertex set.
\end{listing}
\smallskip
	
We first claim that  forms a dominating set of  for each .
Notice that  for the resulting
vertex set . Moreover, only the root  of  is adjacent to a vertex in .
Since  and both  and  form
dominating sets of , we can conclude that  forms a dominating set of .
Then, all vertex subsets appearing in Steps~(1) and (2) above also form
dominating sets of , because each of them is a superset of .

We then claim that  for each .
If , then the claim clearly holds because
we only delete vertices in Step~(2) without adding the vertex  in Step~(1).
We thus consider the case where .
Since  is a dominating set of , Lemma~\ref{lem:treemin} implies
that  in this case.
Therefore, we have .

Note that, since addition is executed only in Step~(1), the maximum cardinality of any dominating set
in the reconfiguration sequence from  to  is at most .
Since  for each , the maximum cardinality
of any dominating set in the reconfiguration sequence
from   to   is at most .
Therefore, there exists a reconfiguration sequence from  to  such that all
intermediate dominating sets are of cardinality at most .
\qed
\end{proof}

\subsection{Details omitted from Section~\ref{dsr:alinterval}}
\subsubsection{Proof of Lemma~\ref{lemma:interval03}}
\begin{proof}
Suppose for a contradiction that  holds for some index .
Assume that the vertices in  are ordered as 
such that .
Then, observe that  holds for every .
In addition,  holds if .

First, we consider the case where both  and  hold;
in this case, both  and  hold.
Since ,  must be
dominated by some vertex  in .
Then,  and hence we have .
Since , by Eq.~(\ref{eq:interval:minimum00}) we have 
and hence .
Therefore,  holds and  must be labeled .
This contradicts the assumption that  is labeled .

We now consider the other case, that is, both 
and  hold for index .
Since ,  must be dominated
by at least one vertex in  or .
If  is dominated by some vertex in , then the same arguments given
above yield a contradiction, i.e.  must be labeled  even though  is in .
Therefore,  must be dominated by some vertex  in .
Then, since , we have .
Furthermore, since , by Eq.~(\ref{eq:interval:minimum00}) we have .
However, recall that  is chosen as the vertex in  which has the maximum -value among all vertices in .
This contradicts the assumption that  is labeled .
\qed
\end{proof}

\subsubsection{Proof of Lemma~\ref{lemma:interval04}}
\begin{proof}
We construct a reconfiguration sequence from  to  such that
each intermediate dominating set is of cardinality at most .

Let .
For each  from  to , we focus on the vertices in , and transform  into  as follows:
\begin{listing}{aaa}
	\item[(1)] add the vertex  to  if ;
	\item[(2)] delete the vertices in  one by one; and
	\item[(3)] let  be the resulting vertex set.
\end{listing}
\smallskip
	
For each , let 
and .
We claim that  forms a dominating set of :
\begin{listing}{a}
	\item[-] Consider a vertex  such that .
				Since  holds,  is dominated by some vertex in .
	\item[-] Consider a vertex  such that .
				Since  holds,  is dominated by some vertex in .
	\item[-] Finally, consider a vertex  such that .
				Then,  and hence  is dominated by .
\end{listing}
Thus,  forms a dominating set of .
Since each vertex subset appearing in Steps~(1) and (2) above is a superset of , it also forms a dominating set of .

By the same arguments as in the proof of Lemma~\ref{lem:treereach}, we can conclude
that the reconfiguration sequence from  to  above consists
only of dominating sets of cardinality at most .
\qed
\end{proof}

\end{document}
