\RequirePackage{amsmath}
\documentclass{llncs}

\usepackage{multicol}

\usepackage{amsmath}
\usepackage{amsfonts}
\usepackage{amssymb}
\usepackage{verbatim}

\usepackage{gastex}
\usepackage{graphicx}
\usepackage{epic}
\usepackage{eepic}
\usepackage{epsfig,float}
\usepackage{pdfsync}

\usepackage{multicol}
\pagestyle{plain}
\DeclareGraphicsRule{.tif}{png}{.png}{`convert #1 `dirname #1`/`basename #1 .tif`.png}

\newcommand{\tid}{\mbox{{\bf 1}}}
\renewcommand{\le}{\leqslant}
\renewcommand{\ge}{\geqslant}

\newcommand{\rt}{\textcolor{red}{}}
\newcommand{\lt}{\textcolor{red}{}}

\newcommand{\co}{companion}
\newcommand{\ol}{\overline}
\newcommand{\eps}{\varepsilon}
\newcommand{\emp}{\emptyset}
\newcommand{\rhoR}{R}
\newcommand{\Sig}{\Sigma}
\newcommand{\sig}{\sigma}
\newcommand{\noin}{\noindent}
\newcommand{\pf}{prefix-focused}
\newcommand{\ur}{uniquely reachable}
\newcommand{\bi}{\begin{itemize}}
\newcommand{\ei}{\end{itemize}}
\newcommand{\be}{\begin{enumerate}}
\newcommand{\ee}{\end{enumerate}}
\newcommand{\bd}{\begin{description}}
\newcommand{\ed}{\end{description}}
\newcommand{\bq}{\begin{quote}}
\newcommand{\eq}{\end{quote}}
\newcommand{\txt}[1]{\mbox{ #1 }}

\newcommand{\etc}{\mbox{\it etc.}}
\newcommand{\ie}{\mbox{\it i.e.}}
\newcommand{\eg}{\mbox{\it e.g.}}

\newcommand{\inv}[1]{\mbox{}}

\newcommand{\stress}[1]{{\fontfamily{cmtt}\selectfont #1}}

\def\shu{\mathbin{\mathchoice
{\rule{.3pt}{1ex}\rule{.3em}{.3pt}\rule{.3pt}{1ex}
\rule{.3em}{.3pt}\rule{.3pt}{1ex}}
{\rule{.3pt}{1ex}\rule{.3em}{.3pt}\rule{.3pt}{1ex}
\rule{.3em}{.3pt}\rule{.3pt}{1ex}}
{\rule{.2pt}{.7ex}\rule{.2em}{.2pt}\rule{.2pt}{.7ex}
\rule{.2em}{.2pt}\rule{.2pt}{.7ex}}
{\rule{.3pt}{1ex}\rule{.3em}{.3pt}\rule{.3pt}{1ex}
\rule{.3em}{.3pt}\rule{.3pt}{1ex}}\mkern2mu}}

\newcommand{\cA}{{\mathcal A}}
\newcommand{\cB}{{\mathcal B}}
\newcommand{\cC}{{\mathcal C}}
\newcommand{\cD}{{\mathcal D}}
\newcommand{\cE}{{\mathcal E}}
\newcommand{\cI}{{\mathcal I}}
\newcommand{\cK}{{\mathcal K}}
\newcommand{\cL}{{\mathcal L}}
\newcommand{\cM}{{\mathcal M}}
\newcommand{\cN}{{\mathcal N}}
\newcommand{\cP}{{\mathcal P}}
\newcommand{\cQ}{{\mathcal Q}}
\newcommand{\cR}{{\mathcal R}}
\newcommand{\cS}{{\mathcal S}}
\newcommand{\cT}{{\mathcal T}}
\newcommand{\cU}{{\mathcal U}}
\newcommand{\cV}{{\mathcal V}}
\newcommand{\cW}{{\mathcal W}}

\newcommand{\one}{{\mathbf 1}}

\newcommand{\Lra}{{\hspace{.1cm}\Leftrightarrow\hspace{.1cm}}}
\newcommand{\lra}{{\hspace{.1cm}\leftrightarrow\hspace{.1cm}}}
\newcommand{\la}{{\hspace{.1cm}\leftarrow\hspace{.1cm}}}
\newcommand{\raL}{{\hspace{.1cm}{\rightarrow_L} \hspace{.1cm}}}
\newcommand{\lraL}{{\hspace{.1cm}{\leftrightarrow_L} \hspace{.1cm}}}

\newcommand{\sn}{{semiautomaton}}
\newcommand{\sa}{{semiautomata}}
\newcommand{\Sn}{{Semiautomaton}}
\newcommand{\Sa}{{Semiautomata}}
\newcommand{\se}{{settable}}
\newcommand{\Se}{{Settable}}
\newcommand{\pc}{{prefix-continuous}}

\newcommand{\cover}{{version}}
\newcommand{\defeq}{\stackrel{\rm def}{=}}
\newcommand{\com}{\mathbb{C}}
\newcommand{\rev}{R}
\newcommand{\deter}{D}
\newcommand{\mini}{\mathbb{M}}
\newcommand{\trim}{\mathbb{T}}

\newcommand{\timg}{\mathop{\mbox{rng}}}
\newcommand{\tdom}{\mathop{\mbox{dom}}}

\newcommand{\qedb}{\hfill} 

\newcommand{\gen}[1]{\langle #1 \rangle}

\newtheorem{open}[theorem]{Open problem}
\spnewtheorem{conj}{Conjecture}{\bfseries}{\rmfamily}

\newcommand{\atoms}{{\bf A}}
\newcommand{\atp}{{\phi(\atoms)}}
\newcommand{\quot}{{\bf K}}

\newcommand{\distlemma}{the Distinguishability Lemma}

\title{Quotient Complexities of Atoms in Regular Ideal Languages \thanks{This work was supported by the Natural Sciences and Engineering Research Council of Canada under grant No.~OGP0000871.}
}

\author{Janusz~Brzozowski\inst{1} \and Sylvie Davies\inst{2}}

\titlerunning{Quotient Complexities of Atoms in Regular Ideal Languages}

\authorrunning{J. Brzozowski and S. Davies}   

\institute{David R. Cheriton School of Computer Science, University of Waterloo \\
Waterloo, ON, Canada N2L 3G1\\
{\tt brzozo@uwaterloo.ca}
\and
Department of Pure Mathematics, University of Waterloo \\
Waterloo, ON, Canada N2L 3G1\\
{\tt sldavies@uwaterloo.ca}
}
\begin{document}

\maketitle
\begin{abstract}
A (left) quotient of a language  by a word  is the language .
The quotient complexity of a regular language  is the number of quotients of ; it is equal to the state complexity of , which is the number of states in a minimal deterministic finite automaton accepting .
An atom of   is an equivalence class of the relation in which two words are equivalent if for each quotient, they either are both in the quotient or both not in it; 
hence  it is a non-empty intersection of complemented and uncomplemented  quotients of .
A right (respectively, left and two-sided) ideal is a language  over an alphabet  that satisfies  (respectively,  and ).
We compute the maximal number of atoms and  the maximal quotient complexities of atoms of right, left and two-sided regular ideals.
\medskip


\noin
{\bf Keywords:}
atom,  quotient, regular language, left ideal, quotient complexity, right ideal, state complexity,  syntactic semigroup, two-sided ideal

\end{abstract}

\section{Introduction}

We assume that the reader is familiar with basic concepts of regular languages and finite automata; more background is given in the next section.
Consider a regular language  over a finite non-empty alphabet .
Let  be a minimal \emph{deterministic finite automaton (DFA)} recognizing , where  is the set of \emph{states},  is the \emph{transition function},  is the \emph{initial} state, and  is the set of \emph{final} states.
There are three natural equivalence relations associated with  and . 

The  \emph{Nerode right congruence}~\cite{Ner58} is defined as follows: Two words  and  are equivalent if for every ,  is in  if and only if  is in . 
The set of all words that ``can follow'' a given word  in  is the \emph{left quotient of  by }, defined by . 
In automaton-theoretic terms  is the set of all words  that are accepted from the state  reached when  is applied to the initial state of ; this is  known as the \emph{right language} of state~, the language accepted by  DFA . The Nerode equivalence class containing  is  known as the \emph{left language} of state~, the language accepted by DFA .
The number  of Nerode equivalence classes is the number of distinct left quotients of ,  known as its \emph{quotient complexity}~\cite{Brz10}. This is the same number as the number of states in , and is therefore known as 's \emph{state complexity}~\cite{Yu01}. 
Quotient/state complexity is now a commonly used measure of complexity of a regular language, and constitutes a basic reference for other measures of complexity. One can also define the quotient complexity of a Nerode equivalence class, that is, of the language accepted by  DFA . In the worst case -- for example, if  is strongly connected --  this is  for every .

The \emph{Myhill congruence}~\cite{Myh57} refines the Nerode right congruence and is a (two-sided) congruence. Here word  is equivalent to word  if for all  and  in ,  is in  if and only if  is in . This is also known as the \emph{syntactic congruence}~\cite{Pin97} of . The quotient set of  by this congruence is the \emph{syntactic semigroup} of . 
In automaton-theoretic terms two words are equivalent if they induce the same transformation of the set of states of a minimal DFA of .
The quotient complexity of Myhill classes has not been studied.

The third equivalence, which we call the \emph{atom congruence} is a left congruence refined by the Myhill congruence. Here two words  and  are equivalent if 
  if and only if   for all . 
 Thus  and  are equivalent if
  if and only if .
 An equivalence class of this relation is called an \emph{atom} of ~\cite{BrTa14}. 
It follows that an atom is a non-empty intersection of complemented and uncomplemented quotients of .

This congruence is related to the Myhill and Nerode congruences in a natural way. Say a congruence on  \emph{recognizes}  if  can be written as a union of the congruence's classes. The Myhill congruence is the unique \emph{coarsest} congruence (that is, the one with the fewest equivalence classes) that recognizes ~\cite{Pin97}. The Nerode and atom congruences are respectively the coarsest \emph{right} and \emph{left} congruences that recognize .

The quotient complexity of atoms of regular languages has been studied in~\cite{BrDa14,BrTa13,Iva14}.
In this paper we study the quotient complexity of atoms in three subclasses of regular languages, namely, right, left, and two-sided ideals.

Ideals are fundamental concepts in semigroup theory. A language  over an alphabet  is a 
\emph{right} (respectively, \emph{left} and \emph{two-sided}) \emph{ideal} if  (respectively,  and ).
The quotient complexity of regular ideal languages has been studied in~\cite{BJL13}, and the reader should refer to that paper for more information about ideals.
Ideals appear in  pattern matching.  A right (left) ideal  () represents the set of all words beginning (ending) with some word of a  given set , and  is the set of all words containing a factor from .


\section{Preliminaries}

It is well known that a language  is regular if and only if it has a finite number of quotients. We denote the number of quotients of   (the \emph{quotient complexity}) by . This is the same as the \emph{state complexity}, the number of states in a minimal DFA of . Since we will not be discussing other measures of complexity, we refer to both quotient and state complexity as just \emph{complexity}.

Let the set of quotients of a regular language  be .
The \emph{quotient automaton} of  is the DFA , where 
 if ,  by convention, and . This DFA is uniquely defined by  and is isomorphic to every minimal DFA of . 


A \emph{transformation} of a set  of  elements is a mapping of  \emph{into} itself, whereas a \emph{permutation}
of  is a mapping of  \emph{onto} itself.
In this paper we consider only transformations of finite sets, and we assume
without loss of generality  that .
An arbitrary transformation has the form

where  for .
The image of element  under transformation  is denoted by .
The image of  is .
The \emph{identity} transformation  maps each element to itself.
For , a transformation (permutation)  is a \emph{-cycle} if there is a set  such that if , and  for all .
A -cycle is denoted by .
A~2-cycle  is called a \emph{transposition}.
A transformation is \emph{constant} if it maps all states to a single state ; we denote it by .
A  transformation  is \emph{unitary}  if ,  and  for all ; we denote it by . 
The following is well-known:

\begin{proposition}
\label{prop:piccard}
The complete transformation monoid  of size  can be generated by any generators of the symmetric group  (the group of all permutations of ) together with a unitary transformation. In particular,  can be generated by , and by .
\end{proposition}

For a DFA  we define the transformations  by  for , and  for . This set is a semigroup under composition and it is called the \emph{transition semigroup} of . 
The transformation  is called the \emph{transformation induced by }. To simplify notation, we usually make no distinction between the word  and the transformation . 
If  is the quotient automaton of , then the transition semigroup of  is isomorphic to the syntactic semigroup of ~\cite{Pin97}.
A state  is \emph{reachable from } if  for some , and \emph{reachable} if it is reachable from .
Two states  are \emph{indistinguishable} if  for all , and \emph{distinguishable} otherwise.
Indistinguishability is an equivalence relation on ; furthermore, if  recognizes a language , we can compute  by counting the number of equivalence classes under  indistinguishability of the reachable states of .
A state is \emph{empty} if its right language (defined in the introduction) is .

\section{Atoms}
Atoms of regular languages were studied in~\cite{BrTa14}, and their complexities  in~\cite{BrDa14a,BrTa13}.
As discussed earlier, atoms are the classes of the \emph{atom congruence}, a left congruence which is the natural counterpart of the Myhill two-sided congruence and Nerode right congruence. 
The Myhill and Nerode congruences are fundamental in regular language theory, but it seems comparatively little attention has been paid to the atom congruence and its classes.
In~\cite{Brz13} it was argued that it is useful to consider the complexity of a language's atoms when searching for highly complex regular languages, since one would expect such languages to have highly complex atoms.

Below we present an alternative characterization of atoms, which we use in our proofs. Earlier papers on atoms such as~\cite{BrDa14a,BrTa13,BrTa14} take this as the definition of atoms, for it was not known until recently that atoms may be viewed as congruence classes (this fact was first noticed by Iv\'an in~\cite{Iva14}).

From now on assume all languages are non-empty.
Denote the complement of a language  by .
Let  and let  be a regular language with quotients . Each subset  of  defines an \emph{atomic intersection} , where .
An \emph{atom} of  is a non-empty atomic intersection. 
Since atoms are pairwise disjoint,  every atom  has a unique atomic intersection associated with it, and this atomic intersection has a unique subset  of  associated with it. This set  is called the \emph{basis} of .

Throughout the paper,  is a regular language of complexity  with quotients  and minimal DFA  such that the language of state  is . 
Let  be an atom. For any  we have 

Since a quotient of a quotient of  is also a quotient of ,  has the form;

where  and , .

The complexity of atoms of a regular language was computed in~\cite{BrTa13} using a unique NFA defined by , called the \emph{\'atomaton}. In that NFA the language of each state  is an atom  of . To find the complexity of that atom, the \'atomaton started in state  was converted to an equivalent DFA. 
A more direct and simpler method was used by Szabolcs Iv\'an~\cite{Iva14} who constructed the DFA for the atom directly from the DFA . We follow that approach here and outline it briefly for completeness. 

For any regular language  an atom  corresponds to the ordered pair ), where  () is the set of subscripts of uncomplemented (complemented) quotients. If  is represented by a DFA , it is more convenient to think of  and  as subsets of .
Similarly, any quotient of  corresponds to a pair  of subsets of . 
For the quotient of  reached when a letter  is applied to the quotient corresponding to  we get

In terms of pairs of subsets of , from  we reach . 
Note that if  in  then the corresponding quotient is empty.
Note also that the quotient of atom  corresponding to  is final if and only if
each quotient  with  contains , and each  with  does not contain .

These considerations lead to the following definition of a DFA for .
\begin{definition}
Suppose  is a DFA and let .
Define the DFA , where
\bi
\item
.
\item
For all ,  if , and  otherwise; and .
\item
. 
\ei
\end{definition}
DFA  recognizes the atomic intersection  of . If  recognizes a non-empty language, 
then  is an atom. 

\section{Complexity of Atoms in Regular Languages}

Upper bounds on the maximal complexity of atoms of regular languages were derived in~\cite{BrTa13}; for completeness we include these results.
For  there is only one non-empty language ; it has one atom, , which is of complexity~1. From now on assume that .

\begin{proposition}
\label{prop:reg}
Let  be a regular language with  quotients. Then  has at most  atoms.
If , then  quotients. Otherwise,

\end{proposition}
\begin{proof}
Since the number of subsets  of   is , there are at most that many atoms.
For atom complexity consider the following three cases:
\be
\item
.
Then   is the intersection of all quotients of . 
For ,  , where . 
Hence there are at most  quotients of this atom.
\item
. 
 Now , and
, where 
.
As in the first case, there are at most   quotients of this atom.
\item
. Then .
Every quotient of  has the form  where  and .
There are two subcases:
	\be
	\item
	If , then .
	\item
	If , there are at most 
	
	quotients of  of this form. This follows since  is the number of ways to choose a set  of size , and once  is fixed,  is the number of ways to choose a set  of size  that is \emph{disjoint} from . Taking the sum over the permissible values of  and  gives the formula above.
	\ee
Adding the results of (a) and (b)  we have the required bound. \qed
\ee
\end{proof}


 It was shown in~\cite{Brz13} that the language  accepted by the minimal DFA  of Definition~\ref{def:reg}, also illustrated  in Figure~\ref{fig:reg}, meets all the complexity bounds for common operations on regular languages.   
 \begin{definition}
\label{def:reg}
For , let , where 

is the set of states,
 is the alphabet, the transition function  is defined 
by
,
,
and ,   state 1 is the initial state, and  is the set of final states.
Let  be the language accepted by~.
(If ,  and  induce the same transformation; hence  suffices.)
\end{definition}


\begin{figure}[th]
\unitlength 7pt
\begin{center}\begin{picture}(37,10)(0,1)
\gasset{Nh=2.5,Nw=3.5,Nmr=1.25,ELdist=0.4,loopdiam=1.5}
\node(1)(1,7){1}\imark(1)
\node(2)(8,7){2}
\node(3)(15,7){3}
\node[Nframe=n](3dots)(22,7){}
	{\small
\node(n-1)(29,7){}
	}
	{\small
\node(n)(36,7){}\rmark(n)
	}
\drawloop(1){}
\drawedge[curvedepth= 1,ELdist=.1](1,2){}
\drawedge[curvedepth= 1,ELdist=-1.2](2,1){}
\drawloop(2){}
\drawedge(2,3){}
\drawloop(3){}
\drawedge(3,3dots){}
\drawedge(3dots,n-1){}
\drawloop(n-1){}
\drawedge(n-1,n){}
\drawedge[curvedepth= 4.5,ELdist=-1.3](n,1){}
\drawloop(n){}
\end{picture}\end{center}
\caption{ DFA  of a regular language whose atoms meet the bounds.}
\label{fig:reg}
\end{figure}


It was proved in \cite{BrTa13} that  has  atoms, all of which are as complex as possible. We include the proof of this theorem following~\cite{Iva14}. We first prove a general result about distinguishability of states in , which we will use throughout the paper.

\begin{lemma}[Distinguishability]
\label{lem:dist}
Let  be a minimal DFA and for , let  be the DFA of the atom . Then:
\be
\item
States  and  of  are distinguishable 
if  and  are both atoms, or if  and  are both atoms.
\item
If one of  or  is an atom, then  is distinguishable from .
\ee
\end{lemma}
\begin{proof}
First note that if  is an atom, then the initial state of  must be non-empty, so there is a word  such that  with , , i.e., . In particular, , since . We also have , since  is sent to a subset of , and  is sent to a subset of . This proves (2): if one of  or  is an atom, then one of  or  is in the transition semigroup of , and hence  can be mapped to a final state but  cannot.
Now, we consider the two cases from (1):
\be
\item
. Suppose . Then , but , since  is a non-empty subset of  and hence gets mapped outside of . Thus  distinguishes these states. If instead we have , then  distinguishes the states. Hence if  are atoms,  and  are in the transition semigroup of , and the states are distinguishable.
\item
. If , then  distinguishes  from ; otherwise,  distinguishes the states. As before, if  are atoms then the states are distinguishable. \qed
\ee
\end{proof}


\begin{theorem}
For , the language  of Definition~\ref{def:reg} has  atoms and each atom meets the bounds of Proposition~\ref{prop:reg}.
\end{theorem}
\begin{proof}
The DFA for the atomic intersection  is 
, where
 . 
 The transition semigroup of  consists of all  transformations of the state set .
Hence  can be mapped to a final state in  by taking a transformation that sends  to  and  to . 
It follows that all  atomic intersections ,  are atoms. By \distlemma, all distinct states in  are distinguishable. It suffices to prove the number of reachable states in each  meets the bounds.

 If ,
 then  is represented by , the reachable states of  are of the form 
, where  is the image of  under some transformation in the transition semigroup.
Since we have all transformations, we can reach all  states , . For  a similar argument works.

If , then for any state  with ,  and , we can find a transformation mapping  onto  and  onto .
So all these states are reachable, and there are  of them.
In addition,  is reachable from  by the constant transformation  and so  the bound is met.
\qed
\end{proof}


\section{Complexity of Atoms in Right Ideals}


If  is a right ideal, one of its quotients is ; by convention we assume that . In any atom  the quotient  must be uncomplemented, that is, we must have . Thus  is not an atom.
The results of this section were stated in~\cite{BrDa14} without proof; for completeness we include the proofs.
\begin{proposition}
\label{prop:bounds_right}
Suppose  is a right ideal with  quotients. Then  has at most  atoms.
The  complexity  of atom  satisfies


\end{proposition}
\begin{proof}
Let  be an atom.
Since  for all ,  always has  uncomplemented; so if  corresponds to , then .
Since the number of subsets  of  containing  is , there are at most that many atoms. Consider two cases:
\be
\item
.
Then  , and each such quotient of  is represented by , where .
Since  is always in , there are at most  quotients of this atom.
\item
. Then   where  and .
We know that if , then . 
Thus we are looking for pairs  such that  and .
To get  we take  and choose  elements from , and then to get  take  elements from .
The number of such pairs is
.
Adding the empty quotient we have our bound. \qed
\ee

\end{proof}

For ,  is a right ideal with one atom of complexity 1. 
For ,  is a right ideal with two atoms  and  of complexity 2.
It was shown in~\cite{BrDa14} that the language of the DFA of Definition~\ref{def:rideal} is most complex in the sense that it meets all the bounds for common operations, but no proof of atom complexity was given. We include this proof here.


\begin{definition}
\label{def:rideal}
For , let , where 
, 
and  is defined by
,
,

and .
Let  be the language accepted by~.
If ,  is not needed; hence  suffices.
Also, let  and . 
\end{definition}

\begin{figure}[ht]
\unitlength 7pt
\begin{center}\begin{picture}(42,10)(0,0)
\gasset{Nh=2.5,Nw=3.5,Nmr=1.25,ELdist=0.4,loopdiam=1.5}
\node(1)(1,7){1}\imark(1)
\node(2)(8,7){2}
\node(3)(15,7){3}
\node[Nframe=n](3dots)(22,7){}
	{\small
\node(n-2)(29,7){}
	}
	{\small
\node(n-1)(36,7){}
	}
	{\small
\node(n)(43,7){}\rmark(n)
	}
\drawloop(1){}
\drawedge(1,2){}
\drawloop(2){}
\drawedge(2,3){}
\drawloop(3){}
\drawedge(3,3dots){}
\drawedge(3dots,n-2){}
\drawedge(n-2,n-1){}
\drawedge(n-1,n){}
\drawedge[curvedepth= 3,ELdist=-1.5](n-1,2){}
\drawedge[curvedepth= 5,ELdist=-1.2](n-1,1){}
\drawloop(n-2){}
\drawloop(n){}
\end{picture}\end{center}
\caption{DFA  of  a right ideal  whose atoms meet the bounds.}
\label{fig:RightIdeal}
\end{figure}


\begin{theorem}
For , the language  of Definition~\ref{def:rideal} is a right ideal that has  atoms and each atom meets the bounds of Proposition~\ref{prop:bounds_right}.
\end{theorem}
\begin{proof}
The cases  are easily verified; hence assume .
By Proposition \ref{prop:piccard}, the transformations  restricted to  generate all transformations of . When  is included, we get all transformations of  that fix . 
For , , consider the DFA , which has initial state . There is a transformation of  fixing  that sends  to the final state . Hence  is an atom if , and so  has  atoms.

We now count reachable and distinguishable states in the DFA of each atom. Suppose . The initial state of  is ; by transformations that fix , we can reach any state  with . There are  such states, and since  is an atom if , all of them are distinguishable by \distlemma.

Suppose . From the initial state , by transformations that fix  we can reach any  with , ,  and . 
There are  such states. For all such states , we have  and , so  and  are both atoms; hence by \distlemma, all of these states are distinguishable from each other and from .
The state  is also reachable by the constant transformation , and so  the bound is met.
\qed
\end{proof}


\section{Complexity of Atoms in Left Ideals}
\label{sec:left}

If  is a left ideal, then , and  contains  for every . By convention we let . 



 
\label{sec:left_bounds}
\begin{proposition}
\label{prop:bounds_left}
Suppose  is a left ideal with  quotients. Then  has at most  atoms. The  complexity  of atom  satisfies
      

\end{proposition}
\begin{proof}
Consider the atomic intersections  such that ; then  (since every quotient contains ), and there are two possibilities:
Either , in which case  , or there is at least one quotient, say  which is complemented. Since  contains , it can be expressed as , where . Then the intersection has the term , and  is not an atom.
Thus for  to be an atom, either   or . Hence there are at most  atoms.

For atom complexity, consider the following cases:
\be
\item
.
Then  , and the  complexity of  is precisely . 
\item
. 
 Now  , and
every quotient of  is an intersection , where 
.
There are  such intersections, but
consider any quotient  of a left ideal; it can be expressed as , where . 
We have 

Thus every  intersection  which has  and does not have  as a term defines the same language as . 
There are  such intersections. Adding 1 for the intersection which just has the single term , we get our bound .

\item
. Then , where neither  nor  is empty.
If   this intersection  is empty, and so is not an atom.
Assume from now on that .
Every quotient of  has the form  where  and .

	\be
	\item
	. We claim that  for all . For suppose that there is a term 		, ,  and a word  such that .
	Since , we have .
	Since also  because  is a left ideal,  we have
	. 
	But  , so  has  as a term.
   Thus , which means .
	Hence  .
	\item
	.
	We are looking for pairs  such that .
As we argued in (2),  for each , so we can assume without loss of generality that .
	To get  we choose  elements from  and to get  	we take  and choose  elements from .
	The number of such pairs is
	

		\ee
		Adding 1 for the empty quotient we have our bound. \qed
\ee
\end{proof}

Next we compare the bounds for left ideals with those for right ideals. To calculate the number of pairs  such that  and  for right ideals, we can first choose  from  and then take  and choose  elements from . The number of such pairs is

If we interchange  and  we note that this is precisely the number of pairs  such that  and  for an atom of a left ideal with a basis of size .
Thus we have
\begin{remark}
\label{rem:symmetry}
Let  be a right ideal of complexity  and let  be an atom of , where .
Let  be a left ideal of complexity  and let  be an atom of . The upper bounds on the complexities of  and  are equal.
\end{remark}

Now we consider the question of tightness of the bounds in Proposition~\ref{prop:bounds_left}.
For ,  is a left ideal with one atom of complexity 1; so the bound of Proposition~\ref{prop:bounds_left} does not hold. 

The DFA of Definition~\ref{def:lideal} and Figure~\ref{fig:lideal} was introduced in~\cite{BrYe11}.
It was shown in~\cite{BrSz14} that the language of this DFA has the largest syntactic semigroup among left ideals of complexity .
Moreover, it was shown in~\cite{BrLiu15} that this language also meets the bounds on the quotient complexity of boolean operations, concatenation and star. Together with our result about the number of atoms and their complexity, this shows that this language is the most complex left ideal.

\begin{definition}
\label{def:lideal}
For , let , where 
, 
and  is defined by
,
,
,
,
and .
If , inputs  and  coincide; hence  suffices.
Also, let , where 
,
,
. 
Let  be the language accepted by~; we have .
\end{definition}


\begin{figure}[h]
\unitlength 7pt
\begin{center}\begin{picture}(43,17)(0,-1)
\gasset{Nh=2.5,Nw=3.5,Nmr=1.25,ELdist=0.4,loopdiam=1.5}
\node(1)(1,7){1}\imark(1)
\node(2)(8,7){2}
\node(3)(15,7){3}
\node(4)(22,7){4}
\node[Nframe=n](4dots)(29,7){}
	{\small
\node(n-1)(36,7){}
	}
	{\small
\node(n)(43,7){}\rmark(n)
	}
\drawedge(1,2){}
\drawloop(1){}
\drawloop(2){}
\drawedge[curvedepth= 1,ELdist=.1](2,3){}
\drawedge[curvedepth= 1,ELdist=.1](3,2){}
\drawloop[loopangle=270,ELpos=25](3){}
\drawedge(3,4){}
\drawedge[curvedepth= -3.5,ELdist=-1](4,2){}
\drawedge(4,4dots){}
\drawedge(4dots,n-1){}
\drawloop(4){}
\drawloop(n-1){}
\drawedge(n-1,n){}
\drawedge[curvedepth= 6.5,ELdist=-1.2](n-1,2){}
\drawedge[curvedepth= -8.5,ELdist=.5](n,2){}
\drawedge[curvedepth= 8.0,ELdist=.5](n,1){}
\drawloop(n){}
\end{picture}\end{center}
\caption{DFA  of a left ideal whose atoms meet the bounds.}
\label{fig:lideal}
\end{figure}

\begin{theorem}
For , the language  of Definition~\ref{def:lideal} is a left ideal that has  atoms and each atom meets the bounds of  Proposition~\ref{prop:bounds_left}.
\end{theorem}
\begin{proof}
It was proved in~\cite{BrYe11} that  is a left ideal of complexity .
The case  is easily verified; hence assume .
It was proved in~\cite{BrSz14} that the transition semigroup of  contains all transformations of  that fix 1 and all constant transformations.
Recall that if  is an atom of a left ideal, then either  or . 
For all  with , from  we can reach the final state  of  (or  for ) by transformations that fix 1.
For , let ; then  is final in .
Hence if  or , then  is an atom of , and so  has  atoms.

We now count reachable and distinguishable states in the DFA of each atom. We know that  has complexity  for all left ideals, so assume .
If , the initial state of  is . By transformations that fix 1 we can reach  for all  with . There are  of these states. Since  does not contain 1,  is an atom, so all of these states are distinguishable by \distlemma.

If , the initial state of  is .
Since , by transformations that fix 1, we can reach any state  with , , , , and .
There are  such states.
They are all distinguishable from each other and from  by \distlemma, since ,  imply that  and  are both atoms.
We can also reach  from  by any constant transformation, and so  the bound is met. \qed
\end{proof}

\begin{comment}
It was also proved in~\cite{BrDa14} that right ideals of complexity  have at most  atoms, and we proved earlier that a left ideal of complexity  has at most  atoms.
In fact, there is a complexity-preserving injection from the set of atoms of  (the most complex right ideal defined in~\cite{BrDa14}) to the set of atoms of our most complex left ideal .
The set of atoms of  is ; if we let  be the transposition , then  is a subset of  that does not contain 1, and thus the map  sends  to an atom  of  with . As we just argued, this means the map preserves quotient complexity of atoms. 
The only atom of  that is not in the range of this injection is , which has complexity  and does not correspond to any atom of .
\end{comment}

\section{Complexity of Atoms in Two-Sided Ideals}
\subsection{Upper Bounds}
A language  is a two-sided ideal if it is both a right  ideal  and a left ideal.


\label{sec:2sided_bounds}
\begin{proposition}
\label{prop:bounds_2sided}
Suppose  is a two-sided ideal with  quotients. Then  has at most  atoms. The  complexity  of atom  satisfies




\end{proposition}
\begin{proof}
Since  is a left ideal,  is an atom only if  or ; since  is a right ideal we must also have .
This gives our upper bound of  atoms.

We know that  has complexity  since this is true for left ideals.
Since  is a right ideal,  is not an atom, so we can assume .



Suppose  is an atom of , with  and .
We proved for left ideals that the number of distinct non-empty quotients of  is bounded by the number of pairs , , , , , .
Since  is a right ideal, we must also have  and . 
There are  possibilities for , since  must contain  and the remaining  elements are taken from .
If  is fixed, there are  possibilities for , since  must contain 1 and the remaining  elements are taken from . Since  always contains ,  the size of  is always .
Summing over the possible sizes of  and  and adding 1 for the empty quotient, we get the required bound.

This  leaves the case of .
Each quotient of  has the form

where , and . We can view the non-empty quotients as states  of the DFA  for , where  is a minimal DFA for .
We must have  and ,  and so . Hence , and there are  choices for .
However, for each  there are potentially  choices for , giving an upper bound of  for the non-empty quotients, which is not tight. We need to look more carefully at the distinguishability relations between states of .

For each  in , define the set . The elements of  are called the \emph{successors} of .
Note that  is not  a successor of itself.

Since  is a left ideal, we have  for all .
It follows that  for all .
Thus in the formula for  above, we have  for all .
But if  for any , then  is empty. 
Thus  for all , which implies .

 must contain , since  is a right ideal. Thus for each , there are at most  distinguishable states .
The index  can range from  to ; if  then  is non-empty.
This gives an upper bound of  for the number of non-empty quotients. 

This bound still is not tight, so we refine it as follows.
Choose  and a non-empty set . 
Then  for all , so we have
.
This means  is indistinguishable from .
Since  is non-empty and does not contain , there are at most  possibilities for .

From this we get a new upper bound for the number of distinguishable states  for a fixed , as follows: first take our previous bound of . Then for each ,   subtract  to account for the states  that are equivalent to . Our new bound is
 
Summing over all possible values of , and adding 1 for the empty quotient, we get the following bound on the complexity of :
 
Noting that  and , we pull out the  case from the outermost summation:
{\small
}
Observe that  is equal to , the bound we are trying to prove. We will show that the  value of the  rest of this formula is always less than or equal to zero. We pull 
 out to the front:
{\small
}
Note that  
, so cancellation occurs:

Now, the value of the innermost summation is always greater than or equal to : for each , , we  know  that  is a successor of , and hence  and . Thus the value of the outermost summation is always less than or equal to zero.
It follows that the number of quotients of  is at most .
\qed
\end{proof}

Next we address the question of tightness of the bounds for two-sided ideals.\
For ,  is a two-sided ideal with one atom of complexity 1; so the bound of Proposition~\ref{prop:bounds_2sided} does not hold. 

The DFA of  Definition~\ref{def:2sided} and Figure~\ref{fig:2sided} was introduced in~\cite{BrYe11}.
It was shown in~\cite{BrSz14} that the language of the DFA of  Definition~\ref{def:2sided} has the largest syntactic semigroup among left ideals of complexity .
 Moreover, it was shown in~\cite{BrLiu15} that this language also meets the bounds on the quotient complexity of boolean operations, concatenation and star. Together with our result about the number of atoms and their complexity, this shows that this language is the most complex two-sided ideal.

\begin{definition}
\label{def:2sided}
Let , and let  
 be the DFA with
 ,
,
,
,
,
,
and .
For ,  inputs  and  coincide.
Also, let , where
,
,
 ,
and 
let , where 
,
,
. 
Let  be the language accepted by .
\end{definition}

\begin{figure}[h]
\unitlength 7pt
\begin{center}\begin{picture}(43,17)(0,-1)
\gasset{Nh=2.5,Nw=4,Nmr=1.25,ELdist=0.4,loopdiam=1.5}
\node(n)(8,14){}\rmark(n)
\drawloop(n){}
\drawedge(2,n){}
\node(1)(1,7){1}\imark(1)
\node(2)(8,7){2}
\node(3)(15,7){3}
\node(4)(22,7){4}
\node[Nframe=n](4dots)(29,7){}
	{\small
\node(n-2)(36,7){}
	}
	{\small
\node(n-1)(43,7){}
	}
\drawedge(1,2){}
\drawloop(1){}
\drawloop[loopangle=270,ELdist=.2](2){}
\drawedge[curvedepth= 1,ELdist=.1](2,3){}
\drawedge[curvedepth= 1,ELdist=-1.2](3,2){}
\drawloop(3){}
\drawedge(3,4){}
\drawedge[curvedepth= 2.5,ELdist=-1](4,2){}
\drawedge(4,4dots){}
\drawedge(4dots,n-2){}
\drawloop(4){}
\drawloop(n-2){}
\drawedge(n-2,n-1){}
\drawedge[curvedepth= 5,ELdist=-1.2](n-2,2){}
\drawedge[curvedepth= -9.5,ELdist=-1.2](n-1,2){}
\drawedge[curvedepth= 9.5,ELdist=-1.5](n-1,1){}
\drawloop(n-1){}
\end{picture}\end{center}
\caption{ DFA of a  two-sided ideal whose atoms meet  the bounds.}
\label{fig:2sided}
\end{figure}

\begin{theorem}
For , the language  of Definition~\ref{def:2sided} is a two-sided ideal that has  atoms and each atom meets the bounds of Proposition~\ref{prop:bounds_2sided}.
\end{theorem}
\begin{proof}

It was proved in~\cite{BrYe11} that  is a two-sided ideal of complexity .
The cases with   are easily verified; hence assume .

The following observations were made in~\cite{BrSz14}:  
Transformations  restricted to  generate all the transformations of . Together with  and , they generate all transformations of  that fix  and .
Also, we have .


Recall that if  is an atom of a two-sided ideal, then , and either  or .
We know  is an atom of complexity  for all left ideals (and hence all two-sided ideals), so assume , .
Then ,  and so from state  in  we can reach the final state  by transformations that fix 1 and .
Hence  is an atom for every  with , . There are  of these atoms, as well as the atom , for a total of .

Consider the atom  for  and . 
In the DFA , the initial state is , and we have , .
By transformations that fix  and , we can reach  for all  such that , , , , .
There are  such states.
Since ,  and ,  we see that  and  are atoms. Hence by \distlemma, all of these states are distinguishable from each other and from .
Since , we can reach  from  by . Hence  the  bound is met.

It remains to show that the complexity of ,  also meets the bound.
The initial state of  is . By transformations that fix  and , we can reach all  states of the form  with . From , we can reach  additional states  for  by .
Finally, we can reach the sink state  from the initial state by . This gives a total of  reachable states, which matches the upper bound.

To see these states are distinguishable, note that  is an atom if . Also,  is an atom. Hence by \distlemma, all states of the form  are distinguishable from each other and from .
Also,  is distinguished from  by , which sends the former state to the non-final state , but sends the latter to some final state  with . And each ,  is a final state, so it is distinguishable from all states of the form ,  and from , since they are not final.
Hence all  reachable states are distinguishable.
\qed
\end{proof}

\section{Some Numerical Results}
The following tables compare the maximal complexities for atoms  of two-sided ideals (first entry), left ideals (second entry) and regular languages (third entry) with complexity . Right ideals are omitted because their complexities are essentially the same as those of left ideals, by Remark \ref{rem:symmetry}.
When the maximal complexity is undefined (e.g., because no languages in a class have atoms  for a particular size of ) this is indicated by an asterisk. The maximum values for each  are in boldface. The  entry in the \emph{ratio} row shows the approximate value of , where  is the  entry in the \emph{max} row.



{\small

}

\section{Conclusions}
We have derived tight upper bounds for the number of atoms and quotient complexity of atoms in right, left and two-sided regular ideal languages.
The recently discovered relationship between atoms and the Myhill and Nerode congruence classes opens up many interesting research questions. 
The quotient complexity of a language is equal to the number of Nerode classes, and 
the number of Myhill classes has also been used as a measure of complexity, called \emph{syntactic complexity} since it is equal to the size of the syntactic semigroup. 
We can view the number of atoms as a third fundamental measure of complexity for regular languages.

It is known~\cite{BrTa13} that the number of atoms of a regular language  is equal to the quotient complexity of the \emph{reversal} of . The quotient complexity of reversal has been studied for various classes of languages in the context of determining the quotient complexity of operations on regular languages. Hence, the maximal number of atoms is known for many language classes.

However, as far as we know the \emph{quotient complexity} of atoms has not been studied outside of regular languages and ideals.
For simplicity, let us call the atom congruence the \emph{left congruence}, the Nerode congruence the \emph{right congruence}, and the Myhill congruence the \emph{central congruence}. 
When computing the quotient complexity of atoms, we are computing the number of \emph{right congruence classes} of each \emph{left congruence class}. 
We can consider other permutations of this idea: how many right classes and left classes do the central classes have? How many central classes do the left classes have? 
These questions are outside the scope of this paper, but we believe they should be investigated.

\providecommand{\noopsort}[1]{}
\begin{thebibliography}{10}
\providecommand{\url}[1]{\texttt{#1}}
\providecommand{\urlprefix}{URL }

\bibitem{Brz10}
Brzozowski, J.: Quotient complexity of regular languages. J. Autom. Lang. Comb.
   15(1/2),  71--89 (2010)

\bibitem{Brz13}
Brzozowski, J.: In search of the most complex regular languages. Int. J. Found.
  Comput. Sci.,  24(6),  691--708 (2013)

\bibitem{BrDa14a}
Brzozowski, J., Davies, G.: Maximally atomic languages. In: \'Esik, Z.,
  F\"ul\"op, Z. (eds.) Proceedings of the 14th International Conference on
  Automata and Formal Languages AFL\/. Electronic Proceedings in
  Theoretical Computer Science, vol. 151, pp. 151--161 (2014)

\bibitem{BrDa14}
Brzozowski, J., Davies, G.: Most complex regular right ideals. In: J\"urgensen,
  H., et. al (eds.) Proceedings of the 16th International Workshop on
  Descriptional Complexity of Formal Systems DCFS\/. Lecture Notes in
  Computer Science, vol. 8614, pp. 90--101. Springer Berlin/Heidelberg (2014)

\bibitem{BJL13}
Brzozowski, J., Jir{\'a}skov{\'a}, G., Li, B.: Quotient complexity of ideal
  languages. Theoret. Comput. Sci.  470,  36--52 (2013)

\bibitem{BrLiu15}
Brzozowski, J., Liu, B.Y.V.: Most complex regular left and two-sided ideals
  (2015), in preparation

\bibitem{BrSz14}
Brzozowski, J., Szyku{\l}a, M.: Upper bounds on syntactic complexity of left
  and two-sided ideals. In: Shur, A.M., Volkov, M.V. (eds.) Proceedings of the
  18th International Conference on Developments in Language Theory DLT\/.
  Lecture Notes in Computer Science, vol. 8633, pp. 13--24. Springer,
  Berlin/Heidelberg (2014)

\bibitem{BrTa13}
Brzozowski, J., Tamm, H.: Quotient complexities of atoms of regular languages.
  Int. J. Found. Comput. Sci.  24(7),  1009--1027 (2013)

\bibitem{BrTa14}
Brzozowski, J., Tamm, H.: Theory of \'atomata. Theoret. Comput. Sci.  539,
  13--27 (2014)

\bibitem{BrYe11}
Brzozowski, J., Ye, Y.: Syntactic complexity of ideal and closed languages. In:
  Mauri, G., Leporati, A. (eds.) Proceedings of the 15th International
  Conference on Developments in Language Theory DLT\/. Lecture Notes in
  Computer Science, vol. 6795, pp. 117--128. Springer, Berlin/Heidelberg (2011)

\bibitem{Iva14}
Iv\'an, S.: Handle atoms with care, unpublished manuscript (2014)

\bibitem{Myh57}
Myhill, J.: Finite automata and representation of events. Wright Air
  Development Center Technical Report  57--624 (1957)

\bibitem{Ner58}
Nerode, A.: Linear automaton transformations. Proc. Amer. Math. Soc.  9,
  541--544 (1958)

\bibitem{Pin97}
Pin, J.E.: Syntactic semigroups. In: Handbook of Formal Languages, vol.~1:
  Word, Language, Grammar, pp. 679--746. Springer, New York, NY, USA (1997)

\bibitem{Yu01}
Yu, S.: State complexity of regular languages. J. Autom. Lang. Comb.  6,
  221--234 (2001)

\end{thebibliography}



\end{document}
