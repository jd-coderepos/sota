

\documentclass{article}
\usepackage[table]{xcolor}
\usepackage{amsthm}
\usepackage{amsmath}
\usepackage{float}
\usepackage{lipsum}
\usepackage{amsfonts}
\restylefloat{figure}


\usepackage[usenames,dvipsnames]{pstricks}
\usepackage{epsfig}
\usepackage{pst-grad} \usepackage{pst-plot} \usepackage{graphicx} \usepackage{subfigure} 

\usepackage{natbib}

\usepackage{algorithm}
\usepackage{algorithmic}

\usepackage{hyperref}

\newcommand{\theHalgorithm}{\arabic{algorithm}}

\usepackage[accepted]{icml2013} 



\icmltitlerunning{Empirical Evaluation of Rectified Activations in Convolutional Network}

\begin{document} 

\twocolumn[
\icmltitle{Empirical Evaluation of Rectified Activations in Convolution Network}

\icmlauthor{Bing Xu}{antinucleon@gmail.com}
\icmladdress{University of Alberta}
\icmlauthor{Naiyan Wang}{winsty@gmail.com}
\icmladdress{Hong Kong University of Science and Technology}
\icmlauthor{Tianqi Chen}{tqchen@cs.washington.edu
}
\icmladdress{University of Washington}
\icmlauthor{Mu Li}{muli@cs.cmu.edu}
\icmladdress{Carnegie Mellon University}
\icmlkeywords{Deep Learning}

\vskip 0.3in
]

\begin{abstract} 

In this paper we investigate the performance of different types of rectified activation functions in convolutional neural network: standard rectified linear unit (ReLU), leaky rectified linear unit (Leaky ReLU), parametric rectified linear unit (PReLU) and a new randomized leaky rectified linear units (RReLU). We evaluate these activation function on standard image classification task. Our experiments suggest that incorporating a non-zero slope for negative part in rectified activation units could consistently improve the results. Thus our findings are negative on the common belief that sparsity is the key of good performance in ReLU. Moreover, on small scale dataset, using deterministic negative slope or learning it are both prone to overfitting. They are not as effective as using their randomized counterpart. By using RReLU, we achieved 75.68\% accuracy on CIFAR-100 test set without multiple test or ensemble. 

\end{abstract} 

\section{Introduction}
Convolutional neural network (CNN) has made great success in various computer vision tasks, such as image classification \citep{krizhevsky2012imagenet, szegedy2014going}, object detection\citep{rcnn} and tracking\citep{sodlt}. Despite its depth, one of the key characteristics of modern deep learning system is to use non-saturated activation function (e.g. ReLU)  to replace its saturated counterpart (e.g. sigmoid, tanh). The advantage of using non-saturated activation function lies in two aspects: The first is to solve the so called ``exploding/vanishing gradient". The second is to accelerate the convergence speed.

In all of these non-saturated activation functions, the most notable one is \emph{rectified linear unit} (ReLU)~\citep{nair2010rectified, sun2014deeply}. Briefly speaking, it is a piecewise linear function which prunes the negative part to zero, and retains the positive part. It has a desirable property that the activations are sparse after passing ReLU. It is commonly believed that the superior performance of ReLU comes from the sparsity~\citep{glorot2011deep, sun2014deeply}. In this paper, we want to ask two questions: \emph{First, is sparsity the most important factor for a good performance? Second, can we design better non-saturated activation functions that could beat ReLU?}



We consider a broader class of activation functions, namely the rectified unit family. In particular, we are interested in the leaky ReLU and its variants. In contrast to ReLU, in which the negative part is totally dropped, leaky ReLU assigns a noon-zero slope to it. The first variant is called \emph{parametric rectified linear unit} (PReLU) \citep{he2015delving}. In PReLU, the slopes of negative part are learned form data rather than predefined. The authors claimed that PReLU is the key factor of  surpassing human-level performance on ImageNet classification~\citep{ILSVRC15} task. The second variant is called \emph{randomized rectified linear unit} (RReLU). In RReLU, the slopes of negative parts are randomized in a given range in the training, and then fixed in the testing. In a recent Kaggle National Data Science Bowl (NDSB) competition\footnote{Kaggle National Data Science Bowl Competition: \url{https://www.kaggle.com/c/datasciencebowl}}, it is reported that RReLU could reduce overfitting due to its randomized nature.

In this paper, we empirically evaluate these four kinds of activation functions. Based on our experiment, we conclude on small dataset, Leaky ReLU and its variants are consistently better than ReLU in convolutional neural networks. RReLU is favorable due to its randomness in training which reduces the risk of overfitting. While in case of large dataset, more investigation should be done in future.

\section{Rectified Units}
In this section, we introduce the four kinds of rectified units: rectified linear (ReLU), leaky rectified linear (Leaky ReLU), parametric rectified linear (PReLU) and randomized rectified linear (RReLU). We illustrate them in Fig.\ref{fig:act} for comparisons. In the sequel, we use   to denote the input of th channel in th example , and  to denote the corresponding output after passing the activation function. In the following subsections, we introduce each rectified unit formally.


\begin{figure}[!htb]
  \centering
    \includegraphics[width=0.5\textwidth]{act.pdf}
      \caption{ReLU, Leaky ReLU, PReLU and  RReLU. For PReLU,  is learned and for Leaky ReLU  is fixed. For RReLU,  is a random variable keeps sampling in a given range, and remains fixed in testing.}
    \label{fig:act}
\end{figure}

\subsection{Rectified Linear Unit}
Rectified Linear is first used in Restricted Boltzmann Machines\citep{nair2010rectified}. Formally, rectified linear activation is defined as:






\subsection{Leaky Rectified Linear Unit}
Leaky Rectified Linear activation is first introduced in acoustic model\citep{maas2013rectifier}. Mathematically, we have

where  is a fixed parameter in range . In original paper, the authors suggest to set  to a large number like 100. In additional to this setting, we also experiment smaller  in our paper.
 
\subsection{Parametric Rectified Linear Unit}
Parametric rectified linear is proposed by \citep{he2015delving}. The authors reported its performance is much better than ReLU in large scale image classification task. It is the same as leaky ReLU (Eqn.\ref{lrelu}) with the exception that  is learned in the training via back propagation.

\subsection{Randomized Leaky Rectified Linear Unit}
Randomized Leaky Rectified Linear is the randomized version of leaky ReLU. It is first proposed and used in Kaggle NDSB Competition. The highlight of RReLU is that in training process,  is a random number sampled from a uniform distribution . Formally, we have:

where

In the test phase, we take average of all the  in training as in the method of dropout~\citep{srivastava2014dropout} , and thus set  to  to get a deterministic result. Suggested by the NDSB competition winner,  is sampled from . We use the same configuration in this paper.

In test time, we use:



\section{Experiment Settings}
We evaluate classification performance on same convolutional network structure with different activation functions. Due to the large parameter searching space, we use two state-of-art convolutional network structure and same hyper parameters for different activation setting. All models are trained by using CXXNET\footnote{CXXNET: \url{https://github.com/dmlc/cxxnet}}.

\subsection{CIFAR-10 and CIFAR-100}
The CIFAR-10 and CIFAR-100 dataset \citep{krizhevsky2009learning} are tiny nature image dataset. CIFAR-10 datasets contains 10 different classes images and CIFAR-100 datasets contains 100 different classes. Each image is an RGB image in size 32x32. There are 50,000 training images and 10,000 test images. We use raw images directly without any pre-processing and augmentation. The result is from on single view test without any ensemble. 

The network structure is shown in Table \ref{tab:nin}. It is taken from Network in Network(NIN)\citep{lin2013network}.



\rowcolors{2}{white}{gray!25}
\begin{table}[!htb]
\centering
    \begin{tabular}{ll}
    Input Size 		& NIN                  \\
            & 5x5, 192             \\ 
           		& 1x1, 160             \\ 
            		& 1x1, 96              \\ 
            		& 3x3 max pooling, /2  \\ 
    
            		& dropout, 0.5         \\ 
             		& 5x5, 192             \\ 
             		& 1x1, 192             \\ 
            		& 1x1, 192             \\ 
             		& 3x3,avg pooling, /2  \\
             		& dropout, 0.5         \\ 
             		& 3x3, 192             \\ 
            		& 1x1, 192             \\ 
            		& 1x1, 10              \\
            		& 8x8, avg pooling, /1 \\ 
    10 or 100       		& softmax             
    \end{tabular}
    \caption{CIFAR-10/CIFAR-100 network structure. Each layer is a convolutional layer if not otherwise specified. Activation function is followed by each convolutional layer.}
    \label{tab:nin}
\end{table}

In CIFAR-100 experiment, we also tested RReLU on Batch Norm Inception Network \citep{ioffe2015batch}. We use a subset of Inception Network which is started from inception-3a module. This network achieved 75.68\% test accuracy without any ensemble or multiple view test \footnote{CIFAR-100 Reproduce code: \url{https://github.com/dmlc/mxnet/blob/master/example/notebooks/cifar-100.ipynb}}.

\subsection{National Data Science Bowl Competition}
The task for National Data Science Bowl competition is to classify plankton animals from image with award of \70 \times 7070 \times 7070 \times 7035 \times 3535 \times 3535 \times 3535 \times 3517 \times 1717 \times 1717 \times 1717 \times 1717 \times 1717 \times 1717 \times 178 \times 88 \times 88 \times 88 \times 88 \times 88 \times 812544 \times 11024 \times 11024 \times 1a=100a = 5.5a=100a=5.5y_{ji} = x_{ji} /\frac{l+u}{2}a=100a=5.5y_{ji} = x_{ji} /\frac{l+u}{2}a=100a=5.5y_{ji} = x_{ji} /\frac{l+u}{2}$)      & 0.8090         & \textbf{0.7292}  
    \end{tabular}
    \end{small}
    \caption{Multi-classes Log-Loss of NDSB Network with different activation function}
    \label{tab:ndsb}
\end{table}



\begin{figure*}[!htb]
\centering
\minipage{0.25\textwidth}
  \includegraphics[width=\linewidth]{cifar10_relu_xelu100.png}

\endminipage\hfill
\minipage{0.25\textwidth}
  \includegraphics[width=\linewidth]{cifar10_relu_xelu.png}

\endminipage\hfill
\minipage{0.25\textwidth}\includegraphics[width=\linewidth]{cifar10_relu_prelu.png}
\endminipage
\minipage{0.25\textwidth}
  \includegraphics[width=\linewidth]{cifar10_relu_rrelu.png}

\endminipage\hfill
\caption{Convergence curves for training and test sets of different activations on CIFAR-10 Network in Network.}
\label{fig:cifar}
\end{figure*}


\begin{figure*}[!htb]
\centering
\minipage{0.25\textwidth}
  \includegraphics[width=\linewidth]{cifar100_relu_xelu100.png}

\endminipage\hfill
\minipage{0.25\textwidth}
  \includegraphics[width=\linewidth]{cifar100_relu_xelu.png}

\endminipage\hfill
\minipage{0.25\textwidth}\includegraphics[width=\linewidth]{cifar100_relu_prelu.png}
\endminipage
\minipage{0.25\textwidth}
  \includegraphics[width=\linewidth]{cifar100_relu_rrelu.png}

\endminipage\hfill
\caption{Convergence curves for training and test sets of different activations on CIFAR-100 Network in Network.}
\label{fig:cifar100}
\end{figure*}

\begin{figure*}[!htb]
\centering
\minipage{0.25\textwidth}
  \includegraphics[width=\linewidth]{ndsb_relu_xelu100.png}

\endminipage\hfill
\minipage{0.25\textwidth}
  \includegraphics[width=\linewidth]{ndsb_relu_xelu.png}

\endminipage\hfill
\minipage{0.25\textwidth}\includegraphics[width=\linewidth]{ndsb_relu_prelu.png}
\endminipage
\minipage{0.25\textwidth}
  \includegraphics[width=\linewidth]{ndsb_relu_rrelu.png}

\endminipage\hfill
\caption{Convergence curves for training and test sets of different activations on NDSB Net.}
\label{fig:ndsb}
\end{figure*}

\section{Conclusion}
In this paper, we analyzed four rectified activation functions using various network architectures on three datasets. Our findings strongly suggest that the most popular activation function ReLU is not the end of story: Three types of (modified) leaky ReLU all consistently outperform the original ReLU. However, the reasons of their superior performances still lack rigorous justification from theoretic aspect. Also, how the activations perform on large scale data is still need to be investigated. This is an open question worth pursuing in the future. 

\section*{Acknowledgement}
We would like to thank Jason Rolfe from D-Wave system for helpful discussion on test network for randomized leaky ReLU.
\bibliography{example_paper}
\bibliographystyle{icml2013}

\end{document} 
