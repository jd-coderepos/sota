
\documentclass{article} \usepackage{iclr2021_conference,times}



\usepackage{amsmath,amsfonts,bm}

\newcommand{\figleft}{{\em (Left)}}
\newcommand{\figcenter}{{\em (Center)}}
\newcommand{\figright}{{\em (Right)}}
\newcommand{\figtop}{{\em (Top)}}
\newcommand{\figbottom}{{\em (Bottom)}}
\newcommand{\captiona}{{\em (a)}}
\newcommand{\captionb}{{\em (b)}}
\newcommand{\captionc}{{\em (c)}}
\newcommand{\captiond}{{\em (d)}}

\newcommand{\newterm}[1]{{\bf #1}}


\def\figref#1{figure~\ref{#1}}
\def\Figref#1{Figure~\ref{#1}}
\def\twofigref#1#2{figures \ref{#1} and \ref{#2}}
\def\quadfigref#1#2#3#4{figures \ref{#1}, \ref{#2}, \ref{#3} and \ref{#4}}
\def\secref#1{section~\ref{#1}}
\def\Secref#1{Section~\ref{#1}}
\def\twosecrefs#1#2{sections \ref{#1} and \ref{#2}}
\def\secrefs#1#2#3{sections \ref{#1}, \ref{#2} and \ref{#3}}
\def\eqref#1{equation~\ref{#1}}
\def\Eqref#1{Equation~\ref{#1}}
\def\plaineqref#1{\ref{#1}}
\def\chapref#1{chapter~\ref{#1}}
\def\Chapref#1{Chapter~\ref{#1}}
\def\rangechapref#1#2{chapters\ref{#1}--\ref{#2}}
\def\algref#1{algorithm~\ref{#1}}
\def\Algref#1{Algorithm~\ref{#1}}
\def\twoalgref#1#2{algorithms \ref{#1} and \ref{#2}}
\def\Twoalgref#1#2{Algorithms \ref{#1} and \ref{#2}}
\def\partref#1{part~\ref{#1}}
\def\Partref#1{Part~\ref{#1}}
\def\twopartref#1#2{parts \ref{#1} and \ref{#2}}

\def\ceil#1{\lceil #1 \rceil}
\def\floor#1{\lfloor #1 \rfloor}
\def\1{\bm{1}}
\newcommand{\train}{\mathcal{D}}
\newcommand{\valid}{\mathcal{D_{\mathrm{valid}}}}
\newcommand{\test}{\mathcal{D_{\mathrm{test}}}}

\def\eps{{\epsilon}}


\def\reta{{\textnormal{}}}
\def\ra{{\textnormal{a}}}
\def\rb{{\textnormal{b}}}
\def\rc{{\textnormal{c}}}
\def\rd{{\textnormal{d}}}
\def\re{{\textnormal{e}}}
\def\rf{{\textnormal{f}}}
\def\rg{{\textnormal{g}}}
\def\rh{{\textnormal{h}}}
\def\ri{{\textnormal{i}}}
\def\rj{{\textnormal{j}}}
\def\rk{{\textnormal{k}}}
\def\rl{{\textnormal{l}}}
\def\rn{{\textnormal{n}}}
\def\ro{{\textnormal{o}}}
\def\rp{{\textnormal{p}}}
\def\rq{{\textnormal{q}}}
\def\rr{{\textnormal{r}}}
\def\rs{{\textnormal{s}}}
\def\rt{{\textnormal{t}}}
\def\ru{{\textnormal{u}}}
\def\rv{{\textnormal{v}}}
\def\rw{{\textnormal{w}}}
\def\rx{{\textnormal{x}}}
\def\ry{{\textnormal{y}}}
\def\rz{{\textnormal{z}}}

\def\rvepsilon{{\mathbf{\epsilon}}}
\def\rvtheta{{\mathbf{\theta}}}
\def\rva{{\mathbf{a}}}
\def\rvb{{\mathbf{b}}}
\def\rvc{{\mathbf{c}}}
\def\rvd{{\mathbf{d}}}
\def\rve{{\mathbf{e}}}
\def\rvf{{\mathbf{f}}}
\def\rvg{{\mathbf{g}}}
\def\rvh{{\mathbf{h}}}
\def\rvu{{\mathbf{i}}}
\def\rvj{{\mathbf{j}}}
\def\rvk{{\mathbf{k}}}
\def\rvl{{\mathbf{l}}}
\def\rvm{{\mathbf{m}}}
\def\rvn{{\mathbf{n}}}
\def\rvo{{\mathbf{o}}}
\def\rvp{{\mathbf{p}}}
\def\rvq{{\mathbf{q}}}
\def\rvr{{\mathbf{r}}}
\def\rvs{{\mathbf{s}}}
\def\rvt{{\mathbf{t}}}
\def\rvu{{\mathbf{u}}}
\def\rvv{{\mathbf{v}}}
\def\rvw{{\mathbf{w}}}
\def\rvx{{\mathbf{x}}}
\def\rvy{{\mathbf{y}}}
\def\rvz{{\mathbf{z}}}

\def\erva{{\textnormal{a}}}
\def\ervb{{\textnormal{b}}}
\def\ervc{{\textnormal{c}}}
\def\ervd{{\textnormal{d}}}
\def\erve{{\textnormal{e}}}
\def\ervf{{\textnormal{f}}}
\def\ervg{{\textnormal{g}}}
\def\ervh{{\textnormal{h}}}
\def\ervi{{\textnormal{i}}}
\def\ervj{{\textnormal{j}}}
\def\ervk{{\textnormal{k}}}
\def\ervl{{\textnormal{l}}}
\def\ervm{{\textnormal{m}}}
\def\ervn{{\textnormal{n}}}
\def\ervo{{\textnormal{o}}}
\def\ervp{{\textnormal{p}}}
\def\ervq{{\textnormal{q}}}
\def\ervr{{\textnormal{r}}}
\def\ervs{{\textnormal{s}}}
\def\ervt{{\textnormal{t}}}
\def\ervu{{\textnormal{u}}}
\def\ervv{{\textnormal{v}}}
\def\ervw{{\textnormal{w}}}
\def\ervx{{\textnormal{x}}}
\def\ervy{{\textnormal{y}}}
\def\ervz{{\textnormal{z}}}

\def\rmA{{\mathbf{A}}}
\def\rmB{{\mathbf{B}}}
\def\rmC{{\mathbf{C}}}
\def\rmD{{\mathbf{D}}}
\def\rmE{{\mathbf{E}}}
\def\rmF{{\mathbf{F}}}
\def\rmG{{\mathbf{G}}}
\def\rmH{{\mathbf{H}}}
\def\rmI{{\mathbf{I}}}
\def\rmJ{{\mathbf{J}}}
\def\rmK{{\mathbf{K}}}
\def\rmL{{\mathbf{L}}}
\def\rmM{{\mathbf{M}}}
\def\rmN{{\mathbf{N}}}
\def\rmO{{\mathbf{O}}}
\def\rmP{{\mathbf{P}}}
\def\rmQ{{\mathbf{Q}}}
\def\rmR{{\mathbf{R}}}
\def\rmS{{\mathbf{S}}}
\def\rmT{{\mathbf{T}}}
\def\rmU{{\mathbf{U}}}
\def\rmV{{\mathbf{V}}}
\def\rmW{{\mathbf{W}}}
\def\rmX{{\mathbf{X}}}
\def\rmY{{\mathbf{Y}}}
\def\rmZ{{\mathbf{Z}}}

\def\ermA{{\textnormal{A}}}
\def\ermB{{\textnormal{B}}}
\def\ermC{{\textnormal{C}}}
\def\ermD{{\textnormal{D}}}
\def\ermE{{\textnormal{E}}}
\def\ermF{{\textnormal{F}}}
\def\ermG{{\textnormal{G}}}
\def\ermH{{\textnormal{H}}}
\def\ermI{{\textnormal{I}}}
\def\ermJ{{\textnormal{J}}}
\def\ermK{{\textnormal{K}}}
\def\ermL{{\textnormal{L}}}
\def\ermM{{\textnormal{M}}}
\def\ermN{{\textnormal{N}}}
\def\ermO{{\textnormal{O}}}
\def\ermP{{\textnormal{P}}}
\def\ermQ{{\textnormal{Q}}}
\def\ermR{{\textnormal{R}}}
\def\ermS{{\textnormal{S}}}
\def\ermT{{\textnormal{T}}}
\def\ermU{{\textnormal{U}}}
\def\ermV{{\textnormal{V}}}
\def\ermW{{\textnormal{W}}}
\def\ermX{{\textnormal{X}}}
\def\ermY{{\textnormal{Y}}}
\def\ermZ{{\textnormal{Z}}}

\def\vzero{{\bm{0}}}
\def\vone{{\bm{1}}}
\def\vmu{{\bm{\mu}}}
\def\vtheta{{\bm{\theta}}}
\def\va{{\bm{a}}}
\def\vb{{\bm{b}}}
\def\vc{{\bm{c}}}
\def\vd{{\bm{d}}}
\def\ve{{\bm{e}}}
\def\vf{{\bm{f}}}
\def\vg{{\bm{g}}}
\def\vh{{\bm{h}}}
\def\vi{{\bm{i}}}
\def\vj{{\bm{j}}}
\def\vk{{\bm{k}}}
\def\vl{{\bm{l}}}
\def\vm{{\bm{m}}}
\def\vn{{\bm{n}}}
\def\vo{{\bm{o}}}
\def\vp{{\bm{p}}}
\def\vq{{\bm{q}}}
\def\vr{{\bm{r}}}
\def\vs{{\bm{s}}}
\def\vt{{\bm{t}}}
\def\vu{{\bm{u}}}
\def\vv{{\bm{v}}}
\def\vw{{\bm{w}}}
\def\vx{{\bm{x}}}
\def\vy{{\bm{y}}}
\def\vz{{\bm{z}}}

\def\evalpha{{\alpha}}
\def\evbeta{{\beta}}
\def\evepsilon{{\epsilon}}
\def\evlambda{{\lambda}}
\def\evomega{{\omega}}
\def\evmu{{\mu}}
\def\evpsi{{\psi}}
\def\evsigma{{\sigma}}
\def\evtheta{{\theta}}
\def\eva{{a}}
\def\evb{{b}}
\def\evc{{c}}
\def\evd{{d}}
\def\eve{{e}}
\def\evf{{f}}
\def\evg{{g}}
\def\evh{{h}}
\def\evi{{i}}
\def\evj{{j}}
\def\evk{{k}}
\def\evl{{l}}
\def\evm{{m}}
\def\evn{{n}}
\def\evo{{o}}
\def\evp{{p}}
\def\evq{{q}}
\def\evr{{r}}
\def\evs{{s}}
\def\evt{{t}}
\def\evu{{u}}
\def\evv{{v}}
\def\evw{{w}}
\def\evx{{x}}
\def\evy{{y}}
\def\evz{{z}}

\def\mA{{\bm{A}}}
\def\mB{{\bm{B}}}
\def\mC{{\bm{C}}}
\def\mD{{\bm{D}}}
\def\mE{{\bm{E}}}
\def\mF{{\bm{F}}}
\def\mG{{\bm{G}}}
\def\mH{{\bm{H}}}
\def\mI{{\bm{I}}}
\def\mJ{{\bm{J}}}
\def\mK{{\bm{K}}}
\def\mL{{\bm{L}}}
\def\mM{{\bm{M}}}
\def\mN{{\bm{N}}}
\def\mO{{\bm{O}}}
\def\mP{{\bm{P}}}
\def\mQ{{\bm{Q}}}
\def\mR{{\bm{R}}}
\def\mS{{\bm{S}}}
\def\mT{{\bm{T}}}
\def\mU{{\bm{U}}}
\def\mV{{\bm{V}}}
\def\mW{{\bm{W}}}
\def\mX{{\bm{X}}}
\def\mY{{\bm{Y}}}
\def\mZ{{\bm{Z}}}
\def\mBeta{{\bm{\beta}}}
\def\mPhi{{\bm{\Phi}}}
\def\mLambda{{\bm{\Lambda}}}
\def\mSigma{{\bm{\Sigma}}}

\DeclareMathAlphabet{\mathsfit}{\encodingdefault}{\sfdefault}{m}{sl}
\SetMathAlphabet{\mathsfit}{bold}{\encodingdefault}{\sfdefault}{bx}{n}
\newcommand{\tens}[1]{\bm{\mathsfit{#1}}}
\def\tA{{\tens{A}}}
\def\tB{{\tens{B}}}
\def\tC{{\tens{C}}}
\def\tD{{\tens{D}}}
\def\tE{{\tens{E}}}
\def\tF{{\tens{F}}}
\def\tG{{\tens{G}}}
\def\tH{{\tens{H}}}
\def\tI{{\tens{I}}}
\def\tJ{{\tens{J}}}
\def\tK{{\tens{K}}}
\def\tL{{\tens{L}}}
\def\tM{{\tens{M}}}
\def\tN{{\tens{N}}}
\def\tO{{\tens{O}}}
\def\tP{{\tens{P}}}
\def\tQ{{\tens{Q}}}
\def\tR{{\tens{R}}}
\def\tS{{\tens{S}}}
\def\tT{{\tens{T}}}
\def\tU{{\tens{U}}}
\def\tV{{\tens{V}}}
\def\tW{{\tens{W}}}
\def\tX{{\tens{X}}}
\def\tY{{\tens{Y}}}
\def\tZ{{\tens{Z}}}


\def\gA{{\mathcal{A}}}
\def\gB{{\mathcal{B}}}
\def\gC{{\mathcal{C}}}
\def\gD{{\mathcal{D}}}
\def\gE{{\mathcal{E}}}
\def\gF{{\mathcal{F}}}
\def\gG{{\mathcal{G}}}
\def\gH{{\mathcal{H}}}
\def\gI{{\mathcal{I}}}
\def\gJ{{\mathcal{J}}}
\def\gK{{\mathcal{K}}}
\def\gL{{\mathcal{L}}}
\def\gM{{\mathcal{M}}}
\def\gN{{\mathcal{N}}}
\def\gO{{\mathcal{O}}}
\def\gP{{\mathcal{P}}}
\def\gQ{{\mathcal{Q}}}
\def\gR{{\mathcal{R}}}
\def\gS{{\mathcal{S}}}
\def\gT{{\mathcal{T}}}
\def\gU{{\mathcal{U}}}
\def\gV{{\mathcal{V}}}
\def\gW{{\mathcal{W}}}
\def\gX{{\mathcal{X}}}
\def\gY{{\mathcal{Y}}}
\def\gZ{{\mathcal{Z}}}

\def\sA{{\mathbb{A}}}
\def\sB{{\mathbb{B}}}
\def\sC{{\mathbb{C}}}
\def\sD{{\mathbb{D}}}
\def\sF{{\mathbb{F}}}
\def\sG{{\mathbb{G}}}
\def\sH{{\mathbb{H}}}
\def\sI{{\mathbb{I}}}
\def\sJ{{\mathbb{J}}}
\def\sK{{\mathbb{K}}}
\def\sL{{\mathbb{L}}}
\def\sM{{\mathbb{M}}}
\def\sN{{\mathbb{N}}}
\def\sO{{\mathbb{O}}}
\def\sP{{\mathbb{P}}}
\def\sQ{{\mathbb{Q}}}
\def\sR{{\mathbb{R}}}
\def\sS{{\mathbb{S}}}
\def\sT{{\mathbb{T}}}
\def\sU{{\mathbb{U}}}
\def\sV{{\mathbb{V}}}
\def\sW{{\mathbb{W}}}
\def\sX{{\mathbb{X}}}
\def\sY{{\mathbb{Y}}}
\def\sZ{{\mathbb{Z}}}

\def\emLambda{{\Lambda}}
\def\emA{{A}}
\def\emB{{B}}
\def\emC{{C}}
\def\emD{{D}}
\def\emE{{E}}
\def\emF{{F}}
\def\emG{{G}}
\def\emH{{H}}
\def\emI{{I}}
\def\emJ{{J}}
\def\emK{{K}}
\def\emL{{L}}
\def\emM{{M}}
\def\emN{{N}}
\def\emO{{O}}
\def\emP{{P}}
\def\emQ{{Q}}
\def\emR{{R}}
\def\emS{{S}}
\def\emT{{T}}
\def\emU{{U}}
\def\emV{{V}}
\def\emW{{W}}
\def\emX{{X}}
\def\emY{{Y}}
\def\emZ{{Z}}
\def\emSigma{{\Sigma}}

\newcommand{\etens}[1]{\mathsfit{#1}}
\def\etLambda{{\etens{\Lambda}}}
\def\etA{{\etens{A}}}
\def\etB{{\etens{B}}}
\def\etC{{\etens{C}}}
\def\etD{{\etens{D}}}
\def\etE{{\etens{E}}}
\def\etF{{\etens{F}}}
\def\etG{{\etens{G}}}
\def\etH{{\etens{H}}}
\def\etI{{\etens{I}}}
\def\etJ{{\etens{J}}}
\def\etK{{\etens{K}}}
\def\etL{{\etens{L}}}
\def\etM{{\etens{M}}}
\def\etN{{\etens{N}}}
\def\etO{{\etens{O}}}
\def\etP{{\etens{P}}}
\def\etQ{{\etens{Q}}}
\def\etR{{\etens{R}}}
\def\etS{{\etens{S}}}
\def\etT{{\etens{T}}}
\def\etU{{\etens{U}}}
\def\etV{{\etens{V}}}
\def\etW{{\etens{W}}}
\def\etX{{\etens{X}}}
\def\etY{{\etens{Y}}}
\def\etZ{{\etens{Z}}}

\newcommand{\pdata}{p_{\rm{data}}}
\newcommand{\ptrain}{\hat{p}_{\rm{data}}}
\newcommand{\Ptrain}{\hat{P}_{\rm{data}}}
\newcommand{\pmodel}{p_{\rm{model}}}
\newcommand{\Pmodel}{P_{\rm{model}}}
\newcommand{\ptildemodel}{\tilde{p}_{\rm{model}}}
\newcommand{\pencode}{p_{\rm{encoder}}}
\newcommand{\pdecode}{p_{\rm{decoder}}}
\newcommand{\precons}{p_{\rm{reconstruct}}}

\newcommand{\laplace}{\mathrm{Laplace}} 

\newcommand{\E}{\mathbb{E}}
\newcommand{\Ls}{\mathcal{L}}
\newcommand{\R}{\mathbb{R}}
\newcommand{\emp}{\tilde{p}}
\newcommand{\lr}{\alpha}
\newcommand{\reg}{\lambda}
\newcommand{\rect}{\mathrm{rectifier}}
\newcommand{\softmax}{\mathrm{softmax}}
\newcommand{\sigmoid}{\sigma}
\newcommand{\softplus}{\zeta}
\newcommand{\KL}{D_{\mathrm{KL}}}
\newcommand{\Var}{\mathrm{Var}}
\newcommand{\standarderror}{\mathrm{SE}}
\newcommand{\Cov}{\mathrm{Cov}}
\newcommand{\normlzero}{L^0}
\newcommand{\normlone}{L^1}
\newcommand{\normltwo}{L^2}
\newcommand{\normlp}{L^p}
\newcommand{\normmax}{L^\infty}

\newcommand{\parents}{Pa} 

\DeclareMathOperator*{\argmax}{arg\,max}
\DeclareMathOperator*{\argmin}{arg\,min}

\DeclareMathOperator{\sign}{sign}
\DeclareMathOperator{\Tr}{Tr}
\let\ab\allowbreak
 
\usepackage{amsmath}
\usepackage{hyperref}
\usepackage{url}
\usepackage{booktabs}
\usepackage{tablefootnote}
\usepackage{multirow}
\usepackage{makecell}
\usepackage{graphicx}
\usepackage{subcaption}



\title{LayoutLMv2: Multi-modal Pre-training for Visually-Rich Document Understanding}





\author{Yang Xu,~ Yiheng Xu, Tengchao Lv, Lei Cui, Furu Wei, Guoxin Wang, Yijuan Lu, \\ \textbf{Dinei Florencio, Cha Zhang, Wanxiang Che, Min Zhang, Lidong Zhou} \\
Harbin Institute of Technology\\
Microsoft Research Asia\\
Microsoft Cloud\&AI Team\\
Soochow University\\
\texttt{\{yxu,car\}@ir.hit.edu.cn}\\
\texttt{\{v-yixu,v-telv,lecu,fuwei,lidongz\}@microsoft.com} \\
\texttt{\{guow,yijlu,dinei,chazhang\}@microsoft.com}\\
\texttt{minzhang@suda.edu.cn}
}


\newcommand{\fix}{\marginpar{FIX}}
\newcommand{\new}{\marginpar{NEW}}

\iclrfinalcopy \begin{document}


\maketitle

\begin{abstract}

Pre-training of text and layout has proved effective in a variety of visually-rich document understanding tasks due to its effective model architecture and the advantage of large-scale unlabeled scanned/digital-born documents. In this paper, we present \textbf{LayoutLMv2} by pre-training text, layout and image in a multi-modal framework, where new model architectures and pre-training tasks are leveraged. Specifically, LayoutLMv2 not only uses the existing masked visual-language modeling task but also the new text-image alignment and text-image matching tasks in the pre-training stage, where cross-modality interaction is better learned. Meanwhile, it also integrates a spatial-aware self-attention mechanism into the Transformer architecture, so that the model can fully understand the relative positional relationship among different text blocks.
Experiment results show that LayoutLMv2 outperforms strong baselines and achieves new state-of-the-art results on a wide variety of downstream visually-rich document understanding tasks, including FUNSD (0.7895~~0.8420), CORD (0.9493~~0.9601), SROIE (0.9524~~0.9781), Kleister-NDA (0.834~~0.852), RVL-CDIP (0.9443~~0.9564), and DocVQA (0.7295~~0.8672). The pre-trained LayoutLMv2 model is publicly available at \url{https://aka.ms/layoutlmv2}.

\end{abstract}

\section{Introduction}

Visually-rich Document Understanding (VrDU) aims to analyze scanned/digital-born business documents (images, PDFs, etc.) where structured information can be automatically extracted and organized for many business applications. Distinct from conventional information extraction tasks, the VrDU task not only relies on textual information, but also visual and layout information that is vital for visually-rich documents. For instance, the documents in Figure \ref{fig:1} include a variety of types such as digital forms, receipts, invoices and financial reports. Different types of documents indicate that the text fields of interest locate at different positions within the document, which is often determined by the style and format of each type as well as the document content. Therefore, to accurately recognize the text fields of interest, it is inevitable to take advantage of the cross-modality nature of visually-rich documents, where the textual, visual and layout information should be jointly modeled and learned end-to-end in a single framework.

The recent progress of VrDU lies primarily in two directions. The first direction is usually built on the shallow fusion between textual and visual/layout/style information~\citep{Yang_2017,liu-etal-2019-graph,ijcai2019-466,yu2020pick,majumder-etal-2020-representation,Wei_2020,zhang2020trie}. These approaches leverage the pre-trained NLP and CV models individually and combine the information from multiple modalities for supervised learning. Although good performance has been achieved, these models often need to be re-trained from scratch once the document type is changed. In addition, the domain knowledge of one document type cannot be easily transferred into another document type, thereby the local invariance in general document layout (e.g. key-value pairs in a left-right layout, tables in a grid layout, etc.) cannot be fully exploited. To this end, the second direction relies on the deep fusion among textual, visual and layout information from a great number of unlabeled documents in different domains, where pre-training techniques play an important role in learning the cross-modality interaction in an end-to-end fashion~\citep{Lockard_2020,10.1145/3394486.3403172}. In this way, the pre-trained models absorb cross-modal knowledge from different document types, where the local invariance among these layout and styles is preserved. Furthermore, when the model needs to be transferred into another domain with different document formats, only a few labeled samples would be sufficient to fine-tune the generic model in order to achieve state-of-the-art accuracy. Therefore, the proposed model in this paper follows the second direction, and we explore how to further improve the pre-training strategies for the VrDU task.


\begin{figure*}[t]
\centering
    \begin{subfigure}[b]{0.23\textwidth}
        \includegraphics[width=\textwidth]{form2.png}
        \caption{Form}
        \label{fig:1a}
    \end{subfigure}
    ~ \begin{subfigure}[b]{0.23\textwidth}
        \includegraphics[width=\textwidth]{receipt.jpg}
        \caption{Receipt}
        \label{fig:1b}
    \end{subfigure}
    ~ \begin{subfigure}[b]{0.23\textwidth}
        \includegraphics[width=\textwidth]{invoice.png}
        \caption{Invoice}
        \label{fig:1c}
    \end{subfigure}
    ~
    \begin{subfigure}[b]{0.23\textwidth}
        \includegraphics[width=\textwidth]{report.png}
        \caption{Report}
        \label{fig:1d}
    \end{subfigure}
    \caption{Visually-rich business documents with different layouts and formats}\label{fig:1}
\end{figure*}




In this paper, we present an improved version of LayoutLM~\citep{10.1145/3394486.3403172}, aka \textbf{LayoutLMv2}. LayoutLM is a simple but effective pre-training method of text and layout for the VrDU task. Distinct from previous text-based pre-trained models, LayoutLM uses 2-D position embeddings and image embeddings in addition to the conventional text embeddings. During the pre-training stage, two training objectives are used, which are 1) a masked visual-language model and 2) multi-label document classification. The model is pre-trained with a great number of unlabeled scanned document images from the IIT-CDIP dataset~\citep{10.1145/1148170.1148307}, and achieves very promising results on several downstream tasks.
Extending the existing research work, we propose new model architectures and pre-training objectives in the LayoutLMv2 model. Different from the vanilla LayoutLM model where image embeddings are combined in the fine-tuning stage, we integrate the image information in the pre-training stage in LayoutLMv2 by taking advantage of the Transformer architecture to learn the cross-modality interaction between visual and textual information.
In addition, inspired by the 1-D relative position representations~\citep{Shaw_2018, raffel2020exploring, bao2020unilmv2}, we propose the spatial-aware self-attention mechanism for the LayoutLMv2, which involves a 2-D relative position representation for token pairs. Different from the absolute 2-D position embeddings, the relative position embeddings explicitly provide a broader view for the contextual spatial modeling.
For the pre-training strategies, we use two new training objectives for the LayoutLMv2 in addition to the masked visual-language model.
The first is the proposed text-image alignment strategy, which covers text-lines in the image and makes predictions on the text-side to classify whether the token is covered or not on the image-side.
The second is the text-image matching strategy that is popular in previous vision-language pre-training models~\citep{tan2019lxmert,lu2019vilbert,su2020vlbert,chen2020uniter,Sun_2019_ICCV}, where some images in the text-image pairs are randomly replaced with another document image to make the model learn whether the image and OCR texts are correlated or not.
In this way, LayoutLMv2 is more capable of learning contextual textual and visual information and the cross-modal correlation in a single framework, which leads to better VrDU performance.
We select 6 publicly available benchmark datasets as the downstream tasks to evaluate the performance of the pre-trained LayoutLMv2 model, which are the FUNSD dataset~\citep{Jaume_2019} for form understanding, the CORD dataset~\citep{park2019cord} and the SROIE dataset~\citep{8977955} for receipt understanding, \iffalse \footnote{\scriptsize \url{https://rrc.cvc.uab.es/?ch=13}} \fi the Kleister-NDA dataset~\citep{graliski2020kleister} for long document understanding with complex layout, the RVL-CDIP dataset~\citep{harley2015icdar} for document image classification, as well as the DocVQA dataset~\citep{mathew2020docvqa} for visual question answering on document images. Experiment results show that the LayoutLMv2 model outperforms strong baselines including the vanilla LayoutLM and achieves new state-of-the-art results in these downstream VrDU tasks, which substantially benefits a great number of real-world document understanding tasks.

\begin{figure*}[t]
    \centering
    \includegraphics[width=0.95\textwidth]{layoutlm-v2.pdf}
    \caption{An illustration of the model architecture and pre-training strategies for LayoutLMv2}
    \label{fig:2.LayoutLMv2}
\end{figure*}

The contributions of this paper are summarized as follows:

\begin{itemize}

    \item We propose a multi-modal Transformer model to integrate the document text, layout and image information in the pre-training stage, which learns the cross-modal interaction end-to-end in a single framework.
    \item In addition to the masked visual-language model, we also add text-image matching and text-image alignment as the new pre-training strategies to enforce the alignment among different modalities. Meanwhile, a spatial-aware self-attention mechanism is also integrated into the Transformer architecture.
    \item LayoutLMv2 not only outperforms the baseline models on the conventional VrDU tasks, but also achieves new SOTA results on the VQA task for document images, which demonstrates the great potential for the multi-modal pre-training for VrDU. The pre-trained LayoutLMv2 model is publicly available at \url{https://aka.ms/layoutlmv2}.
\end{itemize}


\section{Approach}

The overall illustration of the proposed LayoutLMv2 is shown in Figure \ref{fig:2.LayoutLMv2}. In this section, we will introduce the model architecture and pre-training tasks of the LayoutLMv2.

\subsection{Model Architecture}

We build an enhanced Transformer architecture for the VrDU tasks, i.e.\ the multi-modal Transformer as the backbone of LayoutLMv2. The multi-modal Transformer accepts inputs of three modalities: text, image, and layout. The input of each modality is converted to an embedding sequence and fused by the encoder. The model establishes deep interactions within and between modalities by leveraging the powerful Transformer layers. The model details are introduced as follows, where some dropout and normalization layers are omitted.


\paragraph{Text Embedding}

We recognize text and serialize it in a reasonable reading order using off-the-shelf OCR tools and PDF parsers.
Following the common practice, we use WordPiece \citep{wu2016google} to tokenize the text sequence and assign each token to a certain segment . Then, we add a {\tt[CLS]} at the beginning of the token sequence and a {\tt[SEP]} at the end of each text segment. The length of the text sequence is limited to ensure that the length of the final sequence is not greater than the maximum sequence length . Extra {\tt[PAD]} tokens are appended after the last {\tt[SEP]} token to fill the gap if the token sequence is still shorter than  tokens.
In this way, we get the input token sequence like 
The final text embedding is the sum of three embeddings. Token embedding represents the token itself, 1D positional embedding represents the token index, and segment embedding is used to distinguish different text segments. Formally, we have the -th text embedding 

\paragraph{Visual Embedding}

We use ResNeXt-FPN~\citep{Xie2016AggregatedRT,Lin_2017_CVPR} architecture as the backbone of the visual encoder. Given a document page image , it is resized to  then fed into the visual backbone.
After that, the output feature map is average-pooled to a fixed size with the width being  and height being . Next, it is flattened into a visual embedding sequence of length .
A linear projection layer is then applied to each visual token embedding in order to unify the dimensions.
Since the CNN-based visual backbone cannot capture the positional information, we also add a 1D positional embedding to these image token embeddings. The 1D positional embedding is shared with the text embedding layer.
For the segment embedding, we attach all visual tokens to the visual segment {\tt[C]}.
The -th visual embedding can be represented as



\paragraph{Layout Embedding}

The layout embedding layer aims to embed the spatial layout information represented by token bounding boxes in which corner coordinates and box shapes are identified explicitly. Following the vanilla LayoutLM, we normalize and discretize all coordinates to integers in the range , and use two embedding layers to embed -axis features and -axis features separately. Given the normalized bounding box of the -th text/visual token , the layout embedding layer concatenates six bounding box features to construct a token-level layout embedding, aka the 2D positional embedding

Note that CNNs perform local transformation, thus the visual token embeddings can be mapped back to image regions one by one with neither overlap nor omission. In the view of the layout embedding layer, the visual tokens can be treated as some evenly divided grids, so their bounding box coordinates are easy to calculate. An empty bounding box  is attached to special tokens {\tt[CLS]}, {\tt[SEP]} and {\tt[PAD]}.

\paragraph{Multi-modal Encoder with Spatial-Aware Self-Attention Mechanism}
The encoder concatenates visual embeddings  and text embeddings  to a unified sequence  and fuses spatial information by adding the layout embeddings to get the first layer input .

Following the architecture of Transformer, we build our multi-modal encoder with a stack of multi-head self-attention layers followed by a feed-forward network. However, the original self-attention mechanism can only implicitly capture the relationship between the input tokens with the absolute position hints. In order to efficiently model local invariance in the document layout, it is necessary to insert relative position information explicitly. Therefore, we introduce the spatial-aware self-attention mechanism into the self-attention layers. The original self-attention mechanism captures the correlation between query  and key  by projecting the two vectors and calculating the attention score

We jointly model the semantic relative position and spatial relative position as bias terms and explicitly add them to the attention score. Let  ,  and  denote the learnable 1D and 2D relative position biases respectively. The biases are different among attention heads but shared in all encoder layers. Assuming  anchors the top left corner coordinates of the -th bounding box, we obtain the spatial-aware attention score

Finally, the output vectors are represented as the weighted average of all the projected value vectors with respect to normalized spatial-aware attention scores










\subsection{Pre-training}

We adopt three self-supervised tasks simultaneously during the pre-training stage, which are described as follows.

\paragraph{Masked Visual-Language Modeling}

Similar to the vanilla LayoutLM, we use the Masked Visual-Language Modeling (MVLM) to make the model learn better in the language side with the cross-modality clues. We randomly mask some text tokens and ask the model to recover the masked tokens. Meanwhile, the layout information remains unchanged, which means the model knows each masked token's location on the page. The output representations of masked tokens from the encoder are fed into a classifier over the whole vocabulary, driven by a cross-entropy loss.
To avoid visual clue leakage, we mask image regions corresponding to masked tokens on the raw page image input before feeding into the visual encoder.
MVLM helps the model capture nearby tokens features. For instance, a masked blank in a table surrounded by lots of numbers is more likely to be a number. Moreover, given the spatial position of a blank, the model is capable of using visual information around to help predict the token.

\paragraph{Text-Image Alignment}

In addition to the MVLM, we propose the Text-Image Alignment (TIA) as a fine-grained cross-modality alignment task. In the TIA task, some text tokens are randomly selected, and their image regions are covered on the document image. We call this operation covering to avoid confusion with the masking operation in MVLM. During the pre-training, a classification layer is built above the encoder outputs. This layer predicts a label for each text token depending on whether it is covered, i.e., {\tt[Covered]} or {\tt[Not Covered]}, and computes the binary cross-entropy loss.
Considering the input image's resolution is limited, the covering operation is performed at the line-level.
When MVLM and TIA are performed simultaneously, TIA losses of the tokens masked in MVLM are not taken into account. This prevents the model from learning the useless but straightforward correspondence from {\tt[MASK]} to {\tt[Covered]}.

\paragraph{Text-Image Matching}

Furthermore, a coarse-grained cross-modality alignment task, Text-Image Matching (TIM) is applied during the pre-training stage. We feed the output representation at {\tt[CLS]} into a classifier to predict whether the image and text are from the same document page. Regular inputs are positive samples. To construct a negative sample, an image is either replaced by a page image from another document or dropped. To prevent the model from cheating by finding task features, we perform the same masking and covering operations to images in negative samples. The TIA target labels are all set to {\tt[Covered]} in negative samples. We apply the binary cross-entropy loss in the optimization process.

\subsection{Fine-tuning}

LayoutLMv2 produces representations with fused cross-modality information, which benefits a variety of VrDU tasks. Its output sequence provides representations at the token-level. Specifically, the output at {\tt[CLS]} can be used as the global feature. For many downstream tasks, we only need to build a task specified head layer over the LayoutLMv2 outputs and fine-tune the whole model using an appropriate loss. In this way, LayoutLMv2 leads to much better VrDU performance by integrating the text, layout, and image information in a single multi-modal framework, which significantly improves the cross-modal correlation compared to the vanilla LayoutLM model.

\section{Experiments}

\subsection{Data}

In order to pre-train and evaluate LayoutLMv2 models, we select datasets in a wide range from the visually-rich document understanding area. Introduction to the dataset and task definitions along with the description of required data pre-processing are presented as follows.

\paragraph{Pre-training Dataset}

Following LayoutLM, we pre-train LayoutLMv2 on the IIT-CDIP Test Collection~\citep{10.1145/1148170.1148307}, which contains over 11 million scanned document pages. We extract text and corresponding word-level bounding boxes from document page images with the Microsoft Read API.\footnote{\scriptsize \url{https://docs.microsoft.com/en-us/azure/cognitive-services/computer-vision/concept-recognizing-text}}


\paragraph{FUNSD}

FUNSD~\citep{Jaume_2019} is a dataset for form understanding in noisy scanned documents. It contains 199 real, fully annotated, scanned forms where 9,707 semantic entities are annotated above 31,485 words. The 199 samples are split into 149 for training and 50 for testing. The official OCR annotation is directly used with the layout information. The FUNSD dataset is suitable for a variety of tasks, where we focus on semantic entity labeling in this paper. Specifically, the task is assigning to each word a semantic entity label from a set of four predefined categories: question, answer, header or other. The entity-level F1 score is used as the evaluation metric.

\paragraph{CORD}

We also evaluate our model on the receipt key information extraction dataset, i.e. the public available subset of CORD~\citep{park2019cord}. The dataset includes 800 receipts for the training set, 100 for the validation set and 100 for the test set. A photo and a list of OCR annotations are equipped for each receipt. An ROI that encompasses the area of receipt region is provided along with each photo because there can be irrelevant things in the background. We only use the ROI as input instead of the raw photo. The dataset defines 30 fields under 4 categories and the task aims to label each word to the right field. The evaluation metric is entity-level F1. We use the official OCR annotations.

\paragraph{SROIE}

The SROIE dataset (Task 3)~\citep{8977955} aims to extract information from scanned receipts. There are 626 samples for training and 347 samples for testing in the dataset. The task is to extract values from each receipt of up to four predefined keys: company, date, address or total. The evaluation metric is entity-level F1. We use the official OCR annotations and results on the test set are provided by the official evaluation site.

\paragraph{Kleister-NDA}

Kleister-NDA~\citep{graliski2020kleister} contains non-disclosure agreements collected from the EDGAR database, including 254 documents for training, 83 documents for validation, and 203 documents for testing. This task is defined to extract the values of four fixed keys. We get the entity-level F1 score from the official evaluation tools.\footnote{\scriptsize\url{https://gitlab.com/filipg/geval}} Words and bounding boxes are extracted from the raw PDF file. We use heuristics to locate entity spans because the normalized standard answers may not appear in the utterance.

\paragraph{RVL-CDIP}

RVL-CDIP~\citep{harley2015icdar} consists of 400,000 grayscale images, with 8:1:1 for the training set, validation set, and test set. A multi-class single-label classification task is defined on RVL-CDIP. The images are categorized into 16 classes, with 25,000 images per class. The evaluation metric is the overall classification accuracy. Text and layout information is extracted by Microsoft OCR.

\paragraph{DocVQA}

As a VQA dataset on the document understanding field, DocVQA~\citep{mathew2020docvqa} consists of 50,000 questions defined on over 12,000 pages from a variety of documents. Pages are split into the training set, validation set and test set with a ratio of about 8:1:1. The dataset is organized as a set of triples page image, questions, answers. Thus, we use Microsoft Read API to extract text and bounding boxes from images. Heuristics are used to find given answers in the extracted text. The task is evaluated using an edit distance based metric ANLS (aka average normalized Levenshtein similarity). Given that human performance is about 98\% ANLS on the test set, it is reasonable to assume that the found ground truth which reaches over 97\% ANLS on training and validation sets is good enough to train a model. Results on the test set are provided by the official evaluation site.

\subsection{Settings}
\label{sec:Experiments.Settings}

Following the typical pre-training and fine-tuning strategy, we update all parameters and train whole models end-to-end for all the settings.


\paragraph{Pre-training LayoutLMv2}

We train LayoutLMv2 models with two different parameter sizes. We set hidden size  in  and use a 12-layer 12-head Transformer encoder. While in the ,  and its encoder has 24 Transformer layers with 16 heads. Visual backbones in the two models use the same ResNeXt101-FPN architecture. The numbers of parameters are 200M and 426M approximately for  and , respectively.

The model is initialized from the existing pre-trained model checkpoints. For the encoder along with the text embedding layer, LayoutLMv2 uses the same architecture as UniLMv2~\citep{bao2020unilmv2}, thus it is initialized from UniLMv2. For the ResNeXt-FPN part in the visual embedding layer, the backbone of a Mask-RCNN~\citep{He_2017_ICCV} model trained on PubLayNet \citep{zhong2019publaynet} is leveraged.\footnote{\scriptsize ``MaskRCNN ResNeXt101\_32x8d FPN 3X" setting in \url{https://github.com/hpanwar08/detectron2}} The rest of the parameters in the model are randomly initialized. We pre-train LayoutLMv2 models using Adam optimizer \citep{kingma2017adam,loshchilov2018decoupled}, with the learning rate of , weight decay of ,and . The learning rate is linearly warmed up over the first  steps then linearly decayed.  is trained with a batch size of  for  epochs, and  is trained with a batch size of  for  epochs on the IIT-CDIP dataset.

During the pre-training, we sample pages from the IIT-CDIP dataset and select a random sliding window of the text sequence if the sample is too long. We set the maximum sequence length  and assign all text tokens to the segment {\tt[A]}. The output shape of the adaptive pooling layer is set to , so that it transforms the feature map into  image tokens. In MVLM, 15\% text tokens are masked among which 80\% are replaced by a special token {\tt[MASK]}, 10\% are replaced by a random token sampled from the whole vocabulary, and 10\% remains the same. In TIA, 15\% of the lines are covered. In TIM, 15\% images are replaced and 5\% are dropped.

\paragraph{Fine-tuning LayoutLMv2 for Visual Question Answering}

We treat the DocVQA as an extractive QA task and build a token-level classifier on top of the text part of LayoutLMv2 output representations. Question tokens, context tokens and visual tokens are assigned to segment {\tt[A]}, {\tt[B]} and {\tt[C]}, respectively. In the DocVQA paper, experiment results show that the BERT model fine-tuned on the SQuAD dataset~\citep{rajpurkar-etal-2016-squad} outperforms the original BERT model. Inspired by this, we add an extra setting, which is that we first fine-tune LayoutLMv2 on a Question Generation (QG) dataset followed by the DocVQA dataset. The QG dataset contains almost one million question-answer pairs generated by a generation model trained on the SQuAD dataset.

\paragraph{Fine-tuning LayoutLMv2 for Document Image Classification}

This task depends on high-level visual information, thereby we leverage the image features explicitly in the fine-tuning. We pool the visual embeddings into a global pre-encoder feature, and pool the visual part of LayoutLMv2 output representations into a global post-encoder feature. The pre and post-encoder features along with the {\tt[CLS]} output feature are concatenated and fed into the final classification layer.

\paragraph{Fine-tuning LayoutLMv2 for Sequence Labeling}

We formalize FUNSD, SROIE, CORD and Kleister-NDA as the sequence labeling tasks. To fine-tune LayoutLMv2 models on these tasks, we build a token-level classification layer above the text part of the output representations to predict the BIO tags for each entity field.



\paragraph{Baselines}

We select 3 baseline models in the experiments to compare LayoutLMv2 with the SOTA text-only pre-trained models as well as the vanilla LayoutLM model. Specifically, we compare LayoutLMv2 with BERT~\citep{devlin-etal-2019-bert}, UniLMv2~\citep{bao2020unilmv2} and LayoutLM~\citep{10.1145/3394486.3403172} for all the experiment settings. We use the publicly available PyTorch models for BERT~\citep{wolf-etal-2020-transformers} and LayoutLM,\footnote{\scriptsize \url{https://github.com/microsoft/unilm/tree/master/layoutlm}} and use our in-house implementation for the UniLMv2 models. For each baseline approach, experiments are conducted using both the  and  parameter settings.

\subsection{Results}

\paragraph{FUNSD}
Table~\ref{tab:funsd} shows the model accuracy on the FUNSD dataset which is evaluated using entity-level precision, recall and F1 score. For text-only models, the UniLMv2 models outperform the BERT models by a large margin in terms of the  and  settings. For text+layout models, the LayoutLM family brings significant performance improvement over the text-only baselines, especially the LayoutLMv2 models. The best performance is achieved by the , where an improvement of 3\% F1 point is observed compared to the current SOTA results. This illustrates that the multi-modal pre-training in LayoutLMv2 learns better from the interactions from different modalities, thereby leading to the new SOTA on the form understanding task.

\begin{table}[ht]
    \centering
    \small
    \begin{tabular}{lcccc}
    \toprule
       \multicolumn{1}{c}{\bf Model} & \bf Precision & \bf Recall & \bf F1 & \bf \#Parameters  \\\midrule
        & 0.5469 & 0.6710 & 0.6026 & 110M \\
        & 0.6349 & 0.6975 & 0.6648 & 125M  \\
         & 0.6113 & 0.7085 & 0.6563 & 340M \\
        & 0.6780 & 0.7391 & 0.7072 & 355M \\\midrule
       & 0.7597 & 0.8155 & 0.7866 & 113M \\
       & 0.7596 & 0.8219 & 0.7895 & 343M \\\midrule
       & 0.8029 & 0.8539 & 0.8276 & 200M \\
       & \bf 0.8324 & \bf 0.8519 & \bf 0.8420 & 426M \\\midrule
      \midrule
BROS ~\citep{anonymous2021bros} & 0.8056 & 0.8188 & 0.8121 & - \\
     \bottomrule
    \end{tabular}
    \caption{Model accuracy (entity-level Precision, Recall, F1) on the FUNSD dataset}
    \label{tab:funsd}
\end{table}

\paragraph{CORD}
Table~\ref{tab:cord} gives the entity-level precision, recall and F1 scores on the CORD dataset. The LayoutLM family significantly outperforms the text-only pre-trained models including BERT and UniLMv2, especially the LayoutLMv2 models. Compared to the baselines, the LayoutLMv2 models are also superior to the ``SPADE'' decoder method, as well as the ``BROS'' approach that is built on the ``SPADE'' decoder, which confirms the effectiveness of the pre-training for text, layout and image information.


\begin{table}[ht]
    \centering
    \small
    \begin{tabular}{lcccc}
    \toprule
     \multicolumn{1}{c}{\bf Model} & \bf Precision & \bf Recall & \bf F1 & \bf \#Parameters\\
     \midrule
      & 0.8833 & 0.9107 & 0.8968 & 110M\\
      & 0.8987 & 0.9198 & 0.9092 & 125M \\
      & 0.8886 & 0.9168 & 0.9025 & 340M\\
      & 0.9123 & 0.9289 & 0.9205 & 355M \\
     \midrule
      & 0.9437 & 0.9508 & 0.9472 & 113M\\
      & 0.9432 & 0.9554 & 0.9493 & 343M\\
     \midrule
      & 0.9453 & 0.9539 & 0.9495 & 200M \\
      & \bf 0.9565 & \bf 0.9637 & \bf 0.9601 & 426M \\
     \midrule\midrule
     SPADE~\citep{hwang2020spatial} & - & - & 0.9150 & - \\
     BROS~\citep{anonymous2021bros} & 0.9558 & 0.9514 & 0.9536 & - \\
     \bottomrule
    \end{tabular}
    \caption{Model accuracy (entity-level Precision, Recall, F1) on the CORD dataset}
    \label{tab:cord}
\end{table}




\paragraph{SROIE}
Table~\ref{tab:sroie} lists the entity-level precision, recall, and F1 score on Task 3 of the SROIE challenge. Compared to the text-only pre-trained language models, our LayoutLM family models have significant improvement by integrating cross-modal interactions. Moreover, with the same modal information, our LayoutLMv2 models also outperform existing multi-modal approaches~\citep{anonymous2021bros, yu2020pick, zhang2020trie}, which demonstrates the model effectiveness. Eventually, the  single model can even beat the top-1 submission on the SROIE leaderboard.

\begin{table}[t]
    \centering
    \small
    \begin{tabular}{lcccc}
    \toprule
      \multicolumn{1}{c}{\bf Model} & \bf Precision & \bf Recall & \bf F1 & \bf \#Parameters  \\\midrule
       & 0.9099 & 0.9099 & 0.9099 & 110M \\
       & 0.9459 & 0.9459 & 0.9459 & 125M  \\
        & 0.9200 & 0.9200 & 0.9200 & 340M \\
       & 0.9488 & 0.9488 & 0.9488 & 355M \\\midrule
      & 0.9438 & 0.9438 & 0.9438 & 113M \\
      & 0.9524 & 0.9524 & 0.9524 & 343M \\\midrule
      & 0.9625 & 0.9625 & 0.9625 & 200M \\
      & 0.9661 &  0.9661 & 0.9661 & 426M \\
      (Excluding OCR mismatch) & \bf 0.9904 & \bf 0.9661 & \bf 0.9781 & 426M \\
\midrule\midrule
     BROS~\citep{anonymous2021bros} & 0.9493 & 0.9603 & 0.9548 & - \\
     PICK~\citep{yu2020pick} & 0.9679 & 0.9546  & 0.9612 & -\\
     TRIE~\citep{zhang2020trie} & - & - & 0.9618 & -\\

     Top-1 on SROIE Leaderboard (Excluding OCR mismatch)\tablefootnote{\scriptsize Unpublished results, the leaderboard is available at  \url{https://rrc.cvc.uab.es/?ch=13&com=evaluation&task=3}} & 0.9889 & 0.9647  & 0.9767 & - \\
     \bottomrule
    \end{tabular}
    \caption{Model accuracy (entity-level Precision, Recall, F1) on the SROIE dataset (until 2020-12-24)}
    \label{tab:sroie}
\end{table}

\paragraph{Kleister-NDA}

Table~\ref{tab:kleister-nda} gives the entity-level F1 score of the Kleister-NDA dataset. As the labeled answers are normalized into a canonical form, we apply post-processing heuristics to convert the extracted date information into the ``YYYY-MM-DD'' format, and company names into the abbreviations such as ``LLC'' and ``Inc.''. We report the evaluation results on the validation set because the ground-truth labels and the submission website for the test set are not available right now. The experiment results have shown that the LayoutLMv2 models improve the text-only and vanilla LayoutLM models by a large margin for the lengthy NDA documents, which also demonstrates that LayoutLMv2 can handle the complex layout information much better than previous models.

\begin{table}[ht]
    \centering
    \small
\begin{tabular}{lcc}
    \toprule
     \multicolumn{1}{c}{\bf Model} & \bf F1 & \bf \#Parameters  \\\midrule
      & 0.779 & 110M \\
 & 0.795 & 125M  \\
      & 0.791 & 340M \\
      & 0.818 & 355M \\\midrule
      & 0.827 & 113M \\
      & 0.834 & 343M \\\midrule
      & 0.833 & 200M \\
      & \bf 0.852 & 426M \\\midrule\midrule
      in~\citep{graliski2020kleister} & 0.793 & 125M \\
     \bottomrule
    \end{tabular}
    \caption{Model accuracy (entity-level F1) on the validation set of the Kleister-NDA dataset using the official evaluation toolkit}
    \label{tab:kleister-nda}
\end{table}









\paragraph{RVL-CDIP}


Table~\ref{tab:rvlcdip} shows the classification accuracy on the RVL-CDIP dataset, including text-only pre-trained models, the LayoutLM family as well as several image-based baseline models. As shown in the table, both the text and image information is important to the document image classification task because document images are text-intensive and represented by a variety of layouts and formats. Therefore, we observed that the LayoutLM family outperforms those text-only or image-only models as it leverages the multi-modal information within the documents. Specifically, the  model significantly improves the classification accuracy by more than 1.2\% F1 point over the previous SOTA results, which achieves an accuracy of 95.64\%. This also verifies that the pre-trained LayoutLMv2 model not only benefits the information extraction tasks in document understanding but also the document image classification task through the effective model training across different modalities.



\begin{table}[t]
    \centering
    \small
    \begin{tabular}{lcc}
    \toprule
     \multicolumn{1}{c}{\bf Model} & \bf Accuracy   & \bf \#Parameters\\\midrule
       &  89.81\% & 110M\\
       &  90.06\% & 125M\\
        & 89.92\%    & 340M\\
       &  90.20\% & 355M\\\midrule
 (w/ image) & 94.42\% & 160M \\
      (w/ image) & 94.43\% & 390M \\
     \midrule
      & 95.25\% & 200M \\
      & \bf 95.64\% & 426M \\
     \midrule\midrule
      VGG-16~\citep{Afzal2017CuttingTE} & 90.97\% &- \\
     Single model~\citep{Das2018DocumentIC} & 91.11\% & -\\
     Ensemble~\citep{Das2018DocumentIC} & 92.21\% & -\\
     InceptionResNetV2\tablefootnote{\scriptsize \url{ https://medium.com/@jdegange85/benchmarking-modern-cnn-architectures-to-rvl-cdip-9dd0b7ec2955}}~\citep{Szegedy2016Inceptionv4IA} & 92.63\% & -\\
     LadderNet~\citep{ijcai2019-466} & 92.77\% & -\\
     Single model~\citep{Dauphinee2019ModularMA} &  93.03\% & -\\
     Ensemble~\citep{Dauphinee2019ModularMA} &  93.07\% & -\\
     \bottomrule
    \end{tabular}
    \caption{Classification accuracy on the RVL-CDIP dataset}
    \label{tab:rvlcdip}
\end{table}


\paragraph{DocVQA}
Table~\ref{tab:docvqa} lists the Average Normalized Levenshtein Similarity (ANLS) scores on the DocVQA dataset of text-only baselines, LayoutLM family models and the previous top-1 on the leaderboard. With multi-modal pre-training, LayoutLMv2 models outperform LayoutLM models and text-only baselines by a large margin when fine-tuned on the train set. By using all data (train + dev) as the fine-tuning dataset, the  single model outperforms the previous top-1 on the leaderboard which ensembles 30 models. Under the setting of fine-tuning  on a question generation dataset (QG) and the DocVQA dataset successively, the single model performance increases by more than 1.6\% ANLS and achieves the new SOTA.



\begin{table}[ht]
    \centering
    \small
    \begin{tabular}{llcc}
    \toprule
     \multicolumn{1}{c}{\bf Model} & \bf Fine-tuning set & \bf ANLS   & \bf \#Parameters\\
     \midrule
      & train & 0.6354 & 110M \\
      & train & 0.7134 & 125M \\
       & train & 0.6768 & 340M \\
      & train & 0.7709 &  355M \\
     \midrule
      & train & 0.6979 & 113M \\
      & train & 0.7259 & 343M \\
     \midrule
      & train & 0.7808 & 200M \\
      & train & 0.8348 & 426M \\
     \midrule
 & train + dev & 0.8529 & 426M \\

      + QG & train + dev & \bf 0.8672 & 426M \\
     \midrule\midrule
     Top-1 on DocVQA Leaderboard (30 models ensemble)\tablefootnote{\scriptsize Unpublished results, the leaderboard is available at \url{https://rrc.cvc.uab.es/?ch=17&com=evaluation&task=1}} & - & 0.8506 & - \\
     \bottomrule
    \end{tabular}
    \caption{Average Normalized Levenshtein Similarity (ANLS) score on the DocVQA dataset (until 2020-12-24), ``QG" denotes the data augmentation with the question generation dataset.}
    \label{tab:docvqa}
\end{table}







\subsection{Ablation Study}

To fully understand the underlying impact of different components, we conduct an ablation study to explore the effect of visual information, the pre-training tasks, spatial-aware self-attention mechanism, as well as different initialization models. Table~\ref{tab:ablation-docvqa-bert} shows model performance on the DocVQA validation set. Under all the settings, we pre-train the models using all IIT-CDIP data for one epoch. The hyper-parameters are the same as those used to pre-train  in Section~\ref{sec:Experiments.Settings}. ``LayoutLM" denotes the vanilla LayoutLM architecture in~\citep{10.1145/3394486.3403172}, which can be regarded as a LayoutLMv2 architecture without visual module and spatial-aware self-attention mechanism. ``X101-FPN" denotes the ResNeXt101-FPN visual backbone described in Section~\ref{sec:Experiments.Settings}. We first evaluate the effect of introducing visual information. By comparing \#1 and \#2a, we find that LayoutLMv2 pre-trained with only MVLM can leverage visual information effectively. Then, we compare the two cross-modality alignment pre-training tasks TIA and TIM. According to the four results in \#2, both tasks improve the model performance substantially, and the proposed TIA benefits the model more than the commonly used TIM. Using both tasks together is more effective than using either one alone. From the comparison result of \#2d and \#3, the spatial-aware self-attention mechanism can further improve the model accuracy. In the settings \#3 and \#4, we change the text-side initialization checkpoint from BERT to UniLMv2, and confirm that LayoutLMv2 benefits from the better initialization.






\begin{table}[t]
    \centering
    \small
    \begin{tabular}{cllccccc}
        \toprule
        \bf \# & \bf Model Architecture & \bf Initialization & \bf SASAM & \bf MVLM & \bf TIA & \bf TIM & \bf ANLS \\
        \midrule
        1 &  &  & & \checkmark & & & 0.6841 \\
        \midrule
        2a &  &  + X101-FPN & & \checkmark & & & 0.6915 \\
        2b &  &  + X101-FPN & & \checkmark & \checkmark & & 0.7061 \\
        2c &  &  + X101-FPN & & \checkmark & & \checkmark & 0.6955 \\
        2d &  &  + X101-FPN & & \checkmark & \checkmark & \checkmark & 0.7124 \\
        \midrule
        3 &  &  + X101-FPN & \checkmark & \checkmark & \checkmark & \checkmark & 0.7217 \\
        \midrule
        4 &  &  + X101-FPN & \checkmark & \checkmark & \checkmark & \checkmark & 0.7421 \\
        \bottomrule
    \end{tabular}
    \caption{Ablation study on the DocVQA dataset, where ANLS scores on the validation set are reported. ``SASAM" means the spatial-aware self-attention mechanism. ``MVLM", ``TIA" and ``TIM" are the three proposed pre-training tasks. All the models are trained using all IIT-CDIP data for 1 epoch with the  model size.}
    \label{tab:ablation-docvqa-bert}
\end{table}





\section{Related Work}











With the development of conventional machine learning, statistical machine learning approaches~\citep{shilman2005learning,1359749} have become the mainstream for document segmentation tasks during the past decade.~\cite{shilman2005learning} consider the layout information of a document as a parsing problem, and globally search the optimal parsing tree based on a grammar-based loss function. They utilize a machine learning approach to select features and train all parameters during the parsing process. Meanwhile, artificial neural networks~\citep{1359749} have been extensively applied to document analysis and recognition. Most efforts have been devoted to the recognition of isolated handwritten and printed characters with widely recognized successful results. In addition to the ANN model, SVM and GMM~\citep{6628808} have been used in document layout analysis tasks. For machine learning approaches, they are usually time-consuming to design manually crafted features and difficult to obtain a highly abstract semantic context. In addition, these methods usually relied on visual cues but ignored textual information.



Deep learning methods have become the mainstream and de facto standard for many machine learning problems. Theoretically, they can fit any arbitrary functions through the stacking of multi-layer neural networks and have been verified to be effective in many research areas.~\cite{Yang2017LearningTE} treat the document semantic structure extraction task as a pixel-by-pixel classification problem. They propose a multi-modal neural network that considers visual and textual information, while the limitation of this work is that they only used the network to assist heuristic algorithms to classify candidate bounding boxes rather than an end-to-end approach.~\cite{Viana2017FastCD} propose a lightweight model of document layout analysis for mobile and cloud services. The model uses one-dimensional information of images for inference and compares it with the model using two-dimensional information, achieving comparable accuracy in the experiments.~\cite{katti-etal-2018-chargrid} make use of a fully convolutional encoder-decoder network that predicts a segmentation mask and bounding boxes, and the model significantly outperforms approaches based on sequential text or document images.~\cite{soto-yoo-2019-visual} incorporate contextual information into the Faster R-CNN model that involves the inherently localized nature of article contents to improve region detection performance.



In recent years, pre-training techniques have become more and more popular in both NLP and CV areas, and have also been leveraged in the VrDU tasks.~\cite{devlin-etal-2019-bert} introduced a new language representation model called BERT, which is designed to pre-train deep bidirectional representations from the unlabeled text by jointly conditioning on both left and right context in all layers. As a result, the pre-trained BERT model can be fine-tuned with just one additional output layer to create state-of-the-art models for a wide range of tasks.~\cite{bao2020unilmv2} propose to pre-train a unified language model for both autoencoding and partially autoregressive language modeling tasks using a novel training procedure, referred to as a pseudo-masked language model. In addition, the two tasks pre-train a unified language model as a bidirectional encoder and a sequence-to-sequence decoder, respectively.~\cite{lu2019vilbert} proposed ViLBERT for learning task-agnostic joint representations of image content and natural language by extending the popular BERT architecture to a multi-modal two-stream model.~\cite{su2020vlbert} proposed VL-BERT that adopts the Transformer model as the backbone, and extends it to take both visual and linguistic embedded features as input.~\citep{10.1145/3394486.3403172} proposed the LayoutLM to jointly model interactions between text and layout information across scanned document images, which is beneficial for a great number of real-world document image understanding tasks such as information extraction from scanned documents. This work is a natural extension of the vanilla LayoutLM, which takes advantage of textual, layout and visual information in a single multi-modal pre-training framework.



\section{Conclusion}

In this paper, we present a multi-modal pre-training approach for visually-rich document understanding tasks, aka LayoutLMv2. Distinct from existing methods for VrDU, the LayoutLMv2 model not only considers the text and layout information but also integrates the image information in the pre-training stage with a single multi-modal framework. Meanwhile, the spatial-aware self-attention mechanism is integrated into the
Transformer architecture to capture the relative relationship among different bounding boxes. Furthermore, new pre-training objectives are also leveraged to enforce the learning of cross-modal interaction among different modalities. Experiment results on 6 different VrDU tasks have illustrated that the pre-trained LayoutLMv2 model has substantially outperformed the SOTA baselines in the document intelligence area, which greatly benefits a number of real-world document understanding tasks.

For future research, we will further explore the network architecture as well as the pre-training strategies for the LayoutLM family, so that we can push the SOTA results in VrDU to the new height. Meanwhile, we will also investigate the language expansion to make the multi-lingual LayoutLMv2 model available for different languages especially the non-English areas around the world.






\bibliography{iclr2021_conference}
\bibliographystyle{iclr2021_conference}



\end{document}
