




\section{Introduction}
\label{intro}

A common induced subgraph of two graphs  and  is a graph that is isomorphic to an induced subgraph of both graphs.
The problem of finding a common induced subgraph with the maximum number of vertices (or edges) has many applications in a number of domains including bioinformatics and chemistry \cite{Grindley1993,Koch1996,MW81,RW02,Yamaguchi2004}. 
In the decision version of the problem, we are given an integer  and the question is to decide whether there is a solution with at least  vertices.
We say that the solution size  is the \emph{natural parameter} of the 
problem. 

Concerning its classical complexity, \mcis is -complete, and remains so 
on forests. When the common subgraph is required to be connected, the problem 
is in  for trees \cite{Garey1979}.
Moreover, \mcis is also in  when the two input graphs are connected and 
(both) have bounded treewidth and bounded degree \cite{Akutsu1993}. 

A particular case of \mcis is the well known \isi (ISI) decision problem, where the question posed is whether  is isomorphic to an induced 
subgraph of .
In other words, it is equivalent to \mcis where .
In this case,  is called the pattern graph and  is called the host 
graph.
ISI is known to be -hard, even when  is an interval graph and  
is a proper interval graph, but it becomes polynomial-time solvable when  
is in addition connected~\cite{Heggernes}. 
Unlike \textsc{Subgraph Isomorphism}, \isi remains -hard when both
the host graph and the pattern graph consist of a disjoint union of 
paths \cite{Damaschke1991}. 
From the parameterized complexity viewpoint, the problem is -hard in 
general for the natural parameter, by a straightforward reduction 
from \textsc{-Clique}. Therefore MCIS is also -hard. 
Moreover, ISI (and, therefore, MCIS) remains -hard even when both 
graphs are interval graphs~\cite{MarxS13}. 
On the other hand, ISI is  on nowhere dense 
graphs, being expressible by a first-order formula of length
function of the natural parameter ~\cite{GroheKS14}.
This generalizes what was previously known about ISI on 
-minor free graphs~\cite{FG01} 
and graphs of bounded degree~\cite{Cai2006}. 
We observe that whenever ISI in  on a certain graph class, then so 
is MCIS. 
To see this, note that an arbitrary instance  of MCIS can be 
reduced in fpt-time to instances of ISI by enumerating each graph  on 
 vertices and checking whether  is an induced subgraph of  and .
This implies that ISI and MCIS have the same parameterized complexity with 
respect to the natural parameter.  

Another way of dealing with the hardness of a problem is to study its complexity with respect to auxiliary (or structural) parameters, to better  understand its algorithmic behavior (see for example~\cite{FellowsJR13}).
Being -hard on disjoint union of chordless paths~\cite{Damaschke1991}, MCIS is hard on graphs with bounded treewidth as well as graphs where the size of the minimum feedback vertex set is bounded. 
Thus the problem is paraNP-hard when parameterized by the treewidth of the input graphs, or by a bound on the sizes of their minimum feedback vertex sets.
Therefore, we need to look for ``bigger'' parameters.
And indeed, if the parameter  is a bound on the sizes of the minimum vertex 
covers of the input graphs, 
then the problem is in , with a running time of ~\cite{AbuKhzam2014}.
In this paper, we show that this algorithm cannot be significantly improved: 
unless the Exponential Time Hypothesis 
(ETH) fails, there is no algorithm solving MCIS in time , where the  notation suppresses the polynomial factors.
We also prove 
that MCIS does not have a polynomial-size kernel in this case unless . 
These two latter results answer open problems raised in \cite{AbuKhzam2014}.
Finally, we show that \mccis (MCCIS), where the solution should be a connected 
graph, is also fixed-parameter tractable when parameterized by .

\section{Preliminaries}
\label{prelims}

Two finite graphs  and  are \textit{isomorphic} 
if there 
is a bijection  such that . Given a graph , a graph  
 is an \textit{induced subgraph} of  if  and 
  , i.e.  is the edge set with both 
extremities in . We also say that  is the subgraph of  induced by . 

The \textit{girth} of a graph  is the length of the shortest cycle contained in . 
Contracting an edge  consists of deleting  and replacing the vertices  and  by a single vertex  in the incidence relation (edges incident on  or  become incident on ).
A graph  is a \textit{minor} of graph  if  is obtained from a subgraph of  by applying zero or more edge contractions. Given a fixed graph , a family  of graphs is said to be \textit{-minor free} if  is not a minor of any element of .


The \mcis problem is defined formally as follows. 

\PbOpt{\mcis (MCIS)}{Two graphs  and .}{An induced 
subgraph  of  isomorphic to an induced subgraph  of  with a 
maximum number of vertices.}





\mccis (MCCIS) is defined as MCIS with the additional restriction that the solution must be \emph{connected}. 

For completeness, we also give the definition of \isi:

\PbOpt{\isi (ISI)}{Two graphs  and .}{An induced 
subgraph  of  isomorphic to  if it exists.}

\icsi (ICSI) is defined as ISI but  must be connected. 

\paragraph{Parameterized complexity}
A parameterized problem  is \textit{fixed-parameter tractable} (or in 
the class ) with respect to parameter  if it can be solved in 
 time (i.e. in fpt-time), where 
 is any computable function and  is a constant (see 
\cite{Downey2013,Niedermeier2006} 
for more details about fixed-parameter tractability).
The parameterized complexity hierarchy is composed of the classes 
. 
The class  contains problems solvable in time , 
where  and  are unrestricted functions.
A problem is said to be \emph{paraNP-hard} if it is -hard even for a 
constant value of the parameter (it hence cannot be in ).
A -hard problem is not fixed-parameter tractable, unless , and one can prove -hardness by means of a \emph{parameterized reduction} from a -hard problem.
This is a mapping of an instance~ of a problem~ in  time (for any computable function~) into an instance  for~ such that  and  for some function~.

A powerful technique to design parameterized algorithms is \textit{kernelization}.
In short, kernelization is a polynomial-time self-reduction algorithm that takes an instance  of a parameterized problem  as input and computes an equivalent instance  of  such that  for some computable function  and .
The instance  is called a \textit{kernel} in this case.
If the function  is polynomial, we say that  is a polynomial kernel.
It is well known that a problem is in  iff it has a kernel, but this equivalence yields super-polynomial kernels (in general).
To design efficient parameterized algorithms, a kernel of polynomial (or even linear) size in  is important.
However, some lower bounds on the size of the kernel can be shown unless some polynomial hierarchy collapses.
To show this result, we will use the cross composition technique developed by Bodlaender et al.~\cite{Bodlaender2014}.


\begin{definition}[Polynomial equivalence relation \cite{Bodlaender2014}]
  An equivalence relation  on  is said to be \emph{polynomial} if the following two conditions hold:
  (i) There is an algorithm that given two strings  decides whether  and  belong to the same equivalence class in time .
  (ii) For any finite set  the equivalence relation  partitions the elements of  into at most  classes.

\end{definition}

\begin{definition}[OR-cross-composition \cite{Bodlaender2014}] Let  be 
a set and let  be a parameterized problem. We say 
that  \emph{cross-composes} into  if there is a polynomial equivalence relation 
 and an algorithm which, given  strings  belonging to the 
same equivalence class of , computes an instance  
in time polynomial in  such that: (i)  for some . (ii)  is bounded by a polynomial in .

\end{definition}

\begin{proposition}[\cite{Bodlaender2014}]
Let  be a set which is -hard under Karp reductions. If 
 cross-composes into the parameterized problem , then  has no polynomial 
kernel unless .
\end{proposition}


The \emph{Exponential Time Hypothesis} (ETH) is a conjecture by Impagliazzo et al. asserting that there is no -time algorithm for \textsc{3-SAT} on instances with  variables \cite{ImpagliazzoETH}. 
The ETH, together with the sparsification lemma \cite{ImpagliazzoETH}, even implies that there is no -time algorithm solving \textsc{3-SAT}.
Many algorithmic lower bounds have been proved under the ETH, see for example~\cite{LokshtanovSODA}.


We say that a parameterized problem is {\em fixed-parameter enumerable} if all feasible solutions can be enumerated in  time, where  is a computable function of the parameter  only, and  is a constant.

\section{Parameterized Complexity with respect to the natural parameter}\label{sec:nat}


We study the parameterized complexity of \isi, \mcis, \icsi, and \mccis with 
respect to the natural parameter. 
We will in particular study these problems in 
graphs of bounded degeneracy, chordal graphs, and graphs of large girth.



\begin{theorem}\label{thm:complete}
MCIS, MCCIS, ISI, and ICSI are -complete.
\end{theorem}

\begin{proof}
Since those problems are -hard by a straightforward reduction from \textsc{-Clique}, it suffices to show membership in .
In \cite{Cesati03}, it is shown that if a problem can be reduced in FPT time 
to simulating a non-deterministic single-taped Turing machine halting in at 
most  steps, for some function , then it is in .
The Turing machine can have an alphabet and a set of states of size depending 
on the size of the input of the initial problem.
In our case, we can design a Turing machine that guesses in  steps the 
corresponding right  vertices in  (for I(C)SI this part is not 
necessary) and the right  vertices in  (our alphabet being isomorphic 
to an indexing of ) and then check in time  whether 
the two induced subgraphs are isomorphic (and that they are connected for ICSI 
and MCCIS).
\end{proof}



In \cite{MS09} it was shown that 
\textsc{Maximum Induced Matching}\footnote{where one looks for a largest 
subset of vertices that induce a disjoint union of edges} is -hard 
on bipartite graphs. This implies that MCIS is -hard on bipartite 
graphs. In fact, we show that MCIS remains -hard on more restricted 
graph classes, namely -free bipartite graphs with degeneracy .
In particular, those graphs have girth at least .
This result tells us two things about MC(C)IS.
The first is that the fixed-parameter algorithm of Cai et al.~\cite[Theorem 1]{Cai2006} 
cannot be extended from bounded degree to bounded degeneracy (note that some W-hard problems on general graphs become  on graphs with bounded degeneracy, such as the W[2]-complete \textsc{Dominating Set} problem~\cite{AlonG09}).
The second is that short cycles are not making MC(C)IS W[1]-hard; they are W[1]-hard even without them.
In \cite{RamanS08}, the authors present fixed-parameter algorithms on graphs 
of girth 5, for some problems which are W-hard on general graphs.
MCIS and MCCIS are also resistant to this approach.


\begin{theorem}
\isi and \icsi are -complete even when both graphs are -free 
bipartite graphs with degeneracy at most .
\end{theorem}
\begin{proof}
The incidence graph  of any graph , obtained by subdividing each edge of  once, has degeneracy . 
Indeed, graph  is the bipartite graph  where the edges of  are all the  for which , , and  is an endpoint of . 
All the vertices  of  have degree .
Therefore, they can be removed first. 
Then, what is left in  is the independent set .

We transform any input  of \textsc{-Clique}, into the instance  of \ISIC, where both graphs have degeneracy .
The problem consists of finding the incidence graph of a -clique within the incidence graph of . 
We show that it is equivalent to finding a -clique in .
Obviously, if there is a -clique  in , then the graph  is isomorphic to .
Now, let us assume that  is isomorphic to an induced subgraph of .
We denote by  the vertices of  with degree , and by  the vertices of  with degree .
We denote by  the isomorphism from graph  to an induced subgraph of .
Let  for each , and  for each .
We set .
For every ,  since the degree of  in  is  (hence, the degree of  in  is also ).
Now, for every ,  since  has two neighbors in  (recall that  is bipartite).
Therefore,  are  vertices in  inducing precisely  edges.
Hence,  is a -clique in .

Membership in  comes from \autoref{thm:complete}. 
\end{proof}

\begin{corollary}
\mcis and \mccis remain -complete on bipartite graphs of girth 6 and 
degeneracy 2.
\end{corollary}

The absence of triangles and cycles of length four in the input graphs does not make the problems tractable.
We show that the absence of a long induced cycle does not help either (in~\cite{Arumugam11}, the authors show that the -hard problem \textsc{Dominator Coloring} is in  when the input graph is chordal).
More specifically, all four problems are -hard on chordal graphs.
In fact, we can even show that these problems remain -hard on a proper subclass of chordal graphs called split graphs.
A split graph is a graph whose vertex set can be partitioned into a set inducing a clique and an independent set.

\begin{theorem}
ISI (hence MCIS) and ICSI (hence MCCIS) remain -hard on split graphs.
\end{theorem}

\begin{proof}
Similarly to the previous construction, we define  as the graph 
 where the edges of  are the edges  for which 
, , and  is an endpoint of , plus all the edges 
 with . 
The graph  is split:  induces a clique in  and  induces 
an independent set.
From an instance  of \textsc{-Clique} with , we build the 
equivalent instance  of MC(C)IS and I(C)SI.
The soundness can be obtained in the same way as in the previous proof.
\end{proof}









Let us now say some words about the complexity of the connected version.
First we note that MCIS is -hard on forests while MCCIS is solvable in 
polynomial-time in this case: given two forests  and , run the 
polynomial-time MCIS algorithm of Akutsu on every pair of trees 
from  and ~\cite{Akutsu1992}.
From the parameterized complexity standpoint,
\mccis is  whenever \isi is  
since the enumeration of all  possible induced \emph{connected} subgraphs can be used as described in the introduction.
The converse is true on classes of graphs which are closed by adding a universal vertex (i.e., a vertex linked to all the other vertices).
An instance  of ISI can be reduced to an equivalent instance  of MCCIS by letting  be the graph obtained by adding a vertex to  that is made adjacent to all other vertices of .


\section{Structural parameterization}\label{sec:struct}

Let us first recall that , where  (resp. , ) represents the treewidth (resp. the feedback vertex set number, the vertex cover number) of ~\cite{Fellows2013}.
As noted before, if the parameter is the combination of  and  then MCIS is known to be -hard.
Even more, if the parameter is the combination of  and  (which is bigger than the combination of the treewidth), then the problem is not even in  since \mcis and \isi are -hard on disjoint union of chordless paths, a case where the parameter is equal to 0~\cite{Damaschke1991,Garey1979}.



\begin{theorem}[\cite{Damaschke1991,Garey1979}]
\label{notinXP}
\mcis is paraNP-hard when parameterized by  (and hence by ).
\end{theorem}







One can extend this result to make it valid for the connected version.

\begin{theorem}\label{th:fvsone}
\icsi, and as a corollary \mccis, are paraNP-hard when parameterized by .
\end{theorem}


\begin{proof}
  Given an instance of \isi on forests  and  (each with at least 2 trees), we build an instance of \icsi by adding a universal vertex (connected to every node) in  and in .
  Both graph have thus a feedback vertex set of value one.
  One can see that these two universal vertices must be matched together since they are the only ones with sufficiently high degree.
  Then, there is a solution for \isi iff there is a solution for \icsi. 
The result of course holds for MCCIS, too.
\end{proof}












\medskip







It was shown in \cite{AbuKhzam2014} that MCIS is in  if the parameter is the combination of  and . Accordingly, the problem has a kernel, but no polynomial bound is known on its size. We show that, in this case,
the kernel cannot be polynomial unless .


\begin{theorem}\label{th:no poly kernel}
Unless , \mcis has no polynomial kernel when parameterized by the sum of the sizes of vertex covers in the two input graphs.
\end{theorem}

\begin{proof}

We will define an OR-cross-composition from the -complete \clique, problem, where the given instance is a tuple  and the question is whether the graph  contains a clique on  vertices. 

Given  instances, , of \clique, where  is a graph and , we define our equivalence relation  such that any strings that are not encoding valid instances are equivalent, and  are equivalent iff , and . Hereafter, we assume that  and , for any . We will build an instance of \mcis parameterized by the vertex cover  where  and  are two graphs,  and  is a vertex cover of  computed in fpt-time, such that there is a solution of size  for \mcis iff there is an  such that there is a solution of size  in . We will now describe how to build  and .\\

To build  (see also Figure~\ref{fig:g2}):

\begin{itemize}
\item       ,
\item ,
\item   ,
\item   ,
\item   ,
\item .
\end{itemize}

\begin{figure}[ht!]
\centering
\begin{tikzpicture}[scale=0.9,auto]

\node[draw] (r) at (-1,0) {};
\node[draw] (p) at (-1.7,0.5) {};
\node[draw] (q) at (-0.3,0.5) {};

\draw (p) -- (r) -- (q);
\draw (p) -- (q);

\node[draw] (t1) at (-3,-1) {};
\node[draw] (t2) at (-2,-1) {};
\node () at (-0.5,-1) {};
\node[draw] (tt) at (1,-1) {};
\node[draw,rectangle,rounded corners, fit = (t1) (tt)]  () {};

\node[draw] (e12) at (-4,-3) {};
\node[draw] (e13) at (-3,-3) {};
\node () at (-2,-3) {};
\node[draw] (e1n) at (-1,-3) {};
\node[draw] (e23) at (-0,-3) {};
\node () at (1,-3) {};
\node[draw] (en-1n) at (2,-3) {};
\node[draw,rectangle,rounded corners, fit = (en-1n) (e12)]  () {};

\draw[color=black,fill=white,rounded corners] (-3.5,-5.4) rectangle (1.5,-4.6);
\node[draw] (v11) at (-2.8,-5) {};
\node[draw] (v12) at (-1.8,-5) {};
\node () at (-0.5,-5) {};
\node[draw] (v1n) at (1,-5) {};


\draw 	(t1) -- (r) -- (t2);
\draw (r) -- (tt);

\draw[color=black,fill=white,rounded corners=0.0cm] (-3.7,-2.3) rectangle (1.7,-1.7);
\node () at (-1,-2) {};

\draw (-1,-1.35) -- (-1,-1.7);
\draw (-1,-2.3) -- (-1,-2.63) ;

\draw[color=black,fill=white,rounded corners=0.0cm] (-4.3,-3.85) rectangle (2.3,-4.25);
\node () at (-1,-4.05) {};

\draw (-1,-3.36) -- (-1,-3.85);
\draw (-1,-4.25) -- (-1,-4.6);




\end{tikzpicture}
\caption{Illustration of the construction of .}\label{fig:g2}
\end{figure}

To build  (see also Figure~\ref{fig:g1}):
\begin{itemize}
\item     ,
\item ,
\item   ,
\item   ,
\item .
\end{itemize}

\begin{figure}[ht!]
\centering
\begin{tikzpicture}



\node[draw] (r2) at (-0.5,-1.3) {};
\node[draw] (p2) at (-1.2,-0.8) {};
\node[draw] (q2) at (0.2,-0.8) {};

\draw (p2) -- (r2) -- (q2);
\draw (p2) -- (q2);

\node[draw] (ti) at (-0.5,-2) {};

\node[draw] (e1) at (-2,-3) {};
\node[draw] (e2) at (-1,-3) {};
\node () at (0,-3) {};
\node[draw] (e2l) at (1,-3) {};
\node[fit=(e1)(e2)(e2l),draw,rounded corners] (e1g1) {};

\node[draw] (v1) at (-2,-5) {};
\node[draw] (v2) at (-1,-5) {};
\node () at (0,-5) {};
\node[draw] (vl) at (1,-5) {};
\node[fit=(v1)(v2)(vl),draw,rounded corners] (vg1) {};



\node[draw] (eg1) at (-0.5,-4) {};



\draw (-0.5,-3.4) -- (-0.5,-3.7);
\draw (vg1) -- (eg1);

\draw (r2) -- (ti) -- (e1);
\draw (e2) -- (ti) -- (e2l);






\end{tikzpicture}
\caption{Illustration of the construction of .}\label{fig:g1}

\end{figure}


We set , and . It is easy to see that  is indeed a vertex cover for  and that its size is equal to , which is polynomial in  and hence in the size of the largest instance. Note that the size of the graph  does not depend on  and is polynomial in , so the size of its vertex cover is also polynomial in  and independent of .

Let us show that  is an induced subgraph of  iff at least one of the 's has a clique of size .

 Suppose that  has a clique of size . We denote by  a clique of size exactly  in . We show that there is an induced subgraph  of  of size , isomorphic to . We set   . One can easily check that this subgraph is isomorphic to .

 Assume now that  is an induced subgraph of . 
Denote by  the subgraph of  isomorphic to . 
Note that the only triangle in  is .
Indeed,  is bipartite.
The triangle  in  has therefore to match  in .
Moreover,  in  has to match  in  since  and  have no edges besides the clique .
The vertex  in  can only match a vertex  for some .
Then,  up to  in  has to match  vertices in    of  which correspond to actual edges in . 
Finally,  up to  in  has to match  vertices among the 's in .
Note that the number of edges in  between the 's and the 's is 
exactly . 
More precisely, each  touches  edges in  and each  touches  edges in .
In order to get a match in , one should find a set of  edges inducing exactly  vertices.
So, this set of  vertices is a clique in .
\end{proof}

Note that the parameter of MCIS in the previous reduction is exactly the size of  and the graphs used in the proof are connected. 
Therefore, we have the following: 

\begin{corollary}
\isi and \mccis, parameterized by a bound on the minimum vertex covers of input graphs, do not have a polynomial-size kernel unless .
\end{corollary}



The algorithm of~\cite{AbuKhzam2014} is not single-exponential for parameter sum of the vertex cover numbers. 
In fact, we show that a single-exponential algorithm is very unlikely.
This is, to the best of our knowledge, the first result of this type for parameter vertex cover.


\begin{theorem}
Under the ETH, IS(C)I cannot be solved in time  when parameter  is the sum of the vertex cover number of both graphs.
\end{theorem}

\begin{proof}
We give a reduction from  \textsc{Permutation Clique} which linearly preserves the parameter .
It is known that this problem does not admit an algorithm with running time  unless the ETH fails \cite{Lokshtanov2011}.
In the  \textsc{Permutation Clique} problem, one is given a graph over the set of vertices  and the goal is to find a clique of size  such that in each \emph{row} and in each \emph{column} exactly one vertex is part of the clique, where a \emph{row} is the set of vertices  for some , and a column is the set of vertices  for some .

We first describe how the graph  is built from any instance  of  \textsc{Permutation Clique}.
For each row (resp. column) index , we add two vertices  and  (resp.  and ) that we link by an edge.
For , we set  (resp. ) and  (resp. ).
Then, to distinguish row indices from column indices, we add a clique  of size , and we link one designated vertex  of  to all the vertices in .
We also add a clique  of size  with a special vertex  in  such that  is linked to all the vertices in .

Finally, for each edge  of  with  and  \footnote{We ignore the other edges since they are not relevant in finding a permutation clique.}, we add a vertex  that we link to the four vertices , , , and , and a vertex  that we link to the four vertices , , , and .
This ends the construction of  (see Figure~\ref{fig:overall-vc}).
The pattern  depends only on  and is defined as the graph one obtains following the above construction when  have all the edges of the form  and no other edges (in other words,  has a -clique on the diagonal and nothing else). 

\begin{figure}
\centering
\begin{tikzpicture}[]
\def\k{4}
\node[draw,circle] (r) at (-2.2,\k+1.25) {} ;
\node[draw,circle] (dr1) at (-3.8,\k+1.25) {} ;
\node[draw,circle] (dr2) at (-2.66,\k+2) {} ;
\node[draw,circle] (dr3) at (-3.33,\k+2) {} ;
\node[draw,circle] (dr4) at (-2.66,\k+0.5) {} ;
\node[draw,circle] (dr5) at (-3.33,\k+0.5) {} ;
\foreach \i [count=\s from 0] in {1,...,5}{
  \foreach \j in {1,2,...,\s}{
    \draw[very thin] (dr\i) -- (dr\j) ;
}
\draw[very thin] (r) -- (dr\i) ;
}
\node[draw,circle,fit=(r) (dr1) (dr2) (dr3) (dr4) (dr5),inner sep=-0.15cm] (Dr) {} ;
\node[below of=Dr] {} ;
\node[draw,circle] (v) at (-2.2,-2.2) {} ;
\node[draw,circle] (d1) at (-2,-3) {} ;
\node[draw,circle] (d2) at (-2.65,-3.5) {} ;
\node[draw,circle] (d3) at (-3.3,-3) {} ;
\node[draw,circle] (d4) at (-3,-2.2) {} ;
\foreach \i [count=\s from 0] in {1,...,4}{
  \foreach \j in {1,2,...,\s}{
    \draw[very thin] (d\i) -- (d\j) ;
}
\draw[very thin] (v) -- (d\i) ;
}
\node[draw,circle,fit=(v) (d1) (d2) (d3) (d4),inner sep=-0.05cm] (D) {} ;
\node[below of=D] {} ;
\foreach \i in {1,...,\k}{
\foreach \j in {1,2}{
\node[draw,circle,inner sep=0.04cm] (r\i\j) at (0,\i+\k*\j-\k+0.5*\j) {} ;
\draw[very thin] (r) -- (r\i\j) ;
}
\draw[very thin] (r\i1) to [bend right] (r\i2) ;
}
\foreach \j in {1,2}{
\node[draw,rectangle,rounded corners,fit=(r1\j) (r\k\j)] (R\j) {\\} ;
}
\node[draw,rectangle,rounded corners,fit=(R1) (R2)] (R) {\\} ;
\foreach \i in {1,...,\k}{
\draw[very thin] (v) -- (r\i1) ;
}
\foreach \i in {1,...,\k}{
\foreach \j in {1,2}{
\node[draw,circle,inner sep=0.04cm] (c\i\j) at (\i+\k*\j-\k+0.5*\j,0) {} ;
}
\draw[very thin] (c\i1) to [bend right] (c\i2) ;
}
\foreach \j in {1,2}{
\node[draw,rectangle,rounded corners,fit=(c1\j) (c\k\j)] (C\j) {} ;
}
\node[draw,rectangle,rounded corners,fit=(C1) (C2)] (C) {} ;
\foreach \i in {1,...,\k}{
\draw[very thin] (v) -- (c\i1) ;
}
\node[draw,ellipse] (ve11) at (5,9) {} ;
\draw[very thick] (r21) -- (ve11) -- (c11) ;
\draw[very thick] (r32) -- (ve11) -- (c22) ;
\node[draw,ellipse] at (6.2,8.3) (ve12) {} ;
\draw[very thick] (r22) -- (ve12) -- (c12) ;
\draw[very thick] (r31) -- (ve12) -- (c21) ;

\begin{scope}[xshift=3cm,yshift=-1.7cm]
\node[draw,ellipse] (ve21) at (5,9) {} ;
\draw (r31) -- (ve21) -- (c11) ;
\draw (r42) -- (ve21) -- (c32) ;
\node[draw,ellipse] at (6.2,8.3) (ve22) {} ;
\draw (r32) -- (ve22) -- (c12) ;
\draw (r41) -- (ve22) -- (c31) ;
\end{scope}
\end{tikzpicture}
\caption{The overall construction of . We represented only two edges of :  and . For the sake of readability, the edges encoding  are enhanced to distinguish them easily from the edges encoding .}
\label{fig:overall-vc}
\end{figure}

Both  and  have  as a vertex cover of size 
.  has  vertices and  has 
 vertices.
To avoid confusion about vertices in  and  we will denote the 
vertices and sets of vertices of  with a tilde.
We now show that the reduction is valid.

Suppose there is a solution  to the instance of  \textsc{Permutation Clique}.
Then,  is an induced subgraph of  with the following mapping.
We map  to  and  to .
We map  to  and  to  in an arbitrary way.
Then, for each  and , we map  to  and  to .
We observe that this mapping is one-to-one since  is a \emph{permutation} clique, i.e., .
Finally, for any , and any  we map  to .
Note that vertex  always exists precisely because  is a clique.

Conversely, if there is a solution to the IS(C)I instance, we will show that there is a permutation -clique in .
There is only one clique of size  in , so the clique  of size  has to be mapped to .
Then, , as the unique vertex of  of degree larger than , should be mapped to .
Now, for the same reasons,  should be mapped to  and  to .
Vertices of  are the only  unmatched vertices having  as a neighbor, so those vertices should be matched to the only  unmatched vertices having  as a neighbor, namely .
For similar reasons,  should be mapped to .
Now,  can only be mapped to  as the only unmatched vertices having exactly one neighbor in  ().

Thus, the  vertices of  can only be mapped to , such that for ,  is mapped to  and  is mapped to .
The edges  and  (resp.  and ) further constrains the mapping: if  is mapped to  then  has to be mapped to  (resp. if  is mapped to  then  has to be mapped to ).
Hence, we can see the mapping from  to  as two permutations  and  on  elements, such that for , for ,  is mapped to  and  is mapped to .
Then, the current partial mapping can be extended to a solution only if  is a clique in .
Indeed, ,  ,  can only be mapped to a potential  so that vertex has to exist, meaning that there should be an edge in  between  and .

An algorithm solving IS(C)I in time  with  would therefore translate into an algorithm running in time  for  \textsc{Permutation Clique} and contradict the ETH.

\end{proof}

Despite the fact that ISI and MCIS have the same parameterized complexity with respect to the natural parameter, they exhibit different behaviors when considering structural parameters.
In fact, the latter is paraNP-hard when parameterized by the vertex cover of only one of the two graphs, whereas ISI is  when parameterized by the vertex cover of the second (host) graph.
To see this, note that when the host graph has a vertex cover of size , the minimum size of a vertex cover in the pattern graph must be bounded by the parameter ; otherwise we have a NO-instance.
The claim follows from the fixed-parameter tractability of MCIS in this case~\cite{AbuKhzam2014}.  









\medskip











Given the negative result of Theorem~\ref{th:fvsone}, the next question we pose is whether MCCIS is in  with respect to the size of a minimum vertex cover. 
In \cite{AbuKhzam2014}, a parameterized algorithm is presented for MCIS when the parameter is a bound on the minimum vertex cover number of the input graphs.
However, that algorithm cannot help us much for solving MCCIS since it relies on the existence of a feasible solution of size at least  which consists of mapping the two \emph{big} independent sets of the two graphs onto each other.
Of course, this is not a feasible solution for MCCIS.
In the following we prove that MCCIS is fixed-parameter tractable w.r.t. .



\begin{theorem}
\label{mccis_vc}
\mccis parameterized by  is fixed-parameter tractable.
\end{theorem}


\begin{proof}
In time  (even  \cite{CKX10}), we can find minimum vertex covers  and  in  and  respectively.
Let  be the independent set  for .
By assumption, our parameter  is , so we can enumerate all tripartitions of  and  in time .
We denote by ,  and  (respectively ,  and ) the three sets of a tripartition of  (respectively ).
For ,  corresponds to the vertices of  that are not matched, so they may be deleted.
 comprises the vertices matched to  (that is, to the vertex cover of the other graph), and  are the vertices matched to , the independent set of the other graph. See Figure~\ref{vcvc}.

We observe that for ,  can be partitioned into at most  classes of twins: . A class of twins in this context is a set of vertices with an identical neighborhood in the vertex cover and there are at most  subsets of . Potentially, some classes can be empty: they correspond to a subset of the vertex cover  that is not the (exact) neighborhood of any vertex in .

At this point, we can enumerate the mappings between  and  in time  and the mappings between  and  in time .
Indeed, to match a vertex  with a vertex  or a twin of  is equivalent.
Thus, in time  we can enumerate all the solutions of MCIS where only vertices of  could still be matched to vertices of .
The optimal map of the independent sets can be done in polynomial time by matching the greatest number of vertices in each \emph{equivalent} twin class (which is the size of the smaller of the two equivalent twin classes), where a twin class  in  is equivalent to a twin class  in  if the vertices of  and  are in one-to-one correspondence.
\end{proof}

\begin{figure}[htb!]
\begin{center}
\begin{tikzpicture}

\draw[color=black,fill=white] (0,0) rectangle (3,4);
\draw (0,1.5) -- (3,1.5);
\draw[decorate,decoration=brace] (-0.2,0) -- (-0.2,1.45) node[midway,anchor=east,inner sep=3pt, outer sep=3pt]  {};
\draw[decorate,decoration=brace] (-0.2,1.55) -- (-0.2,4) node[midway,anchor=east,inner sep=3pt, outer sep=3pt]  {};
\draw (0,0.5) -- (3,0.5);
\draw (0,1) -- (3,1);
\node() at (1.5, 0.25) {};
\node() at (1.5, 0.75) {};
\node() at (1.5, 1.25) {};

\node () at (1.5,-0.5) {};

\node[draw,circle] (i11) at (0.75,3.5) {};
\node[draw,circle] (i12) at (0.45,2.5) {};
\node[draw,circle] (i13) at (0.753,2) {};
\node[draw,rectangle,rounded corners,dashed,fit=(i11) (i12) (i13)] (i1) {};

\node (d) at (1.5,3) {};

\node[draw,circle] (i2k1) at (2.3,3.5) {};
\node[draw,circle] (i2k2) at (2.5,2.2) {};
\node[draw,rectangle,rounded corners,dashed,fit=(i2k1) (i2k2)] (i2k) {};

\node[draw,circle] (v1) at (0.3,1.25) {};
\node[draw,circle] (v2) at (2.2,1.25) {};
\node[draw,circle] (v2b) at (2.6,1.25) {};


\draw (v1) edge[bend left=30] (i11) ;
\draw (v1) -- (i12) ;
\draw (v1) -- (i13) ;


\draw (i2k1) edge[bend right=25] (v2) ;
\draw (i2k2) -- (v2) ;
\draw (i2k1) edge[bend left=30] (v2b) ;
\draw (i2k2) -- (v2b) ;

\begin{scope}[xshift=1cm]
\draw[color=black,fill=white] (3,0) rectangle (6,4);
\draw[decorate,decoration=brace] (6.2,1.45) -- (6.2,0) node[midway,anchor=west,inner sep=3pt, outer sep=3pt]  {};
\draw[decorate,decoration=brace] (6.2,4) -- (6.2,1.55) node[midway,anchor=west,inner sep=3pt, outer sep=3pt]  {};
\draw (3,1.5) -- (6,1.5);
\draw (3,0.5) -- (6,0.5);
\draw (3,1) -- (6,1);
\node() at (4.5, 0.25) {};
\node() at (4.5, 0.75) {};
\node() at (4.5, 1.25) {};
\node () at (4.5,-0.5) {};

\node[draw,circle] (i11) at (3.5,3.5) {};
\node[draw,circle] (i13) at (3.753,2) {};
\node[draw,rectangle,rounded corners,dashed,fit=(i11)  (i13)] (i1) {};

\node (d) at (4.5,3) {};

\node[draw,circle] (i2k1) at (5.4,3.5) {};
\node[draw,circle] (i2k2) at (5.6,2.2) {};
\node[draw,circle] (i2k3) at (5.2,2.2) {};
\node[draw,rectangle,rounded corners,dashed,fit=(i2k1) (i2k2) (i2k3)] (i2k) {};

\node[draw,circle] (v1) at (3.4,1.25) {};
\node[draw,circle] (v1b) at (3.8,1.25) {};
\node[draw,circle] (v2) at (5.2,1.25) {};


\draw (v1) edge[bend left=15] (i11) ;
\draw (v1) -- (i13) ;
\draw (v1b) edge[bend right=30] (i11) ;
\draw (v1b) -- (i13) ;


\draw (i2k1) edge[bend right=30] (v2) ;
\draw (i2k2) -- (v2) ;
\draw (i2k3) -- (v2) ;
\draw (i2k1) -- (v1b) ;
\draw (i2k2) -- (v1b) ;
\draw (i2k3) -- (v1b) ;



\end{scope}

\draw (2.8,0.75) edge[<->] (4.2,0.75);
\draw (2.8,1.25) edge[<->] (4.2,2.75);
\draw (2.8,2.75) edge[<->] (4.2,1.25);
\draw (2.8,3) edge[<->] (4.2,3);


\end{tikzpicture}


\end{center}



\caption{Illustration of the proof of Theorem \ref{mccis_vc}. Dashed boxes represent the classes inside the independent set. Arrows represent the matching between sets of vertices. Sets  (resp. ) represents a vertex cover for  (resp. ).}
\label{vcvc}
\end{figure}



To find a solution for MCCIS, the algorithm described in the above proof enumerates all possible maximal common induced subgraphs in time . 
The current bottleneck to improve it is when we try to match vertices of the vertex cover with vertices of the independent set.
For the not connected version of the problem, a trivial argument can bound the size of the independent set (if this one is big, there is a trivial solution), which cannot be used for the connected version.
As such, it can be used as an enumeration algorithm for MCIS. 




\begin{corollary}
\mcis parameterized by  is fixed-parameter enumerable.
\end{corollary}





Let us finish this section with some general considerations.
Note that for ISI, the parameter  is not the same 
as . 
In the latter, the parameter is a bound on the vertex cover of  (as well as the feedback vertex set of ) which makes ISI in , while it remains open for . 
We also note that ISI is not in  w.r.t.  by a simple reduction from \textsc{Independent Set}: let  be an edgeless graph on  vertices, then its vertex cover number is 0.

We now turn our attention to the case where MCIS is parameterized by a 
combination of the natural parameter and some structural parameter. We note 
that, in general, such parameterization reduces the problem's complexity. This 
is most often due to the fixed-parameter tractability of MCIS in -minor 
free graphs (again, since ISI is  in this case~\cite{FG01}).
For example, consider the case where the parameter is the sum of some bound  
on the feedback vertex set of the input graphs and the natural parameter . 
The problem is  in this case since graphs of -feedback vertex set are 
-minor free where  is the ``fixed'' graph consisting of a disjoint union 
of  triangles. The same applies to parameterization by treewidth and the 
natural parameter by considering  to be the complete graph on  
vertices, for example.



\section{Conclusion}

We studied the \mcis and \mccis problems with respect to the solution size on special graph classes such as forests, bipartite graphs, bounded degree graphs, bounded degeneracy graphs, graphs without small (length 3 or 4) cycles. 
The two problems are fixed-parameter tractable on -minor free graphs, which include forests, and bounded degree graphs, but they are -complete on bipartite graphs of girth 6 and degeneracy 2.
This ruled out at the same time two approaches to get fixed-parameter algorithms on subclasses of graphs for W-hard problems.

We then considered the use of structural parameters, such as a bound on the minimum vertex covers of the input graphs. 
Although both MCIS and MCCIS are in  in this case, we proved that no kernel of polynomial bound can be obtained unless  and that they cannot be solved in time  under the ETH.
We noted that MCIS is not even in  with respect to other (smaller) auxiliary parameters, such as treewidth and feedback vertex set. A few open problems remain to be addressed. 
For example, is MCIS/MCCIS in  when parameterized by the combination of the vertex cover number and the feedback vertex set number, or by the vertex cover number and the treewidth?
Moreover, it would be interesting to know whether the algorithm for MCCIS of Theorem~\ref{mccis_vc} can be improved to match the lower bound.










