\documentclass{article}
\pdfoutput=1
\textwidth 6.8in \addtolength{\oddsidemargin}{-1.1in}
\textheight 9in
\addtolength{\topmargin}{-0.5in}
\setlength{\parindent}{0.0cm}
\setlength{\parskip}{0.1cm}
\topskip 0.0in
\pagestyle{empty}
\thispagestyle{empty}



\usepackage{algorithmic}
\usepackage{algorithm}
\usepackage{amssymb,amsmath,amsfonts}
\usepackage{graphicx}
\usepackage{verbatim}
\usepackage{subfigure}
\usepackage{url}
\usepackage{empheq}
\usepackage[final]{changes}
\usepackage{changes}
\normalem


\usepackage{tikz}
\usetikzlibrary{positioning,chains,fit,shapes,calc,arrows,automata}


\newtheorem{thm}{Theorem}
\newtheorem{lemma}{Lemma}
\newtheorem{defi}{Definition}
\newtheorem{cor}{Corollary}

\newcommand{\U}{\mathcal{U}}
\newcommand{\A}{\mathcal{A}}
\newcommand{\D}{\mathcal{D}}
\newcommand{\N}{\mathcal{N}}
\newcommand{\T}{\mathcal{T}}
\newcommand{\Ao}{\A^o}
\newcommand{\At}{\A^{24}}
\newcommand{\Uo}{\U^o}
\newcommand{\Ut}{\U^{24}}
\newcommand{\Utb}{\mathcal{U}_b^{24}}
\newcommand{\Uob}{\mathcal{U}_b^{o}}
\newcommand{\Atb}{\mathcal{A}_b^{24}}
\newcommand{\Aob}{\mathcal{A}_b^{o}}



\begin{document}

\title{Designing Low Cost and Energy Efficient Access Network for the Developing World}

\author{
  Kshitiz Verma\\ IIT Kanpur,India \\ UC3M, Spain\\
  \texttt{vermasharp@gmail.com}
  \and
  Shmuel Zaks \\ Technion, Haifa, Israel\\
  \texttt{zaks@cs.technion.ac.il}
  \and
  Alberto Garcia-Martinez \\ UC3M, Spain\\
  \texttt{alberto@it.uc3m.es}
}





\maketitle


\begin{abstract}
Internet is growing rapidly in the developing world now. Our survey of four networks in India, all having at least one thousand users, suggest that both installation cost and recurring cost due to power consumption pose a challenge in its deployment in developing countries. In this paper, we first model the access design problem by dividing the users in two types 1) those that may access the network anytime and 2) those who need it only during office hours on working days. The problem is formulated as a binary integer linear program which turns out to be NP-hard. We then give a distributed heuristic for network design. We evaluate our model and heuristic using real data collected from IIT Kanpur LAN for more than 50 days. Results show that even in a tree topology -- which is a common characteristic of all networks who participated in our study, our design can reduce the energy consumption of the network by up to 11\% in residential-cum-office environments and up to 22\% in office-only environments in comparison with current methods without giving up on the performance. The extra cost incurred due to our design can be compensated in less than an year by saving in electricity bill of the network.
\end{abstract}

\section{Introduction} 
\label{sec:intro}


\added{\subsection{Networks in developing countries} }
The Government of India recently announced its \60-\45\,^{\circ}\mathrm{C}\N\U =\{u_1,u_2, \cdots, u_U\}\A=\{a_1,a_2, \cdots, a_A\}\D = \{d_1,d_2, \cdots, d_D\}su \in\Ua \in\AC_{ua}l_{ua}a \in\Ad \in\DC_{ad}l_{ad}d \in\DsC_{ds}l_{ds}\tau_{copper}\tau_{fiber}\Tss\A\Dua_u\Ta\delta_ad\delta_d\T\T\Tv(v,w)c_vc_{vw}uaduadsc_uc_ac_dc_s(u,a)(a,d)(d,s)c_{ua}c_{ad}c_{ds}c_u=0u \in\UvP_vsdaP_sP_dP_{a}Pc_\ic_{11},l_{11}c_{21},l_{21}c_{31},l_{31}c_{42},l_{42}c_{52},l_{52}c_{63},l_{63}uaa\delta_a\N\T\Usuad(i,j)ijC_{ij}(i,j)l_{ij}(i,j)c_{ij}(i,j)c_iiP_ii\delta_ii\U\U\Uo\Ut\Uo \cup \Ut = \U\Uo \cap \Ut = \emptyset\Uo\Ao\At\Ao \cup \At = \A\Ao \cap \At = \emptyset\Uo\Ao\Ut\At\Ao\mathcal{N}\U= \Uo \cup \Ut = \{1,2, \cdots, \mathcal{U}\}\Uo \cap \Ut = \emptyset\A= \Ao \cup \At = \{1,2, \cdots, \mathcal{A}\}\Ao \cap \At = \emptyset\D=\{1,2, \cdots, \mathcal{D}\}su \in\Ua \in\AC_{ua}\tau_{copper}l_{ua}a \in\Ad \in\DC_{ad}\tau_{copper}l_{ad}d \in\DsC_{ds}\tau_{fiber}l_{ds}\delta_aa \in\A\delta_dd \in\DL,\alpha\geq 1,\beta\geq 0\mathcal{T}\mathcal{N}\Us\Uo\Ut\Ao\AtLx_az_d10adw_{ua}y_{ad}10ua\mathcal{T}ad\delta_aauaaadduu\Aa_u\Tu\T\alpha\betaA_o\in\AoD_oA_o\in \AoD_o\Ao\mathcal{B}b\in \mathcal{B}{\U}_b\A_bb\Uob\Utb\Uob \cup \Utb = \U_b\Uob \cap \Utb = \emptysetb\Aob\Atb\Aob \cup \Atb = \A_b\Aob \cap \Atb = \emptyset\Uob\Aob\Utb\Atbb\Aob\Atb18^{th}9^{th}\Uo\Uo\Uo\Ao\Uo\Ao\Uo\Ao\Ut\Uo\Uo\Ut\Uo\Uo\Uo\Uo\Uo\Ut\Uo\Ut7^{th}9^{th}\Uo\Uo\Ao\Uo\Uo\Ao\Uo\Ao\Uo\Ao\Ao\Uo<\Ao\Ao\Ao_{24}_{24}\)} & 
\textbf{Op.Temp.} \\
 \hline\hline
Juniper  & EX2200-24T-4G & 50 & 1,212.80 & 76 & 2044.80 &    \\ \hline
Cisco  & WS-C2960G-24TC-L & 55 & 2,277.0 & 80 & 4137.00 &   \\ \hline
D-Link  & DGS-3420-28TC & 50.8 & 1,529.99 & 81 & 2549.99 &   \\ \hline
HP & HP 2920-24G (J9726A) & 58 & 1065.60 & 70 & 1873.85 &     \\ \hline
\end{tabular*}
\end{center}
\caption{Some of the available switches for access layer.}
\label{tab:iitk}
\normalsize
\end{table*}




\section{Pinging in the network}
\label{sec:pinging}

\begin{figure}[t]
\centering
\includegraphics[width=12cm]{failures.eps}
\caption{Number of unreachable switches during different dates, plotted hourly. A `+' represents the number of switches which did not echo ping request exactly once whereas a `' represents the number of switches that did not echo ping request more than once in the considered hour.}
\label{fig:unreachable}
\end{figure}

We ping every switch in the network for 55 days starting from  May to  July 2014, every 20 minutes, the minimum time period allowed by the computer center staff for continuous pings. Fig. \ref{fig:unreachable} presents the hourly result of ping experiment collected. We divide the failure of ping in two cases, 1). When the ping request is not echoed exactly once in an hour, 2). When it is not echoed more than once. The idea is to capture long term ans short term failures which can happen due to various reasons like a). Maintenance b). Power outages c). congestion d). ambient temperature. If the computer center (CC) staff turns off a distribution switch for maintenance, that often leads to 20-25 switches being unreachable. As mentioned by CC staff, this happens regularly on a daily basis for different buildings. Sometimes because of power outage in a building, all the switches in the building get switched off (this may account for up to 20-25 switches) and is not a rare event. Thus long durations of failures are mainly due to the first two kinds of reasons. 

One time failures are effected due to rise in ambient temperature, switches reboot frequently they are kept without any cooling whatsoever. We did an analysis of such failures and it turns out that many of such failures were during afternoon 12:00-15:00 hours. It is bound to happen when the ambient temperature is unusually high, the highest ambient temperature this summer was , which is beyond the operational temperature of these switches \cite{2960datasheet}. According to the computer center staff, network becomes crazy during summer. They have to keep running all the time to fix the faults whereas their job is much easier during winters. Congestion level of the network is high but we don't comment much on it in the interest of space. 

In Table \ref{tab:iitk}, we compare some of the available switches that can be used at the access layer of a network. We see that power consumption is more or less the same for all the switches whereas the cost varies drastically. All the prices are quoted from the latest editions of PEPPM \cite{peppm}. We also observe that switches from some vendors have higher operating temperature range, which is beneficial in our case. Hence, choosing all D-Link or HP switches may lead to cheaper networks with better quality of service in summer. A similar comparison of switches at distribution layer is possible. Note that we are considering only 24 port switches, a similar comparison for 48 port switches can be done. 

 

\section{Related work} 
\label{sec:relatedwork}

The need for energy efficiency and its importance in developing countries was first studied in seminal paper by Gupta and Singh \cite{gupta2003greening}. Since then research for designing energy efficient networks have been one of the major research drives and many important references can be found in \cite{bolla2011}. A key observation in the solutions is to exploit the over-provisioning in the networks. Most of the research proposing energy efficiency implicitly considers only the networks in the developed countries as they assume presence of redundant links and/or nodes in the network \cite{mahadevan2010energy} \cite{mahadevan2009energy} \cite{chabarek2008power}. A good amount of literature exists in greening the protocols used for dynamic routing like OSPF \cite{cianfrani2012ospf} \cite{amaldi2011energy}\cite{bianzino2012grida}.

Our work is different from most of these solutions because we provide energy saving network design for a network that 1). has tree topology, i.e., without any redundancy and 2) deploys static routing and 3) is an important network in a developing country like India. Deployment of networks in developing countries pose many challenges \cite{nungu2011powering}. Like us, \cite{he2012greenvlan} study designing of green VLANs but their approach is to make use of the energy efficient hardware, whereas our approach is general enough to be applied to any network to obtain energy savings. \cite{jardosh2009green} propose network on demand approach but for wireless LANs. Switching off of switches by migrating inactive users to wireless network has been proposed in \cite{le2010performance}.

Energy savings in the access network is a difficult task and many innovative methods like \cite{NikosDSLAMSigcom}\cite{reich2010sleepless}\cite{allman2007enabling} have been proposed. A relevant reference for access network design that is close to our design is in \cite{andrews1998access}\cite{gollowitzer2011two}\cite{rodriguez2009improved}  and the references therein. 
 

\section{Conclusion and Future Work} 
\label{sec:conclusion}
Our main observation is the existence of users that allow switching off the networking devices during night hours. However, it is not straightforward to implement the energy savings as the network topology is a tree in which any switched off node may imply disconnection of the users. The issue of electricity saving in the network has to be taken as seriously as for other appliances. We also provide a way of making sure that a switch is not turned off while it is being used by some user. We have also observed that due to extraordinary hot summers, switches that have higher operating temperature range should be used to minimize the reboot of switches due to heat. Our next goal is to do a similar study for a network that uses routers and dynamic routing of packets. We expect more energy savings in such an environment as there are more redundant paths. 

 


\section*{Acknowledgements}
The first author would like to thank Mr. Navpreet Singh, Navneet Sharma, Vijay Gaur and the whole network group at Computer Center, IIT Kanpur for their help with data collection and providing all the necessary and important information related to the LAN of IIT Kanpur.


\bibliographystyle{plain}
\bibliography{paper}

\end{document}
