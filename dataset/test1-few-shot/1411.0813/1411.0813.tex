\documentclass[11pt,a4paper,portrait]{extarticle}
\usepackage{amsmath}
\usepackage{authblk}

\title{An Intuitive Procedure for Converting PDA to CFG, by Construction of Single State PDA}
\author{
Arjun Bhardwaj\thanks{School of Computing and Electrical Engineering, Indian Institute of Technology Mandi, India, bhardwaj.arjun.14@gmail.com} \:\: N.S. Narayanaswamy\thanks{Associate Professor, Department of Computer Science and Engineering, Indian Institute of Technology Madras, India, swamy@iitm.ac.in}
}
\date{Novemeber 1, 2013}


\begin{document}

\maketitle

\section{Introduction}
We present here the proof for an alternative procedure to convert a Push Down Automata (PDA) into a Context Free Grammar (CFG). The procedure involves intermediate conversion to a single state PDA. In view of the authors, this conversion is conceptually simpler than the approach presented in \cite{ull} and can serve as a teaching aid for the relevant topics. For details on CFG and PDA, the reader is referred to \cite{ull}.

\section{Construction Procedure : Multistate PDA to Single State PDA}

For a Push Down Automata (PDA) that accepts by null store : N = , we can construct an equivalent single state PDA :  = ,
such that , and  is of the form [pXq] where , . 
 is constructed by following rules :\\
\begin{enumerate}
\item For every , we add 
\item For every , , we add , 
\item  for all 
\end{enumerate}

\subsection {Proof}

We provide a formal proof for the following claim : 

\subsubsection{(IF part) }

To prove this, we first prove the following claim :\\
\\

This is proved by induction on the number of steps in which X is popped off by the one state automata.\\

Base : . This is possible only if  and . 
Then, according to construction, there is a corresponding entry in transition table of multistate automaton i.e. . \\

Inductive step : assume that if , then ; for . Consider  and  such that  i.e. X is popped off in n+1 steps.
Since  is the string that gets processed as PDA N pops off all of , we can decompose  into  such that ,  ... , where .\\
The rule for single state automaton corresponding to one responsible for , yields . further, by induction hypothesis, ,  and . thus, , proving the claim.\\
As a special case of the claim, consider :  i.e. . then for , . 
Using construction rule 3,  i.e. 

\subsubsection{(ELSE part) }
To prove this, we first prove the following claim :\\
\\
\\
We provide a proof by induction on the no of steps taken by single state automata to pop off [pXq].\\

Base : if . Then there must be a rule in  : , where . 
According to construction, this rule derives from  in N. thus, .\\

Inductive case : assume that if :  then : , for . Consider     . 
Since all the symbols stacked after the first move in place of  will all be popped off, we can decompose  such that , where  . by inductive hypothesis, each of such transitions corresponds to  for PDA N. 
Also, the first transition of  corresponds to presence of rule . thus, . 
Thus , completing the proof of the above mentioned claim.\\\\
Consider . Then for some , . Using the claim, we can assert . \\

Thus, for a single state PDA  constructed from multistate PDA N by the procedure described above, .

\section{construction : single state PDA to grammar G}

For a single State PDA  = , we construct a grammar G = (V,T,P,S), such that L(G) = L(), where V = , T =  and S = . the following rules outline the set of productions, P :\\
\begin{enumerate}
\item For every , we add , where , , 
\end{enumerate}

\subsection{proof}
Here we describe the proof for the following claim : 

\subsubsection{(IF part) }
To prove this we first prove the following claim :\\
\\
, \\
where . Proof is by induction on the number of steps in the derivation of w.\\

Base : . 
This indicates the presence of identical production in P. By construction, we can conclude the presence of . 
Thus, .\\

Induction : assume that if , then , for . 
Consider the left derivation of the string , such that , where , . 
Since v has a leftmost derivation, each  is replaced by a part of the terminal string  i.e. , where , . by inductive hypothesis, . 
Further the first step in derivation indicates the presence of rule  in , such that . 
Thus, \\
For the special case of , the claim yields : if , then . 
Thus, if w is derived by G, then it is accepted by  i.e. .

\subsubsection{(THEN part) }
To prove this we first prove the following claim :\\

, where \\
This is proved by induction on the number of state transitions taken by  to reach .\\

Base : . This is possible only due to the presence of rule . According to construction, G has the following production : Aw. Thus, w is derived by G.

Induction : Assume that if , then , for . Consider  and , such that .
The fist step indicated the presence of the rule  in  and correspondingly the production  in P.
Since all of the stack symbols  are eventually popped off, we can decompose  into  such that , where  and .
By inductive hypothesis, this indicates the presence of derivations . 
Thus, we have a sequence of derivations . Thus proved.

Consider as a special case A=. Then, if  is accepted by , it is derived by G. We can conclude .\\

So, .

\begin{thebibliography}{1}

\bibitem{ull}
{\sc Hopcroft, J.~E., and Ullman, J.~D.}
\newblock {\em Introduction To Automata Theory, Languages, And Computation},
  1st~ed.
\newblock Addison-Wesley Longman Publishing Co., Inc., Boston, MA, USA, 1990.

\end{thebibliography} 

\end{document}
