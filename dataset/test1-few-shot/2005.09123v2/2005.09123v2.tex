

\documentclass[11pt,a4paper]{article}
\usepackage[hyperref]{acl2020}
\usepackage{times}
\usepackage{latexsym}
\usepackage{amsmath, amsthm, amssymb, amsfonts}
\usepackage{url}
\usepackage[disable]{todonotes}   

\newcommand{\note}[4][]{\todo[author=#2,color=#3,size=\scriptsize,fancyline,caption={},#1]{#4}} 

\newcommand{\ramon}[2][]{\note[#1]{Ram\'on}{green!40}{#2}}
\newcommand{\manuel}[2][]{\note[#1]{Manuel}{orange!40}{#2}}
\newcommand{\arafat}[2][]{\note[#1]{Arafat}{blue!40}{#2}}
\newcommand{\tahira}[2][]{\note[#1]{Tahira}{yellow!40}{#2}}
\newcommand{\youngsuk}[2][]{\note[#1]{YoungSuk}{red!40}{#2}}
\newcommand{\Ramon}[2][]{\ramon[inline,#1]{#2}\noindent}
\newcommand{\Manuel}[2][]{\manuel[inline,#1]{#2}\noindent}
\newcommand{\Arafat}[2][]{\arafat[inline,#1]{#2}\noindent}
\newcommand{\Tahira}[2][]{\tahira[inline,#1]{#2}\noindent}


\newcommand{\GPTs}{\texttt{GPT-2S}}
\newcommand{\GPTm}{\texttt{GPT-2M}}
\newcommand{\GPTl}{\texttt{GPT-2L}}
\newcommand{\GPT}{\texttt{GPT-2}}

\aclfinalcopy \def\aclpaperid{2002} 



\newcommand\BibTeX{B\textsc{ib}\TeX}

\title{GPT-too: A language-model-first approach for AMR-to-text generation}

\author{\parbox{\linewidth}{\centering
Manuel Mager{\rm\affmark[1]\thanks{~~This research was done during an internship at IBM Research AI.}}~~~~Ram\'on Fernandez Astudillo{\rm\affmark[2]}~~~~Tahira Naseem{\rm\affmark[2]}~~~~Md Arafat Sultan{\rm\affmark[2]}\\Young-Suk Lee{\rm\affmark[2]}~~~~Radu Florian{\rm\affmark[2]}~~~~Salim Roukos{\rm\affmark[2]}} \vspace{.12cm}
\\
\affaddr{\affmark[1] Institute for Natural Language Processing,\\ 
University of Stuttgart, Germany}\\
\affaddr{\affmark[2] IBM Research AI, Yorktown Heights, NY 10598, USA}\\
\affaddr{\texttt{manuel.mager@ims.uni-stuttgart.de}}\\
\affaddr{\texttt{\{ramon.astudillo,arafat.sultan\}@ibm.com}} \\
\affaddr{\texttt{\{tnaseem, ysuklee\}@us.ibm.com}}} 
\newcommand*{\affaddr}[1]{#1} \newcommand*{\affmark}[1][*]{\textsuperscript{#1}}

\date{}

\begin{document}
\maketitle
\begin{abstract}

Abstract Meaning Representations (AMRs) are broad-coverage sentence-level semantic graphs. Existing approaches to generating text from AMR have focused on training sequence-to-sequence or graph-to-sequence models on AMR annotated data only. In this paper, we propose an alternative approach that combines a strong pre-trained language model with cycle consistency-based re-scoring. Despite the simplicity of the approach, our experimental results show these models outperform all previous techniques on the English LDC2017T10 dataset, including the recent use of transformer architectures. In addition to the standard evaluation metrics, we provide human evaluation experiments that further substantiate the strength of our approach.

\end{abstract}

\section{Introduction}
\label{sec:intro}
Abstract Meaning Representation (AMR) \cite{banarescu2013abstract} is a rooted, directed, acyclic graph with labeled edges (relations) and nodes (concepts) expressing ``who is doing what to whom''. AMR-to-text generates sentences representing the semantics underlying an AMR graph. 

Initial works in AMR-to-text used transducers ~\cite{flanigan2016generation}, phrase-based machine translation \cite{pourdamghani2016generating} and neural sequence-to-sequence (\texttt{seq2seq}) models with linearized graphs \cite{konstas2017neural}.
\newcite{cao2019factorising} leverage constituency parsing for generation. \newcite{beck2018graph} improve upon prior RNN graph encoding \cite{song2018graph} with Levi Graph Transformations. \newcite{damonte2019structural} compare multiple representations and find graph encoders to be the best. \newcite{guo2019densely} use RNN graph encoders with dense graph convolutional encoding. \newcite{ribeiro2019enhancing} use RNN encoders with dual graph representations.  Transformer-based \texttt{seq2seq} \cite{vaswani2017attention} was first applied to AMR-to-text in \cite{sinh2019study}. \newcite{zhu2019modeling} greatly improve over the prior state-of-the-art by modifying self-attention to account for AMR graph structure. Using transformers has also been recently explored by \newcite{wang2020amr} who propose a mutli-head graph attention mechanism and by \newcite{Deng2020graph} who propose a graph transformer architecture.

Pre-trained transformer representations \cite{radford2018improving,devlin2019bert,radford2019language} use transfer learning to yield powerful language models that considerably outperform the prior art. They have also shown great success when fine-tuned to particular text generation tasks \cite{see2019massively,zhang2019dialogpt,keskar2019ctrl}. Given their success, it would be desirable to apply pre-trained transformer models to a graph-to-text task like AMR-to-text, but the need for graph encoding precludes in principle that option. Feeding the network with some sequential representation of the graph, such as a topological sorting, looses some of the graphs representational power. Complex graph annotations, such as AMR, also contain many special symbols and special constructs that departure from natural language and may by not interpretable by a pre-trained language model.

In this paper we explore the possibility of directly fine-tuning a pre-trained transformer language model on a sequential representation of AMR graphs, despite the expected difficulties listed above. For this we re-purpose a GPT-2 language model \cite{radford2019language} to yield an AMR-to-text system. We show that it is surprisingly easy to fine-tune GPT-2 to learn AMR graph to text mapping that outperforms the previous state-of-the-art on automatic evaluation metrics. Since a single graph AMR, graph corresponds to multiple sentences with the same meaning, we also provide human evaluation and semantic similarity metric results \cite{bert-score} which are less dependent on reference text. Human evaluation and semantic similarity results highlight the positive impact of a strong language model strategy. Finally we also introduce a simple re-scoring technique based on cycle-consistency that further improves performance. 
 
\section{Fine-tuning GPT-2 for conditional language generation}
\label{sec:Model}

In order to fine-tune a generative model (\GPT;~\newcite{radford2019language}) for conditional text generation, prior works fine-tune the language model to predict target text starting from the additional source text as context. In our experiments, we found it beneficial to fine-tune on the joint distribution of AMR and text instead i.e. also reconstruct the source. Given a tokenized sentence  and the sequential AMR representation  we maximized the joint probability



A special separator token is added to mark the end of the sequential AMR representation. Special AMR symbols that should not be interpreted literally are assigned tokens from the \GPT~unused token list. In addition to this, we also observed that freezing the input embeddings when fine-tuning had positive impact in performance. 

At test time, we provide the AMR as context as in conventional conditional text generation:



\section{Re-scoring via Cycle Consistency}

The general idea of cycle consistency is to assess the quality of a system's output based on how well an external `reverse' system can reconstruct the input from it. In previous works, cycle-consistency based losses have been used as part of the training objective in machine translation \cite{he2016dual} and speech recognition \cite{hori2019cycle}. It has also been used for filtering synthetic training data for question answering \cite{alberti-etal-2019-synthetic}. Here we propose the use of a cycle consistency measure to re-score the system outputs. 

In particular, we take the top  sentences generated by our system from each gold AMR graph and parse them using an off-the-shelf parser to obtain a second AMR graph. We then re-score each sentence using the standard AMR parsing metric Smatch \cite{cai2013smatch} by comparing the gold and parsed AMRs.

\section{Experimental setup}
\label{sec:exp}

Following Previous works on AMR-to-text, we Use the standard LDC2017T10 AMR corpus for evaluation of the proposed model. This Corpus contains 36,521 training instances of AMR graphs in PENMAN notation and the corresponding texts. It also includes 1368 and 1371 development and test instances, respectively. We tokenize each input text using The JAMR toolkit \cite{flanigan2014discriminative}. The concatenation of an AMR graph and the corresponding text is split into words, special symbols and sub-word units using the \GPT~tokenizer. We add all arc labels seen in the training set and the root node \texttt{:root} to the vocabulary of the \GPT model, but we freeze the embedding layer for training. We use the Hugging Face implementation of \cite{wolf2019transformers} for \GPT~small (\GPTs), medium (\GPTm) and large (\GPTl). Fine-tuning converges after  epochs, which takes just a few hours on a V100 GPU\footnote{Code for this paper is available at: \url{https://github.com/IBM/GPT-too-AMR2text}}. For cycle-consistency re-scoring we use an implementation of \newcite{naseem-etal-2019-rewarding} in PyTorch. For re-scoring experiments, we use a beam size of 15. 

\paragraph{AMR input representation.} we test three variants of AMR representation. First, a depth-first search (DFS) through the graph following \newcite{konstas2017neural}, where the input sequence is the path followed in the graph. Second, to see if GPT-2 is in fact learning from the graph structure, we remove all the edges from the DFS, keeping only the concept nodes. This has the effect of removing the relation information between concepts, such as subject/object relations. As a third option, we use the PENMAN representation without any modification. The three input representations are illustrated below:

\vspace{1mm}
\begin{tabular}{c p{5.5cm}}
\small Nodes   & \small \texttt{recommend advocate-01 it vigorous}\\
\small DFS     & \small \texttt{ recommend :ARG1 advocate-01 :ARG1 it :manner vigorous} \\
\small Penman     & \small \texttt{(r / recommend-01
  :ARG1 (a / advocate-01
          :ARG1 (i / it)
          :manner (v / vigorous)))}
\end{tabular}
\vspace{1mm}

\paragraph{Decoding.} 
For generation, we experiment with greedy decoding, beam search, and nucleus sampling \cite{holtzman2019curious}. For beam search, we explore beam sizes of ,  and . As the system, in some cases, produces repetitive output at the end of the text, we additionally perform a post-processing step to remove these occurrences. 

\paragraph{Metrics.} We considered the three automatic evaluation metrics commonly used in previous works. We compute BLEU \cite{papineni2002bleu} using  SacreBLEU \cite{ma2019results}. We compute chrF++ \cite{popovic2017chrf++} using both SacreBLEU and the scripts used by authors of the baseline systems. We compute METEOR \cite{banerjee2005meteor} with the default values for English of the CMU implementation.\footnote{\url{https://www.cs.cmu.edu/~alavie/METEOR}}


In addition to the standard automatic metrics, we also carry out human evaluation experiments and use the semantic similarity metric BERTScore \cite{bert-score}. Both metrics arguably have less dependency on the surface symbols of the reference text used for evaluation. This is particularly relevant for the AMR-to-text task, since one single AMR graph corresponds to multiple sentences with the same semantic meaning. Conventional metrics for AMR-to-text are are strongly influenced by surface symbols and thus do not capture well the ability of the system to produce a diverse sentences with same underlying semantics.

Human evaluations are carried out by three professional annotators on  randomly selected sentences from the  test sentences, on a 6 point scale, ranging from 0 to 5.
\begin{itemize}
\item {\small 0=Exceptionally poor (No useful information is conveyed at all.)}
\item {\small 1=Poor (Fundamental errors in grammar and vocabulary make it difficult to understand the meaning.)}
\item {\small 2=Not good enough (Errors in grammar, vocabulary and style make it difficult to understand the meaning.)}
\item {\small 3=Good enough (There are errors in the text, but I am reasonably confident that I understand the meaning.)}
\item {\small 4=Very good (There may be minor errors in the text, but I am very confident that I understand the meaning.)}
\item {\small 5=Excellent (The information is presented clearly and with appropriate grammar, vocabulary and style.)}
\end{itemize}
For each system, scores from all annotators are averaged to compute a single score. Inter-annotator agreement was  when measured by Pearson correlation coefficient.




Our system produces de-tokenized cased output after BPE decoding, whereas previous systems produce traditional tokenized lower-cased output. Therefore, we lowercase and tokenize our system outputs to have fair comparisons with previous systems.


\begin{table}[t]
    \centering
    \fontsize{10pt}{12pt}\selectfont
    \setlength{\tabcolsep}{2.0pt}
    \begin{tabular}{c|c||r|r}
 Model   & Input                          &BLEU & chrF++ \\ \hline \hline
\GPTs~Rec.& Only nodes AMR 	             &9.45 	& 41.59 \\
\GPTs~Rec.& Lin. AMR w/o edges.  &11.35 & 43.25 \\
\GPTs~Rec.& Lin. AMR w/edges. 	 &20.14 & 53.12 \\
\GPTs~Rec.& Penman AMR 	                 &22.37 & 53.92 \\ \hline
\GPTm~Rec.& Lin. AMR w/edges.  &22.86 & 55.04 \\
\GPTm~Rec.& Penman AMR 	 &27.99 & 61.26
    \end{tabular}
    \caption{Results on the LDC2017T10 development set using GPT-2 S(mall) and M(edium) with Rec(onstruction) loss (see \S2) for different AMR representations (see \S4).}
    
    \label{tab:representation_results}
\end{table}{}

\begin{table}[t]
    \centering
    \fontsize{10pt}{12pt}\selectfont
    \setlength{\tabcolsep}{2.0pt}
    \begin{tabular}{c|c||r|r}
Approach  & Decoding & BLEU & chrF++ \\ \hline \hline  
\GPTm~Conditional         &Greedy   & 25.73 & 57.2 \\\hline
\GPTm~Rec.              &Greedy   & 30.41 & 61.36 \\
\GPTm~Rec.              &BEAM     & 31.8 	& 62.56 \\
\GPTm~Rec.              &BEAM 10  & \textbf{32.32} & 62.79 \\
\GPTm~Rec.              &Sampling & 28.75 & 61.19
    \end{tabular}
    \caption{Results on the LDC2017T10 development set. Rec(onstruction) uses the AMR reconstruction term (see \S\ref{sec:Model}) whereas Conditional does not.}
    \label{tab:decoding_res}
\end{table}{}

\subsection{Results}
\label{subsec:results}

\begin{table}[h]
    \centering
    \fontsize{10pt}{12pt}\selectfont
    \setlength{\tabcolsep}{2.0pt}
    \begin{tabular}{c| c c c }
        System &  \multicolumn{3}{c}{Performance}\\\hline
               &  BLEU & Meteor & chrF++ \\\hline \hline
        \newcite{beck2018graph}         & 23.30     & -     & 50.40 \\\newcite{damonte2019structural} & 24.54     & 24.07 & - \\
        \newcite{guo2019densely}        & 27.60     & -     & 57.30 \\\newcite{cao2019factorising}    & 26.80     & -     & -  \\
        \newcite{sinh2019study}         & 18.36     & -     & - \\
        \newcite{ribeiro2019enhancing}  & 27.87     & 33.21 & - \\
        \newcite{Deng2020graph}         & 29.80     & 35.10 & 59.4  \\
        \newcite{zhu2019modeling}       & 31.82     & 36.38 & \textbf{64.05} \\\GPTm~Rec.                 &       &  &  \\\GPTl~Rec.                 &       &   & \\\GPTm~Rec. re-scoring       &      &  &  \\\GPTl~Rec. re-scoring       & \bf 33.02 & \bf 37.68 & \bf 63.89 \\\end{tabular}
    \caption{Results on the LDC2017T10 test set for best performing models compared to other results reported in the literature.  indicates statistical significance at ,  at  and , not significant. All significance tests are with respect to \citep{zhu2019modeling}.}
    \label{tab:compared_results}
\end{table}

Regarding the type of AMR representation, as shown in Table \ref{tab:representation_results}, using directly the PENMAN notation for AMR representation leads to the best results outperforming DFS. Edge information, indicating relations between concepts, seems also to play a fundamental role since its absence strongly decreases performance in both DFS and PENMAN representations. Penman notation was chosen for the rest of the experiments. 

The impact of the use of a reconstruction term explained in \S2 is shown in Table \ref{tab:decoding_res}. The model trained using this additional term achieves  BLEU and  chrF++, as opposed to  BLEU and  chrF++ without the term. We therefore use a reconstruction term training in the rest of the experiments.

Beam search improves system performance greatly over the greedy baseline with  BLEU points (see Table \ref{tab:decoding_res}). With beam size , we obtain  BLEU and  chrF++. With nucleus sampling at a cumulative probability mass of , performance drops to  BLEU and  chrF++. Finally, cycle-consistency re-ranking of the beam search outputs improves performance ( BLEU,  chrF++) over the one best output. 

\begin{table}[]
    \centering
    \fontsize{10pt}{12pt}\selectfont
    \setlength{\tabcolsep}{3.0pt}
    \begin{tabular}{c|c c |c }
        System & \multicolumn{3}{c}{LDC2017T10}\\\hline
               & \multicolumn{2}{c|}{Human Eval.} & SemSim \\\hline 
               & Avg. & P45 & F1 \\\hline\hline
               
        \newcite{guo2019densely}       &  & 15.69\% & 92.68  \\
        \newcite{ribeiro2019enhancing} &  & 16.37\% & 92.63  \\
        \newcite{zhu2019modeling}      &  & 20.26\% & 93.31  \\\hline
        \GPTm~Rec.                   &   & 37.91\% & 94.55 \\
        \GPTl~Rec.                   & \bf 3.04 & \bf 41.83\% & \bf 94.63  \\
    \end{tabular}
    \caption{Human evaluation and semantic similarity (SemSim) results on the LDC2017T10 test set. Human evaluations (Human Eval.) show the average (Avg.) of scores (0 to 5) and the ratio of sentence evaluated between 4 and 5 (P45). All results for human evaluation are on  randomly selected sentences and statistically significant at . SemSim results are significant at . All significance tests refer to a comparison with \cite{zhu2019modeling}.}
    \label{tab:huaman_eval}
\end{table}

\begin{table*}[ht!]
    \centering
    \fontsize{10pt}{12pt}\selectfont
    \setlength{\tabcolsep}{2.5pt}
    \begin{tabular}{c r p{13cm}}

         \bf  & \bf System    & \bf Generated text\\\hline \hline
         (1) & \bf REF:    & the doctors gave her medication and it 's made her much better .\\
             & \bf G2S:    & the doctor gives her medications and they make her much better .\\ 
             & \bf Transf: & doctors give her medications and make her much better .\\
             & \bf Our:     & the doctor gave her the medication and made her feel much better. \\
             & \bf Our R.:& the doctor gave her the medication and made her " much better " .\\ \hline

(2) & \bf REF: & at the state scientific center of applied microbiology there is every kind of deadly bacteria that was studied for use in the secret biological weapons program of the soviet union . \\
& \bf G2S: & there are every kind of killing \textless unk\textgreater~in the state scientific center of applied microbiology to use themselves for soviet union 's secret biological weapons programs .  \\
& \bf Transf:  & there is every kind of bacterium , which is studied in using bacterium for the soviet union secret biological weapons program . \\
& \bf Our: & every kind of bacterium that was studied was found at the state scientific center of applied microbiology and was used in soviet secret weapons programs for biological weapons of biology . \\
& \bf Our R.: & every kind of bacterium that has been studied and used in soviet secret programs for biological weapons has been in the state scientific center of applied microbiology . \\\hline

         (3) & \bf REF:    & among the nations that have not signed the treaty only india and israel would qualify for admission to the nsg under the israeli proposal .\\
             & \bf G2S: & only one of the nations who do not sign the treaty are qualified for their proposal to admit the nsg .\\
             & \bf Transf:    & india and israel are only qualified for the nations that do not sign the treaty , but they admitted to the nsg .\\
             & \bf Our:    & india and israel are the only countries eligible to admit to the nsg by proposing a treaty .\\
             & \bf Our R.: & only india and israel are eligible to admit to the nsg by proposing a treaty .\\\hline
    \end{tabular}
    \caption{Output examples from four systems of the LDC2017T10 dataset. REF stands for reference, G2S for \citep{guo2019densely} and Transf. for \citep{zhu2019modeling}. Our is the top beam output for \GPTl~and Our R. is with re-scoring.}
    \label{tab:output_examples}
\end{table*}{}

Table \ref{tab:compared_results} compares the best \GPTm~and \GPTl~results, fine-tuned using the reconstruction term and PENMAN notation. For all scores we test statistical significance with a standard two-tailed student t-test. Our model achieves a large improvement of  BLEU and  METEOR scores over the previous state-of-the-art model using \GPTl~and re-scoring. For chrF++, we get different scores from SacreBLEU and the scripts provided by the authors of our baseline systems, achieving comparable results with the former (), and improving over the best score with the latter () . 

Table \ref{tab:huaman_eval} shows human Evaluation results and semantic similarity scores of \GPTl~and \GPTm~compared to \cite{zhu2019modeling,ribeiro2019enhancing,guo2019densely}. Our approach produces a large number of high-quality sentences with , a significant gain over the previous best system (). Regarding semantic similarity, prior art methods show relatively close scores, a  points difference, while \GPTl~Rec. improves  points over the best of these models. It should be noted that differences with \cite{zhu2019modeling} for \GPTl~Rec. are statistically significantly with , while differences for \GPTm~Rec are not significant due to the small sample size. 

In Table \ref{tab:output_examples} we show three nontrivial examples, where we compare our system outputs with those of previous work. In the first example, the reference sentence contains a grammatical error. Our system not only generates the correct output, but also corrects the error in the reference. The proposed system can generate fluent long sentences as shown in example 2. The third example shows a sentence where all systems including ours fail to generate a correct text. 

\subsection{Discussion}
\label{subsec:discussion}

Due to the large amounts of data they are trained on, pre-trained transformer language models can be expected to generate fluent and diverse text \cite{see2019massively}. It should however be highlighted that fine-tuned \GPT~learns to produce not only fluent but also adequate text, despite using a sequential representation of an AMR graph as input. As shown in the experimental setup, encoding of relations plays as well a fundamental role in AMR-to-text performance, indicating that \GPT~attains a fine-grained understanding of the underlying semantics to reach state of the art performance.

While a sequence of PENMAN notation tokens is far from an optimal encoding of a graph, it is noteworthy how far performance-wise current strong language models can go. Furthermore, It is likely that standard metrics (BLEU, Meteor, chrF++) that rely on a reference text do not properly reflect AMR-to-text quality. An AMR graph corresponds to multiple sentences with the same semantics and these measures are likely biased towards the single available reference. In metrics that are less influenced by the reference text such as human evaluation and semantic similarity, the proposed system shows a larger improvement over the previous systems with close to  of the generated sentences considered excellent or good. 

Finally it is worth considering that leveraging pre-trained transformers greatly expands the vocabulary available on AMR-to-text systems. A single AMR graph can correspond to multiple sentences with markedly different surface realizations, but manual annotation of AMR is a time consuming task. Approaches like the one proposed may be a simple solution for generation of diverse text data for AMR parser training or other applications were diversity play a role.

\section{Conclusions}

In this work, we present a language model-based approach for the AMR-to-text generation task. We show that a strong pre-trained transformer language model (\GPT) can be fine-tuned to generate text directly from the PENMAN notation of an AMR graph. Comparison with state-of-the-art models in BLUE, chrF++, METEOR as well as SemSim and human evaluation metrics show that while simple, this approach can outperform existing methods including methods training transformers from scratch. We also show that cycle consistency-based re-scoring using a conventional AMR parser and the Smatch metric can notably improve the results. Future work will focus on incorporating better encoding of the AMR graph into the current system and exploring data augmentation techniques leveraging the proposed approach.

\section*{Acknowledgments}

We thank the reviewers for their valuable suggestions. We would also like to thank Chunchuan Lyu for his valuable feedback and help.

\bibliography{acl2020}
\bibliographystyle{acl_natbib}

\end{document}
