The atmospheric modeling code COSMO (COnsortium for Small-scale MOdeling) was chosen for tests to validate the power measurement capabilities. COSMO is a non-hydrostatic limited area model used for both weather forecasting by national weather services and for climate research.

The production code for COSMO is based on Fortran 90 with a flat MPI parallelization model, optimized for NEC vector architectures.
A Swiss initiative, lead by the Swiss weather service MeteoSwiss, has ported and optimized the entire COSMO code for both multi-core CPU and GPU architectures using CUDA and OpenACC for different parts of the code~\cite{cug2013,gysi2014,cumming2014,lapillonne2014}.
The new GPU implementation is mature, it currently used for production climate simulations on \daint, so it is suitable for comparing both time and energy to solution across different architectures.

In this section we will discuss the configuration for the COSMO simulation that was used for the power measurement tests, followed by a description of the steps taken to ensure that results measured using the PMDB on XC-30 were comparable to those measured with external power meters on other systems. Then a comparison of time and energy to solution across different Cray systems is presented.

\subsubsection{Simulation configuration:}
The configuration used in our tests is based on the 2-km model of the Swiss Alps, known as COSMO-2, used by MeteoSwiss.
A COSMO-2 run uses 9 nodes on each system, 8 nodes for computation, and an additional node for IO.
This model is currently used for daily weather forecasting runs, and it will be used in ensemble mode from 2015.
An ensemble test runs in the order of 20 instances the same simulation (ensemble members), with perturbations to parameters, to better quantify the uncertainty for forecasts.

The ensemble configuration was used for these tests because it allowed us to run enough instances of the 9-node job to fill a cabinet.
It was necessary to fill a cabinet with jobs to compare different systems, because while the XC-30 make it possible to measure power and energy on a node-by-node basis, we were restricted to measuring the power consumption of a whole cabinet with external power meters on the other systems.





\subsubsection{Measurement and systems overview:}
The measurements for time and energy to solution were performed on three systems at CSCS (\rosa, \daint and \todi), and a hybrid XC-30 system named \clogin that we had early access to at Chippewa Falls.
The number of nodes per cabinet on the XE6 and XK7 systems is half of that on the XC-30 systems, so on each system we used as many members in each ensemble run required to fill a full cabinet.

The difference systems used in the tests are summarised below:
\begin{itemize}
\item
    \textbf{\rosa} and \textbf{\todi} are Cray XE6 and XK7 systems respectively, with to 96 compute nodes per rack. Each XE6 node has two by 16-core AMD Interlagos sockets, and the XK7 replaces a socket with a K20X GPU. Each cabinet has a blower integrated in the base. Power measurements were performed at CSCS with an external power meter, which measured total power for one cabinet, including the blower, at a 1Hz frequency.
\item
    \textbf{\daint} and \textbf{\clogin} were Cray XC-30 systems with 192 nodes per cabinet (on \daint 4 of the nodes are service nodes, so one less ensemble member could be run than on \clogin). \daint had two Sandy Bridge sockets with eight cores each per node (this was before \daint's hybrid upgrade), while each node of \clogin had one Sandy Bridge socket with one K20X GPU.
\end{itemize}

In our tests, time and energy to solution were for total time to solution, i.e. measured directly before and after the \textit{aprun} command.
We were interested in the total energy to solution, including the node, network, blower and AC/DC losses in all measurements, which is precisely what is measured by the external power meters on the XE6 and XK7 systems.
The cabinet level readings from the PMDB did not include the blowers on XC-30, so the value for the blowers was approximated according to formula~\eq{eq:energyEstimateWithBlower}, before scaling for AC/DC conversion.

On each system the ensemble runs were performed multiple times to determine the number of runs required to obtain an accurate average results. Both time to solution and energy to solution were highly reproducible with very small variance, so that only two runs were required on each system to have a 99\% confidence level for the mean of the samples to be accurate to 1\%.

\subsection{Results}

Before using results from the PMDB to compare energy to solution on the XC-30 against the other systems, we first validated that the results from the PMDB could be used to derive measurements equivalent to those from an external power meter.
The \clogin system had two external power meters: the first measured power for one cabinet; and the second measured power ``at the wall'' for the whole system, i.e. for all three cabinets and two blowers.

\begin{table}[hp!]
\centering
\begin{tabular}{|l|rrr|}
            \cline{2-4}
\multicolumn{1}{c|}{} & PMDB (kWh)& external (kWh)& efficiency (PMDB/external)\\
            \hline
run 1                 & 53.63         & 56.45         & 95.00\%    \\
run 2                 & 53.47         & 56.27         & 95.02\%    \\
            \hline
\end{tabular}
\vspace{5pt}
\caption{ETS measured using PMDB and external meter for a three-cabinet hybrid XC-30 system \clogin. The energy is for all three cabinets with the blowers, with the blower power assumed to be \watt{4440} for the PMDB results (see \tbl{tab:blowers}).}
\label{tbl:cosmoValidate}
\end{table}

An ensemble run large enough to fill all three cabinets was run two times, and results from both power meters were compared to those from the PMDB.
The PMDB agreed very well with the power meter for the entire system with the assumed 95\% efficiency in the conversion from AC to DC, as shown in the results in \tbl{tbl:cosmoValidate}.
The power meter on the single cabinet disagreed significantly, by around 5\% with the PMDB, however further investigation showed that this meter required recalibration, and that the PMDB results were correct.

\begin{table}[htp!]
\centering
\begin{tabular}{|l|rrcc|}
\hline
System                  & XE6       & XK7       & XC-30      & hybrid XC-30   \\
Name                    & \rosa     & \todi     & \daint    & \clogin       \\
Configuration           & CPU       & \textbf{GPU} & CPU    & \textbf{GPU}  \\
\hline
Ensemble Size           & 10        & 10        & 20        & 21            \\
       \rowcolor{lightred}
Wall Time (s)           & 3683      & 2579      & 2083      & 1539          \\
Mean Power (kW)         & 40.22     & 62.07     & 28.27     & 41.56         \\
Peak Power (kW)         & 43.00     & 64.96     & 31.55     & 43.97         \\
Energy to solution (kWh)& 41.14     & 44.47     & 16.36     & 17.77         \\
       \rowcolor{lightblue}
Energy per member (kWh) & 4.11      & 2.22      & 1.636     & 0.85          \\
\hline
       \rowcolor{lightred}
Speedup time to solution& 1.0       & 1.42      & 1.76      & 2.39          \\
       \rowcolor{lightblue}
Speedup energy to solution & 1.0    & 1.85      & 2.51      & 4.84          \\
\hline
\end{tabular}
\vspace{5pt}
\caption{Results of time to solution and energy to solution for COSMO-2 ensemble runs on different Cray systems.}
\label{tbl:cosmoResults}
\end{table}

\tbl{tbl:cosmoResults} shows the result for all of the Cray systems. There are two clear trends in terms of architecture. The first is that moving from one generation of system to another, from XE6 to XC-30 and from XK7 to hybrid-XC-30, improves both TTS and ETS in all cases. This makes sense, due to improvements including CPUs (Interlagos to Sandy Bridge) and network (Gemini to Aires). The second trend is that TTS and ETS are also better on GPU-based systems. It is interesting to note that the ETS improvements are greater than TTS improvements in all cases: indeed energy to solution per ensemble member is a factor of 4.8, while TTS improves by a factor of 2.4, going from XE6 to hybrid XC-30.

The original aim of this investigation was to use external power meters for all systems, to provide comprehensive energy results.
We had the opportunity to validate the PMDB counters with early access to a hybrid XC-30 system, and the results from those experiments, presented above gave us confidence that the built in power monitoring features can be used to accurately measure the true energy to solution for applications.
Measurements with the PMDB were certainly much easier than using external meters, and the PMDB has become the go to tool for follow on work with COSMO~\cite{cumming2014}.
