\documentclass[11pt]{elsarticle}
\usepackage{enumerate}
\usepackage{hyperref}
\usepackage{amsmath,amssymb,amsthm}
\usepackage{tikz}
\usepackage{subcaption}

\usepackage{algorithm}\usepackage{algpseudocode}
\usepackage[margin=1.3in]{geometry}
\usepackage{todonotes}

\hypersetup{
    pdftitle = {A Single-Exponential Fixed-Parameter Tractable Algorithm for Distance-Hereditary Vertex Deletion},
   pdfauthor=  {Eduard Eiben and Robert Ganian and O-joung Kwon}
   }

\newcommand\abs[1]{\lvert #1\rvert}
\newtheorem{theorem}{Theorem}[section]
\newtheorem{lemma}[theorem]{Lemma}
\newtheorem{corollary}[theorem]{Corollary}
\newtheorem{PROP}[theorem]{Proposition}
\newtheorem{observation}[theorem]{Observation}

\newtheorem{claim}{Claim}
\newenvironment{clproof}{\begin{list}{}{\setlength{\leftmargin}{5mm}} \item {\it Proof.} }{\hfill\end{list}\medskip}
              
\newtheorem{definition}{Definition}
\newtheorem{RRULE}{Reduction Rule}
\newtheorem{BRULE}{Branching Rule}
\newcommand{\YES}{\textsc{Yes}}
\newcommand{\NO}{\textsc{No}}

\newcommand{\cc}{\operatorname{cc}}
\newcommand{\twinclass}{\textbf{tc}(G-S)}
\newcommand{\DHVD}{\textsc{Distance-Hereditary Vertex Deletion}}
\newcommand{\disjointDHVD}{\textsc{Disjoint Distance-Hereditary Vertex Deletion}}
\newcommand{\bad}{a bypassing vertex}
\newcommand{\comp}{\operatorname{comp}}
\newcommand{\btwn}{\operatorname{path}}
\newcommand{\bigoh}{\mathcal{O}}




\begin{document}
\title{A Single-Exponential Fixed-Parameter Algorithm for Distance-Hereditary Vertex Deletion}
\author[Vienna]{Eduard Eiben}
\ead{eduard.eiben@gmail.com}
\author[Vienna]{Robert Ganian}
\ead{rganian@gmail.com}
\author[Kwon]{O-joung Kwon}
\ead{ojoungkwon@gmail.com}
\address[Vienna]{Algorithms and Complexity Group, TU Wien, Vienna, Austria}
\address[Kwon]{Logic and Semantics, Technische Universit\"at Berlin, Germany}
\date{\today}
\begin{abstract}
Vertex deletion problems ask whether it is possible to delete at most  vertices from a graph so that the resulting graph belongs to a specified graph class. Over the past years, the parameterized complexity of vertex deletion to a plethora of graph classes has been systematically researched. Here we present the first single-exponential fixed-parameter tractable algorithm for vertex deletion to distance-hereditary graphs, a well-studied graph class which is particularly important in the context of vertex deletion due to its connection to the graph parameter rank-width. We complement our result with matching asymptotic lower bounds based on the exponential time hypothesis. As an application of our algorithm, we show that a vertex deletion set to distance-hereditary graphs can be used as a parameter which allows single-exponential fixed-parameter tractable algorithms for classical NP-hard problems.
\end{abstract}
\begin{keyword}
 distance-hereditary graphs, fixed-parameter algorithms, rank-width
\end{keyword}


\maketitle




\section{Introduction}\label{sec:introduction}
Vertex deletion problems include some of the best studied NP-hard problems in theoretical computer science, including \textsc{Vertex Cover} or \textsc{Feedback Vertex Set}. In general, the problem asks whether it is possible to delete at most  vertices from a graph so that the resulting graph belongs to a specified graph class. While these problems are studied in a variety of contexts, they are of special importance for the parameterized complexity paradigm~\cite{DowneyF13,CyganFKLMPPS15}, which measures the performance of algorithms not only with respect to the input size but also with respect to an additional numerical parameter. The notion of vertex deletion allows a highly natural choice of the parameter (specifically, ), especially for problems where the solution size is not defined or cannot be used. Many vertex deletion problems are known to admit so-called \emph{single-exponential fixed-parameter tractable (FPT) algorithms}, which are algorithms running in time  for input size  and some constant .



Over the past years, the parameterized complexity of vertex deletion to a plethora of graph classes has been systematically researched, 
and in particular, if the target class admits efficient algorithms for many NP-hard problems, then such a class get more attention.
For this reason, classes of graphs of constant treewidth have been studied in detail, 
and Fomin et al.~\cite{FominLMS12} and Kim et al.~\cite{KLPRRSS13} showed that the corresponding \textsc{Treewidth- Vertex Deletion}\footnote{\textsc{Treewidth- Vertex Deletion} asks whether it is possible to delete  vertices so that the resulting graph has treewidth at most .}
problem is solvable in single-exponential FPT time.
Interestingly, this problem is a special case of general \textsc{Planar -Deletion} problems, which ask whether one can hit all of minor models of graphs in  by at most  vertices, when  contains at least one planar graph. 
The condition that  contains a planar graph is essential because it tells that the outside of any solution should have bounded treewidth, by the grid-minor theorem~\cite{RobertsonS1986}.
Several authors~\cite{FominLMS12,KLPRRSS13} have used this fact to design single-exponential FPT algorithms.

The successful development of single-exponential FPT algorithms for \textsc{Treewidth- Vertex Deletion} motivates us to study \textsc{Rank-width -Vertex Deletion}, which is analogous to \textsc{Treewidth- Vertex Deletion} but replaces treewidth with rank-width.
Rank-width~\cite{OS2004, Oum05} is a graph parameter introduced for generalizing graph classes of bounded treewidth into dense graph classes; for example,
complete graphs have unbounded treewidth but rank-width .
Generally, classes of graphs of bounded rank-width capture the graphs that can be recursively decomposable along vertex bipartitions  where the number of distinct neighborhood types from one side to the other is bounded. 
Courcelle, Makowski, and Rotics~\cite{CourcelleMR2000} proved that every MSO-expressible problem can be solved in
polynomial time on graphs of bounded rank-width (see also the work of Ganian and Hlin\v en\' y~\cite{GanianH10}).

Kant\'e et al.~\cite{KanteKKP2015} observed that \textsc{Rank-width -Vertex Deletion} is fixed parameter tractable using the general framework of Courcelle, Makowski, and Rotics.
However, this algorithm does not provide any reasonable function for . Thus Kant\'e et al. naturally asked whether it is solvable in reasonably better running time.
For instance, it is actually open whether \textsc{Rank-width -Vertex Deletion} can even be solved in time , where  is the size of the deletion set.

In this paper, we focus on graphs of rank-width at most , which are \emph{distance-hereditary graphs}.
Distance-hereditary graphs were introduced by Howorka~\cite{howorka77} in 1977, long before the discovery of rank-width~\cite{OS2004} and the observation by Oum~\cite{Oum05} that the class of graphs of rank-width at most  are precisely distance-hereditary graphs.
Bandelt and Mulder~\cite{BM1986} found all the minimal induced subgraph obstructions for distance-hereditary graphs. 
Distance-hereditary graphs are naturally related to split decompositions, where they are exactly the graphs that are completely decomposable into stars and complete graphs~\cite{Bouchet1988a}.
We explain these structural properties in more detail in Section~\ref{sec:prelim}.
This structure has led to the development of a number of algorithms for distance-hereditary graphs~\cite{CogisT2005, HungC2005, HsiehHH2006, RuoM2007, MarcD2007, NakanoUU2007, GassnerH2008}.
Given the above, we view the vertex deletion problem for distance-hereditary graphs as a first step towards handling \textsc{Rank-width- Vertex Deletion}.




\subsection*{Our Contribution.}
A graph  is called \emph{distance-hereditary} if for every connected induced subgraph  of  and every , 
the distance between  and  in  is the same as the distance between  and  in . 
We study the following problem. 

  
\medskip
\noindent
\fbox{\parbox{0.97\textwidth}{
\DHVD \\
\emph{Instance:} A graph  and an integer . \\
\emph{Parameter:} . \\
\emph{Task:} Is there a vertex set  with  such that  is distance-hereditary? }}
\vskip 0.2cm

The main result of this paper is a single-exponential FPT algorithm for \DHVD.

\begin{theorem}
\DHVD\ can be solved in time .
\end{theorem}

We note that this solves an open problem of Kant\' e, Kim, Kwon, and Paul~\cite{KanteKKP2015}. The core of our approach exploits two distinct characterizations of distance-hereditary graphs: one by forbidden induced subgraphs (obstructions), and the other by admitting a special kind of split decomposition~\cite{Cunningham1982}. 


The algorithm can be conceptually divided into three parts. 

\begin{enumerate}
\item \textbf{Iterative Compression}. This technique allows us to reduce the problem to the easier \textsc{Disjoint Distance-Hereditary Vertex Deletion}, where we assume that the instance additionally contains a certain form of advice to aid us in our computation. Specifically, this advice is a vertex deletion set  to distance-hereditary graphs which is disjoint from and slightly larger than the desired solution.
  
  \item \textbf{Branching Rules}. We exhaustively apply two branching rules to simplify the given instance of \textsc{Disjoint Distance-Hereditary Vertex Deletion}. At a high level, these branching rules allow us to assume that the resulting instance contains no small obstructions and furthermore that certain connectivity conditions hold on .
  
  \item \textbf{Simplification of Split Decomposition}. We compute the split decomposition of  and exploit the properties of our instance  guaranteed by branching to prune the decomposition. In particular, we show that the connectivity conditions and non-existence of small obstructions mean that  must interact with the split decomposition of  in a special way, and this allows us to identify irrelevant vertices in . This is by far the most technically challenging part of the algorithm.
\end{enumerate}

A more detailed explanation of our algorithm is provided in Section~\ref{sec:general}, after the definition of required notions. We complement this result with an algorithmic lower bound which rules out a subexponential FPT algorithm for \textsc{Distance-Hereditary Vertex Deletion} under well-established complexity assumptions. We also note that the naive approach of simply hitting all known ``obstructions'' (i.e., forbidden induced subgraphs) for distance-hereditary graphs does not lead to an FPT algorithm. Indeed, the set of induced subgraph obstructions for distance-hereditary graphs includes induced cycles of length at least .
Heggernes et al.~\cite{Heggernes2013} showed that 
the problem asking whether it is possible to delete  vertices so that the resulting graph has no induced cycles of length at least  is W[2]-hard.
Therefore, unless the Exponential Time Hypothesis fails, one cannot obtain a single-exponential FPT algorithm for \DHVD\ by simply finding and hitting all forbidden induced subgraphs for the class.

The paper is organized as follows. Section~\ref{sec:prelim} contains the necessary preliminaries and notions required for our results. In Section~\ref{sec:general}, we set the stage for the process of simplifying the split decomposition, which entails the definition of \textsc{Disjoint Distance-Hereditary Vertex Deletion}, introduction of our branching rules, and a few technical lemmas which will be useful throughout the later sections.
Section~\ref{sec:rules} then introduces and proves the safeness of five polynomial-time reduction rules; crucially, the exhaustive application of these rules guarantees that the resulting instance will have a certain ``inseparability'' property. Using this structural result, we prove that one of reduction rules is applicable until the remaining instance is trivial. Finally, the proof of our main result as well as the corresponding lower bound are presented in Section~\ref{sec:completing}. Section~\ref{sec:completing} also illustrates one potential application of our result: we show that a vertex deletion set to distance-hereditary graphs can be used as a parameter which allows single-exponential FPT algorithms for classical NP-hard problems.

\section{Preliminaries}\label{sec:prelim}

All graphs in this paper are simple and undirected.
For a graph , let  and  denote the vertex set and the edge set of , respectively. 
For , let  denote the subgraph of  induced by . For  and , let  be the graph obtained from  by removing , and let  be the graph obtained by removing all vertices in . 
For , let  denote the graph obtained from  by removing all edges in .
For , the set of neighbors of  in  is denoted by .
For , let  denote the set of vertices in  that have a neighbor in .
We denote by  the number of connected components of .
An edge  of a connected graph  is a \emph{cut edge} if the graph obtained from  by removing  is disconnected.

The \emph{length} of a path is the number of edges on the path.
For  and a subgraph  of , we say  is adjacent to  if it has a neighbor in .
A \emph{star} is a tree with a distinguished vertex, called the \emph{center}, adjacent to all other vertices. A \emph{complete graph} is a graph with all possible edges.

Two vertices  and  in a graph  are called \emph{twins} if they have the same set of neighbors in . 
For two vertex sets  and , we say that 
\begin{itemize}
\item  is \emph{complete} to  if for every , ,  is adjacent to , 
\item  is \emph{anti-complete} to  if for every , ,  is not adjacent to .
\end{itemize}

In parameterized complexity, an instance of a parameterized problem consists in a pair , where  is a secondary measurement, called the \emph{parameter}. 
A parameterized problem  is \emph{fixed-parameter tractable} (\emph{FPT}) if there is an algorithm which decides whether  belongs to  in time  for some computable function . 


\subsection{Distance-Hereditary Graphs}
A graph  is called \emph{distance-hereditary} if for every connected induced subgraph  of  and every , 
the distance between  and  in  is the same as the distance between  and  in . 
For instance, the induced cycle  is not distance-hereditary, because the distance from  to  is , 
but if we take an induced subgraph on , then the distance becomes .
This graph class was first introduced by Howorka~\cite{howorka77}, and deeply studied by Bandelt and Mulder~\cite{BM1986}.
There are several other, equivalent characterizations of distance-hereditary graphs. One of the most prominent ones links it to the structural parameter \emph{rank-width}~\cite{Oum05}; specifically, distance-hereditary graphs are precisely the graphs of rank-width ~\cite{Oum05}.
However, in this paper we will exploit two other characterizations of the graph class: one by forbidden induced subgraphs (given below), and one via split decompositions (given in the following subsection).

The house, the gem, and the domino graphs are depicted in Figure~\ref{fig:obsdh}.
A graph isomorphic to one of the house, the gem, the domino, and induced cycles of length at least  will be called a \emph{distance-hereditary obstruction} or shortly a \emph{DH obstruction}.
A DH obstruction with at most  vertices will be called a \emph{small DH obstruction}.
Note that every DH obstruction does not contain any twins.

\begin{figure}[t]
\centerline{\includegraphics[scale=1]{figure11.pdf} \quad\quad
\includegraphics[scale=1]{figure12.pdf} \quad\quad
\includegraphics[scale=1]{figure13.pdf} }
\caption{Small DH obstructions which are not cycles.}
\label{fig:obsdh}
\end{figure}

\begin{theorem}[Bandelt and Mulder~\cite{BM1986}]
A graph is distance-hereditary if and only if it contains no DH obstructions as induced subgraphs.
\end{theorem}
We state an observation which will be useful later on.

\begin{observation}
\label{obs:sub}
For any DH obstruction  and any edge  in , it holds that the graph  obtained by subdividing  also contains a DH obstruction as an induced subgraph.
\end{observation}
The following lemma will be used to find DH obstructions later on.

\begin{lemma}[Kant\`e, Kim, Kwon, and Paul, Lemma 4.3 of \cite{KanteKKP2015}]\label{lem:dhobs}
Let  be a graph obtained from an induced path of length at least  by adding a vertex  adjacent to its end vertices where
 may be adjacent to some internal vertices of the path.
Then  has a DH obstruction containing .
In particular, if the given path has length at most , then  has a small DH obstruction containing .
\end{lemma}
\begin{proof}
The first statement was shown in Lemma 4.3 of \cite{KanteKKP2015}. If the given path has length at most , then  has at most  vertices, and thus  contains a small DH obstruction containing .
\end{proof}

\subsection{Split decompositions}

We follow the notations used by Bouchet~\cite{Bouchet1988a}.
A \emph{split} of a connected graph  is a vertex partition  of  such that , and  is complete to . 
See Figure~\ref{fig:examplesplit} for an example.
Splits are also called \emph{1-joins}, or simply \emph{joins}~\cite{GSH1989}. 
A connected graph  is called a \emph{prime graph} if  and it has no split.



\begin{figure}[t]
\centerline{\includegraphics[scale=0.42]{examplesplit}\qquad\qquad  \includegraphics[scale=0.42]{examplesplit2}}
\caption{An example of a split  of a graph. Its simple decomposition is presented in the second picture, where the red edge is the newly introduced marked edge.}
\label{fig:examplesplit}
\end{figure}


A connected graph  with a distinguished set of edges  is called a \emph{marked graph} if
the edges in  form a matching and each edge in  is a cut edge.
An edge in  is called a \emph{marked edge}, and every other edge is called an \emph{unmarked edge}.
A vertex incident with a marked edge is called a \emph{marked vertex},
and every other vertex is called an \emph{unmarked vertex}.
Each connected component of  is called a \emph{bag} of .

When a connected marked graph , which will be a bag of a marked graph, admits a split , we construct a marked graph  on the vertex set  such that
\begin{itemize}
\item for vertices  with  or ,  if and only if ,
\item  is a new marked edge,
\item  is anti-complete to ,
\item  is complete to  and  is complete to  (with unmarked edges). 
\end{itemize}
The marked graph  is called a \emph{simple decomposition of} .
A \emph{split decomposition} of a connected graph  is a marked graph  defined inductively to be either  or a marked graph defined from a split decomposition 
of  by replacing a connected component  of  with a simple decomposition of .  See Figure~\ref{fig:example} for an example of a split decomposition.
The following lemma provides an important property. An example of an alternating path described in Lemma~\ref{lem:splitadj} is presented in Figure~\ref{fig:example}.

\begin{lemma}[See Adler, Kant\'e, and Kwon, Lemma 2.10 of \cite{AKK2014}]\label{lem:splitadj}
Let  be a split decomposition of a connected graph  and 
 be two vertices in .
Then  if and only if there is a path from  to  in  where its first and last edges are unmarked, and
an unmarked edge and a marked edge alternatively appear in the path. 
\end{lemma}
 
Naturally, we can define a reverse operation of decomposing into a simple decomposition; for a marked edge  of a split decomposition , 
\emph{recomposing } is the operation of removing two vertices  and  and making  complete to  with unmarked edges.
It is not hard to observe that if  is a split decomposition of , then  can be obtained from  by recomposing all marked edges.

Note that there are many ways of decomposing a complete graph or a star, because every its non-trivial vertex partition is a split.
Cunningham and Edmonds \cite{CunninghamE80} developed a canonical way to decompose a graph into a split decomposition by not allowing to decompose a bag which is a star or a complete graph.
A split decomposition  of  is called a \emph{canonical split decomposition} if each bag of  is either a prime graph, a star, or a complete graph, and 
every recomposing of a marked edge in  results in a split decomposition without the same property. It is not hard to observe that every canonical split decomposition has no marked edge linking two complete bags, and no marked edge linking a leaf of a star bag and the center of another star bag~\cite{Bouchet1988a}. Furthermore, for each pair of twins  and  in , it holds that  and  must both be located in the same bag of the canonical split decomposition. 



\begin{figure}[t]
\centerline{\includegraphics[scale=0.66]{exampleorigin}\qquad \includegraphics[scale=0.55]{example}}
\caption{A graph  and its canonical split decomposition . Marked edges are represented by dashed edges, and bags are indicated by circles. 
Note that  and  is the set of -separator bags, and  is the set of -separator bags. 
The shortest path from  to  in  is a path from  to  where its first and last edges are unmarked, and
an unmarked edge and a marked edge alternatively appear in the path. The existence of such a path exactly corresponds to the adjacency relation in the original graph.
The distance between  and  in  is , and there are two -separator bags.}
\label{fig:example}
\end{figure}


\begin{theorem}[Cunningham and Edmonds~\cite{CunninghamE80}] \label{thm:CED} 
Every connected graph has a unique canonical split decomposition, up to isomorphism.
\end{theorem}
\begin{theorem}[Dahlhaus~\cite{Dahlhaus00}]\label{thm:dahlhaus}
The canonical split decomposition of a graph   can be computed in time .
\end{theorem}


We can now give the second characterization of distance-hereditary graphs that is crucial for our results.
For convenience, we call a bag a \emph{star bag} or a \emph{complete bag} if it is a star or a complete graph, respectively.

\begin{theorem}[Bouchet~\cite{Bouchet1988a}]\label{thm:bouchet}
A graph is a distance-hereditary graph if and only if every bag in its canonical split decomposition is either a star bag or a complete bag.
\end{theorem}

We will later on also need a little bit of additional notation related to split decompositions of distance-hereditary graphs.
Let  be a canonical split decomposition of a distance-hereditary graph.
For two distinct bags  and , we denote by  the connected component of  containing .
Technically, when , we define  to be the empty set.
For two bags  and , we denote by  the set of all bags containing a vertex in a shortest path from  to  in .
In other words, when we obtain a tree from  by contracting every bag  into a node ,
 is the set of all bags corresponding to nodes of the unique path from  to  in the tree. 
See Figure~\ref{fig:example}.


Let  and  be two disjoint vertex subsets of  such that  and  are sets of unmarked vertices contained in (not necessarily distinct) bags  and , respectively. 
A bag  is called \emph{a -separator bag}
  if  is a star bag contained in  whose center is adjacent to neither  nor .
 We remark that  can be  for some , and especially when , 
  is a star bag and each  consists of leaves of  and  is the unique -separator bag.
For convenience, we also say that a bag  is \emph{a -separator bag}
if   is a star bag contained in  whose center is adjacent to neither  nor .
For this notation,  cannot be  nor .

We observe that the distance between  and  in the original graph 
  is exactly the same as one plus the number of -separator bags. 

\begin{observation}\label{obser:separator}
The distance between  and  in the original graph 
  is exactly the same as one plus the number of -separator bags. 
\end{observation}


\section{Setting the Stage}
\label{sec:general}

We begin by applying the \emph{iterative compression technique}, first introduced by Reed, Smith and Vetta~\cite{ReedSV2004} to show that \textsc{Odd Cycle Transversal} can be solved in single-exponential FPT time. This technique allows us to transform our original target problem to one that is easier to handle, which we call \disjointDHVD. Our goal for the majority of the paper will be to obtain a single-exponential FPT algorithm for \disjointDHVD; this is then used to obtain an algorithm for {\DHVD} in Section~\ref{sec:completing}.

    \smallskip
\noindent
\fbox{\parbox{0.97\textwidth}{
\disjointDHVD \\
\emph{Instance :} A graph , an integer , and  such that  is distance-hereditary. \\
\emph{Task :} Is there  with  such that  is distance-hereditary? }}
\vskip 0.2cm

  
We will denote an instance of {\disjointDHVD} as a tuple .
The major part of our result is to prove that this problem can be solved in time , 
where  denotes the number of connected components of .
We note that any instance of \disjointDHVD{} such that  is not distance-hereditary must clearly be a \NO-instance; hence we will assume that we reject all such instances immediately. 

Before explaining the general approach for solving \disjointDHVD, it will be useful to introduce a few definitions.
Since the canonical split decomposition guaranteed by Theorem~\ref{thm:bouchet} only helps us classify twins in  and not in , we explicitly define an equivalence  on the vertices of  which allows us to classify twins in :
\begin{center}
for two vertices ,  iff they are twins in .
\end{center}

We denote by  the set of equivalence classes of  on , and each individual equivalence class will be called a \emph{twin class} in .
We can observe that if  lies in a single connected component of , then  must be contained in precisely one bag of the split decomposition of this connected component of , as  is a set of twins in  as well.
A twin class is \emph{-attached} if it has a neighbor in , and 
\emph{non--attached} if it has no neighbors in .
Similarly, we say that a bag in the canonical split decomposition of  is \emph{-attached} if it has a neighbor in , and
\emph{non--attached} otherwise.



We frequently use a special type of star bags.
A star bag  is called \emph{simple} if 
its center is either unmarked or adjacent to a connected component of  consisting of one non--attached bag.



\subsection{Overview of the Approach}
Now that we have introduced the required terminology, we can provide a high-level overview of our approach for solving \disjointDHVD.
\begin{enumerate}
\item We exhaustively apply the branching rules described in Section~\ref{subsec:branching}. 
Branching rules will be applied when  has a small subset  such that  induces a DH obstruction, or 
there is a small connected subset  such that adding  to  decreases the number of connected components in .
\item We exhaustively apply the initial reduction rules described in Section~\ref{sec:rules}. 
Each of these rules runs in polynomial time, finds a part in the canonical split decomposition of a connected component of  that can be simplified, and modifies the decomposition.
Each application of a reduction rule from Section~\ref{sec:rules} 
either reduces the number of vertices in  or reduces the total number of bags in the canonical split decomposition (of a connected component of ). 
It is well known that the total number of bags in the canonical split decomposition of a graph is linear in the number of vertices.
Therefore, the total number of application of these initial reduction rules will also be at most linear in the number of vertices.
\item We say that  and the canonical split decompositions of  are \emph{reduced} if the branching rules in Section~\ref{subsec:branching} and reduction rules in Section~\ref{sec:rules} cannot be applied anymore.
We will obtain the following simple structure of the decompositions in the reduced instance:
\begin{itemize}
\item Each canonical split decomposition  of a connected component of  contains at least two distinct -attached twin classes 
(Lemma~\ref{lem:onesattached}).
\item Each bag contains at most one -attached twin class 
(Lemma~\ref{lem:twinclassreduction}).
\item When  is a bag and  is a connected component of  containing no bags having a neighbor in , 
 consists of one bag and  is a star bag whose center is adjacent to  (Lemma~\ref{lem:smallbranch}).
In this case,  is a simple star bag whose center is adjacent to .
\item When  is a bag and  is a connected component of  such that
 contains exactly one -attached bag ,
there is no -separator bag (Lemma~\ref{lem:simplifynearsattached2}).
\end{itemize}
\item Using these structures, we prove in Subsection~\ref{sec:twinclass} that if a split decomposition of a connected component of  contains two -attached twin classes, 
then one of reduction rules should be applied. 
For this, we assign any bag as a root bag  of  and 
choose a bag  with maximum  such that there are two descendant bags of  having -attached twin classes  and , respectively.
Then the distance from  to  in  is at most , and thus their neighbors on  should be close to each other, as branching rules cannot be applied further.
Depending on the type of  and the distance from  to , 
we show separately that one of reduction rules can be applied.


It will imply that we can apply one of all rules recursively until  is empty or  becomes .
Then we can test whether the resulting instance is distance-hereditary or not in polynomial time, and output an answer.
\end{enumerate}


Let us also say a few words about the running time of the algorithm. Let . Each of our branching rules will reduce  and branch into at most  subinstances.
Each reduction rule takes polynomial time, and the reduction rules will be applied at most  times. 
Whenever we introduce a new rule, we need to show that it is \emph{safe}; for branching rules this means that there exists at least one subinstance resulting from the rule which is a \YES-instance if and only if the original graph was a \YES-instance, while for reduction rules this means that the application of the rule preserves the property of being a \YES-instance.




A vertex  in  is called \emph{irrelevant} if  is a \YES-instance if and only if  is a \YES-instance. We will be identifying and removing irrelevant vertices in several of our reduction rules. When removing a vertex  from , it is easy to modify the canonical split decomposition containing , and thus it is not necessary to recompute the canonical split decomposition of the resulting graph from scratch.
More details regarding such modifications of split decompositions can be found in the work of Gioan and Paul~\cite{GP2012}.

\subsection{Branching Rules}\label{subsec:branching}
We state our two branching rules below.

\begin{BRULE}\label{brule:threevertices}
For every vertex subset  of  with , 
if   is not distance-hereditary, then we remove one of the vertices in , and reduce  by .
\end{BRULE}
\begin{BRULE}\label{brule:reducecomponent}
For every vertex subset  of  with  such that  is connected and the set  is not contained in a connected component of , 
then we either remove one of the vertices in  and reduce  by , or put all of them into  (which reduces the number of connected components of ).
\end{BRULE}

The safeness of Branching Rules~\ref{brule:threevertices} and \ref{brule:reducecomponent} are clear, and these rules can be performed in polynomial time.
The exhaustive application of these branching rules guarantees the following structure of the instance.


\begin{lemma}
\label{lem:shortdistance}
Let  be an instance reduced under Branching Rules~\ref{brule:threevertices} and \ref{brule:reducecomponent}.
\begin{enumerate}[(1)]
\item  has no small DH obstructions.
\item Let . For every two vertices , 
they are contained in the same connected component of  and there is no induced path of length at least  from  to  in .
Specifically, if , then there is an induced path  for some .
\item There is no induced path  of length  in  where  and  have neighbors in  but  and  have no neighbors in .
\item There is no induced path  of length  in  where  and  have neighbors in  but  has no neighbors in .
\end{enumerate}
\end{lemma}
\begin{proof}
(1) Suppose  has a small DH obstruction . Since  is distance-hereditary, .
Thus, , and it can be reduced under Branching Rule~\ref{brule:threevertices}.

(2) First, by Branching Rule~\ref{brule:reducecomponent}, 
 and  are contained in the same connected component of .
Suppose there is an induced path of length at least  from  to  in .
Then by Lemma~\ref{lem:dhobs},  contains a DH obstruction, contradicting our assumption that  is reduced under Branching Rule~\ref{brule:threevertices}.
So, if , then there is an induced path of length  from  to  in .

(3) Suppose there is an induced path  of length  in  where  and  have neighbors on  but  and  have no neighbors on .
By Branching Rule~\ref{brule:reducecomponent}, 
we know that  and  are contained in the same connected component of .
Let  be a shortest path from  to  (if  and  have a common neighbor, then we choose a common neighbor).
Then  is an induced path of length at least  and  is adjacent to its end vertices.
So,  contains a DH obstruction, contradicting our assumption that  is reduced under Branching Rule~\ref{brule:threevertices}.

(4) The same argument in (3) holds.
\end{proof}


Lemma~\ref{lem:shortdistance}, and especially point  in the lemma, is used in many parts of our proofs. Since we will apply the branching rules exhaustively at the beginning and also after each new application of a reduction rule, these properties will be implicitly assumed to hold in subsequent sections.

We will make use of two more lemmas based on our branching rules. These
will be used in Section~\ref{sec:twinclass} as well as
in the proof of Lemma~\ref{lem:twinclassreduction} in Section~\ref{subsec:structure}.

\begin{lemma}
 \label{lem:relationt1t2}
Let  be an instance reduced under Branching Rules~\ref{brule:threevertices} and \ref{brule:reducecomponent}.
Let  be two distinct -attached twin classes of  such that  is anti-complete to , and . Then:
\begin{enumerate}[(1)]
\item .
\item For every  and every , 
if  is adjacent to , then  is adjacent to  as well.
It implies that  is adjacent to either all of vertices in  or neither of them.
\item For every  and every , 
if there is a path  for some  (not necessarily induced), then  is adjacent to  as well.
It implies that  is adjacent to either all of vertices in  or neither of them.
\end{enumerate} 
 \end{lemma}
 \begin{proof}
 For each  let  and let .

(1) Suppose  and  are disjoint.
Let us choose a vertex  in , which is not an empty set by assumption.
Thus,  induces a connected subgraph of .
If  and  are not contained in one connected component of ,
then we can apply Branching Rule~\ref{brule:reducecomponent}. 
As our instance was reduced under Branching Rule~\ref{brule:reducecomponent},
we know that  and  are contained in the same connected component of . 

Let  be a shortest path from  to  in .
Clearly,  contains at most one vertex from each .
As  is anti-complete to , 
 is an induced path of length at least  and  is adjacent to its end vertices. 
By Lemma~\ref{lem:dhobs},  contains a DH obstruction, contradicting the assumption that
 is reduced under Branching Rule~\ref{brule:threevertices}. 


(2) For contradiction, suppose  and .  Then  is an induced path of length  and  is adjacent to its end vertices.
By Lemma~\ref{lem:dhobs},  contains a small DH obstruction, contradiction. 

(3) Suppose there is a path  for some  and  is not adjacent to .   
First assume that .
If , then  is an induced path, and
otherwise,  is an induced path.
Since  is adjacent to  and , by Lemma~\ref{lem:dhobs},  contains a small DH obstruction, contradiction. 
When ,  becomes an induced path of length  and  is adjacent to its end vertices, 
and thus  contains a small DH obstruction. We conclude that  is adjacent to .
 \end{proof}
 

\begin{lemma}\label{lem:completerelationt1t2}
 Let  be an instance reduced under Branching Rules~\ref{brule:threevertices}, and \ref{brule:reducecomponent}.
Let  be two distinct twin classes of  such that  is  complete to . Then:
\begin{enumerate}[(1)]
\item For every  and every , 
if  is adjacent to , then either it is adjacent to  as well, or  is adjacent to .
\item For every  and every , 
if there is a path  for some  (not necessarily induced), then .
\end{enumerate} 
 \end{lemma}
\begin{proof}
For each  and let  and let .

(1) Suppose  and .  Then  is isomorphic to the gem, contradiction.

(2) Suppose there is a path  for some .   
Then  is an induced path of length , and  is adjacent to its end vertices.
By Lemma~\ref{lem:dhobs}, 
 contains a small DH obstruction, contradiction.
\end{proof}




\section{Reduction Rules in Split Decompositions}\label{sec:rules}






   








In this section, we assume that the given instance  is reduced under Branching Rules~\ref{brule:threevertices} and \ref{brule:reducecomponent}.
The reduction rules introduced here either remove some irrelevant vertex, move some vertex into , or reduce the number of bags in the decomposition by modifying the instance into an equivalent instance.
After we apply any of these reduction rules, we will run the two branching rules from Section~\ref{sec:general} again.

In Subsection~\ref{subsec:bypassing}, we introduce the notion of a \emph{bypassing vertex}, which is a crucial concept that will frequently appear in our proofs.  
In Subsection~\ref{subsec:sixrules}, we present five reduction rules and prove their correctness.
Then in Subsection~\ref{subsec:structure}, we discuss structural properties of the obtained instance after exhaustive application of all of the presented branching rules and reduction rules.
These properties will be used in Section~\ref{sec:twinclass} to argue that if the instance is non-trivial, then one can apply one of reduction rules.

\subsection{Bypassing Vertices}\label{subsec:bypassing}

We introduce a generic way of finding an irrelevant vertex which will be used in many reduction rules.
For a vertex  in  and an induced path  in  where ,
a vertex  in  is called a \emph{bypassing vertex} for  and  if  is adjacent to  and . 
When  is clear from the context, we simply say that  is a bypassing vertex for . 
If such a vertex  exists, it is clear that  is not contained in .
The following property is essential.

\begin{lemma}
\label{lem:badvertex}
Let  be an instance reduced under Branching Rules~\ref{brule:threevertices} and \ref{brule:reducecomponent}.
Let  be a vertex in  such that for every induced path  where , 
there is a bypassing vertex for  and .
Then  is irrelevant.
\end{lemma}
\begin{proof}
We claim that  is a \YES-instance if and only if  is a \YES-instance.
The forward direction is clear.
Suppose that  has a vertex set  such that , , and  is distance-hereditary. 
If  is distance-hereditary, then we are done.
Suppose that  has a DH obstruction . Since Branching Rule~\ref{brule:threevertices} does not apply,  has no small DH obstructions, and therefore
 is an induced cycle of length at least . 
Let  be the subpath of  such that .
By the assumption, there is a bypassing vertex  for  and .
Note that , as  would be a cycle of length .
Also,  is an induced path of length at least . 
Thus  contains another DH obstruction by Lemma~\ref{lem:dhobs}. Note that  and hence also  and, in particular, . 
This contradicts the fact that  is distance-hereditary.
\end{proof}

\begin{lemma}
\label{lem:twovertexinS}
Let  be an instance reduced under Branching Rules~\ref{brule:threevertices} and \ref{brule:reducecomponent}.
Let  be a vertex in  and  be an induced path where .
If , then there is a bypassing vertex for  and .
\end{lemma}
\begin{proof}
Note that .
Thus, by (2) of Lemma~\ref{lem:shortdistance}, 
there is an induced path  for some , and  is a bypassing vertex.
\end{proof}

\subsection{Five Reduction Rules}\label{subsec:sixrules}

We are now ready to start with our reduction rules. 
For the remainder of this section, let us fix a canonical split decomposition  of a connected component of .

We start with a simple reduction rule that can be applied when  contains at most one -attached twin class.
\begin{RRULE}\label{rrule:dhcomponent}
If  has at most one -attached twin class, then we remove all unmarked vertices of  from .
\end{RRULE}
\begin{lemma}
\label{lem:onesattached}
Reduction Rule~\ref{rrule:dhcomponent} is safe.
\end{lemma}

\begin{proof}
If  has no -attached twin class, then its underlying graph is a distance-hereditary connected component of .
Thus, we can safely remove all its unmarked vertices. We may assume that  has one -attached twin class .

Since  is the only -attached twin class and every induced cycle of length at least  contains no twins,
no induced cycle of length at least  contains an unmarked vertex in .
Thus, we can safely remove all of unmarked vertices other than vertices in .
Now we assume that . We claim that every vertex in  is also irrelevant.

To apply Lemma~\ref{lem:badvertex}, 
suppose there is an induced path
  where .
 Since there are no twins in , 
  contains at most one vertex of .
Thus,  and  are contained in . Since  and  is reduced under Branching Rules~\ref{brule:threevertices} and \ref{brule:reducecomponent}, 
by Lemma~\ref{lem:twovertexinS}, there is a bypassing vertex for  and .
Since  was arbitrarily chosen, by Lemma~\ref{lem:badvertex},  is irrelevant.
\end{proof}

The next rule deals with a vertex of degree  in .
 
\begin{RRULE}\label{rrule:leaftoS}
Let  be a star bag whose center is unmarked, and let  be a leaf unmarked vertex in . 
If  has no neighbor in , then we remove .
If  has a neighbor in , then we move  into .
\end{RRULE}

\begin{lemma}
\label{lem:leaftoS}
Reduction Rule~\ref{rrule:leaftoS} is safe.
\end{lemma}
\begin{proof}
Let  be the center of  and let  be a leaf unmarked vertex in .
If  has no neighbor in , then  has degree  in , and we can safely remove it.
We assume that  has a neighbor in .
We claim that  is a \YES-instance if and only if  is a \YES-instance.
The converse direction is easy. Suppose that  contains a vertex set  where , , and  is distance-hereditary. 
Let  if , and otherwise, we remove  from  and add  to , and call it .
We claim that 
 is distance-hereditary, which implies that  is a \YES-instance.

Suppose  is not distance-hereditary.
Since  has no small DH obstructions,  contains an induced cycle  of length at least .
First assume that  contains . 
Then  is not contained in , and therefore,  was not contained in , and we have  by the construction. Thus  also contains , contradiction.
Thus, we have . 
If , then  is an induced subgraph of  because  and  only differ at .
This implies that  and . Thus  contains .

Let  be the subpath of  where .
As  contains , 
 and  are contained in .
As  is reduced under Branching Rules~\ref{brule:threevertices} and \ref{brule:reducecomponent}, 
by Lemma~\ref{lem:twovertexinS}, there is a bypassing vertex for  and .
Thus,  contains another DH obstruction.
It contradicts the fact that  is distance-hereditary, as  is an induced subgraph of .
\end{proof}

We remark that when we move  into  in Reduction Rule~\ref{rrule:leaftoS}, 
 does not increase, and the size of  decreases.
After applying Reduction Rule~\ref{rrule:leaftoS} exhaustively, we obtain that 
if the center of a star bag is unmarked, then this bag contains no unmarked leaves.


The next reduction rule arises directly from the definition of bypassing vertices.

\begin{RRULE}\label{rrule:p5middle}
Let  be a vertex in  such that 
for every induced path  with , 
there is a bypassing vertex for  and .
Then we remove . In  particular, when there is no such an induced path, we remove .
\end{RRULE}

For fixed , we can apply Reduction Rule~\ref{rrule:p5middle} in time  by considering all vertex subsets of size , 
and testing whether  and  have a common neighbor in .

\begin{lemma}
  \label{lem:p5middle}
Reduction Rule~\ref{rrule:p5middle} is safe.
\end{lemma}
\begin{proof}
This follows from Lemma~\ref{lem:badvertex}.
\end{proof}



We proceed by introducing a reduction rule which sequentially arranges bags containing exactly one twin class.
The operation of \emph{swapping the adjacency} between two vertices  and  in a graph is to remove  if  was an edge, and otherwise add an edge between  and .
The number of bags in  is strictly reduced when applying Reduction Rule~\ref{rrule:leaf}.

\begin{RRULE}\label{rrule:leaf}
Let  be a leaf bag and  be the neighbor bag of .
\begin{enumerate}[(1)]
\item If  is a complete bag having exactly one twin class in  and  is a star bag whose leaf is adjacent to , 
then we swap the adjacency between every two unmarked vertices in . By swapping the adjacency,  becomes a star whose center is adjacent to , and thus we can recompose the marked edge connecting  and .
We recompose the marked edge connecting  and .
\item If  is a star bag having exactly one twin class in , the center of  is adjacent to , and  is a complete bag,  
then we swap the adjacency between every two unmarked vertices in . By swapping the adjacency,  becomes a complete graph, and thus we can recompose the marked edge connecting  and .
We recompose the marked edge connecting  and .
\end{enumerate}
\end{RRULE}

We use the following lemma.
\begin{lemma}\label{lem:swaptwinproperty}
Let  be a set of vertices that are pairwise twins in . Let  be the graph obtained from  by swapping the adjacency relation between every pair of two distinct vertices in .
Then  is a \YES-instance if and only if  is a \YES-instance.
\end{lemma}
\begin{proof}
Note that either  is a complete graph or it has no edges. Therefore,  is again a set of vertices that are pairwise twins in .
Since each DH obstruction contains at most one vertex from a set of twins (and hence, at most one vertex from ), swapping the adjacency on  will neither introduce nor remove DH obstructions from .
Hence it is easy to check that   is a \YES-instance if and only if  is a \YES-instance.
\end{proof}
\begin{lemma}
Reduction Rule~\ref{rrule:leaf} is safe.
\end{lemma}
\begin{proof}
This follows from Lemma~\ref{lem:swaptwinproperty}.
\end{proof}


The last rule consider bags near to some leaf bag.
We illustrate in Figure~\ref{fig:bypassing2}.
Recall that 
a star bag  is simple if 
its center is either unmarked or adjacent to a connected component of  consisting of one non--attached bag.


\begin{RRULE}\label{rrule:bypassing2}
Let  be distinct bags in  such that  
\begin{itemize}
\item  is a non--attached leaf bag whose neighbor bag is , and it is not a star whose leaf is adjacent to , 
\item  has exactly two neighbor bags  and , it is a star whose center is adjacent to , and the set of unmarked vertices in  is the unique -attached twin class  in ,  and
\item  is a simple star bag.
\end{itemize}
Let  be the set of unmarked vertices in .
Then we remove  and , and
add a leaf set of unmarked vertices  with  vertices to , that is complete to  and has no other neighbors in .
\end{RRULE}


Note that this rule can potentially create an induced cycle of length . So, we need to run Branching Rule~\ref{brule:threevertices} after applying Reduction Rule~\ref{rrule:bypassing2}. 
We confirm the safeness of Reduction Rule~\ref{rrule:bypassing2} in the following lemma.

  \begin{figure}[t]
    
    \centering
      \includegraphics[scale=0.55]{rrule81}
    \qquad
      \includegraphics[scale=0.55]{rrule82}
        
    \caption{Reduction Rule~\ref{rrule:bypassing2}.} \label{fig:bypassing2}
      
  \end{figure}

\begin{lemma}
\label{lem:bypassing2}
Reduction Rule~\ref{rrule:bypassing2} is safe.
\end{lemma}
\begin{proof}
As  is a simple star bag and  has no neighbors in , 
 is exactly 
the center of  if it is unmarked, and otherwise, the set of unmarked vertices in the bag where the center of  is adjacent.
Let . 
We remark that  is a twin class. 


Let  be the resulting graph obtained by applying Reduction Rule~\ref{rrule:bypassing2}.
Note that  is a set of pairwise twins in  (it may not be a twin class), 
and .

We claim that  is a \YES-instance if and only if  is a \YES-instance.
Suppose  has a minimum vertex set  such that , , and  is distance-hereditary.
We divide cases depending on whether  contains a vertex of  or not.

\subparagraph{\textbf{Case 1.}  contains a vertex in :}  
We observe that since  is a twin class and  is a minimum solution, 
if  contains a vertex of , then  contains all vertices in .
Thus,   fully contains one of  and .
Since , the set  has size at most . Moreover, we conclude that
 is distance-hereditary, as it is an induced subgraph of . 


\subparagraph{\textbf{Case 2.}  contains no vertex in :}  

Suppose that  contains a DH obstruction .
If  does not contain a vertex in , then  is an induced subgraph of , contradicting our assumption.
Thus,  contains a vertex in , and as every pair of two distinct vertices in  is a twin, we have .
Let  be the vertex in , and let  be the two neighbors of  in . As  is a twin class in , 
at least one of  and  is contained in . Without loss of generality, we assume .

If , then we can obtain a DH obstruction by replacing  with a vertex of  in ,
which implies that  contains a DH obstruction. Thus, we may assume that  is contained in , and henceforth we have .


For two vertices  and , 
we can obtain a DH obstruction in  from  by removing  and adding , which is equivalent (up to isomorphism) to subdividing the unique edge in  incident to  and a vertex in . By Observation~\ref{obs:sub}, we know that the resulting graph  must then also contain a DH obstruction, contradicting our assumption.

\medskip

For the converse direction, suppose that  has a minimum vertex set  such that , , and  is distance-hereditary.
Similar to the forward direction, we divide cases depending on whether  contains a vertex in  or not.

\subparagraph{\textbf{Case 1.}  contains no vertex in :}  
Suppose  has a DH obstruction . Since  has no small DH obstructions due to the application of branching rules,  should be an induced cycle of length at least .
We have , otherwise  is an induced subgraph of , which is contradiction.
As  and  are twin classes,  contains at most one vertex from each of  and .

We claim that  contains one vertex from each of  and .
Suppose  and . Then the two neighbors of the vertex on  belong to , 
since . But  forms a twin class, and an induced cycle of length at least  cannot contain two vertices from the same twin class; a contradiction.
Suppose  but .  
Then the two neighbors of the vertex  in  in  are contained in .
Let  be the subpath of  where .
By Lemma~\ref{lem:twovertexinS}, there is a bypassing vertex for  and , and thus  contains a DH obstruction, which is also contained in .
This constitutes a contradiction. We conclude that  contains one vertex from each of  and .




 It further implies that  contains one vertex from each of  and , because . 
Since  has length at least , we can obtain an induced cycle of length at least  in  from  by removing the vertices in  and adding one vertex of , which is contradiction.

\subparagraph{\textbf{Case 2.}  contains a vertex in :}  
As  is a twin class and  is a minimum solution for , we have .
We obtain a set  from  by removing , and adding  if  and adding  if . If , then we add one of them chosen arbitrarily.
Clearly, .
We claim that  is distance-hereditary.

In case when , we observe that every induced cycle of length at least  containing a vertex in  has to contain two vertices in , which is not possible.
Thus, we may assume .
Note that . Thus, whenever there is an induced cycle of length at least  in  containing a vertex in , by Lemma~\ref{lem:twovertexinS} there exists another DH obstruction which does not contain any vertex in , contradicting the assumption that  is distance-hereditary.
Hence we conclude that  is distance-hereditary.
\end{proof}


\begin{PROP}
\label{prop:applicationRR}
 Let  be an instance reduced under Branching Rules~\ref{brule:threevertices} and \ref{brule:reducecomponent}.
Given a connected component  of , we can in time  either apply one of Reduction Rules~\ref{rrule:dhcomponent}--\ref{rrule:bypassing2}, or correctly answer that Reduction Rules~\ref{rrule:dhcomponent}--\ref{rrule:bypassing2} cannot be applied anymore. 
\end{PROP}
\begin{proof}
We first compute the canonical split decomposition  of  in time   using Theorem~\ref{thm:dahlhaus}.
Then we classify twin classes in  by testing two unmarked vertices in a bag have the same neigbhorhood in  or not. 
This can be done in time . 
At the same time, we can also test whether a twin class is -attached or not. 
Note that the total number of bags in canonical split decompositions of connected components of  is .

We can apply Reduction Rules~\ref{rrule:dhcomponent}, \ref{rrule:leaftoS}, \ref{rrule:leaf} in time , if one of them can be applied. 
We can apply Reduction Rule~\ref{rrule:p5middle} in time  for fixed vertex , 
and thus, we can test for all vertices  in time .
For Reduction Rule~\ref{rrule:bypassing2}, we need to consider three bags, which are uniquely identified by the first (leaf) bag among them, to check whether they satisfy preconditions of the rule. We can verify the preconditions of Reduction Rule~\ref{rrule:bypassing2} in constant time and 
thus this step takes time .
We conclude that we can in time  either apply one of Reduction Rules~\ref{rrule:dhcomponent}--\ref{rrule:bypassing2}, or correctly answer that Reduction Rules~\ref{rrule:dhcomponent}--\ref{rrule:bypassing2} cannot be applied anymore. 
\end{proof}




\subsection{Structural Properties obtained after Exhaustive Application of Rules}\label{subsec:structure}

In this subsection, we discuss structural properties obtained after the exhaustive application of both branching and reduction rules.
We say that  and the canonical split decompositions of connected components of  are \emph{reduced} 
if Branching Rules~\ref{brule:threevertices}--\ref{brule:reducecomponent} and Reduction Rules~\ref{rrule:dhcomponent}--\ref{rrule:bypassing2} cannot be applied anymore.
We assume that the given instance is reduced in this subsection.

The following observation is a direct consequence of the exhaustive application 
of Reduction Rule~\ref{rrule:leaftoS}.


\begin{observation}\label{obs:starrestriction}
If the center of a star bag in  is unmarked, then this bag contains no unmarked leaves.
\end{observation}


Our next goal is to establish the following lemma.




\begin{lemma}\label{lem:twinclassreduction}
Every bag of  contains at most one -attached twin class.
\end{lemma}

Before we formally prove Lemma~\ref{lem:twinclassreduction}, we briefly explain how the argument works.
Let  and  be two distinct -attached twin classes in a bag  such that neither of them consists of the center of a star and  is non-empty.
If  is a star, then  is anti-complete to  and ,  have a common neighbor in , and thus,  and  satisfy preconditions of Lemma~\ref{lem:relationt1t2}.
Lemma~\ref{lem:relationt1t2} implies that  is non-empty. 
Let  and . 
We argue that whenever there is an induced path  with  there is a bypassing vertex for .
We describe an example case. For instance, when  and  are both contained in , they are contained in . 
So,  while .
Thus, by (2) of Lemma~\ref{lem:relationt1t2},
if  is adjacent to , then  should be adjacent to  and , which means that  is a bypassing vertex for .
If  is not adjacent to , then we could apply (3) of Lemma~\ref{lem:relationt1t2} to find a bypassing vertex for .
We do a careful analysis depending on the places of  and , and also consider the case when  is a complete bag.





 \begin{proof}[Proof of Lemma~\ref{lem:twinclassreduction}]
Suppose there is a bag containing two distinct -attached twin classes  and .
By Observation~\ref{obs:starrestriction}, if  consists of the center of a star, then there are no other unmarked vertices in the bag, and thus it is not possible.
Therefore,  does not consist of the center of a star bag.
As  and  are distinct twin classes, , and thus we have 
either  or .
Without loss of generality, we assume  is non-empty.


For each , let  and let . Let .
We observe that .
This is because  and  are contained in , which is a complete bag or a star bag whose center is marked.

We claim that for every  and every induced path ,
there is a bypassing vertex for  and . 
This will imply that we can apply Reduction Rule~\ref{rrule:p5middle}, which leads a contradiction.

If , then by Lemma~\ref{lem:twovertexinS}, there is a bypassing vertex for .
We may assume that  or  is contained in .
Without loss of generality, we assume that  is contained in .
We depict cases in Figures~\ref{fig:lemma45star1} and \ref{fig:lemma45star2}.



  \begin{figure}[t]
\begin{subfigure}[b]{0.5\textwidth}
      \includegraphics[scale=0.5]{lemma451}
      \caption{}
      \end{subfigure}
      \qquad
\begin{subfigure}[b]{0.5\textwidth}
      \includegraphics[scale=0.5]{lemma452}
      \caption{}
      \end{subfigure}
            \caption{When  is a star bag in Lemma~\ref{lem:twinclassreduction}.
The red thick edges illustrate the edges whose existence is guaranteed by Lemma~\ref{lem:relationt1t2}. } \label{fig:lemma45star1}
  \end{figure}


\subparagraph{\textbf{Case 1.}  is a star bag:}  
In this case,  is anti-complete to  and .
By (1) of Lemma~\ref{lem:relationt1t2}, we have . Let . 
We divide cases depending on whether  or .

Suppose . Note that both  and  are contained in .
Since , by (2) of Lemma~\ref{lem:relationt1t2},
 is not adjacent to .
Since  and  are neighbors of  and  are not adjacent, by (2) of Lemma~\ref{lem:shortdistance}, there is an induced path  for some .
Then by (3) of Lemma~\ref{lem:relationt1t2},  is adjacent to , and therefore,  is a bypassing vertex.

Suppose . Recall that  is a vertex in . If  is adjacent to , then by (2) of Lemma~\ref{lem:relationt1t2},
 is adjacent to both  and , and thus  is a bypassing vertex.
We may assume that .
Then by (2) of Lemma~\ref{lem:shortdistance}, there is an induced path  for some .
By (3) of Lemma~\ref{lem:relationt1t2},  is adjacent to both  and , and therefore,  is a bypassing vertex, as required.


  \begin{figure}[t]
\begin{subfigure}[b]{0.5\textwidth}
      \includegraphics[scale=0.5]{lemma453}
       \caption{.}
\end{subfigure}
\qquad
\begin{subfigure}[b]{0.5\textwidth}
      \includegraphics[scale=0.5]{lemma454}
       \caption{.}
\end{subfigure}
\center
\begin{subfigure}[b]{0.5\textwidth}
      \includegraphics[scale=0.5]{lemma455}
       \caption{.}
\end{subfigure}
            \caption{When  is a complete bag and  in Lemma~\ref{lem:twinclassreduction}.
            The red thick edges illustrate the edges whose existence is guaranteed by Lemma~\ref{lem:completerelationt1t2} or non-existence of small DH obstructions. } \label{fig:lemma45star2}
  \end{figure}

\subparagraph{\textbf{Case 2.}  is a complete bag:}  
Note that  is complete to , and   is contained in either   or .
We first discuss when  is contained in .

Suppose . As  is not adjacent to ,  cannot be in , and furthermore  cannot be in  as  is a complete bag. 
Thus . 
If  has a neighbor in , then the neighbor is a bypassing vertex, because it is adjacent to both  and .
We may assume that  has no neighbors in .
Observe that  and  are contained in the same connected component of , otherwise, Branching Rule~\ref{brule:reducecomponent} can be applied.
Let us take a shortest path  from  to  in .
Then  is an induced path of length at least  and  is adjacent to its end vertices, and thus  has a DH obstruction by Lemma~\ref{lem:dhobs}, which contradicts the assumption that  is reduced under Branching Rule~\ref{brule:threevertices}.

Now, suppose . In this case,  is not in . We distinguish subcases by the places of .
We illustrate cases in Figure~\ref{fig:lemma45star2}.

\subparagraph{\textbf{Case 2-1.}  :}  
Let  be a shortest path from  to .
If  has length at least , then  is an induced path of length at least  and  is adjacent to its end vertices.
So,  contains a DH obstruction, which is a contradiction.
Thus,  has a neighbor in , say .
If  is not adjacent to , then  is an induced path and  is adjacent to its end vertices.
This contradicts the assumption that  is reduced under Branching Rule~\ref{brule:threevertices}.
Thus,  and  is a bypassing vertex.

\subparagraph{\textbf{Case 2-2.}  :}  
Recall that  is a vertex in .
If  is adjacent to , then by (1) of Lemma~\ref{lem:completerelationt1t2}, 
 is adjacent to 
because  is not adjacent to . Thus,  is a bypassing vertex.
So, we may assume that  is not adjacent to .
By (2) of Lemma~\ref{lem:shortdistance}, there is an induced path  for some .

If , then  is adjacent to  by (1) of Lemma~\ref{lem:completerelationt1t2}, and thus  is a bypassing vertex.
Assume . If  is adjacent to , then  is a bypassing vertex, and we are done.
Otherwise, by (1) of Lemma~\ref{lem:completerelationt1t2},  should be adjacent to both  and , since .
Therefore, we may assume that . 
Then 
by (2) of Lemma~\ref{lem:completerelationt1t2}, 
we have , and again by (1) of Lemma~\ref{lem:completerelationt1t2},
either  or .
Since ,  becomes a bypassing vertex.

\subparagraph{\textbf{Case 2-3.}  :}  
If  is adjacent to  or , then by (1) of Lemma~\ref{lem:completerelationt1t2},  
 is adjacent to both  and , because .
Then  is a bypassing vertex. Therefore, we may assume  is adjacent to neither  nor .
We take a shortest path  from  to .
If  has length at least , then  is an induced path of length at least , and 
since  is adjacent to its end vertices,  contains a DH obstruction by Lemma~\ref{lem:dhobs}.
But this contradicts the assumption that  is reduced under Branching Rule~\ref{brule:threevertices}.
We may assume that  has length , and let  be a neighbor of  in .
Observe that if  is not adjacent to  or , then  or  is an induced path, respectively, 
and  is adjacent to its end vertices. It contradicts the assumption that  is reduced under Branching Rule~\ref{brule:threevertices}.
Therefore  is adjacent to both  and , which implies that  is a bypassing vertex.

\medskip 
We conclude that, for every induced path , there exists a bypassing vertex for . It contradicts the assumption that  is reduced under Reduction Rule~\ref{rrule:p5middle}.
Therefore, every bag of  contains at most one -attached twin class.
\end{proof}


Next, we show that for a bag  of , if a connected component of  contains no -attached bags, then  is a simple star bag adjacent to the component.

\begin{lemma}
\label{lem:smallbranch}
Let  be a bag and  be a connected component of  containing no -attached bags.
Then  is a simple star bag whose center is adjacent to .
\end{lemma}
\begin{proof}
Let  be the neighbor bag of  contained in .
First claim that . Suppose  contains at least one bag other than .
We regard  as the root bag of , and choose a bag  in  with maximum . 
Clearly,  is a leaf bag in .
Let  be the neighbor bag of .


Suppose  is a star.
As we choose  with maximum , 
every child of  is a leaf bag.
We claim that there is no leaf bag of  pending to a leaf of .
Suppose for contradiction there exists such a bag .
Since  is canonical,  is not a star whose center is adjacent to .
If  is a star whose leaf is adjacent to , then it can be reduced under Reduction Rule~\ref{rrule:leaftoS}.
If  is a complete graph, then it can be reduced under Reduction Rule~\ref{rrule:leaf}, 
which is a contradiction. 
Therefore, there is no leaf bag of  pending to a leaf of , 
and it implies that 
the center of  is adjacent to , and
 is the unique child of .

Let  be an unmarked vertex of . As  is the set of unmarked vertices in  which is a twin class, 
any induced path of length  could not contain  as the third vertex.
Therefore, Reduction Rule~\ref{rrule:p5middle} can be applied to remove , which is a contradiction.


Suppose  is a complete graph.
Since  is canonical,  is not a complete graph.
If  is a star whose leaf is adjacent to , 
then all unmarked leaves in  can be removed by Reduction Rule~\ref{rrule:leaftoS}.
If  is a star whose center is adjacent to , then 
we can apply Reduction Rule~\ref{rrule:leaf} to  and .

We conclude that . Moreover, if  is not a star whose center is adjacent to , then we can reduce  using Reduction Rule~\ref{rrule:leaftoS} or \ref{rrule:leaf}.
Thus  is a star whose center is adjacent to .
\end{proof}

The following structure is illustrated in Figure~\ref{fig:rrule2}.



  \begin{figure}[t]
      \centering
      \includegraphics[scale=0.55]{rrule31}
\caption{Lemma~\ref{lem:starcenter}.} \label{fig:rrule2}
  \end{figure}
  
\begin{lemma}
  \label{lem:starcenter}
Let  be a leaf bag containing at most  one -attached twin class and  be a bag distinct from  such that
\begin{itemize}
\item  is a star bag whose center is adjacent to .
\item every bag in  is not a -separator bag, and has exactly two neighbor bags, and 
\item for every bag  in  that is not a star whose center is adjacent to , 
 is non--attached.
\end{itemize}
Then  contains no non--attached twin class .
\end{lemma}
\begin{proof}
Suppose  contains a non--attached twin class, and 
let  be a vertex in the class.
We claim that there is no induced path , which implies that Reduction Rule~\ref{rrule:p5middle} can be applied.
Suppose there is such a path.

We claim that either  or . If  then  should be adjacent to , which contradicts the fact that 
 is an induced path. The same argument holds when .

Let  and  be the bags containing  and , respectively. As  is a star whose center is adjacent to ,
 and  are bags in . 

First assume . In this case, since  is not adjacent to ,  is a star bag.
Note that no DH obstruction contains two twins, and therefore,  and  are contained in distinct twin classes.
Since  contains at most one -attached twin class by Lemma~\ref{lem:twinclassreduction}, 
one of  and  is contained in the non--attached twin class in .
Say  is such a vertex. Then we have , because  and  are twins in .

Now, we assume at least one of  and  is not equal to .
We further assume  is contained in . The same argument holds when  is contained in .

Since  and ,  is not a complete bag. 
Thus,  is a star bag whose center is adjacent to .
As  and it is not -attached by the assumption, all neighbors of  in  are neighbors of .
Then  should be adjacent to , which is a contradiction.



We conclude that there is no such path , and Reduction Rule~\ref{rrule:p5middle} can be applied to remove . Therefore,  contains no non--attached twin class .
\end{proof}


The following structure is illustrated in Figure~\ref{fig:rrule3}.
 
\begin{lemma}\label{rrule:sattachedleaf}
Let  be a leaf bag having exactly one -attached twin class and  be a simple star bag distinct from  such that 
\begin{itemize}
\item  is not a star whose leaf is adjacent to a neighbor bag,
\item every bag in  is non--attached, not a -separator bag and has exactly two neighbor bags.
\end{itemize}
Then  contains no non--attached twin class.
\end{lemma}
\begin{proof}
We claim that if  contains a non--attached twin class, then we can apply a reduction rule to remove it.
Suppose  contains a non--attached twin class , 
and let  be the -attached twin class in .

Let  and we claim that there is no induced path .
If this is true, then we can apply Reduction Rule~\ref{rrule:p5middle}. 
Suppose there is such an induced path.

Assume that . In this case,  should be a complete bag.
Therefore,  is adjacent to , 
because , and ,  are twins in .
This contradicts the assumption that  is an induced path. Thus, we can assume that , and similarly, .


By symmetry, we assume , where  and  are bags containing  and , respectively.
Since  is not adjacent to ,  should be a star bag whose center is adjacent to the component . 
Therefore, every neighbor of  in  is adjacent to , and in particular,  is adjacent to . 
This contradicts the assumption that  is induced.
This proves the claim.
It contradicts the assumption that  is reduced.
\end{proof}



  \begin{figure}[t]
    \centering
      \includegraphics[scale=0.5]{rrule41}


    \caption{Lemma~\ref{rrule:sattachedleaf}.} \label{fig:rrule3}
      
  \end{figure}



\begin{lemma}
\label{lem:simplifynearsattached}
Let  be a simple star bag, and let  be a  connected component of  such that
\begin{itemize}
\item  contains exactly one -attached bag , and
\item there is no -separator bag.
\end{itemize}
Then  is a star whose leaf is adjacent to  and there is a leaf bag  where the center of  is adjacent to .
\end{lemma}
\begin{proof}
We first claim that  is a star whose leaf is adjacent to . We prove this by a sequence of auxiliary claims.
Suppose for contradiction that  does not satisfy the property; that is, either  is a complete bag or a star bag whose center is adjacent to .

\begin{claim}
There is no connected component of  other than .
\end{claim}
\begin{clproof}
If there is such a component , then by the assumption, it contains no -attached bag.
By Lemma~\ref{lem:smallbranch},  is a star whose center is adjacent to , 
contradicting our assumption. Thus, there is no connected component of  other than . 
\end{clproof}

We observe that  contains one -attached twin class by Lemma~\ref{lem:twinclassreduction}.
Also, all bags in  have exactly two neighbor bags.
This follows from Lemma~\ref{lem:smallbranch} and the fact that every bag in  is not a -separator bag. 
Now, we can observe that  and  satisfy the conditions of Lemma~\ref{rrule:sattachedleaf}.
Therefore,  contains no non--attached twin class.

\begin{claim}\label{claim:starreverse}
There is no star bag  whose center is adjacent to .
\end{claim}
\begin{clproof}
Suppose there is such a star bag .
Then  and  satisfy conditions in Lemma~\ref{lem:starcenter}.
Thus,  has no non--attached twin class.
But this is impossible as  has only two neighbor bags and  has no -attached twin class.
Thus, such a bag  does not exist.
\end{clproof}

By Claim~\ref{claim:starreverse}, we observe that  and its parent bag satisfy the condition (1) or (2) of Reduction Rule~\ref{rrule:leaf}, and thus we can apply the rule.
It contradicts the assumption that  is reduced.
Thus,  is a star whose leaf is adjacent to .

Now, suppose there is no bag  where the center of  is adjacent to .
Since there is no bag pending to a leaf of  by Lemma~\ref{lem:smallbranch},
 is a leaf bag. In this case, we can reduce using Reduction Rule~\ref{rrule:leaftoS}, which is a contradiction.
Therefore, there is a leaf bag  where the center of  is adjacent to , as required.
\end{proof}





The next lemma describes the structure of  where  and  are simple star bags, and there is no -attached bag in the connected component of  containing bags in .

\begin{lemma}\label{rrule:bypassing1}
Let  and  be two simple star bags in  such that 
\begin{itemize}
\item every bag in  is a non--attached bag, has two neighbor bags, and is not a -separator bag. 
\end{itemize}
Then  and  are neighbor bags.
\end{lemma}
\begin{proof}
Suppose for contradiction that .
Let  and  be an unmarked vertex of .

We claim that there is no induced path .
Suppose there is such an induced path.
By symmetry, we assume , where  and  are bags containing  and , respectively.
If  and  are contained in the different connected components of , 
then because  is the not -separator bag,  should be a complete bag.
But then  is adjacent to , contradiction.
Thus,  and  are contained in the same connected component of .
Without loss of generality, we may assume that such a connected component contains .

Suppose there is a bag  where the center of  is adjacent to .
Since  is simple,  contains only non--attached twin class.
Thus one of  and  are not contained in , as they are not twins in the path .

We may assume  is in .
Then, every neighbor of  in  is adjacent to , in particular,  is adjacent to . 
This contradicts the assumption that  is induced.

This proves the claim. Since there is no such a path , we can apply Reduction Rule~\ref{rrule:p5middle} to remove .
It contradicts the assumption that  is reduced.
We conclude that  and  are neighbor bags.
\end{proof}



Finally, we claim that our instance has the desired inseparability property. We formalize and prove this property below.


\begin{lemma}
\label{lem:simplifynearsattached2}
Let  be a bag and let  be a connected component of  such that
 contains exactly one -attached bag .
Then there is no -separator bag.
\end{lemma}
\begin{proof}
For contradiction, suppose that there is a -separator bag. We choose such a bag   with minimum .
From the choice of , there is no -separator bag. 

We verify preconditions of Lemma~\ref{lem:simplifynearsattached} for  and .
Clearly,  has exactly one -attached bag , 
and there is no -separator bag. 
To see that  is a simple star bag, let us assume that there is a connected component  of  where the center of  is adjacent to ; if there is no such a component, it is clear by definition.
As  contains no -attached bag, by Lemma~\ref{lem:smallbranch},  consists of one bag and  is a simple star bag.

By applying Lemma~\ref{lem:simplifynearsattached} for  and , 
we can observe that  is a star whose leaf is adjacent to , and 
there is a leaf bag  where the center of  is adjacent to .
We can also observe that  is a simple star bag.
By Lemma~\ref{lem:smallbranch}, there is no connected component of  pending to leaves of  other than the leaf adjacent to its parent.

Note that every bag  in  has two neighbor bags, because it is not a -separator bag and by Lemma~\ref{lem:smallbranch} there is no other component  pending to .
Therefore  by Lemma~\ref{rrule:bypassing1},  is a neighbor bag of .

Now, by Lemma~\ref{lem:starcenter}, there is no non--attached twin class in , 
which means that the unmarked vertices of  form a unique -attached twin class. 
Then, we can apply Reduction Rule~\ref{rrule:bypassing2} to , a contradiction.

We conclude that there is no -separator bag.
\end{proof}






\subsection{Connected components with two -attached bags}\label{sec:twinclass}




This section is devoted to showing that if  is reduced and contains two distinct -attached classes, then we can apply a reduction rule.
Suppose  is reduced and contains two distinct -attached classes, and we choose a root bag of . 
Let  be a farthest bag from the root bag 
 such that there are two descendant bags  and  of  having distinct -attached twin classes  and , respectively.

First, we verify that the distance from  to  in  is at most .


\begin{lemma}
\label{lem:distancec1c2}
The distance from  to  in  is at most .
\end{lemma}
\begin{proof}
Let us take a shortest sequence of twin classes  from  to  in  such that for  with ,  is complete to  if  and they are anti-complete, otherwise. We note that each  except  and  corresponds to a -separator bag.
 Clearly, at most one of  possibly has a neighbor in  because  is the unique -attached twin class in  if .
 By (3) and (4) of Lemma~\ref{lem:shortdistance}, 
 the length from  to  in  cannot be  or , and thus  cannot be  or .
 Also, by Lemma~\ref{lem:simplifynearsattached2}, 
 we know that there is no -separator bag when .
 Thus,  cannot be larger than .
So, the distance from  to  in  is at most  . 
\end{proof}

\begin{PROP}
  \label{prop:anticomplete1}
 The bag  is not a -separator bag.
\end{PROP}
\begin{proof}
For each  let .
Since by Lemma~\ref{lem:distancec1c2} the distance from  to  is at most , it follows from Observation~\ref{obser:separator} that there exists at most one -separator bag. Suppose that  is the -separator bag.
Note that  for some  because  and  are distinct. Without loss of generality, we assume that .
We verify the proposition by a sequence of claims.

\begin{claim}
 is not a star bag whose leaf is adjacent to .
\end{claim}
\begin{clproof}
Suppose  is a star bag whose leaf is adjacent to . As  is the unique -attached bag in , 
by Lemma~\ref{lem:smallbranch}, there is no bag pending to a leaf of .
Also, the center of  is marked, otherwise, we can apply Reduction Rule~\ref{rrule:leaftoS}, and by Lemma~\ref{lem:smallbranch},  is a simple star bag.
Therefore,  consists of leaves of , and  is a -separator bag. But it contradicts the assumption that  and  is the only -separator bag.
\end{clproof}

Note that  is either a complete graph, or a star whose center is adjacent to .
We observe that  contains a non--attached twin class.
\begin{claim}
 contains a non--attached twin class.
\end{claim}
\begin{clproof}
Suppose for contradiction that  contains no non--attached twin class, that is,  is exactly the set of unmarked vertices of . 
Let  be the parent bag of . 
If  is not a star whose center is adjacent to , then we can apply Reduction Rule~\ref{rrule:leaf}.
We may assume  is a star whose center is adjacent to . But in this case, , 
and thus,  has no -attached twin classes. By Lemma~\ref{lem:smallbranch},  has exactly two neighbor bags, and by Lemma~\ref{lem:starcenter},  
it contains no non--attached twin class. But this is impossible.
We conclude that  contains a non--attached twin class.
\end{clproof}

\begin{claim}
There is a vertex  in  contained in a complete bag such that  has no neighbors in .
\end{claim}
\begin{clproof}
If  is a complete bag, then the non--attached twin class is contained in .
Assume  is a star. Since  is a star whose leaf is adjacent to , 
there is at least one bag in . Moreover, there is no star bag  in  whose center is adjacent to  by Lemma~\ref{lem:starcenter}.
Therefore, there is at least one complete bag in , which contains a vertex in .
We choose  to be such a vertex. Then  is a vertex in  having no neighbors in , and 
also contained in a complete bag.
\end{clproof}

Since  is contained in a complete bag,  has a neighbor in .
By (1) of Lemma~\ref{lem:relationt1t2}, we have .
Since  and  is adjacent to a vertex in , by (2) of Lemma~\ref{lem:relationt1t2},  should be adjacent to all vertices in , which contradicts the fact that  has no neighbors in .
\end{proof}

The following lemma describes all possible cases.

\begin{lemma}\label{lem:twosattachedbags}
Let  be a farthest bag from the root bag 
 such that there are two descendant bags  and  of  having distinct -attached twin classes  and , respectively.
Then  and one of the following happens:
\begin{enumerate}
\item The distance from  to  in  is  and the unique -separator bag is contained in  for some .
\item   is complete to  and either
\begin{itemize}
\item  is a star bag and  is the set consisting of the center of  for some , or
\item   is a star bag whose center is adjacent to  for some .
\end{itemize}
\item   is complete to  and  is a complete bag.
\end{enumerate}
\end{lemma}
\begin{proof}


 
By Lemma~\ref{lem:twinclassreduction}, each bag contains at most one -attached twin class and it follows that .
By Lemma~\ref{lem:distancec1c2}
the distance from  to  in  is at most .
Suppose that the distance from  to  in  is . 
Then, there is a unique -separator bag in .
By Proposition~\ref{prop:anticomplete1},  cannot be the -separator bag.
Thus, the unique -separator bag in contained in  for some .
If the distance from  to  is , then  is complete to , 
and in this case if  is a star, then its center either consists of one class  or is adjacent to one of  and .
\end{proof}


We show that in each of three cases in Lemma~\ref{lem:twosattachedbags}, 
we can apply a reduction rule.







\begin{PROP}
  \label{prop:anticomplete2}
  Suppose the distance from  to  in  is  and the unique -separator bag is contained in .
  Then for every induced path  with , 
  there is a bypassing vertex for  and .
\end{PROP}
\begin{proof}
Let  for each . 
 We start by proving the following claim.
\begin{claim}
The bag  is the -separator bag.
\end{claim}
\begin{clproof}
For a contradiction, suppose that  is not the -separator bag and let  be the -separator bag.
Then  is a -separator bag.
However, since  has exactly one -attached bag , 
by Lemma~\ref{lem:simplifynearsattached2}, 
there is no -separator bag, which is a contradiction.
\end{clproof}

By Lemma~\ref{lem:smallbranch},  has no child pending to a leaf of .
If the center of  is unmarked, then we can reduce it using Reduction Rule~\ref{rrule:leaftoS}.
Thus there is component attached to the center of , and by Lemma~\ref{lem:smallbranch} this component is a single leaf bag.
We call the leaf bag , and let  be the set of unmarked vertices of . Note that  is a non--attached twin class.
Also, by Lemma~\ref{lem:starcenter},  contains no non--attached twin class.




Suppose  there is an induced path  with .
We want to show that there is a bypassing vertex for  and .
Observe that every neighbor of  in  is either in  or in .
As  is a twin class, it contains at most one of  and .
If  and  are contained in , then by Lemma~\ref{lem:twovertexinS}, there is a bypassing vertex for .
Thus, we may assume that one of  and  is contained in  and the other is contained in .




By symmetry, we may assume that  and .
Note that since ,  has no neighbors in .
Furthermore, 
as the distance from  to  is exactly ,  is complete to .
It implies that .

By (1) of Lemma~\ref{lem:relationt1t2}, we have .
Let .
We divide cases depending on whether  is in  or .

\subparagraph{\textbf{Case 1.} :}  

Note that  and .
Since  is not adjacent to , by (2) of Lemma~\ref{lem:relationt1t2},
 is not adjacent to  as well. As  and  are neighbors of  and  is reduced under Branching Rule~\ref{brule:reducecomponent}, 
 and  are contained in the same connected component of . Moreover, since  is reduced under Branching Rule~\ref{brule:threevertices}, 
there is no induced path of length at least  from  to  in , 
and thus the distance from  to  in  is exactly .
Let  be an induced path for some . 

If  is contained in , then  should be adjacent to  by (2) of Lemma~\ref{lem:relationt1t2}.
Thus,   is not contained in .
If , then by (2) of Lemma~\ref{lem:relationt1t2},  is adjacent to , but  has no neighbors in , a contradiction. Lastly, assume that . In this case,  by (3) of Lemma~\ref{lem:relationt1t2} with , 
 is adjacent to , again a contradiction. 

\subparagraph{\textbf{Case 2.} :}  
Let .
If  and  have a common neighbor  in  that is adjacent to neither  nor , then
 is isomorphic to the house.
So, there are no such vertices. 
This implies that for each , there is no complete bag in ,
and if  or  is a complete bag, then it contains no non--attached twin class.

We claim that  contains at most  bags.

\begin{claim}\label{claim:restricted}
 contains at most  bags, and when it contains  bags, the bag in  is a star bag whose center is adjacent to .
\end{claim}
\begin{clproof}
Suppose  contains more than  bags, and let  be the parent bag of  and  be the parent of .
As   contains no complete bags,
both  and  are star bags. 
Thus,  is a star bag whose center is adjacent to .
Such a bag  does not exist by Lemma~\ref{lem:starcenter}. It proves the claim.
\end{clproof}

In particular, Claim~\ref{claim:restricted} implies that every neighbor of  is either in  or not contained in the component of  containing .






We divide into subcases depending on the shape of .

\subparagraph{\textbf{Case 2-1.}  is a complete bag:}

First assume that . As  contains no non--attached twin class and ,
 is in the neighborhood of  in . Then  is adjacent to end vertices of , and by Lemma~\ref{lem:dhobs}, 
 contains a small DH obstruction. This is a contradiction.
We may assume .

As  is canonical, 
the parent bag  of  is a star bag.
We claim that . Suppose , that is,  is contained in .
Since there is no -separator bag, the center of  is adjacent to either  or .
As  is the unique -attached bag in , 
by Lemma~\ref{lem:smallbranch},  has exactly two neighbor bags.
Also, again by Lemma~\ref{lem:smallbranch},  is a leaf bag.
Therefore, by Lemma~\ref{lem:starcenter}, the center of  is not adjacent to .
On the other hand, if the center of  is adjacent to , 
then we can apply Reduction Rule~\ref{rrule:leaf}, as  contains no non--attached twin class, which is a contradiction.
Thus, we have , and the same argument using Reduction Rule~\ref{rrule:leaf} implies that the center of  is adjacent to .
Then  should be contained in  and adjacent to , which is impossible.




\subparagraph{\textbf{Case 2-2.}  is a star bag:}
First assume that . 
As  is complete to , the center of  is adjacent to .
If  and  are neighbor bags, then the marked edge connecting them can be recomposable.
Thus, in this case,  contains  bags. Let  be the bag in , and  be an unmarked vertex in .
It is not difficult to observe that  should be adjacent to , since  cannot be in .
Then  is an induced path and  is adjacent to its end vertices, and thus  contains a small DH obstruction.
It is a contradiction. We may assume .


Similar to the case when  is a complete bag, we can show that the parent of  is  and  is a star whose center is adjacent to .
In this case,  is contained in the non--attached twin class, as  is not adjacent to .
As  has at least  vertices, there is a vertex  where  is adjacent to , but not adjacent to .
If  is adjacent to , then we have a small DH obstruction.
Otherwise,  is an induced path, and  is adjacent to its end vertices. 
It contradicts the non-existence of a small DH obstruction.

\medskip
We conclude that for every induced path  with , 
  there is a bypassing vertex for  and .
\end{proof}

Next, we deal with the case when  is complete to . We prove the case when  is a star bag in Proposition~\ref{prop:complete2}, 
and the case when  is a complete bag in Proposition~\ref{prop:complete1}. 


\begin{PROP}
  \label{prop:complete2}
  Suppose  is complete to , and either 
 \begin{itemize}
 \item  and  consists of the center of  or
 \item  , and  is a star bag   whose center is adjacent to .
 \end{itemize}
  Then for every induced path  with , 
  there is a bypassing vertex for  and .  
  \end{PROP}
\begin{proof}
For each  and let  and .
We first observe that there is no child bag  pending to  except the possible child in  when .
Suppose there is such a bag, and let  be the connected component of  containing .
As  has no -attached bags by the choice of , by Lemma~\ref{lem:smallbranch},  is a star whose center is adjacent to . 
Then  becomes a -separator bag, contradicting the assumption that  is complete to .
We conclude the claim.

As  is the unique -attached bag in  when , 
by Lemma~\ref{lem:smallbranch}, 
every bag in  has exactly two neighbor bags.
Since either 
\begin{itemize}
 \item  and  consists of the center of  or
\item  is a star bag whose center is adjacent to ,
\end{itemize}
every neighbor of a vertex in  is contained in  or .

Suppose there is an induced path  with . We will show that there is a bypassing vertex for .
If , then it follows from Lemma~\ref{lem:twovertexinS}. 
Without loss of generality, we assume .
We distinguish cases depending on whether  or not.



\subparagraph{\textbf{Case 1.}  :}  

We choose a vertex .
Since every neighbor of a vertex in  is contained in  or , 
 is contained in  or .
Since  is not contained in ,  has no neighbors in .

We claim that .

\begin{claim}
.
\end{claim}
\begin{clproof}
Suppose  and  are disjoint.
As  is complete to  and  is reduced under Branching Rule~\ref{brule:reducecomponent},  and  are contained in the same conncected component of .
Let  be a shortest path from  to  in .
Since  is an induced path of length at least  and  is adjacent to its end vertices, 
 contains a DH obstruction, which contradicts the assumption that  is reduced under Branching Rule~\ref{brule:threevertices}.
Therefore, . 
\end{clproof}

Let .
Clearly,  is not adjacent to .
Next, we claim that .

\begin{claim}
.
\end{claim}
\begin{clproof}
Suppose there is a vertex  in . 
Since , if , then  by (1) of Lemma~\ref{lem:completerelationt1t2}. 
But  has no neighbors in . Thus, we have .
Since  or , there is an induced path  for some .
We assume ; the symmetric argument holds when .
If , then by (1) of Lemma~\ref{lem:completerelationt1t2},  or  should be adjacent to , which is a contradiction.
On the other hand, by (2) of Lemma~\ref{lem:completerelationt1t2},  cannot be in .
We conclude that .
\end{clproof}


Suppose that . 
We know that  and  are contained in some bags in .
By symmetry, we assume , where  and  are bags containing  and , respectively.

Since ,  is not a complete bag, and thus it is a star whose center is adjacent to .
In case when , we may assume that  is contained in the non--attached twin class. 
Then  should be adjacent to , contradiction.

We may assume that  and , because . Since , we have .
We can observe that  has no neighbors in  as  is contained in some bag in , and it is not contained in .

If  is adjacent to , 
then  is adjacent to the end vertices of an induced path , implying that  has a small DH obstruction, which is a contradiction.
We may assume that  is not adjacent to .
One can observe that in this case,  is in some bag of , and thus
 is adjacent to . It is a contradiction.




\subparagraph{\textbf{Case 2.}  :}  
This implies that there are no complete bags in , and especially, if  or  is a complete bag, then it has no non--attached twin class.
We first claim that ,  and  is the parent bag of .

\begin{claim}
,  and  is the parent bag of .
\end{claim}
\begin{clproof}
Suppose , and let  be the parent bag of .
As  is complete to , 
 is a star whose center is adjacent to  or a complete bag. 
By Lemma~\ref{lem:smallbranch}, there is no child of , and thus  is a leaf bag.
Also, by Lemma~\ref{lem:smallbranch},  has exactly two neighbor bags unless .

We observe that  should be a star whose center is adjacent to .
When  is a star,  is a star whose center is adjacent to , as there is no complete bag in .
When  is a complete bag, if  is a star whose leaf is adjacent to , 
then we can apply Reduction Rule~\ref{rrule:leaf} because  contains no non--attached twin class.
Thus  is a star whose center is adjacent to .
Due to Lemma~\ref{lem:starcenter}, such a bag  cannot exist. 
Therefore, we have . 

Since  by Lemma~\ref{lem:twosattachedbags}, we have . 
In the same reason, there are no bags in . It implies that  is the parent of .
\end{clproof}






  \begin{figure}[t]
    \centering
      \includegraphics[scale=0.5]{prop422}


    \caption{The case when  in Proposition~\ref{prop:complete2}.} \label{fig:prop422}
      
  \end{figure}

See Figure~\ref{fig:prop422} for an illustration.
Recall that  is contained in . Thus,  should be contained in  or
 has the non--attached twin class and  is contained in this class.

Suppose . Then . 
If  has a neighbor in , then we have a bypassing vertex.
So, we may assume that  has no neighbors in .
As  is complete to  and Branching Rule~\ref{brule:reducecomponent} is exhaustively applied,
 and  are contained in the same connected component of .
Let  be a shortest path from  to  in .
Then  is an induced path of length at least  and  is adjacent to its end vertices, and therefore  contains a DH obstruction 
which contradicts the exhaustive application of Branching Rule~\ref{brule:threevertices}. 

Now, suppose . It implies that  is a star bag having a non--attached twin class, and  is contained in the set.
Then  should be a common neighbor of  and .  
Let  be the shortest path from  to  in .
First assume that  has a neighbor in . Among neighbors of  in , we choose the vertex  such that the distance between  and  in  is shortest.
Let  be the subpath of  from  to .
Then  is an induced path of length at least  since  is not adjacent to , 
and  is adjacent to its end vertices. 
Therefore,  contains a DH obstruction, contradicting our assumption that 
 is reduced under Branching Rule~\ref{brule:threevertices}.
We may assume that  has no neighbors in .
In this case, 
 is an induced path of length at least , and  is adjacent to its end vertices.
Therefore,  contains a DH obstruction, contradicting our assumption that 
 is reduced under Branching Rule~\ref{brule:threevertices}.

\medskip
We conclude that for every induced path  with , 
  there is a bypassing vertex for  and .
\end{proof}




\begin{PROP}
\label{prop:complete1}
  Suppose  is complete to , , and  is a complete bag. 
  Then  contains a non--attached twin class  and 
  for every induced path  with , 
  there is a bypassing vertex for  and . 
   \end{PROP}
  \begin{proof}
For each  and let  and .
Let  be the parent bag of .
As  is complete to ,
 is either a complete bag or a star whose center is adjacent to .
We observe that  has a non--attached class, and .

\begin{claim}
 has a non--attached class.
\end{claim}
\begin{clproof}
Suppose for contradiction that  has no non--attached class, that is, its unmarked vertices form one -attached twin class.
We verify that there is no child bag of .
Suppose for contradiction that there is a child  of  and let  be the component .
Since  contains no -attached bag, by Lemma~\ref{lem:smallbranch},  should be a star whose center is adjacent to , which is a contradiction.
Also, every bag in   is not a -separator bag, and by Lemma~\ref{lem:smallbranch},  it has exactly two neighbor bags.

Assume  is a complete bag. Then its parent  is a star 
and thus . 
By Lemma~\ref{lem:starcenter},
 the center of  is not adjacent to .
Thus, the center of  is adjacent to its parent.
As  has no non--attached class, 
we can apply Reduction Rule~\ref{rrule:leaf} to reduce the split decomposition, which is a contradiction.
Assume  is a star whose center is adjacent to its parent.
Similarly, by Lemma~\ref{lem:starcenter}, 
 is not a star whose center is adjacent to .
 Thus, we may assume the parent of  is a complete bag, but in this case, we can apply Reduction Rule~\ref{rrule:leaf}.

We conclude that  contains a non--attached twin class.
\end{clproof}

Let  be the non--attached twin class in . As  and  have the same neighborhood in , 
 is complete to .
  
\begin{claim}
 contains a vertex that has no neighbors in .
\end{claim}
\begin{clproof}
If  is a star bag, then  and .
If  is a complete bag, then since  is a complete bag,  and the unmarked vertices in  are contained in .
Let  be an unmarked vertex in .
By the choice of ,  has no neighbors in .
\end{clproof}

Let 
 be a vertex in  having no neighbors in .





Suppose there is an induced path  with . 
We will prove that there is a bypassing vertex for  and .
Let  be the connected component of  containing the parent of .

We claim that  and  are contained in .
\begin{claim}
 and  are contained in .
\end{claim}
\begin{clproof}
Note that  or  is contained in either  or .
If both  and  are contained in , then this is clear.
If both  and  are contained in , then without loss of generality, 
we assume that  where  and  are bags containing  and , respectively.
Since there is no -separator bag,  should be adjacent to , a contradiction.
Lastly, we assume that one of  and  is in , but the other is in .
By symmetry we assume . 
If , then  is contained in a complete bag, and thus  is adjacent to .
If , then  is clearly adjacent to , as  is a complete bag.
Both cases are not possible.

We conclude that  and  are contained in .
\end{clproof}


Suppose . Let .
Since we know that  has no neighbors in , we have , 
and by (1) of Lemma~\ref{lem:completerelationt1t2},
 should be adjacent to both  and . Thus,  is a bypassing vertex, as required.
We may assume that .

Since  is complete to , by Branching Rule~\ref{brule:reducecomponent}, 
we know that  and  are contained in the same connected component of .
Let  be a shortest path from  to .
If  has length at least , then  is an induced cycle of length at least , contradicting our assumption that  is reduced under Branching Rule~\ref{brule:threevertices}.
Thus,  has length . Let  be the path where  is a neighbor of  for each .
 
 Note that ,  and .
 If  or  is adjacent to one of  and , the by (1) of Lemma~\ref{lem:completerelationt1t2}, 
 it is adjacent to both  and .
 This means that it becomes a bypassing vertex, as required. 
 Therefore, we may assume that for each ,  is adjacent to neither  nor .
 So,  is an induced path of length , and  is adjacent to its end vertices. 
 It implies that  has a small DH obstruction, which is a contradiction.


\medskip
We conclude that for every induced path  with , 
  there is a bypassing vertex for  and .
\end{proof}

\begin{PROP}\label{prop:finish}
If  is a reduced canonical split decomposition of a connected component of ,
then  is empty.
\end{PROP}
\begin{proof}
Suppose a reduced canonical split decomposition  of a connected component of  contains a vertex.
If it contains at most one -attached twin class, then Reduction Rule~\ref{rrule:dhcomponent} can be applied.
We may assume that  contains at least two -attached twin classes.

We choose a root bag, and let  be a farthest bag from the root bag 
 such that there are two descendant bags  and  of  having distinct -attached twin classes  and , respectively.
By Lemma~\ref{lem:twosattachedbags},
, and one of the following happens:
\begin{enumerate}[(1)]
\item The distance from  to  in  is  and the unique -separator bag is contained in  for some .
\item   is complete to  and either
\begin{itemize}
\item  is a star bag and  is the set consisting of the center of  for some , or
\item  is a star bag whose center is adjacent to  for some .
\end{itemize}
\item   is complete to  and  is a complete bag.
\end{enumerate}
If (1) happens, then by Proposition~\ref{prop:anticomplete2}, Reduction Rule~\ref{rrule:p5middle} is applied to remove .
If (2) happens, then by Proposition~\ref{prop:complete2}, Reduction Rule~\ref{rrule:p5middle} is applied to remove .
If (3) happens, then by Proposition~\ref{prop:complete1},  Reduction Rule~\ref{rrule:p5middle} is applied to remove the non--attached twin class in one of  and .
But this contradicts the assumption that  is reduced.
\end{proof}





\section{The Algorithm, Lower Bounds and Applications}
\label{sec:completing}

Our goal in this section is to give a proof of our main result, Theorem~\ref{thm:main}, and obtain corresponding lower bounds. 

\subsection{The Algorithm}
Below, we use the presented reduction and branching rules to give an algorithm for \disjointDHVD. This is then followed by a proof of our main algorithmic result.




\begin{theorem}
\label{thm:main2}
\disjointDHVD\ can be solved in time .
\end{theorem}
\begin{proof}
Let  be an instance of \disjointDHVD. 
We exhaustively apply Branching Rules~\ref{brule:threevertices}--\ref{brule:reducecomponent} and Reduction Rules~\ref{rrule:dhcomponent}--\ref{rrule:bypassing2}.
We prove that one of rules can be applied until  is empty or  becomes .
In both cases, we can test whether the resulting instance is distance-hereditary or not in polynomial time, and output an answer.
Suppose  does not reach . Then by Proposition~\ref{prop:finish},  contains no vertices.
Therefore, the resulting instance satisfies that  is empty, as mentioned. 






We argue that the runtime bounds hold. For convenience, we will denote  by  and  by .
First notice that each branching rule reduces either  or the number of connected components in  and branches into at most  subinstances. Moreover, none of the reduction rules change  or the number of components in . Hence a branching rule is applied at most  times. Similarly, every reduction rule reduces either the number of vertices in  or the number of bags in canonical split decomposition of . Therefore, it is not hard to observe that the branching tree of the algorithm will have at most  leaves and each leaf will be in depth at most  and hence the branching tree will have at most  nodes. In the following we will discuss that the runtime in every node will not exceed . In each node, we go through the branching and reduction rules, in the order they are introduced in the paper, and apply the first rule that can be applied. 
Let us start with detecting and applying  Branching Rule~\ref{brule:threevertices}.  
Our algorithm is going through all sets  of size at most  and checking, whether  is distance-hereditary. It follows from Theorems~\ref{thm:dahlhaus} and \ref{thm:bouchet} that we check whether a graph is distance-hereditary in time . If the graph is not distance-hereditary, application of the rule can be done in constant time. Hence, the Branching Rule~\ref{brule:threevertices} can be verified in time  . Similarly, for Branching Rule~\ref{brule:reducecomponent} for every set  of size at most , we can in time , e.g. using breadth-first search, verify that the neighborhood of  is in the same connected component and the same running bound follows.  
After verifying that the graph is actually reduced under Branching Rules~\ref{brule:threevertices} and \ref{brule:reducecomponent} it follows from Proposition~\ref{prop:applicationRR} that we can in time  either apply one of Reduction Rules~\ref{rrule:dhcomponent}--\ref{rrule:bypassing2} or correctly deduce that the graph is reduced also under Reduction Rules~\ref{rrule:dhcomponent}--\ref{rrule:bypassing2}. Hence, the whole algorithm for \disjointDHVD\ can be implemented in time .
\end{proof}

\begin{theorem}
\label{thm:main}
\DHVD\ can be solved in time .
\end{theorem}
\begin{proof}
We apply the standard iterative compressing technique. The algorithm involves a two-step reduction of \DHVD: we first reduce \DHVD\ to the \textsc{Compression} problem, which reduces to
\disjointDHVD. 

For convenience, we denote for this proof  and .
Fix an arbitrary labeling  of  and let  be a the graph  for . From  up to , we consider the following the \textsc{Compression} problem for \DHVD: given a graph  and  such that  is distance-hereditary and , we aim to find a set  such that  is distance-hereditary and , if one exists, and output \textsc{No} otherwise. Since distance-hereditary graphs are closed under taking induced subgraphs,  is \textsc{Yes}-instance of \DHVD\ if and only if  is a \textsc{Yes}-instance for 
\textsc{Compression} for all , where  is a legitimate instance. Hence we correctly output that  is \textsc{No}-instance of \DHVD\ if  is a  \textsc{No}-instance for some . Moreover, if  is a solution to the -th instance of  \textsc{Compression}, then  is a legitimate instance for -th
instance of  \textsc{Compression}. 

Given an instance  of \textsc{Compression}, we enumerate all possible intersections  of  and a desired solution to . For each guessed set , we solve the instance  of \disjointDHVD\ using Theorem~\ref{thm:main2}. Note that   is \textsc{Yes}-instance if and only if   is \textsc{Yes}-instance for some . If  is a solution to  , then clearly  is a solution to the instance  of \textsc{Compression}. Conversely, if  is a solution to  the instance  of \textsc{Compression} then for the set  the instance  is  \textsc{Yes}-instance for \disjointDHVD. Therefore, using the algorithm from Theorem~\ref{thm:main2} we can correctly solve \DHVD.

It remains to prove the complexity of the algorithm. Given an instance  we guess at most  sets  of size  for each . Note that  has size at most , and in particular  has at most  connected components. Therefore, we can solve the resulting instance  of \disjointDHVD\ in time . Summing up, \DHVD\ can be solved by running an algorithm for \textsc{Compression} at most  times, which yields the claimed running time 

Note that the equality follows from the use of the binomial theorem, which states that  (see, e.g., Chapter 10 in Cygan~et~al.~\cite{CyganFKLMPPS15}).
\end{proof}

\subsection{Lower Bounds}
Here we will present our lower bound result, based on the well-established exponential time hypothesis~\cite{ImpagliazzoRF2001}.

\begin{definition}[Exponential Time Hypothesis (ETH)]
 There exists a constant  such that \textsc{3-CNF-SAT} with  variables and  clauses cannot be solved in time .
\end{definition}

Our result uses the fact that the classical \textsc{Vertex Cover} problem cannot be solved in subexponential time under ETH.
  
\begin{theorem}[Cai and Juedes \cite{CaiJuedes03}]\label{thm:ethVC}
There is no  algorithm for \textsc{Vertex Cover}, unless ETH fails.
\end{theorem}

\begin{theorem}
\label{thm:main3}
There is no  algorithm for \DHVD\, unless ETH fails.
\end{theorem}
\begin{proof}
For a graph , we will denote  by  and  by .
For contradiction suppose there exists an algorithm for solving the \DHVD\ problem in time . We show that we can solve \textsc{Vertex Cover} in time . Let  be an instance of \textsc{Vertex Cover} problem. We construct a graph  as follows. We replace every edge  of  with two vertex disjoint paths of length  between  and . Note that for every edge  in  the two disjoint paths of length  in  form an induced subgraph isomorphic to . Moreover we have . We claim that  has a vertex set  of size at most  such that  has no edges if and only if  has a vertex deletion set of size at most  to a distance-hereditary graph. Suppose that  has such a vertex cover . It is easy to confirm that  is a  disjoint union of subdivisions of stars, which is distance hereditary. 

For the converse direction, suppose  has a distance-hereditary vertex deletion set  of size at most . 
Let us fix an arbitrary edge  in . Note that no DH obstruction contains a pendant vertex. Hence we observe that if  is a DH obstruction containing a vertex  on a shortest  path in , then  contains both vertices  and  as well. Therefore, if , then also graphs  and  are distance-hereditary. Since the choice of the edge  was arbitrary, we can find a set , such that , , and  is a distance-hereditary graph. Clearly for every edge  in ,  contains  or , otherwise  contains an induced . We conclude that  is a vertex cover of , which finishes the proof. 
\end{proof}

\subsection{Example Applications}
There is an established line of research studying the algorithmic applications of vertex deletion sets to specific graph classes~\cite{GajarskyHOORRVS13,EibenGanianSzeider15,EibenGanianSzeider15b,FellowsLMRS08}.
In this context, it is natural to ask whether Theorem~\ref{thm:main} allows the development of single-exponential algorithms for problems parameterized by the size of a vertex deletion set to distance-hereditary graphs. 

Clearly, any problem that is FPT when parameterized by clique-width (and rank-width) must also be FPT when parameterized by the size of a vertex deletion set to distance-hereditary graphs. However, the existence of a single-exponential FPT algorithm parameterized by clique-width does not immediately imply that the problem also admits a single-exponential FPT algorithm parameterized by our parameter, since the addition of  vertices to a graph may increase clique-width by a factor of up to ~\cite{Gurski2016}. On the other hand, known FPT algorithms parameterized by rank-width usually do not have a single-exponential dependency on the parameter. As a consequence, one cannot obtain the following examples of single-exponential algorithms by simply solving these problems via known FPT algorithms parameterized by rank-width or clique-width.

\begin{lemma}
 \textsc{Vertex Cover} and \textsc{-Coloring} admit a single-exponential FPT algorithm when parameteried by the size of a vertex deletion set to distance-hereditary graphs.
\end{lemma}

\begin{proof}
For each of the presented problems, we always begin by invoking Theorem~\ref{thm:main} to compute a vertex deletion set  to distance-hereditary graphs of size at most .

In the case of \textsc{Vertex Cover}, we can apply standard branching algorithms to solve the problem. In particular, we begin by branching over the at most  options of how  intersects with a (hypothetical) solution; let  be one such subset of  and let . 
After branching we proceed by testing the validity of a branch (i.e., whether each edge with both endpoints in  is covered by ). For each valid branch, we delete  and the set  of all neighbors of  in . Next, we find a minimum vertex cover  in the remaining distance-hereditary subgraph of  in polynomial time. Finally, for each branch we compare the desired solution size with ; clearly, a  graph is a YES-instance of \textsc{Vertex Cover} if and only if at least one selection of  results in a value of  which is at most the desired solution size.

For \textsc{-Coloring}, we also begin by branching over the at most  -colorings of . For each such proper -coloring of , we construct an instance of \textsc{-List Coloring} as follows: the input graph is , and the list of admissible colors for each vertex  contains all colors that are not used by a neighbor of  in . The \textsc{-List Coloring} problem can be solved in polynomial time on distance-hereditary graphs: indeed, the problem can easily be reduced to the MSO model checking problem over labeled graphs with (at most)  labels. Since  has rank-width at most , the polynomial-time tractability of the problem follows for instance from Courcelle's Theorem~\cite{CourcelleMR00}. All that remains now is to test whether at least one of the considered  branches give rise to a yes-instance of \textsc{-List Coloring} on .
\end{proof}


\section{Concluding Notes}
We conclude with a few remarks on why we believe that the presented algorithm is of high interest. First, it intrinsically exploits the properties guaranteed by distinct, seemingly unrelated characterizations of distance-hereditary graphs; this approach can likely be used to design or improve algorithms for other vertex deletion problems. Second, it uses highly nontrivial reduction rules which simplify canonical split decompositions, and an adaptation or extension of the presented rules could be highly relevant for other graph classes characterized by special canonical split decompositions, such as parity graphs~\cite{CiceroneS1999} or circle graphs~\cite{GSH1989}. Third, it is the first of its kind which targets a ``full'' class of graphs of bounded rank-width (contrasting previous results for specific subclasses of graphs of rank-width~~\cite{HuffnerKMN2010, AgrawalKLS16,KimK2015, KanteKKP2015}).

It is worth noting that there remains a number of interesting open problems in this general area. Perhaps the most prominent one is the question of whether vertex deletion to graphs of rank-width , for any constant , admits a single-exponential fixed-parameter algorithm. Our algorithm represents the first steps in this general direction. 
Recently, Kim and the third author~\cite{KimK2016} announced a polynomial kernel for \textsc{Distance-Hereditary Vertex Deletion}.
The existence of a polynomial kernel or an approximation algorithm for such vertex deletion problems for  remains open.

\section*{Acknowledgment}
The first and second authors were supported by the Austrian Science Fund (FWF, projects P26696 and W1255-N23). Robert Ganian is also affiliated with FI MU, Brno, Czech Republic. The third author was supported by ERC Starting Grant PARAMTIGHT (No. 280152) and also supported by the European Research Council (ERC) under the European Union's Horizon 2020 research and innovation programme (ERC consolidator grant DISTRUCT, agreement No. 648527). 

The authors would like to thank the anonymous reviewers for helpful suggestions. In particular, one reviewer suggested merging several reduction rules to one, and this makes it easier to understand the algorithm.
The third author would like to thank Eun Jung Kim and Sang-il Oum for initial discussions on this problem.

\section*{References}

\begin{thebibliography}{10}

\bibitem{AKK2014}
I.~Adler, M.~M. Kant\'e, and O.~Kwon.
\newblock Linear rank-width of distance-hereditary graphs {I}. {A}
  polynomial-time algorithm.
\newblock {\em Algorithmica}, 78(1):342--377, 2017.

\bibitem{AgrawalKLS16}
A.~Agrawal, S.~Kolay, D.~Lokshtanov, and S.~Saurabh.
\newblock A faster {FPT} algorithm and a smaller kernel for block graph vertex
  deletion.
\newblock In {\em {LATIN} 2016: Theoretical Informatics - 12th Latin American
  Symposium}, volume 9644 of {\em LNCS}, pages 1--13. Springer, 2016.

\bibitem{BM1986}
H.-J. Bandelt and H.~M. Mulder.
\newblock Distance-hereditary graphs.
\newblock {\em J. Comb. Theory Ser. {B}}, 41(2):182--208, 1986.

\bibitem{Bouchet1988a}
A.~Bouchet.
\newblock Transforming trees by successive local complementations.
\newblock {\em J. Graph Theory}, 12(2):195--207, 1988.

\bibitem{CaiJuedes03}
L.~Cai and D.~Juedes.
\newblock On the existence of subexponential parameterized algorithms.
\newblock {\em J. Comput. System Sci.}, 67(4):789--807, 2003.

\bibitem{CiceroneS1999}
S.~Cicerone and G.~Di~Stefano.
\newblock On the extension of bipartite to parity graphs.
\newblock {\em Discrete Appl. Math.}, 95(1-3):181--195, 1999.

\bibitem{CogisT2005}
O.~Cogis and E.~Thierry.
\newblock Computing maximum stable sets for distance-hereditary graphs.
\newblock {\em Discrete Optim.}, 2(2):185--188, 2005.

\bibitem{CourcelleMR2000}
B.~Courcelle, J.~A. Makowsky, and U.~Rotics.
\newblock Linear time solvable optimization problems on graphs of bounded
  clique-width.
\newblock {\em Theory Comput. Syst.}, 33(2):125--150, 2000.

\bibitem{CourcelleMR00}
B.~Courcelle, J.~A. Makowsky, and U.~Rotics.
\newblock Linear time solvable optimization problems on graphs of bounded
  clique-width.
\newblock {\em Theory Comput. Syst.}, 33(2):125--150, 2000.

\bibitem{Cunningham1982}
W.~H. Cunningham.
\newblock Decomposition of directed graphs.
\newblock {\em SIAM J. Algebraic Discrete Methods}, 3(2):214--228, 1982.

\bibitem{CunninghamE80}
W.~H. Cunningham and J.~Edmonds.
\newblock A combinatorial decomposition theory.
\newblock {\em Canad. J. Math.}, 32(3):734--765, 1980.

\bibitem{CyganFKLMPPS15}
M.~Cygan, F.~V. Fomin, L.~Kowalik, D.~Lokshtanov, D.~Marx, M.~Pilipczuk,
  M.~Pilipczuk, and S.~Saurabh.
\newblock {\em Parameterized Algorithms}.
\newblock Springer, 2015.

\bibitem{Dahlhaus00}
E.~Dahlhaus.
\newblock Parallel algorithms for hierarchical clustering and applications to
  split decomposition and parity graph recognition.
\newblock {\em J. Algorithms}, 36(2):205--240, 2000.

\bibitem{DowneyF13}
R.~G. Downey and M.~R. Fellows.
\newblock {\em Fundamentals of Parameterized Complexity}.
\newblock Texts in Computer Science. Springer, 2013.

\bibitem{EibenGanianSzeider15b}
E.~Eiben, R.~Ganian, and S.~Szeider.
\newblock Meta-kernelization using well-structured modulators.
\newblock In {\em 10th International Symposium on Parameterized and Exact
  Computation, {IPEC} 2015}, volume~43 of {\em LIPIcs}, pages 114--126. Schloss
  Dagstuhl - Leibniz-Zentrum fuer Informatik, 2015.

\bibitem{EibenGanianSzeider15}
E.~Eiben, R.~Ganian, and S.~Szeider.
\newblock Solving problems on graphs of high rank-width.
\newblock In {\em Algorithms and Data Structures - 14th International
  Symposium, {WADS} 2015}, volume 9214 of {\em LNCS}, pages 314--326. Springer,
  2015.

\bibitem{FellowsLMRS08}
M.~R. Fellows, D.~Lokshtanov, N.~Misra, F.~A. Rosamond, and S.~Saurabh.
\newblock Graph layout problems parameterized by vertex cover.
\newblock In {\em Algorithms and Computation, 19th International Symposium,
  {ISAAC} 2008}, volume 5369 of {\em LNCS}, pages 294--305. Springer, 2008.

\bibitem{FominLMS12}
F.~V. Fomin, D.~Lokshtanov, N.~Misra, and S.~Saurabh.
\newblock Planar {F}-{D}eletion: Approximation, {K}ernelization and {O}ptimal
  {FPT} {A}lgorithms.
\newblock In {\em 53rd Annual {IEEE} Symposium on Foundations of Computer
  Science, {FOCS} 2012}, pages 470--479. {IEEE} Computer Society, 2012.

\bibitem{GSH1989}
C.~P. Gabor, K.~J. Supowit, and W.~L. Hsu.
\newblock Recognizing circle graphs in polynomial time.
\newblock {\em J. Assoc. Comput. Mach.}, 36(3):435--473, 1989.

\bibitem{GajarskyHOORRVS13}
J.~Gajarsk{\'{y}}, P.~Hlin{\v{e}}n{\'y}, J.~Obdr\v{z}\'{a}lek, S.~Ordyniak,
  F.~Reidl, P.~Rossmanith, F.~S. Villaamil, and S.~Sikdar.
\newblock Kernelization using structural parameters on sparse graph classes.
\newblock In {\em Algorithms - {ESA} 2013 - 21st Annual European Symposium,
  Sophia Antipolis}, volume 8125 of {\em LNCS}, pages 529--540. Springer, 2013.

\bibitem{GanianH10}
R.~Ganian and P.~Hlin{\v{e}}n{\'y}.
\newblock On parse trees and {M}yhill-{N}erode-type tools for handling graphs
  of bounded rank-width.
\newblock {\em Discrete Appl. Math.}, 158(7):851--867, 2010.

\bibitem{GassnerH2008}
E.~Gassner and J.~Hatzl.
\newblock A parity domination problem in graphs with bounded treewidth and
  distance-hereditary graphs.
\newblock {\em Computing}, 82(2-3):171--187, 2008.

\bibitem{GP2012}
E.~Gioan and C.~Paul.
\newblock Split decomposition and graph-labelled trees: characterizations and
  fully dynamic algorithms for totally decomposable graphs.
\newblock {\em Discrete Appl. Math.}, 160(6):708--733, 2012.

\bibitem{Gurski2016}
F.~Gurski.
\newblock The behavior of clique-width under graph operations and graph
  transformations.
\newblock {\em Theory Comput. Syst.}, 60(2):346--376, 2017.

\bibitem{Heggernes2013}
P.~Heggernes, P.~van~'t Hof, B.~M.~P. Jansen, S.~Kratsch, and Y.~Villanger.
\newblock Parameterized complexity of vertex deletion into perfect graph
  classes.
\newblock {\em Theoret. Comput. Sci.}, 511:172--180, 2013.

\bibitem{howorka77}
E.~Howorka.
\newblock A characterization of distance-hereditary graphs.
\newblock {\em Quart. J. Math. Oxford Ser. (2)}, 28(112):417--420, 1977.

\bibitem{HsiehHH2006}
S.-Y. Hsieh, C.-W. Ho, T.-S. Hsu, and M.-T. Ko.
\newblock The {H}amiltonian problem on distance-hereditary graphs.
\newblock {\em Discrete Appl. Math.}, 154(3):508--524, 2006.

\bibitem{HuffnerKMN2010}
F.~H{\"u}ffner, C.~Komusiewicz, H.~Moser, and R.~Niedermeier.
\newblock Fixed-parameter algorithms for cluster vertex deletion.
\newblock {\em Theory Comput. Syst.}, 47(1):196--217, 2010.

\bibitem{HungC2005}
R.-W. Hung and M.-S. Chang.
\newblock Linear-time algorithms for the {H}amiltonian problems on
  distance-hereditary graphs.
\newblock {\em Theoret. Comput. Sci.}, 341(1-3):411--440, 2005.

\bibitem{RuoM2007}
R.-W. Hung and M.-S. Chang.
\newblock Finding a minimum path cover of a distance-hereditary graph in
  polynomial time.
\newblock {\em Discrete Appl. Math.}, 155(17):2242--2256, 2007.

\bibitem{ImpagliazzoRF2001}
R.~Impagliazzo, R.~Paturi, and F.~Zane.
\newblock Which problems have strongly exponential complexity?
\newblock {\em J. Comput. System Sci.}, 63(4):512--530, 2001.
\newblock Special issue on FOCS 98 (Palo Alto, CA).

\bibitem{KanteKKP2015}
M.~M. Kant\'e, E.~J. Kim, O.~Kwon, and C.~Paul.
\newblock An {FPT} {A}lgorithm and a {P}olynomial {K}ernel for {L}inear
  {R}ankwidth-1 {V}ertex {D}eletion.
\newblock {\em Algorithmica}, 79(1):66--95, 2017.

\bibitem{KimK2015}
E.~J. Kim and O.~Kwon.
\newblock A {P}olynomial {K}ernel for {B}lock {G}raph {D}eletion.
\newblock {\em Algorithmica}, 79(1):251--270, 2017.

\bibitem{KimK2016}
E.~J. Kim and O.~Kwon.
\newblock A polynomial kernel for distance-hereditary vertex deletion.
\newblock {\em Algorithms and Data Structures Symposium, {WADS} 2017, To
  Appear}, arxiv.org/abs/1610.07229, 2017.

\bibitem{KLPRRSS13}
E.~J. Kim, A.~Langer, C.~Paul, F.~Reidl, P.~Rossmanith, I.~Sau, and S.~Sikdar.
\newblock Linear kernels and single-exponential algorithms via protrusion
  decompositions.
\newblock {\em ACM Trans. Algorithms}, 12(2):Art. 21, 41, 2016.

\bibitem{NakanoUU2007}
S.-i. Nakano, R.~Uehara, and T.~Uno.
\newblock A new approach to graph recognition and applications to
  distance-hereditary graphs.
\newblock In {\em Theory and applications of models of computation}, volume
  4484 of {\em Lecture Notes in Comput. Sci.}, pages 115--127. Springer,
  Berlin, 2007.

\bibitem{Oum05}
S.~Oum.
\newblock Rank-width and vertex-minors.
\newblock {\em J. Comb. Theory, Ser. {B}}, 95(1):79--100, 2005.

\bibitem{OS2004}
S.~Oum and P.~Seymour.
\newblock Approximating clique-width and branch-width.
\newblock {\em J. Comb. Theory Ser. {B}}, 96(4):514--528, 2006.

\bibitem{ReedSV2004}
B.~Reed, K.~Smith, and A.~Vetta.
\newblock Finding odd cycle transversals.
\newblock {\em Oper. Res. Lett.}, 32(4):299--301, 2004.

\bibitem{RobertsonS1986}
N.~Robertson and P.~D. Seymour.
\newblock Graph minors. {V}. {E}xcluding a planar graph.
\newblock {\em J. Comb. Theory Ser. {B}}, 41(1):92--114, 1986.

\bibitem{MarcD2007}
M.~Tedder and D.~Corneil.
\newblock An optimal, edges-only fully dynamic algorithm for
  distance-hereditary graphs.
\newblock In {\em S{TACS} 2007}, volume 4393 of {\em Lecture Notes in Comput.
  Sci.}, pages 344--355. Springer, Berlin, 2007.

\end{thebibliography}


\end{document}
