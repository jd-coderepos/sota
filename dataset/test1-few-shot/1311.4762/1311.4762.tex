\documentclass[a4paper,11pt]{article}

\pdfoutput=1

 
\newcommand{\defaultlinespread}{\linespread{0.94}} 


\usepackage{geometry}
\geometry{papersize={199.5mm,282.15mm}, layoutsize={210mm,297mm} }

\hoffset=-4.5mm 
\textwidth=145mm 
\textheight=202mm 
\setlength{\columnsep}{4.0mm}
\defaultlinespread{}
\newenvironment{mtwocols}{\begin{multicols}{2}}{\end{multicols}}



 
\usepackage{array} 
\newcolumntype{L}[1]{>{\raggedright\let\newline\\\arraybackslash\hspace{0pt}}m{#1}}
\newcolumntype{C}[1]{>{\centering\let\newline\\\arraybackslash\hspace{0pt}}m{#1}}
\newcolumntype{R}[1]{>{\raggedleft\let\newline\\\arraybackslash\hspace{0pt}}m{#1}}

\newcolumntype{M}{>{\centering\arraybackslash}}

\usepackage{environ}
\NewEnviron{mtvDisplayMath}{\vspace{4mm}
\noindent\colorbox{magenta!43!cyan!10!white}{\begin{tabular}{@{}M@{}} \-1mm]
\end{tabular}}
\vspace{1mm}
}

\usepackage{amsmath}
\usepackage{amsfonts} 

\usepackage{authblk}


\usepackage[T1]{fontenc}
\usepackage{fourier}

\usepackage{fancyhdr}
\usepackage{forloop}
\usepackage[usenames,dvipsnames]{color}
\usepackage[hyphens]{url}
\usepackage{hyphenat} 

\usepackage[pagebackref=true]{hyperref}
\definecolor{Brown}{rgb}{0.61,0.08,0.23} 
\definecolor{Blues}{rgb}{.12,.12,.48}
\hypersetup{
  colorlinks   = true,  urlcolor     = Blues, linkcolor    = Brown, citecolor    = Brown  }


\usepackage{cite}
\usepackage{multicol}
\usepackage{fancybox}
\usepackage{tikz}
\usepackage{float}

\newcommand{\golabel}[1]{\ensuremath{{}^{{}^{\hypertarget{refback:#1}{\,}\!}}\mkern-1.6mu}}
\newcommand{\gotonolabel}[1]{\hyperlink{ref:#1}{#1}}
\newcommand{\goto}[1]{\golabel{#1}\gotonolabel{#1}}
\newcommand{\activeitem}[1]{\item \hypertarget{ref:#1}{}}

\newcommand{\Nbibitem}[2]{\bibitem{#2}}

\newcommand{\gofrom}[2]{}

\newcommand{\gourl}[1]{\url{#1}}
\newcommand{\goDOI}[1]{\href{http://dx.doi.org/#1}{DOI:#1}}
\newcommand{\goScholar}[1]{{Google~Scholar}:~\href{http://scholar.google.com/scholar?cluster=#1}{#1}}

\newcommand{\partitle}[1]{\noindent {\large\textbf{\textsf{#1}}}}
\usepackage{titlesec}
\titleformat*{\section}{\Large\bf\sffamily}

\newcommand{\reftitle}[1]{\textbf{#1}}
\newcounter{mylabelnumber}
\newcommand{\golabelrange}[2]{{\forloop{mylabelnumber}{#1}{\value{mylabelnumber} < #2}{\golabel{\arabic{mylabelnumber}}}\golabel{#2}}}
\newcommand{\gotorange}[2]{{\gotonolabel{#1}--\gotonolabel{#2}\golabelrange{#1}{#2}}}


\newcommand{\DTM}{\ensuremath{\mbox{D-}\!\mbox{TM}}}

\newcommand{\checkendpaper}{
\setlength{\leftmargin}{0pt}
\setlength{\itemindent}{0em}

\setlength{\itemsep}{0mm}
\setlength{\parskip}{0mm}
\setlength{\parsep}{0mm} 
\linespread{0.92} 
}


\renewcommand*{\backref}[1]{}
    \renewcommand*{\backrefalt}[4]{\ifcase #1 (Not cited).
    \or
        (page~#2).\else
        (pages~#2).
    \fi}


\usepackage{lettrine} 
\usepackage{colortbl} 


\makeatletter
\def\newparshape{\parshape\@npshape0{}}
\def\@npshape#1#2#3{\ifx\\#3\expandafter\@@@npshape\else\expandafter\@@npshape\fi
{#1}{#2}{#3}}
\def\@@npshape#1#2#3#4#5{\ifnum#3>\z@\expandafter\@firstoftwo\else\expandafter\@secondoftwo\fi
{\expandafter\@@npshape\expandafter{\the\numexpr#1+1\relax}{#2 #4 #5}{\numexpr#3-1\relax}{#4}{#5}}{\@npshape{#1}{#2}}}
\def\@@@npshape#1#2#3{#1 #2 }
\makeatother

\newcommand{\myquote}[2]{{
\rlap{\hspace{\dimexpr0.63\columnwidth+\columnsep}\vspace{-#1}\colorbox{black!10}{
\em\small\begin{tabular}{L{\dimexpr0.63\columnwidth-\columnsep}}
\mbox{\hspace{1pt}}\2mm]
\end{tabular}}}}}



\pagestyle{fancy}
\fancyhead{}
\fancyhead[C]{\footnotesize de Rigo, D., 2013. Software Uncertainty in Integrated Environmental Modelling: the role of Semantics and Open Science. {\em Geophys Res Abstr 15}, \href{http://meetingorganizer.copernicus.org/EGU2013/EGU2013-13292.pdf}{13292}+. ISSN 1607-7962. (EGU General Assembly 2013).\vspace{1mm}}

\title{\vspace{-20mm}\bf \Large \textsf{Software Uncertainty in Integrated Environmental Modelling: the role of Semantics and Open Science}\vspace{3mm}}
\author[1,2]{{\Large \textsf{Daniele de Rigo}} \vspace{2mm}}

\affil[1]{\small \;European Commission, Joint Research Centre, Institute for Environment and Sustainability\\

Via E. Fermi 2749, I-21027 Ispra (VA), Italy\smallskip }
\affil[2]{\;Politecnico di Milano, Dipartimento di Elettronica, Informazione e Bioingegneria\\

Via Ponzio 34/5, I-20133 Milano, Italy\vspace{-4mm}}
\date{}
\begin{document}

  \maketitle
  
\noindent\colorbox{black!10}{\small
\arrayrulecolor{white}\color{black!80}\begin{tabular}{|p{138mm}|}
\hline
\vspace{0mm}Copyright {\copyright} 2013 Daniele de Rigo.\6pt]
This is the author's version of the work. The definitive version has been published in the Vol. 15 of Geophysical Research Abstracts (ISSN 1607-7962) and presented at the European Geosciences Union (EGU) General Assembly 2013, Vienna, Austria, 07--12 April 2013 \\
{\url{http://www.egu2013.eu/}}\7pt]
{\sloppy \parbox[l]{138mm}{
   de Rigo, D., 2013. 
   {\bf Software Uncertainty in Integrated Environmental Modelling: the role of Semantics and Open Science}. 
   {\em Geophys Res Abstr 15}, 
   \href{http://meetingorganizer.copernicus.org/EGU2013/EGU2013-13292.pdf}{13292}+ }
}\1mm]
\hline
\end{tabular}}


\vspace{9mm}\begin{mtwocols} 
\noindent \lettrine[lines=2]{C}{omputational} aspects increasingly shape environmental sciences \cite{Casagrandi_Guariso_2009}. Actually, transdisciplinary modelling of complex and uncertain environmental systems is challenging computational science (CS) and also the science-policy interface \cite{de_Rigo_F10002013,Gomes_2009,Easterbrook_Johns_2009,Hamarat_etal_2012,Bankes_2002,Kandlikar_etal_2005}.

\smallskip

Large spatial-scale problems falling within this category -- i.e. wide-scale transdisciplinary modelling for environment (WSTMe) \cite{de_Rigo_etal_EGU2013,Rodriguez_Aseretto_etal_2013,de_Rigo_etal_exp2013}
  -- often deal with factors \textbf{(a)} for which deep-uncertainty \cite{de_Rigo_F10002013,Lempert_2002,Kandlikar_etal_2005,Gober_Kirkwood_2010}
  may prevent usual statistical analysis of modelled quantities and need different ways for providing policy-making with science-based~support. 
Here, practical recommendations are proposed for tempering a peculiar -- not infrequently underestimated -- source of uncertainty. 
Software errors in complex WSTMe may subtly affect the outcomes with possible consequences even on collective environmental decision-making. 
Semantic transparency in CS \cite{de_Rigo_F10002013,de_Rigo_etal_EGU2013,de_Rigo_etal_exp2013,de_Rigo_iEMSs2012,de_Rigo_SemAP2012} and free software \cite{FSF_2012,Stallman_2009} are discussed as possible mitigations \textbf{(b)}.

\bigskip\medskip

\partitle{Software uncertainty,}\vspace{0mm}

\partitle{black-boxes and free software}

\bigskip
\noindent Integrated natural resources modelling and management (INRMM) \cite{de_Rigo_INRMM2012} frequently exploits chains of nontrivial data-transformation models ({\DTM}), each of them affected by uncertainties and errors.  

\medskip

\noindent Those {\DTM} chains may be packaged as monolithic specialized models, maybe only accessible as black-box executables (if accessible at all) \cite{Morin_etal_2012}. For end-users, black-boxes merely transform inputs in the final outputs, relying on classical peer-reviewed publications for describing the internal mechanism. 

\smallskip

\noindent While software tautologically plays a vital role in CS, it is often neglected in favour of more theoretical~\mbox{aspects}. 

\end{mtwocols}

\newcommand{\myrightdef}[1]{\ensuremath{\text{\parbox[t]{72mm}{#1}}}}
\newcommand{\myrightdefAO}[2]{\ensuremath{\text{\begin{tabular}{@{}L{#1}@{}}{\nohyphens{#2}}\end{tabular}}}}
\newcommand{\myrightdefA}[1]{\ensuremath{\hspace{3mm}\myrightdefAO{82mm}{#1}\vspace{2mm}}}
\newcommand{\myrightdefB}[1]{\ensuremath{\hspace{9mm}\myrightdefAO{76mm}{#1}\vspace{2mm}}}
\vspace{-1mm}

\begin{mtvDisplayMath}
\mbox{\textbf{(a)}} \qquad
{\begin{array}{ll}
{ \begin{array}{l}
  \mbox{Complexity} \\
  \end{array} } & { = \quad \left \{
{ \begin{array}{l}
  \myrightdefA{Transdisciplinary integration (e.g. systems of systems)}\\
  \myrightdefA{Environmental system(s) heterogeneity (e.g. geospatial fragmentation)}\\
  \myrightdefA{Data heterogeneity (formats, definitions, spatiotemporal density, ...)}\\
  \myrightdefA{Software complexity (algorithms, dependencies, languages, interfaces, ...)}\3mm]
\-2mm]
  \end{array} }
\right . } 
\3mm]
{ \begin{array}{l}
  \mbox{Dynamic} \\
  \mbox{behaviour} \\
  \end{array} } & { = \quad \left \{
{ \begin{array}{l}
  \myrightdefA{Uncertainty propagation via:} \\
  \myrightdefB{Propagation in the network of interconnected WSTMe components \cite{de_Rigo_F10002013,de_Rigo_SemAP2012,Green_Sadedin_2005,de_Rigo_INRMM2012,Baker_etal_2012,de_Rigo_etal_IPRMW2012,Thompson_etal_2009,FAO_2005,Bonan_2008}} \\
  \myrightdefB{Iterations within nonlinear optimization steps \cite{Hamarat_etal_2012,Ferreira_etal_2012,de_Rigo_2001,Bond_etal_2011,de_Rigo_etal_2005,Phillis_Kouikoglou_2012,Cavallo_Nardo_2008,Castelletti_etal_2008}} \\
  \myrightdefB{Data fusion, harmonization, integration \cite{Rodriguez_Aseretto_etal_2013,Kempeneers_etal_2011,Sedano_etal_2012,de_Rigo_Bosco_2011,Voinov_Shugart_2013}} \\
  \myrightdefB{Steps for computing and aggregating criteria and indices \cite{Bankes_2002,Lempert_2002,Kandlikar_etal_2005,Mendoza_Martins_2006,OFarrell_Anderson_2010,Dale_Beyeler_2001,Gilbert_2010}} \3mm] 
\3mm] 
\3mm] 
\1mm]
\ought sem( Y, f, \theta,X,\zeta\, ) \\
\end{array}} \right . }
\end{array} }\\
\mbox{\textbf{(b)}}
\2mm]
 X_{i} \in \mathbf{C}^{N_{i1} \times \cdots \times N_{in_i}} 
\mbox{ is a multi-dimensional array (e.g. a two-dimensional } \\
  \mbox{\qquad raster layer)} \2mm]
\mbox{the modal/deontic logic operator \;} {\ought p } \mbox{\; means: it ought to be that } p .
\end{array} \right . }
\end{array} }
\end{mtvDisplayMath}


\begin{mtwocols}
\noindent Applying this paradigm to WSTMe, an alternative strategy to black-boxes would suggest exposing not only final outputs but also key intermediate layers of data and information along with the corresponding free software {\DTM} modules. 
\end{mtwocols}

\begin{mtwocols}  
\newparshape{12}{0pt}{0.63\columnwidth}{1}{0pt}{\columnwidth}\\
\myquote{50mm}{Software errors in complex WSTMe may subtly affect the outcomes with possible consequences even on collective environmental decision-making''\0.2mm]
``The chain of free-software modules should be transparent}
\noindent A concise, semantically-enhanced modularization \cite{de_Rigo_iEMSs2012,de_Rigo_SemAP2012} may help not only to see the code (as a very basic prerequisite for semantic transparency) but also to understand -- and correct -- it \cite{Iverson_1980}. Semantically-enhanced, concise modularization is e.g. supported by semantic array programming (SemAP) \cite{de_Rigo_iEMSs2012,de_Rigo_SemAP2012} and its extension to geospatial problems \cite{de_Rigo_etal_EGU2013,de_Rigo_etal_exp2013}. 

\smallskip

\newparshape{6}{0pt}{\columnwidth}{12}{0.37\columnwidth}{0.63\columnwidth}{1}{0pt}{\columnwidth}\\
Some WSTMe may surely be classified in the subset of software systems which ``are growing well past the ability of a small group of people to completely understand the content'', while ``data from these systems are often used for critical decision \mbox{making}''~\cite{Sanders_Kelly_2008}.  
In this context, the further uncertainty arising from the unpredicted ``(not to say unpredictable)'' \cite{Cerf_2012} behaviour of software errors propagation in WSTMe should be explicitly considered as software uncertainty~\cite{Lehman_1989,Lehman_Ramil_2002} (see {\bf b}). 
The data and information flow of a black-box {\DTM} is often a (hidden) composition of {\DTM} modules:
\begin{figure}[H]
\vspace{-0.5mm} 
\centerline{\includegraphics[scale=0.18]{blackbox_semap_DTM.png}}
\vspace{-2.5mm} 
\end{figure}
\noindent This chain of free-software {\DTM} modules
(each of them semantically-enhanced) should be transparent:
\end{mtwocols}


\begin{figure}[H]
\vspace{6mm} 
\centerline{\includegraphics[scale=0.18]{nonblackbox_semap_DTM.png}}
\vspace{5mm} 
\end{figure}
\bigskip


\begin{mtwocols}
\partitle{Semantics and design diversity}
\bigskip\vspace{0mm}

\noindent Silent faults \cite{Hook_Kelly_2009} are a critical class of software errors altering computation output without evident symptoms -- such as computation premature interruption (exceptions, error messages, ...), obviously unrealistic results or computation patterns (e.g. noticeably shorter/longer or endless computations). As it has been underlined, ``many scientific results are corrupted, perhaps fatally so, by undiscovered mistakes in the software used to calculate and present those results''~\cite{Hatton_2007}. 

\newparshape{15}{0pt}{0.63\columnwidth}{4}{0pt}{\columnwidth}{15}{0.37\columnwidth}{0.63\columnwidth}{1}{0pt}{\columnwidth}\\
\myquote{67mm}{\vspace{0mm}Semantic modularization might help to catch at least a subset of silent faults, when misusing intermediate data outside the expected semantic context''\1mm]
``Where the complexity and scale may lead to deep uncertainty, techniques such as ensemble modelling may be recommendable\vspace{1mm}}
\noindent Despite the ubiquity of software errors~\cite{Lehman_1989,Lehman_Ramil_2002,Hook_Kelly_2009,Hatton_2007,Hatton_1997,Hatton_2012,Lehman_1996,Oberkampf_etal_2002,Wilson_2006}, the structural role of scientific software uncertainty seems dramatically underestimated  \cite{de_Rigo_F10002013,Cerf_2012}. Semantic {\DTM} modularization might help to catch at least a subset of silent faults, when misusing intermediate data outside the expected semantic context of a given {\DTM} module \textbf{(b)}. 
Where the complexity and scale of WSTMe may lead unavoidable software-uncertainty to induce or worsen deep-uncertainty \cite{de_Rigo_F10002013}, techniques such as ensemble modelling may be recommendable \cite{Lempert_2002,Kandlikar_etal_2005,Gober_Kirkwood_2010}. Adapting those techniques for glancing at the software-uncertainty of a given WSTMe would imply availability of multiple instances (implementations) of the same abstract WSTMe. 
Independently re-implementing the same WSTMe (design diversity \cite{Rebaudengo_2011}) might of course be extremely expensive. However, partly independent re-implementations of critical {\DTM} modules may be more affordable and examples of comparison between supposedly equivalent {\DTM} algorithms seem to corroborate the interest of this research option \cite{Cai_etal_2012,Beaudette_2008,Barnes_Jones_2011}.
\end{mtwocols}

\medskip
\checkendpaper{}

\begin{footnotesize}


\renewcommand*\labelenumi{[\theenumi]}


\raggedright
\nohyphens{
\begin{thebibliography}{}
\bibitem{Casagrandi_Guariso_2009} Casagrandi, R., Guariso, G., 2009. \textbf{Impact of ICT in environmental sciences: A citation analysis 1990-2007.} \emph{Environmental Modelling \& Software 24} (7), 865-871. \goDOI{10.1016/j.envsoft.2008.11.013} \goScholar{10214045160670186637}

\bibitem{de_Rigo_F10002013} de Rigo, D., (exp.) 2013. \textbf{Behind the horizon of reproducible integrated environmental modelling at European scale: ethics and practice of scientific knowledge freedom.} \emph{F1000 Research}. Submitted

\bibitem{Gomes_2009} Gomes, C. P., 2009. \textbf{Computational sustainability: Computational methods for a sustainable environment, economy, and society.} \emph{The Bridge 39} (4), 5-13. \gourl {http://www.nae.edu/File.aspx?id=17673}

\bibitem{Easterbrook_Johns_2009} Easterbrook, S. M., Johns, T. C., 2009. \textbf{Engineering the software for understanding climate change.} \emph{Computing in Science \& Engineering 11} (6), 65-74. \goDOI{10.1109/MCSE.2009.193} \goScholar{5281658010101603741}

\bibitem{Hamarat_etal_2012} Hamarat, C., Kwakkel, J. H., Pruyt, E., 2012. \textbf{Adaptive robust design under deep uncertainty.} Technological Forecasting and Social Change. \goDOI{10.1016/j.techfore.2012.10.004} \goScholar{1225653171705658912}

\bibitem{Bankes_2002} Bankes, S. C., 2002. \textbf{Tools and techniques for developing policies for complex and uncertain systems.} \emph{Proceedings of the National Academy of Sciences of the United States of America 99} (Suppl 3), 7263-7266. \goDOI{10.1073/pnas.092081399} \goScholar{2377027768488561757}

\bibitem{Kandlikar_etal_2005} Kandlikar, M., Risbey, J., Dessai, S., 2005. \textbf{Representing and communicating deep uncertainty in climate-change assessments.} \emph{Comptes Rendus Geoscience 337} (4), 443-455. \goDOI{10.1016/j.crte.2004.10.010} \goScholar{8596177795590020821}

\bibitem{de_Rigo_etal_EGU2013} de Rigo, D., Corti, P., Caudullo, G., McInerney, D., Di Leo, M., San-Miguel-Ayanz, J., 2013. \textbf{Toward Open Science at the European scale: Geospatial Semantic Array Programming for Integrated Environmental Modelling.} \emph{Geophysical Research Abstracts 15}, \href{http://meetingorganizer.copernicus.org/EGU2013/EGU2013-13245.pdf}{13245}+ \goDOI{10.6084/m9.figshare.155703} \goScholar{17118262245556811911}

\bibitem{Rodriguez_Aseretto_etal_2013} Rodriguez Aseretto, D., Di Leo, M., de Rigo, D., Corti, P., McInerney, D., Camia, A., San Miguel-Ayanz, J., 2013. \textbf{Free and Open Source Software underpinning the European Forest Data Centre.} \emph{Geophysical Research Abstracts 15}, \href{http://meetingorganizer.copernicus.org/EGU2013/EGU2013-12101.pdf}{12101}+ \goDOI{10.6084/m9.figshare.155700} \goScholar{14482024956822192435}

\bibitem{de_Rigo_etal_exp2013} de Rigo, D., Corti, P., Caudullo, G., McInerney, D., Di Leo, M., San-Miguel-Ayanz, J., (exp.) 2013. \textbf{Supporting Environmental Modelling and Science-Policy Interface at European Scale with Geospatial Semantic Array Programming.} In prep.

\bibitem{Lempert_2002} Lempert, R. J., 2002. \textbf{A new decision sciences for complex systems.} \emph{Proceedings of the National Academy of Sciences of the United States of America 99} (Suppl 3), 7309-7313. \goDOI{10.1073/pnas.082081699} \goScholar{3105059191542042306}


\bibitem{Gober_Kirkwood_2010} Gober, P., Kirkwood, C. W., 2010. \textbf{Vulnerability assessment of climate-induced water shortage in Phoenix.} \emph{Proceedings of the National Academy of Sciences 107} (50), 21295-21299. \goDOI{10.1073/pnas.0911113107} \goScholar{13404393835259418552}

\bibitem{de_Rigo_iEMSs2012} de Rigo, D., 2012. \textbf{Semantic Array Programming for Environmental Modelling: Application of the Mastrave library.} In: Seppelt, R., Voinov, A. A., Lange, S., Bankamp, D. (Eds.), International Environmental Modelling and Software Society (iEMSs) 2012 International Congress on Environmental Modelling and Software. Managing Resources of a Limited Planet: Pathways and Visions under Uncertainty, Sixth Biennial Meeting. pp. 1167-1176. \gourl {http://www.iemss.org/iemss2012/proceedings/D3\_1\_0715\_deRigo.pdf} \goScholar{6628751141895151391}

\bibitem{de_Rigo_SemAP2012} de Rigo, D., 2012. \textbf{Semantic Array Programming with Mastrave - Introduction to Semantic Computational Modelling.} \gourl {http://mastrave.org/doc/MTV-1.012-1} \goScholar{6848554969929557252}

\bibitem{FSF_2012} Free Software Foundation, 2012. \textbf{What is free software?} \gourl {http://www.gnu.org/philosophy/free-sw.html} (revision 1.118 archived at \gourl {http://www.webcitation.org/6DXqCFAN3} ) \goScholar{7470103647360812109}

\bibitem{Stallman_2009} Stallman, R. M., 2009. \textbf{Viewpoint: Why "open source" misses the point of free software.} \emph{Communications of the ACM 52} (6), 31-33. \goDOI{10.1145/1516046.1516058} (free access version: \gourl {http://www.gnu.org/philosophy/open-source-misses-the-point.html} ) \goScholar{17751536887456926788}

\bibitem{de_Rigo_INRMM2012} de Rigo, D., 2012. \textbf{Integrated Natural Resources Modelling and Management: minimal redefinition of a known challenge for environmental modelling.} Excerpt from the \emph{Call for a shared research agenda toward scientific knowledge freedom}, Maieutike Research Initiative. \gourl {http://www.citeulike.org/groupfunc/15400/home}

\bibitem{Morin_etal_2012} Morin, A., Urban, J., Adams, P. D., Foster, I., Sali, A., Baker, D., Sliz, P., 2012. \textbf{Shining light into black boxes.} \emph{Science 336} (6078), 159-160. \goDOI{10.1126/science.1218263} \goScholar{12575758499484368256}


\bibitem{Lempert_Schlesinger_2001} Lempert, R., Schlesinger, M. E., 2001. \textbf{Climate-change strategy needs to be robust.} \emph{Nature 412} (6845), 375. \goDOI{10.1038/35086617} \goScholar{3059012525521592083}

\bibitem{Shell_2012} Shell, K. M., 2012. \textbf{Constraining cloud feedbacks.} \emph{Science 338} (6108), 755-756. \goDOI{10.1126/science.1231083} \goScholar{17586687489186735372}

\bibitem{van_der_Sluijs_2012} van der Sluijs, J. P., 2012. \textbf{Uncertainty and dissent in climate risk assessment: A Post-Normal perspective.} \emph{Nature and Culture 7} (2), 174-195. \goDOI{10.3167/nc.2012.070204} \goScholar{2453345655060253463}

\bibitem{Lenton_etal_2008} Lenton, T. M., Held, H., Kriegler, E., Hall, J. W., Lucht, W., Rahmstorf, S., Schellnhuber, H. J., 2008. \textbf{Tipping elements in the earth's climate system.} \emph{Proceedings of the National Academy of Sciences 105} (6), 1786-1793. \goDOI{10.1073/pnas.0705414105} \goScholar{9014054360036592667}

\bibitem{Hastings_Wysham_2010} Hastings, A., Wysham, D. B., 2010. \textbf{Regime shifts in ecological systems can occur with no warning.} \emph{Ecology Letters 13} (4), 464-472. \goDOI{10.1111/j.1461-0248.2010.01439.x} \goScholar{10306576888461248930}

\bibitem{Barnosky_etal_2012} Barnosky, A. D., Hadly, E. A., Bascompte, J., Berlow, E. L., Brown, J. H., Fortelius, M., Getz, W. M., Harte, J., Hastings, A., Marquet, P. A., Martinez, N. D., Mooers, A., Roopnarine, P., Vermeij, G., Williams, J. W., Gillespie, R., Kitzes, J., Marshall, C., Matzke, N., Mindell, D. P., Revilla, E., Smith, A. B., 2012. \textbf{Approaching a state shift in earth's biosphere.} \emph{Nature 486} (7401), 52-58. \goDOI{10.1038/nature11018} \goScholar{17836601480741683208}


\bibitem{Milly_etal_2008} Milly, P. C. D., Betancourt, J., Falkenmark, M., Hirsch, R. M., Kundzewicz, Z. W., Lettenmaier, D. P., Stouffer, R. J., 2008. \textbf{Stationarity is dead: Whither water management?} \emph{Science 319} (5863), 573-574. \goDOI{10.1126/science.1151915} \goScholar{16810436092791821916}

\bibitem{Sloan_Pelletier_2012} Sloan, S., Pelletier, J., 2012. \textbf{How accurately may we project tropical forest-cover change? A validation of a forward-looking baseline for REDD.} \emph{Global Environmental Change 22} (2), 440-453. \goDOI{10.1016/j.gloenvcha.2012.02.001} \goScholar{10187012353585922701}

\bibitem{Nabuurs_etal_2008} Nabuurs, G. J., van Putten, B., Knippers, T. S., Mohren, G. M. J., 2008. \textbf{Comparison of uncertainties in carbon sequestration estimates for a tropical and a temperate forest.} \emph{Forest Ecology and Management 256} (3), 237-245. \goDOI{10.1016/j.foreco.2008.04.010} \goScholar{6421873803139267676}

\bibitem{Green_Sadedin_2005} Green, D. G., Sadedin, S., 2005. \textbf{Interactions matter -- complexity in landscapes and ecosystems.} \emph{Ecological Complexity 2} (2), 117-130. \goDOI{10.1016/j.ecocom.2004.11.006} \goScholar{558815488529198010}


\bibitem{Baker_etal_2012} Baker, R., Koch, F., Kriticos, D., Rafoss, T., Venette, R., van der Werf, W. (Eds.), 2012. \textbf{Advancing risk assessment models for invasive alien species in the food chain: contending with climate change, economics and uncertainty.} \emph{Bioforsk FOKUS 7}. Bioforsk, Frederik A. Dahls vei 20, 1432 \r{A}s, Norway. \gourl {http://www.pestrisk.org/2012/BioforskFOKUS7-10\_IPRMW-VI.pdf}

\bibitem{de_Rigo_etal_IPRMW2012} de Rigo, D., Caudullo, G., San-Miguel-Ayanz, J., Stancanelli, G., 2012. \textbf{Mapping European forest tree species distribution to support pest risk assessment.} In: Baker, R., Koch, F., Kriticos, D., Rafoss, T., Venette, R., van der Werf, W. (Eds.), Advancing risk assessment models for invasive alien species in the food chain: contending with climate change, economics and uncertainty. \emph{Bioforsk FOKUS 7}. Bioforsk, Frederik A. Dahls vei 20, 1432 \r{a}s, Norway. \gourl {http://www.pestrisk.org/2012/BioforskFOKUS7-10\_IPRMW-VI.pdf} \goScholar{6508055261897528514}

\bibitem{Thompson_etal_2009} Thompson, I., Mackey, B., McNulty, S., Mosseler, A., 2009. \textbf{Forest resilience, biodiversity, and climate change: a synthesis of the biodiversity/resilience/stability relationship in forest ecosystems.} Vol. 43 of Technical Series. Secretariat of the Convention on Biological Diversity. ISBN: 9292251376 \goScholar{16541384134391848156}

\bibitem{FAO_2005} Center for International Forestry Research., FAO Regional Office for Asia and the Pacific, 2005. \textbf{Forests and floods: drowning in fiction or thriving on facts?} Center for International Forestry Research; Food and Agriculture Organization of the United Nations, Regional Office for Asia and the Pacific. \gourl {http://www.worldcat.org/isbn/9793361646}

\bibitem{Bonan_2008} Bonan, G. B., 2008. \textbf{Forests and climate change: Forcings, feedbacks, and the climate benefits of forests.} \emph{Science 320} (5882), 1444-1449. \goDOI{10.1126/science.1155121} \goScholar{12659549051072980947}

\bibitem{Ferreira_etal_2012} Ferreira, L., Constantino, M. F., Borges, J. G., Garcia-Gonzalo, J., 2012. \textbf{A stochastic dynamic programming approach to optimize Short-Rotation coppice systems management scheduling: An application to eucalypt plantations under wildfire risk in portugal.} \emph{Forest Science}, 353-365. \goDOI{10.5849/forsci.10-084} \goScholar{15823418833962927194}

\bibitem{de_Rigo_2001} de Rigo, D., Rizzoli, A. E., Soncini-Sessa, R., Weber, E., Zenesi, P., 2001. \textbf{Neuro-dynamic programming for the efficient management of reservoir networks.} In: Proceedings of MODSIM 2001, International Congress on Modelling and Simulation. Vol. 4. Modelling and Simulation Society of Australia and New Zealand, pp. 1949-1954. \goDOI {10.5281/zenodo.7481} \goScholar{16120008708786398621}

\bibitem{Bond_etal_2011} Bond, C. A., Champ, P., Meldrum, J., Schoettle, A., 2011. \textbf{Investigating the optimality of proactive management of an invasive forest pest.} In: Keane, R. E., Tomback, D. F., Murray, M. P., Smith, C. M. (Eds.), The future of high-elevation, five-needle white pines in Western North America: Proceedings of the High Five Symposium. U.S. Department of Agriculture, Forest Service, Rocky Mountain Research Station, pp. 295-302. \gourl {http://www.treesearch.fs.fed.us/pubs/38241} \goScholar{15999737409186644552}

\bibitem{de_Rigo_etal_2005} de Rigo, D., Castelletti, A., Rizzoli, A. E., Soncini-Sessa, R., Weber, E., 2005. \textbf{A selective improvement technique for fastening neuro-dynamic programming in water resources network management.} In: Z\'itek, P. (Ed.), Proceedings of the 16th IFAC World Congress. Vol. 16. International Federation of Automatic Control (IFAC), pp. 7-12. \goDOI{10.3182/20050703-6-CZ-1902.02172} \goScholar{3101827154043147997}

\bibitem{Phillis_Kouikoglou_2012} Phillis, Y. A., Kouikoglou, V. S., 2012. \textbf{System-of-Systems hierarchy of biodiversity conservation problems.} \emph{Ecological Modelling 235-236}, 36-48.  \goDOI{10.1016/j.ecolmodel.2012.03.032} \goScholar{14693363263013674243}

\bibitem{Cavallo_Nardo_2008} Cavallo, A., Nardo, A., 2008. \textbf{Optimal fuzzy management of reservoir based on genetic algorithm.} In: Lowen, R., Verschoren, A. (Eds.), Foundations of Generic Optimization. Vol. 24 of Mathematical Modelling: Theory and Applications. Springer Netherlands, pp. 139-159. \goDOI{10.1007/978-1-4020-6668-9\_2} \goScholar{596109054217603623}

\bibitem{Castelletti_etal_2008} Castelletti, A., de Rigo, D., Tepsich, L., Soncini-Sessa, R., Weber, E., 2008. \textbf{On-Line design of water reservoir policies based on inflow prediction.} In: Myung, C., Misra, P. (Eds.), Proceedings of the 17th IFAC World Congress. Vol. 17. International Federation of Automatic Control (IFAC), pp. 14540-14545. \goDOI{10.3182/20080706-5-KR-1001.02463} \goScholar{17611910683549006538}

\bibitem{Kempeneers_etal_2011} Kempeneers, P., Sedano, F., Seebach, L. M., Strobl, P., San-Miguel-Ayanz, J., 2011. \textbf{Data fusion of different spatial resolution remote sensing images applied to Forest-Type mapping.} \emph{IEEE Transactions on Geoscience and Remote Sensing 49} (12), 4977-4986. \goDOI{10.1109/TGRS.2011.2158548} \goScholar{1622266323511963573}

\bibitem{Sedano_etal_2012} Sedano, F., Kempeneers, P., Strobl, P., McInerney, D., San-Miguel-Ayanz, J., 2012. \textbf{Increasing Spatial Detail of Burned Scar Maps Using IRS-AWiFS Data for Mediterranean Europe.} \emph{Remote Sensing 4} (3), 726-744. \goDOI{10.3390/rs4030726} \goScholar{16654372463982241297}

\bibitem{de_Rigo_Bosco_2011} de Rigo, D., Bosco, C., 2011. \textbf{Architecture of a Pan-European Framework for Integrated Soil Water Erosion Assessment.} \emph{IFIP Advances in Information and Communication Technology 359}, 310-318. \goDOI{10.1007/978-3-642-22285-6\_34} \goScholar{3950024085016158193}

\bibitem{Voinov_Shugart_2013} Voinov, A., Shugart, H. H., 2013. \textbf{'Integronsters', integral and integrated modeling.} \emph{Environmental Modelling \& Software 39}, 149-158. \goDOI{10.1016/j.envsoft.2012.05.014} \goScholar{17438791388360294950}

\bibitem{Mendoza_Martins_2006} Mendoza, G. A., Martins, H., 2006. \textbf{Multi-criteria decision analysis in natural resource management: A critical review of methods and new modelling paradigms.} \emph{Forest Ecology and Management 230} (1-3), 1-22. \goDOI{10.1016/j.foreco.2006.03.023} \goScholar{17320958744182864860}

\bibitem{OFarrell_Anderson_2010} O'Farrell, P. J., Anderson, P. M. L., 2010. \textbf{Sustainable multifunctional landscapes: a review to implementation.} \emph{Current Opinion in Environmental Sustainability 2} (1-2), 59-65. \goDOI{10.1016/j.cosust.2010.02.005} \goScholar{11738195672718592653}

\bibitem{Dale_Beyeler_2001} Dale, V. H., Beyeler, S. C., 2001. \textbf{Challenges in the development and use of ecological indicators.} \emph{Ecological Indicators 1} (1), 3-10. \goDOI{10.1016/S1470-160X(01)00003-6} \goScholar{6821471585672677713}

\bibitem{Gilbert_2010} Gilbert, N., 2010. \textbf{Balancing water supply and wildlife.} \emph{Nature}. \goDOI{10.1038/news.2010.505} \goScholar{4413680200627277742}

\bibitem{Barnes_Jones_2011} Barnes, N., Jones, D., 2011. \textbf{Clear climate code: Rewriting legacy science software for clarity.} \emph{Software, IEEE 28} (6), 36-42. \goDOI{10.1109/MS.2011.113} \goScholar{12280553201622082598}


\bibitem{Sanders_Kelly_2008} Sanders, R., Kelly, D., 2008. \textbf{Dealing with risk in scientific software development.} \emph{Software, IEEE 25} (4), 21-28. \goDOI{10.1109/MS.2008.84} \goScholar{5979969088190020135}

\bibitem{Cerf_2012} Cerf, V. G., 2012. \textbf{Where is the science in computer science?} \emph{Commun. ACM 55} (10), 5. \goDOI{10.1145/2347736.2347737} \goScholar{15391984385653800704}

\bibitem{Pincas_2011} Pincas, U., 2011. \textbf{Program verification and functioning of operative computing revisited: How about mathematics engineering?} \emph{Minds and Machines 21} (2), 337-359. \goDOI{10.1007/s11023-011-9237-z} \goScholar{8660462933112825884}

\bibitem{Sanders_2009} Sanders, P., 2009. \textbf{Algorithm engineering -- an attempt at a definition.} In: Albers, S., Alt, H., N\"aher, S. (Eds.), Efficient Algorithms. Vol. 5760 of Lecture Notes in Computer Science. Springer Berlin Heidelberg, pp. 321-340. \goDOI{10.1007/978-3-642-03456-5\_22} \goScholar{16310481551473795186}

\bibitem{Kleiner_2011} Kleiner, K., 2011. \textbf{Data on demand.} \emph{Nature Climate Change 1} (1), 10-12. \goDOI{10.1038/nclimate1057} \goScholar{13711304819619162709}

\bibitem{Nature_2011} Nature, 2011. \textbf{Devil in the details.} \emph{Nature 470} (7334), 305-306. \goDOI{10.1038/470305b}

\bibitem{Peng_2011} Peng, R. D., 2011. \textbf{Reproducible research in computational science.} \emph{Science 334} (6060), 1226-1227. \goDOI{10.1126/science.1213847} \goScholar{905554772905069177}

\bibitem{Cai_etal_2012} Cai, Y., Judd, K. L., Lontzek, T. S., 2012. \textbf{Open science is necessary.} \emph{Nature Climate Change 2} (5), 299. \goDOI{10.1038/nclimate1509} \goScholar{10156741028963436768}

\bibitem{Ghisla_etal_2012} Ghisla, A., Rocchini, D., Neteler, M., F\"orster, M., Kleinschmit, B., 2012. \textbf{Species distribution modelling and open source GIS: why are they still so loosely connected?} In: Seppelt, R., Voinov, A. A., Lange, S., Bankamp, D. (Eds.), International Environmental Modelling and Software Society (iEMSs) 2012 International Congress on Environmental Modelling and Software. Managing Resources of a Limited Planet: Pathways and Visions under Uncertainty, Sixth Biennial Meeting. pp. 1481-1488. \gourl {http://www.iemss.org/iemss2012/proceedings/D6\_0897\_Ghisla\_et\_al.pdf} \goScholar{6265264564814017260}

\bibitem{Iverson_1980} Iverson, K. E., 1980. \textbf{Notation as a tool of thought.} \emph{Communications of the ACM 23} (8), 444-465. \gourl {http://awards.acm.org/images/awards/140/articles/9147499.pdf} \goScholar{15203139354397204728}

\bibitem{Lehman_1989} Lehman, M. M., 1989. \textbf{Uncertainty in computer application and its control through the engineering of software.} \emph{J. Softw. Maint: Res. Pract. 1} (1), 3-27. \goDOI{10.1002/smr.4360010103} \goScholar{11699944162229777276}

\bibitem{Lehman_Ramil_2002} Lehman, M. M., Ramil, J. F., 2002. \textbf{Software uncertainty.} In: Bustard, D., Liu, W., Sterritt, R. (Eds.), Soft-Ware 2002: Computing in an Imperfect World. Vol. 2311 of Lecture Notes in Computer Science. Springer Berlin / Heidelberg, Ch. 14, pp. 477-514. \goDOI{10.1007/3-540-46019-5\_14} \goScholar{15328955307076618013}

\bibitem{Hook_Kelly_2009} Hook, D., Kelly, D., 2009. \textbf{Testing for trustworthiness in scientific software.} In: Software Engineering for Computational Science and Engineering, 2009. SECSE '09. ICSE Workshop on. IEEE, Washington, DC, USA, pp. 59-64. \goDOI{10.1109/SECSE.2009.5069163} \goScholar{9959381250960735693}

\bibitem{Hatton_2007} Hatton, L., 2007. \textbf{The chimera of software quality.} \emph{Computer 40} (8), 104-103. \goDOI{10.1109/MC.2007.292} \goScholar{691364900894680251}

\bibitem{Hatton_1997} Hatton, L., 1997. \textbf{The t experiments: errors in scientific software.} \emph{Computational Science \& Engineering, IEEE} 4 (2), 27-38. \goDOI{10.1109/99.609829} \goScholar{8585904454769261330}

\bibitem{Hatton_2012} Hatton, L., 2012. \textbf{Defects, scientific computation and the scientific method uncertainty quantification in scientific computing.} \emph{IFIP Advances in Information and Communication Technology 377}, 123-138. \goDOI{10.1007/978-3-642-32677-6\_8} \goScholar{1399353442026811576}

\bibitem{Lehman_1996} Lehman, M. M., 1996. \textbf{Laws of software evolution revisited software process technology.} In: Montangero, C. (Ed.), Software Process Technology. Vol. 1149 of Lecture Notes in Computer Science. Springer Berlin/Heidelberg, Ch. 12, pp. 108-124. \goDOI{10.1007/BFb0017737} \goScholar{3754503458653974527}

\bibitem{Oberkampf_etal_2002} Oberkampf, W. L., DeLand, S. M., Rutherford, B. M., Diegert, K. V., Alvin, K. F., 2002. \textbf{Error and uncertainty in modeling and simulation.} \emph{Reliability Engineering \& System Safety 75} (3), 333-357. \goDOI{10.1016/S0951-8320(01)00120-X} \goScholar{10613684146262342334}

\bibitem{Wilson_2006} Wilson, G., 2006. \textbf{Where's the real bottleneck in scientific computing?} \emph{American Scientist 94} (1), 5+. \goDOI{10.1511/2006.1.5} \goScholar{12828974620192082182}

\bibitem{Rebaudengo_2011} Rebaudengo, M., Reorda, M., Violante, M., 2011. \textbf{Software-Level Soft-Error mitigation techniques.} In: Nicolaidis, M. (Ed.), Soft Errors in Modern Electronic Systems. Vol. 41 of Frontiers in Electronic Testing. Springer US, pp. 253-285. \goDOI{10.1007/978-1-4419-6993-4\_9} \goScholar{15742657402676184752}

\bibitem{Beaudette_2008} Beaudette, D., 2008. \textbf{Simple comparison of two Least-Cost path approaches.} In: Open Source Software Tools for Soil Scientists. \gourl {http://casoilresource.lawr.ucdavis.edu/drupal/node/544} (archived at: \gourl {http://www.webcitation.org/6D0LHBRXW} ) \goScholar{14623451910171059303}

\end{thebibliography}
}
\end{footnotesize}

\end{document}
