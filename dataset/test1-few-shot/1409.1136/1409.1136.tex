

In Section \ref{sec:introduction} we discussed how data values fail to provide a good model for modelling computations in which names are used hierarchically, such as a system of concurrent processes which can spawn subprocesses.
Motivated by these applications, in this section we introduce a notion of nested data values in which the data set has a forest-structure.  This is a stylistically different presentation 
to earlier work on nested data in that~\cite{bjorklund10,Decker14} require that each position in the words considered have a data value in each of a fixed number of levels.  By giving the data set a forest-structure, we can explicitly handle variable levels of nesting within a word.  However, we note that there is a natural translation between the two presentations.



\begin{definition}
A \emph{rooted tree} (henceforth, just \emph{tree}) is a simple directed graph , where  is the predecessor map defined on every node of the tree except the root, such that every node has a unique path to the root. A node  of a tree has \emph{level}  just if  is the root (thus the root has level 1). A tree has \emph{bounded level} just if there exists a least  such that every node has level no more than ; we say that such a tree has level .

We define a \emph{nested dataset}  to be a forest of infinitely many trees of level  which is \emph{full} in the sense that for each data value  of level less than , 
 has infinitely many children.
\end{definition}



We now extend CMA to nested data by allowing the nested data class memory automaton (NDCMA), on reading a data value , to access the class memory function's memory of not only , but each ancestor of  in the nested data set.  Once a transition has been made, the class memory function updates the remembered state not only of , but also of each of its ancestors.  Formally:



\begin{definition}
Fix a nested data set of level .  A \emph{Nested Data CMA of level } is a tuple  where  is a finite set of states,  is the initial state,  are sets of globally and locally accepting states respectively, and  is the transition map.   is given by a union  where each  is a function:

The automaton is \emph{deterministic} if each set in the image of  is a singleton; and is \emph{weak} if .
A configuration is a pair  where , and  is a class memory function (i.e.  for all but finitely many ).  The initial configuration is  where  is the class memory function mapping every data value to .  A configuration  is final if  and  for all .
The automaton can transition from configuration  to configuration  on reading input  just if  is a level- data value, , and .
A run  is accepting if the configuration  is final.   if there is an accepting run of  on .
\end{definition}

It is clear that level-1 NDCMA are equivalent to normal CMA.  
We know that emptiness of class memory automata is equivalent to reachability in Petri nets; it is natural to ask whether there is any analogous correspondence -- to some kind of high-level Petri net -- once nested data is used. 

\begin{example}\label{ex:ndcma-pnwr}
In Example \ref{ex:wcma-pncoverability}, we showed how CMA (resp. WCMA) can encode Petri net reachability (resp. coverability).  A similar technique allows reachability (resp. coverability) of Petri nets with reset arcs to be reduced to emptiness of NDCMA (resp. weak NDCMA).  
The key idea is to have, for each place in the net, a level-1 data value -- essentially as a ``bag'' holding the tokens for that place. Nested under the level-1 data value, level-2 data values are used to represent tokens just as before.  
When a reset arc is fired, the corresponding level-1 data value is moved to a ``dead'' state -- from where it and the data values nested under it are not moved again -- and a fresh level-1 data value is then used to hold subsequently added tokens to that place.
\end{example}

\begin{theorem}\label{thm:ndcma-emptiness}
The emptiness problem for NDCMA is undecidable.  Emptiness of Weak NDCMA is decidable, but Ackermann-hard.
\end{theorem}
\begin{proof}
This result follows from Theorem \ref{thm:NDCMA-pNDA} together with results in \cite{Decker14}, though we also provide a direct proof.

We show decidability by reduction to a well-structured transition system \cite{FinkelS01} constructed as follows: a class memory function on a nested data set can be viewed as a labelling of the data set by labels from the set of states.  Since we only care about the shape of the class memory function (i.e. up to automorphisms of the nested data set), we can remove the nodes labelled by , and view a class memory function as a finite set of labelled trees.  The set of finite forests of finite trees of bounded depth with the order given by  iff there is a forest homomorphism from  to  (where a forest is the natural extension of tree homomorphisms to forests) is a well-quasi-order \cite{Meyer08}, which provides the basis for the well-structured transition system.  A full proof is given in Appendix \ref{appendix-ndcma-wsts}.

Undecidability for NDCMA and Ackermann-hardness for Weak NDCMA follow from the ideas in Example \ref{ex:ndcma-pnwr}: the reachability (resp. coverability) problem for Petri nets with reset arcs is encodable in NDCMA (resp. Weak NDCMA), and this is undecidable~\cite{ArakiK76} (resp. Ackermann-hard \cite{Schnoebelen02}).
\end{proof}
Weak Nested Data CMA have similar closure properties to Weak CMA.
\begin{proposition}
\begin{enumerate}[(i)]
\item Weak NDCMA are closed under intersection and union.
\item Deterministic Weak NDCMA are closed under all Boolean operations.
\end{enumerate}
\end{proposition}
\begin{proof}
Again these can be shown by the same techniques as for DFA.
\end{proof}

\begin{corollary}
The containment and equivalence problems for
Deterministic Weak NDCMA are decidable.
\end{corollary}

\subsection{Link with Nested Data Automata}
In \cite{Decker14} Decker et al. also examined ``Nested Data Automata'' (NDA), and showed the locally prefix-closed NDA (pNDA) to have decidable emptiness (via reduction to well-structured transition systems).  In fact, these NDA precisely correspond to NDCMA, and again being locally prefix-closed corresponds to weakness. In this section we briefly outline this connection.

\textbf{Nested Data Automata.}  A -nested data automaton is a tuple  where  is a data automaton for each .  Such automata run on words over the alphabet , where  is a (normal, unstructured) dataset.  As for normal data automata, the transducer, , runs on the string projection of the word, giving output .  Then for each  the class automaton  runs on each subsequence of  corresponding to the positions which agree on the first  data values.   The NDA is locally prefix-closed if each  is.

Since these NDA are defined on a slightly different presentation of nested data, we provide the following presentation of NDCMA over multiple levels of data.
\begin{definition}
A \emph{Nested Data CMA of level } over the alphabet  is a tuple  where  is a finite set of states, , , and  is the transition map.

A configuration is a tuple , where each  maps an -tuple of data values to a state in the automaton (or ).  The initial configuration is  where  maps every tuple in the domain to .  A configuration  is final if each  maps into .  The automaton can transition from configuration  to configuration  on reading input  just if , and each .  
\end{definition}
Using ideas from the proof of equivalence between CMA and DA in~\cite{bjorklund10}, we can show the following result:
\begin{theorem}\label{thm:NDCMA-pNDA}
NDCMA (resp. weak NDCMA) and NDA (resp. pNDA) are expressively equivalent, with effective translations.
\end{theorem}

\subsection{Link with Higher-Order Multicounter Automata}

In \cite{BjorklundB07} the authors examined a link between nested data values and shuffle expressions.  In doing so, they introduced higher-order multicounter automata (HOMCA).  While not explicitly over nested data values, they are closely related to the ideas involved, and in fact we show that, just as multicounter automata and CMA are equivalent, there is a natural translation between HOMCA and the NDCMA we have introduced.  Further, just as the equivalence between MCA and CMA descends to one between weak multicounter automata and weak CMA, we find an equivalence between weak NDCMA and ``weak'' HOMCA in which the corresponding acceptance condition -- hereditary emptiness -- is dropped.  
To show this, we introduce HOMCA', which add restrictions to the  and  counter operations analogous to the restriction for the  operation.
\begin{wrapfigure}{r}{0.7\textwidth}
   \vspace{-30pt}

   \vspace{-10pt}
\caption{A diagram showing translations between HOMCA, HOMCA', NDCMA, and their weak counterparts.}
   \vspace{-10pt}
\label{fig:homca-equivalences}
\end{wrapfigure}
We show that these HOMCA' are equivalent to HOMCA, and that HOMCA' are equivalent to NDCMA, with both of these equivalences descending to the weak versions.  These equivalences are summarised in Figure~\ref{fig:homca-equivalences}.

\begin{definition}
We define \emph{weak} HOMCA to be just as HOMCA, but without the hereditary-emptiness condition on acceptance,  i.e. a run is accepting just if it ends in a final state.
\end{definition}


\begin{definition}
We define HOMCA' to be the same as HOMCA, except for the following changes to the counter operations:
\begin{inparaenum}[(i)]
\item  operations are only enabled when ; and
\item  operations are only enabled when  and .
\end{inparaenum}

As for HOMCA, we define \emph{weak} HOMCA' to be HOMCA' without the hereditary-emptiness condition. 
\end{definition}
This means that each reachable configuration  is such that there is a unique  such that for all ,  and each , .

\begin{theorem}
HOMCA (resp. weak HOMCA) and HOMCA' (resp. weak HOMCA') are expressively equivalent, with effective translations between them.
\end{theorem}
\begin{proof}
This requires simulating the HOMCA operations  and  in HOMCA': which can be difficult if, for instance, the HOMCA is carrying out a  operation when it has a current level- multiset in memory.  The trick is to move the level- multiset across to be nested under a new level- multiset, and this can be done one element at a time in a ``folding-and-unfolding'' method.  The hereditary emptiness condition checks that the each of these movements was completed, i.e. no element was left unmoved. In the weak case some elements not being moved could not change an accepting run to a non-accepting run, so the fallibility of the moving method does not matter.   We provide a more detailed proof sketch in Appendix~\ref{appendix-homca}.
\end{proof}

We now have the main result of this section:
\begin{theorem}\label{thm:ndcma-homca-equivalence}
For every (weak) level- NDCMA, , there is a (weak) level- HOMCA', , such that , and vice-versa. 
\end{theorem}
\begin{proof}
This proof rests on the strong similarity between the nesting of data values, and the nesting of level- multisets in level- multisets.  

For a NDCMA to simulate a HOMCA', we use level- data values to represent instances of the multiset letters, level- data values to represent level-1 multisets, and so on, up to level- data values representing level- multisets.  Since each run of a HOMCA' can have at most one level- multiset, this does not need to be encoded in data values.  

Conversely, when simulating a NDCMA with a HOMCA', a level- data value is represented by an instance of an appropriate multiset letter.  The letter contains the information on which state the data value was last seen in.  Level- data values are represented by level-1 multisets, which also include a multiset letter storing the state that data value was last seen in.  We provide a full proof of the theorem in Appendix~\ref{appendix:ndcma-homca}.
\end{proof}





