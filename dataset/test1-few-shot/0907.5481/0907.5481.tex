\documentclass[11pt]{article}
\pdfoutput=1
\usepackage[centertags]{amsmath}
\usepackage{amssymb}

\newtheorem{theorem}{Theorem}
\newtheorem{proposition}{Proposition}[section]
\newtheorem{definition}{Definition}[section]
\newtheorem{corollary}{Corollary}[section]
\newtheorem{lemma}{Lemma}[section]
\newtheorem{example}{Example}[section]
\newtheorem{problem}{Problem}


\providecommand{\expectation}[2]{\mathbb{E}_{#2}\left[#1\right]}
\providecommand{\probab}[2]{\mathbb{P}_{#2}\left\{#1\right\}}

\providecommand{\whp}{\textbf{whp}}

\providecommand{\rg}[1]{G(#1)}
\providecommand{\rhg}[1]{\mathcal{G}(#1)}
\providecommand{\rig}[1]{G_{\mathcal{I}}(#1)}
\providecommand{\rsg}[1]{G_{\mathcal{S}}(#1)}


\providecommand{\rigs}{random intersection graphs}
\providecommand{\ktree}{-tree}
\providecommand{\ktrees}{-trees}



\providecommand{\binom}[2]{{#1\choose#2}}


\providecommand{\degree}[2]{{\textrm{deg}_{#1}(#2)}}
\providecommand{\vd}{V_D}
\providecommand{\elabel}[1]{L(#1)}
\providecommand{\vlabel}[1]{\ell(#1)}



\providecommand{\qed}{\hfill }

\newenvironment{proof}[0]{\textit{Proof.} }{\hfill  \qed} 

\newenvironment{problem-env}[1]{
\vskip 3mm
\begin{problem} 
\textbf{#1}
}
{
\end{problem}
}


\newenvironment{neat-list}[3]{
  \vskip 1mm
  \begin{list}{}
  {\itemsep 2mm  \labelsep #1 \labelwidth #2 \leftmargin #3
   \partopsep 0mm \topsep 0mm \parsep 0mm  \itemindent 0mm}
}
{
  \end{list}
  \vskip 3mm
}


\newenvironment{code}[1]{
  \vskip 3mm
  \hrule \noindent
  \textbf{#1}
  \vskip 3mm
  \hrule \vskip 3mm
  }
{
  \hrule
  \vskip 3mm}


\newenvironment{Algorithm}[3]{
  \vskip 1mm
  \hrule \noindent
  \textbf{Algorithm} #1
   \begin{neat-list}
   \item \textbf{Input: }#2 
   \item \textbf{Output: }#3
   \end{neat-list}
  \vskip 1mm
  \hrule \vskip 3mm
  }
{
  \vskip 2mm
  \hrule
  \vskip 1mm}

\newcommand{\cmnt}[2]
{ \ \\
\fbox{
\begin{minipage}[b]{\textwidth - 1cm}
{\itshape #1}
\end{minipage}
\par

\hfill \footnotesize \today
}
\\ }






\newcommand{\union}{\cup}
\newcommand{\intersect}{\cap}
\newcommand{\nbr}{N}
\newcommand{\V}{V}
\newcommand{\E}{E}
\newcommand{\G}{G}
\newcommand {\setdelim}        {\; | \;}

 

\newcommand{\weicnf}[3]{{\mathcal F}_{#3}^{#1,#2}}
\newcommand{\weicnfa}[3]{{\mathcal F}_{#3}^{#1,#2}(A)}
\newcommand{\bdtnm}{{\mathcal B}^{d,t}_{n,m}}
\newcommand{\model}[1]{\bdtnm[{#1}]}
\newcommand{\flawless}{\model{{\mathcal M}}}
\newcommand{\genmodel}{\model{\albl}}
\newcommand{\modelb}{\bdtnm}
\newcommand{\modelsc}{\model{\SC}}
\newcommand{\modelwc}{\model{\WC}}

 
 
\newcommand{\evar}[2]{\var_{#2}:{#1}}
\newcommand{\envar}[1]{\var:{#1}}
\newcommand{\clause}[1]{{#1} }
\newcommand{\bnot}[1]{\overline{#1}}
\newcommand{\bor}{\vee}
\newcommand{\instance}{{\mathcal{I}}}
\newcommand{\cnf}{{\mathrm{CNF}}}
\newcommand{\vars}{{\mathrm{var}}}

\newcommand{\mdl}{{\mindeg_{\ell}}}
\newcommand{\sizel}{\sized_{\ell}}
\newcommand{\xl}{{\var_{\ell}}}
\newcommand{\dl}{{\D_{\ell}}}
\newcommand{\vl}{{\V_{\ell}}}
\newcommand{\pil}{{\biject_{\ell}}}
\newcommand{\vall}{\val_{\ell}}
\newcommand{\idl}{\iota_{\ell}}


\newcommand{\xr}{{\var_r}}
\newcommand{\dr}{{\D_r}}
\newcommand{\vr}{{\V_r}}
\newcommand{\pir}{{\biject_r}}
\newcommand{\sizer}{\sized_r}
\newcommand{\mdr}{{\mindeg_r}}
\newcommand{\valr}{\val_r}
\newcommand{\idr}{\iota_{r}}

 
 

\renewcommand{\theequation}{\thesection.\arabic{equation}}

\newcommand{\figref}[1]{Figure~\ref{#1}}
\newcommand{\tabref}[1]{Table~\ref{#1}}
\newcommand{\lemref}[1]{Lemma~\ref{#1}}
\newcommand{\defref}[1]{Definition~\ref{#1}}
\newcommand{\thmref}[1]{Theorem~\ref{#1}}
\newcommand{\sectref}[1]{Section~\ref{#1}}

\newcommand{\Figref}[1]{Figure~\ref{#1}}
\newcommand{\Tabref}[1]{Table~\ref{#1}}
\newcommand{\Lemref}[1]{Lemma~\ref{#1}}
\newcommand{\Defref}[1]{Definition~\ref{#1}}
\newcommand{\Thmref}[1]{Theorem~\ref{#1}}
\newcommand{\Sectref}[1]{Section~\ref{#1}}



 
\topmargin -1.1 cm
\oddsidemargin 0.2 in
\evensidemargin 0 in
\textwidth 6.0 in
\textheight 8.6 in 
\parskip 0.1cm

\begin{document}

 
\title{Treewidth of Erd\"{o}s-R\'{e}nyi Random Graphs, Random Intersection Graphs, and 
  Scale-Free Random Graphs
} 

 
\author{Yong Gao \thanks{Supported in part by NSERC Discovery Grant RGPIN 327587-09} \\
    Department of Computer Science, \\
    Irving K. Barber School of Arts and Sciences \\
    University of British Columbia Okanagan, \\
    Kelowna, Canada V1V 1V7 \\
}   
\maketitle 

\begin{abstract}
We prove that the treewidth of an Erd\"{o}s-R\'{e}nyi random graph  is, with
high probability, greater than  for some constant  if 
the edge/vertex ratio  is greater than 1.073. Our lower bound 
 improves the only previously-known lower bound established in \cite{kloks94}.  
We also study the treewidth of random graphs under two other random models for large-scale complex networks. In particular, our result on the treewidth of \rigs~strengths a previous observation in \cite{karonski99cpc} on the average case behavior of the \textit{gate matrix layout} problem. For scale-free random graphs based on the  Barab\'{a}si-Albert preferential-attachment model, our result shows that  if more than 12 vertices are attached to a new vertex, then the treewidth of the obtained network is linear in the size of the network with high probability.   
\end{abstract}
 

\section{Introduction}
\label{Introduction}
Treewidth plays an important role in characterizing the structural properties of a graph
and the complexity of a variety of algorithmic problems of practical importance
\cite{bodlaender93,kloks94}. When restricted to instances with bounded treewidth, 
many NP-hard problems are polynomially sovable. Dynamic programming algorithms based on the
tree-decomposition of graphs have found many applications in research field
such as computational biology and artificial intelligence \cite{dalmau02,dechter01}.

The theory of random graphs pioneered by the work of Erd\"{o}s and R\'{enyi} 
\cite{erdos60} deals with the probabilistic behavior of various graph properties such as the connectivity, the colorability, and the size of (connected) components
\cite{erdos60,bollobas01,achlioptas99sharp,friedgut99}.
Random intersection graphs and scale-free random graphs were proposed 
as more realistic models for large-scale complex networks arising in
real-world domains such as communication networks (Internet, WWW, Wireless and P2P networks), computational biology (protein networks), and sociology 
(social networks). It has been hoped that these new models will be able to capture    
the common features of these networks in a better way and in the mean time, are mathematically
approachable and algorithmically tractable \cite{cooper05,ferrante08tcs,gao09tcs,silvio09stoc}.  

As treewidth is one of the most important structural parameters used to capture the 
algorithmic tractability of computationally hard problems, it is interesting 
to see how large the treewidth of  a typical graph is in these random models. Of course, studying the probabilistic behavior of the treewidth of these random graphs is itself an interesting combinatorial problem. 
Except for a result in \cite{kloks94} establishing an lower bound  on the threshold of having a linear treewidth of the Erd\"{o}s-R\'{e}nyi random graph, we are not aware of any other work in the literature. In the paper, we study the treewidth of random graphs under 
the following three random models:
\begin{enumerate}
\item \textbf{The Erd\"{o}s-R\'{e}nyi model \cite{bollobas01,erdos60}}. An Erd\"{o}s-R\'{e}nyi random graph  is defined on  vertices and contain  edges selected from the 
 potential edges uniformly at random and without replacement.
\item \textbf{The random intersection model \cite{karonski99cpc}}. A random intersection graph  on  vertices is defined as follows. Let  be a fixed universe of size . Each vertex   is associated with a subset  that is obtained by including each element in  independently with probability . These 's are determined independently as well. There is an edge between a pair of vertices  and  if and only if . 
\item \textbf{The Barab\'{a}si-Albert scale-free model \cite{albert02complex}}. 
A Barab\'{a}si-Albert random graph   on a set of  vertices  is defined by a graph evolution process in which vertices are added to the graph one at a time.
In each step, the newly-added vertex is connected to
 existing vertices selected according to 
the \textit{preferential attachment} mechanism, i.e. an existing vertex is selected with probability in proportion to its degree.          
\end{enumerate} 

We establish a lower bound  1.073 on the edge/vertex ratio  
above which an Erd\"{o}s-R\'{e}nyi random graph  has a treewidth 
linear to the number of vertices with high probability. Our lower bound
improves the previous one  in \cite{kloks94}. 
We obtain similar results on the behavior of the treewidth 
for the random intersection graph  and the  Barab\'{a}si-Albert scale-free random graph . Our result on  complements an observation in \cite{karonski99cpc} on the average case behavior of the \textit{gate matrix layout} problem. 
Our result on the scale-free random graph  shows that if more than 12 vertices are attached to a new vertex, then the treewidth of the obtained network is linear in the size of the network with high probability.             
Our results are summarized in the following theorems: 
 
\begin{theorem}
\label{thm-treewidth-bound}
Let  be an Erd\"{o}s-R\'{e}nyi random graph. 
For any , there is a constant
 such that

\end{theorem}


\begin{theorem}
\label{thm-intersection-graph}
Let  be a random intersection graph with the universe 
 and . 
For any  and , there exists a constant  such that
  
\end{theorem}

\begin{theorem}
\label{thm-power-law-graph}
Let  be the Barab\'{a}si-Albert random graph. For any , there is 
a constant  such that    

\end{theorem}


\subsection{Technical Contribution}
The approach used in \cite{kloks94} is essentially an application of 
the first-moment method
to the random variable that counts the total number of the \textit{balanced partitions}
 where the size of the separator  is at most  
(See Section~\ref{sec-thm-1} for the formal definition of a balanced partition.)  It is further commented in \cite{kloks94} that it was not known whether the 1.18 lower bound 
can be improved and that the treewidth of the random graph  with 
 is unknown. 

Our main contribution in this paper is in the proof of our improved lower bound  
. We note that a more refined analytical calculation is able to 
improve the lower bound 1.18 in \cite{kloks94} to 1.083. The difficulty lies in bringing 
down the lower bound further from 1.083 to 1.073.  To achieve this, we introduce the notion of
-rigid and balanced partitions  which are maximally balanced 
in the sense that no vertex subset of certain size from the larger part, say , can be moved 
to the smaller one  to create a new balanced partition. The motivation is that by considering the expected number of these more restricted partitions, 
we will be able to get a more accurate estimation when applying Markov's inequality\footnote{The idea of restricting the kinds of combinatorial objects to be considered have been used in the study of the threshold for the satisfiability of random CNF formulas and the chromatic number of random graphs\cite{kirousis94threshold,achlioptas99thesis,kirousis09threshold}}. 
 
The difficulty we have to overcome in the case of treewidth is the estimation of the expected 
number of -rigid and balanced partition  in . To do this, 
an exponentially small upper bound is required on the probability that the induced
subgraph  of the random graph  doesn't have small-sized tree components.

We managed to obtain such an exponentially small upper bound in a ``conditional" probability space, which is equivalent to the Erd\"{o}-R\'{e}yni random model as far as the size of the treewidth is concerned,  by  using a Hoeffding-Azuma style inequality.
To achieve the best possible  Lipschitz constant in our application of the Hoeffding-Azuma inequality, we used a ``weighted" count on the number of tree components of size up to a fixed constant .  We are not aware of any other application of the Hoeffding-Azuma inequality  
in the study of random discrete structures where this idea of weighted counts is beneficial.       



\subsection{Outline of the Paper}
The next section fixes our notation and contains preliminaries. Also discussed in this section
is a variant of the Erd\"{o}-R\'{e}nyi model for random graphs which we will be using in our proofs. Sections 3 - 5 contain the proofs of Theorem \ref{thm-treewidth-bound}, 
Theorem \ref{thm-intersection-graph}, and Theorem \ref{thm-power-law-graph} respectively. The two appendices contain the proof of some necessary lemmas.     


 
\section{Notation and Preliminaries}

Throughout this paper, all logarithms are natural logarithms, i.e., 
to the base . The cardinality of a set  is denoted by .   
All graphs are undirected and standard 
terminologies in graph theory \cite{west01} are used.  Given a graph  and a vertex
, we use  to denote the set of neighbors of , i.e.,

Given a vertex subset , we use  to denote the neighborhood of , i.e.,
  
The induced subgraph on a subset of vertices  is denoted by .
By a component of a graph, we mean a maximal connected subgraph.  

In the proofs, we will be using the following upper bound on  that 
can be derived from Stirling's formula:
\begin{lemma}
\label{lem-stirling}
For any constants , 
 
where  is a constant.
\end{lemma}

We also need the  following three lemmas on the properties of some useful functions.  
The proof of these lemmas are incldued in Appendix 2.

\begin{lemma}
\label{lem-function-0}
On internal , the function 
  
attains its minimum at   and .  Furthermore,  is decreasing on the interval  and decreasing on the 
interval .
\end{lemma}

\begin{lemma}
\label{lem-function-1}
Let  is a function defined as

where  is a constant. For any  and sufficiently small ,  
 is decreasing on the interval .
\end{lemma}

\begin{lemma}
\label{lem-function-2}
Let  be a function defined as

where  and  are constants. Then for sufficiently small ,
 is increasing on .
\end{lemma}

\subsection{Treewidth and Random Graphs}
\label{subsec-treewidth}
The notion of treewith plays an important role in graph theory and in real world computing.   
Several equivalent definitions of treewidth exist and the one based on \ktrees~ is probably 
the easiest to explain. The graph class of \ktrees~is defined recursively as follows
\cite{kloks94}:
\begin{enumerate}
    \item A clique with k+1 vertices is a \ktree;
    \item Given a \ktree~  with n vertices, a \ktree~with
         vertices is constructed by adding to  a new
        vertex and connecting it to a -clique of .
\end{enumerate}
A graph is called a \textbf{partial \ktree}~if it is a subgraph of a
\ktree. The treewidth  of a graph  is the minimum value  such that
 is a partial \ktree.

Since the seminal work of Erd\"{o}s and R\'{e}nyi 
\cite{erdos60}, the theory of random graphs  has been an active research area in  graph theory and combinatorics. The probabilistic behavior of various graph properties such as the connectivity, the colorability, and the size of (connected) components, have been extensively 
studied. The theory of random graphs has also motivated the study 
of the probabilistic properties of random instances of other important combinatorial 
optimization problems, most notably that of the satisfiability of random logic formulas in
conjunctive normal form (CNF).    

We use  to denote an Erd\"{o}s-R\'{e}nyi random graph \cite{bollobas01} on 
 vertices with  edges selected from the  possible edges 
uniformly at random and without replacement.  Throughout this paper by ``with high probability",
abbreviated as \whp,  we mean that the probability of the event under consideration
is  as  goes to infinity.   

We will be working with a random graph model  that is slightly different
from  in that the  edges are selected independently and uniformly
at random, \textbf{but with replacement}. 
There is a one-to-one correspondence between the random graph  and the 
product probability space 

defined as follows:
\begin{enumerate} 
\item  where each 
 is the set of all  possible edges. This is a finite and discrete  sample space.
\item  is the -field consisting of all subsets of .  
\item The probability measure   is 

\end{enumerate}
A sample point  is interpreted as an outcome of the 
random experiment that selects  edges independently, uniformly at random with replacement
from the set of all possible edges. Note that the graph corresponding to a sample point   is actually a multi-graph, i.e., a graph in which parallel edges are allowed.    

It turns out that as far as the property of having a treewidth linear in the number of vertices is concerned, the two random graph models   and 
 are equivalent. In fact, the equivalence holds for any monotone increasing combinatorial property in random discrete structures,    
as has been observed in \cite{kirousis94threshold,achlioptas99thesis} and formally proved in \cite{kirousis96tech}. For completeness, we will include in Appendix~\ref{appendix-1} an alternative pure measure-theory style proof.   
\begin{proposition}
\label{prop-equiv}
If there exists a constant  such that

then
 
\end{proposition}
Due to Proposition~\ref{prop-equiv}, we will continue to use the notation  
instead of  throughout this paper, but with the understanding that the  edges are selected independently and uniformly at random with replacement.


In the rest of the paper, we will always subscript operations such as 
a probability measure  and a mathematical 
expectation  to clear indicate the underlying probability space in which these operations are applied. 
 
In \cite{kloks94}, it proved that the treewidth of an Erd\"{o}s-R\'{e}nyi random 
graph  is linear in the number of vertices \whp~  if the edge/vertex ratio is
greater than .  It is mentioned
in \cite{kloks94} that it was unclear whether the lower bound 1.18 can 
be further improved, and that
the treewidth of a random graph  with 
is unknown \cite{kloks94}.  The main result of this paper improves
the bound to 1.073. 



\subsection{Random Intersection Graphs}
The intersection model for random graphs was introduced by Karon\'{n}ski, et al. \cite{karonski99cpc}. As one of the motivations, Karon\'{n}ski, et al. discussed 
the application of this model in the average-case analysis of algorithmic problems
in gate matrix circuit design \cite{karonski99cpc}.  Other motivations for the recent interests in random intersection graphs include the possible applications in modeling 
large-scale complex networks arising in wireless communications \cite{nikoletseas08tcs} 
and social networks. 

A random intersection graph  over a vertex set  is defined  
by a universe  and  three parameters  (the number of vertices), 
, and 
. Associated with a vertex  is a random subset  formed 
by selecting each element in  independently with probability . A pair of vertices 
and  is an edge in  if and only if .   

An alternative view of  is as follows. Let  be a set of 
 subsets of vertices. Each  is formed independently by including each vertex independently with probability . A pair of vertices  and  is an edge in 
  if and only if some  contains both  and . In this sense, a random intersection graph is actually the primal graph of a random hypergraph consisting of 
 hyperedges each of which contains a vertex with probability . 
 
\subsection{Barab\'{a}si-Albert Random Graphs} 
In recent years, there has been growing interests in random models for large-scale communication networks, biological networks, and social networks. A remarkable observation is that 
the degree distribution of these large-scale networks follow a power law, i.e., the fraction
of vertices of a given degree  is proportional to  for some constant
. 

The Barab\'{a}si-Albert model for random graphs is proposed in  \cite{albert02complex}          
and has been shown to have a power law degree distribution \cite{bollobas01scalefree}. In addition to the purpose of modeling, it is also hoped that features such as a power-law degree distribution may be exploited  algorithmically and/or mathematically 
to help solve real-world problems defined on these large-scale networks. See, for example, the work and arguments in \cite{cooper05,ferrante08tcs,silvio09stoc, gao09tcs}.     

Following the formal definition given in \cite{bollobas01scalefree},
a Barab\'{a}si-Albert random graph   on a set of  vertices 
 is defined by a graph evolution process in which vertices are added to the graph one at a time. In each step, the newly-added vertex is connected to
 existing vertices selected according to 
the \textit{preferential attachment} mechanism, i.e. an existing vertex is selected with probability in proportion to its degree. To be more precisely, let  be the vertex to be added and let  be the graph obtained after vertex  is added. 
The  neighbors of  are selected in  steps. In step ,  
the probability that 
an existing vertex  is selected as the neighbor of the new vertex  is 
          
where 
\begin{enumerate}
\item  is the total degree
of ,
\item  is the number of times  has been picked as the neighbor of  in the first  trials, and 
\item the term  takes into consideration the increase of the total degree as a result of the first  neighbors. 
\end{enumerate}
One also needs to take care of the initial phase, but that won't have any impact on our analyses. 

\section{Treewidth of Erd\"{o}s-R\'{e}nyi Random Graphs: Proof of Theorem \ref{thm-treewidth-bound}}
\label{sec-thm-1}
In this section, we prove Theorem~\ref{thm-treewidth-bound} to establish 
the lower bound  on the edge/vertex ratio  such that  
whenever ,  
the treewidth of an  Erd\"{o}s-R\'{e}nyi random graph  
is \whp~greater than  for some constant .
To begin with, we introduce the following concept which will be used as a necessary condition 
for a graph to have a treewidth of certain size. The following notion of
balanced -partition was used in \cite{kloks94} to establish the lower bound 1.18.
\begin{definition}[\cite{kloks94}]
\label{rigid_partition_def}
Let  be a graph with . 
Let  be a triple of disjoint vertex subsets such that 
 and .  

We say that  is balanced if 
. 
Without lose of generality, we will always assume that .

We say that  is an -partition if 
 separates  and , i.e., there are no edges between vertices of  and vertices of . 
\end{definition}

The following notion of a -rigid partition plays an important role in establishing
our improved lower bound: 
\begin{definition}
Let  be an integer. A triple  with  is said to be -rigid if there is no subset of vertices  with  that induces
a connected component of .
\end{definition}

A -rigid and balanced -partition generalizes Kloks's balanced -partition by
requiring that any vertex set of size at most  in the larger subset of a partition cannot be
moved to the other subset of the partition, and hence the word ``rigid". As we will have to consider all the vertex sets of size at most  to get the best possible estimation, 
the requirement of connectivity is a kind of ``maximality" condition to avoid repeated counting 
of vertex sets of different sizes. For the case of 
, being -rigid means that  has no isolated vertices.

We note that the idea of imposing various restrictions on the combinatorial objects  under consideration has been used in recent years to increase the power of 
the first moment method when dealing with combinatorial problems in 
discrete random structures such as the satisfiability of 
random CNF formulas \cite{kirousis94threshold,kirousis09threshold} and 
the colorability of random graphs \cite{achlioptas99thesis}.      
Our result is a further illustration of the power of this idea in the context 
of treewidth of random graphs.

\begin{lemma}
\label{treewidth-partition-lem}
Let   be an integer.
Any graph with a treewidth at most  must have a balanced -partition 
 such that either  or
 is -rigid. 
\end{lemma}
\begin{proof}
From \cite{kloks94}, any graph with treewidth at most  must have
a balanced -partition . If , we are done. Otherwise, if the triple  is not -rigid, then there must be a vertex subset  that induces a component of  and consequently 

Therefore, we can move  from  to  and create a new balanced -partition
with the size of  decreased by .
Continuing this process until either  or 
the partition becomes -rigid. 
\end{proof}

\subsection{Conditional Probability of a -rigid and balanced -partition}

We now bound the conditional probability that a balanced triple 
with  and  is -rigid given that it is an 
-partition of .
To facilitate the presentation,  we define the following function

where .

\begin{theorem}
\label{thm-rigid-conditional}
Let  be a random graph and let 
be a balanced triple such that . Let 
be a constant integer less than .  
Then for  sufficiently large,

where ,

and 


\end{theorem}
\begin{proof} 
Conditional on that  is an -partition of , each of the
 edges can only be selected from the set of edges

where  denotes the set of unordered pair of vertices. Let  be the size of 
, we have


In the rest of the proof, we will work on the conditional probability space
 where
 and 
 for each . A sample point 
 corresponds to an outcome 
of selecting  edges from  uniformly at random and with replacement such that
 is a balanced -partition of the graph determined by . 
The probability measure  is
    
The following lemma guarantees that we can obtain Equation~(\ref{eq-rigid-probab})
by studying the probability :  

\begin{lemma}
\label{lem-equiv-condi}

\end{lemma}
\begin{proof}
Recall that  is the probability measure for the probability 
space  and 
 is the probability measure 
for the probability space .
Note that  is the set of sample points  in  such that 
 is an -partition in the graph determined by .
Let  be the set of sample points  such that  
 is -rigid in the graph determined by .
We have 

This proves the lemma. 
\end{proof}

Continuing the proof of Theorem~\ref{thm-rigid-conditional}, we need to bound 
. 
To make thing simpler, we will 
bound the probability that there exist tree components, instead of general connected components, of size at most  in the subgraph of  induced on the vertex set .
We use the following variate of Hoeffding-Azuma inequality: 
\begin{lemma}[Lemma 1.2 \cite{mcdiarmid89} and Theorem 1.19 \cite{bollobas01}]
\label{lem-bounded-diff}
Let  be a independent 
product probability space where each  is a finite set, and 
be a random variable satisfying the following Lipschitz condition

if  differs only in one coordinate. Then, for any ,
   
\end{lemma} 

In our case, the probability space is 
and we may use any the function  such that  the total number 
of tree components of size at most  is larger than zero whenever . To achieve  
the best possible Lipschitz constant  in Equation (\ref{eq-bounded-diff}), we consider
a weighted sum  of all tree components of size at most  defined as follows.      

For any ,  let  be the collection of size- vertex sets in  
and let 
 
For a vertex set , we use  to denote the indicator function of the event that   is a tree component of , i.e.,  is a tree and
. Define 
  
where . The idea is that instead of counting the total number of tree components of size at most , we use the random  variable  as  a ``weighted count" 
to which the contribution of a tree component on a vertex set of size  is . Note that the constant  can be made arbitrarily small by taking an arbitrarily large (but constant) . The purpose is to make   
as close to 1 as possible for every pair  and  that differs only on 
one coordinate. 

It is obvious that  if and only if  that the total number of 
tree components of size at most 
is greater than zero. By the definition of a -rigid triple, we have
  
By Lemma~\ref{lem-equiv-condi} and Lemma~\ref{lem-bounded-diff}, we have 

where  with the maximum taken over
all pairs of  and  in   that differ only on one coordinate.
The following lemma bounds . 
(Note that if we had used the unweighted sum , 
the best we can have is  .)
\begin{lemma}
\label{lem-max-diff}
For any  that differ only in one coordinate,  

\end{lemma}
\begin{proof}
Note that  and  represent two possible outcomes of the  independent random experiments that select the  edges of a random graph.   
If  differ only in one coordinate, say the -th coordinate, 
then the edge sets of the corresponding graphs  and  only differ
in the -th edge.   

Let us 
consider the change of the value of  when we modify  to 
by removing the -th edge of  and adding the -th edge of .
First, removing the -th edge can only increase  by . The maximum increase
occurs situations where a tree component  is broken up into two smaller
tree components  and . Suppose that there are  vertices in  and  vertices 
in , we have 

where  if  and  otherwise. 
If , we have
 
If , we have (since )
     
Secondly, adding the -th edge can only decrease  by . The maximum
decrease occurs in situations where two tree components are merged into a larger one, and
 as well. 

Therefore, the maximum net change of  is  and is achieved when 
 and , or    
 and . Consequently, 

The proves the lemma.
\end{proof}

To complete the proof of Theorem~\ref{thm-rigid-conditional}, we estimate in the following 
lemma the expected number of tree components .   
\begin{lemma}
\label{lem-expected-tree}
Let  be the number of tree components on at most  vertices in . 
We have 
 
\end{lemma}
\begin{proof}
Let  be a vertex set in  and recall that in , the 
 edges are selected uniformly at random and with replacement. Conditional on the 
event that  is a balanced -partition, the  edges are
selected from the set  uniformly at random with replacement. Therefore for 
, the probability that  is an induced tree component in  is

For the case of , the probability  is the 
probability that the single vertex in  is isolated in , and thus

Since there are  vertex subsets of size  in , the expected 
number of tree components in  on at most  vertices is

Since ,
we have that for sufficiently large 

This proves Lemma~\ref{lem-expected-tree}.
\end{proof}

To complete the proof of Theorem~\ref{thm-rigid-conditional}, we see that
Equation~(\ref{eq-rigid-probab}) follows from 
Lemma~\ref{lem-max-diff}, Lemma~\ref{lem-expected-tree}, 
and Equation~(\ref{eq-bounded-difference}). 
\end{proof}

 
\subsection{Proof of Theorem \ref{thm-treewidth-bound}}
We prove Theorem~\ref{thm-treewidth-bound} by applying Markov's inequality and 
the upper bound obtained in Section 3.1 on the conditional probability
of a -rigid and balanced -partition. 

Let  where  is a sufficiently small number to be determined
at the end of the proof.  Let  be the total number of balanced -partition
 such that , and  
let  be the total number of balanced -partition
 such that  and  is -rigid.    

By Lemma~\ref{treewidth-partition-lem}, if the treewidth of  is at most , then 
either  or . It follows that
 
If we can show that  tends to zero  as  goes to infinity, 
Theorem~\ref{thm-treewidth-bound} follows from Markov's inequality. 

Define

For the expectation of , we have   
\begin{lemma}
\label{lem-j-1} 
For any , there is a constant  such that for any ,  
.
\end{lemma}   
\begin{proof}
Consider a partition  of
the vertices of  such that . Since
, we see that 
 if and only if 
. 

The probability that  is a balanced -partition is
 
For a fixed vertex set , there are  ways
() to choose the pair  such that
one of them has the size . It follows that

Since  attains its maximum at  and the function 
 is increasing in the interval , we have
by Stirling's formula (Lemma~\ref{lem-stirling}) that
            
For any , there is some  such that 
for any . Since 
, there
exists some  such that 
.

Taking , we see that for any ,  

where . Lemma~\ref{lem-j-1} follows. 
\end{proof}   
   
For the expectation of , we need to take into consideration the requirement of 
being -rigid in order to get a better bound.  
\begin{lemma}
\label{lem-j-2}
For , there is a constant  such that for any , 
. 
\end{lemma}    
\begin{proof}
Consider a partition  of
the vertices of  such that
. Let 
be the indicator function of the event that
 is a -rigid and balanced -partition. We have

From Theorem \ref{thm-rigid-conditional}, we know that

By the definition of a balanced partition,

For a fixed vertex set  with , there are  ways
() to choose the pair  such that
. Therefore,

By Lemma~\ref{lem-stirling}, we have for  large enough

Recall that  

and see Equation (\ref{eq-rgx-def}) for the definition of  and .
By Lemma~\ref{lem-function-1},  and  are decreasing on . Consequently  is increasing on .
It follows that

By Lemma~\ref{lem-function-2},

Therefore,

Consider the function 

Numerical calculations using MATLAB shows that for ,\ and , we have

Since  is continuous 
in  and  on , there exist constants 
 and  such that 

By Lemma~\ref{eq-function-0}, there exits a constant  such that 

Let . It follows that
 for any  and ,   
 
for some constant .
This proves Lemma~\ref{lem-j-2}.
\end{proof}


It follows from Equation (\ref{eq-markov}) that for any ,

Since the property that the treewidth of a graph is greater 
is a monotone increasing graph property, we have that 
for any ,
   
Theorem~\ref{thm-treewidth-bound} follows.
\qed


\section{Treewidth of Random Intersection Graphs: Proof of Theorems~\ref{thm-intersection-graph}}
\label{sec-thm-2}




Let .
Consider a balanced triple  with  and
. We upper bound the 
probability that  is a balanced -partition and then use Markov's inequality. By the definition of random intersection graphs, there is no edge between the two vertex sets  and  if and only if 

which in turn is equivalent to the following: for every , 

Since 's are formed independently and since  for any , the probability for the event in Equation~(\ref{eq-thm2-2}) to occur is
   
It follows that 

There are  ways to choose  and for each fixed , there
are  ways to choose  with . Since the treewidth of  is at most  implies that there is a balanced -partition, we have by Markov's inequality  that for ,

where last inequality is because the function 
is decreasing on  for any .
Note that .
Therefore, for sufficiently small , we have 

This proves Theorem~\ref{thm-intersection-graph}. 
\qed

\section{The Barab\'{a}si-Albert Model: Proof of Theorem~\ref{thm-power-law-graph}}
\label{sec-thm-3}
Let  be the set of vertices in  and
. Without loss of generality, assume that the vertices are added to  in this order in the 
iterative construction of .  Let  be the first half of the vertices, i.e, , and 
be the second half . 

Let  be a balanced triple of disjoint vertex subsets 
such that . (See Definition~\ref{rigid_partition_def} for the details).
Write  and . Assume, without loss of generality, that  
so that .   
Considering the way in which  and  intersect with  and , let us write

where  and  shall satisfy
 

We upper bound the probability 
.
Let  be the event that  is a balanced -partition, and
focus on what happens when the second half of the vertices, i.e. those in , are added
to . Define the following events

We have

Therefore,

The following lemma  bounds the conditional probability of  given . 
\begin{lemma}

\end{lemma}
\begin{proof}
Consider a vertex  (The case that  is similar).  
The total vertex degree of 
is . The total vertex degree of the vertices in  is at least 
. Note that the event  occurs implies that none of the vertices in 
 is selected as the neighbor of  in the -step procedure to pick 
's neighbors.

By the definition of preferential attachment mechanism in the Barab\'{a}si-Albert model, Equation~(\ref{eq-preferential-prob}), we have that 

This proves the lemma.   
\end{proof}

Continue the proof of Theorem~\ref{thm-power-law-graph}. From Lemma,   
we have
 
Taking into consideration that , we see  that

Consider the behavior of the function 

for  and . 
We have
\begin{lemma} 
There is a constant  such that for any , 

\end{lemma}
\begin{proof} 
Note that the last term  of  can be made arbitrarily 
to 1 by requiring that  is less than a sufficiently small number, say .  
We, therefore, only need to consider the function 


First, we claim that 
for any . 
To see this, we  take the logarithm on both sides of Equation~(\ref{eq-thm3-f-def}) to obtain

Taking derivative on both sides in the above, we get

Since for any  and ,  
 and
, we see that
. The claim
holds since .

Now consider the interval . Let  be a constant such that for any 
, .
Split  into  segments
and consider the  intervals  where
. Since 

is decreasing in ,  for any  and 
, and  is decreasing in , we have
  
Numerical calculations\footnote{We also tried  up to 10, 000, and found out that the value seems to converge to 0.9424} using  gives us 
.
Take , we get Equation~(\ref{eq-thm3-max}).  
\end{proof}


To complete the proof of Theorem~\ref{thm-power-law-graph}, we see from Markov inequality
that the expected number of balanced -partition is at most
 
Numerical calculation shows that . Since , 
we have by Lemma~\ref{lem-stirling}
and Lemma~\ref{eq-function-0} that there is a constant  such that for any 
, 


Let  where  is the constant required 
in Lemma~\ref{eq-thm3-max}. It follows that
for any , the expected number of balanced -partitions in 
 tends to zero, and consequently 
 
This completes the proof of Theorem~\ref{thm-power-law-graph}. 

\appendix 
\section{Proof of Proposition~\ref{prop-equiv}}
\label{appendix-1}
The result actually holds for any monotone increasing combinatorial property in 
random discrete structures,    
as has been observed in \cite{kirousis94threshold,achlioptas99thesis} and formally proved in \cite{kirousis96tech}. For completeness, we give an alternative pure measure-theory style 
proof here.   

Recall that a random graph can be identified with a properly-defined 
probability space. The Erd\"{o}s-R\'{e}nyi random graph  corresponds to the probability space   where 
is the collection of the  subsets of  edges (), and 
 is 
    
Each sample point  corresponds to a set of  edges selected 
uniformly at random without replacement from the  potential edges.  

The random graph  , where the  are selected uniformly at random, 
but with replacement, can be identified with the following 
probability space   

where
\begin{enumerate} 
\item  where each 
 is the set of all  possible edges. 
A sample point 

corresponds to a multip-graph with  edges.   
\item The probability measure   is 

Each sample point  is an outcome of the 
random experiment of selecting  edges independently and uniformly at random with replacement
from the set of all possible edges. Also note that the graph represented by 
a sample point in  is actually a multi-graph, i.e., there are may be
more than one edges between a pair of vertices.  
\end{enumerate}

Let  be a fixed constant.
Let  be the set of sample points 
 such that the treewidth of the multi-graph determined by 
 is greater than , and let  
 be the set of sample points  such that the treewidth of the simple graph determined by  is greater than . 
   
For each , let 
 be the set of distinct edges 
that  has, and
let 
be the set of sample points in  that have exactly  distinct edges.
For each sample point , define 
   
We claim that   satisfies the following



If there is an  that belongs to both  and , 
then it must be the case that . Therefore, . To see that 
, note that any one-to-one mapping  between
the two sets of edges  and  defines a one-to-one mapping between
 and . 

From Equations (\ref{eq-app-1-par-1}) through (\ref{eq-app-1-par-3}), 
the additive property of a probability measure, and the fact  that ,
we see that for any ,


Since parallel edges have no impact on treewidth, we have 

and consequently
 
We have

where the second last inequality is due to the fact that the graph property represented 
by the set of sample points   is 
monotone increasing and Theorem 2.1 in \cite{bollobas01} on the probability 
of monotone increasing properties in the Erd\"{o}s-R\'{e}nyi random graph .  
This completes the proof of the proposition.
\qed
 
\section{Proof of Lemmas~\ref{lem-function-0},~\ref{lem-function-1} and~\ref{lem-function-2}}
\label{appendix-2}


\subsection{Proof of Lemma~\ref{lem-function-0}}
Taking derivative on both sides of 

we see that  is increasing on  and decreasing
on . The lemma follows.
\qed

\subsection{Proof of Lemma~\ref{lem-function-1}}
To show that the function

is decreasing in  on the interval , we show that
the its derivative . 
To this end, we take take the derivative of the logarithm of   

to get

Since  and , 
we only need to show that the numerator of the right-hand
side in the above, i.e.,
the function

is less than zero.
 
Note  and
. As  is continuous, we have 
that for sufficiently small ,  as well.     
It is thus sufficient to show
that  itself is monotone. The first and second derivatives of the function
 are respectively

and

Note that as a quadratic polynomial,
 can be shown to be always less
than 0 for any .  As  is continuous and
, we see that for sufficiently small
,  as well.
It follows that
. Therefore
 is monotone as required.
\qed

\subsection{Proof of Lemma~\ref{lem-function-2}}
First, since both   and 
are increasing on the interval , we have that
 
is increasing on the interval   .

Focusing now on the interval , 
let us consider the  logarithm of the function ,

The derivative of  is

and . The second-order derivative of  is

where

First, assume that . On the interval ,
we have


and

It follows that

Since the family of functions  are uniformly
continuous on , we have that for small enough
, . 
It follows that the second-order derivative  is always greater than zero.
Since , we have that . It follows that  is increasing. Consequently,  is also increasing since 
, 
\qed




 
\section*{Acknowledgment}
\noindent
A  weaker version of Theorem \ref{thm-treewidth-bound} was reported  in \cite{yong06treewidth}.
The research is supported by Natural Science and Engineering Research Council of Canada (NSERC) RGPIN 327587-06 and RGPIN 327587-09.

\bibliographystyle{plain}
\bibliography{../treewidth-cocoon06,bib/random_graph,bib/phase_transition,bib/complex_networks}
\end{document}
