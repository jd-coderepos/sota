\documentclass[a4paper,10pt]{article}
\usepackage[utf8]{inputenc}
\usepackage{amsmath,amsfonts,amssymb}

\newtheorem{theorem}{Theorem}
\newtheorem{lemma}[theorem]{Lemma}
\newtheorem{definition}[theorem]{Definition}
\newtheorem{remark}[theorem]{Remark}
\newtheorem{example}[theorem]{Example}
\newtheorem{corollary}[theorem]{Corollary}
\newtheorem{proposition}[theorem]{Proposition}
\newenvironment{proof}[1][Proof]{\noindent\textbf{#1: }}{\ \rule{0.5em}{0.5em}}


\title{Weighted paths between partitions}
\author{Giovanni Rossi\\
\footnotesize{Department of Computer Science and Engineering - DISI}\\
\footnotesize{Mura Anteo Zamboni 7, Bologna 40126, Italy, giovanni.rossi6@unibo.it}}

\begin{document}

\maketitle

\begin{abstract}
How to quantify the distance between any two partitions of a finite set is an important issue in statistical classification, whenever different clustering results
need to be compared. Developing from the traditional Hamming distance between subsets or cardinality of their symmetric difference, this work considers alternative
metric distances between partitions. With one exception, all of them obtain as minimum-weight paths in the undirected graph corresponding to the Hasse diagram of the
partition lattice. Firstly, by focusing on the atoms of the lattice, one well-known partition distance is recognized to be in fact the analog of the Hamming distance
between subsets, with weights on edges of the Hasse diagram determined through the number of atoms in the unique maximal join-decomposition of partitions. Secondly,
another partition distance known as ``variation of information'' is seen to correspond to a minimum-weight path with edge weights determined by the entropy of
partitions. These two distances are next compared in terms of their upper and lower bounds over all pairs of partitions that are complements of one another. What
emerges is that the two distances share the same minimizers and maximizers, while a much rawer behavior is observed for the partition distance which does not 
correspond to a minimum-weight path. The idea of measuring the distance between partitions by means of minimum-weight paths in the Hasse diagram is further explored
by considering alternative symmetric and order-preserving/inverting partition functions (such as the the rank, in the simplest case) for assigning weights to edges.
What matters most, in such a general setting, turns out to be whether the weighting function is supermodular or else submodular, as this makes any minimum-weight path
visit the meet or else the join of the two partitions, depending on order preserving/inverting. Finally, two appendices are devoted respectively to a definition of
Euclidean distance between fuzzy partitions and the consensus partition (combinatorial optimization) problem.

\textbf{Keywords:} partition lattice, symmetric function, Hamming distance, Hasse diagram, geodesic distance, indicator function, graph of a polytope.

\textbf{MSC numbers:} 05A18, 05C12.
\end{abstract}

\section{Introduction}
Partitions are key instruments in many applicative scenarios at the interface of computer science, artificial intelligence and engineering, including pattern recognition,
data mining and bioinformatics, while also being \textit{``of central importance in the study of symmetric functions, a class of functions that pervades mathematics in
general''} \cite[p. 39]{Knuth2005} (see also \cite[Chapter 5]{RotaWay2009}, \cite{RosasSagan2006} and \cite[Chapter 7]{Stanley2012EnuCom} on symmetric function theory).
Below, symmetric functions are employed to define metric distances between partitions, which in turn are useful when different clustering results need to be compared. In
statistical classification, partitions of a data set may indeed be referred to as ``clusterings'', although the latter term relates to a richer set of structures than the
former. The issue addressed here typically arises since a local search clustering algorithm generally provides different outputs when initialized with different candidate
solutions (or inputs). On the other hand, a chosen clustering algorithm shall allow for different parametrizations, each yielding different results for the same data.
Finally, alternative clustering algorithms commonly partition the same set in alternative ways. In all these cases, a distance measure is essential for assessing the
proximity between diverse partitions \cite{Meila2007,Mirkin2013,CentralPartition}.

The issue is attracting considerable attention since the mid 60s \cite{Lerman1981,Rand1971,Renier1965}. More recently, since measuring the distance between partitions of a
population is fundamental for sibling relationship reconstruction in bioinformatics, several contributions over the last decade adopted a combinatorial approach for studying
one specific such a distance measure, here denoted by MMD as it relies on \textit{maximum matching}
\cite{Berger-Wolf+al2007,Konovalov2006,Konovalov+al2005B,Konovalov+al2005A,Sheikh+al2010}. More precisely, MMD can be shown \cite{Almudevar+1999,Day1981,Gusfield2002} to be
computable via the assignment problem \cite{KorteVygen2002}. Also, in most recent years sibship reconstruction has been tackled by means of a further partition distance
measure \cite{Brown+2012}, obtained axiomatically from information theory \cite{Meila2007} and called \textsl{variation of information} VI.

In this work, entire families of metric distances between partitions are considered, the principle aim being to have consistency and generalizations in terms of order (i.e.
lattice) theory. In fact, the general leading idea is the same as in \cite{LeclercMondjardet+1986,Leclerc1993,LeclercMondjardetLatticialConsensus,Monjardet1981},
namely to define distances between elements that are (partially) comparable in terms of a binary order relation, although attention is not limited to posets (partially
ordered sets), distributive lattices and semi-lattices, but mainly extends to the geometric lattice of partitions. Metrics for distributive lattices are usually defined in
terms of valuations (or modular lattice functions, such as the rank or cardinality of subsets in the Boolean case, see below). Conversely, valuations of the partition lattice
are constant functions \cite{Aigner79}, and therefore useless for defining metrics. Thus, the method proposed here relies on super/submodular lattice functions (referred to
as lower/upper valuations in \cite{LeclercMondjardet+1986}).

The first goal is to reproduce the traditional Hamming distance between two subsets, given by the number of atoms of the subset lattice included in either one but not in both
(i.e. the cardinality of their symmetric difference, see \cite{Bollobas86}). Such a benchmark is extended to the geometric lattice of partitions by focusing on atoms and
join-decompositions of lattice elements \cite{Aigner79,Stern99}. While every subset admits a unique such a decomposition, involving a number of atoms equal to the cardinality
(or rank) of the subset, a generic partition admits different join-decompositions, most of which redundant. The number of atoms involved in the unique maximal
join-decomposition of a partition is here referred to as the \textit{size} of that partition, yielding a function taking positive integer values, like the rank. In fact, the
two coincide for subset lattices but differ crucially for partition lattices. Roughly speaking, replacing the rank with the size yields a (i.e. the) Hamming distance between
partitions, denoted by HD. Apart from the resulting consonance in terms of ordered structures, HD and VI share important characterizing axioms (see \cite{Meila2007}).
Computable through scalar products between Boolean vectors, without any algorithmic issue, HD has a large range, and thus fine measurement sensitivity too.

The traditional Hamming distance between two subsets of a -set is also the length of a shortest path between them in the Hasse diagram of the Boolean lattice of subsets.
Such a diagram is in fact the graph of the polytope \cite{BronstedConvex,Branko2001} given by the -dimensional unit hypercube , thus it has  vertices that
bijectively correspond to subsets, and an edge links any two vertices when the two corresponding subsets are comparable in terms of the covering relation (see
\cite{Bollobas86,AlgeGraph} and below). In order to have exactly the same for the Hamming distance between partitions, these latter must be seen to correspond bijectively to
those graphs on  (labelled) vertices each of whose components is complete. More precisely, denoting by  the complete graph on vertex set ,
with , partitions correspond bijectively to those graphs  each of whose components is a maximal complete subgraph or
a clique, and the geometric lattice of partitions of  is the so-called polygon matroid defined on the edges of  \cite[pp. 54, 259]{Aigner79}. The associated Hasse
diagram is thus recognized to be the graph of a polytope strictly included in the -dimensional unit hypercube . Specifically, the
-set  of hypercube vertices identifies the -set of distinct graphs on vertex set , whereas linear dependence
\cite{Whitney1935} entails that partitions only span  hypercube vertices, where  is the \textit{Bell number} of partitions of a
-set () \cite{Aigner79,GrahamKnuth+1994,Rota1964}. While the covering relation between subsets assigns a unit weight to every edge of the -cube \cite{Sebo+04},
edges of the polytope of partitions must be weighted through the size, which matches precisely the number of edges of the -cube that collapse into a unique
edge of the included polytope. With these weights, the Hamming distance HD between partitions (like between subsets) is the minimum weight of a path connecting them.

The analysis then continues by observing that the size may be replaced with any alternative symmetric and (strictly) order-preserving/inverting partition function, such as
rank, entropy, logical entropy \cite{EllermanLogicalEntropy,EllermanLogic} and co-size (see below). Then, polytope edges have weights obtained as the difference between the
greater and the smaller value taken by the chosen function on the associated endpoints. Accordingly, the distance between two partitions remains the minimum weight of a path
connecting them. In particular, if the function assigning weights to edges is order-preserving and supermodular (like the size) or else submodular (like the rank), then the
minimum-weight path between any two partitions visits their meet or else their join, respectively. Analog results obtain for order-inverting and symmetric functions which
are either supermodular or else submodular.

Section 2 outlines the needed background, with emphasis on lattice functions and Hamming distances in general, while Section 3 introduces the proposed Hamming distance between
partitions, including an axiomatic characterization. Section 4 is devoted to bounding both the Hamming and variation-of-information distances over all pairs of partitions that
are complements of one another. Section 5 frames distances as minimum-weight paths in the Hasse diagram. Section 6 considers two further functions assigning weights to edges,
namely logical entropy and co-size. Sections 7 and 8 are two appendices detailing respectively a definition of Euclidean distance between fuzzy partitions and an exact
solution for the consensus partition (combinatorial optimization) problem. Section 9 concludes the paper with some final remarks.


\section{Preliminaries}
Throughout this work, the general concern is with metric distances  between elements  of a poset , i.e. a (finite) set  endowed with a
partial order relation . Additionally,  shall also be endowed with the meet  and join  operators, so that  is a (complete) lattice
(see \cite{DaveyPriestley}). The ordered structures to be considered are grounded on a finite set , where integers  possibly denote the indices
of a data set. In particular, attention is going to be placed on the Boolean lattice  of subsets of  ordered by inclusion  and, mostly, on the
geometric lattice  of partitions of  ordered by coarsening  (see \cite{Aigner79,Stern99}). Generic subsets and partitions are denoted
respectively by  and . Recall that a partition  is a collection of (non-empty) pair-wise disjoint subsets, called
blocks, whose union is . For any , if , then every block  is included in some block , i.e. . Hence the bottom
partition is  (like the bottom subset is ), while the top one is  (like  is the top subset). Also, among
partitions the meet  is the coarsest-finer-than operator, while the join  is the finest-coarser-than operator. The number  of
partitions of  is defined recursively by  and  (see \cite{Aigner79,GrahamKnuth+1994,Rota1964} on
Bell numbers).

For all ordered pairs  of poset elements, the associated interval or segment is , and  is said to cover ,
denoted by , if . The Hasse diagram of poset  is the graph  whose vertices are elements  and edges are given by the covering
relation, i.e. . Although these edges are sometimes assumed to be directed, thereby also indicating what elemets are covered/covering, still in the
present setting they are more fuitfully regarded as undirected, for this allows to consider paths where edges may be used in both directions. In fact, the distance between any
two vertices in a graph is the length of any shorthest path between them. More generally, if the graph is weighted, meaning that every edge has an associated (strictly positive)
weight, then the distance between any two vertices is the weight of a lightest path between them, where the weight of a path is the sum over its edges of their weight.

In a lattice  with bottom element , the set  of atoms consists of all lattice elements that cover the bottom one.
In atomic lattices, every element  admits a decomposition  as a join of atoms . Both the Boolean lattice
 of subsets of  and the geometric lattice  of partitions of  are atomic. For the former, atoms are the  singletons
. For the latter, atoms are the  partitions consisting of  blocks, out of which  are singletons while the remaining one is a pair. Most
importantly, every subset  admits a unique join-decomposition, namely . Conversely, partitions generally admit several join-decompositions.
However, every partition  admits a unique maximal join-decomposition, which includes all atoms finer than . In the sequel, a great deal of attention shall
be placed on such a number of atoms finer than any given partition, to be referred to as the size of partitions.

\subsection{Lattice functions}
In order to consider alternative weights over the edges of the Hasse diagram, it is necessary to deal with different lattice functions . Firstly,
from a geometric perspective,  is a point in a vector space. A well-known basis of this vector space is , where
 for all . Thus, any  is a linear combination
 of basis elements, with coefficients  given by M\"obius inversion , where this latter obeys the
following recursion:  for all , hence  and  for all  (see
\cite{Aigner79,RotaMobius,Stern99}).

A lattice function  is said to be:
\begin{itemize}
\item strictly order-preserving if  for all  such that ,
\item strictly order-inverting if  for all  such that ,
\item supermodular if  for all ,
\item submodular if  for all ,
\item modular if  for all ,
\item totally positive if  for all .
\end{itemize}
\textbf{Observation:} if  is totally positive, then it is supermodular. To see this, firstly note that if  and  are comparable, i.e. say , then there is
nothing to show as  and , and thus the inequality defining supermodularity is satisfied with equality. Apart from this trivial case, if  and  are
uncomparable, i.e. , then substituting the general M\"obius inversion formula above, i.e. , into the inequality
defining supermodularity formula yields 

where of course  by definition of meet.

Further lattice functions to be considered are symmetric ones, i.e. those that are invariant under the action of the symmetric group  consisting of all 
permutations  (see \cite[p. 161]{Aigner79}). Symmetric functions are generally very important in mathematics; for reasons of space only essential facts are here
exposed, with focus on lattices  and . For any , let , where 
is the index mapped into the -th position by . A set function  is symmetric if  for all . Thus, 
is symmetric if  for all  such that . As for partitions, for every  let  be the class or
type of  (see \cite{RotaMobius}), that is to say . For all  and , let
. A partition function  is symmetric if  for all .
Thus,  is symmetric if  for all  such that .

\subsection{Hamming distance between subsets}
First of all recall that measures of the distance between elements of any (i.e. possibly non-ordered) set are referred to as ``Hamming distances'' when these elements are represented
as arrays or matrices and the distance between two of them is the number of entries where their array or matrix representations differ. The issue introduced in Section 1, namely how to
measure a distance  between any two partitions , is firstly addressed in the following Section 3 by reproducing the traditional Hamming distance 
between subsets , where . This distance measure can also be expressed as ,
where  is the rank function, i.e.  for all . The essential combinatorial feature of  is that it counts how many atoms
 of Boolean lattice  are included in either  or else  but not in both. Also,  is a Hamming distance since subsets 
are represented as Boolean -vectors , with characteristic function  defined by  if  and  if
, for all . Thus,  is precisely the number of entries where  and  differ
\cite{Aigner79,Bollobas86}. Evidently, characteristic functions  provide a bijection between the -set of subsets  and the vertices  of
the -dimensional unit hypercube . In fact, the graph of this latter polytope \cite{BronstedConvex,Branko2001} is the Hasse diagram of Boolean lattice , the
two sharing the same vertices and edges, and  is the length of a shortest path connecting vertices  and . Clearly, a shortest path is also a minimum-weight
path as long as each edge has unit weight, which is precisely what happens when edges are weighted by the rank.

For any two points  in the unit -cube, let  denote their scalar product. Since  is the -vector all
of whose entries equal 1, for all  it holds . Three further expressions for the Hamming distance between subsets  are



Furthermore, the following two observations are immediately checked.
\begin{itemize}
\item  is a strictly order-preserving, symmetric and modular lattice (i.e. set) function, and
\item  is a metric: for all ,
\begin{enumerate}
\item ,
\item , with equality if and only if ,
\item , or triangle inequality.
\end{enumerate}
\end{itemize}

\section{Partition distances}
In a (simple) graph  with vertex set  the edge set  is included in the -set of unordered pairs
of vertices. As already mentioned, the complete graph on these  (labelled) vertices is , and the Hamming distance  between partitions defined in the sequel
reproduces  while keeping into account that partitions of  correspond bijectively to those graphs with vertex set  whose components are each a complete subgraph
\cite{Aigner79}.

The combinatorial analog of  in terms of partitions , namely the number of atoms of  finer than either  or  but not finer than both,
exists in the literature \cite{Meila2007,Rand1971}, but is commonly not recognised to be such an analog. Conversely, the name ``Hamming distance between partitions'' is often customarily
maintained for a metric obtained by representing partitions  as Boolean matrices , despite these latter correspond in fact to generic binary relations on 
\cite[p. 393]{Mirkin1996}. Since partitions only correspond to equivalence relations, it is readily seen there are  binary relations which are \textit{not} equivalence
relations, yielding both conceptual and quantitative ambiguities (detailed below). In addition, the metric obtained by representing partitions  as matrices  does not yield any
shortest path between vertices of the Hasse diagram of partitions. In general, it seems desirable that the distance between elements of a ordered set (such as  and ) is
measured in terms of the order relation, like  is specified in terms of . That is to say, in formal notation,
.

There exist many partition distance measures available in the literature, \cite[Sections 10.2, 10.3, pp. 191-193]{DezaDeza2013}, \cite[Chapter 5]{Mirkin1996}
\cite{Day1981,HubertArabie1985,Warrens2008}. Towards a clear disambiguation between the so-called \textit{Hamming distance between (matrices representing) partitions}
\cite{Meila2007,Mirkin+1970,Mirkin+2008} mentioned above and what is proposed here, recall that a binary relation  on  is a subset  of
\textit{ordered} pairs  of elements  (hence unordered pairs  satisfy , while  for ordered ones). The collection of all
such binary relations is a Boolean lattice . If symmetry  and  \textit{transitivity}
 hold, then  is an \textit{equivalence} relation, or a partition of  into equivalence classes: -maximal
subsets  such that  for all  are precisely its blocks. A binary relation  may be represented as a Boolean matrix
 with entries  if  and  if . Now let two equivalence
relations  have associated partitions  and representing matrices . The distance  between
subsets  can be computed as
. This is the number of 1s in matrix
 modulo 2. While providing a distance between partitions  and , this is in fact the traditional Hamming
distance between certain subsets , while generic such subsets  correspond to partitions only in very special cases, as
lattice  contains  elements, or binary relations, that do not correspond to partitions, or equivalence relations. The argument
also applies when partitions are represented as Boolean -matrices through the complement  of equivalence relations , known as
\textit{apartness relations} in computer science \cite{EllermanLogicalEntropy,EllermanLogic}, i.e.  (this is detailed below).

The point is that in finite sets such as  and  where there is no ``natural'' metric (like the Euclidean norm in ), the distance between elements
 and  must be quantified, in some way, by the number of elements  between  and , where ``between'' means that  must be comparable, in terms of the order relation, with
 and/or . To achieve this, in the present setting, consider that the partition lattice  is a matroid (see \cite{Aigner79,Stern99} and above). However
regarded, it is necessarily embedded into a larger subset lattice, with which some elements are shared while some others are not. Apart from binary relations just described, a na\"ive
example comes from noticing that partitions  are collections of subsets, i.e. , and thus the distance between  and  might be computed as the Hamming distance
 between elements of subset lattice , i.e. the number of subsets  that are blocks of either one but not both. Again, there are really many (i.e.
) set systems (or collections  of subsets) that do not correspond to partitions. This feature is maintained even when  and  are decomposed
as joins of atoms, for they generally admit several such join-decompositions \cite[Chapter II]{Aigner79}. Yet, when regarded from this perspective partition lattice
 is seen to be included in subset lattice , with the two sharing the same  atoms. In fact,  is the \textit{minimal}
Boolean lattice including the partition lattice. Accordingly, the Hamming distance between partitions HD proposed below relies precisely on representing partitions as Boolean
-vectors, although only  distinct such vectors correspond to partitions. In particular, HD is the traditional Hamming distance 
between edge sets  of graphs on vertex set , with these latter corresponding to partitions only when in both graphs  each component is a complete
subgraph.

\subsection{Hamming distance between partitions}
In combinatorial theory, both  and  are geometric lattices \cite[p. 54]{Aigner79}. As such, they are atomic, meaning that every
element is decomposable as a join of atoms (see above). The rank function  of the partition lattice is , with height
 and  for the top and bottom elements, respectively. As already outlined, atoms are immediately above , with rank , in the associated
Hasse diagram \cite[p. 889]{Meila2007}, where coarser partitions occupy upper levels. Thus, atoms are those partitions consisting of  blocks, namely  singletons and one
pair. These  pairs  are the same atoms as in Boolean lattice . Notationally, it is now convenient to let 
be the atom where the unique -cardinal block is pair  (this is denoted by  in \cite[p. 150]{LeclercMondjardetLatticialConsensus}, where  are elements of
the partitioned set while  denotes the generic partition).

In order to have a combinatorially congruhent reproduction of the Hamming distance between partitions, let  be the -set of atoms
of the partition lattice, with isomorphism . The analog of characteristic function  is \textit{indicator function}
, defined by
  
In words, if pair  is included in some block  of , i.e. , then partition  is coarser than atom , and the corresponding position
 of indicator array  has entry . Otherwise, that position is . For the top partition , indicator function  is the
-vector with all entries equal to 1. For the bottom partition , analogously  is the -vector all of whose
entries equal 0. The number  of atoms finer than any partition  is \cite{Rossi2011} the \textit{size} 
mentioned in Section 1, i.e.

While the cardinality  of subsets takes every integer value between  and , the size  of partitions
does not the same between  and . Minimally, this is already observable for , as there are  partitions: the finest
 and coarsest  ones, together with the  atoms ,  and .
Thus, there is no partition with size equal to , as . Available sizes of partitions of a -set, for
, are in Table 1 below.
\begin{table}[htbp]
\caption{\textsl{Available sizes of partitions of a -set, .}}
\label{tab: available sizes}
\begin{center}
\begin{tabular}{|c|c|c|}
\hline
 &  (available sizes)\\
\hline
1& \\
\hline
2 & \\
\hline
3 & \\
\hline
4 & \\
\hline
5 & \\
\hline
6 & \\
\hline
7 & \\
\hline
\end{tabular}
\end{center}
\end{table}

Both lattices  and  are atomic, with every element  and  admitting a decomposition as a join of atoms. Yet, while subsets
 (or edge sets of graphs with vertex set ) admit a unique such a decomposition, namely , partitions generally admit several such
decompositions . For  as above, the coarsest partition  decomposes either as the join of any two atoms, or else as the join of all
the three available atoms at once. In particular, the rank  of  is the \textit{minimum} number of atoms involved in a join-decomposition of , while the size  is
the \textit{maximum} number of atoms involved in such a decomposition. Hence, the coarsest partition  of a -cardinal set has rank  and size
. 

The rank  of partitions is well-known to be strictly order-preserving, symmetric and submodular, while the size  is strictly order-preserving, symmetric and
supermodular. This is shown below.

\begin{lemma}
The size is a strictly order-preserving partition function: if , then , for all .
\end{lemma}
\begin{proof}
If , then every  is the union of some , i.e. , with  for at least one . The
union  of any  increases the size by

which is strictly positive as blocks are non-empty.
\end{proof}

In order to reproduce expressions (1-2) of Section 2.2 above, Hamming distance HD between partitions has to count the number of atoms finer than either one of any two partitions
but not finer than both. Thus, in terms of cardinalities of subsets of atoms, distance  is given by

The size and the indicator function allow to obtain HD as follows:

Also note that , and this is the maximal decomposition of  as a join of atoms, namely that involving
 atoms. Therefore,

In view of expressions (1-4), there seems to remain no doubt that, from a combinatorial perspective,  is in fact the faithful translation of the traditional Hamming
distance  from subsets  to partitions .

\subsection{Two further partition distances}

Two non-Hamming partition distances are now briefly introduced, since they provide a term of comparison for the following sections and also in view of the recent literature in
bioinformatics cited in Section 1. Any subset  has a unique complement . For all partitions  and all non-empty subsets , let
 denote the partition of  induced by . Maximum matching distance  between partitions  is

This is the minimum number of elements  that must be deleted in order for the two residual induced partitions to coincide. Also,  \textit{``is the minimum number
of elements that must be moved between clusters of  so that the resulting partition equals ''} \cite[p. 160]{Gusfield2002}. It is computable as a maximum matching or
assignment problem \cite{Day1981}, \cite[chapter 11]{KorteVygen2002}. In a graph a matching is a set of pairwise disjoint edges, i.e. the endpoints are all different vertices. Now consider the
bipartite graph  with  vertices, one for each block of each partition, and join any two of them  and  with an edge  if
. In addition, let  be the weight of the edge. Then, determining  amounts to find a maximum-weight matching  in , that is one
where the sum  of edge weights is maximal. In fact, the minimum number  of elements that must be removed for the two residual partitions to
coincide is the sum  over all selected edges of the cardinality of the symmetric difference between the associated endpoints.

Another important measure of the distance between any two partitions  and  is the variation of information , obtained axiomatically from information theory (see
\cite[Expressions (15)-(22), pages 879-80]{Meila2007}). Entropy  of partitions  (binary logarithm) enables to measure the distance between  and  as

Notice that while the range of MMD is , VI ranges in a finite subset of interval . Most importantly,
the entropy  of partitions  is strictly order-inverting, symmetric and submodular, with  and . To see submodularity, simply consider
 as before, and set  and , yielding  and . Then, 

Finally observe that , in turn, conversely is strictly order-preserving, symmetric and supermodular. It can also be anticipated that VI is in the broad class of metric
distances defined in the sequel, but MMD is not.

\subsection{HD and VI: axioms}
Following \cite{Meila2007}, attention is now placed on those axioms that characterize both partition distance measures HD and VI. An alternative axiomatic characterization
of HD appears in \cite{MirkinCherny1970}. The following proposition may be compared with \cite[pp. 880-881, Property 1]{Meila2007}.
\begin{proposition}
HD is a \textsl{metric}: for all ,
\begin{enumerate}
\item ,
\item , with equality if and only if ,
\item , i.e. triangle inequality.
\end{enumerate}
\end{proposition}

\begin{proof}
The first condition is obvious. In view of lemma 1 above, the second one is also immediate as . In fact,  is the sum
 of two positive integers, while  (for any atom ). Concerning
triangle inequality, difference 

must be shown to be positive for all triplets . For any , size  is given, and thus 
has to be minimized by suitably choosing . Firstly, sum  is maximized when both  (or ) and  (or
) hold. Secondly, if , then the whole difference is minimized when . Thus, HD satisfies triangle inequality as long as the
size satisfies supermodularity: . The simplest way to see that this is indeed the case is by focusing on
M\"obius inversion of lattice (or more generally poset) functions (see \cite{Aigner79,RotaMobius} and above). By definition, the size  has M\"obius inversion
 given by  if  is an atom (i.e.  or ), and  otherwise. In fact,
 for all . The size thus satisfies a sufficient (but not necessary) condition for supermoduarity, in that its M\"obius
inversion takes only positive values (see Section 2.1). This completes the proof.
\end{proof}

Triangle inequality is satisfied with equality by both HD and VI as long as  (for VI, see \cite[pp. 883, 888]{Meila2007} Properties 6 and 10(A.2)).

\begin{proposition}
HD satisfies horizontal collinearity:

\end{proposition}

\begin{proof}
 as well as .
\end{proof}

Briefly aticipating the forthcoming analysis, it may be noted that horizontal collinearity may well be conceived in terms of the join, rather than the meet, of any two partitions,
since it is not hard to define distances  satisfying triangle inequality with equality when ; that is to say,
 for all . This is in fact the so-called  ``betweenness'' relation proposed in \cite[p. 176]{Monjardet1981}.

Collinearity also applies to distances between partitions  that are comparable, i.e. either  or . Firstly consider the case involving the top
 and bottom  elements (for VI, see \cite[p. 888]{Meila2007} property 10(A.1)).

\begin{proposition}
HD satisfies vertical collinearity:

\end{proposition}

\begin{proof}
 independently from , as well as .
\end{proof}

Vertical collinearity may be generalized for arbitrary comparable partitions , in that  for all ,
where  is an interval or segment \cite{RotaMobius} of  (see above). In fact, this is precisely the
``interval betweenness'' property considered in \cite[p. 179]{Monjardet1981} (for valuations of distributive lattices).

\section{Distances between complementary partitions}
The distance between the bottom and top elements in vertical collinearity leads to regard such lattice elements as complements, thereby focusing on the distance between other,
generic complements. Maintaining the traditional Hamming distance between subsets as the fundamental benchmark, it must be taken into account that the subset and partition
lattices are very different in terms of complementation. In particular, every subset  has a unique complement , and the distance between any two such complements
equals the distance between the bottom and top elements, i.e.  for all . Conversely, partitions  generally have several and quite
different complements \cite{Aigner79}, which are all those  such that  as well as . In statistical classification, partitions 
satisfying only the former condition, i.e. , are commonly referred to as ``dual partitions'' and investigated as those where the addjusted Rand index ARI
\cite{HubertArabie1985} takes negative values; see \cite[pp. 237-238, 389]{Mirkin1996}, \cite[pp. 429-430]{KovalevaMirkin} and \cite{SchreiderSharov1982}. Apart from this,
concerning complementation and partition distances MMD, VI and HD, the former measures the distance between any two complements  solely through their cardinalities ,
while VI and HD provide a fine distinction between different complements, and also agree on which are closer and which are remoter. The issue may be exemplified with
 and partitions  and  and  (where vertical bar  separates blocks). Both  and  are complements of
, that is  and . Distances MMD, VI and HD are:

Concerning MMD, this examples generalizes as follows.

\begin{proposition}
For any two complementary partitions ,

\end{proposition}

\begin{proof}
If , then every edge  of the bipartite graph  defined in Section 2 above has unit weight . Hence, a
maximum-weight matching simply is one including the maximum number of feasible edges. Such a number is , because each block (of either
partition) can be the endpoint of at most one edge included in a matching. Also, the number of elements  that must be deleted for the two residual partitions to coincide
is . On the other hand,  entails

as desired.
\end{proof}

As shown by the above example, a partition generally has different complements with different classes. The set of complements of any partition  is denoted by
.
A \textit{modular element} of the partition lattice \cite{Aigner79,Stanley1971,Stern99} is any  where all blocks are singletons apart from only one, at most,
i.e. . The sublattice  consisting of modular elements contains the bottom and top elements, together with
all partitions of the form  with , where  is the finest partition of . Hence there are  modular partitions (with
 for ).
Here, the main link between modular elements and complementation is that an element is modular if and only if no two of its complements are comparable
\cite[Theorem 1]{Stanley1971}. Therefore, if , then there are  such that . It seems thus important that the distance
between  and  differs from the distance between  and . The following result bounds the Hamming distance HD between a partition and any of its complements. 

\begin{proposition}
For all , if , then

where the upper bound is always tight, while the lower one is tight only if

\end{proposition}

\begin{proof}
Firstly note that if , then . Hence,

Any complement of partition  has join-decompositions minimally involving  atoms , with
associated pairs  such that . Considering the upper bound first, observe that size
 attains its maximum when  for all , in which case
 for all . This bound is tight because such a complement  always exists, whatever
the class  of . In fact,  has  blocks, out of which  are singletons, while the remaining one  is -cardinal and
satisfies  for all , i.e. . Thus .

Turning to the lower bound, observe that size  attains its minimum, ideally, when  for all
, in which case  for all . Yet, this is not always possible because each block  can have non-empty
intersection with a number of pair-wise disjoint pairs  which is bounded above by , entailing that the constraint is given by the number 
of singletons . Specifically, nesting together  non-singleton blocks requires  pairs . If these latter
have to be pair-wise disjoint, then the maximum number of elements  in non-singleton blocks available to match (into pair-wise disjoint pairs) those elements
 in singletons is .
\end{proof}

In words, if the number  of singleton blocks of partition  exceeds the number  of elements  available to match, into pair-wise
disjoint pairs, those elements  in singletons, then basically a complement  of , in order to yield the top partition  through the join , must
necessarily consist of blocks larger than pairs. Of course, in the limit, if the blocks of  are all singletons, then the unique complement  has to be the
coarsest or top partition, consisting of a unique block. Thus, the greater the number  of singleton blocks of , the fewer and larger the blocks  of a complement
 of  have to be. In this view, the following Proposition 6 shows that the more the cardinality  of these blocks  is evenly distributed, the lower the size
 of the complement . In fact, on any level  of the partition lattice, the size attains its maximum value on
modular partitions (consisting of  singletons and one -cardinal block) and its minimum value on those partitions each of whose block has cardinality between
 and , where  is the floor of , i.e. the greatest integer , while
 is the ceiling of , i.e. the smallest integer , for . As detailed by Proposition 8 next, the opposite occurs for the
entropy of partitions.

\begin{proposition}
If  satisfies , then

where .
\end{proposition}

\begin{proof}
If , then the above proof of Proposition 5 entails that the maximum number  of blocks of a complement of
 is . On the other hand, for , among -cardinal partitions  of a -set the size is minimized when
 for all . Bound  above is
the size of a -cardinal partition  with  for all
. In particular, the number of -cardinal blocks is
, while the number of -cardinal blocks is
.
\end{proof}
 
\begin{proposition}
Among complements  of any , HD and VI have common minimizers, i.e.
,
and common maximizers, i.e. .
\end{proposition}

\begin{proof}
Firstly,  entails . Thus,  is minimized or else maximized when  is, respectively, maximized or else minimized. On
the other hand, if , then all complements  have same rank. Otherwise, as already observed, there are comparable complements, i.e. with
different rank. Therefore, in general, among complements  entropy  is minimized when  is minimized and, in addition, . This
is precisely where size  is maximized. Similarly,  is maximized when  is maximized and, in addition,
 for all . This is where  is minimized.
\end{proof}

\section{Minimum-weight paths between partitions}
This section provides an analysis similar, in spirit, to that provided in \cite[Section 3]{Monjardet1981}, although the generic posets and semilattices considered there are replaced
here with the geometric lattice of partitions. Similarly, the covering graph becomes the graph  of polytope  below, and despite posets lack the join and meet
operators, still \cite{Monjardet1981} defines upper/lower valuations, which correspond to sub/supermodular partition functions in the present setting. Apart from these
differences, still the general idea to define metrics through weighted paths in the graph induced by the covering relation is the same. In fact, 
Hamming distance  between edge sets  is the length of a shortest path between vertices  of the
-dimensional unit hypercube , where  is the characteristic function defined in Section 2, i.e.
 if  and 0 otherwise. Recall that a polytope naturally defines a graph with its same vertices and edges \cite[p. 93]{BronstedConvex}, and the
hypercube is perhaps the main example of polytope. In particular, the graph of hypercube  is the Hasse diagram of Boolean lattice , for its
edges correspond to the covering relation, that is to say  is an edge of the hypercube if either  or else the converse, i.e.
.

Clearly, a shortest path is a minimum-weight path as long as every edge has unit weight. This simple observation is the starting point toward an analog view of the Hamming
distance HD between partitions, namely as the weight of a minimum-weight path in the associated Hasse diagram when edge weights are determined by the size. More generally, if edge
weights are determined by a symmetric and strictly order preserving/inverting partition function (like rank or entropy), then minimum-weight paths across edges of the
Hasse diagram equivalently yield well-defined metric distances.
In this view, consider the convex hull  whose extreme points
\cite{BronstedConvex,Branko2001} are all the  Boolean -vectors defined by the indicator functions  of partitions. Note that
 is a (so-called ``hull onest'') 0/1-polytope that might be included in the classifying literature \cite{Aicholzer+96,Ziegler2000} as a type in and of itself. Here, it may be
referred to as ``the polytope of partitions'', since its graph  basically is the Hasse diagram of partition lattice .
Specifically, edges correspond to the covering relation between partitions, i.e.  if either  or else  (see above). Let
 denote the covering relation between partitions, while  is the set of extreme points or
vertices of . For , polytope  is strictly included in  and its five vertices are , , ,  and
. Thus, vertices ,  and  of  are excluded from , as they correspond to those 3 graphs with vertex set
 whose edge set is 2-cardinal. That is,  is precisely the number of graphs with vertex set  that do not coincide with their closure
(see Sections 1 and 2). Geometrically, for , polytope  is the union of a lower tetrahedron, whose volume is , and an upper tetrahedron,
whose volume is , hence the whole volume is . The former is , while the latter is .
Thus  is the polyhedron obtained as the intersection of 6 half-spaces, namely those three above the hyperplanes each including one of the three facets (different from
unit simplex ) of the lower tetrahedron, and those three below the hyperplanes each including one of the three facets (again different from the unit
simplex) of the upper tetrahedron.
Although this situation for  is quite simple, still for generic  the associated polytope  is more complex. When , for example,
 is the convex hull of the 15 vertices corresponding to the rows of Table 2 below, with columns indexed, from left to right, by the  atoms
[12], [13], [14], [23], [24] and [34] of . Corresponding partitions are in the far left column, with vertical bar  separating blocks.

\begin{table}[htbp]
\caption{\textsl{Extreme points of 0/1-polytope  for }}
\label{tab: extreme points}
\smallskip
\begin{center}
\begin{tabular}{|c|c|c|c|c|c|c|}
\hline
  ;   & [12] & [13] & [14] & [23] & [24] & [34]\\
\hline
 &0 & 0 & 0 & 0 & 0 & 0 \\
\hline
 & 1 & 0 & 0 & 0 & 0 & 0\\
\hline
 & 0 & 1 & 0 & 0 & 0 & 0\\
\hline
 & 0 & 0 & 1 & 0 & 0 & 0\\
\hline
 & 0 & 0 & 0 & 1 & 0 & 0\\
\hline
 & 0 & 0 & 0 & 0 & 1 & 0\\
\hline
 & 0 & 0 & 0 & 0 & 0 & 1\\
\hline
 & 1 & 0 & 0 & 0 & 0 & 1\\
\hline
 & 0 & 1 & 0 & 0 & 1 & 0\\
\hline
 & 0 & 0 & 1 & 1 & 0 & 0\\
\hline
 & 1 & 1 & 0 & 1 & 0 & 0\\
\hline
 & 1 & 0 & 1 & 0 & 1 & 0\\
\hline
 & 0 & 1 & 1 & 0 & 0 & 1\\
\hline
 & 0 & 0 & 0 & 1 & 1 & 1\\
\hline
 & 1 & 1 & 1 & 1 & 1 & 1\\
\hline
\end{tabular}
\end{center}
\end{table}

As for weights on edges  of (covering) graph , let  be the vector space of strictly
order-preserving/inverting and symmetric partition functions . As already mentioned entropy, rank and size are in , and the former
is order-inverting, while the latter two are order-preserving. Given any , define weights
 on edges  by

For all pairs , let  contain all -paths in graph , noting that this latter is highly connected (or dense), as every partition
 is covered by  partitions  and covers  partitions , hence  for all . Recall that a path
 is a subgraph  where

with  for . Also, the weight of a path  is


\begin{definition}
Minimum--weight partition distance  is

\end{definition}

\begin{proposition}
For all  and all , every minimum--weight -path visits  or  or both; that is to say, if path  satisfies
, then .
\end{proposition}

\begin{proof}
If  are comparable, say , then  for \textit{all} paths , in that  and ;
in particular, if , then the unique minimum--weight  path consists of vertices  and  together with the edge  linking them. On the
other hand, if  are not comparable, i.e. , then any path  visits some vertex  comparable with both , and either  or
else . Hence , with , for some -path  and some -path , entailing that the weight
of such a  is . Finally, since  is strictly order-preserving/inverting and symmetric,  minimizes
 over all partitions  while  minimizes  over all .
\end{proof}

Whether a minimum--weight path visits the join or else the meet of any two incomparable partitions clearly depends on . A generic  may have associated
minimum-weight paths visiting the meet of some incomparable partitions  and the join of some others . In fact, whether minimum-weight paths awlays visit the meet or
else the join of any two incomparable partitions depends on whether  or else  is supermodular. As already observed, if  is supermodular, then  is submodular, i.e.
 (and viceversa).

\begin{proposition}
For any strictly order-preserving , if  is supermodular, then the minimum--weight partition distance is

while if  is submodular, then the minimum--weight partition distance is

\end{proposition}

\begin{proof}
Supermodularity entails

whereas submodularity entails

for all .
\end{proof}

\begin{proposition}
For any strictly order-inverting , if  is supermodular, then the minimum--weight partition distance is

while if  is submodular, then the minimum--weight partition distance is

\end{proposition}

\begin{proof}
Supermodularity entails

whereas submodularity entails

for all .
\end{proof}

Since the size  is supermodular (see Proposition 1) and order-preserving,  is the minimum--weight partition distance, i.e.  for all . On the
other hand, the rank  is submodular \cite[pp. 259, 265, 274]{Aigner79} and order-preserving, hence  is the
minimum--weight partition distance. In particular,  for all edges , and therefore  is in fact the shortest-path distance. This
is detailed below by means of Example 2. Finally, entropy  is order-inverting and submodular, hence the minimum--weight distance  is the VI distance
, as shown in Example 1 hereafter. Propositions 9 and 10 are summarized in Table 3 below.

\begin{example}
\textbf{Entropy-based minimum-weight path distance:} for any two atoms  such that , the VI distance is

and this is indeed the minimum--weight distance. On the other hand,

. In fact,  as .
\end{example}

\begin{example}
\textbf{Rank-based shortest path distance:} let  and consider partitions  and  (with vertical bar  separating blocks as in Table
2 above). Then,  as well as . Accordingly,


as  is the length of a shortest path between  and . Such a path visits  and for instance may be across edges

and finally  of  (or equivalently of Hasse diagram  introduced above). On the other hand, a shortest -path \textit{forced to visit}
 has length 7 and for instance may be across edges


and finally . Note that the rank assigns to every edge  of  unit weight , and thus  is indeed the shortest path distance.
\end{example}

\begin{table}[htbp]
\caption{\textsl{ for  symmetric, strictly order preserving/inverting, super/submodular.}}
\label{tab: minimum-weight distances}
\begin{center}
\begin{tabular}{|c|c|c|c|}
\hline
 symmetric &  strictly order-preserving &  strictly order-inverting\\
\hline
 supermodular&  & \\
\hline
 submodular &  & \\
\hline
\end{tabular}
\end{center}
\end{table}

\section{Distinctions, co-atoms and fields}
A further measure of partition entropy, called logical entropy, has been recently proposed \cite{EllermanLogicalEntropy} in terms of distinctions, i.e. \textit{ordered} pairs
 (see Section 2). In statistical classification, the same concept is also referred to as the ``Gini coefficient''
\cite[pp. 53-54, 247-250, 257, 334]{Mirkin2013}. If distinctions are replaced with unordered pairs , then \textit{mutatis mutandis} the non-normalized logical
entropy of partitions  is the analog of , providing a further minimum-weight partition distance. Furthermore, since in information theory partitions are
generally evaluated by means of order-inverting functions, the approach developed thus far may be applied to the upside-down Hasse diagram of the partition lattice, with co-atoms
(or dual atoms \cite{RotaMobius}) in place of atoms. In this way, the distance between partitions is the distance between the associated fields of subsets.

A partition  distinguishes between  and  if  while  with , and the set of such distinctions has been
recently proposed as the logical analog of the complement of , with the (normalized) number of distinctions providing a novel measure of the (logical) entropy of partitions
\cite{EllermanLogicalEntropy,EllermanLogic}. In particular, this is achieved through apartness binary relations , wich are the complement of equivalence relations
 (see again Section 2). In terms of atoms  of the partition lattice, the logical entropy  of
partitions \cite[p. 127]{EllermanLogicalEntropy} is

with  and .

\begin{proposition}
The logical entropy-based minimum-weight distance  is

\end{proposition}

\begin{proof}
Logical entropy  satisfies  and is strictly order-inverting. Also, apart from constant terms,  varies with , which is submodular because  is
supermodular. That is to say,

Thus  and the desired conclusion follows from Proposition 10.
\end{proof}

A field of subsets is a set system  closed under union, intersection and complementation, i.e.  for all
. Every partition  generates the field  containing all subsets  obtained as the union of blocks , with
 and . There are  minimal fields (generated by partitions) that strictly include
; they are those  with .
On the other hand,
2-cardinal partitions  are the co-atoms \cite{Aigner79} (or dual atoms \cite{RotaMobius}) of partition lattice  ordered
by coarsening. In fact, in information theory finer partitions are generally more valuable than coarser ones, and thus attention is placed on order-inverting partition functions.
In this view, the partition lattice is often dealt with as ordered by refinement and thus with the upside-down Hasse diagram. Accordingly,
a distance between partitions also obtains by counting co-atoms rather than atoms. To this end, define the co-size
 by , with  and . In words,  is
the number of co-atoms coarser than . 

\begin{proposition}
The minimum--weight partition distance is

\end{proposition}

\begin{proof}
Denote by  the M\"obius inversion from above \cite{Aigner79,RotaMobius} of the co-size, with
 for all . By definition,  if  and 0 otherwise. Like for the size in Proposition 1, this entails
supermodularity, i.e. . Furthermore,  is order-inverting. Therefore,

for all .
\end{proof}

Denote by  the lattice whose elements are the  fields of subsets  generated by partitions , ordered by
inclusion . The meet and join are, respectively,  and .
The set of atoms is the collection  of minimal fields; that is to say,
 for all . Therefore,  may also be regarded as an analog of
the traditional Hamming distance between subsets:

In words, this is the number of minimal fields  included in either  or else in , but not in both.

\section{Appendix: Euclidean distance between fuzzy partitions}
The leading idea of this section is to propose a measure of the distance between fuzzy partitions, like in \cite{Brouwer2009}. Together with theoretical worthiness, from an applicative
perspective this distance is useful for comparing alternative results of objective function-based fuzzy clustering algorithms (such as the fuzzy
C-means, see \cite{FuzzyCmeansBook2008,Valente+07} for a comprehensive treatment). More precisely, these algorithms usually rely on local search methods, and their output takes the
form of a membership matrix, where rows and columns are indexed by data and clusters, respectively. For a given data set, the chosen algorithm typically outputs different membership
matrices depending on alternative initial candidate solutions and/or parametrizations, and these varying outputs are commonly ranked through a validity index
(see \cite{WangZhang2007} for a recent overview). A key input is the desired number of clusters, which is not chosen autonomously through optimization, but is conversely maintained
fixed over the search. Conceiving several runs for each reasonable number of clusters, a common situation is thus one where alternative outputs score best on the chosen validity index.
Then, the proposed distance measure allows to compare these outputs, each with a different number of clusters and with highest validity score for that number.

Fuzzy clusterings are collections  of subsets of  endowed with  membership distributions , where 
quantifies the membership of  in . A fuzzy clustering thus is a -collection of fuzzy subsets  of 
\cite{FuzzySubset}, and  since \textit{every} non-empty subset  may have an associated fuzzy subset .
Membership matrices  satisfy  for all . The traditional Euclidean distance  between
 simply is , i.e. the  norm in . For measuring the
same distance  between  and  it must be , in which case
. Yet, as already observed, very likely there are fuzzy clusterings with high scores
in terms of the chosen validity index such that . In this view, the proposed method is dimension-free, i.e. regardless of whether  or .

When considering that the  singletons , 
are the atoms of Boolean lattice , a fuzzy subset is readily seen to consist, in fact, of  memberships  indexed by the  atoms. In
this view, from a combinatorial
perspective fuzzy elements of atomic lattices may be defined to be collections of [0,1]-memberships, one for each atom. Insofar as lattice theory is concerned, fuzzy partitions may thus be
regarded as points in the 0/1-polytope  introduced in Section 5, with variables 
indexed by atoms .

\begin{definition}
A fuzzy partition is any ,
and  is in fact non-fuzzy (or hard), while  is properly fuzzy.
\end{definition}

A fuzzy partition thus is a point in the polytope  included in the -cube, with axes indexed by atoms , and a
non-fuzzy partition  corresponds to a vertex of  identified by indicator function . 

Denote by  the set of all membership matrices. For ,
let  be the associated subsets, i.e.  is the membership of  in , while  for all . Thus, for instance, if , then
 for .

\begin{proposition}
Mapping  defined by

satisfies: (i) if  for all , then , and (ii) .
\end{proposition}

\begin{proof}
Firstly,  for all  entails that the summation yields a positive quantity never exceeding 1, that is .
Concerning (i), if , then  corresponds to a non-fuzzy partition , i.e.
.
The  columns  of \textbf x are thus given by the  characteristic functions
 of subsets  (see above), with  for some partition . Hence , in that

for all atoms .
Finally, coming to (ii), observe that  obtains as a suitable convex combination of vertices  of the polytope. That is to say,
 with  and .
These partitions  and coefficients  are determined through a
fairly simple recursive procedure. Starting from the top partition , with coefficient ,
let  be the atom corresponding to this minimum. Next, atom  corresponds to minimum  and
 is a coarsest partition satisfying , while coefficient  obtains
incrementally. At the generic -th step, atom  corresponds to minimum , while the selected partition
 is a coarsest one satisfying  and the coefficient  is given by
. These steps continue through partitions that are either finer or else incomparable with respect to the previous
ones, while reaching the atoms themselves and, if necessary, the bottom  too.
\end{proof}

\begin{example}
For , consider the collections

 and

 of subsets, with membership matrices  and  given by:

, ,  and

, ,  and

, ,  and

, ,  for the former, while

, ,  and

, , ,  and

, , ,  and

, ,  for the latter. Let  for notational convenience. Expression (10) yields:

,

,

,

,

,

 for the former collection, and

,

,

,

,

,

 for the latter. Concerning the convex combinations corresponding to 
and , for the former collection , since , firstly  and
, thus partitions  coming next satisfy .
As  for , a coarsest  is . Hence

and therefore  for all subsequest partitions .
The new minimum is , and the above constraints yield   as the coarsest available partition, with
. After updating,  is the novel minimum, with  and
.
The last partitions  are atoms themselves, namely  and , with associated coefficients
 as well as
.
Since the sum of these six coefficients yields , the bottom partition finally has coefficient
.
Thus the sought convex combination of indicator functions or vertices  is 

A generic point in polytope  generally admits alternative (equivalent)
convex combinations of vertices. For instance,  also admits

Coming to the second collection  of subsets, the first coefficient is 
since . Next, rather straightforwardly,

,

,

,

,

. These six coefficients add up to , hence the bottom partition has coefficient
. A sought convex combination thus is

\end{example}

The Euclidean distances between fuzzy partitions  given by the  and  norms, denoted by  and  respectively,
are the usual distances between points in a Euclidean vector space (i.e. ), namely

where  is the absolute value. Both are well-known metrics (see above). In particular, triangle inequality
may be considered in conjunction with the order relation and the meet of fuzzy partitions.

\subsection{Order, meet and join}
The order relation , the meet  and the join  for partitions  may be extended from vertices  of polytope 
to the whole of this latter. Specifically, . In the same way,
for any two fuzzy partitions ,

For the discrete setting provided by vertices of the polytope, the following condition has been already considered in terms of ``vertical collinearity'' \cite{Meila2007} or
``interval betweenness'' (for valuations of distributive lattices) \cite{Monjardet1981}.

\begin{proposition}
For any fuzzy partitions , if , then  and  satisfy triangle inequality with equality, that is

\end{proposition}

\begin{proof}
If , then for all atoms 

and of course

hence .
\end{proof}

The same does not hold for , which conversely satisfies triangle inequality with equality if and only if  lies on the line segment between  and .

Turning attention to the meet  of fuzzy partitions , firstly consider that for the characteristic functions 
of subsets the meet or intersection is given by  for all , i.e. by the product. Analogously, the meet of partitions
, with indicator functions , is given by

as . Then, the meet  of fuzzy partitions  also obtains through the product:
.

\begin{proposition}
For all ,

\end{proposition}

\begin{proof}
Firstly note that  entails .

Now, 

 as wanted.
\end{proof}

A similar expression may be provided for the squared Euclidean distance or  norm .

\begin{proposition}


\end{proposition}

\begin{proof}
By direct substitution: 

 as wanted.
\end{proof}

The join  of two (fuzzy) partitions leads to a more complex setting, because it brings about the closure yielding the partition lattice as the
polygon matroid defined on the edges of the complete graph  (see Sections 1, 2). As already observed, for , the meet
 is coarser than all and only those atoms
 finer than both  and . Thus, the meet of partitions is basically the analog of the intersection of subsets , and indeed
in the same way obtains through the pair-wise product of indicator functions  (see above). Conversely, when regarded as a pair-wise operation between indicator
functions , the join is very different from the union of subsets. In particular, recall that for , with characteristic functions
, the union  obtains as follows: .
In words, Boolean vector  has entry 1 where  and/or  have entry 1. The same does not apply to partitions ,
as indicator function  may have entry 1 even where both  and  have entry 0. As before,
this can be observed already in the simple case where . To this end, arrange the entries of  by
, hence  and
. Then,
, but ,
i.e. .

\begin{definition}
In terms of indicator functions , for all atoms 
the join  of partitions  is

\end{definition}

In the same way, the join  of fuzzy partitions  is given by
.


\section{Appendix: the consensus partition problem}
Hamming distance between partitions HD was considered for the first time in the mid '60s \cite{Renier1965} in terms of the \textit{consensus (or central) partition} problem, which
is important in many applicative scenarios concerned with statistical classification. From a combinatorial optimization perspective, the problem has generic instance consisting of
a -collection , , and is characterized by firstly selecting a measure of the distance between any two partitions, i.e. a metric
. Given this, the objective is to find a partition  minimizing the sum of its distances from the 
partitions. That is to say, any  satisfying  for all  is a consensus
partition. For generic , finding a solution  is tipically hard. In particular, if , then each distance  for any
 is computable in  time \cite[p. 236]{KorteVygen2002}, whereas if , then in view of expression (6) above (see Section 3) each
distance  is computable more rapidly through scalar products. In any case, independently from the chosen metric , the main issue is that the size
 of the search space  makes all approaches relying on direct enumeration simply unviable, at least for relevant values of . The
problem is thus commonly interpreted in terms of heuristics \cite{CeleuxEtAl1989,CentralPartition}, and if  is large and/or  are very far from each other, then
figuring out where to concentrate the search is the fundametal issue.

Although the consensus problem is generally harsh, especially in terms of the required exploration of , still the analysis conducted thus far identifies conditions
where exact solutions are easy to find. In fact, if the chosen metric is a minimum--weight partition distance, i.e.  with , and weighting
function  is either supermodular or else submodular (but not both, see below), then either the meet  or else the join
 of instance elements are consensus partitions. Specifically, the former case applies to Hamming distance or size-based  and to logical
entropy-based , while the latter applies to rank-based  and to co-size-based . Hence, the computational burden reduces solely to assessing the
 distances between instance elements and their meet (or else their join), with no search need.

\begin{proposition}
If distances between partitions are measured by HD, then the meet of all instance elements achieves consensus, i.e.

for all  and all instances .
\end{proposition}

\begin{proof}
Firstly note that for  this consensus condition is in fact a restatement of horizontal collinearity and triangle inequality (see Propositions 1 and 2). Hence,
in order to use induction, assume that the condition holds for some , and denote by  the solution or consensus partition of a -instance
. By assumption,  is a solution of instance , thus novel solution  minimizes the sum of its
distances from the previous solution  and from the novel instance element , i.e.

for all . Then, horizontal collinearity and triangle inequality entail

with equality if .
\end{proof}  

Concerning the value taken by the sum  of distances between instance elements and the consensus partition,
observe that for all  and all 

By triangle inequality,

with equality if  for all , which is not possible unless . Now consider partition function
 defined by

where  denotes the given instance. Function  attains its minimum at consensus partition
, where

as  for all .

Exactly the same argument applies to logical entropy-based , entailing that
 for all  and all instances .

For rank-based  and co-size-based  distances, horizontal collinearity holds in terms of the join (rather than in terms of the meet of any
, see above), meaning that  yields

with equality if . Thus the join (rather than the meet) of instance elements achieves consesus, i.e.
 for all  and all instances , while analog
results apply, \textit{mutatis mutandis}, to partition function .

The setting developed thus far also enables to frame the consensus partition problem in a novel manner, which in turn widens the spectrum of conceivable fuzzy models
for partitions. In order to briefly outline such new possibilities, firstly recall that a fuzzy subset of  is a function  or,
from an equivalent geometric perspective, a point  in the -dimensional unit hypercube, where , . Accordingly, a fuzzy
partition is commonly intended as a partition  with associated  points  in the hypercube such that 
for all  and all . On the other hand, a fuzzy graph with vertex set  may be seen as one whose edge set is a fuzzy subset of ,
i.e. a function  or, from an equivalent geometric perspective, a point in the -dimensional unit hypercube, i.e.
.

By looking at partitions of  as graphs with vertex set  each of whose components is complete, fuzzy partitions can be regarded as fuzzy graphs with complete components.
Along this route, the fuzzy consensus partition  associated with instance  may be defined to be the point in the interior of
the polytope  of partitions (see above) corresponding to the center of the convex hull  given by all convex combinations of the
indicator functions  of instance elements. In this way, the fuzzy consensus partition is a function ranging in the unit interval  and taking values
on the atoms of , i.e. . In particular,


In this framework, the \textit{strong patterns} of instance  considered in \cite{CentralPartition} are the blocks of partition  obtained through
defuzzification of  as follows:

In words,  obtains as the join of all atoms where the fuzzy consensus partition attains its maximum, i.e. 1.

\section{Conclusion}
This work considers distances between partitions
by focusing on lattice theory and relying on discrete methods. Specifically, it firstly develops from the idea of reproducing the traditional Hamming distance between subsets by
counting unordered pairs of partitioned elements or atoms of the partition lattice. Although counting ordered and/or unordered pairs is not new (see \cite[Section 2.1]{Meila2007}
for a survey), still the Hamming distance between partitions HD is here analyzed from a novel geometric perspective. Special attention is placed on the distance between complements
in comparison with two alternative partition distance measures proposed in recent years, namely MMD and VI. Given its low computational complexity combined with fine measurement
sensitivity, HD may be considered as an alternative to MMD and VI for applications.

Like the cardinality of the symmetric difference between subsets is a count of atoms of a Boolean lattice, in the same way HD relies on the size, which counts the atoms
finer than partitions, but while the cardinality or rank of subsets is a valuation, i.e. both supermodular and submodular, the size of partitions is supermodular, in that valuations
of the partition lattice are constant partition functions \cite{Aigner79}. Also, in view of expression  for the Hamming distance between subsets
, it may seem reasonable to consider distances between partitions  of the form  for some symmetric and order preserving/inverting ,
i.e. . However, such a distance takes the same value  whenever  and  are complementary partitions (see Section 4), and this
should be avoided in view of \cite[Theorem 1]{Stanley1971}.

The geometric approach adopted here enables to analyze further partition distances obtained by replacing the size with alternative partition functions such as entropy, rank,
logical entropy and co-size. In general, any symmetric and order-preserving/inverting partition function  provides a distance between partitions  by considering the four
values  and . Specifically,  defines weights on edges of the Hasse diagram (or 0/1-polytope) of partitions such that the so-called minimum--weight
distance between any  is the weight of a lightest -path. Depending on whether  is supermodular or else submodular and order-preserving or else order-inverting, a
minimum--weight path between  and  visits their meet or else their join, and viceversa. These four possibilities are summarized in Table 3, Section 5. In particular, HD is the
minimum--weight distance , where partition function  is the size, while VI is the minimum--weight distance , where partition function  is the entropy.

Any distance is of course normalized when considered as the ratio to its maximum value . On the other hand, it may be relevant to consider such a maximum as a function
 of the number  of partitioned elements, with focus on the first-order difference  and on the second-order one
. For HD both differences are strictly positive:  and , and
these are exactly the same values  as for the traditional Hamming distance  between subsets . For entropy-based
distance VI, the former  is positive while the latter  is negative by concavity of
the  function. For maximum matching distance  while , and the same applies to rank-based minimum-weight distance
 outlined in Example 2, Section 5. For logical entropy-based minimum-weight distance  and
.

By extending attention from edges and vertices of the 0/1-polytope of partitions to the whole of this latter, the general aproach based on atoms also applies to the fuzzufication
of partitions. In particular, fuzzy clusterings or membership matrices of any dimension are turned into fuzzy subsets of atoms of the partition lattice, and thus distances between
such matrices may be computed through common Euclidean norm in . 


\bibliographystyle{abbrv}
\bibliography{HammingDistanceLatticeMetrics}


\end{document}
