\documentclass[11pt]{article}

\usepackage{color}               
\usepackage{a4wide}
\usepackage{graphicx}
\usepackage{amssymb, amsthm}
\usepackage{epstopdf}
\usepackage{amsmath}
\usepackage{hyperref}

\newtheorem{prop}{Proposition}
\newtheorem{cor}{Corollary}
\newtheorem{theorem}{Theorem}
\newtheorem{lemma}{Lemma}
\newtheorem{claim}{Claim}

\newcommand{\ds}{\displaystyle}
\newcommand{\OPT}{\textrm{OPT}}

\title{Minimum Entropy Orientations}

\author{
Jean Cardinal\thanks{Universit\'e Libre de Bruxelles, D\'epartement d'Informatique, c.p.~212, B-1050 Brussels, Belgium, jcardin@ulb.ac.be.}
 \and 
Samuel Fiorini\thanks{Universit\'e Libre de Bruxelles, D\'epartement de Math\'ematique, c.p. 216,  B-1050 Brussels, Belgium,  sfiorini@ulb.ac.be.}
\and 
Gwena\"el Joret\thanks{Universit\'e Libre de Bruxelles, D\'epartement d'Informatique, c.p.~212,  B-1050 Brussels, Belgium, gjoret@ulb.ac.be. G. Joret is a Research Fellow of the Fonds 
National de la Recherche Scientifique (F.R.S.--FNRS).}
}

\date{}

\begin{document}
\sloppy

\maketitle

\begin{abstract}
We study graph orientations that minimize the entropy of the in-degree sequence. We prove that the minimum entropy orientation problem is NP-hard even if the graph is planar, and that there exists a simple linear-time algorithm that returns an approximate solution with an additive error guarantee of 1 bit. 
\end{abstract}

{\noindent {\bf Keywords:}
Approximation algorithm; Entropy; Graph orientation; Vertex cover 
}

\section{Introduction}

All graphs considered here are finite, undirected and loopless, but multiple edges are allowed. Let  be a graph with  vertices and  edges, and consider any directed graph  obtained by orienting the edges of . The {\em in-degree distribution}  of this orientation is defined by , where  denotes the in-degree of  in .

In this paper, we consider the problem of finding an orientation whose corresponding in-degree distribution is as unbalanced as possible. As a balance measure, we use the {\em entropy}

where  denotes the base  logarithm, and . The {\em minimum entropy orientation} problem (MINEO) is the problem of finding an orientation of  with an in-degree distribution  minimizing .

The study of MINEO is motivated by that of the {\em minimum entropy set cover} problem (MINESC), introduced by Halperin and Karp~\cite{HK05}. In the latter problem, we are given a ground set  and a collection  of subsets of  whose union is , and we have to assign each element of  to a subset  containing it. This assignment partitions  into classes  of elements assigned to the same subset, and the objective is to minimize the entropy of the probability distribution  
defined by this partition. Hence, MINEO is the special case of MINESC where each element of the ground set can be covered by exactly two sets from .
(To see this, take  and let  with .)

The main motivation of Halperin and Karp for introducing MINESC was to solve a haplotyping problem, of importance in computational biology. This problem involves covering a set  of {\em partial haplotypes} of length , defined as words in the set , by {\em complete haplotypes}, defined as words in . Each subset of  corresponds to a complete haplotype  and contains all partial haplotypes that are {\em compatible} with , that is, partial haplotypes whose symbols match in every non-`' positions with . The `' positions in the partial haplotypes are interpreted as measurements error. It is shown that, under some probabilistic assumptions, minimizing the entropy of the covering amounts to maximizing its likelihood~\cite{HK05}. 

An application of MINEO is the special case of partial haplotyping
in which each partial haplotype has at most one `' in it. Partial haplotypes are then edges of a -dimensional hypercube, and the subsets  correspond to vertices of this hypercube. In other words, this special case of the partial haplotyping problem is a minimum entropy orientation problem in a partial hypercube.

The well-known greedy algorithm for the set cover problem is applicable to MINEO. It involves iterating the following steps: choose a maximum degree vertex  in , orient all edges incident to  toward , and remove  from . The performance of the greedy algorithm for MINESC has been studied thoroughly. Halperin and Karp~\cite{HK05} first showed that the greedy algorithm approximates MINESC to within some additive constant.  Then the current authors improved their analysis and showed that the greedy algorithm returns a solution whose entropy is at most the optimum plus  bits, with  bits~\cite{CFJ08}. Moreover, they proved that it is NP-hard to approximate MINESC to within an additive error of , for every constant . Since MINEO is a special case of MINESC, the first result implies that the greedy algorithm also approximates MINEO to within an additive error of  bits. We note that there exist instances of MINEO where the latter bound is (asymptotically) attained; see for example those described in Figure~\ref{fig:tight}.

\begin{figure}
\begin{center}
\includegraphics[scale=2]{tightgreedyVC.pdf}
\end{center}
\caption{\label{fig:tight}Construction of tight examples for the greedy  algorithm. Given a value  (in our illustration, ), let  be a set of  independent vertices (the red vertices). Then for each  in , construct  independent vertices of degree  with neighbors in , such that their neighborhoods partition  into subsets of size . The optimal solution is obtained by orienting each edge toward its endpoint in . The greedy algorithm may orient each edge in the opposite direction. This example is equivalent to the tight example previously given for the minimum entropy set cover problem~\cite{CFJ08}.}
\end{figure}

In this paper, we first prove that MINEO is NP-hard, even if the input
graph is planar (Section~\ref{sec-hardness}). The reduction is from a restricted version of the -in- Satisfiability problem. Then, we show in Section~\ref{sec-approx} that there exists a simple linear-time approximation algorithm for MINEO with an improved approximation guarantee of  bit. 

To conclude the introduction, we mention that MINEO is also related to
the {\em minimum sum vertex cover} problem (MINSVC), introduced by Feige, Lov{\'a}sz, and Tetali~\cite{FLT04}. In that problem, we are given a graph , and have to find an ordering  of its vertices such that the average cover time of an edge is minimized, that is, minimizing

where  denotes the set of edges that are incident to  but to no vertex  with . This problem can also be seen as a graph orientation problem in which the most unbalanced orientation is sought, although with a different balance measure than that of MINEO. Feige {\em et al.}~\cite{FLT04} proved that MINSVC is APX-hard, and gave a -approximation algorithm, based on randomized rounding of a natural linear programming relaxation. Using a different rounding technique, Berenholz, Feige, and Peleg~\cite{BFP06} recently derived an improved approximation factor of .

Although MINEO and MINSVC share some common properties, the main difference between these two problems is perhaps how they behave with respect to instances that are the union of smaller ones: As noted by Berenholz {\em et al.}~\cite{BFP06}, MINSVC is not ``linear'', in the sense that an optimal solution to the union of two disjoint graphs  and  is not necessarily a combination of an optimal solution to  and , respectively. (In particular, the APX-hardness proof of Feige {\em et al.}~\cite{FLT04} relies on this non-linearity.) On the other hand, MINEO {\em is} linear, as can be easily checked.

\section{Hardness}
\label{sec-hardness}

Let  and  be two sequences of non-negative integers sorted in non-increasing order, and such that . We say that sequence  {\em dominates} sequence  if 

for every , and moreover~(\ref{eq-dominance}) holds with strict inequality for at least one such . We emphasize that  whenever  dominates . The following lemma is a standard consequence of the strict concavity of the function ; see e.g.~\cite{MEC-ISAAC05, FHM-soda, HLP88} for different proofs. 

\begin{lemma}
\label{lem:jungle}
If  dominates , then . 
\end{lemma}

\begin{theorem}
\label{th-NPhard}
Finding a minimum entropy orientation of a planar graph is NP-hard.
\end{theorem}
\begin{proof}
In the -in- Satisfiability problem, we are given a 3-SAT formula in input, and we have to decide whether there exists a truth assignment of the variables such that each clause is satisfied by {\em exactly} one of its three literals. Moore and Robson~\cite{MR01} proved that this problem is NP-complete, even if every variable appears in exactly three clauses, there is no negation in the formula, 
and the bipartite graph obtained by linking a variable and a clause if and only if the variable appears in the clause, is planar.

We will reduce the latter restriction of the -in- Satisfiability problem to MINEO. It will be convenient for the proof to restate Moore and Robson's result in the context of the Exact Cover problem. The latter asks, given a set system , to decide if  can be covered using pairwise disjoint sets from . As noted by Li and Toulouse~\cite{LT06}, the NP-completeness of the version of -in- Satisfiability described above directly implies that Exact Cover is NP-complete even when every set in  has cardinality exactly 3, each element in  is included in exactly three sets of , and the ``elements versus sets'' incidence graph is planar. (To see this, consider the set system  where  and  are associated with the clauses and the variables of the -in- Satisfiability instance, respectively.) Let  be any such set system, and denote by  and  the elements of  and the sets in , respectively.
We may assume without loss of generality that  is a multiple of 3, since otherwise
  has no exact cover.

We construct a graph  as follows: First, create a vertex  per set . Then, for each element , add a copy of the gadget depicted in Figure~\ref{fig-point-gadget}, and link  () to , where  are the indices of the three sets containing . 
\begin{figure}
\centering
\includegraphics[width=4.5cm]{fig-point-gadget.pdf}
\caption{\label{fig-point-gadget}Gadget for element .}
\end{figure}
The fact that  is planar directly follows from the planarity of the bipartite graph underlying the set system . 

Let  be any orientation of  with minimum entropy, and denote by  its arc set. For a subset  of vertices,
we use  for the number of arcs going from  to  in . We define  the {\em in-degree sequence} of , denoted , as the sequence of in-degrees of the vertices in , sorted in non-increasing order. Let also , for every .

\begin{claim}
\label{claim-gadget}
The in-degree sequence of the set  in  is given by the following table:

\end{claim}
\begin{proof}
Denote by  the in-degree sequence of the set  in . Arguing by contradiction, we suppose that this sequence is different from the one given in the table.  Considering the structure of the element-gadget, and in particular its two disjoint triangles, we infer the following bounds:

It follows from the inequalities above and  that  is dominated (in the sense of Lemma~\ref{lem:jungle}) by the corresponding sequence in the table. Now, it is always possible to re-orient the arcs of  with both endpoints in  in such a way that   realizes the latter sequence, as illustrated in Figure~\ref{fig-gadget-orientation}. 
\begin{figure}
\centering
\includegraphics[width=5cm]{fig-gadget-orientation.pdf}
\caption{\label{fig-gadget-orientation}A good orientation of the element-gadget. It is assumed that the arcs going out of  occur clockwise from the top of the figure.}
\end{figure}
Because this modification leaves the in-degrees of the vertices in  unchanged, we deduce from Lemma~\ref{lem:jungle} that the new orientation has an entropy strictly smaller than , a contradiction.
\end{proof}

Using Claim~\ref{claim-gadget}, we may assume without loss of generality that  is isomorphic to the orientation of the element-gadget given in Figure~\ref{fig-gadget-orientation}. 
Renaming the vertices if necessary, we may thus suppose that , ,  have in-degree , ,  in , respectively.

Consider now the three edges  of . Recall that  denote the indices of the three sets in  containing the element . Because  has minimum entropy, we may assume that the first of these three edges is oriented out of  and the last two toward . Indeed, if we have  then , , and changing the orientation of  would decrease the entropy of  by Lemma~\ref{lem:jungle}, a contradiction. Moreover, if , then re-orienting  either leaves the entropy of  unchanged or decreases it. A similar argument holds when .

It follows that there is exactly one arc going out of  in , for every . We proceed with a second (and last) claim.

\begin{claim}
\label{claim-exact-cover}
Let  (\,). Then the entropy of  is at least

with equality if and only if there exists an exact cover of .
\end{claim}
\begin{proof}
As we have seen, we have without loss of generality  for every . Then, each component of , where , is clearly at most 3, and the sum of all of them equals . 
Combining these observations with Lemma~\ref{lem:jungle}, we deduce that the entropy of  is at least the lower bound given in the claim, and that equality occurs if and only if  equals . We show that the latter happens if and only if there exists an exact cover of .

Suppose first  , and define  as

It is then easily seen that the collection  is an exact cover for the set system .

Now assume that  is an exact cover of . After permuting the indices ,  and , we can assume that the set of  containing  is . Orienting each edge  of  toward  if , toward  otherwise, and using the orientation of the element-gadgets given in Figure~\ref{fig-gadget-orientation}, we obtain an orientation  of  where each 
 has in-degree sequence  and  has in-degree sequence . Hence,  must also have the same in-degree sequence in , since otherwise  would have entropy strictly less than , contradicting the optimality of the latter orientation. The claim follows.
\end{proof}

By Claim~\ref{claim-exact-cover}, a polynomial-time algorithm finding
a minimum entropy orientation of  could be used to decide, in polynomial time, if there exists an exact cover of . This completes the proof of the theorem.
\end{proof}

Let us say that a graph orientation problem has the {\em strict dominance property} if the objective function  to minimize is such that 

whenever  and  are two orientations of a graph  such that the in-degree sequence of  in  dominates that of  in  . Hence, Lemma~\ref{lem:jungle} says exactly that MINEO has the strict dominance property. We remark that, since the proof of Theorem~\ref{th-NPhard} relies solely on that lemma, it follows more generally that every orientation problem with the strict dominance property is NP-hard on planar graphs.

\section{Approximation}
\label{sec-approx}

Throughout this section, we denote by  the minimum entropy
of an orientation of . An orientation of  is {\em biased} if each edge  with  is oriented toward . It turns out that biased orientations have entropy close to the minimum achievable:

\begin{theorem}
\label{th-approx}
The entropy of any biased orientation of  is at most .
\end{theorem}

Since finding a biased orientation can easily be done in linear time,
Theorem~\ref{th-approx} yields the following corollary:

\begin{cor}
\label{cor-approx}
MINEO can be approximated within an additive error of  bit, in linear time.
\end{cor}

Let  denote the number of edges of  and  the normalized degree of a vertex . Given two discrete probability distributions  and  over a common domain , 
we denote by  their relative entropy (or Kullback-Leibler distance), 
defined as follows: 

It is known that  is always non-negative (see for instance
Cover and Thomas~\cite[Section 2.6]{TC} for a proof). 

\begin{proof}[Proof of Theorem~\ref{th-approx}.]
Let  be an optimal orientation of . Denote respectively by  and  the in-degree distribution and arc set of . We first rewrite the entropy of  as follows:

Now we observe that for any vertex  we have . We let  be a biased orientation,  its arc set, and  the corresponding in-degree distribution. From our observation, we have
 
Since , we have , which concludes the proof.
\end{proof}

We note that the bound given in Theorem~\ref{th-approx} is tight: consider for instance the case where  is a cycle.\\

We end this section with some remarks. For , let  denote the fraction of edges of  incident to a vertex in . Thus . It is well-known that  is a submodular function, that is, satisfies  for all . We denote the base polytope of  by . 
Letting  for , 
we thus have

It follows from standard results on polymatroids that  is the convex hull of all the in-degree distributions of orientations of ;
see for instance Schrijver~\cite{S03-b}. The vertices of  correspond to the acyclic orientations of .

Consider now the following generic linear program, where for each ,  is a fixed non-negative cost:
2ex]
\textrm{s.t.} &p \in P(G).
\end{array}

\label{prob-MINEO-LP}
\begin{array}{rl@{\qquad}l}
\min          &\ds \sum_{v \in V} -p_v \cdot \log \frac{\deg(v)}{m} \
Since MINEO can be formulated as
2ex]
\textrm{s.t.} &\ds p \in P(G),
\end{array}

and  trivially holds for every ,
the optimum value of~\eqref{prob-MINEO-LP} gives a lower bound on .
Now, observe that performing the greedy algorithm to solve~\eqref{prob-MINEO-LP} 
amounts to finding a biased orientation of . Moreover, the in-degree sequence of every such orientation can be produced by the algorithm. 

To conclude, we mention that the natural counterpart of MINEO where one aims at finding an orientation of  with {\em maximum} entropy is polynomial. This is because maximizing a separable concave function over  
can be done in polynomial time; see~\cite{G91,  H94, MS04}.

\section*{Acknowledgments}

The authors wish to thank Olivier Roussel from the \'Ecole Normale Sup\'erieure de Cachan 
for his preliminary investigations on this topic. We also thank an anonymous referee for an observation that 
made the proof of Theorem~\ref{th-approx} more concise. 

This work was supported by 
the {\em Actions de Recherche Concert\'ees (ARC)\,} fund of the {\em Communaut\'e fran\c{c}aise de Belgique}.

\bibliographystyle{plain}
\bibliography{e-vc}

\end{document}  
