\documentclass{article} \usepackage{iclr2018_conference,times}
\usepackage{hyperref}
\usepackage{url}

\usepackage{amsmath}
\usepackage{subfigure}
\usepackage{graphicx}


\title{Residual Gated Graph ConvNets}



\author{Xavier Bresson\thanks{XB is supported by NRF Fellowship NRFF2017-10.}\\
School of Computer Science and Engineering\\
Nanyang Technological University, Singapore\\
\texttt{xbresson@ntu.edu.sg} \\
\And
Thomas Laurent \\
Department of Mathematics \\
Loyola Marymount University \\
\texttt{tlaurent@lmu.edu} 
}



\newcommand{\fix}{\marginpar{FIX}}
\newcommand{\new}{\marginpar{NEW}}

\iclrfinalcopy 




\begin{document}


\maketitle

\begin{abstract}
Graph-structured data such as social networks, functional brain networks, gene regulatory networks, communications networks have brought the interest in generalizing deep learning techniques to graph domains. In this paper, we are interested to design neural networks for graphs with variable length in order to solve learning problems such as vertex classification, graph classification, graph regression, and graph generative tasks. Most existing works have focused on recurrent neural networks (RNNs) to learn meaningful representations of graphs, and more recently new convolutional neural networks (ConvNets) have been introduced. In this work, we want to compare rigorously these two fundamental families of architectures to solve graph learning tasks. We review existing graph RNN and ConvNet architectures, and propose natural extension of LSTM and ConvNet to graphs with arbitrary size. Then, we design a set of analytically controlled experiments on two basic graph problems, i.e. subgraph matching and graph clustering, to test the different architectures.  Numerical results show that the proposed graph ConvNets are 3-17\% more accurate and 1.5-4x faster than graph RNNs. Graph ConvNets are also 36\% more accurate than variational (non-learning) techniques. Finally, the most effective graph ConvNet architecture uses gated edges and residuality. Residuality plays an essential role to learn multi-layer architectures as they provide a 10\% gain of performance.
\end{abstract}



\section{Introduction}

Convolutional neural networks of \cite{pro:LeCunBottouBengioHaffner98MNIST} and recurrent neural networks of \cite{art:HochreiterSchmidhuber97LSTM} are deep learning architectures that have been applied with great success to computer vision (CV) and natural language processing (NLP) tasks. Such models require the data domain to be regular, such as 2D or 3D Euclidean grids for CV and 1D line for NLP. Beyond CV and NLP, data does not usually lie on regular domains but on heterogeneous graph domains. Users on social networks, functional time series on brain structures, gene DNA on regulatory networks, IP packets on telecommunication networks are a a few examples to motivate the development of new neural network techniques that can be applied to graphs. One possible classification of these techniques is to consider neural network architectures with fixed length graphs and variable length graphs. 


In the case of graphs with fixed length, a family of convolutional neural networks has been developed on spectral graph theory by \cite{book:Chung97Spectral}. The early work of \cite{art:BrunaZarembaSzlamLeCun13DLgraphs} proposed to formulate graph convolutional operations in the spectral domain with the graph Laplacian, as an analogy of the Euclidean Fourier transform as proposed by \cite{art:HammondVandergheynstGribonval11GraphWav}. This work was extended by \cite{art:HenaffBrunaLeCun15DLgraphs} to smooth spectral filters for spatial localization.  \cite{art:DefferrardBressonVandergheynst16FastSpecCNN} used Chebyshev polynomials to achieve linear complexity for sparse graphs, \cite{art:LevieMontiBressonBronstein17CayleyCNN} applied Cayley polynomials to focus on narrow-band frequencies, and  \cite{art:MontiBronsteinBresson17Recom} dealt with multiple (fixed) graphs. Finally, \cite{art:KipfWelling17GCN} simplified the spectral convnets architecture using 1-hop filters to solve the semi-supervised clustering task. For related works, see also the works of \cite{Bronstein17reviewGDL}, \cite{art:BronsteinBressonSzlamBrunaLeCun17cvpr} and references therein.


For graphs with variable length, a generic formulation was proposed by \cite{art:GoriMonfardiniScarselli05GNN,art:ScarselliGoriTsoiHagenbuchnerMonfardini09} based on recurrent neural networks. The authors defined a multilayer perceptron of a vanilla RNN. This work was  extended by \cite{art:LiTarlowBrockschmidtZemel16GNN} using a GRU architecture and a hidden state that captures the average information in local neighborhoods of the graph. The work of \cite{art:SukhbaatarSzlamFergus16ComAgents} introduced a vanilla graph ConvNet and used this new architecture to solve learning communication tasks. \cite{marcheggiani2017encoding} introduced an edge gating mechanism in graph ConvNets for semantic role labeling. Finally, \cite{art:BrunaLi17ComDetect} designed a network to learn non-linear approximations of the power of graph Laplacian operators, and applied it to the unsupervised graph clustering problem. Other works for drugs design, computer graphics and vision  are presented by \cite{duvenaud2015convolutional,boscaini2016learning,monti2016geometric}.




In this work, we study the two fundamental classes of neural networks, RNNs and ConvNets, in the context of graphs with arbitrary length.  Section \ref{sec2} reviews the existing techniques. Section  \ref{sec_our_models} presents the new graph NN models.  Section \ref{sec4} reports the numerical experiments. 











\section{Neural networks for graphs with arbitrary length}
\label{sec2}


\subsection{Recurrent neural networks}

{\bf Generic formulation.} Consider a standard RNN for word prediction in natural language processing. Let  be the feature vector associated with word  in the sequence. In a regular vanilla RNN,   is computed with the feature vector  from the previous step and the current word , so we have:

The notion of neighborhood for regular RNNs is the previous step in the sequence. For graphs, the notion of neighborhood is given by the graph structure. If  stands for the feature vector of vertex , then the most generic version of a feature vector for a graph RNN is

where  refers to a data vector and  denotes the set of feature vectors of the neighboring vertices. Observe that the set  is unordered, meaning that  is intrinsic, i.e. invariant by vertex re-indexing (no vertex matching between graphs is required). Other properties of  are locality as only neighbors of vertex  are considered, weight sharing, and such vector is independent of the graph length. In summary, to define a feature vector in a graph RNN, one needs a mapping  that takes as input an unordered set of vectors , i.e. the feature vectors of all neighboring vertices, and a data vector , Figure \ref{fig0a}. 

We refer to the mapping  as the neighborhood transfer function in graph RNNs. In a regular RNN, each neighbor as a distinct position relatively to the current word (1 position left from the center). In a graph, if the edges are not weighted or annotated, neighbors are not distinguishable. The only vertex which is special is the center vertex around which the neighborhood is built. This explains the generic formulation of Eq. \eqref{generic1}.  This type of  formalism for deep learning for graphs with variable length is described by \cite{art:ScarselliGoriTsoiHagenbuchnerMonfardini09,art:GilmerSchoenholzRileyVinyalsDahl17Molecule,art:BronsteinBressonSzlamBrunaLeCun17cvpr} with slightly different terminology and notations. 






{\bf Graph Neural Networks of \cite{art:ScarselliGoriTsoiHagenbuchnerMonfardini09}.} The earliest work of graph RNNs for arbitrary graphs was introduced by \cite{art:GoriMonfardiniScarselli05GNN,art:ScarselliGoriTsoiHagenbuchnerMonfardini09}. The authors proposed to use a vanilla RNN with a multilayer perceptron to define the feature vector :

with

and  is the sigmoid function,  are the weight parameters to learn. 




Minimization of Eq. \eqref{eq_scarselli} does not hold a closed-form solution as the dependence computational graph of the model is not a directed acyclic graph (DAG). \cite{art:ScarselliGoriTsoiHagenbuchnerMonfardini09} proposed a fixed-point iterative scheme:  for 

The iterative scheme is guaranteed to converge as long as the mapping is contractive, which can be a strong assumption. Besides, a large number of iterations can be computational expensive. 







{\bf Gated Graph Neural Networks of \cite{art:LiTarlowBrockschmidtZemel16GNN}.} In this work, the authors use the gated recurrent units (GRU) of \cite{art:ChungGulcehreChoBengio14GRU}:

As the minimization of Eq. \eqref{eq_li} does not have an analytical solution,  \cite{art:LiTarlowBrockschmidtZemel16GNN} designed the following iterative scheme: 

and  is equal to

where  is the Hadamard point-wise multiplication operator. This model was used for NLP tasks by \cite{art:LiTarlowBrockschmidtZemel16GNN} and also in quantum chemistry by \cite{art:GilmerSchoenholzRileyVinyalsDahl17Molecule} for fast organic molecule properties estimation, for which standard techniques (DFT) require expensive computational time.

 

\begin{figure}[t]
\centering
\subfigure[Graph RNN]{\includegraphics[height=4.3cm]{pic/generic_feature_rnn.pdf}\label{fig0a}}
\hspace{1.5cm}
\subfigure[Graph ConvNet]{\includegraphics[height=5cm]{pic/generic_feature_cnn.pdf}\label{fig0b}}
\caption{Generic feature representation  of vertex  on a graph RNN (a) and a graph convNet (b).}
\label{fig0}
\end{figure}



{\bf Tree-Structured LSTM of \cite{art:TaiSocherManning15TreeLSTM}.} The authors extended the original LSTM model of \cite{art:HochreiterSchmidhuber97LSTM} to a tree-graph structure:

where   refers the set of children of node .  is equal to


Unlike the works of \cite{art:ScarselliGoriTsoiHagenbuchnerMonfardini09,art:LiTarlowBrockschmidtZemel16GNN}, Tree-LSTM does not require an iterative process to update its feature vector  as the tree structure is also a DAG as original LSTM. Consequently, the feature representation \eqref{eq_treelstm} can be updated with a recurrent formula. Nevertheless, a tree is a special case of graphs, and such recurrence formula cannot be directly applied to arbitrary graph structure. A key property of this model is the function  which acts as a gate on the edge from neighbor  to vertex . Given the task, the gate will close to let the information flow from neighbor  to vertex , or it will open to stop it. It seems to be an essential property for learning systems on graphs as some neighbors can be irrelevant. For example, for the community detection task, the graph neural network should learn which neighbors to communicate (same community) and which neighbors to ignore (different community). In different contexts, \cite{art:DauphinFanAuliGrangier17gatedCNN} added a gated mechanism inside the regular ConvNets in order to improve language modeling for translation tasks, and \cite{van2016conditional} considered a gated unit with the convolutional layers after activation, and used it for image generation.








\subsection{Convolutional neural networks}


{\bf Generic formulation.} Consider now a classical ConvNet for computer vision. Let   denote the feature vector at layer  associated with pixel . In a regular ConvNet,  is obtained by applying a non linear transformation to the feature vectors  for all pixels  in a neighborhood of pixel . For example, with  filters, we would have:

In the above, the notation   denote the concatenation of all feature vectors  belonging to the  neighborhood of vertex . In ConvNets, the notion of neighborhood is given by the euclidian distance. As previously noticed, for graphs, the notion of neighborhood is given by the graph structure. Thus, the most generic version of a feature vector  at vertex  for a graph ConvNet is

where  denotes the set of feature vectors of the neighboring vertices. In other words, to define a graph ConvNet, one needs a mapping  taking as input  a vector   (the feature vector of the center vertex) as well as an unordered set of vectors  (the feature vectors of all neighboring vertices), see Figure \ref{fig0b}. We also refer to the mapping  as the neighborhood transfer function.   In a regular ConvNet, each neighbor as a distinct position relatively to the center pixel (for example 1 pixel up and 1 pixel left from the center). As for graph RNNs, the only vertex which is special for graph ConvNets is the center vertex around which the neighborhood is built.





{\bf CommNets of \cite{art:SukhbaatarSzlamFergus16ComAgents}.} The authors introduced one of the simplest instantiations of a graph ConvNet with the following neighborhood transfer function:

where  denotes the layer level, and  is the rectified linear unit. We will refer to this architecture as the vanilla graph ConvNet. \cite{art:SukhbaatarSzlamFergus16ComAgents} used this graph neural network to learn the communication between multiple agents to solve multiple tasks like traffic control.



{\bf Syntactic Graph Convolutional Networks of \cite{marcheggiani2017encoding}.} The authors proposed the following transfer function:

where  act as edge gates, and are computed by:

These gated edges are very similar in spirit to the Tree-LSTM proposed in \cite{art:TaiSocherManning15TreeLSTM}. We believe this mechanism to be important for graphs, as they will be able to learn what edges are important for the graph learning task to be solved. 








\section{Models}
\label{sec_our_models}


{\bf Proposed Graph LSTM. } First, we propose to extend the Tree-LSTM of \cite{art:TaiSocherManning15TreeLSTM} to arbitrary graphs and multiple layers:





As there is no recurrent formula is the general case of graphs, we proceed as \cite{art:ScarselliGoriTsoiHagenbuchnerMonfardini09} and use an iterative process to solve Eq. \eqref{eq_graph_lstm}: At layer , for 


In other words, the vector  is computed by running the model  from  at layer . It produces the vector  which becomes   and also the input  for the next layer. The proposed Graph LSTM model differs from \cite{liang2016semantic,peng2017cross} mostly because the cell  in these previous models is not iterated over multiple times , which reduces the performance of Graph LSTM (see numerical experiments on Figure \ref{fig1e}).


{\bf Proposed Gated Graph ConvNets. } We leverage the vanilla graph ConvNet architecture of \cite{art:SukhbaatarSzlamFergus16ComAgents}, Eq.\eqref{eq_commnets}, and the edge gating mechanism of \cite{marcheggiani2017encoding}, Eq.\eqref{eq_gated_convnet1}, by considering the following model:

where , and the edge gates  are defined in Eq. \eqref{eq_gates}. This model is the most generic formulation of a graph ConvNet (because it uses both the feature vector  of the center vertex and the feature vectors  of neighboring vertices) with the edge gating property. 




{\bf Residual Gated Graph ConvNets. } In addition, we formulate a multi-layer gated graph ConvNet using residual networks (ResNets) introduced by \cite{art:HeZhangRenSun16ResNet}. This boils down to add the identity operator between successive convolutional layers:

As we will see, such multi-layer strategy work very well for graph neural networks.  








\section{Experiments}
\label{sec4}


\begin{figure}[h!]
\centering
\subfigure[Subgraph matching]{\includegraphics[height=3.5cm]{pic/subgraph_matching.pdf}\label{fig2a}}
\hspace{0.5cm}
\subfigure[Semi-supervised clustering]{\includegraphics[height=3.5cm]{pic/clustering2.pdf}\label{fig2b}}
\caption{Graph learning tasks.}
\label{fig2}
\end{figure}




\subsection{Subgraph matching}

We consider the subgraph matching problem presented by \cite{art:ScarselliGoriTsoiHagenbuchnerMonfardini09}, see Figure \ref{fig2a}. The goal is to find the vertices of a given subgraph  in larger graphs  with variable sizes. Identifying similar localized patterns in different graphs is one of the most basic tasks for graph neural networks. The subgraph  and larger graph  are generated with the stochastic block model (SBM), see for example \cite{art:Abbe17SBM}. A SBM is a random graph which assigns communities to each node as follows: any two vertices are connected with the probability  if they belong to the same community, or they are connected with the probability  if they belong to different communities. For all experiments, we generate a subgraph  of  nodes with a SBM , and the signal on  is generated with a uniform random distribution with a vocabulary of size , i.e. . Larger graphs  are composed of  communities with sizes randomly generated between 15 and 25. The SBM of each community is . The value of , which acts as the noise level, is , unless otherwise specified. Besides, the signal on  is also randomly generated between . Inputs of all neural networks are the graphs with variable size, and outputs are vertex classification vectors of input graphs. Finally, the output of neural networks are simple fully connected layers from the hidden states.





All reported results are averaged over 5 trails. We run 5 algorithms; Gated Graph Neural Networks of  \cite{art:LiTarlowBrockschmidtZemel16GNN}, CommNets of \cite{art:SukhbaatarSzlamFergus16ComAgents}, SyntacticNets of \cite{marcheggiani2017encoding}, and the proposed Graph LSTM and Gated ConvNets from Section \ref{sec_our_models}. We upgrade the existing models of \cite{art:LiTarlowBrockschmidtZemel16GNN,art:SukhbaatarSzlamFergus16ComAgents,marcheggiani2017encoding} with a multilayer version for \cite{art:LiTarlowBrockschmidtZemel16GNN} and using ResNets for all three architectures. We also use the batch normalization technique of \cite{art:IoffeSzegedy15BatchNorm} to speed up learning convergence for our algorithms, and also for \cite{art:LiTarlowBrockschmidtZemel16GNN,art:SukhbaatarSzlamFergus16ComAgents,marcheggiani2017encoding}. The learning schedule is as follows: the maximum number of iterations, or equivalently the number of randomly generated graphs with the attached subgraph is 5,000 and the learning rate is decreased by a factor  if the loss averaged over 100 iterations does not decrease. The loss is the cross-entropy with 2 classes (the subgraph  class and  the class of the larger graph ) respectively weighted by their  sizes. The accuracy is the average of the diagonal of the normalized confusion matrix w.r.t. the cluster sizes (the confusion matrix measures the number of nodes correctly and badly classified for each class). We also report the time for a batch of 100 generated graphs. The choice of the architectures will be given for each experiment. All algorithms are optimized as follow. We fix a budget of parameters of  and a number of layers . The number of hidden neurons  for each layer is automatically computed. Then we manually select the optimizer and learning rate for each architecture that best minimize the loss. For this task, \cite{art:LiTarlowBrockschmidtZemel16GNN,art:SukhbaatarSzlamFergus16ComAgents,marcheggiani2017encoding} and our gated ConvNets work well with Adam and learning rate . Graph LSTM uses SGD with learning rate . Besides, the value of inner iterative steps  for graph LSTM and \cite{art:LiTarlowBrockschmidtZemel16GNN} is . 












The first experiment focuses on shallow graph neural networks, i.e. with a single layer . We also vary the level of noise, that is the probability  in the SBM that connects two vertices in two different communities (the higher  the more mixed are the communities). The hyper-parameters are selected as follows. Besides , the budget is  and the number of hidden neurons  is automatically computed for each architecture to satisfy the budget. First row of Figure \ref{fig_matching} reports the accuracy and time for the five algorithms and for different levels of noise . RNN architectures are plotted in dashed lines and ConvNet architectures in solid lines. For shallow networks, all RNN architectures (graph LSTM and  \cite{art:LiTarlowBrockschmidtZemel16GNN}) performs much better, but they also take more time than the graph ConvNets architectures we propose, as well as  \cite{art:SukhbaatarSzlamFergus16ComAgents,marcheggiani2017encoding}. As expected, all algorithms performances decrease when the noise increases.



The second experiment demonstrates the importance of having multiple layers compared to shallow networks. We vary the number of layers  and we fix the number of hidden neurons to . Notice that the budget is not the same for all architectures. Second row of Figure \ref{fig_matching} reports the accuracy and time w.r.t.  (middle figure is a zoom in the left figure). All models clearly benefit with more layers, but RNN-based architectures see their performances decrease for a large number of layers. The ConvNet architectures benefit from large  values, with the proposed graph ConvNet performing slightly better than \cite{art:SukhbaatarSzlamFergus16ComAgents,marcheggiani2017encoding}. Besides, all ConvNet models are faster than RNN models.





In the third experiment, we evaluate the algorithms for different budgets of parameters . For this experiment, we fix the number of layers  and the number of neurons  is automatically computed given the budget .  The results are reported in the third row of Figure \ref{fig_matching}. For this task, the proposed graph ConvNet best performs for a large budget, while being faster than RNNs.




\begin{figure}[h!]
\centering
\hspace{-0.53cm}
\includegraphics[height=3.2cm]{pic/fig00b.pdf}
\hspace{-0.53cm}
\includegraphics[height=3.2cm]{pic/fig00c.pdf}\\
\includegraphics[height=3.2cm]{pic/fig02b.pdf}
\hspace{-0.53cm}
\includegraphics[height=3.2cm]{pic/fig02d.pdf}
\hspace{-0.53cm}
\includegraphics[height=3.2cm]{pic/fig02c.pdf}\\
\includegraphics[height=3.2cm]{pic/fig05b.pdf}
\hspace{-0.53cm}
\includegraphics[height=3.2cm]{pic/fig05d.pdf}
\hspace{-0.53cm}
\includegraphics[height=3.2cm]{pic/fig05c.pdf}
\caption{Subgraph matching: First row studies shallow networks w.r.t. noise. Second row investigates multilayer graph networks. Third row reports graph architectures w.r.t. budget.}
\label{fig_matching}
\end{figure}








We also show the influence of hyper-parameter  for \cite{art:LiTarlowBrockschmidtZemel16GNN} and the proposed graph LSTM. We fix ,  and . Figure \ref{fig1e} reports the results for . The  value has an undesirable impact on the performance of graph LSTM. Multi-layer \cite{art:LiTarlowBrockschmidtZemel16GNN} is not really influenced by . Finally, the computational time naturally increases with larger  values. 
\begin{figure}[h!]
\centering
\includegraphics[height=3.2cm]{pic/fig04b.pdf}
\hspace{-0.53cm}
\includegraphics[height=3.2cm]{pic/fig14b.pdf}
\hspace{-0.53cm}
\includegraphics[height=3.2cm]{pic/fig14c.pdf}
\caption{Influence of  hyper-parameter  on RNN architectures. Left figure is for graph matching, middle figure for semi-supervised clustering, and right figure are the batch time for the clustering task (same trend for matching).}
\label{fig1e}
\end{figure}








\subsection{Semi-supervised clustering}


In this section, we consider the semi-supervised clustering problem, see Figure  \ref{fig2b}. This is also a standard task in network science. For this work, it consists in finding  communities on a graph given  single label for each community. This problem is more discriminative w.r.t. to the architectures than the previous single pattern matching problem where there were only  clusters to find (i.e. 50\% random chance). For clustering, we have  clusters (around 10\% random chance). As in the previous section, we use SBM to generate graphs of communities with variable length. The size for each community is randomly generated between  and , and the label is randomly selected in each community. Probability  is 0.5, and  depends on the experiment. For this task, \cite{art:LiTarlowBrockschmidtZemel16GNN,art:SukhbaatarSzlamFergus16ComAgents,marcheggiani2017encoding} and the proposed gated ConvNets work well with Adam and learning rate . Graph LSTM uses SGD with learning rate . The value of  for graph LSTM and \cite{art:LiTarlowBrockschmidtZemel16GNN} is .

The same set of experiments as in the previous task are reported in Figure \ref{fig_clustering}. ConvNet architectures get clearly better than RNNs when the number of layers increase (middle row), with the proposed Gated ConvNet outperforming the other architectures. For a fixed number of layers , our graph ConvNets and \cite{marcheggiani2017encoding} best perform for all budgets, while paying a reasonable computational cost.
\begin{figure}[h!]
\centering
\hspace{-0.53cm}
\includegraphics[height=3.2cm]{pic/fig10b.pdf}
\hspace{-0.53cm}
\includegraphics[height=3.2cm]{pic/fig10c.pdf}\\
\includegraphics[height=3.2cm]{pic/fig12b.pdf}
\hspace{-0.53cm}
\includegraphics[height=3.2cm]{pic/fig12d.pdf}
\hspace{-0.53cm}
\includegraphics[height=3.2cm]{pic/fig12c.pdf}\\
\includegraphics[height=3.2cm]{pic/fig15b.pdf}
\hspace{-0.53cm}
\includegraphics[height=3.2cm]{pic/fig15d.pdf}
\hspace{-0.53cm}
\includegraphics[height=3.2cm]{pic/fig15c.pdf}
\caption{Semi-supervised clustering: First row reports shallow networks w.r.t. noise . Second row shows multilayer graph networks w.r.t. . Third row is about graph architectures w.r.t. budget .}
\label{fig_clustering}
\end{figure}








Next, we report the learning speed of the models. We fix ,  with  being automatically computed to satisfy the budget. Figure \ref{fig_learning_speed} reports the accuracy w.r.t. time. The ConvNet architectures converge faster than RNNs, in particular for the semi-supervised task. 
\begin{figure}[h!]
\centering
\includegraphics[height=3.3cm]{pic/fig06b.pdf}
\hspace{-0cm}
\includegraphics[height=3.3cm]{pic/fig16b.pdf}
\caption{Learning speed of RNN and ConvNet architectures. Left figure is for graph matching and right figure semi-supervised clustering.}
\label{fig_learning_speed}
\end{figure}



To close this study, we are interested in comparing learning based approaches to non-learning variational ones. To this aim, we solve the variational Dirichlet problem with labeled and unlabelled data as proposed by \cite{art:Grady06RandomWalk}. We run 100 experiments and report an average accuracy of 45.37\% using the  same setting as the learning techniques (one label per class). The performance of the best learning model is 82\%. Learning techniques produce better performances with a different paradigm as they use training data with ground truth, while variational techniques do not use such information. The downside is the need to see 2000 training graphs to get to 82\%. However, when the training is done, the test complexity of these learning techniques is , where  is the number of edges in the graph. This is an advantage over the variational Dirichlet model that solves a sparse linear system of equations with complexity , see \cite{art:LiptonRoseTarjan79spalinsys}. 





\section{Conclusion}

This work explores the choice of graph neural network architectures for solving learning tasks with graphs of variable length. We developed analytically controlled experiments for two fundamental graph learning problems, that are subgraph matching and graph clustering. Numerical experiments showed that graph ConvNets had a monotonous increase of accuracy when the network gets deeper, unlike graph RNNs for which performance decreases for a large number of layers. This led us to consider the most generic formulation of gated graph ConvNets, Eq. \eqref{eq_gated_convnet2}. We also explored the benefit of residuality for graphs, Eq. \eqref{eq_resnet}. Without residuality, existing graph neural networks are not able to stack more than a few layers. This makes this property essential for graph neural networks, which receive a 10\% boost of accuracy when more than 6 layers were stacked. Future work will focus on solving domain-specific problems in chemistry, physics, and neuroscience. 




\bibliography{iclr2018_conference}
\bibliographystyle{iclr2018_conference}




















\end{document}