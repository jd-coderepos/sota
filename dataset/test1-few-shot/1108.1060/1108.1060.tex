

A {\em directed graph}  is the pair  where  is a finite non-empty set of vertices and  is a binary
relation.
The elements of  are called {\em arcs}. An arc  is oriented from  to . 
We use the term {\em graph} to refer to a {\em directed graph}\footnote{An {\em undirected graph} is a graph whose arc set  is symmetrical, i.e.  iff .}.
For the sake of clarity we may not be fully formal in the following definitions. More details and formal definitions can be found in \cite{DBLP:conf/wea/Lopez-PresaA09,jllopez2009}.

An \emph{isomorphism} of graphs  and  is a bijection between the vertex sets
of then, , such that  iff .
Graphs  and  are {\em isomorphic}, written , if there is some
\emph{isomorphism} of them. An {\em automorphism} of  is an isomorphism of  and itself.

Given a graph , we assume  to be given as an {\em adjacency matrix}
 with size  as follows.

Given a vertex  and a subset , the \emph{available degree} of  with  is the 3-tuple  counting the number of vertices
in  connected to  with the 3 types of adjacencies 3, 2, and 1, respectively. If all the elements of the tuple are zero then,  is disconnected from . The available degree
is used in our algorithms to order the vertices of .





\paragraph{Sequences of partitions}

The conauto \cite{DBLP:conf/wea/Lopez-PresaA09,jllopez2009} algorithms all use the same basic approach to test for isomorphism between graphs  and : they build a sequence of partitions for one of the graphs and try to find another sequence of partitions for the other graph with the same underlying structure (compatible). A sequence of partitions defines an ordering of the vertices of the graph. Then, if compatible sequences of partitions are found, the corresponding orderings yield the isomorphism between  and .



Let us consider a graph .
A \emph{partition} of a set  is a sequence  of disjoint
nonempty subsets of  such that .  The sets  are called
the \emph{cells} of .
A \emph{sequence of partitions} of  is a list of partitions . The sequence
starts with the trivial partition , and partition  is obtained
from partition  using some refinement. Each refinement can be a \emph{set refinement} or a \emph{vertex refinement} (or vertex individualization).
In both cases a pivot cell is chosen. In a set refinement, each cell of the initial partition  is divided into smaller cells in  as a function of the
available degree of each of its vertices with respect to the pivot cell. In the vertex refinement, a vertex  of the pivot cell is chosen, and each
cell of the initial partition is divided into smaller cells as a function of the
adjacency of each of its vertices with respect to .
Each refinement can discard some vertices from the partition, either because they have been used as pivot
vertex in a vertex refinement, or because they have no more links with the rest of vertices in the partition.
The sequence ends when all remaining cells are singleton.
In a sequence of partitions, \emph{level}  refers to  and its associated
parameters (e.g., type of refinement, and pivot cell used).



















Consider two graphs  and , and sequences of partitions  and  of them.
We say that the sequences are \emph{compatible} if both have the same number of levels , at each level the corresponding partitions are compatible as well (i.e., have the same number of cells, pairwise with the same size and the same available degree between them), the refinements applied at each level  is the same, and the pivot cell is in the same position, and the
final partitions  and  have the same adjacencies between cells.
The following theorem shows that having compatible sequences of partitions is equivalent to being isomorphic.

\begin{theorem}[\cite{DBLP:conf/wea/Lopez-PresaA09}]
\label{iso-iif-seq}
Two graphs  and  are isomorphic iff there are two compatible sequences of partitions  and  for graphs  and  respectively.
\end{theorem}

The challenge to derive fast isomorphism testing algorithms based on sequence of partitions is to deal with backtracking in the search for a compatible sequence of partitions.
Backtracking occurs when vertex refinement is used in the original sequence,
and the pivot cell has more than one vertex. Our algorithms always guarantee that this happens only when the partition has no singleton cells 
and it is not possible to refine the partition by means of a set refinement. When this happens at level , we say that we have a \emph{backtracking point} at that level, or that  is a
\emph{backtracking level}.

In order to deal with backtracking points, the conauto algorithms use vertex equivalence and automorphism detection. 
In a graph , two vertices  are \emph{equivalent at the level } of a sequence of partitions if there is another sequence of partitions, compatible with the first one, in which the first  partitions are the same. 
Observe that two vertices that are equivalent at level  are equivalent at all levels . Two vertices equivalent at some level belong to the same \emph{orbit}.




\section{Description of the Algorithm \emph{conauto-2.0}}
\label{s-algorithm}

Algorithm \emph{conauto-2.0} compares two graphs  and  for isomorphism. It works in the following way. First it computes a sequence of partitions for graph . Then, unlike previous versions of conauto which only perform a partial (polynomial time) search for automorphisms, \emph{conauto-2.0} performs a full search for automorphisms. This yields a complete set of generators for the automorphism group of the graph.  The automorphisms found allow removing backtracking points from the sequence of partitions (a backtracking point at a level  is removed, when all the vertices of the pivot cell in level  are equivalent at that level). If, after removing backtracking points, the resulting sequence of partitions of  still has backtracking points, a sequence of partitions for graph  is generated, and the search for automorphisms and backtracking point removal in that graph is performed. Note that, if the sequence of partitions of  does not have backtracking points, then it is not necessary to generate the sequence of partitions for . 

Then, a backtracking process is used to find a match between the vertices of the two graphs. In this process, the sequence of partitions of one of the graphs is used as target, and an attempt is made to find a compatible sequence of partitions for the other graph. This uses the knowledge of the automorphism groups of the graphs, what drastically prunes the search space. The sequence of partitions with less backtracking points is chosen as the target for the match. Observe that if the target sequence of partitions has no backtracking points, the matching process has no backtracking.

\paragraph{Pivot cell selection.}
When the sequence of partitions of a graph is computed, at each backtracking point, a pivot cell must be chosen for vertex individualization. In conauto-2.0, at a level , for each possible combination of cell size and available degree, the first cell in the partition, with that size and  available degree, is considered.  
The first vertex in each of these cells is individualized and the obtained partition is subsequently refined until an equitable partition is reached\footnote{Partition  is {\em equitable} if, for all  and all , the available degree of  and  with  is the same.}.
For each vertex , assume the equitable partition is reached at level . If the partition at that level  is a sub-partition of the partition at level , then the cell of  is chosen without testing any other. Otherwise, we sum the number of discarded vertices 
and the number of cells of this resulting partition. Then, the cell of the vertex  which yields the biggest such sum is chosen as the pivot cell. Note that, since only one vertex in each cell is considered, the choice is not isomorphism invariant and, hence, cannot be used for canonical labeling. However, it works very well in practice for computing the automorphism group of a graph.

\paragraph{Recorded failures.}
In \cite{jllopez2009} one of the authors of this work suggested that recording failures during the search for automorphisms could help pruning the search space. Recently, the same idea has been proposed by Junttila and Kaski in \cite{springerlink:10.1007/978-3-642-19754-3_16}. The use of failures in conauto (version 2.0) and bliss (version 0.65 and above) are similar, but in our case they are used both in the search for automorphisms and during the matching process.

\paragraph{Search for automorphisms.}
The search for automorphisms is performed iteratively traversing all the backtracking points of the sequence of partitions from the last to the first. At each backtracking point, every vertex of the pivot cell (except the pivot vertex  used in the original sequence of partitions), which has not been found yet to be equivalent to , is chosen for vertex individualization. It is used to generate an alternative sequence of partitions using the same pivot cells and refinement procedures used in the original sequence of partitions. If a compatible sequence of partitions is found, the corresponding automorphism is recorded as a generator, and the vertex equivalences are updated accordingly. If all the vertices of the pivot cell at some level are found to be equivalent, then the backtracking point is removed.

The search for a compatible sequence of partitions that induces an automorphism of the graph is performed by a backtracking process that explores every feasible path in the search tree (all feasible pivot vertices at each backtracking point traversed). Known automorphisms and sub-partitions (as described below) are used to prune this search.

\paragraph{Use of sub-partitions to prune the search for automorphisms.} Consider the following definition.
\begin{definition}
\label{subpartition-def}
Let  and , , be two backtracking levels. We say that  is a \emph{sub-partition} of  if  such that , , and . (I.e., each cell of  is included in a different cell of .)
\end{definition}
As it will be proved in next section, when the search for automorphisms is being performed at level , it is enough to consider partitions up to level , where  is the first level reached such that  is a sub-partition of . If a compatible alternative sequence of partitions can be generated up to level , then it is possible to infer an automorphism without the need to generate any further partition. Observe that the property that partition  is a sub-partition of partition  can be evaluated before the search for automorphisms starts.

