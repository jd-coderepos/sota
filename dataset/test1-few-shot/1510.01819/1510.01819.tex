\documentclass{article}
\usepackage{theorem,graphicx,amssymb,amsmath}
\usepackage[retainorgcmds]{IEEEtrantools}
\usepackage{lineno}




\theorembodyfont{\slshape}

\newtheorem{theorem}{Theorem}
\newtheorem{lemma}[theorem]{Lemma}
\newtheorem{cor}[theorem]{Corollary}
\newtheorem{prop}[theorem]{Proposition}
\newtheorem{invar}{Invariant}
\newtheorem{obs}{Observation}
\newtheorem{conj}{Conjecture}
\newtheorem{defini}{Definition}

\def\QED{\ensuremath{{\square}}}

\def\markatright#1{\leavevmode\unskip\nobreak\quad\hspace*{\fill}{#1}}
\newenvironment{proof}
  {\begin{trivlist}\item[\hskip\labelsep{\bf Proof.}]}
  {\markatright{\QED}\end{trivlist}}

\newcommand{\conv}{\operatorname{Conv}}
\newcommand{\vol}{\operatorname{Area}}




\title{Balanced Islands in Two Colored Point Sets in the Plane\footnote{
Oswin Aichholzer and Birgit Vogtenhuber were supported by the ESF EUROCORES programme EuroGIGA--ComPoSe, Austrian Science Fund (FWF): I 648-N18.
Pablo Perez-Lantero was partially supported by projects CONICYT FONDECYT/Iniciaci\'on 11110069 (Chile), and Millennium Nucleus Information and Coordination in Networks ICM/FIC RC130003 (Chile).}}

\author{Oswin Aichholzer\thanks{Institute of Software Technology, Graz University of Technology, Graz, Austria.\texttt{[oaich|bvogt]@ist.tugraz.at}}
\and Nieves Atienza\thanks{Departamento de Matem\'atica Aplicada, Universidad de Sevilla, Spain. \texttt{natienza@us.es}}
\and Jos\'e M. D\'iaz-B\'a\~nez\thanks{Departamento de Matem\'atica Aplicada II, Universidad de Sevilla, Spain. \texttt{dbanez@us.es}}
\and Ruy Fabila-Monroy\thanks{Departamento de Matem\'aticas, Cinvestav, D.F., M\'exico. \texttt{ruyfabila@math.cinvestav.edu.mx}}
\and David Flores-Pe\~naloza\thanks{Departamento de Matem\'aticas, Facultad de Ciencias, UNAM, Mexico. \texttt{dflorespenaloza@gmail.com}}
\and Pablo P\'erez-Lantero\thanks{Departamento de Matem\'atica y Ciencia de
la Computaci\'on, Universidad de Santiago, Santiago, Chile. \texttt{pablo.perez.l@usach.cl}}
\and Birgit Vogtenhuber\footnotemark[2]
\and Jorge Urrutia\thanks{Instituto de Matem\'aticas, UNAM, M\'exico. \texttt{urrutia@matem.unam.mx}}}
        



\begin{document}
\maketitle



\begin{abstract}
Let  be a set of  points in general position in the plane,  of which are red and  of which are blue.
In this paper we prove that there exist:
for every , a convex set containing exactly  red points
and exactly  blue points of ; a convex set containing exactly  red points
and exactly  blue points of .
Furthermore, we present polynomial time algorithms to find these convex sets.
In the first case we provide an  time algorithm  and
an  time algorithm in the second case.
Finally, if  is small, that is, not much larger
than , we improve the running time to .
\end{abstract}

\newpage


\section{Introduction}

Let  be a set of  points in the plane,  of which are red and  of which are blue. 
Without loss of generality, we assume that  (and any other finite point set in this paper) is in general position, that is, no three points lie on a common line. 
A large class of problems in Discrete and Computational Geometry involves partitioning such point sets.
A typical question in this context is whether a given 2-colored point set may be partitioned into parts that satisfy certain predefined properties.
In this paper, we present algorithms for computing convex sets that contain a balanced proportion of points of  of each color.

The Ham Sandwich theorem states that there exists a straight line that simultaneously partitions the red points and the blue points in half. 
As a consequence, there exists a convex set  containing half of the red points and half of the blue points of .
This result can be generalized as follows.

\begin{theorem}\label{thm:hobby}\textbf{(The Balanced Island Theorem)}
Let  be a set of  red points and  blue points in the plane.
Then, for every   there exists:
\begin{enumerate}
  \item a convex set containing exactly  red points and exactly  blue points of ;
  \item a convex set containing exactly  red points and exactly  blue points of .
\end{enumerate}
\end{theorem}

An \emph{island} of  is a subset  of  such that .
The first case of Theorem~\ref{thm:hobby} implies in particular, that if  then
for every  there exists an island containing 
 red points and  blue points of ; such an island contains a balanced
number of red and blue points.  See Figure~\ref{fig:example} for an example.

On the other hand, consider the following construction.
Place  red points at the vertices of a regular -gon and place
 blue points close to its center. Every convex set containing 
 red points contains all the blue points.
The gap between  and 
is filled by the second case of Theorem~\ref{thm:hobby}.

\begin{figure}
  \begin{center}
   \includegraphics[width=0.5\textwidth]{figs/example}
\end{center}
\caption{An example of a balanced island for  and .}
\label{fig:example}
\end{figure}

In higher dimensions the Ham Sandwich theorem states that the color 
classes of a -colored finite set of points  in  can be 
simultaneously bisected by a hyperplane. Although the Ham Sandwich 
theorem can be easily proven in dimension two, already
for dimension three, tools from Algebraic Topology 
are needed. This is also the case for Theorem~\ref{thm:hobby}.
The first case of Theorem~\ref{thm:hobby} can be derived from
a theorem of Blagojevi\'c and Dimitrijevi\'c Blagojevi\'c (Theorem 3.2 in \cite{equivariant}), 
which is a prime example of the applications of Algebraic Topology in Combinatorial Geometry.
This implication was also noted by Sober\'on as a remark in his PhD thesis~\cite{pablo}. 

The connection between Algebraic Topology and Combinatorial Geometry
can sometimes be hard to understand without the proper background.
For the sake of self-containment, 
we include an expository account of this connection in Section~\ref{sec:topo}.

We prove Theorem~\ref{thm:hobby} in Section~\ref{sec:hobby}. In Lemma~\ref{lem:gen}
we show how to derive the first case of Theorem~\ref{thm:hobby} from the result
of \cite{equivariant};  the second case of Theorem~\ref{thm:hobby} needs a separate proof, 
which we give in Lemma~\ref{lem:n+1}; the argument used is also topological in essence. 


Finally,  in Section~\ref{sec:algo}, we consider the algorithmic facet of Theorem~\ref{thm:hobby}. In 
 Theorem~\ref{thm:hobby_alg} we show that the convex set guaranteed by Theorem~\ref{thm:hobby} can
be found in  time in the first case, and in   time
in the second case. We also show in Theorem~\ref{thm:alg_2} that if 
 is small, that is, not much larger than
, the running time  can be improved
to . 



\section{Topological preliminaries}\label{sec:topo}

\subsection{The Ham Sandwich and Borsuk-Ulam theorems}\label{sec:ham}

The statement that there exists a straight line simultaneously
bisecting the color classes of a two-colored finite point
set in the plane, is what many computational geometers would
recognize as the Ham Sandwich theorem. It generalizes
to higher dimensions as follows.

\begin{theorem} \textbf{(Discrete Ham Sandwich theorem).} \label{thm:ham_d}

Let  be finite point sets in .
Then there exists a hyperplane that simultaneously bisects\footnote{Each of the two open half-spaces defined
by the hyperplane contain at most  points
of  . } each .
\end{theorem}

At first sight it might be hard to see the connection
of Theorem~\ref{thm:ham_d} with Topology. This connection
perhaps is more apparent in the following continuous version of Theorem~\ref{thm:ham_d}.

\begin{theorem} \textbf{(Continuous Ham Sandwich theorem).} \label{thm:ham_c}

Let  be bounded open sets in .
Then there exists a hyperplane that simultaneously bisects\footnote{Each of the two open half-spaces defined
by the hyperplane contain half of the volume of .} each .
\end{theorem}

We point out that Theorem~\ref{thm:ham_c} is usually
stated in the more general setting of finite Borel measures. To keep our 
exposition as self-contained as possible, we opted to use volumes
of open sets instead. 

The discrete version of the Ham Sandwich theorem can be proven using the 
continuous version. Given  
finite point sets in , the first step
is to replace each point with a ball of radius 
centered at the point. The continuous Ham Sandwich theorem ensures that 
there exists a hyperplane that simultaneously bisects
these expanded 's. If we let  tend to zero this hyperplane
converges to a hyperplane that simultaneously bisects the original 's.

The continuous version of the Ham Sandwich theorem can be
proven using the Borsuk-Ulam theorem. The Borsuk-Ulam theorem has
many equivalent formulations. One of them 
states that for any map (continuous function) from the  -dimensional sphere 
to , there exists a pair of antipodal points
with the same image. 

\begin{theorem} \textbf{(Borsuk-Ulam theorem A).} \label{thm:borsuk_p}

For every map  there
exists a point  such that .
\end{theorem}

To prove the continuous version of the Ham Sandwich theorem, one first chooses a set
of a given family of bounded open sets  of .
Say  is chosen. For every possible direction , consider the set of oriented
hyperplanes, orthogonal to , that bisect . These hyperplanes form an interval
along this direction. Let  be the first such hyperplane. 

It can be checked that the set 
is topologically equivalent (homeomorphic) to . In this setting, 
pairs of antipodal points
correspond to pairs of oriented planes with parallel supporting planes and 
with opposite orientation, such that the volume of  that
is contained between them is equal to zero. A map  from this space 
of oriented planes to , is defined by 
mapping each such plane  to the point  whose
-th coordinate is the fraction of the volume of  that lies above .
By the Borsuk-Ulam theorem, there exists a plane  that has
the same fraction of the volume of  above it as its antipodal plane  has. 
Therefore,  and  simultaneously
bisect every .

For more applications of the Borsuk-Ulam theorem see Matou{\v{s}}ek's book~\cite{borsuk}.

\subsection{Equivariant maps}\label{sec:eq_map}

The Borsuk-Ulam theorem as stated in Theorem~\ref{thm:borsuk_p} is
formulated in a positive way---it ensures the existence of a pair
of antipodal points of  with a certain property.
It can also be formulated in a negative way---that no map 
from  to  with a certain property (antipodality) exists. A continuous function 
 is \emph{antipodal} if  for all
. We explicitly give this negative formulation of the Borsuk-Ulam theorem.

\begin{theorem}  \textbf{(Borsuk-Ulam theorem B).} \label{thm:borsuk_n}

There is no antipodal map from  to .
\end{theorem}

Antipodality and the negative formulation of the Borsuk-Ulam theorem
are examples of a more general phenomena. In the case of antipodality,
 consider the map that sends every point 
to its antipodal point. This map together with the identity on 
form a group under function composition. This group is isomorphic to the group  
(the unique group with two elements).
So  is said to \emph{act} on . The formal and more
general definition is the following.

\begin{defini}
An \textbf{action} of a group  on a topological space  is 
an assignment of an homeomorphism  of  to every
element  of , such that

\begin{itemize}

\item  is the identity on  if and only if  is the identity element of .

\item  for all .

\end{itemize} 

\end{defini}

The \emph{action} of an element  on a point  is defined as the point
. If the action is already specified or implied,
we simply write . We say that   is \emph{free}
if  for some  implies that .
In other words, the only homeomorphism that
maps some point to itself is the one assigned
to the identity element. 

For a given group  acting on two topological
spaces  and  a -\emph{equivariant} map
is a map 
such that  for all  and .
An antipodal map from  to  is just a 
-equivariant map. The negative formulation of the
Borsuk-Ulam theorem can be reinterpreted as the statement that 
there is no -equivariant map
from  to , where  acts
freely on both  and .

The negative formulation of the Borsuk-Ulam theorem can
be used to prove the Ham Sandwich theorem in the following way.
Assume that there exists a family of bounded open sets 
of  for which there is no hyperplane that simultaneously
bisects all of them. To apply the negative version
of the Borsuk-Ulam theorem, we use this assumption to define
an antipodal (-equivariant) map 
from  to ; thus arriving to a contradiction.

We proceed in a similar way as when using the positive
formulation of the Borsuk-Ulam theorem. Again, we use the map 
 from the set of oriented planes that bisect  to .
(Recall that  is the point of  whose 
-th coordinate is the fraction of the volume of  that lies above
the hyperplane .)
The image of  is actually the -dimensional cube  rather
than all of ; we will regard
 as a map from  to . 

The topological space  can be equipped with an ``antipodal''
function by mapping every point  to the point 
whose -th coordinate is . 
The assumption that there is no hyperplane simultaneously
bisecting all the 's, is equivalent to the assumption
that no bisecting plane of  is mapped to the point 
.
Actually,  is a -equivariant map from
 to . 

Let  be a point
in  and let  be the
infinite ray with apex  that passes through . 
Let  be the map that sends  to the intersection of  and
the boundary of . The boundary of  is homeomorphic
to  and the antipodal function on 
defines a free -action when restricted to it.
The function  is the desired 
antipodal map from  to . 

The method just described to prove the Ham Sandwich
theorem is certainly more involved than the method described in 
Section~\ref{sec:ham}. However, it is this approach that has been streamlined
to prove many equipartition theorems in what has been
called the ``Configuration Space-Test Map'' (CS-TM) scheme. 
For a nice survey of the CS-TM scheme 
see {\v{Z}}ivaljevi{\'c}'s paper~\cite{user_guide}.

\subsection{Configuration Space-Test Map Scheme}

Suppose that we want to prove that an object with a certain
 property exists. (In the
previous example we searched for a hyperplane that simultaneously
bisects a given family  of bounded
open sets of .)
The approach of the CS-TM scheme
is as follows.

A set of candidates for the solution is first defined. 
This set is then given a topology and a free action of a group .
This space  is called the \emph{configuration space}. 
In our previous example,  was the space of oriented hyperplanes
that bisect .

Afterwards, a map  from  to a space  is defined.  
The desired object is then shown to exist
if and only if some point of  is mapped by  to a certain
subspace  of . The space  and its subspace
 are called the \emph{test space}; the map  is called
the \emph{test map}. In our previous examples
the roles of  and  where played by the boundary of , 
and , respectively.

An action of  on  is defined
so that  is a -equivariant map; it is also required that
 acts freely on .
When restricted to , the map  becomes a -equivariant map
from  to . Finally, one shows that no such maps can
exist---contradicting the assumption that our desired object does not exist.
This last part of proving the non-existence of equivariant maps
is where tools from Algebraic Topology typically come into play.

\subsubsection{Blagojevi\'c's and Dimitrijevi\'c Blagojevi\'c's Theorem under the CS-TM Scheme}\label{sec:k_fan}

The theorem of Blagojevi\'c and Dimitrijevi\'c Blagojevi\'c
is an equipartition result of -fans on the sphere; 
we now illustrate how their proof fits into the CS-TM scheme.
For the following definitions regarding -fans we follow the exposition
in B\'ar\'any's and Matou{\v{s}}ek's paper~\cite{k-fan}.
The study of equipartition results using -fans was initiated
by Akiyama Kaneko, Kano, Nakamura, Rivera-Campo, Tokunaga and Urrutia
in \cite{jorge}.

A \emph{-fan} in the plane is a set of  infinite rays,
that emanate from the same point. This point is called 
its \emph{apex}. A -fan in the plane can also be a set of  parallel
lines. Given a -fan  in the plane, we call the connected
open regions of  
the \emph{wedges} of . In the case where 
 consists of parallel lines, a wedge is also the union of the two open regions of  
 that are bounded by a single
line.

The inclusion of  parallel lines in the definition
of -fans and the last exception in the definition of its wedges
may seem awkward. However, -fans consisting of rays
and -fans consisting of parallel lines are closely related.
This connection will become clear once we consider
-fans in the sphere and their connection with -fans in the plane.

A \emph{-fan} in the two dimensional \emph{sphere}  is a set
of  great semicircles that emanate from the same two antipodal points. (Recall that 
is a \emph{two}-dimensional surface, but it is normally regarded as embedded in .)
These two points are called its apices.  Given a -fan  in , we call the connected
open regions of  the \emph{wedges} of . See Figure~\ref{fig:wedge}.

\begin{figure}
  \begin{center}
   \includegraphics[width=0.5\textwidth]{figs/wedge}
\end{center}
\caption{A -fan on .}
\label{fig:wedge}
\end{figure}
The connection between -fans in the sphere
and -fans in the plane is given by the following map from
the open southern hemisphere of  to .
Assume that  lies on  and identify
 with a horizontal plane lying below .
From the center of  project every point on the southern
hemisphere of  to . Let  be this
map. The image under  of a -fan for which all its great semicircles
intersect the open southern hemisphere of  is a -fan
in . If the apices of this -fan are on the
equator of  then the image of this -fan corresponds
to a set of  parallel lines in . Conversely
the preimage under  of a -fan in  corresponds
to a -fan in  for which all its semicircles 
intersect the open southern hemisphere. 
This connection between -fans in the sphere and in the plane
allows us to translate partition results by -fans
in the sphere to partition results by -fans in the plane. 

Let  and  be two finite Borel measures on  and
 be two real numbers such that . 
The result of \cite{equivariant} states that there exists
a -fan such that two of its wedges have an  proportion
of  and , and the remaining wedge contains a 
proportion of  and .  Keeping up with our convention we reformulate 
that statement in terms of areas of open sets. 

\begin{theorem}\label{thm:fan}\textbf{(Theorem 3.2 in \cite{equivariant})}
Let  and  be two open sets on , and let
 be two real numbers such that . 
Then there exists a -fan on  such that its corresponding
wedges  and  satisfy:
\begin{itemize}

\item ,
\item  and
\item .
\end{itemize}
\end{theorem}

The proof of Theorem~\ref{thm:fan} in~\cite{equivariant} is done 
when  and  are rational numbers. Afterwards, using standard
arguments it is shown that Theorem~\ref{thm:fan} holds when  and  
are real numbers. So assume that  and  are rational numbers.
Let  be natural numbers
such that  and . Note that
. 


\subsubsection*{Configuration Space}

We consider -fans rather than -fans. 
We need to ``orient'' the set of -fans, similar to as 
we did with the halving planes in Section~\ref{sec:ham}.
Let  be a -fan in .  Recall that 
has two apices, and  great semicircles. The orientation is given 
by choosing a tuple  where  is one of its apices, and  is 
one of its great semicircles.
 Once an orientation  is chosen for ,
the  great semicircles  of  are assumed to 
be sorted counterclockwise around , with  being the first
one. For , let  be the wedge of  
bounded by  and , and let  be the wedge bounded by
 and .

Let  be the set of all -fans that equipartition . 
That is, all -fans  such that
 for .
Let  be such a -fan and let  be the straight line
through its two apices. Note that each semicircle
 of  can be moved around a maximal interval  (of possibly one point) of great semicircles around ,
so that  does not change. If  has more than one point, define  as the wedge bounded
by the great semicircles at
the endpoints of . In particular, if  has more than one point then  is equal to zero.
Let  be the -fan that is obtained from  by placing each  at the midpoint
of . Note that . 

The configuration space is the subset of , given by .
We parametrize this space by assigning a pair of orthogonal unit vectors to each  as follows. 
Assume that  is of radius one.  Assign to  the tuple 
where  is the unit vector with endpoint at , and  is the
unit vector orthogonal to  whose endpoint lies in .
Note that since  equipartitions , and  is in ,  and  determine
all of . Thus, a tuple  of orthogonal 
vectors corresponds to
exactly one of these -fans. 
With this correspondence in mind, the 
configuration space is the space, ,
 of  all tuples of unit orthogonal
vectors in . (In the literature 
is known as the Stiefel manifold~\cite{stiefel}.) 


We specify a group action on . Let  be the homeomorphism
of  that sends  to , and let 
be the homeomorphism that sends  to . The
group generated by  and  is isomorphic 
to the dihedral group ---the
group of symmetries of the regular -gon. This is clear
once one realizes that  corresponds to a reflection of the regular 
-gon, and that  corresponds to a clockwise rotation 
of the -gon by one. We assume that  acts
on  via  and ; note that this
action is free.

\subsubsection*{Test Space and Test Map}

Let  be the linear subspace of  defined
by 


Let  be the subspace of  defined by the equations

\begin{IEEEeqnarray*}{lCll}
  x_1 & +\dots+ & x_{a_1} & = 0,\\
  x_{a_1+1} & +\dots+& x_{a_1+a_2} &  = 0, \\
  x_{a_1+a_2+1} & +\dots+& x_{k} &  = 0.
\end{IEEEeqnarray*}



Let  be the map defined by



Let  be the -fan with apex  and with great semicircles  and .
Note that if  then:

\begin{IEEEeqnarray*}{cccccc}
 \vol(\gamma_1' \cap B)/ \vol(B) & = & \sum_{i=1}^{a_1} \vol(\gamma_i \cap B)/\vol(B)~       & = & ~\alpha, &\\ 
 \vol(\gamma_2' \cap B)/ \vol(B) & = & \sum_{i=a_1+1}^{a_1+a_2} \vol(\gamma_i \cap B)/\vol(B)~ & = & ~\beta, & \textrm{ and} \\
 \vol(\gamma_3' \cap B)/  \vol(B)& = & \sum_{i=a_1+a_2}^{k} \vol(\gamma_i \cap B)/ \vol(B)~   & = & ~\alpha.& 
\end{IEEEeqnarray*}

In this case,  is the desired -fan of Theorem~\ref{thm:fan}. 
We now equip  with a free -action. Let  be the homeomorphism
of  that sends  to , and
let  be the homeomorphism of  that sends  to 
. The group generated by  and  is isomorphic to 
. We assume that  acts on  via  and
. Note that this action is free and  is a -equivariant
map from  to . If no -fan is mapped by  to  then  
is an equivariant map from  to . In \cite{equivariant},
Blagojevi\'c and Dimitrijevi\'c Blagojevi\'c prove that no such
map exists. The proof of this last part is far from trivial. Indeed, it is
the gist of the proof of Theorem~\ref{thm:fan}--we have merely presented the prelude.
Unfortunately, a detailed account of this is
 beyond an expository account. 



\section{Proof of the Balanced Island Theorem}\label{sec:hobby}
We are now ready to prove Theorem~\ref{thm:hobby}. We show the first
case in Lemma~\ref{lem:gen} and the second case in Lemma~\ref{lem:n+1}.

A way to prove partition theorems on points sets
from similar partition theorems on finite Borel measures (or areas of open sets in our case)
is the following.
First enlarge each point to a disk of radius , and use the measure
theorem to find a solution. Then let  tend to zero and show
that the limit of these solutions exists and that it is  the desired
solution for point sets. We applied this approach 
in Section~\ref{sec:ham}, when we sketched how to obtain the
 discrete version (Theorem~\ref{thm:ham_d}) of the Ham Sandwich theorem
from its continuous version (Theorem~\ref{thm:ham_c}). We follow
this approach again in the proof of Lemma~\ref{lem:gen}.

\begin{lemma}\label{lem:gen} 

Let  be a set of  red points and  blue points
in the plane. Then for every 
there exists a convex set containing exactly  red points
and exactly  blue points of . Moreover, this convex
set is either a convex wedge or a strip.
\end{lemma}
\begin{proof}

Assume without loss of generality that  and  are rationals.
We project the open southern hemisphere of  to  
as in Section~\ref{sec:k_fan}. Assume that  lies on  and identify
 with an horizontal plane lying below . 
From the center of  project every point on the southern
hemisphere of  to . Let  be this map.

Let .
On every red point 
place an open disk of radius  centered at this point; 
let  be the union of all these disks. Likewise, 
on every blue point  place an open disk of radius  centered at this point; 
let  be the union of all these disks. 
Let  be the -fan of  given by 
Theorem~\ref{thm:fan} for , ,  and 
. Choose  small enough so that   and  lie on the southern hemisphere of ;
if necessary, perturb  so that the three semicircles
of  intersect the southern hemisphere of . Note that 
is a -fan in . 

The -fan  defines three wedges: two of which contain
an  proportion of the area of  and ; the remaining wedge, ,
contains a  proportion of the area  and . Assume that 
 is oriented so that  is bounded
 by the first and the second semicircle. At least one of the two wedges that contain
an  proportion of the area of  and  is convex when projected
under~. Of these two wedges, let  be the first wedge clockwise from  that is convex
when projected under~.

Note that once the apex, , and the first semicircle, , 
of  are fixed, all of  is determined.
This implies that once  is oriented we can identify 
it with a tuple of unit orthogonal vectors. 
Assume that  is the 
unit sphere centered at the origin. Set the first vector
of the tuple to be  and the second to be the unit vector orthogonal 
to  that lies on . It can be verified that this space, , 
of tuples of orthogonal unit vectors in  is compact.  

Consider any sequence of 's that converges to zero; the corresponding
sequence of 's has a limit point  .
Let   be a subsequence of this sequence that converges
to  and let  be the wedge
of  that  converges to. ~is convex  since
each of the terms in  is convex when projected
under~. 

Note that since   is in general position no three points  are in a
common great semicircle of . Furthermore, each of the terms in 
 contain an  proportion of the
area of  and . This implies that the closure of  (in )
contains at least   and at most
  red points of . By
the same token, it contains at least   and at most
  blue points of .
If  has one more red point than  ,
then a red point lies in one of the bounding semicircles of . Similarly, 
if  has one more blue point than  ,
then a blue point lies in one of the bounding semicircles of .
In both cases a small perturbation of  ensures that it
contains exactly  red points
and exactly  blue points.
The convex wedge  is the desired convex set
and the result follows.
\end{proof}


\begin{figure}
  	\begin{center}
   	\includegraphics[scale = 0.65, page = 2]{figs/gridnew}
	\end{center}
\caption{Illustration of the proof of Lemma~\ref{lem:n+1} for  and . 
	The points of  are colored like the according points of . 
	The part of  that corresponds to the desired subset of  is marked with a gray background.}
\label{fig:grid}
\end{figure}

\begin{lemma}\label{lem:n+1}
Let  be a set of  red points and  blue points in the plane. Then 
there exists a strip containing exactly  red points and exactly  blue points of .
\end{lemma}

\begin{proof}
Let  and .
For a given oriented line , let  be the orthogonal projection of  to .
Assume that  is such that no two points of  are projected to the same point in . 
Further assume that the points of  are sorted by the order in which they appear on . 
Note that if  contains an interval (i.e., a contiguous subsequence of points) of exactly  red points and exactly  blue points,
then there exists a strip bounded by two lines orthogonal to  and containing exactly  red points and  blue points of .

Let  be the  integer grid graph with vertex set , 
	in which two vertices are adjacent if in one of their coordinates they are equal and in the other they differ by one. 
We assign a path  in  to .
We define it as follows. 
The first vertex  is equal to . 
For , the -th vertex  is equal to  if the -st element of  is red, and equal to   if it is blue.
Note that  always ends at .


Let  be the translation of  by the vector .
That is,  has length , and the -th vertex of  has coordinates . 
If  and  have a common vertex  then  contains the vertex . 
This implies that in the contiguous sequence from the -th to the -st point of  there are exactly  red and  blue points.
Hence it suffices to show that  and  intersect.

We may assume that  does not contain the vertex , since it would imply that  also ends at . 
Therefore,  must intersect the path 
	, , ,  or the path 
	, ,  , .
Without loss of generality, assume that  intersects  (if not we interchange the colors), and let  be an intersection point of  and .

Suppose first that  intersects the path  , ,  , . 
Then, since  ends at  it must intersect the subpath of  that starts at  and ends at , see Figure~\ref{fig:grid}. 

Assume now that  does not intersect .
Therefore,  contains the vertices ,  and  . 
In particular, the -th and the -th point of  are blue. 
Move  continuously until it reaches a line parallel to it but with opposite orientation. 
Throughout the motion, contiguous elements in  exchange their positions until  is inverted. 
During these exchanges,  also changes somewhat continuously: at each step, one rightwards-upwards corner may become upwards-rightwards, or vice versa. 
This implies that at some line  the -th or the -th point of  is red and  and  intersect.
\end{proof}



\section{Algorithms} \label{sec:algo}

A drawback of using topological methods is that they tend  to provide
existential rather than constructive proofs. This is the case
for the proof of Theorem~\ref{thm:hobby}. In this section we give polynomial time
algorithms to find balanced islands.

Our strategy is to use the fact that the convex sets in 
Theorem~\ref{thm:hobby} are either strips or wedges (Lemmas~\ref{lem:gen} and \ref{lem:n+1}). We discretize
the space of candidates (wedges or strips) and efficiently visit them.
 If we are looking for a wedge we first discretize
the set of possible apices; if we are looking for a strip
we first discretize the set of possible directions for the boundaries of the strip.
These two steps are very similar. Indeed,
strips and wedges are the same object on the sphere; 
fixing a direction of a strip is equivalent to fixing
an apex for the corresponding wedge on the sphere (see Section~\ref{sec:k_fan}). 

Suppose that we are looking for a wedge.  We know that
the solution must have exactly  points of  (regardless of color).
Suppose that a finite set of candidates apices has been chosen, and
that  is the first candidate apex visited.
The wedges with apex  containing exactly  points
of  can be discretized as follows. Sort  clockwise by angle
around   in  time. Consider the set of 
all intervals  of  contiguous points of  in this ordering; 
the set of candidate wedges with apex  are those wedges 
whose bounding rays pass through the endpoints
of these intervals. Note that for every wedge with
apex  and containing  points of , there exists
a candidate wedge containing exactly the same points of . 
This set of wedges around  can be constructed in   
time, and as we construct them we can check whether any of them
is balanced and convex. The candidate apices will then be visited in such a way so as
to minimize the changes in this set of candidate wedges; the only possible
changes are that two points transpose in the order around
the apex, or that a wedge ceases to be or becomes convex. In particular, 
at most two wedges can change: the two wedges whose  associated intervals
have endpoints at the points that transpose or the wedge that ceases to be or becomes 
convex. As a result, we can update our set of candidates in constant time when we go 
from one candidate apex to the next.  

When we are looking for strips we proceed in a similar way.
First we discretize the the set of possible directions for the strips.
We visit the first direction  and sort the points in the orthogonal
direction to  in  time. Like before,
the candidate strips are the set of intervals containing  points in this
ordering. When we visit the next direction the only change that occurs is that 
two points transpose in this ordering. Again, we can update our set
of candidate strips in constant time. In the proof of 
Lemmas~\ref{lem:wedge_alg} and \ref{lem:strips_alg}, we detail how to compute
and visit these sets of candidates.


\begin{lemma}\label{lem:wedge_alg}
Let  be a set of  points in general position in the plane,  of which are red and  of which are blue.
Then for any  and , finding a convex wedge 
containing exactly  red points and exactly  
points of , or determining that no such wedge exists,
can be done in  time.
\end{lemma}
\begin{proof}

Let  be the line arrangement generated by the set of lines 
that pass through every pair of points in . As there are  such lines, 
 can be constructed in  time using standard
algorithms for constructing line arrangements. Also note that 
has  cells.

Let  be a cell of  and let  be 
a point in its interior. Starting at
an arbitrary point of , sort the points of  clockwise by angle around . 
 This is done in  time;
let  be this sorted set.

Let .
For , let  be the interval of 
that starts at the -th point and ends at the -th
point of  (modulo ). We compute the number of red and blue points in each 
in  time, by computing this parameter for  and updating it
when we move from  to .  We keep pointers from the first and last vertices
of  to itself. Let  be the wedge with apex , that contains , and
whose bounding rays pass through the first and last vertex of . We also keep
a record of whether  is convex. Note that neither the number of red and blue points of 
 inside , nor whether it is convex, depend on the choice of  within . 

We visit all the cells in  by doing a DFS search in
 time on its
dual graph, so that we move between adjacent cells. We choose
a point  in the interior of each of them. 
The only changes than can occur is that two consecutive
points of  transpose or that a wedge ceases to be or becomes convex.
This is determined by which line of  was crossed
when visiting the next cell. By knowing which pair of points define
this line,  we can update the corresponding 's in constant time.

Consider any convex wedge  with apex  that contains  points
of . Some cell of  contains
 and at some point in our algorithm we chose a point  in this cell. 
One of the candidate wedges  with apex  contains exactly the same 
points as  and the result follows.
\end{proof}

\begin{lemma}\label{lem:strips_alg}
Let  be a set of  points in general position in the plane,
 of which are red and  of which are blue.
Then for any  and , finding a strip 
containing containing exactly  red points and exactly  blue
points of , or determining that no such strip exists,
can be done in  time.
\end{lemma}
\begin{proof}

Let  be the set of lines generated by every pair of 
points in . Sort the lines in  by slope in 
time. Let  be the first line of . Project  orthogonally
to  and let  be this set. Note that strictly speaking,
 the pair of points that define  are mapped
to the same point. We wish to avoid this, so we actually choose a line with a slightly larger slope
than  (but smaller than the next line in ).
 We will make this choice each time we visit a line in .

 Sort in  time 
the points in  by the order in which they appear on . 
Note that if  contains an interval of  exactly  red points
and exactly  blue points
then there  exists a strip bounded by two lines, orthogonal to , containing 
exactly  red points and 
exactly  blue points of .

Let . For , let  be the interval of 
that starts at the -th point and ends at the -th
point of . We compute the number of red and blue points in each 
in  time, by computing this parameter for  in  time
and updating it when we move from  to .
We keep pointers from the first and last vertices
of  to itself. 

We visit the 's in order, while maintaining 
the 's and their respective number of red and blue points. This can be
done in constant time per line. At each step only two consecutive
points of  interchange their positions. The only intervals
that change their endpoints---and thus their number of red and blue
points---are precisely the two intervals that start and end at these
points.

Note that every strip can be rotated without changing
the points of  it contains until its bounding lines
are orthogonal to a line in . Therefore,
finding a strip containing
exactly  red points and exactly  blue points of , or
determining that no such a strip exists,
can be done in  time. 
\end{proof}

We now state the algorithmic version of Theorem~\ref{thm:hobby}.
\begin{theorem}\label{thm:hobby_alg}
Let  be a set of  points in general position in the plane,
 of which are red and  of which are blue; 
let , then:
\begin{enumerate}
  \item  a convex set containing exactly  red points
and exactly  blue points of  can be found in  time;

  \item a convex set containing exactly  red points
and exactly  blue points of  can be found in  time.
\end{enumerate}
\end{theorem}
\begin{proof}
The existence of these convex sets follows from Lemmas~\ref{lem:gen} and \ref{lem:n+1}.
The running times of the algorithms to find them follow from  Lemmas~\ref{lem:wedge_alg} 
and \ref{lem:strips_alg}.
\end{proof}



\subsection{Balanced Islands in  Time}

The running times of the previous algorithms can be improved significantly for many values of .  
For example, Lo and Steiger~\cite{ham_alg} gave an optimal  time algorithm for finding a Ham Sandwich cut for , by this giving an optimal algorithm for .
Using the results from the following lemmas, we will present a significantly improved algorithm for a large range of values  in Theorem~\ref{thm:alg_2}.  


To this end, we first introduce the concept of weighted islands. 
Assume that every red point in  is given a positive weight of  and every blue point in  is given a negative weight of . 
For a given island of  let its weight be the sum of the weight of its points.
To obtain a balanced island for a given , we want 
	an island of weight  
	and containing  points. 
If  and   are both integers then the weight of this island is equal to zero. 
Otherwise, it is as close to zero as possible, among the islands of  with   points.
If an island has weight larger than , we call it \emph{positive}; 
if it has weight smaller than this value we call it \emph{negative}. 

Suppose that we have found a positive island  and a negative island , both with   points of . 
A promising approach would be to move ``continuously'' from  to , so that somewhere in the middle 
	we find a balanced island of  points.
This indeed can be done. In \cite{island_graph} Bautista-Santiago et al.\ defined a graph whose vertices are all the islands of  of a given size~, 
	where two of them are adjacent if their symmetric difference has a fixed cardinality . 
They showed that under mild assumptions on  and  that this graph is connected. 
In our case we have  and . 
A path from  to  in the resulting graph is our desired sequence; somewhere in the middle of such a sequence there is a balanced island. 
We show how to compute this sequence.

\begin{lemma}\label{lem:island_graph}
Let  be a set of   points in the plane and 
let  and  be islands of  of  points each.
Then there exists a sequence of  islands, starting at  and ending at , such that the symmetric difference between two consecutive islands is a pair of points. 
This sequence can be computed in  time.
\end{lemma}

\begin{proof}
Let  be the graph whose vertices are all the islands of  with  points, two of which are adjacent in  if their symmetric difference is a pair of points. 
We look for a path of linear length from  to  in . 
Without loss of generality, assume that no two points of  have the same -coordinate.
We sort the points in  by their -coordinate. 
The subsets of  consisting of  consecutive points in this ordering are islands of  (thus vertices of ).
Let  be the subgraph of  induced by these islands.
Note that  is a path of length at most , and can be computed in  time.  
To complete the proof, we show how to compute a path of linear length from  to a vertex of  in  time. 
A path from  to a vertex of  can be computed in a similar way. 

Compute the convex hull, , of  in  time.  
Let  be the points of  that lie in the vertical strip between the leftmost and rightmost point of . 
Initialize a priority min-queue  with the vertices of .  
Store the points in  according to their shortest distance to . 
This distance can be computed in  time per point, by doing a binary search on  .  
Set . 

Assume that  has  been computed and that  is its rightmost point.  
We extract points from  until we find a point  that is to the left of . 
Note that  is the point to the left of  closest to . Set . 
The rightmost point  of  can be computed in constant time, since it is either  or the first point of  to the left of .  
If  is empty or no such point is found then  is an island of  and we are done.  
Note that by construction the symmetric difference between  and  is a pair of points. 
It remains to show that the 's are islands of .

Suppose that some  is not an island of . 
Then there exists a point  contained in the convex hull of .  
By Caratheodory's theorem there exist three points  and  of  that contain  in their convex hull. 
One of these points, say , is farther away from  than .  
In particular  is not in . Therefore,  was added to create some  with .  
When   was created,  was chosen because it was the point of  closest to  and to the left of . 
This is a contradiction since  is also a point of  to the left of , and it is closer to  than .
\end{proof}

The algorithm to find a balanced wedge in Theorem~\ref{thm:hobby_alg} computes a set of candidate apices, such that
one of them is guaranteed to be the apex of a balanced wedge. 
The set of candidates however is quite large; it has size . 
If  is not too large, we can somehow reduce this set of candidates 
to a single point . This point has the property that either there is a balanced convex wedge with
apex , or there exists both a negative convex wedge with apex  and a positive convex wedge with
apex . In the latter case we can apply Lemma~\ref{lem:island_graph}.
This point  is given by the following lemma.

\begin{lemma}\textbf{(\cite{ceder, ceder_alg})}\label{lem:ceder}
There exist three lines, concurrent at a point , that divide the plane into six open regions, 
		with the property that every region contains at least  points of . 
Moreover,  can be found in  time.
\end{lemma}

The existence of the point  given in Lemma~\ref{lem:ceder} was shown by Ceder~\cite{ceder} 
	using a theorem of Buck and Buck~\cite{buck} on equipartitions of convex sets in the plane. 
The algorithm for finding  is due to Sambuddha and Steiger~\cite{ceder_alg}.
The point  has the property that many of the wedges with apex  and whose bounding rays pass through points of  are convex. 
This is quantified in Lemma~\ref{lem:many_convex}.

\begin{lemma}\label{lem:many_convex}
Let  be a set of  points in the plane, and  a point given by Lemma~\ref{lem:ceder}. 
Let  be  a positive integer.
If   then all the wedges with apex  that contain
 points of  and whose bounding rays pass through points of  are convex; 
if  then at least  of them are convex.
\end{lemma}

\begin{proof}
The six regions in Lemma~\ref{lem:ceder} are all wedges with apex . 
Let  be these wedges sorted clockwise around . 
Note that since each  contains at least  points of , any wedge with apex  
	that contains less than  points is contained in the union of at most three consecutive 's.
Thus any such wedge is convex.

Assume that  and set . 
Sort the points of  clockwise by angle around . 
Let  and  be the first and last  points of , respectively, in this order. 
Let  be a wedge with apex , containing  points of  and whose bounding rays pass through points of . 
Note that if the first vertex of  lies in  then its last vertex lies in  (modulo 6).
In this case  is convex. 
Therefore, for  to be non-convex, its first vertex must be in some . 
Now, if the first vertex of  is in  then its last vertex is in  (modulo 6). 
Let  be the set of wedges with apex  that contain  points of , 
	whose bounding rays pass through points of , and whose first vertex is in . 
Since ,  and  are disjoint.  
Therefore, all the wedges in  are convex or all the wedges in  (modulo 6) are convex. 
This gives  extra convex wedges.
Summarizing, we have  convex wedges not starting at some  plus at least  extra convex wedges starting at some . 
So in total we have at least  convex wedges with apex  that contain  points of .
\end{proof}

\begin{theorem}\label{thm:alg_2}
Let  be a set of  red and  blue points in the plane and 
let  be such that  
	
Then an island containing exactly  red and exactly  blue points of  can be found in  time.
\end{theorem}

\begin{proof}
Let .
Find  as in Lemma~\ref{lem:ceder} in  time. 
We compute the set of wedges,  that have apex , whose bounding rays pass through points of , and that contain  points of  in  time.
While computing the wedges , we also compute their weights.
We are done if there is a convex wedge of weight  in . 
So assume that every convex wedge in  has weight greater or less than  (that is, has positive or negative weight). 

We show that  contains a convex negative wedge and a convex positive wedge.
Afterwards, we apply Lemma~\ref{lem:island_graph} to find a balanced island of exactly  red 
	and exactly  blue points of  in   time.
We only give the proof that  contains a positive convex wedge. 
The proof that it contains a negative convex wedge is similar. 
 
Let  be the number of non-negative wedges in  and let   be the sum of the weights of the non-negative wedges.
Note that the sum over all wedges in  is zero as every point appears in the same number of wedges. 
Therefore, the sum of the weights of the negative wedges is equal to . 
In particular, this implies that there exist both negative and positive wedges.  
If  then by Lemma~\ref{lem:many_convex} all the wedges in  are convex and we are done.  
Assume that .

Note that the weight of two consecutive wedges in  differs by at most . 
For  to achieve its largest possible value, the non-negative wedges must lie consecutively as follows.  
The first wedge has weight .
Subsequent wedges increase in weight by  until they reach a maximum. 
Afterwards, subsequent wedges decrease in weight by  until they reach  again. 
Depending on whether  is even or odd the sequence will stay at this maximum value for one or two wedges of the sequence. 
In both cases, simple arithmetic shows that . 

The largest possible negative weight is . 
Therefore the sum of the negative weights is at most . 
Thus, , and 
		 
Solving for , we have that  
		 
Given that ,  and , this implies that 
	 
By Lemma~\ref{lem:many_convex}, at least  of the wedges of  are convex. 
Thus, since , 
	the number of non-convex wedges of  is at most .  
Therefore, at least one of the convex wedges of  is non-negative.
Since all balanced wedges are non-convex by assumption, this wedge must be positive and the result follows.
\end{proof}

\subsection*{Acknowledgments}

We thank Pablo Sober\'on for pointing out that the Theorem of Blagojevi\'c and Dimitrijevi\'c Blagojevi\'c
implies the first case of Theorem~\ref{thm:hobby}.

\small 
\bibliographystyle{abbrv}
\bibliography{balancedbib}

\end{document}
