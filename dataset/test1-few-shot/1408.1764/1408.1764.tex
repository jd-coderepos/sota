\documentclass[12pt, draftcls, onecolumn]{IEEEtranTCOM}

\usepackage{amssymb,bbm}
\usepackage{color,graphicx}
\usepackage{cite}
\usepackage[cmex10]{amsmath}
\usepackage{amsopn}
\usepackage{floatrow}
\usepackage{cite}


\newtheorem{theorem}{Theorem}
\newtheorem{lemma}[theorem]{Lemma}
\newtheorem{proposition}[theorem]{Proposition}
\newtheorem{definition}{Definition}
\newtheorem{corollary}[theorem]{Corollary}
\newtheorem{example}[theorem]{Example}
\newtheorem{remark}{Remark}

\newcommand\qed{\hfill \rule{1.2mm}{2.8mm}}

\newcommand{\one}{ {\mathbbm 1}}

\newcommand{\R}{\ensuremath{\mathbb{R}}}

\newcommand{\ceil}[1]{\lceil #1 \rceil}
\newcommand{\floor}[1]{\lfloor #1 \rfloor}
\newcommand{\euclidnorm}[1]{\lvert \lvert #1 \rvert \rvert}
\newcommand{\spnorm}[1]{\lvert \lvert #1 \rvert \rvert_1}
\newcommand{\maxnorm}[1]{\lvert \lvert #1 \rvert \rvert_\infty}
\newcommand{\opnorm}[1]{\lvert \lvert #1 \rvert \rvert_2}

\newcommand{\x}{{\bf x}} \newcommand{\X}{{\bf X}}
\newcommand{\e}{{\bf e}}
\newcommand{\ba}{{\bf a}}
\newcommand{\bb}{{\bf b}}
\newcommand{\bd}{{\bf d}}
\newcommand{\bc}{{\bf c}}\newcommand{\bC}{{\bf C}}
\newcommand{\bh}{{\bf h}}
\newcommand{\bn}{{\bf n}}
\newcommand{\bu}{{\bf u}}
\newcommand{\bv}{{\bf v}}  \newcommand{\bV}{{\bf V}}
\newcommand{\bW}{{\bf W}}
\newcommand{\bY}{{\bf Y}}
\newcommand{\bfv}{{\bf v}}

\newcommand{\bx}{{\bf x}}
\newcommand{\bz}{{\bf z}}
\newcommand{\la}{\lambda}


\newcommand{\by}{{\bf y}}
\newcommand{\bw}{{\bf w}}
\newcommand{\z}{{\bf z}}
\newcommand{\uy}{\underline{y}}
\newcommand{\uY}{\underline{Y}}
\newcommand{\ua}{\underline{a}}
\newcommand{\uA}{\underline{A}}
\newcommand{\ux}{\underline{x}}
\newcommand{\uX}{\underline{X}}
\newcommand{\uz}{\underline{z}}
\newcommand{\uZ}{\underline{Z}}
\newcommand{\uV}{\underline{V}}
\newcommand{\uv}{\underline{v}}
\newcommand{\uf}{\underline{f}}

\newcommand{\ocE}{\overline{\E}}

\newcommand{\qs}[1]{|#1\rangle}
\newcommand{\tn}{{2^n}}
\newcommand{\be}{}

\newcommand{\bea}{}

\newcommand{\hs}[1]{\hspace{#1}}
\newcommand{\diag}[1]{{\rm diag}\left\{ #1 \right\}}

\newcommand{\proof}{\noindent {\bf Proof}\ \ }
\newcommand{\Proof}{\noindent {\bf Proof}\ \ }

\newcommand\ff{{\mathbb F}}
\newcommand\rr{{\mathbb R}}
\newcommand\cc{{\mathbb C}}
\newcommand\cmplx{{\mathbb C}}
\newcommand\nn{{\mathbb N}}
\newcommand\zz{{\mathbb Z}}
\newcommand{\ZO}{{\mathbb N}_0}
\newcommand{\NN}{{\mathbb N}}

\newcommand{\wheps}{\widehat{\bf \epsilon}}
\newcommand{\beps}{{\bf \epsilon}}


\newcommand\wt{\mbox{{\rm wt}\,}}
\newcommand\Tr{\mbox{{\rm Tr}\,}}
\newcommand\tr{\mbox{{\rm tr}\,}}
\newcommand\dist{\mbox{{\rm dist}\,}}
\newcommand\A{{\cal A}}
\newcommand\C{{\cal C}}
\newcommand\G{{\cal G}}
\newcommand\N{{\cal N}}\newcommand\tA{\widetilde A}
\newcommand\K{{\cal K}}
\newcommand\caL{{\cal L}}
\newcommand\E{{\cal E}}
\newcommand\cS{{\cal S}}
\newcommand\Z{{\cal Z}}
\newcommand{\remove}[1]{}

\newcommand{\wH}{\widehat{{\bf H}}}
\newcommand{\half}{\frac{1}{2}}
\newcommand\CGauss{{\cal CN}}
\newcommand{\abs}[1]{\lvert #1 \rvert}
\newcommand{\prob}[1]{{\mathbb P}\left\{#1\right\}}
\newcommand{\expect}[1]{{\mathbb E}\left[  #1\right]}
\newcommand{\var}[1]{{\mathbb V}\left[  #1\right]}
\newcommand{\perm}[2]{\left(\begin{array}{c}
                          #1 \\ #2
                            \end{array} \right)}
\newcommand{\lb}{\left(}
\newcommand{\rb}{\right)}
\newcommand{\lc}{\left\{}
\newcommand{\rc}{\right\}}
\newcommand{\ls}{\left[}
\newcommand{\rs}{\right]}
\newcommand{\RRe}[1]{\Re\left( #1 \right)}
\newcommand{\IIm}[1]{\Im\left( #1 \right)}
\newcommand{\rf}{\rho_f}

\newcommand{\FT}{{\cal F}}
\newcommand{\FTI}{{\cal F}^{-1}}
\newcommand{\modul}[1]{\vert\vert #1 \vert\vert^2}
\newcommand{\sqrtmodul}[1]{\vert\vert #1 \vert\vert}
\newcommand{\trce}[1]{\mbox{{\rm Tr}}\left\{#1\right\}}
\renewcommand{\thefootnote}{\fnsymbol{footnote}}
\renewcommand{\le}{\leqslant}\renewcommand{\ge}{\geqslant}

\newcommand{\Ua}{U_\alpha}

\newcommand{\cC}{{\mathcal C}}
\newcommand{\cN}{{\mathcal N}}

\newcommand{\Dlk}{\Delta^{(\ell)}_k}
\newcommand{\Xlk}{\Xi^{(\ell)}_k}
\newcommand{\DL}{\boldsymbol{\Delta}{\bf L}}
\newcommand{\DLF}{\boldsymbol{\Delta}{\bf L}_F}
\newcommand{\Ds}{\boldsymbol{\Delta}}
\newcommand{\Dss}{\boldsymbol{\Delta}(n)}
\newcommand{\lambold}{\boldsymbol{\lambda}}
\newcommand{\elk}{\eta^{(\ell)}_k}

\newcommand{\Rlk}{R^{(\ell)}_k}
\newcommand{\Slk}{S^{(\ell)}_k}
\newcommand{\Nlk}{N^{(\ell)}_k}
\newcommand{\Glk}{G^{(\ell)}_k}
\newcommand{\Tlk}{T^{(\ell)}_k}
\newcommand{\Alk}{A^{(\ell)}_k}
\newcommand{\Wlk}{W^{(\ell)}_k}
\newcommand{\DDlk}{D^{(\ell)}_k}
\newcommand{\bT}{{\bf T}}
\newcommand{\XXlk}{\xi^{(\ell)}_k}
\newcommand{\tlk}{\left( k, \ell \right)}

\newcommand{\lamlk}{\lambda^{(\ell)}_k}
\newcommand{\llamT}{\lambda_T}
\newcommand{\LamZ}{\Lambda_0}
\newcommand{\alk}{a^{(\ell)}_k}
\newcommand{\blk}{b^{(\ell)}_k}
\newcommand{\ylk}{y^{(\ell)}_k}
\newcommand{\xlk}{x^{(\ell)}_k}

\newcommand{\zlk}{\zeta^{(\ell)}_k}
\newcommand{\ulk}{\upsilon^{(\ell)}_k}

\newcommand{\Kl}{K_{\ell}}
\newcommand{\Ke}{K_\epsilon}

\newcommand{\Nflow}[1]{{\bf N}^{ (#1)}}
\newcommand{\bNN}{{\bf N}}
\newcommand{\bTT}{{\bf T}}
\newcommand{\Ns}{{\bf N}(s)}
\newcommand{\Nt}{{\bf N}(t)}
\newcommand{\Nn}{{\bf N}(n)}
\newcommand{\us}{{\bf u}}

\newcommand{\LAZ}{\Lambda_0}

\newcommand{\fstar}{f^*}
\newcommand{\xstar}{x^*}
\newcommand{\ystar}{y^*}
\newcommand{\tstar}{\tau^*}
\newcommand{\lstar}{\lambda_S^*}
\newcommand{\Lfstar}{L_{f^*}}
\newcommand{\rhostar}{\rho^*}
\newcommand{\Keps}{K_\varepsilon}
\newcommand{\VTp}{V^{\pi}_T}
\newcommand{\NTp}{N^{\pi}_T}
\newcommand{\fpT}{f^\pi_T}
\newcommand{\Rmax}{\overline{R}}
\newcommand{\Rmin}{\underline{R}}
\newcommand{\bfa}{{\bf a}}
\newcommand{\rhoint}{[ \underline{\rho}_\ell, \overline{\rho}_\ell ]}
\newcommand{\tbab}{\tau_{b,T}^{({\bf a})}}
\newcommand{\fbab}{f_{T}^{({\bf a})}}
\newcommand{\fsbab}{f^{({\bf a})}}
\newcommand{\ybab}{y_\ell^{({\bf a})}}
\newcommand{\xbab}{x_\ell^{({\bf a})}}
\newcommand{\al}{a_\ell}
\newcommand{\taubab}{\tau^{({\bf a})}}
\newcommand{\taubTLP}{\tau_{b,T}^{(LP)}}

\newcommand{\VLP}{V^{(LP)}_T}
\newcommand{\NTLn}{N_T^{(\ell,n)}}
\newcommand{\NTHLn}{N_{T,H}^{(\ell,n)}}
\newcommand{\tVLP}{\tilde{V}^{(LP)}_T}
\newcommand{\ATLP}{V^{({\bf a}_T)}}
\newcommand{\rell}{\rho_\ell}
\newcommand{\bldups}{\boldsymbol{\upsilon}}
\newcommand{\abups}{{\bf a}_{\boldsymbol{\upsilon},T}}
\newcommand{\nbups}{{\bf n}_{\boldsymbol{\upsilon},T}}
\newcommand{\xpt}{{\bf x}^\pi_T}


\newcommand{\xm}[1]{\lambda\left( #1 \right)}
\newcommand{\nm}[1]{\eta\left( #1 \right)}
\newcommand{\lamT}[1]{\lambda_T\left( #1 \right)}
\newcommand{\bmu}{\boldsymbol{\mu}}
\newcommand{\tdvarphi}{\tilde{\varphi}}
\newcommand{\tdtheta}{\tilde{\theta}}
\newcommand{\tp}{\tilde{\psi}}
\newcommand{\Wc}{W^o}
\newcommand{\suml}{\sum_{\ell = 1}^{L}}
\newcommand{\maxl}{\max_{\ell=1}^L}
\newcommand{\calC}{\ensuremath{\mathcal{C}}}

\newcommand{\vellT}{v^{(\ell)}_T}
\newcommand{\vell}{v^{(\ell)}}

\newcommand{\rhoellscre}{\underline{\rho}_\ell}
\newcommand{\rhoellbar}{\overline{\rho}_\ell}

\newcommand{\fbar}{\overline{f}}
\newcommand{\fellbar}{\overline{f}_\ell}
\newcommand{\Dbar}{\overline{D}}
\newcommand{\Nbar}{\overline{N}}
\newcommand{\Tbar}{\overline{T}}

\newcommand{\barsigma}{\overline{\sigma}}

\newcommand{\bartau}{\overline{\tau}}
\newcommand{\Rsellxi}{R^\sigma_\ell(\xi)}
\newcommand{\fsgma}{f^\sigma}
\newcommand{\esell}{\mu^\sigma_\ell}
\newcommand{\es}{\mu^\sigma}
\newcommand{\er}{\eta^\varrho}
\newcommand{\erell}{\mu^\varrho_\ell}
\newcommand{\Rselln}{R^\sigma_{\ell,n}}
\newcommand{\aselln}{a^\sigma_{\ell,n}}
\newcommand{\bselln}{b^\sigma_{\ell,n}}
\newcommand{\rhosgmaell}{\rho^\sigma_\ell}
\newcommand{\barxsell}{\overline{x}^\sigma_\ell}
\newcommand{\rhorhoell}{\rho^\varrho_\ell}
\newcommand{\barfsgma}{\overline{f}^\sigma}
\newcommand{\Aseta}{A^\sigma_{{\boldsymbol \mu}}}
\newcommand{\Uell}{U_\ell}
\newcommand{\FellP}{\overline{F}^P_\ell}
\newcommand{\FellM}{\overline{F}^M_\ell}
\newcommand{\FzeroM}{\overline{F}^M_0}
\newcommand{\FM}{\overline{F}^M}

\newcommand{\deltf}{\delta f}
\newcommand{\trho}{\tilde{\rho}}
\newcommand{\uB}{\overline{D}}



\begin{document}

\sloppy

\title{Capacity and Stable Scheduling in Heterogeneous Wireless Networks}



\author{Stephen V. Hanly, Chunshan Liu, Philip Whiting
\thanks{The authors are with the Department of Engineering, Macquarie University, Sydney, NSW, 2109, Australia. email: \{stephen.hanly,chunshan.liu,philip.whiting\}@mq.edu.au.}
}


\maketitle


\begin{abstract}
Heterogeneous wireless networks (HetNets) provide a means to increase network capacity by introducing
small cells and adopting a layered architecture. HetNets allocate resources flexibly
through time sharing and cell range expansion/contraction allowing a
wide range of possible schedulers. In this paper we define the capacity of a HetNet down link in terms of the maximum number of downloads
per second which can be achieved for a given offered traffic density. Given this definition
we show that the capacity is determined via the solution to a continuous linear program (LP).
If the solution is smaller than 1 then there is a scheduler such that the
number of mobiles in the network has ergodic properties with finite mean waiting time.
If the solution is greater than 1 then no such scheduler exists. The above results continue to hold if a more general class of schedulers is considered.
\end{abstract}

\begin{IEEEkeywords}
HetNets, Capacity, Stability, Stochastic Networks.
\end{IEEEkeywords}


\section{Introduction}
\label{sec_intro}
HetNets have been proposed as a means to increase the
capacity of wireless networks by introducing short range pico cells
into the existing coverage area of macro cells. Significant increase in user data
throughput is possible mainly because the pico cells can operate simultaneously, providing better time or frequency re-use. This is particulary the case if traffic is
spatially concentrated into small parts of the coverage area known
as traffic ``hot-spots''.

A brief description of HetNet operation is as follows. First, time
is divided into small duration frames known as Almost Blanking Subframes
(ABS), \cite{standards}. These are used to time share between macro cell
transmissions and pico cell transmissions.  In this way interference between
the macro cell(s) and the pico cells is avoided. Additional
flexibility is provided by allowing the pico cells to adjust their
areas of coverage, that is to expand and serve more mobiles or to
contract to serve fewer mobiles, but at potentially higher rates. This
procedure is known as Cell Range Expansion (CRE). As described in standards, CRE uses received signal strength to
determine to which Base Station (BS) a mobile should connect. To make the cell
expand a bias (on the order of 0 - 16 dB) is added to the received signal strength
to determine cell size. Under our model of fixed (at the time
of arrival) location dependent rates we show that an optimal scheme assigns mobiles on
the basis of their rate ratios.



A problem central to the operation of HetNets is how to maintain
stable operation (prevent indefinite build up of backlogged traffic) against
unknown traffic demands. Results associated with this problem
were obtained  in the following papers \cite{Andrews2012,Chen2011,Hu2011,Rudolf2012,VTC2013,ICC2013}. None of these papers address the question of
dynamic stability in conjunction with the joint HetNet ABS and
CRE controls.

In fact the very flexibility in setting ABS slots over time in
conjunction with the pico coverage areas, makes it more difficult
to determine whether a particular offered traffic density can be
supported or not. Nevertheless, we show that given an
offered traffic density (fraction of mobiles arriving in the vicinity
of a location) then there is a well defined notion of capacity. This capacity
is determined as the supremum of all arrival rates which can be supported
with finite queueing delays. This depends not only on the
traffic locations of arriving mobiles but also on the data demands made by the mobiles. In this
paper we suppose all mobiles have a single file of random but bounded length
to download. Once downloaded the mobile departs the system
and does not return. Generalizations of this assumption can be
made of course but the notion of capacity will still be upheld.

The remainder of the paper is as follows. Section
\ref{sec_model} presents our model for the HetNet network, along the lines of
\cite{ICC2013} as well as some preliminary results.  The first of these identify how to
``clear'' a given set of mobiles from the network in minimum time. The second is
a continuous analogue in which total transmission time per unit time is
minimized. This latter problem takes the form of a continuous Linear Program~(LP). (Perhaps more strictly
this should be seen as a problem in the calculus of variations with integral constraints, with bounded
integrable functions, see Section \ref{sec_model}. However we retain this more informal nomenclature for this class
of problems.) With these constructions in hand, we proceed by characterising their respective solutions.
In Section \ref{sec_capres} we show that if for a given overall arrival rate 
the solution to the continuous LP is smaller than 1, then a scheduling policy exists
such that the network is stable.

Section \ref{sec_converse} demonstrates a converse result, i.e.,
the number of mobiles queueing for service must steadily build up (at linear rate, almost surely) if
the solution to the continuous LP is greater than 1.  We then go on to
look at a more general case where individual picos may be on or off
during ABS time with the effect that physical rates in on cells will be higher
as the interference is reduced. Finally, Section \ref{sec_numer}
presents some illustrative numerical experiments and in Section \ref{sec_conc} there are some brief conclusions
and suggestions for further work.

\section{Model and Preliminaries}
\label{sec_model}
\subsection{Model}
We consider a heterogeneous network consisting of a single macro cell
containing  pico BS, see Figure \ref{fig_macropico}.  denotes the (compact) coverage area of
the macro BS whilst  denote the
respective coverage areas of the pico BSs. The pico BS coverage
areas are assumed disjoint. A user at a location in  can obtain service from
both the macro BS and the pico BS . We denote the set  by , which is the set of
locations not covered by any pico BS. Any user in  will obtain all its support from
the macro BS.

The macro BS is assumed to use a much higher transmit power than the pico BSs, to allow it to provide
full coverage of the region . We will therefore only consider scheduling policies in which the
macro time and pico times are disjoint, to avoid excessive interference at the pico BSs from the macro BS. On the other hand, pico BSs are spatially separated and use much lower power. We will therefore consider scheduling policies in which the pico BSs are allowed to operate at the same time.

A pico BS  will use any allocated pico time to send to its users, unless there is no demand for data from within , in
which case it switches off. The switching on and off of pico BSs has the potential to complicate the analysis, as we will see in Section~\ref{sec_genschedule}. Initially, we avoid this complication, by assuming that picocells don't interfere. In this case, the raw bit rates offered by the BSs
do not depend on the traffic in the network; they are deterministic functions of the location of the user in the network (see below). In Section~\ref{sec_genschedule}, we will allow picocells to interfere, at the expense of a more complicated model.

\begin{figure}
\centering
\includegraphics[width=2in]{coverageSC}
\caption{A Macro Cell with  Pico Cells}
\label{fig_macropico}
\end{figure}




We suppose that users want to download files from the network, and that file requests arrive as a Poisson stream with net arrival
rate  files/sec and that the th arrival is for a single file of (random) length  bits to
be downloaded from the network. File lengths are independent and identically distributed ({\it i.i.d.}), from a common distribution , with  bits. The locations of the arrivals are chosen
independently at random according to a continuous density 
with support on  and bounded uniformly away from 0. Mobiles remain
fixed at their initial location until they obtain their file. Hence
in unit time, the expected number of arrivals at the vicinity of a point
 in the macro cell coverage area  is given by
. 

The probability that an arrival file request is allocated to pico-region  is given by

including also the region  that is served only by the macro BS.
The arrivals to each region are independent Poisson processes with rates . The conditional density in region  is given by



The physical transmission rate of a user is determined by the user's fixed
location . All locations are in the macro BS coverage area, and the
corresponding physical rate provided by the macro BS to location , if scheduled, is  bits/sec. If the location is
within the coverage area of pico BS , then an alternative rate is  bits/sec, provided by pico BS , if scheduled. The macro and pico BSs are scheduled at different times, so a mobile can receive data from
both types of BS. The average data rate offered to a location will depend on the higher-layer controls: the cell association (pico or macro) and the time allocation
offered by the selected BS(s). 

We assume that rates depend continuously on location
with  for pico cell rates
and with corresponding bounds for the macro cell rates,  for .
As such they are random variables with measure induced by the density .
Define  and  similarly. Next
define  to be the rate ratios for respective pico cell users. Clearly
 and
. Finally we suppose that the rate-ratio
random variable has a continuous density  and 



We now define the class of scheduling policies under consideration in this paper. Time is slotted with index  where  is the set of non-negative integers. Scheduling decisions are made at the start of each timeslot .  In particular, the first time that a file can be scheduled is in the first full time slot after its arrival. At each time slot, , the scheduler chooses a fraction of the timeslot, , that it will devote to the pico BSs, and the remaining fraction, , is allocated to the macro BS.  Once allocated this time may be divided continuously between the macro and pico users. Note that the pico BSs can all transmit in parallel, as they do not interfere. In addition, users can receive their file partly from the macro BS and partly from their pico BS, if they are within coverage. The fraction of each will be determined by the scheduling policy, to be described below. BSs transmit to one user at a time but can switch between users arbitrarily.

We will only consider non-random scheduling policies so that the scheduling outcome is determined
by the sequence of inter-arrival times of the users, together with their locations and the file sizes themselves i.e. 
where  denotes the set of non-negative real numbers.  Also we restrict to policies which clear users -
that is those schedules which transmit all user files within finite time. This is a mild assumption ensuring
all users will leave the network eventually. We term such schedules clearing schedules. Schedules such as first come first served, or processor sharing, have this property,
under Poisson arrivals with constant rate.

Within this class of schedules, are those which determine the amount of service from the macro cell and the pico cell only according to location and file size. For such schedules ,  we define
  to be the number of bits delivered by  pico , for a user which  arrives at point , with data demand  bits, and the remainder is therefore delivered by the macro BS. We denote the remainder by . In the case that , we define , and . The network can be stabilized using only clearing policies of this type.

Given such a schedule ,  and an outcome   define , to
be the number of mobiles present at the start of slot , under schedule . We are now ready to make the following definition,
\begin{definition}
The network is said to be {\em stable} under a clearing schedule  iff there exists  such that

\end{definition}

For policies  such that a limiting stationary distribution exists, stability implies that transmission delays have bounded first moments.


\subsection{A Continuous Linear Program}
\label{sec_ctsLP}
We begin by investigating the build up of work in the network over time.
To be specific suppose we fix a schedule  and an outcome . Then
define  to be the total transmission time needed to clear all users arriving in the
first  slots under policy . We now seek to construct a
location based policy  with the property that \eqref{eqn_stability} holds, and

almost surely. 

Consider a scenario in which there is a fixed set of users present in a HetNet with their files ready for transmission. The question arises what is the shortest time for all user files to be transmitted? The solution to this problem can be obtained via a LP as shown in \cite{ICC2013}, and which we now
present.

Let the time allocated to the pico cells (which can serve their users concurrently) be denoted by  seconds.
Also, let  and  represent the amount of data (in bits)
received by user  in pico cell~
from that pico cell and the macro cell, respectively. Finally let  be the corresponding
rates,

Using obvious notation, the minimum value can be written , where  identifies the respective mobile's picocells . It was shown in
\cite{ICC2013} that there is a set of constants
, which determine the optimal solution. For users in pico cell 
such that

it is optimal to have  and if the reverse inequality holds then . Only where there is equality in \eqref{eqn_rate_ratio_test} can it be that both
 are positive in the optimal solution. The thresholds  depend on the locations and demands of all the mobiles in the network.


The above LP has a continuous analogue in which  are replaced by
 where  are (Lebesgue) integrable functions.
The LP becomes,

Here, the individual mobiles are replaced by a continuous mean density and sums are
replaced by integrals. The data demands are replaced by the mean demand, . The objective (\ref{eqn_cts_overall_objective}) can be interpreted as
the fraction of time that the network must be active (i.e. transmitting) in order that the traffic
as determined by  can be met. The problem itself is one in the calculus of variations,
over functions  and non-zero only in .  can be taken
as the maximum in (\ref{eqn_cts_overall_dataconstraint}). Since the space of such functions is
compact in , and the map leading to the objective is continuous, the minimum is achieved.

Before we present a characterization of the solution to (\ref{eqn_cts_overall_objective})-\eqref{eqn_cts_overall_dataconstraint}, let us consider further the interpretation in terms of
a continuous mean density of demand for data from the network. If pico  carries all the (mean) data demand from region  then it needs time . Let  be the maximum such time,
and let  be the pico BS index that requires the maximum such time over all the pico BSs. If pico time  is available, and  then the only demand on the macro BS
will come from the region . Typically, however, such values of  are excessive, and a smaller value in (\ref{eqn_cts_overall_objective}) can be
obtained, as in the optimal solution characterized in Theorem~\ref{thm_ctsLPsoln}.
To state the theorem define .

\begin{theorem}
\label{thm_ctsLPsoln}
Suppose  is an optimal choice for \eqref{eqn_cts_overall_objective}-\eqref{eqn_cts_overall_dataconstraint} and that the optimum value is .
Then there exist  such that
{\small

}
where , and
{\small

}
and

\end{theorem}

{\bf Proof}: See Appendix~\ref{appendix_main_proof}.

As might be expected, the optimal solution takes the form of a rate ratio rule, as it did with the discrete LP.
Also the optimal  is unique up to a set of measure 0.
This is immaterial as far as determining the capacity is concerned. Finally
Theorem~\ref{thm_ctsLPsoln} only states that suitable rate ratio thresholds can be found but does not
show how to determine them. It is desirable to obtain numerical solutions to (\ref{eqn_cts_overall_objective})-\eqref{eqn_cts_overall_dataconstraint}, so we now obtain additional results which characterise the solution further.

Given  and an index, , denote by  the optimal solution to the
subproblem,

It is readily seen that  is decreasing and convex in  and that . Define

for , where  is understood. By definition
 and . By continuity of the integral,

and, moreover,  is strictly decreasing, although not necessarily continuous.

Given  for suitable ,

where , by the mean value theorem. It follows that
the left derivative satisfies  and similarly for the right
derivative, .

We thus see that  is strictly convex in  over the interval  because 
is strictly decreasing. Hence so is the
function  which therefore has a unique optimum at .
It follows that the edge condition

must hold and with the reverse inequality on replacing  with . Equality holds in \eqref{eqn_edge} if 
is differentiable at . It follows that in case ,  (no pico time) is optimal and in case ,  (all pico time) is optimal. Otherwise the solution
satisfies .

Numerical results can be obtained by first checking the edge condition (\ref{eqn_edge}) for extreme choices
of . In case  the optimum may be
found by determining . This in turn can be evaluated once  is obtained, which
can be done by a one dimensional search and numerical integration.


\section{Capacity Results}
\label{sec_capres}
\subsection{The Achievability Result}
\label{sec_achievable}
In this section, we assume that the solution, , to \eqref{eqn_cts_overall_objective}-\eqref{eqn_cts_overall_dataconstraint} satisfies . We will now propose a scheduling policy that we will demonstrate is stable. In fact, the discussion in Section \ref{sec_ctsLP} suggests the following approach to scheduling: A rate ratio policy is defined via
a vector . It works by assigning pico time according to , that is mobiles in
pico cell  at location  are transmitted only using pico time if
,
all other mobiles are transmitted using only macro cell time. In the following we choose
. To each timeslot allocate a fraction
 to the pico cells and the remaining time to the macro cell. We will suppose that time is
divided continuously during the slot so that  seconds into the slot  has been allocated to
the picos.

Files are transmitted at a rate which is reduced by the time sharing fractions as stated above. This is for all  servers, the  picos and the macro cell. For the purposes
of analysis we further suppose that the multiple mobiles, which can arrive between one slot and the next,
have their files merged in order of arrival, to be counted as a single job to be served. We assume that the server allocates its time to these jobs in first come first served (FCFS) order.
Clearly the sequence of merged job service
times  (in seconds) at the start of each slot are independent and identically distributed ({{\it i.i.d.}) by construction. There is a positive probability of a job requiring zero time (corresponding to no arrivals of files during the previous slot).

There is one job arrival at the beginning of each time slot, and the service time is generally distributed, hence each queue is D/G/1. To compute the mean job duration, we note that each individual file has a random length, with average value  bits. The number of arrivals to the pico-cell  queue in a slot is Poisson with mean , where  is the length of the slot (in seconds) and  is the probability of the file being allocated to pico BS , namely
{\small

}
The average time each individual file requires from the pico BS  is
{\small

}
Thus, the workload, measured in seconds of work per second is,
{\small

}
the last inequality from \eqref{eqn_fstarcon}. The service rate is  seconds per second, so the utilization is at most .


The number of arrivals to the macro-cell queue in a slot is Poisson with mean . The average time each individual file requires from the pico BS  is
{\small

}
where
{\small

}
Thus, the workload at the macro-cell queue, measured in seconds of work per second, is
{\small

}
But the RHS of \eqref{eqn_expected_service_macro} is no more than , by \eqref{eqn_tstar}. The service rate is  seconds per second, so the utilization of the macro-cell queue is . 

As far as delay is concerned, this can be broken down to i) time before entering queue, ii) queueing time, iii)
time to clear earlier files in the merged job. The mean delay for i) is at most  and for iii)
no more than . As far as ii) is concerned we have just shown the utilisation for picos and macros is no more than .
The well known theory of the  queue, see \cite{Spitzer} shows that the sequence of waiting times  has a limiting distribution with finite expectation provided that the sum on the RHS of
{\small

}
is finite, where . This is
readily shown to be the case. This is because the mean centred version of  has
finite moments as a consequence of our assumptions. 

From the above we may deduce the following theorem,
\begin{theorem}
\label{theorem_fwd}
If the solution to the continuous LP, satisfies  determined via (\ref{eqn_cts_overall_objective})-\eqref{eqn_cts_overall_dataconstraint}
then there is a policy   and a constant  such that the expected number
 of mobiles still in the network at slot  satisfies

and  is the time sharing scheduler as determined by  and .
\end{theorem}

Of course, scheduling file downloads in a FCFS fashion as described above, is far from practical. A more
practical scheme is to transmit files simultaneously by sharing the available bandwidth evenly amongst
the waiting users. Moreover since the ABS slots  are very small compared with times to transmit
files we can treat these times as infinitesimal and model the transmission delays using the
M/G/1/PS model; see Section~\ref{sec-PS}. This model will provide representative, if not ideal, performance results for actual networks and results are presented in Section \ref{sec_numer}.


\subsection{Converse}
\label{sec_converse}

In this section we will show that if , the infimum of the solutions to (\ref{eqn_cts_overall_objective}), is strictly greater than 1
then for any clearing schedule  the total residual transmission time
increases to infinity. To be more specific, given a scheduling policy 
let  be the (random) total time that transmission of any of the
arrivals during  is taking place under schedule . In other
words, the total time the network is actively transmitting the file of at least
one of the users which arrived in . This time is not assumed to be accrued continuously: There can be interspersed services of later arrivals, but we do not count the
service of the later arrivals in .

For fixed  and a given
set of user arrivals,  is upper bounded by the time which would be
taken if user files are transmitted one after the other at rate .
It is lower bounded by the time taken if the files are transmitted sequentially
at rate .

We now state the converse theorem.
\begin{theorem}
\label{thm_main}
If  then there is a constant  such that for any clearing schedule 
it holds that

almost surely.
\end{theorem}

Hence after time  the residual work (time to clear
the remaining users) is at least  for all  sufficiently large. The implication is that with probability 1 and for all 
sufficiently large there must be at least

users present at time . This is because no clearing schedule transmits files at a rate smaller than  at any time. Thus any clearing schedule must have at least  bits to be transmitted and as there are at most  bits per file there must be at least as many users still in the network as expressed in the RHS of (\ref{eqn_Ntogo}).



The idea of the proof is that first, for any realisation over
 slots we can never use less time than the solution to the corresponding discrete
LP, see \eqref{eqn_overall_objective}. Second it will be shown that if the optimal
solution to (\ref{eqn_cts_overall_objective}) is

then there is a constant  such that

almost surely. This implies that the residual amount of work grows
linearly over time. \\
{\bf Proof of Theorem \ref{thm_main}} \\
Assume that , and let  be the number of file requests which arrive in the interval .
Set  and  being set similarly.
Define  which is clearly a well defined random variable.
The optimal solution to the LP is characterized by rate-ratio thresholds, , and which are random variables (proof omitted).


 lower bounds the time needed to clear the users who arrive in  under any clearing schedule:
\begin{lemma}
\label{lem_sample}
For all sample paths  and for any clearing schedule ,

\end{lemma}
{\bf Proof} \\
Recall that  is the number of file requests which arrive in the interval  and we have indexed the users in order of arrival,
. Let  denote the number of bits sent by the th users pico cell (if any) and the
bits sent by the macro cell under clearing policy . Since  is a clearing schedule,  it must be the case that,

Let  be the total amount of
time during which at least one of the  mobiles is getting its file from the pico BS.
It follows that,

for each . Finally by definition,

From (\ref{eqn_filepi}),(\ref{eqn_picopi}) it can be seen that the LP constraints \eqref{eqn_overall_dataconstraint}-\eqref{eqn_overall_positivity_constraint} are satisfied. Therefore
, for any sample path.  We thus obtain
(\ref{eqn_liminf}) which completes the proof. \qed.

To proceed, consider an alternative scenario in which files, instead of arriving according to a Poisson process during the interval , are all present in the system from time zero. Subsequent files are discarded.  is then the minimum amount of time required to service these users. In this LP assignment, the pico BSs are allocated time , and the macro is allocated time , and the users are assigned according to the rate-ratio thresholds as obtained from the solution to \eqref{eqn_overall_objective}. We would like to show \eqref{eqn_LPbuildsup} directly, but it is easier to first analyze a simpler, static approach.


Consider a simpler static system in which fixed rate-ratio thresholds are used for the assignment of user to BS. To this end, let  be a fixed vector of rate-ratio thresholds. Define

and set . Additionally define,

Lemma~\ref{lem_LLN} below shows that the law of large numbers applies to these quantities, with the following deterministic limits

and . Also define,

where


\begin{lemma}
\label{lem_LLN}
The following limits hold almost surely, 
\end{lemma}
{\bf Proof}: See Appendix~\ref{app_lem_LLN}.

Define   to be the total time used. Since the  in \eqref{eqn_yell_rhoa} (and the corresponding ) are feasible for \eqref{eqn_cts_overall_objective}-\eqref{eqn_cts_overall_dataconstraint},  it follows from the above lemma
that there exists  such that 


Any vector  can be considered as an assignment rule, since it indicates to which BS the arrival will go for service. We now define a discrete set  of such assignment rules. For a fixed integer , to be taken large, define the constants , , by

and define

so the interval  is divided into  equal subintervals, , and  is the set of left endpoints of those subintervals. Define  to be the set of  vectors given by


The next step is to approximate the rate-ratio vector selected by the LP assignment, , with a vector that is close to it in the set . For each , let  be the interval containing , with corresponding left endpoint . Define the vector  by

Note that  is a random vector, taking values in the set .


Now let  be the assignment rule. This is the same as under the LP allocation, except for users in the interval in which the LP rate-ratio threshold lies. Lemma~\ref{lem_LDR} shows that for sufficiently large , the number of users in each of these intervals is upper bounded by .

\begin{lemma}
\label{lem_LDR}
Let  be the number of arrivals into  during in slots 
whose rate ratio falls into . Then for any  there is a fixed set of intervals, with  sufficiently large, and a corresponding random variable , such that for , the following holds, for all  and :

\end{lemma}
{\bf Proof} The proof of Lemma \ref{lem_LDR} is given in Appendix \ref{app_lemLDR}. \qed

Since the minimum rate is , the time to service the users in each of these intervals is upper bounded by . It follows that for ,
{\small

}
Thus,
{\small
}
the last inequality from \eqref{eqn_Vaeta}. Thus, if we take , we obtain
{\small
}
The proof of the Theorem concludes by noting that
{\small
}
holds for {\it any} clearing schedule , due to Lemma~\ref{lem_sample}, and hence the workload must build linearly over time for any clearing schedule.
\qed


\section{Delays under Processor Sharing}
\label{sec-PS}
The proof of Theorem~\ref{theorem_fwd} only requires the consideration of a simple scheduler based on first come, first served (FCFS) processing of jobs in the queues. Theorem~\ref{thm_main} shows that this scheduler is good enough, as far as achieving stability is concerned. However, it is well known that Processor Sharing will provide much better delay performance than FCFS. Since we are only interested in the stable case in this section, we assume that .

In Processor Sharing, all customers in any queue (macro or pico cell) get service from the BS at all times; the service rate is split equally amongst all customers in the queue. Such systems are examples of the symmetric queues considered in Section 3.3 of Kelly~\cite{Kelly79}; in particular, they are server-sharing queues. In \cite{Kelly79} the arrivals to each queue are independent Poisson processes. We obtain Processor Sharing in the limit as we reduce the slot duration to zero, and allocate an equal time share between the jobs (files) in the system. The arrival processes are independent Poisson processes in the limit as the slot duration goes to zero.

We ignore queues which are underloaded, and which typically have much smaller delays, and instead focus on the ``bottleneck'' queues, which are the ones for which there is equality in \eqref{eqn_fstarcon} and . The macro cell queue is also a bottleneck queue. The utilization of the bottleneck servers does not depend on the service discipline, and their utilization is .

It follows from Theorem~3.10 in \cite{Kelly79} that the number of customers queued in each bottleneck queue is Geometric, with parameter , and this result is insensitive to the distribution of the service time of each customer, depending only on the mean service time. It follows that each bottleneck queue size is Geometric with mean . The mean delay in queue  can be computed from Little's law, and is given by

where  is any index in  corresponding to a bottleneck queue. It can be seen that the delays in each bottleneck queue are not the same, unless the arrival rates are the same. In practice, average delays could be reduced further by equalizing delays rather than equalizing utilization.

Lower delays can be achieved if the time-share between macro and picos is allowed to adapt to the traffic. Even better, the rate-ratio thresholds can be adaptive to the traffic, instead of being fixed constants. Our results show that such approaches cannot increase capacity, but one can expect the delay performance to be improved, possibly dramatically in some cases.

\section{On-off scheduling}
\label{sec_genschedule}
Our results heretofore are based on the assumption that  the physical bit rates, , depend only on the location, , of the mobile in the network. This assumption is somewhat restrictive and so in this section we
show how the notion of capacity extends to the case where the physical rates are dependent on network state.
(Note, however, that our prior assumptions are valid for networks with regions  which are sufficiently geographically separated so that between pico cell interference is negligible and are also valid in a network with non-negligible inter-pico interference, but in which the pico BSs transmit with constant power and never go silent. These determine lower/upper bounds for capacity and upper bounds for delay in arbitrary networks which may be adequate for some purposes.)


In what follows we will restrict to the assumption that rates at a particular location depend only on the on/off state of
the picos (these are all off if the macro is on). Hence define  to be
be all subsets of  with at least two members. Clearly the empty set is redundant, and the singleton sets
can be subsumed as macro rates by redefining the macro rate of a mobile to be the maximum of its macro rate and its
pico rate when scheduled by itself. Hence if  the network states are .
Denote a given on-off state by , then for each pico cell  and each location there is a corresponding
pico rate . ABS times are now periods devoted to the various
network states and we may suppose that a period  is made available for operation of
the network in state . 


We may now revisit the set up that we had earlier in much the same way as before. First
the discrete LP for a given set of users file lengths etc. can be solved along similar
lines to the one presented earlier. We leave the reader to check this is the case.
Second a continuous version of the LP can be obtained by replacing summation with
integration over a non-homogeneous Poisson mean measure, as was done earlier. At this point,
two issues arise. First, are there corresponding multipliers generalizing the  used
earlier? and second, does the optimal solution determine capacity as it did before? The answer to both
questions is yes, and can be demonstrated along similar lines to those detailed earlier. We therefore confine ourselves to setting up the LP, and identifying the optimal structure.




To set up the continuous LP define,  to be the number of bits transmitted to mobiles
at location  in cell , when the pico on state is . Let  denote the fraction of
time per unit slot allocated to on state . Also let  and define  to be the bits transmitted by the macro cell for mobiles in pico cell  so that
 where  is mean file size as before. The continuous LP is,


We now characterize the optimal solution for (\ref{eqn_genctsobj}) in the following theorem.
\begin{theorem}
\label{thm_ctsLPsoln_modes}
Let  denote the optimal value, and  the optimizer of (\ref{eqn_genctsobj}). Then there
are Lagrange multipliers  such
that the optimal  satisfies

where 
and if  then

Moreover, for all  such that  and , then \eqref{eqn_ctscon} holds with equality.
\end{theorem}
{\bf Proof}: See Appendix \ref{app_ctsLPsoln_modes}.

It is helpful to interpret the form of the solution in the above theorem.  First the Lagrange multipliers  can be interpreted as the rate of exchange of macro time for time used when operating in state . Since
necessarily   these quantities can thus be thought of as a reuse gain. Also it is
the rate ratios with the macro cell which are in the numerator and if the maximum over all states is
smaller than 1, the mobile is assigned to the macro cell (although this may be by construction actually
time with just the given pico cell on.

The final statement in Theorem~\ref{thm_ctsLPsoln_modes} means that that all picos in all modes saturate together, as capacity is approached.



\section{Numerical Results}

\label{sec_numer}
We consider a circular macrocell, with three pico BSs (), as illustrated
in Figure~\ref{fig_scenario}. The macrocell has radius  km, and the hotspots (pico regions),  each have radius  m. The macro BS is located at the origin,
and the three pico BSs are located at coordinates  , respectively, with distances measured in metres.

\begin{figure}
\centering
\includegraphics[width=2.5in]{Exp1.png}
\caption{Macro-cell containing  Pico Cells}
\label{fig_scenario}
\end{figure}

File requests arrive as a Poisson process of rate  arrivals/sec, and they go in hotspots  with probabilities , respectively. With probability  the arrival selects the macro-only region . File requests falling in a hot spot are distributed uniformly in an
annulus outer radius  and inner radius  File requests falling in  are distributed uniformly in the allowable region, which excludes the pico-regions, and excludes the area within ~m of the macro BS. The mean file size to be transferred was fixed at
 .

The macro BS transmits with power  with an additional antenna gain of  dBi. The macro BS to user pathloss model is
 dB, where  is in metres from the macro BS. The pico BS transmits with power  with an additional antenna gain of  dBi. The pico BS to user pathloss model is  dB, where  is in metres from the pico BS.  The receiver noise at the mobiles is assumed to be  dBm. This model specifies the SNR that can be achieved at any location in the macrocell, from any BS.

We consider two interference cases below. In the no-interference scenario, we base the achieveable rate on Shannon's formula for the AWGN channel,  bits/sec. In the alternative, we take into account interference from other pico BSs, when a mobile is receiving from a pico BS. In this case, the SINR is calculated, using the given pico BS path-loss model for the  interfering signals, and then using  bits/sec to calculate the achievable rate in bits/sec. Interference only applies for pico links, the SNR is used for the macro links in both scenarios. The bandwidth,  is taken to be  MHz.


In the following results, all relevant integrals were estimated using Monte-Carlo simulation.

Figure \ref{fig_rhof} plots the three rate-ratio threshold functions, , described in \eqref{eqn_defrho}, for the no interference scenario. The figure illustrates the fact that these functions are decreasing. Figure~\ref{fig_tau} illustrates the fact that the minimum of  coincides with the unique solution to the edge-rate condition \eqref{eqn_edge}. In these scenarios, the function  is differentiable, so there is equality in~\eqref{eqn_edge}.

 \begin{figure}[t]
    \centering
    \begin{floatrow}
      \ffigbox[\FBwidth]{\caption{Fraction ABS Time  versus Rate Ratio   Pico Cells}\label{fig_rhof}}{\includegraphics[width=3in]{FvsRHO.png}
      }
      \ffigbox[\FBwidth]{\caption{Total time  and Edge Condition Trade off as a Function of }\label{fig_tau}}{\includegraphics[width=3in]{CparePicoInterf.png}
      }
    \end{floatrow}
  \end{figure}



We should expect that the case with interference should give a more accurate account of the real system, but since the pico cells are well separated geographically, we also provide results excluding inter-pico interference to see if this interference has significant impact. As discussed earlier, no-interference is a ``best-case'' assumption and  the interference assumption is actually ``worst-case.'' A more realistic model will have its performance bounded by these two cases.

It can be seen from Figure~\ref{fig_tau} that the optimal  is not very sensitive (in this case) to whether or not there is interference between the pico cells. However, Figure~\ref{fig_feasiblef}, indicates that the capacity is more sensitive to the interference. A lower bound of  arrivals/sec is provided by the interference case. 

Since it may be difficult to adapt the parameter  (the ABS slot size) on a fast time-scale, it is of interest to measure the sensitivity of the range of feasible , as a function of the arrival rate. Figure \ref{fig_feasiblef} depicts the range of ABS values which, when fixed, would allow stable operation,
against arrival rate . As can be seen, there is only significant sensitivity when the arrival rate is close to capacity.


Under the processor sharing model we also obtained the following results for mean time to send
( transmission time + queueing delays), under no interference. These results depend on the file size only through its
mean. The results for the optimal case are shown in the midlle graph of figure \ref{fig_meandelay} for various overall arrival rates and were obtained using Little's law. The graph to the left shows the same results if the ABS value is taken to be  and the right graph with .
The graphs show that the mean time to send is sensitive to the choice of ; compare for example the arrival rates
to attain a mean time to send of 4 seconds in the 3 graphs.

Finally, Figure~\ref{fig_meandelay} illustrates a case in which pico cell  is not saturated at the optimal solution (there is strict inequality in \eqref{eqn_cts_overall_dataconstraint} for pico cell ). It follows that capacity can be enhanced if we allow the pico cells to be switched on and off, as in Section~\ref{sec_genschedule}.

 \begin{figure}[t]
    \centering
    \begin{floatrow}
      \ffigbox[\FBwidth]{\caption{Feasible ABS periods against arrival rate }\label{fig_feasiblef}}{\includegraphics[width=3in]{FeasibleF.png}
      }
      \ffigbox[\FBwidth]{\caption{Fraction ABS Time  versus Rate Ratio   Pico Cells}\label{fig_meandelay}}{\includegraphics[width=3in]{DelaySenstvtyF.png}
      }
    \end{floatrow}
\end{figure}



\section{Conclusions}
\label{sec_conc}
This paper has presented a definition of capacity for the down link of a HetNet in terms of the maximum supportable traffic. It can be obtained once the traffic density (probability of arrival at a given location)
as well as the  macro and pico physical rates are given as well as other traffic parameters. The
capacity limit applies irrespective of how the HetNet is scheduled and is subject only to the constraint
that mobiles are cleared from the network.  It is a natural generalization of a criteria to
clear a static network in minimum time.

The capacity can be evaluated as the numerical
solution to a continuous linear program which can be solved efficiently as has been described. Thus
the results of this paper can be used for capacity planning of future networks. Representative
results for transfer delays, numbers of active mobiles and so on, can also be obtained as has
been shown. These results are optimistic in that they suppose the offered traffic given.
They do not take into account the potentially significant gains in transfer delay which an
adaptive scheduler might achieve however. This is a topic for further investigation.

Finally, the characterization of capacity extends to networks with controls such as on-off scheduling.
As indicated an analogous continuous LP applies in such cases.


\appendices







\section{Proof of Theorem \ref{thm_ctsLPsoln}}
\label{appendix_main_proof}
Given any  we will obtain  so that the Lagrangian is minimized
for the  as stated in the theorem. For each  form the convex dual function,
{\small

}
which is non-negative and convex. On differentiating under the integral sign, we find that
{\small

}
which is continuous. Define  to be a minimizer if   in which case
{\small

}
otherwise choose . In either case the corresponding  is feasible.

Now consider maximizing the Lagrangian with the above choice of  as multipliers,
{\small

}
The optimal choice of  whenever the expression in brackets is positive, which corresponds
to the choice stated in the theorem and feasible as we have already observed. It follows by the Lagrange Sufficiency Theorem, \cite{Whittle71} that  is the minimizer for arbitrary . Since this includes the optimal
 the theorem is proved. 
\section{Proof of Lemma~\ref{lem_LLN}}
\label{app_lem_LLN}

For  and , define ,  by

For , , and , define


Let  be the empirical point measure on  defined by the points , . Then

By independence of  and , we have

see \cite{Billingsley68}.

Since  and  are non-negative, bounded and measurable functions on , the strong law of large numbers implies

almost surely. But  almost surely by continuity of .

Similar arguments show that



\section{Proof of Lemma \ref{lem_LDR}}
\label{app_lemLDR}
Define 
to be the expected number of pico  mobiles arriving per unit time with rate ratios falling
in .
Given  choose a set of intervals  with  sufficiently
large so that  for each interval  and pico .

Now for  any fixed  there exists 

This  follows from standard large deviation arguments applied
to Poisson variates. Given  define  to be the union of the above events.

From the union bound,

for some .  It follows from the first Borel-Cantelli lemma
that for any given set of intervals the event  will occur
finitely many times with probability 1, there being a last slot
 almost surely. We obtain the result on choosing a fixed . \qed



\section{Proof of Theorem \ref{thm_ctsLPsoln_modes}}
\label{app_ctsLPsoln_modes}
Let us consider the subproblem in which the s are given and each pico  minimizes its macro time independently of the others. We drop the pico cell index and restrict all discussion to on-states for which the given cell is active. Consider the dual optimization problem,

and for which any solution must satisfy .
Clearly it is optimal to take . If 
it can be deleted from the problem by taking .

Observe that  (\ref{eqn_ctsdual}) is convex and continuously differentiable in . If the optimum is at an interior point, then differentiating we obtain that

Suppose that there is a  for which . Then it follows that  and since
this is the maximum value which can be taken it follows that . By taking derivatives on
the right,

which shows that no macro time is needed. Additional arguments show that the optimum cannot occur at 
since this implies , which contradicts our assumption that .

In the primal problem we may take the above solution  as multipliers and form the Lagrangian,

and set  only on  and  otherwise. Thus  multiplies the largest
coefficient  over  and the
Lagrangian (\ref{eqn_primal}) is maximal. Since the proposed s are feasible they are optimal
by the Lagrange sufficiency Theorem \cite{Whittle71}.

Now consider the overall optimization problem where there are  pico cells so that the values in the vector  are
the same for all pico cells. Since the objective is convex and therefore continuous, the optimal choice  must lie in a compact set and we suppose that the optimal  is given. We take as Lagrange multipliers the corresponding
 which we have shown to exist. For  such that ,
 can be taken to be .

Now denote the macro time for any given  by  If  then by standard Lagrangian theory,

Hence on differentiating  at , it follows that
 with equality if .


For the final statement, suppose that the constraint \eqref{eqn_ctscon} holds with strict inequality for some , in the optimal solution, and let . There must be at least one other  for which \eqref{eqn_ctscon} is tight, or else  can be reduced (contradiction). But moving time from  to  improves the data rates of users in picos in , because interference for them is less in  than in . A small amount of exchanged time will not violate \eqref{eqn_ctscon} for . But since rates have increased,  can be reduced further, contradicting the optimality of .



This completes the proof of the theorem. \qed
\bibliographystyle{IEEETran}
\begin{thebibliography}{10}
\providecommand{\url}[1]{#1}
\csname url@samestyle\endcsname
\providecommand{\newblock}{\relax}
\providecommand{\bibinfo}[2]{#2}
\providecommand{\BIBentrySTDinterwordspacing}{\spaceskip=0pt\relax}
\providecommand{\BIBentryALTinterwordstretchfactor}{4}
\providecommand{\BIBentryALTinterwordspacing}{\spaceskip=\fontdimen2\font plus
\BIBentryALTinterwordstretchfactor\fontdimen3\font minus
  \fontdimen4\font\relax}
\providecommand{\BIBforeignlanguage}[2]{{\expandafter\ifx\csname l@#1\endcsname\relax
\typeout{** WARNING: IEEEtran.bst: No hyphenation pattern has been}\typeout{** loaded for the language `#1'. Using the pattern for}\typeout{** the default language instead.}\else
\language=\csname l@#1\endcsname
\fi
#2}}
\providecommand{\BIBdecl}{\relax}
\BIBdecl

\bibitem{standards}
R1-094225, ``Dl performance with hotzone cells,'' in \emph{3GPP TSG-RAN WG1
  Meeting \# 58bis}.\hskip 1em plus 0.5em minus 0.4em\relax Qualcomm Europe,
  2009.

\bibitem{Andrews2012}
J.~G. Andrews, H.~Claussen, M.~Dohler, S.~Rangan, and M.~C. Reed, ``Femtocells:
  Past, present, and future,'' \emph{Selected Areas in Communications, IEEE
  Journal on}, vol.~30, no.~3, pp. 497--508, 2012.

\bibitem{Chen2011}
C.~S. Chen, F.~Baccelli, and L.~Roullet, ``Joint optimization of radio
  resources in small and macro cell networks,'' in \emph{Vehicular Technology
  Conference (VTC Spring), 2011 IEEE 73rd}.\hskip 1em plus 0.5em minus
  0.4em\relax IEEE, 2011, pp. 1--5.

\bibitem{Hu2011}
R.~Q. Hu, Y.~Qian, S.~Kota, and G.~Giambene, ``Hetnets-a new paradigm for
  increasing cellular capacity and coverage [guest editorial],'' \emph{Wireless
  Communications, IEEE}, vol.~18, no.~3, pp. 8--9, 2011.

\bibitem{Rudolf2012}
S.~Corroy, L.~Falconetti, and R.~Mathar, ``Dynamic cell association for
  downlink sum rate maximization in multi-cell heterogeneous networks,'' in
  \emph{Communications (ICC), 2012 IEEE International Conference on}.\hskip 1em
  plus 0.5em minus 0.4em\relax IEEE, 2012, pp. 2457--2461.

\bibitem{VTC2013}
S.~Borst, S.~Hanly, and P.~Whiting, ``Optimal resource allocation in hetnets,''
  in \emph{Communications (ICC), 2013 IEEE International Conference on}.\hskip
  1em plus 0.5em minus 0.4em\relax IEEE, 2013, pp. 5437--5441.

\bibitem{ICC2013}
------, ``Optimal resource allocation in hetnets,'' in \emph{Communications
  (ICC), 2013 IEEE International Conference on}.\hskip 1em plus 0.5em minus
  0.4em\relax IEEE, 2013, pp. 5437--5441.

\bibitem{Spitzer}
F.~Spitzer, ``A combinatorial lemma and its application to probability
  theory,'' \emph{Trans. Amer. Math. Soc}, vol.~82, no.~2, pp. 323--339, 1956.

\bibitem{Kelly79}
F.~Kelly, \emph{Reversibility and Stochastic Networks}.\hskip 1em plus 0.5em
  minus 0.4em\relax Wiley Series in Probability and Mathematical Statistics,
  1979.

\bibitem{Whittle71}
P.~Whittle, \emph{Optimization under constraints: theory and applications of
  nonlinear programming}.\hskip 1em plus 0.5em minus 0.4em\relax
  Wiley-Interscience, 1971.

\bibitem{Billingsley68}
P.~Billingsley, \emph{Convergence of probability measures}.\hskip 1em plus
  0.5em minus 0.4em\relax New York, John Wiley \& Sons, 1968.

\end{thebibliography}



\end{document}
