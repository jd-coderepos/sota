\documentclass{llncs}
\usepackage{graphicx}
\usepackage{amsmath}
\usepackage{amssymb}
\usepackage{amsmath}
\usepackage{amssymb}
\usepackage{algorithm}

\newcommand{\be}{}
\newcommand{\nin}{\noindent}
\newcommand{\pF}{{\Pr}_{func}}
\newcommand{\pf}{{\Pr}_{perm}}
\newcommand{\QED}{\begin{flushright}  \end{flushright}}
\pagestyle{plain}
\newcommand{\half}{\frac{1}{2}}

\numberwithin{equation}{section}
\numberwithin{lemma}{section}
\numberwithin{proposition}{section}

\textheight 220mm       \topmargin -7mm         
\begin{document}

\title{How many queries are needed to distinguish a truncated random permutation from a random function?}

\author{Shoni Gilboa \inst {1}, Shay Gueron \inst {2, 3} \and Ben Morris \inst{4}}
\institute{
The Open University of Israel, Raanana 43107 , Israel
\and
University of Haifa, Haifa 31905, Israel,   Intel Corporation
\newline
 UC Davis
}

\maketitle
\centerline{ \today }
 \begin{abstract}

An oracle chooses a function  from the set of  bits strings to itself, which is either a randomly chosen permutation or a randomly chosen function. When queried by an -bit string , the oracle computes , truncates the  last bits, and returns only the first  bits of . How many queries does a querying adversary need to submit in order to distinguish the truncated permutation from the (truncated) function? \\
In 1998, Hall et al. \cite{Hall} showed an algorithm for determining (with high probability) whether or not  is a permutation, using  queries. They also showed that if , a smaller number of queries will not suffice. For , their method gives a weaker bound.
In this note, we first show how a modification of the approximation method used by Hall et al. can solve the problem completely. It extends the result to practically any , showing that  queries are needed to get a non-negligible distinguishing advantage. However, more surprisingly, a better bound for the distinguishing advantage can be obtained from a result of Stam \cite{Stam} published, in a different context, already in 1978. We also show that, at least in some cases, the bound in \cite{Stam} is tight. 
\end{abstract}

{\small
\begin{quote}
\textbf{Keywords:} Pseudo random permutations, pseudo random functions, advantage.
\end{quote}}

\section{Introduction}

Distinguishing a randomly chosen permutation from a random function is a combinatorial problem which is fundamental in cryptology. A few examples where this problem plays an important role are the security analysis of block ciphers, hash and MAC schemes.
 
One formulation of this problem is the following. An oracle chooses a function , which is either a randomly (uniformly) chosen permutation of , or a randomly (uniformly) chosen function from  to . An adversary selects a ``querying and guessing'' algorithm. He first uses it to submit  (adaptive) queries to the oracle, and the oracle responds with  to the query . After collecting the  responses, the adversary uses his algorithm to guess whether or not  is a permutation.  The quality of such an algorithm (in the cryptographic context) is the ability to distinguish between the two cases (rather than successfully guessing which one it is). It is measured by the difference between the probability that the algorithm outputs a certain answer, given that the oracle chose a permutation, and the probability that the algorithm outputs the same answer, given that the oracle chose a function. This difference is called the "advantage" of the algorithm. We are interested in estimating \emph{Adv}, which is the maximal advantage of the adversary, over all possible algorithms, as a function of a budget of  queries. 

The well known answer to this problem is based on the simple ``collision test'' and the Birthday Problem:

Since for every 

we get, for , that

This result implies that the number of queries required to distinguish a random permutation from a random function, with success probability significantly larger than, say, , is .
We now consider the following generalization of this problem: \\


\nin {\bf Problem 1. [Distinguishing a truncated permutation]}
Let  be integers. An oracle chooses . If , it picks a permutation  of  uniformly at random, and if , it picks a function  uniformly at random. 
An adversary is allowed to submit queries  to the oracle. The oracle computes  (if ) or  (if ), truncates (with no loss of generality) the last  bits from , and replies with the remaining  bits. The adversary has a budget of  (adaptive) queries, and after exhausting this budget, is expected to guess . 
{\it How many queries does the adversary need in order to gain non-negligible advantage?} \\ Specifically, we seek  as a function of  and .

\section{So, how many queries are really needed?}

\subsubsection{The Birthday bound (folklore):}
We start with remarking that the classical "Birthday" bound  is obviously valid as a bound for the adversary's advantage in Problem 1. 
In fact, any algorithm that the adversary can use with the truncated replies of  bits from  can also be used by the adversary who sees the full  (he can ignore  bits and apply the same algorithm). \newline

Of course, we are looking for a better upper bound that would reflect the fact that the adversary receives less information when  is truncated. 
We have the following bounds for Problem 1.

\subsubsection{Hall et al. \cite{Hall} (1988):}
Problem 1 was studied by Hall et al. \cite{Hall} in 1998. The authors showed an algorithm that gives a non-negligible distinguishing advantage using  queries (for any ). They also proved the following upper bound:


For  the bound in (\ref{Hall_result}) implies that . However, for larger values of , the bound on that is offered by (\ref{Hall_result}) deteriorates, and becomes (already for )worse than the trivial "Birthday" bound . 

Hall et al. \cite{Hall} conjectured that  queries are needed in order to get a non-negligible advantage, in the general case. 

\subsubsection{Bellare and Impagliazzo \cite{BI} (1999):}

Theorem 4.2  in \cite{BI} states that 

whenever . 

This implies that  for .



\subsubsection{Gilboa and Gueron \cite{GG} (2013):}

The method used to show \eqref{Hall_result} can be pushed to prove the conjecture in \cite{Hall} for (almost) every . 
In particular, it can be shown that if  then

and if  then


This implies that 
 for any .


\subsubsection{Stam \cite{Stam} (1978):}
Surprisingly, it turns out that Problem 1 was solved  years before Hall et al. \cite{Hall}, in a different context. The bound

which is valid for all , follows directly from a result of Stam \cite[Theorem 2.3]{Stam}. 
(Note that if  then (\ref{eq:Stam}) can be simplified to 
).

This implies that 
 for any , confirming the conjecture of \cite{Hall} in all generality ( years before the conjecture was raised). 

\begin{remark}
The bound \eqref{eq:Stam} is tighter than all the bounds mentioned above, with one exception: the elementary upper bound \eqref{birthday} is better than \eqref{eq:Stam} for . 
\end{remark}

\section{Different methods give different bounds}

It is interesting to see how different approaches yield different bounds for Problem 1. To this end, we first define some notations.

For fixed  and  denote . We view  as the set of all possible sequences of replies that can be given by the oracle (to the adversary's  queries).

\noindent For any  ,  let


\noindent 
For  and , let 
be the set of sequences of replies that are the same as  up to the -th query.  


For  let  and  be the probabilities that  is the actual sequence of replies that the oracle gives to the adversary's  queries, in the case the oracle chose a random permutation or a random function, respectively.


For , let 


\noindent 
Note that 


\subsection{The proof method of Hall et al.}
The proof of \eqref{Hall_result} uses the general bound 

that holds for any .
It is applied to the set
 \\

The expression 
  is bounded by direct computations. The expression  is bounded by combining the Union Bound and the Chebyshev inequality. Finally,  is chosen to minimize the resulting bounds.


\subsection{The proof method of Gilboa and Gueron}

To get \eqref{thm1} (for ), we can apply the slightly better (than \eqref{Adv_S:Hall}) bound

to the set

Here,  are chosen to minimize the bound.
Again,  is bounded by direct (elaborate) computation, and  is bounded by combining (via the Union Bound) the Chebyshev inequality and the Markov inequality.

The bound \eqref{thm2} (for ) follows similarly by examining the set

for  and  which are chosen to optimize the bound. 

\subsection{The proof method of Bellare and Impagliazzo}

Bellare and Impagliazzo also used \eqref{eq:Adv_S}, for the set  of all  satisfying (for suitable  and ):


\begin{enumerate}
\item For any ,
 
\item  For any ,
 
\item 
 
\end{enumerate}

The expression  is bounded by combining the Azuma inequality and the observation that for any ,


\subsection{The proof method of Stam}

Stam's approach observes that by Pinsker's inequality (1960) \cite{Pinsker} 
\footnote{
The inequality as used in \eqref{Pinsker} was established independently by Csisz\'ar (1967) \cite{Csiszar}, Kemperman (1968) \cite{Kemperman}, and Kullback \cite{Kullback}. Pinsker proved the inequality with a worse constant.} we have


He then uses the decomposition 

direct (exact) computations, and the concavity of the  function. 

\section{Stam's bound is sometimes sharp}\label{sec:Stam_sharp}
In the case  (i.e., the oracle returns only  bit), \eqref{eq:Stam} gives 

In this section we show that this bound is essentially sharp. 

With no loss of generality we may assume  is even and . We define the following adversarial algorithm. 

\begin{itemize}
\item []
{\bf Algorithm 1.} \\ 
Collect the answers (which are, in this case, just bits) of  arbitrary queries. \\ 
Compute the difference  between the number of 's and 's. \\
If  , guess that the oracle was using a truncated random permutation. Otherwise, guess that the oracle was using a random function.
\end{itemize}


The advantage of Algorithm 1 is

We show that 

for any  such that . From this, we can conclude that 


First, note that for .

Since for any 

we get that for any 


Now, using \eqref{eq:b_q/2} and \eqref{eq:p_q/2} we get

The proof of \eqref{eq:b} and \eqref{eq:p} for  is similar.

\section{An open problem}

By combining \eqref{birthday}, \eqref{eq:Stam}, and the trivial bound , we can conclude that the best known bound for Problem 1 is


\noindent
By the lower bound in \eqref{birthday}, we know that the bound in \eqref{ALL} is essentially sharp for . By our proof in Section \ref{sec:Stam_sharp}, we know that the bound in \eqref{ALL} is essentially sharp . The natural question that remains open is whether the bound \eqref{ALL} is essentially sharp for all .

\begin{thebibliography}{999999999}

\bibitem{BI}  M. Bellare, R. Impagliazzo, "A tool for obtaining tighter security analyses of pseudorandom function based constructions, with applications to PRP to PRF conversion", ePrint 1999/024, http://eprint.iacr.org/1999/024 (1999).

\bibitem{GG} S. Gilboa, S. Gueron, "Distinguishing a truncated random permutation from a random function", {\it manuscript} (2013)


\bibitem{Csiszar}
I. Csisz\'ar, Information-type measures of difference of probability distributions and indirect observations, Studia Sci. Math. Hungar. {\bf 2} (1967), 299--318.

\bibitem{Hall}
C. Hall, D. Wagner, J. Kelsey, B. Schneier, Building prfs from prps, in: Proceedings of
CRYPTO-98: Advances in Cryptography, Springer Verlag, 1998, pp. 370-389.

\bibitem{Kemperman}
J. H. B. Kemperman, On the optimum rate of transmitting information, in {\it Probability and Information Theory (Proc. Internat. Sympos., McMaster Univ., Hamilton, Ont., 1968)}, 126--169, Springer, Berlin.

\bibitem{Kullback}
S. Kullback, "A lower bound for discrimination information in terms of variation", IEEE Trans. Information Theory IT-13 (1967), 126--127. 

\bibitem{Pinsker}
M. S. Pinsker, {\it Information and informational stability of random variables and processes} (Russian), Problemy Pereda\v ci Informacii, Vyp. 7, Akad. Nauk SSSR, Moscow, 1960.

\bibitem{Stam}
A. J. Stam, Distance between sampling with and without replacement, Statist. Neerlandica {\bf 32} (1978), no.~2, 81--91. 

\end{thebibliography}

\end{document}
