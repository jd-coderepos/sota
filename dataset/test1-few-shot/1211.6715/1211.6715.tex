\documentclass[11pt]{article}

\usepackage{layout}
\usepackage{amsmath, amsthm}
\usepackage{amssymb, amsrefs}
\usepackage{graphicx}
\usepackage{arydshln}

\newtheorem{theorem}{Theorem}[section]
\newtheorem{lemma}[theorem]{Lemma}
\newtheorem{proposition}[theorem]{Proposition}
\newtheorem{corollary}[theorem]{Corollary}
\newtheorem{definition}[theorem]{Definition}
\newtheorem{ex}[theorem]{Example}

\begin{document}
\pagestyle{headings}


 \title{\sffamily On the Concrete Categories of Graphs}
\author{
    \textsc{George McRae, Demitri Plessas, Liam Rafferty} \0.25em]
    }

\maketitle



\vspace*{-2em}

\begin{abstract}
\noindent
\par{In the standard Category of Graphs, the graphs allow only one edge to be incident to any two vertices, not necessarily distinct, and the graph morphisms must map edges to edges and vertices to vertices while preserving incidence. We refer to these graph morphisms as Strict Morphisms. We relax the condition on the graphs allowing any number of edges to be incident to any two vertices, as well as relaxing the condition on graph morphisms by allowing edges to be mapped to vertices, provided that incidence is still preserved. We call this broader graph category The Category of Conceptual Graphs, and define four other graph categories created by combinations of restrictions of the graph morphisms as well as restrictions on the allowed graphs.}
\par{We investigate which Lawvere axioms for the category of Sets and Functions apply to each of these Categories of Graphs, as well as the other categorial constructions of free objects, projective objects, generators, and their categorial duals.}
\end{abstract}

\tableofcontents

\section{Introduction - Conceptual Graphs and Their Morphisms}
\indent Often the study of morphisms of any mathematical object starts with the study of automorphisms. In graph theory, this study produced representation theorems for groups as automorphism groups of graphs. The study of automorphisms also produced characterizations of graphs (e.g. vertex-transitive graphs and distance-transitive graphs). More recently, finding restrictions of the automorphism set for the graph by considering automorphisms that fix vertex colors or by considering automorphisms that fix certain sets of vertices produced useful new graph parameters (the distinguishing number \cite{Albertson} and fixing number \cite{Courtney} for a graph).\\
\indent The study of morphisms then delves into the study of endomorphisms and finally homomorphisms. In graph theory, a certain class of graph homomorphisms generalize vertex-coloring, and are now being widely studied. In 2004, a textbook was published about these graph homomorphisms \cite{HN2004}.\\
\indent It is natural in the study of morphisms to use category theory, and our goal in this paper is to create the categorial framework to be used in further study of graph homomorphisms. Most of the proofs found in this paper will be combinatorial in flavor and many of the results lend themselves to combinatorial enumeration.\\
\indent The most common category considered in (undirected) graph theory is a category where graphs are defined as having at most one edge incident to any two vertices and at most one loop incident to any vertex. The morphisms are usually described as a pair of functions between the vertex sets and edge sets that respect edge incidence.\\
\indent In this paper, we will be relaxing these restrictions an investigating the concrete categories that are created. The rest of this section is concerned with defining our graphs as well as defining the restrictions. The second section defines six different categories of graphs, and proves that they are indeed categories. The third section investigates which of the six axioms for \textbf{Sets}, the category of sets and functions, apply to each of these categories of graphs. The last section investigates the other categorial constructions of free objects, projective objects, and generators, as well as their duals, and gives complete categorizations of these objects in the six categories of graphs.\\
\indent In our graphs, we want to start out with as great a generality as possible and add restrictions later. This means we want to allow graphs to have multiple edges between any two vertices and multiple loops at any vertex. We will define our graphs in the style of Bondy and Murty \cite{Bondy}, namely, graphs are sets of two kinds of parts: ``edges'' and ``vertices'' together with an ``incidence'' function. 
\begin{definition}
A \emph{conceptual graph}  consists of\\
\rotatebox[origin=c]{90}{} where  is the set of parts of ,  is the set of vertices of , \rotatebox[origin=c]{90}{} is the set of unordered pairs of vertices of ,  is the incidence map from the set of parts to the unordered pairs of vertices,  is the inclusion map of the vertex set into the part set, and for \rotatebox[origin=c]{90}{}, the unordered diagonal map, .
\end{definition}
\indent We define the set of edges of a graph, , to be . Henceforth, we will frequently abbreviate conceptual graph to graph. In our study here, we have no need to restrict our edge sets and vertex sets of our graphs to be finite sets.\\
\indent In \cite{Bondy} a graph does not have the inclusion map, , but such a map will be critical when defining a graph homomorphism. In this way, we can think of the vertex ``part'' of the graph and the edge ``part'' of the graph in the same ``part'' set. We do allow , i.e. , the empty graph, to be considered a graph. However, since  is required to be a function, if  then . We also allow  and  (``no edges''), i.e. .\\
\indent We now note the following. First, we naturally use the topologist's ``boundary'' symbol for incidence. Second, an unordered pair in \rotatebox[origin=c]{90}{} is denoted \textunderscore or \textunderscore, for vertices . Thus the natural unordered diagonal map \rotatebox[origin=c]{90}{} is given by \textunderscore or \textunderscore. Finally, we have chosen to consider our vertex set and edge set to be combined into a ``part'' set. Thus as an abstract data structure our graphs are a pair of sets: a set of parts with a distinguished subset called ``vertices''. This is done to make the description of morphisms more natural, i.e. functions between the ``over'' sets (of parts) that takes the distinguished subset to the other distinguished subset. This is similar to the construction of the Category of Topological Pairs of Spaces: for example, an object  is a topological space  with a subspace  and a morphism  is a continuous function from the topological space  to the topological space  with .\\
\indent We now define our morphisms for conceptual graphs.
\begin{definition}
 is a \emph{graph (homo)morphism} of conceptual graphs from  to  if  is a function  and  that preserves incidence, i.e. \textunderscore whenever \textunderscore, for all  and some .
\end{definition}
\begin{figure}[h]
\centering \includegraphics[scale=.6]{ghom.PNG}
\caption{The Graph Morphism}
\end{figure}
\indent This definition allows a graph homomorphism to map an edge to a vertex as long as the incidence of the edges are preserved. As an edge, , can be mapped to the part set of the co-domain graph, , so that it is the image of a vertex, i.e.  for some .\\
\indent We will now define some specialized classes of graphs and a specialized graph morphism. Our first restriction is a common restriction in graph theory. The set of graphs is restricted to allow only one edge between any two vertices (see \cite{HN2004}), and at most one edge between a vertex and itself (a loop). We call these graphs \textit{simple graphs} and define them in terms of conceptual graphs.
\begin{definition}
A \emph{simple} graph  is a conceptual graph such that for all  with , there is at most one  such that \textunderscore, and for all  there is at most one  such that \textunderscore (where \textunderscore is the unordered pair of vertices  and ).
\end{definition}
Thus, a graph is simple if and only if the incidence map is injective.\\
\indent Another common restriction is to not allow loops at all. This restriction is often required when discussing vertex coloring. We call these graphs \textit{loopless graphs}.
\begin{definition}
A \emph{loopless} graph  is a conceptual graph such that for all vertices  there is no edge  such that \textunderscore.
\end{definition}
\indent This is not the usual notion of a ``simple'' graph often common in graph theory; that notion is the simple and loopless graph by our definition. This is a departure from standard nomenclature, but it fits our categorial discussion best.\\
\indent We now define the most common notion for a graph morphism in literature, we call it a \emph{strict} morphism because it always takes an edge part to a strict edge part (and not just a part, e.g. a vertex). The following definition is a modified form of the definition presented in \cite{HN2004} to apply to conceptual graphs.
\begin{definition}
Let  and  be conceptual graphs. A \emph{strict graph homomorphism} (or \emph{strict morphism})  is a graph morphism  such that the strict edge condition holds: for all edges , , i.e. the image under the strict morphism  of an edge is again and edge. 
\end{definition}
\indent The condition, \textunderscore whenever \textunderscore, assures that the incidence of the edges in  is preserved in  under . Note that the above definition also requires that vertices be mapped to vertices and edges be mapped (strictly) to edges. However, sometimes it may be beneficial to allow edges to be mapped to vertices. Such a morphism would allow a graph to naturally map to the contraction or quotient graph obtained by the contraction of an edge, but this could not be a strict morphism.\\
\indent Now that we have defined our graphs and graph homomorphisms, we are ready to discuss the various Categories of Graphs.


\section{The Categories of Graphs}
\begin{definition} \emph{Concerete Categories} \cite{Mac} are categories whose objects are sets with structure and whose morphisms are functions that preserve that structure.
\end{definition}
\indent We will now define six concrete categories of graphs using the various restrictions of the previous section. We do not include all combinations of restrictions, but instead focus on the combinations of restrictions often seen in literature.
\begin{definition}
The \emph{Category of Conceptual Graphs}, \textbf{Grphs}, is a (concrete) category where the objects are conceptual graphs and the morphisms are graph morphisms.
\end{definition}
\indent Keith Ken Williams \cite{KKWil} proved that the axioms of a category are satisfied by this definition. 
\begin{proposition}
\textbf{Grphs} is a category.
\end{proposition}
\indent This category, \textbf{Grphs}, we will think of as the big ``mother'' category of graphs. We now define five other commonly studied concrete subcategories of \textbf{Grphs}.
\begin{definition}
The \emph{Category of Simple Graphs with Conceptual Morphisms}, \textbf{SiGrphs}, is the (concrete) category where the objects are simple graphs, and the morphisms are conceptual graph morphisms.
\end{definition}
\begin{definition}
The \emph{Category of Simple Loopless Graphs with Conceptual Morphisms}, \textbf{SiLlGrphs}, is the (concrete) category where the objects are simple graphs without loops, and the morphisms are conceptual graph morphisms.
\end{definition}
\begin{definition}
The \emph{Category of Conceptual Graphs with Strict Morphisms}, \textbf{StGrphs}, is the (concrete) category where the objects are conceptual graphs, and the morphisms are strict graph morphisms.
\end{definition}
\begin{definition}
The \emph{Category of Simple Graphs with Strict Morphisms}, \textbf{SiStGrphs}, is the (concrete) category where the objects are simple graphs and the morphisms are strict graph morphisms.
\end{definition}
\indent This last category we defined is most often referred to as the ``category of graphs'' and is the main category of graphs discussed in \cites{HN2004, Dochtermann}, namely, graphs with at most one edge between vertices, at most one loop at a vertex, and all the morphisms are \emph{strict} (i.e. take and edge or loop \emph{strictly} to an edge or loop).
\begin{definition}
The \emph{Category of Simple Loopless Graphs with Strict Morphisms}, \textbf{SiLlStGrphs}, is the (concrete) category where the objects are simple graphs without loops, and the morphisms are strict graph morphisms.
\end{definition}
As the composition of strict morphisms are strict morphisms, and the identity morphism is a strict morphism, these are in fact categories. We now have a containment picture of our six different (concrete) categories of graphs, with our mother category at the top. At the bottom, we have also included as another graph category, the category of sets (and functions), where a set is considered as a graph with no edges and any function is a (strict) morphism of such graphs.
\begin{figure}[h]
\centering \includegraphics[scale=.6]{bigpic2.png}
\caption{The Categories of Graphs}
\end{figure}

\section[Categorial Comparisons - Graphs vs. Sets]{Categorial Comparisons - The Categories of Graphs vs. The Category of Sets}
\subsection{The Lawvere Axioms for \textbf{Sets}}
A natural representation question is, what characterizing properties must an abstract category have in order for it to be \textbf{Sets} (up to functor equivalence of categories)? There are six characterizing conditions for an arbitrary category to be the category \textbf{Sets} called the Lawvere-Tierney Axioms \cites{LT, Lawvere, T}. An arbitrary category is the category \textbf{Sets} if and only if all six of the following conditions are satisfied.\par
\begin{list}{}
\item \textbf{(L1)} \textbf{Sets} has Limits (and Colimits):
\item \textbf{(L2)} \textbf{Sets} has exponentiation with evaluation.
\item \textbf{(L3)} \textbf{Sets} has a subobject classifier.
\item \textbf{(L4)} \textbf{Sets} has a natural number object.
\item \textbf{(L5)} \textbf{Sets} has the Axiom of Choice.
\item \textbf{(L6)} The subobject classifier in \textbf{Sets} is two-valued.
\end{list}

Each of these categorial properties are defined abstractly \cite{Goldblatt}. That is to say, in the definitions only objects and morphisms will be used, not the structure of the objects.\par
We investigate each of our categories of graphs to see which of the axioms for \textbf{Sets} they satisfy and which axioms they do not. We start with \textbf{Grphs}. It has already been shown that \textbf{Grphs} satisfies (L1) \cite{KKWil}, and that \textbf{Grphs} fails to have the (L2) construction \cite{BMS}. We will provide the constructions for (L1) for completeness and provide an alternate proof for the failure of the existence of (L2) that has a combinatorial flavor, shows that an exponentiation object can exist, and shows exponentiation with evaluation fails due to an evaluation morphism that fails to satisfy the universal mapping property (as opposed to \cite{BMS} who show the failure of a necessary adjoint relationship).\par
\subsection{Lawvere-type Axioms for \textbf{Grphs}}
\begin{proposition}
\textbf{Grphs} satisfies axioms (L1), (L3), and (L4) and does not satisfy axioms (L2), (L5), and (L6).
\end{proposition}
\begin{proof}
\textbf{Axiom L1 (``limits'') holds for} \textbf{Grphs}: We note that limits and colimits exist by the existence of a terminal object, products, equalizers, and their duals \cite{Mac}. The one vertex graph  (the classical ``complete graph on one vertex'') is the terminal object, which we will denote , and the empty vertex set (and edge set) graph  is the initial object, which we will denote .\\
\indent For products, let  and  be graphs in \textbf{Grphs}. To keep track of the incidence relation we will need a total ordering (arbitrary but fixed) on the vertex sets. Let  be the total ordering on  and  be the total ordering on . For any element  with \textunderscore and , and any element  with \textunderscore and , there is an element  in  whose incidence is \textunderscore, (hence ), and if  and  then there is an additional element in , , whose incidence is \textunderscore. The projections from the product are  (regardless of the bar,  ) and  (regardless of the bar,  ). So we have a vertex set, a part set, and an incidence relation from the edge set to an unordered product of the vertex set with itself; therefore this product is in fact, a graph. It can easily be shown it satisfies the universal mapping properties for a product.\par
\textbf{Remark}: for our figures with graphs, we provide ``pictures'' (with picture frames) for the graphs. This helps to distinguish the graphs from the morphisms, especially in the case of graphs with multiple components. It also emphasizes that we are often choosing representative graphs from an isomorphism class of graphs.\\
\begin{figure}[h]
\centering \includegraphics[scale=.7]{GrphsPrds.png}
\caption{An example of the product in \textbf{Grphs} with pictures.}
\end{figure}\par
\indent The coproduct of two graphs in \textbf{Grphs} is the disjoint union of the two graphs. This construction satisfies the universal mapping property for a coproduct.\\
\indent Let  be two morphisms in \textbf{Grphs}. We define the equalizer's part set as  and if \textunderscore then  and . We have required that a edge is in the part set of the equalizer implies that its incident vertices are in the part set of the equalizer. So the part set of the equalizer is a subset of  and the vertex set of the equalizer is a subset of . And by construction the incidence function from  can be restricted to the part set of the equalizer since we made sure to only include edges if their incident vertices were in the equalizer. So we have that the equalizer is a graph and that  is just the inclusion morphism. This construction satisfies the universal mapping property for an equalizer.\\
\indent Again, let  be two morphisms in \textbf{Grphs}.
In order to construct the coequalizer, , we will need to mod out  by an equivalence relation. Let , define  if there exists a sequence  such that , , ,....,  and  or  (or switch the 's and 's in the previous statement). \par
When we identify equivalent elements of  we get a set, which we will call , and we will define  as the set function which maps elements to their equivalence class in .\\
\indent We now show  is a graph. First we note that if an edge or vertex is identified with another edge or vertex, then their incident vertices must be identified. This is due to the fact that if  with \textunderscore and \textunderscore by the properties of graph morphisms  and  or  and . Either way we can construct a sequence between the vertices of  and  of their incidences which will be the sequence necessary for the equivalence of their vertices. It is worth noting that some equivalence classes in \textbf{Grphs} have both vertices and edges in them, and if an equivalence class contains any vertices, it is a vertex in the coequalizer graph. Hence by construction  is a graph morphism and . This construction satisfies the universal mapping property for a coequalizer.\\
\indent \textbf{Axiom L2 (``exponentiation with evaluation'') fails for} \textbf{Grphs}: By way of contradiction let us suppose that categorial exponentiation with evaluation (by the universal mapping property of exponentiation with evaluation) exists in \textbf{Grphs}. Then \textbf{Grphs} has a terminal object, products and exponentiation with evaluation. Hence there is an adjoint functor relationship between  and  for all graphs  and . Hence there is a bijection between the set of morphisms  and the set of morphisms .\par
We construct the counterexample to the existence of exponentiation and evaluation in \textbf{Grphs} in two steps. First, we use the adjoint functor relationship to completely determine (by a ``brute force'' count) the vertices, edges, and incidence of  as a graph where , the graph with a single vertex with a loop on the vertex, and , the classical ``complete graph on two vertices''. Second, we show that for all morphisms  that satisfy the commuting morphism equations of the evaluation universal mapping property fails to have the uniqueness requirement for the universal mapping property for exponentiation with evaluation.\par
To begin, we will use the above mentioned adjoint bijection, but for various choices of ``test'' objects . First, for  = terminal object = a single vertex graph, .\par
Any morphism from  to  must send both vertices to the single vertex in , the edge may be sent to either the loop or to the vertex. So there are two maps here. Therefore, there must be two morphisms from  to . Since  is just a single vertex,  must have exactly two vertices.\par
Second, suppose  is a vertex with a single loop. Then  is a graph on two vertices, with a loop at each vertex, and two edges incident to the two distinct vertices.\par
Again, both the vertices of  must be sent to the single vertex of . Now there are four edges in , each edge maybe sent to either the loop or to the vertex (independent of where the other edges are sent). So there are  morphisms here. Therefore there must be exactly  morphisms from  to . There are exactly two morphisms which send both the edge and the vertex of  to a single vertex (since we have already determined that there are only two vertices in ). Which leaves 14 more morphisms to account for. Since the vertex of  must be sent to a vertex, and the loop must be either sent to a loop or a vertex, we conclude that there are  loops distributed between the two vertices (we do not know how they are divided between the two, but we know that there are  of them).\par
Third, suppose  is two vertices with a non-loop edge between them. Then \par
Again, all four vertices of  must be sent to the single vertex of . The six edges of  can be sent to either the loop or to the vertex (independent of where the other ones are sent). So there are  morphisms here.\par
Therefore there must be  morphisms from  to .  has only one edge, it can either be sent to a vertex, a loop, or a non-loop edge. There are two ways to send it to a vertex (and this will force both its vertices to be sent to this vertex to preserve incidence). Since there are  loops, there are  ways to send it to a loop (and since incidence must be preserved both the vertices of  must be sent to the vertex incident on this loop, it is worth noting that we still don't know where these  loops are, but it doesn't matter counting these morphisms). Which leaves us with  morphisms to account for, which must send the edge of  to a non-loop edge of  since we have accounted for the other possibilities. There are only two vertices in  so there is only place to send a non-loop edge. Also, each non-loop edge in  will give us two morphisms from  to that edge (once you decide which of the vertices to send one vertex of  to, the edge must be sent to the  edge and the other vertex of  to the other vertex of ). Therefore there must be precisely  non-loop edges connecting the two vertices of . We still do not know where the  loops are in  but we have a pretty good idea of what it must look like.\par
Now we will test what  must be by testing with one more  to determine the placement of the loops. So for the fourth test choice of , suppose  is two vertices with one loop and one non-loop edge.\par
\begin{figure}[h]
\centering \includegraphics[scale=.6]{grphsexp.png}
\caption{A picture of , for the fourth test choice of .}
\end{figure}\par
Again, all four vertices of  must be sent to the single vertex in , and each of the 9 edges can either be sent to the loop or to the vertex (independent of where the other edges go). So there are  morphisms here.\par
Now we will count the number of morphisms  by considering the following six disjoint types of morphisms whose union are all the morphisms: everything in  can be sent to a single vertex, everything in  but the loop can be sent to a vertex with the loop to a loop, the non-loop edge can be sent to a loop with everything else sent to a vertex, the non-loop edge can be sent to a non-loop edge with the loop sent to a vertex, the loop can be sent to a loop and the non-loop edge sent to a non-loop edge, or both the loop and the non-loop edge can be sent to loops.\par
There are two ways to send everything in  to a vertex since  only has two vertices. There are  ways to send the loop of  to a loop and everything else to a vertex since  has  loops. Likewise there are  ways to send the non-loop edge to a loop with everything else going to the incident vertex. As discussed before, there are  ways to send the non-loop edge to a non-loop edge and the loop to a vertex.\par
Now we will count the number of ways to send the loop of  to a loop in  and the non-loop edge of  to a non-loop edge of . There are  choices of where to send the loop, and this choice determines where vertex incident on the loop is sent. After this choice is made, there will be  non-loop edges in  to send the non-loop edge of  to (note again that we don't know which vertex the loops are on, but it does not effect our count of this type of morphism). So there are  of this type of morphism.\par
We have now accounted for  morphisms, which leaves  morphisms to account for. The only other type of morphism is one which sends both the loop and the non-loop edge of  to loops in . Suppose there are  loops on one vertex of  and  loops on the other. Then there is  morphisms which send both edges of  to a loop, and . Solving this system of equations yields the unique solution of  and . Hence the  loops are distributed evenly between the two vertices of .\par
So we now have a complete description of what we will call the ``exponential object'' , for the given  and given . (This assumed that categorial exponentiation with evaluation exists).\par
We've determined that  is a graph with two vertices (which we will label  and )  non-loop edges (which we will label  for ), and 7 loops on each vertex (which we will label  and  for ). It will also help us to label the graphs  and . Label the vertices of  as  and , and the edge as . Label the vertex of  by  and the loop by .\par
\begin{figure}[h]
\centering\includegraphics[scale=.6]{grphsexpobj.png}
\caption{Pictures of the graphs for the counterexample to categorial ``exponentiation'' in \textbf{Grphs}.}
\end{figure}
\par
But even though this exponential object exists, we have yet to show that its evaluation satisfies the uniqueness feature of the universal mapping property for exponentiation with evaluation. So we investigate evaluation by again using test objects in the universal mapping property for exponentiation with evaluation, which states that there exists  such that for all  and , there is a unique  such that .\par
For the first choice of test objects, let  be the single vertex graph  (which we have denoted ) with vertex . Then as  is the terminal object, . Thus there are two morphisms from  to . Let  be the morphism which maps all of  to the vertex  of , and let  by the morphism that maps the edge of  to the loop  of .\par
Consider , by the universal mapping property there is a unique  such that . There are two morphisms from  to ,  or . If , then . If , then .\par
As  is unique, only one of the two above possibilities holds. Since there is an automorphism of  that exchanges  and  (exchange all labels of  and  and swap  with ), without loss of generality we can choose . Then as  is unique, . Hence .\par
For the second choice of test objects, test with  to achieve a contradiction.\par
We claim that  if and only if  and  if and only if . For, since  and , . Hence  if and only if  and  if and only if .\par
A similar argument shows  if and only if  and  if and only if .\par
Then for  with  and , we have  and . Hence for such a , as  must preserve incidence,  for some . We now notice the following useful observation.\par
\textbf{(1)} If  for some  then , , , and .\par
For each  there are two choices of where to map each of , , , and  in a morphism from  (either to  or ). Thus for a fixed , there are 16 possible ways to map the edges , , , and  to . However, there are 24 such indicies. Thus by the pigeonhole principle,\par
\textbf{(2)} there exists  with , , , , and .\par
So define a morphism  by  for all vertices , , , , , , and  (incidence is trivially preserved). Then there is a unique  such that .\par
However, by (1)  and ,  is such a morphism and by (2) , , and  is another. Hence no such unique morphism exists and (L2) does not hold in \textbf{Grphs}.\par

\textbf{Axiom L3 (``subobject classifier'') holds in} \textbf{Grphs}: In \textbf{Grphs} the subobject classifier is the following graph:\par
\begin{figure}[h]
\begin{center}
\includegraphics[scale=.6]{grphssubobject.png}
\end{center}
\caption{A picture of the subobject classifier  in \textbf{Grphs}}
\end{figure}
together with the canonical morphism  from the terminal object  to the subobject classifier , which maps the vertex of  to the vertex labeled ``True'' in the above picture.\par
For any subgraph  of a graph  we must map all of  to this single true vertex. And we must map all the vertices of  to the other vertex. Any edge which is not in the image of  but is incident only to vertices in the image of  is mapped to the loop. Any edge with only one incident vertex in the image of  is mapped to the non-loop edge. This construction satisfies the universal mapping property for subobject classifiers.\\
\indent \textbf{Axiom L4 (``natural number object'') holds in} \textbf{Grphs}: The natural number object in \textbf{Grphs}, , is the graph with no edges, and a countably infinite number of vertices labeled by the natural numbers, coupled with the initial morphism  from the terminal object  defined by mapping the single vertex of the terminal object to the vertex labeled , and successor function  where given a vertex labeled , . This construction satisfies the universal mapping property for a natural number object.\\

\indent \textbf{Axiom L5 (``choice'') fails in} \textbf{Grphs}: Consider the graph morphism in figure 8 where  and .\par
\begin{figure}[h]
\begin{center}
\includegraphics[scale=.6]{accounter.png}
\end{center}
\caption{A picture for the counterexample to (L5 - ``choice'') in \textbf{Grphs}}
\end{figure}
For any ,  must send the edge  to one of the vertices. Without loss of generality, assume , then we must have  and  to preserve incidence. But then . So we have an example where there does not exist a  such that .\\
\indent \textbf{Axiom L6 (``two-valued'') fails for} \textbf{Grphs}: By applying the definition of terminal object and coproduct we have that  is a two vertex graph with no edges. But  has edges so it can not be isomorphic to  and is therefore not two-valued.
\end{proof}

\subsection{Lawvere-type Axioms for \textbf{SiGrphs}}
\begin{proposition}
\textbf{SiGrphs} satisfies axioms (L1), (L3), and (L4) and does not satisfy axioms (L2), (L5) and (L6).
\end{proposition}
\begin{proof}
\textbf{Axiom L1 (``limits'') holds for} \textbf{SiGrphs}: The proof of existence products and coproducts in \textbf{SiGrphs} follows similarly to the proof of existence of products and coproducts in \textbf{Grphs} using same constructions, by identifying any multiple edges that occur as a single edge and any multiple loops that occur as a single loop.\\
\indent \textbf{Axiom L2 (``exponentiation with evaluation'') fails for} \textbf{SiGrphs}: Suppose that exponentiation with evaluation exists in \textbf{SiGrphs}. Then \textbf{SiGrphs} has a terminal object, products and exponentiation with evaluation. Thus there is a standard adjoint functor relationship creating a bijection between the set of morphisms  and the set of morphisms .\\
\indent To construct our counterexample to the existence of exponentiation with evaluation in \textbf{SiGrphs}, let both  be  the graph with a single vertex with a loop, and  be . To begin, we will let  be a single vertex (this is the multiplicative identity, , in \textbf{SiGrphs} and hence ). As  admits 2 morphisms to  in \textbf{SiGrphs}, then  has 2 vertices, identified by .\\
\indent Now let  be . Then  is a graph with two vertices with a loop at each vertex and an edge between the two vertices.  admits 8 morphism to , as the edge and each loop can be mapped to either the vertex or the loop. Hence  admits 8 morphisms to . However, in \textbf{SiGrphs},  admits at most 4 morphisms to any graph on two vertices, 2 morphisms that map the loop to a vertex, and 2 that map the loop to another loop. Hence we have a contradiction and exponentiation with evaluation does not exist in \textbf{SiGrphs}.\\
\indent \textbf{Axioms L3 and L4 (``subobject classifier'' and ``natural number object'') both hold in} \textbf{SiGrphs}: Both the subobject classifier, and the natural number object for \textbf{SiGrphs} is the same as it is for \textbf{Grphs}.\\
\indent \textbf{Axioms L5 and L6 (``choice'' and ``two-valued'') fail for} \textbf{SiGrphs}: The same counterexample for \textbf{Grphs} applies here.
\end{proof}

\subsection{Lawvere-type Axioms for \textbf{SiLlGrphs}}
\begin{proposition}
\textbf{SiLlGrphs} satisfies axioms (L1), (L2), and (L4), and does not satisfy axioms (L3), (L5), and (L6).
\end{proposition}
\begin{proof}
\textbf{Axiom L1 (``limits'') holds for} \textbf{SiLlGrphs}: To show the existence of limits, we note that the terminal object, products, and equalizers are defined as in \textbf{SiGrphs}. For colimits, the initial object, and the coproduct are the same as in \textbf{SiGrphs} with the coequalizer being the construction given in \textbf{Grphs} with multiple edges identified as a single edge, and loops identified with the incident vertex.\\
\indent \textbf{Axiom L2 (``exponentiation with evaluation'') holds for} \textbf{SiLlGrphs}: Given graphs  and , define  by  hom, and  with \textunderscore if for all  with \textunderscore, there exists  with \textunderscore.\\
\indent Then define  by  for all vertices  and for  such that \textunderscore define  for  with \textunderscore. Such a  exists by construction of , and by construction of ,  is a graph morphism.\\
\indent Now let  be a graph with morphism . We show there is a unique morphism  such that .\\
\indent Let  and consider  for some  Then  induces a function  defined by . Then for  to hold, define , and  is a vertex set function uniquely determined by .\\
\indent Now let  with \textunderscore. Consider \textunderscore for some . Note that for a part , \textunderscore for some  implies there is a part  such that \textunderscore.\\
\indent For such a , since  preserves incidence, \textunderscore\textunderscore.
Then for  to hold, define  where \textunderscore which exists by definition of , and is uniquely determined by . Clearly  is a morphism in  and is uniquely determined by .\\
\indent \textbf{Axiom L3 (``subobject classifier'') fails for} \textbf{SiLlGrphs}: 
Assume a subobject classifier, , exists with morphism . Consider  having vertices  and  with  the unique morphism to the terminal object. Let  be inclusion where  is   with edge . Then there exists a unique  such that  is the pullback of  and . Then .
\begin{figure}[h]
\centering \includegraphics[scale=.6]{cdiag2.PNG}
\caption{A picture for the counterexample to the existence of a subobject classifier in \textbf{SiLlGrphs}.}
\end{figure}\par
Since  for  the vertex of  and, and since morphisms much map vertices to vertices, . Since , . Since graphs in  are loopless and incidence must be preserved, .\\
\indent Now consider the pullback of  and . It is the vertex induced subgraph of  on . However, since  and ,  and . This contradicts that  is the pullback of  and . Hence no subobject classifier exists.\\
\indent \textbf{Axioms L4 (``natural number object'') holds for } \textbf{SiLlGrphs}, \textbf{L5 and L6 (``choice'' and ``two-valued'') fails for} \textbf{SiLlGrphs}: The natural number object of \textbf{SiLlGrphs} is the same as in \textbf{Grphs}, and the counterexample to choice in \textbf{Grphs} applies as well. Since \textbf{SiLlGrphs} does not have a subobject classifier, Axiom 6 does not apply.
\end{proof}

\subsection{Lawvere-type Axioms for \textbf{StGrphs}}

\begin{proposition}
\textbf{StGrphs} satisfies axioms (L1), (L3), and (L4) and does not satisfy axioms (L2), (L5), and (L6).
\end{proposition}
\begin{proof}
\textbf{Axiom L1 (``limits'') holds for} \textbf{StGrphs}: As in the proof for \textbf{Grphs}, we will show the existence of limits and colimits by constructing the terminal object, the product, the equalizer, and their duals. It is easily shown that the graph with a single vertex and a loop, , is the terminal object and the empty vertex set (and part set) graph  is the initial object.\\
\indent For products, since strict morphisms map vertices to vertices the vertex set will be precisely the same as in the \textbf{Grphs} product, . However, the edges in the product will be different from the product in \textbf{Grphs}. In order to find the part set of the product of  and , we  take the product in \textbf{Grphs} and delete every part which is labeled  where exactly one of  or  was a vertex (these edges were projected to a vertex in one graph and an edge in the other which is impossible with strict morphisms). The projection morphisms are based on the first and second coordinate as they were in \textbf{Grphs}.
\begin{figure}[h]
\centering \includegraphics[scale=.7]{StGrphsPrds.png}
\caption{An example of the product in \textbf{StGrphs} with pictures.}
\end{figure}\\
\indent The coproduct and equalizer in \textbf{StGrphs} is precisely the same as in \textbf{Grphs}.\\
\indent For the coequalizer in \textbf{StGrphs} we do not need to check again that identifying equivalent edges gives us a graph since the graph object is the same and the morphisms are a subcollection of the morphisms from \textbf{Grphs}. We note, however, that the equivalence classes in \textbf{StGrphs} can contain either edges or vertices but not both.\\
\indent \textbf{Axiom L2 (``exponentiation with evaluation'') fails for} \textbf{StGrphs}: Suppose \textbf{StGrphs} did have exponentiation with evaluation. Then it has a terminal object, products and exponentiation with evaluation, and therefore there is a standard adjoint functor relationship providing a bijection between the set of morphisms  and the set of morphisms .\par
Let  be a vertex with two loops, and let  be two vertices with two edges between them. As in \textbf{Grphs}, we will try to determine what  would have to be, if it existed, by testing various choices of  and counting morphisms.\par
Let's begin with an  as just a vertex. The product  is simply two vertices, and there is only one morphism from this product to  (both vertices go to the only vertex in ). This means that there is only one morphism from a single vertex to , therefore  must have exactly one vertex.\par
Now let's test this object against a different . Let  be a vertex with one loop (note that this is the multiplicative identity, , for \textbf{StGrphs}). Hence .\par 
Both of the vertices in  must go to the only vertex in , and there are two choices where to send each of the edges in  (since both vertices are sent to a single vertex the each edge can go to either loop independent of where the other edge was sent, and since these are strict morphisms an edge must go to an edge), so there are  morphisms here. Therefore there must be exactly  morphisms from  to . We've already determined that  has exactly one vertex, and since  is a vertex with a loop,  must be a vertex with four loops.\par
Now we will test this object against one more  to derive a contradiction. Let  be two vertices and a single edge between them.\par
\begin{figure}[h]
\centering\includegraphics[scale=.6]{stgrphsexp.png}
\caption{A picture of , for the third test choice of .}
\end{figure}
Again, since  has only one vertex, all four vertices of  must be sent there. Which leaves us with four edges and two places where each can be sent (and each choice of where to send an edge is independent of the others), so we have  morphisms here. But there are only four strict morphisms from  to . So exponentiation with evaluation does not exist in \textbf{StGrphs}.\\
\indent \textbf{Axiom L3 (``subobject classifier'') holds for} \textbf{StGrphs}: The subobject classifier in \textbf{StGrphs} is the following graph:\par
\begin{figure}[h]
\centering\includegraphics[scale=.6]{stsubobject.png}
\caption{A picture of the subobject classifier  in \textbf{StGrphs}}
\end{figure}
together with the canonical strict morphism  from the terminal object , which sends the vertex and the loop of  to the vertex and loop labeled ``True" above.\par
For any subgraph  of a graph  we must map every vertex of  to the ``True" vertex in  and every edge of the image of  to the ``True" edge in . Every vertex in  must then be sent to the other vertex of . This then forces where the image of any other edge not in the image of  is mapped. If the edge has both its incident vertices in the image of  it is mapped to the loop at the ``true" vertex which is not labeled ``true". If the edge has one incident vertex in the image of  and the other incident vertex in  it is mapped to the non-loop edge between the two vertices of . If the edge has both its incident vertices in  it is mapped to the loop at the vertex which is not labeled ``true". This construction satisfies the universal mapping property for subobject classifiers.\\
\indent \textbf{Axiom L4 (``natural number object'') holds for} \textbf{StGrphs}: The natural number object in \textbf{StGrphs}, , will be countably many vertices with loops labeled with the natural numbers, coupled with the initial morphism  defined by mapping the single vertex and loop of the terminal object to the vertex and loop labeled , and successor function  where given a vertex with a loop labeled , . The natural number object works similarly as it did in \textbf{Grphs} only now instead of going through successive vertices, it can only go through successive vertices with loops.\\
\indent \textbf{Axiom L5 (``choice'') fails for} \textbf{StGrphs}: The same counterexample from \textbf{Grphs} applies here.\\
\indent \textbf{Axiom L6 (``two-valued'') fails for} \textbf{StGrphs}: By applying the definition of terminal object and coproduct we have that  is a graph with two vertices and a single loop at each vertex. But  has a non-loop edge, so it can not be isomorphic to  and is therefore not two-valued.
\end{proof}
\subsection{Lawvere-type Axioms for \textbf{SiLlStGrphs}}
\begin{proposition}
\textbf{SiLlStGrphs} does not satisfy any of the axioms of \textbf{Sets}.
\end{proposition}
\begin{proof}
\textbf{Axiom L1 (``limits'') fails for} \textbf{SiLlStGrphs}: We prove that \textbf{SiLlStGrphs} fails to have limits and colimits by proving that no terminal object exists and coequalizers do not exist.\\
\indent We show no terminal object exists by examining the two cases for a graph  in \textbf{SiLlStGrphs}. Either  has no edges, or  has an edge. If , as strict graph homomorphisms must send edges to edges, a graph that does contain an edge does not admit a strict graph homomorphism to . Hence,  cannot be a terminal object.\\
\indent If there is an edge , since the graphs in \textbf{SiLlStGrphs} are loopless, \textunderscore for some  where  and  are distinct. The consider the morphisms from , the graph containing only a single vertex, , to . Since  and  are distinct, there are two distinct morphisms,  defined by  and . Hence  is not a terminal object since not every graph admits a unique morphism to . Hence no terminal object exists in \textbf{SiLlStGrphs}\\
\indent Assume coequalizers exist. Let , the complete graph on 2 vertices  and  with edge , and consider the following two morphisms  where  is the identity morphism and  is the morphism where , , and . The coequalizer,  with morphism  such that , exists by hypothesis. Since  and , , and since morphisms must send edges to edges,  is an edge of .\\
\indent Let \textunderscore for some . Then since morphisms preserve incidence,  is incident to , and  is incident to . Hence  or .\\
\indent Without loss of generality, let . Then , and since , . Hence  is a loop of , which contradicts our hypothesis that  was in \textbf{SiLlStGrphs}. Thus coequalizers do not exist in \textbf{SiLlStGrphs}.\\
\indent \textbf{Axiom L2 (``exponentiation with evaluation'') fails for} \textbf{SiLlStGrphs}: Suppose \textbf{StLlStGrphs} did have exponentiation with evaluation. Then as it has binary products and exponentiation with evaluation, there is a standard adjoint functor relationship providing a bijection between the set of morphisms  and the set of morphisms .\par
Let  and  be , the graph with two vertices joined by a single edge. As in \textbf{Grphs}, we will try to determine what  would have to be, if it existed, by testing various choices of  and counting morphisms.\par
We begin with a test object  or a graph with only a vertex. The product  is simply two vertices, and there are only four morphisms from this product to . This means that there is four morphisms from a single vertex to , therefore  must have exactly four vertices.\par
Now test this object against a different . Let  be another copy of . Then  is two disjoint copies of . As edges must be sent to edges by strict morphisms, there are four morphisms from  to  (two choices for each edge of ). Hence there are four morphisms from  to . As there are no loops, each pair of morphisms from  to a  identifies as edge of . Thus  has two edges. \par 
Finally, we will test , the complete graph of three vertices. Now  is an isomorphic copy of  or the cycle graph on six vertices. There are two morphisms from  to  as a morphism of  to  is determined by how a single edge and its two incident vertices are mapped into . Thus there are two morphisms from  to . However, as  is isomorphic to , this implies  contains an odd cycle. However,  has only two edges and cannot contain an odd cycle, a contradiction. Hence exponentiation with evaluation does not exist in \textbf{SiLlStGrphs}.\\
\indent \textbf{Axioms L3, L4, L5, and L6 (``subobject classifier'', ``natural number object'', ``choice'' and ``two-valued'') fails for} \textbf{SiLlStGrphs}: As no terminal object exists in \textbf{SiLlStGrphs}, neither does a subobject classifier, nor does a natural number object. Since no subobject classifier exists, axiom 6 does not apply. The same counterexample for choice in \textbf{Grphs} applies here.
\end{proof}
\indent We note that products, coproducts, equalizers, and an initial object exist in \textbf{SiLlStGrphs}, by the following constructions in \textbf{SiStGrphs}.\\

\subsection{Lawvere-type Axioms for \textbf{SiStGrphs}}
\indent As \textbf{SiStGrphs} is the most common category of graphs studied in literature, axioms (L1) and (L2) in the following result are already known \cites{HN2004, Dochtermann}, but we include it for completeness.
\begin{proposition}
\textbf{SiStGrphs} satisfies axioms (L1), (L2), and (L4) and does not satisfy axioms (L3), (L5), and (L6).
\end{proposition}
\begin{proof}
Axioms (L1) (``limits'') and (L2) (``exponentiation with evaluation'') hold for \textbf{SiStGrphs} with proofs in \cites{HN2004, Dochtermann}.\\
\indent \textbf{Axiom L3 (``subobject classifier'') fails for} \textbf{SiStGrphs}: Assume a subobject classifier, , exists with morphism . Consider  having vertices  and  with an edge  between them with  the unique morphism to the terminal object. Let  be inclusion where  is  together with a loops  and  at vertices  and  respectively. Then there exists a unique  such that  is the pullback of  and . Then .
\begin{figure}[h]
\centering \includegraphics[scale=.6]{cdiag.PNG}
\caption{A picture for the counterexample to the existence of a subobject classifier in \textbf{SiStGrphs}.}
\end{figure}\\
\indent Since  for  the vertex of  and  for  the loop of , and since morphisms much send edges to edges,  and  where \textunderscore. Since , . Then since morphisms preserve incidence, \textunderscore. Since graphs in \textbf{SiStGrphs} can have at most one loop at any vertex, and morphisms must send edges to edges, .\\
\indent Now consider the pullback of  and . It is the vertex induced subgraph of  on . However, since  and ,  and . This contradicts that  is the pullback of  and . Hence no subobject classifier exists.\\
\indent \textbf{Axioms L4 (``natural number object'') holds for } \textbf{SiStGrphs}, \textbf{L5 and L6 (``choice'' and ``two-valued'') fails for} \textbf{SiStGrphs}: The natural number object for \textbf{SiStGrphs} is the same as in \textbf{StGrphs}, and the counterexample in \textbf{Grphs} of choice applies here as well. Since \textbf{SiStGrphs} does not have a subobject classifier, axiom 6 does not apply.
\end{proof}
\indent Note that the three axioms (L1)-(L3) define a topos and that \textbf{SiStGrphs} and \textbf{SiLlGrphs} are missing a subobject classifier, while \textbf{SiGrphs}, \textbf{StGrphs}, and \textbf{Grphs} are missing exponentiation with evaluation. We provide a reference table for the Lawvere-type Axioms (Table 1).\\
\begin{table}[h]
\caption{Lawvere-type Axioms for categories of graphs.}
\begin{tabular}{ r || l || l | l | l | l | l | l | }			
  \quad & \textbf{Sets} & \textbf{SiLlStG} & \textbf{SiLlG} & \textbf{SiStG} & \textbf{SiG} & \textbf{StG} & \textbf{G} \\
\hline
\hline
 (L1) Limits & Y & N & Y & Y & Y & Y & Y \\
\hline  
  (Colimits) & Y & N & Y & Y & Y & Y & Y \\
\hdashline
   & Y & N & Y & Y & Y & Y & Y \\
\hline 
   & Y & Y & Y & Y & Y & Y & Y \\
\hline 
   & Y & Y & Y & Y & Y & Y & Y  \\
\hline 
   & Y & Y & Y & Y & Y & Y & Y \\
\hline 
  Equalizer & Y & Y & Y & Y & Y & Y & Y \\
\hline 
  (Coequalizer) & Y & N & Y & Y & Y & Y & Y \\
\hline 
\hline
 (L2) Exp. with Eval. & Y & N & Y & Y & N & N & N \\
\hline 
\hline
 (L3) Subobj. Classifier & Y & N & N & N & Y & Y & Y \\
\hline 
\hline
\hline
 (L4) Nat. Num. Obj. & Y & N & Y & Y & Y & Y & Y \\
\hline 
\hline
 (L5) Choice & Y & N & N & N & N & N & N \\
\hline 
\hline
 (L6) 2-valued & Y & N & N & N & N & N & N \\
\hline 
\end{tabular}
\end{table}
\\
\section{Other Categorial Properties of the Categories of Graphs}
\subsection{Epis and Monos in Categories of Graphs}
\indent We begin our investigation of other properties by first giving characterizations of epimorphisms and monomorphisms in the categories of graphs. These characterizations of epimorphisms and monomorphisms were known in \textbf{Grphs} \cite{KKWil}.
\begin{proposition}
A morphism in \textbf{Grphs}, \textbf{StGrphs}, and \textbf{SiGrphs} is an epimorphism if and only if it is a surjective function of the part sets, and a morphism in the above categories is a monomorphism if and only if it is an injective function of the part sets.
\end{proposition}
\begin{proof}
It is trivial to show that surjections are always epimorphisms and injections are always monomorphisms, we prove the converses.\\
\indent So let  be an epimorphism in \textbf{Grphs}, and suppose  is not surjective. Then there exists .\\
\indent First suppose . Construct the graph  by appending a vertex  to  such that  is adjacent to every vertex  is adjacent to. By construction  is a subgraph of .\\
\indent Since , no edge incident to  is in the image of . Now consider  the inclusion morphism and  defined by  for all , ,  for all edges  not incident to , and for edge  incident to , set  to be the corresponding edge incident to . This is clearly a morphism (actually it is strict). Then  but , a contradiction to  being an epimorphism.\\
\indent Now suppose  is an edge of . Construct the graph  by appending an edge  to  such that  has the same incidence as . Then by construction  is a subgraph of .\\
\indent Now consider  the inclusion morphism and  defined by  for all  and . As the incidence of  is the same as  this is a morphism (it is actually strict). Then  but , a contradiction to  being an epimorphism. Hence epimorphisms in \textbf{Grphs} are surjective functions of the corresponding part sets. A similar proof applies or \textbf{StGrphs}.\\
\indent For \textbf{SiGrphs}, a similar proof applies. However, in the case that  is an edge, a different construction is required. Let  be , the graph with one vertex and one loop. Let  be the morphism that maps everything to the vertex, and  be the morphism that maps everything except  to the vertex, and maps  to the edge. Then  but , a contradiction.\\
\indent Now let  be an monomorphism in \textbf{Grphs}, and suppose  is not injective. Then there exists  such that . Consider  where  maps the edge to , and the vertices to the vertices incident to  and  maps the edge to  and the vertices to the vertices incident to . Then as  must preserve incidence,  but , a contradiction to  being a monomorphism. A similar proof applies to \textbf{StGrphs} and \textbf{SiGrphs}
\end{proof}
\indent This changes if we add enough restrictions, as seen in the following proposition.
\begin{proposition}
A morphism in \textbf{SiStGrphs}, \textbf{SiLlGrphs}, and \textbf{SiLlStGrphs} is an epimorphism if and only if it is a surjective function of vertex sets, and a morphism in the above categories is a monomorphism if and only if it is an injective function of the vertex sets.
\end{proposition}
\begin{proof}
Let  be an epimorphism in \textbf{SiLlGrphs}. Suppose  is not surjective. Then there exists . Construct the graph  by appending a vertex  to  such that  is adjacent to every vertex  is adjacent to. By construction  is a subgraph of .\\
\indent Since , no edge incident to  is in the image of . Now consider  the inclusion morphism and  defined by  for all , ,  for all edges  not incident to , and for edge  incident to , set  to be the corresponding edge incident to . Then  but , a contradiction to  being an epimorphism. Hence epimorphisms in \textbf{SiLlGrphs} have surjective vertex set functions. A similar proof applies to \textbf{SiStGrphs} and \textbf{SiLlStGrphs}.\\
\indent Suppose  is a morphism in \textbf{SiLlGrphs} and  is surjective. Consider morphisms  such that . Since  is surjective and , . So if  there exists an edge  such that , even though . There are two possibilities for  and , either as different vertices or edges.\\
\indent If  and  are different vertices, as , the incident vertices to  in  are both mapped to the same vertex, so for incidence to hold  and  would also be mapped to that vertex and . If  and  are mapped to different edges, since  they must have the same incidence. Since graphs in \textbf{SiLlGrphs} are simple and loopless, . Hence both possibilities lead to contradictions.
A similar proof holds for \textbf{SiLlStGrphs}, and for \textbf{SiStGrphs} a third possiblity arises for  and  to be different loops. However, in this case, as simple graphs have only one loop and , they must be mapped to the same loop.\\
\indent Now let  be a monomorphism in \textbf{SiLlGrphs}. Suppose  is not injective. Then there exists  such that . Then consider the two morphisms  defined by  mapping the single vertex of  to , and  mapping the single vertex of  to . Clearly  but  a contradiction to  being a monomorphism. Hence monomorphisms in \textbf{SiLlGrphs} have injective vertex set functions. A similar proof applies to \textbf{SiStGrphs} and \textbf{SiLlStGrphs}.\\
\indent Suppose  is a morphism in \textbf{SiLlGrphs} and  is injective. Consider morphisms  such that . Since  is injective, . Thus if there exists  such that , then  must be an edge of . Since  is injective and ,  and  cannot both be vertices in . Without loss of generality assume  is an edge.\\
\indent Note that  cannot be a vertex of , for both incident vertices of  in  are mapped to  as well. Then since , morphisms preserve incidence, and the graphs are loopless,  is mapped to a vertex. Hence  must be an edge of . Since ,  and  have the same incident vertices, and since the graphs are simple, , a contradiction. Hence  is a monomorphism.\\
\indent Similar proofs apply in \textbf{SiStGrphs} and \textbf{SiLlStGrphs}. 
\end{proof}

\subsection{Free Objects and Cofree Objects in the Categories of Graphs}
\indent We now consider the underlying vertex set functor  defined for each of the categories of graphs, where   for a graph  and  for a morphism . We define the free graph functor  to be such that  is left adjoint to . We similarly define the cofree graph functor  to be such that  is left adjoint to . The following two proposition characterizes the free and cofree objects in the categories of graphs.
\begin{proposition}
Given a set , in all six categories of graphs, the free graph on  is just the empty edge graph on the vertex set , and the free objects are the empty edge set graphs, denoted  for finite vertex sets.
\end{proposition}
\begin{proof}
Let  be a set in \textbf{Sets} with  elements, and let  where . Now let  be a graph such that there is a function . We show there is a unique graph morphism  such that  for some . Note that . Hence define the function  as .\\
\indent Let  be the pair of function maps  and . Since there are no edges in , incidence is clearly preserved (and the morphism is strict). Then since  must hold, , and ,  is uniquely determined by . 
\end{proof}
\begin{proposition}
Given a set ,
\begin{enumerate}
\item in \textbf{Grphs},  \textbf{SiGrphs}, and  \textbf{SiLlGrphs} the cofree graph on  is the complete graph on the vertex set  and the cofree objects are the complete graphs with at least one vertex, denoted  for finite vertex sets with ,
\item in \textbf{StGrphs}, \textbf{SiStGrphs} the cofree graph on  is a complete graph with a single loop on each vertex having the vertex set , and the cofree objects are the complete graphs with a loop at each vertex with at least one vertex, denoted  for finite vertex sets with ,
\item in \textbf{SiLlStGrphs} no cofree graph exists. 
\end{enumerate}
\end{proposition}
\begin{proof}
\textbf{Part 1}: Let  be a set in \textbf{Sets} and define  as the complete graph with the vertex set . Let  be a graph in \textbf{Grphs} with set function . We show that there is a unique graph morphism  such that  for some set function   Note that . Hence we define  as .\\
\indent For  to hold,  is uniquely determined. Then let  be an edge of  incident to vertices  where  and  are not necessarily distinct. Then since graph morphisms must preserve incidence, for  to be a morphism,  must map to the part  of  incident to vertices  and . By the definition of  such a part  exists, even if it is a vertex. Hence  exists and is uniquely determined by . A similar proof applies for \textbf{SiGrphs} and \textbf{SiLlGrphs}.\\
\indent \textbf{Part 2}: Let  be a set in \textbf{Sets} and define  as the complete graph with a loop at every vertex with the vertex set . Let  be a graph in \textbf{StGrphs} with set function . We show that there is a unique strict graph homomorphism  such that  for some set function   Note that . Hence we define  as .\\
\indent For  to hold,  is uniquely determined. Then let  be an edge of  incident to vertices  where  and  are not necessarily distinct. Then since strict graph homomorphisms must send edges to edges and preserve incidence, for  to be a strict graph homomorphism,  must map to the edge  of  incident to vertices  and . By the definition of  such an edge  exists. Hence  exists and is uniquely determined by . A similar proof applies for \textbf{SiStGrphs}. \\
\indent \textbf{Part 3}: Assume cofree graphs exist. Let  in \textbf{Sets} and  be the cofree graph associated with  and function . Consider  with vertices  and  and edge  and set function  defined by . Then since  is a cofree object, there is a unique morphism in \textbf{SiLlStGrphs}, , such that . Since  is a strict graph homomorphism, it must send  to an edge in . Thus  for some . Since graph homomorphisms preserve incidence,  is incident to  for some vertex  and  for some vertex .\\
\indent Since  is loopless, . Then since ,  and , . Now consider the morphism  defined by ,  and . Clearly . Then  and . Thus , and  is not unique, which is a contradiction to the universal mapping property of the cofree object.
\end{proof}

\subsection{Projective and Injective Objects in the Categories of Graphs}
\indent The definitions for free objects and cofree objects are dependent on the category being a concrete category. We move on to other categorial constructions that are defined for any abstract category. We start with the injective objects and projective objects.
\begin{proposition}
\begin{enumerate}
\item In \textbf{Grphs}, \textbf{SiGrphs}, and \textbf{StGrphs}, all graphs with at most one edge per component are precisely the projective objects, and there are enough projective objects.
\item In \textbf{SiLlGrphs}, \textbf{SiStGrphs}, and \textbf{SiLlStGrphs}, the projective objects are precisely the free objects, and there are a enough projective objects.
\end{enumerate}
\end{proposition}
The projective objects in \textbf{Grphs} are found in \cite{KKWil}, we provide an alternate proof.
\begin{proof}
\textbf{Part 1}: First note that if  is an epimorphism in \textbf{Grphs} then  is a surjective map of the associated part sets. So let  be a graph with at most one edge in each component with morphism  for some graph . Let  be a graph with an epimorphism .\\
\indent Consider a component of . If the component is composed of a single vertex,  without a loop, then since  is a surjection, there exists  such that . If the component is composed of a single vertex  with a loop , and under  the loop is identified with , then as before there exists  such that . If the loop is not identified with the vertex, then as  is a surjection, there exists a loop  such that .\\
\indent If the component has an edge , and two vertices  and , and under  the two vertices are identified with the edge, then there exists  such that . If under  the two vertices are identified, and the edge is sent to a loop, then there exists  such that  and . If under  the two vertices are not identified, then  is mapped to the incident edge of their images. Then since  is a surjection, there exists  such that , , and . Then the definition for  such that  is obvious, and since each component can be mapped independently from other components, this is a graph morphism.\\
\indent Now suppose  is a graph with at least two edges in some component, called  and . Consider the graph  created by ``splitting''  at each vertex incident to more than two edges. That is, for every vertex  incident to at least two edges  and , create  and  in  such that  is incident to  and  is incident to  with no edge between  and , with  and  replacing . Then  admits an epimorphism  to  by re-identifying these split vertices.\\
\indent However, with morphism ,  does not admit a morphism  to  such that  as edges  and  must be sent to the same component to preserve incidence. Hence  is not projective.\\
\indent Let  be a graph in \textbf{Grphs}. To show there are enough projectives, we show there is a projective object  and an epimorphism . As above, construct  by ``splitting'' . Then  admits an epimorphism to , and since  does not have more than one edge per component,  is projective. A similar proof applies to \textbf{SiGrphs} and \textbf{StGrphs}\\
\indent \textbf{Part 2}: First note that if  is an epimorphism in \textbf{SiLlGrphs} then the vertex set function  is surjective. We show that the free objects are projective objects. Clearly the empty graph  is projective since it is the initial object. Now let  be a non-empty set in \textbf{Sets},  be a graph with a morphism , and  be a graph with an epimorphism . We show that there is a morphism  such that .\\
\indent Since  is an epimorphism,  is a surjective function. Hence for all , there is a  such that . Then define  for every . Then  for every vertex  of . Since  contains no edges,  is a graph morphism (and strict). Thus  is projective.\\
\indent Now let  be a graph with at least 1 edge, and consider , the complete graph on , with an inclusion morphism . By Proposition 4.2. there is an epimorphism  for  the empty edge graph on . Since  has an edge any morphism from  to  must identify at least two vertices, and hence no such morphism  exists such that . Thus  is not projective.\\
\indent Let  be a graph in \textbf{SiLlGrphs}. To show there are enough projectives, we show there is a projective object  and an epimorphism . By Proposition 4.2., the projective object ,the empty edge graph on , admits an epimorphism to .  A similar proof applies to \textbf{SiStGrphs} and \textbf{SiLlStGrphs}.\\
\end{proof}
\begin{proposition}
\begin{enumerate}
\item In \textbf{Grphs}, \textbf{SiGrphs}, and \textbf{SiLlGrphs} the injective objects are precisely the graphs containing the cofree objects as spanning subgraphs and there are enough injective objects.
\item In \textbf{StGrphs} and \textbf{SiStGrphs}, the injective objects are precisely the graphs containing the cofree objects as spanning subgraphs and there are enough injective objects.
\item In \textbf{SiLlStGrphs}, there are no injective objects.
\end{enumerate}
\end{proposition}
The injective objects in \textbf{Grphs} are found in \cite{KKWil}, we provide an alternate proof.
\begin{proof}
\textbf{Part 1}: Let  be a graph that contains a cofree spanning subgraph in \textbf{Grphs}, and let  be graphs in \textbf{Grphs} with a morphism  and a monomorphism . We show there is a morphism  such that .\\
\indent Since  is a monomorphism, it is an injection of the part sets. Then for all  there is a unique  such that . Since  is non-empty, it has a vertex . Define  by  if  and  if  is not in the image but a vertex. If  is an edge that is not in the image with \textunderscore, then define  where  is some edge with \textunderscore. One exists since  contains a spanning cofree subgraph, and in the case that  the vertex suffices. By this construction,  is a  morphism and .\\
\indent Now let  be a graph in \textbf{Grphs} that does not contain a cofree spanning subgraph. Assume it is an injective object of \textbf{Grphs}. Then there are distinct vertices  such that there is no edge  with \textunderscore.\\
\indent Then consider  with morphism  defined by  and , for  and  the two vertices of , and  the inclusion morphism. Since the inclusion morphism is a monomorphism, there is a morphism  such that . Then  and . Since morphisms preserve incidence, \textunderscore\textunderscore, and there is an edge  such that \textunderscore, a contradiction. Hence  is not an injective object.\\
\indent To show there are enough injective objects we show that for any graph  in \textbf{Grphs}, there is an injective object  with a monomorphism . If  is not the initial object,  is an injective object and , the inclusion morphism, is a monomorphism. If  then  suffices. Hence there are enough injective objects in \textbf{Grphs}. A similar proof applies to \textbf{SiGrphs} as well as \textbf{SiLlGrphs} that relies on monomorphisms as injections of the vertex sets and the fact that there is at most one edge between any two distinct vertices.\\
\indent \textbf{Part 2}: Let  be a graph that contains a cofree spanning subgraph in \textbf{StGrphs}, and let  be graphs in \textbf{StGrphs} with a morphism  and a monomorphism . We show there is a morphism  such that .\\
\indent Since  is a monomorphism, it is an injection of the part sets. Then for all  there is a unique  such that . Since  is non-empty, it has a vertex . Defined  by  if  and  if  is not in the image but a vertex. If  is an edge that is not in the image with \textunderscore, then define  where  is some edge with \textunderscore. One exists since  contains a spanning cofree subgraph. By this construction,  is a strict graph morphism and .\\
\indent Now let  be a graph in \textbf{StGrphs} that does not contain a cofree spanning subgraph. Assume it is an injective object of \textbf{StGrphs}. Then there are vertices  (not necessarily distinct) such that there is no edge  with \textunderscore.\\
\indent Then consider  with morphism  defined by  and , for  and  the two vertices of , and  the inclusion morphism. Since the inclusion morphism is a monomorphism, there is a morphism  such that . Then  and . Since morphisms preserve incidence, \textunderscore\textunderscore, there is an edge  such that \textunderscore, a contradiction. Hence  is not an injective object.\\
\indent To show there are enough injective objects we show that for any graph  in \textbf{StGrphs}, there is an injective object  with a monomorphism . If  is not the initial object,  is an injective object and , the inclusion morphism, is a monomorphism. If  then  suffices. Hence there are enough injective objects in \textbf{StGrphs}. A similar proof applies to \textbf{SiStGrphs} that relies on monomorphisms as injections of the vertex set and the fact that there is at most one edge between any two (not necessarily distinct) vertices.\\
\indent \textbf{Part 3}: Suppose there exist injective objects. Let  be an injective object in \textbf{SiLlStGrphs}. Consider the complete graph  with the cardinality of  is greater than that of . Then consider the morphisms  the identity on  and , the inclusion morphism of  into . Since inclusion morphisms are injections, they are monomorphisms.\\
\indent Then since  is injective, there is a morphism  such that . Since the cardinality of  is greater than the cardinality of  and  is a set map, there are two distinct vertices  such that . Since  is a strict morphism and  is a complete graph, the edge  incident to  and  in  must be sent to an edge in . Since graph homomorphisms preserve incidence and , \textunderscore, and hence  is a loop. This contradicts  being loopless. Hence no injective objects exist.
\end{proof}

\subsection{Generators and Cogenerators in the Categories of Graphs}
\indent The last property we characterize in the six categories of graphs is a classification of generators and cogenerators.
\begin{proposition}
\begin{enumerate}
\item In \textbf{Grphs} and \textbf{SiGrphs}, all graphs containing a non-loop edge are precisely the generators,
\item in \textbf{SiLlGrphs} all nonempty graphs are generators,
\item in \textbf{StGrphs} no generators exist,
\item in \textbf{SiStGrphs} and \textbf{SiLlStGrphs}, the empty edge graphs with at least one vertex are precisely the generators.
\end{enumerate}
\end{proposition}
\begin{proof}
\textbf{Part 1}: Let  be a graph in \textbf{Grphs} with a non-loop edge, , with vertices  and  incident to . Consider  with vertices  and  with incident edge . Then  has an epimorphism  defined by  and  for all vertices  and where every loop incident to  is mapped to  and every non-loop edge incident to  (including ) is mapped to , and all other edges mapped to .\\
\indent Hence, we only need to show  is a generator. Let  be such that . Hence  for some . First suppose  is a vertex. Then the morphism  from  to  mapping the two vertices and edge of  to  suffices. If  is an edge of , then the morphism that maps the edge of  to  and the incident vertices of the edge to the incident vertices of  suffices.\\
\indent Now suppose  is a graph containing no non-loop edges. Then no morphism from  to  can distinguish between , where  maps the two vertices and edge of  to the vertex of  and  maps the two vertices of  to the single vertex of  and the edge to the loop. Hence  is not a generator. The same proof applies to \textbf{SiGrphs}.\\
\indent \textbf{Part 2}: The empty graph is in the initial object of \textbf{SiLlGrphs} and thus cannot be a generator.\\
\indent Since in all graphs of \textbf{SiLlGrphs} there is at most one edge between any two distinct vertices, if  agree on the vertex sets, they agree on the edge sets and . Hence if  are distinct, then for some vertex  of , . So let  be a non-empty graph. Then  has a vertex and the morphism , where  maps all of  to the vertex  suffices.\\
\indent \textbf{Part 3}: Suppose a generator  in \textbf{StGrphs} exists. Then consider , where  maps the vertex of  to one vertex of  and  maps the vertex of  to the other vertex of . Since  is a generator, it admits a morphism  such that . Since morphisms are strict, edges must be mapped to edges. However,  has no edge, and thus  is edgeless.\\
\indent Now consider  and  , where  is a graph consisting of two vertices with two parallel edges between the two vertices. Define  by  mapping the edge of  to one edge of , and  mapping the edge of  to the other edge of , but mapping the vertices of  in tandum. Then no morphism from  can distinguish between  and  as  has no edges. Hence  is not a generator, a contradiction, and no generators exist in \textbf{StGrphs}.\\
\indent \textbf{Part 4}: First we show , an empty edge graph with at least one vertex, is a generator, then we show that any graph with an edge is not a generator. Let  and  be graphs in \textbf{SiStGrphs} with morphisms  such that . Then there is a vertex  such that , otherwise since the morphisms preserve incidence and there is at most one edge between any two vertices,  for all edges  and .\\
\indent First note . Now consider the map  that sends the single vertex of , , to . Then . Hence . Hence  is a generator.\\
\indent To show  is a generator, we consider the morphism  that sends every vertex of  to . Then clearly , and hence  is a generator of \textbf{SiStGrphs}.\\
\indent Now let  be a graph in \textbf{SiStGrphs} with at least one edge. Consider  with two vertices  and  with two morphisms  where  is the identity morphism and  is the ``twist'' morphism defined by  and . Clearly , but since morphisms must send edges to edges,  admits no map to . Thus  is not a generator. A similar proof applies for \textbf{SiLlStGrphs}.\\
\end{proof}
\begin{proposition}
\begin{enumerate}
\item In \textbf{Grphs} and \textbf{SiGrphs}, the graphs containing a loop and a non-loop edge are precisely the cogenerators,
\item In \textbf{SiLlGrphs}, the graphs containing an edge are the cogenerators,
\item in \textbf{StGrphs}, the graphs containing both a vertex with two distinct loops and containing a subgraph isomorphic to  are precisely the cogenerators,
\item in \textbf{SiStGrphs}, the cogenerators are precisely the graphs containing a subgraph isomorphic to ,
\item in \textbf{SiLlStGrphs} no cogenerators exist.
\end{enumerate}
\end{proposition}
\begin{proof}
\textbf{Part 1}: Let  be a graph composed of two disconnected components. One component contains a vertex  with a loop , and the other component contains two vertices  and  with an edge  between them. If  is a cogenerator, then any graph containing a loop and a non-loop edge is also a cogenerator by the appropriate inclusion morphism.\\
\indent Let  be two distinct morphisms in \textbf{Grphs}. Then  for some . If both  and  are vertices, define  by ,  for all ,  for all non-loop edges  incident to  in ,  for all loops  incident to , and  for all other edges  of . Hence .\\
\indent If  is a vertex of  and  is not, define  by ,  and  for all . Hence . If  is an edge of , then define  by  and  for all . Hence , and  is a cogenerator.\\
\indent If  is a graph not containing any loops, then no morphism exists from  to  that can distinguish between , where  is the identity morphism, and  is the morphism that maps the loop and vertex of  to the vertex of . If  is a graph not containing any non-loop edges, then no morphism exists from  to  that can distinguish between  where  maps the single vertex of  to one vertex of , and  maps the single vertex of  to the other vertex of . Hence all cogenerators require a loop and a non-loop edge. A similar proof applies to \textbf{SiGrphs}.\\
\indent \textbf{Part 2}: Let  be a graph with an edge  incident to vertices  and . As in the proof for generators in \textbf{SiLlGrphs}, if  are distinct, then there is a vertex  such that . Define  by ,  for all ,  for all edges of  incident to , and  for all other edges in . Hence , and  is a cogenerator.\\
\indent If  is a graph not containing any edges, then no morphism exists from  to  that can distinguish between  where  maps the single vertex of  to one vertex of , and  maps the single vertex of  to the other vertex of . Hence all cogenerators in \textbf{SiLlGrphs} require an edge.\\
\indent \textbf{Part 3}: Let  be a graph composed of two disconnected components. One component contains a vertex  with two loops  and , and the other component contains two vertices  and  with an edge  between them, and a loop  and  on  and  respectively. If  is a cogenerator, then any graph containing a loop and a non-loop edge is also a cogenerator.\\
\indent Let  be two distinct morphisms in \textbf{StGrphs}. Then  for some . If  is a vertex, define  by ,  for all ,  for all non-loop edges  incident to  in ,  for all loops  incident to , and  for all other edges  of . Hence .\\
\indent If  is an edge of , then define  by   and  for all  and  for all other edges  in . Hence , and  is a cogenerator.\\
\indent Now suppose  is a graph that has no vertex that is incident to two loops. Then consider , where  is a graph composed of one vertex with two loops  and ,  maps the loop of  to  and  maps the loop to . No morphism exists from  to  that can distinguish between  and .\\
\indent Now suppose  is a graph that does not contain any subgraph isomorphic to , then there is no edge incident two any two vertices that have loops. Then there exists no morphism from  that can distinguish between , where  maps the vertex and loop of  to one of the vertices of  and its incident loop, and  maps the vertex and loop of  to the other vertex of  and its incident loop. Hence all cogenerators in \textbf{StGrphs} require a vertex with two loops, and a subgraph isomorphic to .\\
\indent \textbf{Part 4}: Let  have vertices  and  with edge  incident to  and  and loops  and  on  and  respectively. Let  and  be graphs in 
\textbf{SiStGrphs} with morphisms  such that . Since there is at most one loop at a vertex and at most one edge between any two vertices, there is a vertex  such that . Define a map  by  and  for all vertices , and for ,  if \textunderscore for ,  if \textunderscore, and  if \textunderscore for .\\
\indent We now show  is a strict graph homomorphism. Let . If  then \textunderscore\textunderscore for some . If  then \textunderscore\textunderscore. If  then \textunderscore\textunderscore for some . Hence  preserves incidence, and since  sends edges to edges;  is a strict graph homomorphism.\\
\indent Since ,  and . Hence , and  is a cogenerator of \textbf{SiStGrphs}.\\
\indent If  is a graph in \textbf{SiStGrphs}  that contains a subgraph isomorphic to  then clearly , where  is the inclusion morphism (over the isomorphism) . Hence  is a cogenerator of \textbf{SiStGrphs}.\\
\indent Suppose  does not have a subgraph isomorphic to  but  is a cogenerator. Then no two vertices of  with loops are incident to the same edge. Consider the two morphisms  where  is the identity morphism and  is the morphism defined by , , , , and . Since  is a cogenerator, there is a morphism  such that .\\
\indent Let  for some  and  for some . If , then since edges must be sent to edges and incidence is preserved, \textunderscore\textunderscore\textunderscore. Since there is at most one loop at a vertex, then . Similarly  and . Hence , a contradiction. Thus .\\
\indent Since morphisms must send edges to edges, \textunderscore, and \textunderscore,  has a loop  and  has a loop . Now consider . Since \textunderscore\textunderscore,  and  are two vertices with loops adjacent to the same edge, a contradiction. Hence  must contain a subgraph isomorphic to .\\
\indent \textbf{Part 5}:
To show there are no cogenerators in \textbf{SiLlStGrphs}, we show there is no graph  such that any graph  admits a morphism to . Assume such a graph  exists. Consider the complete graph  such that the cardinailty of  is greater than the cardinality of . By hypothesis there is a morphism . Since  and  is a set map, there are two distinct vertices  such that . Let  be the edge in  incident to both  and . Since graph homomorphisms preserve incidence and , \textunderscore. Then since edges must be sent to edges,  is a loop. This contradicts  being loopless. Hence no such object exists.\\
\indent We now note that every graph  with at least two vertices admits at least two distinct morphisms to the complete graph  with the cardinality of  equal to the cardinality of , as the automorphism group of  is the symmetric group on . This fact coupled with the fact there is no graph such that any other graph admits a morphism to it proves no cogenerators exist.
\end{proof}
\indent We will now provide a reference table.
\begin{table}[h]
\caption{Categorial Properties in the categories of graphs.}
\begin{tabular}{ r || l | l | l | l | l | l | l | }			
  \quad & \textbf{SiLlStG} & \textbf{SiLlG} & \textbf{SiStG} & \textbf{SiGrphs} & \textbf{StG} & \textbf{G} \\
\hline
  Epis are surj. on &  vert. sets & vert. sets & vert. sets & part sets & part sets & part sets \\
\hline  
  Monos are inj. on  &  vert. sets & vert. sets & vert. sets & part sets & part sets & part sets \\
\hline 
  Free Graphs &  &  &  &  &  &  \\
\hline 
  Cofree Graphs &   &  &  &  &  &  \\
\hline 
  Projective Objects & Y & Y & Y & Y & Y & Y  \\
\hline 
  Enough Proj. & Y & Y & Y & Y & Y & Y \\
\hline 
  Injective Objects  & N & Y & Y & Y & Y & Y \\
\hline 
  Enough Inj. & N & Y & Y & Y & Y & Y \\
\hline 
  Generators & Y & Y & Y & Y & N & Y\\
\hline
  Cogenerators & N & Y & Y & Y & Y & Y\\
\hline
\end{tabular}
\end{table}


\section{Conclusion - Making Informed Categorial Choices}
\indent When using category theory in the study of graph morphisms, a choice must be made of which category of graphs to use. An informed choice of category can be based on what categorial properties would be useful to the study.\\
\indent From only the Lawvere-type Axioms, the six categories of graphs break up into three types. The first types consists of the ``smallest'' category in terms of morphisms and objects: \textbf{SiLlStGrphs}. It fails to satisfy any of the Lawvere-type Axioms. The other five categories fall into the other two types and they have limits and a natural number object, but they fail to have choice and are not two-valued.\\
\indent The second type consists of \textbf{SiStGrphs} and \textbf{SiLlGrphs}. These categories have exponentiation with evaluation, but they lack a subobject classifier. The last type consists of \textbf{SiGrphs}, \textbf{StGrphs} and \textbf{Grphs}, each of which have a subobject classifier but they do not have exponentiation with evaluation. We note that all six categories of graphs fail to be topoi.\\
\indent However, within these 3 types, there are vast differences when we look at the other categorial properties. The terminal object, free objects, and cofree objects depend on the type of morphism chosen for the category, highlighting a difference between the categories in our second type, \textbf{SiLlGrphs} and \textbf{SiStGrphs}. This property also differentiates \textbf{StGrphs} from \textbf{SiGrphs} and \textbf{Grphs}. Another categorial property that singles out \textbf{StGrphs} from the third type is a lack of generators.\\
\indent \textbf{Grphs} and \textbf{SiGrphs} have the same categorial properties investigated in this paper, but they can satisfy constructions in different ways. For example, consider the coequalizer for the two morphisms  both of which send the edge of  to the non-loop edge of  but differ on where they send the two incident vertices. In \textbf{Grphs} the coequalizer object will be a graph with one vertex and two loops, whereas in \textbf{SiGrphs} the coequalizer object will be isomorphic to , as the two loops must be identified. When considering categorial quotients, this difference in construction can be crucial.\\


\addcontentsline{toc}{section}{References}
\bibliographystyle{alpha}
\bibliography{concgrphs}
\noindent \textsc{Department of Mathematical Sciences, The University of Montana} \\
\textsc{Missoula, MT 59812-0864, USA} \\
\textit{E-mail}: george.mcrae@umontana.edu\\
\indent \qquad demitri.plessas@umontana.edu\\
\indent \qquad liam.rafferty@umontana.edu\\

\end{document}
