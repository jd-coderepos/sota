\documentclass{acm_proc_article-sp}
\usepackage[utf8]{inputenc}
\usepackage[numbers,sort]{natbib}
\pdfoutput=1
\usepackage{tikz}
\usetikzlibrary{shapes,arrows}

\pagenumbering{arabic} \usepackage[english]{babel}
\usepackage{url}
\usepackage{balance}
\usepackage{graphicx}
\usepackage{caption}
\DeclareCaptionType{copyrightbox}
\usepackage{subcaption}
\usepackage{multirow}
\usepackage{todonotes}

\usepackage{setspace}
\renewcommand{\topfraction}{0.85}
\renewcommand{\textfraction}{0.1}
\renewcommand{\floatpagefraction}{0.75}
\DeclareMathOperator*{\argmin}{arg\,min}

\usepackage{color}
\usepackage{soul}
\newcommand{\hilight}[1]{#1}



\title{
Can't you hear me knocking: \\Identification of user actions on Android apps\\via traffic analysis
}


\numberofauthors{4} \author{
\alignauthor Mauro~Conti\\ 
\affaddr{University of Padua}\\
\affaddr{Padua, Italy}\\
\email{conti@math.unipd.it}
\alignauthor Luigi~V.~Mancini\\
       \affaddr{Sapienza University of Rome}\\
       \affaddr{Rome, Italy}\\
       \email{lv.mancini@di.uniroma1.it}
\alignauthor Riccardo Spolaor\\
       \affaddr{University of Padua}\\
       \affaddr{Padua, Italy}\\
       \email{spolaor.riccado@gmail.com}
\and  \alignauthor Nino~V.~Verde\\
       \affaddr{Sapienza University of Rome}\\
       \affaddr{Rome, Italy}\\
       \email{verde@di.uniroma1.it}
}










\begin{document}

\maketitle

\begin{abstract}
While smartphone usage become more and more pervasive, people start also asking to which extent
such devices can be maliciously exploited as ``tracking devices''. The concern is not only related
to an adversary taking physical or remote control of the device (e.g., via a malicious app),
but also to what a passive adversary (without the above capabilities) can observe from the device communications.
Work in this latter direction aimed, for example, at inferring the apps a user has installed
on his device, or identifying the presence of a specific user within a network.

In this paper, we move a step forward: we investigate to which extent it is feasible to identify the specific
actions that a user is doing on his mobile device, by simply eavesdropping the device's network traffic.
In particular, we aim at identifying actions like \hilight{browsing someone's profile on a social network, posting a message on a
friend's wall, or sending an email.}
We design a system that achieves this goal starting from encrypted TCP/IP packets: it works through identification of network 
flows and application of machine learning techniques. We did a complete implementation of this system and
run a thorough set of experiments, which show that it can achieve accuracy and precision higher than 95\%,
for most of the considered actions.
\end{abstract}

\section{Introduction}
\label{Introduction}

Smartphones become widely used and pervasive devices. People continuously carry those devices with them and use them more and more for daily communication activities, including not only voice calls and SMS but also emails and social network interaction.
In the last years, several concerns have been raised about the capabilities of those portable devices to invade the privacy of the users and actually becoming ``tracking devices''. In particular, one aspect is concerned with the possibility of continuously localize an individual \cite{iphonetrackposition:14,angelaTrack:14}.
Another relevant aspect is related to the fact that malicious apps can go even a step further in tracing and spying on someone life. For example, a malicious app that has access to the microphone and networking capabilities, could in principle continuously eavesdrop the audio and send it over the Internet to an adversary \cite{schlegel2011soundcomber}. 

Even when the adversary has no actual control of the phone (either physical control or remote via malicious apps) other attacks in the same directions are possible to violate the privacy of the communications. 
If the network traffic is not encrypted, the task of the eavesdropper is simple, since he can analyze the payload and read the content of each packet. However, many mobile apps use the \hilight{Secure Sockets Layer (SSL), and its successor Transport Layer  Security (TLS),} as a building block for encrypted communications.
In a typical SSL/TLS usage scenario, a server is configured with a certificate 
containing a public key as well as a matching private key. As part of the 
handshake between an SSL/TLS client and server, the server proves it has the 
private key by signing its certificate with public-key cryptography. 
Unfortunately there is often a gap between theory and practice, e.g., leveraging the SSL vulnerabilities of smartphone apps \cite{Fahl:2012:WEM:2382196.2382205,Georgiev:2012:MDC:2382196.2382204} one might run an SSL man-in-the-middle attack to compromise the confidentiality of communications. 


We believe that while people become more familiar with mobile technologies and their related privacy threats (also thanks to the attention raised by the media, e.g., see the recent attention on NSA for supposedly eavesdropping foreign governments leader such as Angela Merkel \cite{angelaTrack:14}), users start adopting some good practices that better adapt to their privacy feeling and understanding. For examples, solutions to identify and isolate malware running on smartphones \cite{InfFlow-TIFS,MOSES-TPDS,survey13suarez} as well as to protect against attacks coming from the network \cite{DBLP:ContiDG13mithys,LeoneConti.6732964} might significantly reduce current threats to user privacy.

Unfortunately, we believe that even adopting such good practices would not close the door to malicious adversaries willing to trace people. In fact, the wireless and pervasive nature of mobile devices would still leave many practical options for adversarial tracing. 
In particular, even when such solutions are in place, the adversary can still infer a significant amount of information from the properly encrypted traffic.
For example, work leveraging analysis of encrypted traffic already highlighted the possibility of understanding the apps a user has installed on his device \cite{Stober:2013:YSY:2462096.2462099}, or identify the presence of a specific user within a network \cite{NinoVerdeNATleftBehind}.

This work focuses on understanding whether the user profiling made through analyzing encrypted traffic can be pushed up to understand exactly what actions the user is doing on his phone: as concrete examples, we aim at identifying actions such as the user sending an email, receiving an email, browsing someone profile in a social network, \hilight{rather than publishing a post or a tweet.} The underlying issue we leverage in our work is that SSL and TLS protect the content of a packet, while they do not 
prevent the detection of networks packets patterns that instead may reveal some information about the user behavior.

\hilight{
An adversary may use our approach in several practical ways to threaten the privacy of the user.
In the following, we report some possible attacks:}
\begin{itemize}
\item \hilight{
A censorship government may try to identify a dissident who spreads anti-government propaganda 
using an anonymous social network account. Comparing the time of the public posts with the time of the actions
(inferred with our method), the government can guess the identity of that anonymous dissident.}
\item \hilight{By tracing the actions performed by two users, and taking into account the 
communication latency, an adversary may guess (even if with some probability of error) whether there is a communication 
between them. Multiple observations could reduce the probability of errors.}
\item \hilight{An adversary can build a behavioral profile of a target victim based on the habits of the latter one
(e.g., wake up time, work time).
For example, this could be used to improve user fingerprinting methods, to infer the presence of a 
particular user in a network}~\cite{NinoVerdeNATleftBehind}, \hilight{even when he accesses the network 
with different type of devices.}

\end{itemize}






\paragraph{Contributions}
In this paper, we propose a framework to infer which particular actions the 
user executed on some app installed on his mobile-phone, by only looking 
at the network traffic that the phone generates. In particular, we assume the traffic is encrypted and the adversary eavesdrops (without modifying them) the messages exchanged between the user's device and the web services that he uses. 

Our framework analyzes the network communications and leverages information available 
in TCP/IP packets (like IP addresses and ports), together with other 
information like the size, direction (incoming/outgoing), and timing. By 
using an approach based on machine learning, each app that is of interest is analyzed independently. 
To set up our system, for each app we first pre-process a dataset of network packets labeled with 
the user actions that originated them, we cluster them in flow typologies that represent recurrent network flows, and 
finally  we analyze them in order to create a training set that will be used to feed a 
classifier. The trained classifier will be then able to classify new traffic 
traces that have never been seen before.
We fully implemented our system, and we run a thorough set of experiments to evaluate our solution considering three very popular apps: Facebook, Gmail, and Twitter.
The results shows that it can achieve accuracy and precision higher than 95\%, for most of the considered actions done by the user with those apps.


 \paragraph{Organization}
The remainder of the paper is organized as follows. 
In Section \ref{Related Work}, we revise the state of the art around our research topic. 
In Section \ref{ML and DM tools}, we introduce some background knowledge, used in our work, on machine learning and data mining tools.
In Section \ref{OurFramework}, we present our framework, describing it in all its components.
We present the evaluation of our solution for identifying user actions in Section \ref{ExperimentalResults}, 
while in Section \ref{Countermeasures} we discuss about possible countermeasures against the attack.
Finally, in Section \ref{Conclusions} we draw some conclusions.



\section{Related Work}
\label{Related Work}

Our main claim in this paper is that network traffic analysis and machine learning can be used to infer
private information about the user, i.e., the actions that he executes with his mobile phone,
even thought the traffic is encrypted.
To position our contribution with respect to the state of the art, in this section we survey the works that belong to two main research areas that focus on  
similar issues: \textit{privacy attacks via traffic analysis} (not necessarily focusing on mobile devices) and \textit{traffic analysis of mobile devices} (not necessarily focusing on privacy). 





\paragraph{Privacy attacks via traffic analysis}

In the literature, several works proposed to track user activities on the web 
by analyzing unencrypted HTTP requests and responses~\cite{Atterer:2006:KUM:1135777.1135811,Schneider:2009:UOS:1644893.1644899,Benevenuto:2012:CUN:2169463.2169583}.
With this analysis it was possible to understand user actions inferring interests and habits. 
However, in recent years, websites and social networks started to use SSL/TLS encryption protocol, 
both for web and mobile services. This means that communications between endpoints are encrypted and this type of analysis cannot be performed anymore. 

Different works surveyed possible attacks that can be performed using traffic analysis assuming a 
very strong adversary (e.g., a national security agency) which is able to observe all communication 
links \cite{Raymond:2001:TAP:371931.371972,Berthold:2000:PLU:332186.332211}.
In~\cite{Liberatore:2006:ISE:1180405.1180437}, Liberatore et al.\ evaluated the effectiveness of two traffic analysis 
techniques based on naive Bayes and on Jaccard’s coefficient for identifying encrypted HTTP streams. 
Such an attack was outperformed by~\cite{Herrmann:2009:WFA:1655008.1655013}, where the authors presented 
a method that applies common text mining techniques to the normalized frequency distribution of observable 
IP packet sizes, obtaining a classifier that correctly identifies up to 97\% of requests.
The technique was further refined in~\cite{Panchenko:2011:WFO:2046556.2046570}, where the authors presented a support 
vector machine classifier that was able to correctly identify web pages, even when the victim used both encryption and 
anonymization networks such as Tor~\cite{Dingledine:2004:TSO:1251375}.
\hilight{Finally, Cai et al.}~\cite{cai2012touching} 
\hilight{ presented a web pages fingerprinting attack and proved its effectiveness despite traffic analysis countermeasures, such as Tor randomized pipelining} ~\cite{tor:websites:fingerprint} or HTTPOS~\cite{Luo11httpos:sealing}.

Unfortunately, none of the aforementioned works was designed for (or could easily be extended) to mobile devices. In fact, all of them focus on web pages identification in desktop environment (in particular, in desktop browsers), where the generated HTTP traffic strictly depends on how web pages are designed. Conversely, mobile users mostly access the contents through the apps installed on their devices~\cite{Go:2013:TAA:2444776.2444779}. These apps communicate with a service provider (e.g., Facebook) through a set of APIs.
\hilight{An example of such differences between desktop web browsers and mobile apps is the validation of SSL certificates}~\cite{Georgiev:2012:MDC:2382196.2382204,DBLP:ContiDG13mithys}.

Traffic analysis has been applied not only to HTTP but also to other protocols. 
For example, Song et al.~\cite{Song:2001:TAK:1251327.1251352} prove that SSH is not secure. In particular, they show that 
 even very simple statistical techniques suffice to reveal sensitive 
information such as login passwords. More importantly, the authors show that by using 
more advanced statistical techniques on timing information collected from the 
network, the eavesdropper can also learn significant information about what users 
type in SSH sessions. 
SSH is not the only 
protocol that has been target of such attacks. 
Another example is Voice Over IP (VoIP). In particular, in~\cite{Wright:2008:SMY:1397759.1398055}, the authors 
show how the length of encrypted VoIP packets can be used to identify spoken 
phrases of a variable bit rate encoded call. Their work indicates that a profile 
Hidden Markov Model trained using speaker- and phrase-independent data can 
detect the presence of some phrases within encrypted VoIP calls with recall and 
precision exceeding 90\%.

In ~\cite{shuosp2010}, \hilight{the authors show that despite encryption, 
also web applications suffer from side-channel leakages. The system model considered is different from our. 
In particular, their focus is on web applications. On the contrary, we focus on mobile applications. 
More importantly, the authors leverage three fundamental features of web applications: stateful communication; low entropy input; 
significant traffic distinction. 
We believe that in most mobile applications two of these features (stateful communication, low entropy input) 
are not very useful to characterize user actions. In contrast to this work, we adopt a solution that only needs 
information about packet sizes and their order.
}


\paragraph{Traffic analysis of mobile devices}
Focusing on mobile devices, traffic analysis has been successfully used to detect 
information leaks \cite{Enck:2010:TIT:1924943.1924971}, 
to profile users by their set of installed apps~\cite{Stober:2013:YSY:2462096.2462099}, 
\hilight{and to produce the fingerprint of an app from its unencrypted HTTP traffic} \cite{DaiTWNS:2013}.
Traffic analysis has also been used to understand network traffic characteristics, 
with particular attention on energy saving \cite{Falaki:2010:FLT:1879141.1879176}. Stober et al. \cite{Stober:2013:YSY:2462096.2462099} shown that it is possible to identify the set of apps installed on an Android device, 
by eavesdropping the 3G/UMTS traffic that those apps generate.
Similarly, Tongaonkar et al. \cite{DaiTWNS:2013} \hilight{introduced an automatic app profiler that creates the network fingerprint
of an Android app relying on packet payload inspection, in order to re-identify its HTTP traffic.} 
In~\cite{zhouccs2013}, \hilight{Zhou et al. present a work that take in consideration the traffic produced a user action performed with Twitter app.
Unfortunately, the authors focused on a single user action (i.e., send a tweet) without distinguish that action from the other ones a user could perform.}

None of the works mentioned in this section aim at \hilight{inferring and distinguish user actions performed by the user with his mobile apps},
which is the goal of our paper.




\section{Machine Learning and Data Mining Background}
\label{ML and DM tools}
In this section, we briefly recall several machine learning and data mining concepts that we use in our paper, while we point the reader to appropriate references for a complete introduction on those topics.


\subsection{Dynamic Time Warping}
\label{DTW}
Dynamic Time Warping (DTW)~\cite{Muller:2007:IRM:1324818} is a useful method to find
alignments between two time-dependent sequences (also referred as time series) which 
may vary in time or speed. This method is also used to measure the distance or 
similarity between time series.

Let us consider two sequences that 
represent two discrete signals:
 of length 
; and   of length .
DTW uses a local distance measure 
 to calculate 
a cost matrix , s.t., each cell  reports the distance between  and . 
The goal is to find an alignment between  and  having minimal 
overall distance. Intuitively, such an optimal alignment runs along a ``valley'' of 
low cost cells within the cost
matrix . More formally, a \emph{warping path} is defined as a sequence 
 with ,  
satisfying the following three conditions:
\begin{enumerate}
 \item Boundary condition:  and ;
 \item Monotonicity condition:  and ;
 \item Step size condition:  for 
.
\end{enumerate}
The total cost of a 
warping path is calculated as the sum of all the local distances of its 
elements. An \emph{optimal warping path} is a warping path  
having 
minimal total cost among all possible working paths. The total cost of an 
\emph{optimal warping path} is also used as a 
distance measure between two sequences  and . In this paper, we will 
indicate the cost of an \emph{optimal warping path} with .


\subsection{Hierarchical Clustering}
\label{Hierarchical clustering}  

Hierarchical clustering is a cluster analysis method which 
seeks to build a hierarchy of clusters. This clustering method has the distinct 
advantage that any valid measure of distance can be used. In fact, 
the observations themselves are not required: all that is used is a matrix of 
distances.

In the following we will use a type of hierarchical clustering that is called 
agglomerative: each observation starts in its own cluster, and pairs 
of clusters are merged as one moves up the hierarchy. In order to decide which 
clusters should be combined, a metric (a measure of distance between 
pairs of observations) and a linkage criterion are required. Since we will 
clusterize time-dependent sequences, we will use the total cost of an 
\emph{optimal warping path} as distance metric. As for the linkage 
criterion, that determines the distance between sets of observations as a 
function of the pairwise distances between observations, we will use the 
average 
distance, that is defined as:
 
where  is a distance function, and  and  are two clusters of  and  elements, respectively.
More details about Hierarchical clustering can be found in~\cite{hastie09statisticallearning}.


\subsection{Supervised Learning}
 \label{Classification}
Supervised machine learning algorithms learns from labeled instances or 
examples, which are collected in the past and represent past experiences in 
some real-world applications. They produce an inferred model, which can be then 
used for mapping or classifying new instances. An optimal scenario will allow 
for the algorithm to correctly determine the class labels for unseen instances.

In this paper, we will use an ensemble classifier that is called Random 
Forest~\cite{Breiman:2001:RF:570181.570182}.
The main principle behind ensemble methods is that a group of ``weak learners'' 
can be combined together to form a ``strong learner''. 
Random forest leverages a standard machine learning technique called 
 ``decision tree'', which, in ensemble terms, corresponds to the weak learner. 
In practice, it combines together the results of several decision trees trained 
with different portions of the training dataset and different subsets of 
features. More details about the Random Forest classifier can be 
found in~\cite{Breiman:2001:RF:570181.570182}.




\section{Our Framework}
\label{OurFramework}
In this section we describe our framework. In particular, Section~\ref{sec:networkTrafficModellingAndPreprocessing} introduces the 
pre-processing steps that allow us to model the network traffic. Section~\ref{ClassificationUserActions}
describes the methodology used to build training and test dataset, and the procedure used to classify user actions. 









\subsection{Network Traffic Pre-Processing}
\label{sec:networkTrafficModellingAndPreprocessing}
Mobile apps generally rely on SSL/TLS to securely communicate with 
peers. These protocols are built on the top of the TCP/IP suite. 
The TCP layer receives encrypted data from the above layer, it divides data 
into chunks if the packets exceeds a give size. Then, for each chunk it adds a TCP header
creating a TCP segment. Each TCP segment is encapsulated into an Internet Protocol (IP) datagram, and exchanged with 
peers.
Since TCP packets do not include a session identifier, both endpoints 
identify a TCP session using the client's IP address and the port number.



A fundamental entity considered in this paper is the traffic \emph{flow}: with this 
term we indicate a time ordered sequence of TCP packets exchanged between two peers 
during a single TCP session. 
We model each network flow as a set of time series:
(i) a time series is obtained by considering the bytes transported by 
incoming packets only; (ii) another one is obtained by considering bytes transported by 
outgoing packets only; (iii) a third one is obtained by combining (ordered by time) bytes transported by both incoming and 
outgoing packets. Additional time series may be obtained for example
by considering other parameters such as the time-gap between different packets. However, we will only use the first three types mentioned above.
\hilight{The use of time series allows to represent network flows making them independent from the properties of the connection. 
}
Table~\ref{tab:exampleOfTS} reports an example of time series generated from three network flows, while Figure~\ref{fig:fbflowsoverlap}
graphically represents these flows through a cumulative chart. The lower side of the chart represents incoming traffic, while
the upper side represents outgoing traffic. This is only one of the possible representations, and it shows that the ``shapes''
of these three network flows are quite different. Intuitively, our classification approach aims to identify the ``shape'' of \hilight{unseen flows}.

\begin{table*}[ht!]
       \centering
        \begin{center}
\begin{tabular}[t]{|l|l|l|}\hline
\textbf{Flow ID} & \textbf{Time series type} & \textbf{Time series}\\\hline
\multirow{2}{*}{Flow 1} & Incoming & [1514, 1514, 315, 113, 477]\\
 & Outgoing & [282, 188, 514, 96, 1514, 179, 603, 98, 801, 98]\\
 & Complete & [282, -1514, -1514, -315, 188, -113, 514, 96, 1514, 179, 603, 98, 801, 98, -477]\\\hline
 \multirow{2}{*}{Flow 2} & Incoming & [1514, 1514, 1266, 582, 113, 661]
\\
 & Outgoing & [282, 188, 692, 423]\\
 & Complete & [282, -1514, -1514, -1266, -582, 188, -113, 692, 423, -661]
\\\hline
  \multirow{2}{*}{Flow 3} & Incoming & [1245, 1514, 107, 465, 172, 111]
\\
 & Outgoing & [926, 655, 136, 913, 1514, 1514, 863]
\\
 & Complete & [926, 655, 136, -1245, 913, 1514, 1514, 863, -1514, -107, -465, -172, -111]
\\\hline
        \end{tabular}
        \end{center}
\caption{Example of time series generated from three network flows. Values within square brackets
represent the amount of bytes exchanged per packet: negative values in complete time series indicate incoming bytes,
while positive values indicate outgoing bytes.}
\label{tab:exampleOfTS}
\end{table*}    



\begin{figure} [h!]
       \centering
    \epsfig{file=plotClusterFlowsFill_facebook_merge_067_TEST_TWA6_longr1_tw-4_s7_statuspostselection_cut_06.png,width=8.5cm}
    \caption{Representation of flows time series.}
    \label{fig:fbflowsoverlap}  
\end{figure}



Before generating for each flow the corresponding set of time series, 
a few pre-processing steps have to be 
performed. In particular: 1) we apply a domain filtering to select only flows belonging to the analyzed 
app; 2) we filter the remaining flows, in order to delete packets that 
may degrade the precision of our approach (i.e., we filter out ACK and retransmitted 
packets); 3) we limit the length of the generated time-series.
In the following, we will detail these three pre-processing steps.




\paragraph{Domain filtering}
\label{Domain filtering}

\hilight{The network traffic generated by an application is generally directed toward a back-end infrastructure, that may be identified with a single server, or a set of servers that may be even behind a load balance. Since we analyze each app independently, we need to make 
sure that traffic generated from apps other than the considered one (or traffic generated by the OS) do not interfere with the 
analysis. Different methods can be used in order to identify the app that generated each network flows. The destination IP address is a trivial discriminating parameter. However, in case of a load balanced back-end, we should know all the individual IP addresses that can be involved in the communication. The same happens when the back-end is composed by several components such as different web services, databases, etc. To overcome this problem we use another strategy: we take into consideration for further analysis only the flows which destination IP addresses owners have been clearly identified as related to the considered app. In the implementation of our framework, we leverage the WHOIS protocol for this purpose, but we want to highlight that this is only one of the possible way. Business and other context information may be used in order to perform the domain filtering.
}
We also take into consideration the traffic related to third parties services (such as Akamai or Amazon) that are indeed used by several applications~\cite{Wei:2012:PMP:2348543.2348563}. 




\paragraph{Packets filtering}
\label{PacketFiltering}

Due to network congestion, traffic load balancing, or other unpredictable 
network behavior, IP packets can be lost, duplicated, or delivered out of order. 
TCP detects these problems, hence requesting retransmission of lost data, and 
reordering out-of-order data. It comes out that several TCP packets that do not 
carry data, may hinder the analysis process.
In the data exchange phase, for example, the receiver sends a packet with the ACK flag set to notify the correct reception of a chunk of data. 
These ACK packets are transmitted in asynchronous mode so they are affected by many factors related to round trip time of the connection link. 
The order of the received packets may hinder the evaluation of the similarity between two network flows.
For this reason, we filter out all packets retransmissions, as well as packets marked with the ACK flag.
\hilight{Note that the metric that we will use in order to measure similarity between flows (see Section }
\ref{ClassificationUserActions}) \hilight{will mitigate the consequences of a missing packet.
}
We filter out also other packets that do not bring any additional information helpful in characterize flows. In particular, we filter out the three way handshake executed to open a TCP connection, and the packets exchanged to close it.





\paragraph{Timeout and packets interval}
\label{ActiveInactiveTimeouts}




Two different techniques are used to limit the length of the generated time series: 
a \emph{timeout} mechanism and the specification of a \emph{packets interval}. 
The \emph{timeout} mechanism is used to terminate the flows that did not receive any new packet
since  seconds. Indeed, it has been proved experimentally that 95\% of all packets arrive at most  seconds after
their predecessors~\cite{Stober:2013:YSY:2462096.2462099}. The \emph{packets interval} 
specifies the first and the last packet to be considered. 
For example, considering a flow  composed by  packets, and the interval  with  and , 
the corresponding time series will be composed by  values that report the bytes of the 
 to the  packet. 
This simple 
mechanism allows us to focus on particular portions of the flow. The first part, for example, is often the 
more significant. In the experimental part, we report the results for different 
configurations of packets intervals, showing that the best configuration is 
app dependent.  










  


\subsection{Classification of User Action} 
\label{ClassificationUserActions}


\label{LabeledDataset}
Since we use a supervised learning approach, it is necessary to create a labeled dataset that
describes the user actions that we want to classify. 
In order to build this dataset, we simulate a series of user actions, and for each one we identify the 
flows generated after the execution of the action itself. For each app that we analyze
we focus on actions that are significant for that particular app. 

In most cases, a single user action generates a set of different flows (i.e., not just a single one). 
Furthermore, different user actions may generate different sets of flows. 
Our classification method is based on the detection of the sets of flows that 
are distinctive of a particular user action.
In order to elicit these distinctive sets of flows, we build clusters
of flows by using the agglomerative clustering approach described in 
Section~\ref{Hierarchical clustering}. 
\hilight{Flows that are similar one to each other will be grouped together in the same cluster, while not similar flows 
will be assigned to in different clusters.} The average distance is used as 
linkage criterion, while the computation of the distance between two flows combines
the distances of the corresponding time series. 
Supposing that each 
flow  is decomposed into a set of  time series 
, the distance between  and  is defined as:

where  is a weight assigned to the particular time series. Weights can be 
assigned in such a way to give more importance to some type of time series with 
respect to others. For example, it is possible to give more weight to the 
time series that represent incoming packets, and less weight to those that 
represent outgoing packets. 




In order to reduce the computational burden of the subsequent classification, a 
leader is elected for each cluster. 
Leaders will be the representative flows of their clusters. 
Given a cluster  containing the flows 
, the leader is elected by selecting the flow  that has 
minimum overall distance from the other members of the cluster, that is: 



Clustering is executed over the set of flows that will be used to build the training dataset. 
\hilight{The cluster leaders will be used to build both the training and the test datasets. The user 
actions will be the instances of the datasets, while the class of each instance is a label representing 
the action. We will have one integer feature for 
each cluster identified through the agglomerative clustering. The value of
each feature is determined 
by analyzing the flows related to an action.}
Each flow  captured after the execution of an action will 
be assigned to the cluster that 
minimizes the distance between  and the leader of the cluster.
The  feature 
will therefore indicate the number of flows that have been 
assigned to the cluster  after the execution of that action.
For example, for the 
action \emph{send mail}, the  feature will be equal to  
if there are  flows labeled with \emph{send mail}
assigned to the cluster .

Finally, we execute the classification with Random Forest algorithm. 
The main idea behind the overall approach is that different actions will ``trigger'' different
sets of clusters. The classification algorithm will therefore learn which are these sets, 
and will be able to correctly determine the class labels for unseen instances. 





\section{Experimental Results}
\label{ExperimentalResults}


In order to assess the performance of our proposal, we considered three widespread apps: Gmail, Facebook and Twitter. 
\hilight{We select these apps because of their high popularity}~\cite{androidRank}.
\hilight{Indeed, Gmail is the world's largest email service}~\cite{gmailbeatshotmail},
\hilight{and its Android app has over one billion downloads.}
\hilight{On the other hand, Facebook and Twitter are not only the most popular Online Social Networks}~\cite{ebizmbaRank},
\hilight{but they also had a leading role in the Arab spring}~\cite{fbtwarabstring}
\hilight{and the Istanbul's Taksim Gezi Park protests}~\cite{fbtwturkey2}
\hilight{(when Turkish government blocked Twitter)}.
\hilight{We believe that the results we obtain in our analysis also hold for other apps that provide similar functionalities (e.g., Yahoo mail, WhatsApp or LinkedIn), while a thorough evaluation of this claim is left as future work. }To collect the network traffic related to different user actions, we set up a \hilight{monitored environment.} 
In this section we present the elements that compose this environment (Section~\ref{HDWandNetworkConfiguration}), 
the methodology used to collect the data (Section~\ref{datacollection}), and the results of the evaluation (Section~\ref{ClassificationPerformances}). 


\subsection{Hardware and Network Configuration}
\label{HDWandNetworkConfiguration}



For the evaluation of our solution, we used a Galaxy Nexus (GT-I9250) smartphone, running the Android 4.1.2 (Jelly bean) 
operative system.
We enabled the ``\emph{Android Debug}'' option in order to allow the usage of the ADB (Android Debug Bridge) interface via USB cable. 
We used a Wi-Fi access point 
(U.S. Robotics USR808054) to provide wireless connectivity to the mobile phone. 
Finally, we used a server (Intel Pentium Processor dual core E5400 2.7GHz with 4 GB DDR2 RAM)
with two network cards running Ubuntu Server 11.04 LTS to route the traffic from the access point to the Internet, and vice versa. 

To eavesdrop network packets flowing through the server, we used Wireshark 
software. From a Wireshark capture file, we 
created a comma separated file (csv), where each row describes a
packet captured from the access point's interface. 
For every packet we reported source and destination IP addresses, ports, size in bytes and
time in seconds from Unix epoch (i.e., 00:00:00 UTC, 01 January 1970), protocol type and TCP/IP flags. Since the payload is not relevant to our analysis, it has been omitted. 
This data have been then used to generate the time series as 
explained in Section~\ref{sec:networkTrafficModellingAndPreprocessing}.


\subsection{Dataset Collection and Analysis}
\label{datacollection}

For our study we considered three apps installed from the official Android market: Gmail v4.7.2 , Facebook v3.8, and Twitter v4.1.10.
For each app, we created 10 accounts that have been considered in two different categories of users: ``\emph{active}'' and 
``\emph{passive}'' users. ``\emph{Active}'' users simulated the behavior of 
users that actively use the app by sending posts, email, tweets, surfing 
the various menus, etc. ``\emph{Passive}'' users simulated the behavior of 
users that passively use the app, just by receiving messages or posts. 
The accounts of both passive and active users have been configured in such a way 
to have several friends/followers within the group. We avoided to 
configure the accounts with actual friends or followers, in order to avoid 
interference
due to notifications of external users activities that were not under our control.

To reach a particular target, a user may have to perform several actions in 
a precise order. An action could 
be simple (e.g., a tap on a button, a swipe, or a selection of 
edit box), or complex (e.g., type a text, which is a sequence of keyboard inputs).
For example, a user has to perform three actions in a precise sequence to post a message on his Facebook wall.
He has to be sure that the Facebook app shows the ``\emph{user's wall}'', 
then he has to tap on the ``\emph{write a post}'' button (1), fill the edit box with some text (2), and finally tap on the ``\emph{post}'' button (3). It is important to highlight that we do not use static text to fill in 
text boxes, but the text is randomly selected from a large set of sentences.
A script submits the sequence of actions to the
mobile phone through the ADB commands, and it 
captures the network traffic that is generated.
\hilight{The script records also the execution time of each action.
By using the recorded execution time of each action, it is then possible to label the flows extracted from the 
network traffic with the user action that produced it.}
\hilight{For each app, we choose a set of actions that are more sensitive than others from user privacy point of view (e.g., send an email or a message, for the reasons we report in} Section~\ref{Introduction}).
The list of these actions is reported in Table~\ref{tab:appcations}. 
\hilight{We underline that we do not ignore other user actions, but we label them as \emph{other}.
In such a way we have several benefits}~\cite{mitchell1997machine}:
\hilight{we obtain a greater representation of data in terms of variety and variance of examples; 
we reduce the chances of overfitting; we improve the performance of the classifier on relevant user actions.}


\hilight{We collected and labeled the traffic generated by 220 sequences of actions for each app, where a sequence is composed by 50 types of actions 
(for a total of  examples of actions for Gmail,  for Twitter, and  for Facebook).
The user action examples in the dataset was divided in a training set  and a test set.
We use the training set to train the classifier, while we use the test set to evaluate its accuracy. 
We underline that to build the test set we used accounts that have not been used to create the training set.
By using different accounts to generate 
the training and the test set, it is possible to assure that the results 
of the classification do not depend on the specific accounts that have been 
analyzed.}






 
\begin{table}[h!] 
\begin{center} {\begin{tabular}{|l|l|}
\multicolumn{2}{c}{}  \\
  \multicolumn{2}{c}{\textbf{Facebook}}  \\
  \hline
  \textbf{Action} & \textbf{Description}\\
  \hline
  \emph{send message} & {send a direct message to a friend} \\
  \emph{post user status} &  { post a status on the user's wall}\\
  \emph{open user profile} &   {select user profile page from menu}\\
  \emph{open message} &	{select a conversation on messages page}\\
  \emph{status button} & 	{select ``\emph{write a post}'' on user's wall} \\
  \emph{post on wall} & 	{post a message on a friend's wall}  \\
  \emph{open facebook} & 	{open the Facebook app} \\
 
 \hline
 \multicolumn{2}{c}{}  \\
 \multicolumn{2}{c}{\textbf{Gmail}}  \\
 \hline
  \textbf{Action} & \textbf{Description}\\
  \hline
 \emph{send mail} &  { send a new mail} \\
\emph{reply button}    &  {tap on the reply button} \\
  \emph{open chats}    &  {select chats page from menu}\\
\emph{send reply} &   {send a reply to a received mail}\\

  \hline
  \multicolumn{2}{c}{}  \\
  \multicolumn{2}{c}{\textbf{Twitter}}  \\
  \hline
  \textbf{Action} & \textbf{Description}\\
  \hline
  \emph{refresh home}   &  { Refresh the home page}  \\
  \emph{open contacts}   &  {select contacts on menu}   \\
  \emph{tweet/message}&  { publish tweet or send message}\\ 
  \emph{open messages}  &  { select direct messages page}\\
  \emph{open twitter}    &  { open the Twitter app}\\
  \emph{open tweets}     & { select tweets page}\\

  \hline
\end{tabular} }
\end{center}
\caption{Description of the relevant actions for each app.}
\label{tab:appcations}
\end{table}





As explained in Section~\ref{sec:networkTrafficModellingAndPreprocessing}, 
each network flow is modeled as a set of time series. 
Table \ref{tab:weights} reports the weights and the intervals for several configurations (``Conf." in the table) used to limit the length of the time series generated by each app.
We used different weights configurations, and we selected the packets
intervals by analyzing the statistical length of the flows. 
Figure~\ref{fig:distributionflowlength} reports the
statistical distribution of the length of the flows app by app. 
The first quartile, the median and the third quartile are highlighted by 
using a notched box plot. In particular, the median value and the third quartile have been used 
as thresholds to limit the maximum length of the generated time series. 
For the Twitter app, in some cases we set the interval in such a way to focus 
only on the last three or four packets.
Indeed, we noticed that the first part of the time series was identical for each flow.



\begin{table}[h!]
       \centering
        \begin{center}
\begin{tabular}[t]{|l|l|c|c|c|c|}\hline
\textbf{{Apps}} & \textbf{{Sets}} & \textbf{{Weights}} & \textbf{{In}} & \textbf{{Out}} & \textbf{{Complete}} \\
\hline
\multirow{6}{*}{{Gmail}} 
& \multirow{2}{*}{{Conf.~1}} & 0.80 & [1,4]& [1,2]& [1,6] \\
		                 &  & 0.20 & [1,6]& [1,3]& [1,9]\\ \cline{2-6}
& \multirow{2}{*}{{Conf.~2}} & 0.66 & [1,4]& [1,2]& [1,6] \\
		                 &  & 0.33 & [1,6]& [1,3]& [1,9]\\ \cline{2-6}
& \multirow{2}{*}{{Conf.~3}} & 0.33 & [1,4]& [1,2]& [1,6] \\
		                 &  & 0.66 & [1,6]& [1,3]& [1,9]\\
\hline \multirow{6}{*}{{Facebook}}
& \multirow{2}{*}{{Conf.~1}} & 0.66 & [1,3]& [1,5]& [1,7] \\
				 &  & 0.33 & [1,6]& [1,7]& [1,12]\\ \cline{2-6}
& \multirow{2}{*}{{Conf.~2}} & 0.33 & [1,3]& [1,5]& [1,7] \\
				 &  & 0.66 & [1,6]& [1,7]& [1,12]\\ \cline{2-6}
& \multirow{2}{*}{{Conf.~3}} & 0.20 & [1,3]& [1,5]& [1,7] \\
				 &  & 0.80 & [1,6]& [1,7]& [1,12]\\ \cline{2-6}
\hline \multirow{6}{*}{{Twitter}} 
& \multirow{2}{*}{{Conf.~1}} & 0.95 & - & - &  [7,10] \\ &  &  0.05 & - & -& [1,10]\\ \cline{2-6}
 & \multirow{2}{*}{{Conf.~2}} & 0.95 & - & -& [8,11] \\ &  &  0.05 & - & -& [1,11]\\ \cline{2-6}
& \multirow{2}{*}{{Conf.~3}} & 0.95 & - & -& [8,10] \\ &  &  0.05 & - & -& [1,10]\\ 
\hline

\end{tabular}
        \end{center}
\caption{Weights set configurations and packets intervals for Gmail, Facebook and Twitter apps.}
\label{tab:weights}
\end{table}    


\begin{figure}[h!]
    \centering
\epsfig{file=plotFlowLengthStudy_merge_067_TEST_TWA6_longr1_tw-4_s7.png,
width=\linewidth}
    \caption{Statistical distribution of the length of the complete time series 
extracted from the network traffic. First and third quartile are 
represented as the left and right side of the notched box. The notch of the box 
represents the median value. Lines that extend horizontally from the boxes 
indicate the ~percentile (left) and the ~percentile (right).}
    \label{fig:distributionflowlength}
  \end{figure}

  

To confirm this statement we report in 
Figure \ref{fig:gmailflows} and Figure \ref{fig:twitterflows} the 
graphical representation of the flows that occur when executing three different 
actions in Gmail and Twitter respectively. Comparing the two figures, it can be 
noticed that the shapes of the actions drastically change for Gmail, while they 
are almost unvaried for Twitter. As a matter of fact, different Twitter actions 
just differ in their last packets. Nevertheless, our approach 
reaches very good performance also for this app.
\hilight{In our experiments, we used the Random forest classifier implemented by \textit{scikit-learn} libraries}~\cite{scikitlearn}.
\hilight{The classifier is trained using 40 estimators, with the square root of the number of the features as max feature selection value for each estimator,
and the bootstrap option activated. Each estimator (i.e., weak learner) consists in a decision tree without any restrictions on its depth limit.}



\begin{figure} [h!]
        \centering
        \begin{minipage}[b]{0.45\textwidth}
                
\includegraphics[width=8.20cm]{plotFlowFlows4_gmail_merge_067_TEST_TWA6_longr1_tw-4_s7.png}
    \caption{Representation of three different Gmail actions.}
    \label{fig:gmailflows}
        \end{minipage}\\
        \vspace{0.3cm}
\begin{minipage}[b]{0.45\textwidth}
\includegraphics[width=8.30cm]{plotFlowFlows4_twitter_merge_067_TEST_TWA6_longr1_tw-4_s7.png}
    \caption{Representation of three different Twitter actions.}
    \label{fig:twitterflows}
        \end{minipage}
\end{figure}


\begin{figure}[h!]
    \centering
    \vspace{0.2cm}
\includegraphics[width=8.60cm]{plotf1_0_RandomForestClassifier.png}
    \caption{Classification accuracy over number of clusters.}
    \label{fig:accuracyNClusters}
  \end{figure}














  
 
 
  



 
  




\subsection{Classification Performance}\label{ClassificationPerformances}  

One of the issues to discuss before proceeding to the classification of the user actions
is \hilight{the number of clusters to consider.} In order to establish a reasonable value for this parameter, we used a validation dataset
to study the accuracy of the classification when varying the number of clusters. Figure~\ref{fig:accuracyNClusters}
reports the achieved results. For each app, we therefore considered
the number of clusters that maximized the accuracy, in terms of averaged F-measure. In the following, we report the 
results of the classification app by app. 
In particular, we discuss the average accuracy reached 
when detecting each sensitive user action, we report the corresponding confusion matrices, and 
detailed results for the precision, the recall and the F-measure metrics. 






\subsubsection{Facebook}
  \label{ClassPerf:Facebook}

We focused on seven different actions that may be sensitive when using the 
Facebook app. On average, the F-measure is equal 
to 99\%, with a precision and a recall of 99\% and 98\% respectively. 
Performance reached with different configurations of weights and packets intervals 
constraints are reported in Figure~\ref{fig:facebookperf}. 
For each action at least one of the configurations exceeds 94\% of 
accuracy, while the 
worst performing is always higher than 74\%. 

Table \ref{tab:facebookreport} reports precision, recall and F-measure reached 
by using {Configuration 3}. We noticed that all the actions have a precision higher 
96\%. The recall is higher than 95\% for all the actions but the \emph{open user profile}, that reaches 91\%. 
In effect we realized that this particular action is 
classified as \emph{other} in 9\% of the examples, as we can see from the confusion matrix 
reported in Figure~\ref{fig:facebookconfusion}. 

  \begin{table}[h!]
\begin{center} {\begin{tabular}{|l|r|r|r|}
  \hline
\textbf{Actions} & \textbf{Precision}& \textbf{Recall} & \textbf{F-measure}\\
  \hline
  \emph{send message} & 1.00 & 1.00 & 1.00 \\ 
  \emph{post user status} & 1.00 & 0.95 & 0.97 \\ 
\emph{open user profile} & 0.96 & 0.91 & 0.94 \\ 
\emph{open message} & 0.98 & 1.00 & 0.99 \\ 
\emph{status button} & 1.00 & 1.00 & 1.00 \\ 
\emph{post on wall} & 1.00 & 0.98 & 0.99 \\ 
\emph{open facebook} & 1.00 & 1.00 & 1.00 \\ 
\emph{other} & 0.99 & 1.00 & 0.99 \\ 
\hline
Average & 0.99 & 0.98 & 0.99 \\ 
  \hline
\end{tabular} }
\end{center}
\caption{ Classification results of Facebook actions by using {Configuration~3}.}

\label{tab:facebookreport}
\end{table}


\subsubsection{Gmail}
\label{ClassPerf:Gmail}
We analyzed four specific user actions of the Gmail app: \emph{send mail}, 
\emph{reply button}, \emph{open chats} and \emph{send reply}. 
Figure~\ref{fig:gmailperf} shows the classification accuracy that has been 
reached. We observe that we are able to distinguish with high 
accuracy the action of sending of a new mail, from that of replying to a 
previously received message, as well as the tap over the reply button. The 
\emph{open chats} action is instead more difficult to distinguish. 
Table~\ref{tab:gmailreport} reports precision, recall and F-measure for 
different configurations of weights and packets intervals 
constraints. We can observe that the action \emph{open chats} 
(that allows to read past chats) achieves a 
low precision but a high recall. Analyzing the confusion matrix depicted in 
Figure ~\ref{fig:gmailconfusion} it is possible to notice that 
 of \emph{other} actions wrongly classified as \emph{open 
chats}. This is the reason of such a low precision.

\begin{table}[t]
\begin{center} {\begin{tabular}{|l|r|r|r|}
    \hline
\textbf{Actions} & \textbf{Precision}& \textbf{Recall} & 
\textbf{F-measure}\\
    \hline
 \emph{send mail} &      1.00   &   1.00 &     1.00 \\
\emph{reply button}    &   0.85    &  1.00   &   0.92 \\
  \emph{open chats}    &   0.36    &  0.94  &    0.52 \\
\emph{send reply} &      0.98   &   1.00   &   0.99 \\
\emph{other}   &    0.99  &    0.82   &   0.90  \\
\hline
Average    &   0.83   &   0.85  &    0.86  \\ \hline
\end{tabular} }
\end{center}
\caption{Classification results of Gmail actions reached by using {Configuration~1}.}
\label{tab:gmailreport}
\end{table}

\begin{figure*}
        \centering
        \begin{minipage}[t]{0.45\textwidth}
    \includegraphics[width=8cm]{facebook_RandomForestClassifier_60_f1scores_cut_10.png}
	\caption{Classification accuracy of the Facebook user actions.}
	\label{fig:facebookperf}
        \end{minipage}\hfill
        \begin{minipage}[t]{0.45\textwidth}
              \centering
\includegraphics[width=8cm]{facebook_RandomForestClassifier_60_TWA6_longr0_tw-3_s7_cut_10.png}
  \caption{Facebook user actions confusion matrix for Configuration 3.}
  \label{fig:facebookconfusion}  
        \end{minipage}\\
\begin{minipage}[t]{0.45\textwidth}
	  \centering
	\includegraphics[width=8cm]{gmail_RandomForestClassifier_60_f1scores_cut_06.png}
	  \caption{Classification accuracy of the Gmail user actions.}
	  \label{fig:gmailperf}
        \end{minipage}
        \hfill
        \begin{minipage}[t]{0.45\textwidth}
                \centering
\includegraphics[width=8cm]{gmail_RandomForestClassifier_60_TWA6_short0_tw-3_s6_cut_06.png}
    \caption{Gmail user actions confusion matrix for Configuration 1.}
    \label{fig:gmailconfusion}    
        \end{minipage}
        \\
        \begin{minipage}[t]{0.45\textwidth}
	  \centering
      \includegraphics[width=8cm]{twitter_RandomForestClassifier_60_f1scores_cut_11.png}
	  \caption{Classification accuracy of the Twitter user actions.}
	  \label{fig:twitterperf}
        \end{minipage}
	\hfill
	\begin{minipage}[t]{0.45\textwidth}
      \centering
\includegraphics[width=8cm]{twitter_RandomForestClassifier_60_TWA6_short1_tw-4_s6_cut_11.png}
  \caption{Twitter user actions confusion matrix for Configuration 1.}
  \label{fig:twitterconfusion}           
        \end{minipage}
\end{figure*}

\subsubsection{Twitter}
\label{ClassPerf:Twitter}
During the analysis we noticed that Twitter actions may be more 
difficult to classify than Gmail and Facebook actions. Indeed, different 
Twitter actions generate similar time series that have in common a large 
portion. Only the last three or four packets of each time series show some 
difference. Nevertheless, we have been able to reach outstanding results also 
for this app. In particular, we focus on six specific user actions: \emph{refresh 
home}, \emph{open contacts}, \emph{tweet/message},  \emph{open messages}, 
\emph{open twitter}, \emph{open tweets}. 

On average, the F-measure is equal 
to 97\%, with a precision and a recall of 98\% and 97\% respectively (see 
Table~\ref{tab:twitterreport}). 
Performance reached are reported in Figure~\ref{fig:twitterperf}. 
For each action at least one of the configurations exceeds 96\% of 
accuracy, while the worst configuration has an accuracy in any case higher than 91\%.
The action \emph{open twitter} has accuracy and recall equal to  
100\%, independently of the Configuration set used for the clustering phase. As a 
consequence, none of examples of the test set have been wrongly classified. 
Figure~\ref{fig:twitterconfusion} reports the confusion matrix obtained by 
considering the Twitter actions.
Three of the six analyzed actions are correctly classified in more than the 99\% of the cases. 
However, the other three actions, \emph{open contacts}, \emph{open messages} and \emph{open tweets} are correctly classified in more than 95\% of the cases.


\begin{table}[t]
\begin{center} {\begin{tabular}{|l|r|r|r|}
  \hline
\textbf{Actions} & \textbf{Precision}& \textbf{Recall} & \textbf{F-measure}\\
  \hline
\emph{refresh home} & 0.94 & 0.99 & 0.96 \\ 
\emph{open contacts} & 0.97 & 0.96 & 0.97 \\ 
\emph{tweet/message} & 0.97 & 1.00 & 0.98 \\ 
\emph{open messages} & 1.00 & 0.95 & 0.97 \\ 
\emph{open twitter} & 1.00 & 1.00 & 1.00 \\ 
\emph{open tweets} & 1.00 & 0.95 & 0.97 \\
\emph{other} & 0.96 & 0.96 & 0.96 \\ 
\hline
Average & 0.98 & 0.97 & 0.97 \\  
  \hline
\end{tabular} }
\end{center}
\caption{
Classification results of Twitter actions reached 
by using the {Configuration~1}.}
\label{tab:twitterreport}
\end{table}





 





\section{Possible Countermeasures}
\label{Countermeasures}

Users and service providers might believe that their two parties communications 
are secure if they use the right encryption and authentication mechanisms. 
Unfortunately, current secure communication mechanisms limit their 
traffic encryption actions to the syntax of the transmitted data. The semantic 
of the communication is not protected in any 
way~\cite{Krishnamurthy:2013:POS:2498345.2498600}. For this reason, it has been 
possible for example to develop classifiers for TLS/SSL encrypted traffic 
that are able to discriminate between applications. 


The contribution of this paper was to investigate to which extent it is feasible to identify the specific
actions that a user is doing on his mobile device, by simply eavesdropping the device's network traffic.
While it is out of the scope of the paper to investigate possible countermeasure to the proposed attack, we discuss in the following some related issues.

One common belief is that simple padding techniques may be effective against traffic analysis approaches. 
However, it has to be considered that padding countermeasures are already standardized in TLS, 
explicitly to ``frustrate attacks on a protocol that are based on
analysis of the lengths of exchanged messages''~\cite{rfc5246}. 
Nevertheless, our attack worked against TLS encrypted traffic.
More advanced techniques have been proposed in the literature, such as traffic morphing and direct target sampling \cite{conf/ndss/WrightCM09,Wright:2008:SMY:1397759.1398055}.
However, a recent result showed that none of the existing countermeasures are effective~\cite{Dyer:2012:PIS:2310656.2310689}.
The intuition is that coarse information is unlikely to be hidden efficiently, and the analysis of 
these features may still allow an accurate analysis. 
On the light of these results, we believe it is not trivial to propose effective countermeasures to the attack we shown in this paper.
Indeed, it is intention of the authors to highlight a problem that is becoming even more
alarming after the revelation about the mass surveillance programs that are nowadays
adopted by governments and nation states. 

\section{Conclusions}
\label{Conclusions}

We proposed a framework to analyze encrypted network traffic and to infer which particular actions the 
user executed on some apps installed on his mobile-phone. 
\hilight{We demonstrated that despite the use of SSL/TLS, our traffic analysis 
approach is an effective tool that an eavesdropper 
can leverage to undermine the privacy of mobile users. 
With this tool an adversary may easily learn habits of the target users. The adversary may aggregate
data of thousand users in order to gain some commercial or intelligence advantage against some competitor. In addition, a powerful attacker such as a Government, could use these insights in order to de-anonimize user actions that may be of particular interest.}
We hope that this work will \hilight{shed light on} the possible attacks that may undermine the 
user privacy, and that it will stimulate researchers to work on efficient countermeasures that can be
adopted also on mobile devices. These countermeasures may require a kind of trade-off between 
power efficiency and the required privacy level.






\section{Acknowledgments}
  \label{Acknowledgement}
\hilight{Mauro Conti is supported by a Marie Curie Fellowship funded by the
European Commission under the agreement n. PCIG11-GA-2012-321980.
This work has been partially supported by the TENACE PRIN Project
20103P34XC funded by the Italian MIUR, and by the Project ``Tackling
Mobile Malware with Innovative Machine Learning Techniques'' funded by
the University of Padua.}
  We would like to thank Fabio Aiolli, Michele Donini, and Mauro Scanagatta for their insightful comments.


\bibliographystyle{abbrv}
\balance
\bibliography{bibliography_small}
\end{document}
