\documentclass{article}

\usepackage[preprint, nonatbib]{neurips_2020}

\usepackage[utf8]{inputenc} \usepackage{hyperref}       \usepackage{url}            \usepackage{booktabs}       \usepackage{amsfonts}       \usepackage{amsmath}
\usepackage{graphicx,wrapfig,lipsum}
\usepackage{booktabs}
\usepackage{amsthm}
\usepackage{caption}
\usepackage{subcaption}
\usepackage{mathrsfs}  
\usepackage{amssymb}
\usepackage{wrapfig}
\usepackage{tikz}
\usepackage{bm}
\usepackage[inline]{enumitem}

\usepackage{rotating}
\usepackage{xcolor}
\usepackage{algorithm}
\usepackage{algorithmic}
\usepackage{siunitx}
\usepackage{wrapfig}
\usepackage{multirow}

\tikzset{rect_sharp/.style={rectangle, draw=black, minimum width=30, minimum height=20, outer sep=2pt, inner sep=2.0pt},
	rect/.style={rectangle, rounded corners=2.5pt, draw=black, minimum width=30, minimum height=20, outer sep=2pt, inner sep=2.0pt},
	rect2/.style={rectangle, rounded corners=2.5pt, draw=black, minimum width=40, minimum height=20, outer sep=2pt, inner sep=2.0pt},
	circ/.style={circle, draw=black, minimum size=25, outer sep=2pt, inner sep=0.0pt},
	seq/.style={rectangle, text centered, minimum height=12.5pt, minimum width=100pt, rotate=90, outer sep=0.5pt},
	wideseq/.style={rectangle, text centered, minimum height=20pt, minimum width=100pt, rotate=90, outer sep=0.5pt},
	seq2/.style={rectangle, text centered, minimum height=12.5pt, minimum width=100pt, outer sep=0.5pt},
	wideseq2/.style={rectangle, text centered, minimum height=20pt, minimum width=100pt, outer sep=0.5pt},
	every picture/.style={line width=0.75pt},
	ncbar angle/.initial=90,
	ncbar/.style={
		to path=(\tikztostart)
		-- ()
		-- ((\tikztostart)!#1!\pgfkeysvalueof{/tikz/ncbar angle}:(\tikztotarget))
		-- (\tikztotarget)
	},
	ncbar/.default=0.5cm,
}

\tikzset{square left brace/.style={ncbar=0.15cm}}
\tikzset{square right brace/.style={ncbar=-0.2cm}}

\usetikzlibrary{calc,positioning,decorations.pathreplacing,shapes.misc}

\definecolor{grey}{HTML}{F5F5F5}
\definecolor{darkgrey}{HTML}{666666}
\definecolor{orange}{HTML}{FFF2CC}
\definecolor{brown}{HTML}{D6B656}
\definecolor{lightblue}{HTML}{DAE8FC}
\definecolor{blue}{HTML}{6C8EBF}
\definecolor{lightgreen}{HTML}{D5E8D4}
\definecolor{green}{HTML}{82B366}
\definecolor{purple}{HTML}{E1D5E7}
\definecolor{darkerpurple}{HTML}{9673A6}

\newcommand{\R}{\mathbb{R}}
\newcommand{\cN}{\mathcal{N}}
\newcommand{\cS}{\mathcal{S}}
\newcommand{\bx}{\mathbf{x}}
\newcommand{\bX}{\mathbf{X}}
\newcommand{\bz}{\mathbf{z}}
\newcommand{\m}{\mathbf{m}}
\newcommand{\bh}{\mathbf{h}}
\newcommand{\bZ}{\mathbf{Z}}
\newcommand{\by}{\mathbf{y}}
\newcommand{\NN}{\mathbb{N}}
\newcommand{\EE}{\mathbb{E}}

\newcommand{\KL}[2]{D_{KL}\left[#1 \;\|\; #2\right]}

\newcommand{\cX}{\mathcal{X}}
\newcommand{\cY}{\mathcal{Y}}
\newcommand{\cD}{\mathcal{D}}
\newcommand{\vs}{V}

\newcommand\given[1][]{\:#1\vert\:}


\newcommand{\T}[1]{\mathrm{T}({#1})}
\newcommand{\Tr}[2]{\mathrm{T}^{{#1}}({#2})}
\newcommand{\TAf}[1]{\mathrm{T}({#1})} 
\newcommand{\ba}{\mathbf{a}}\newcommand{\bb}{\mathbf{b}}\newcommand{\bPhi}{\bm{\Phi}}
\newcommand{\bt}{\mathbf{t}}
\newcommand{\bs}{\mathbf{s}}
\newcommand{\Seq}[1]{\operatorname{Seq}(#1)}
\newcommand{\bphi}{\bm\Phi}


\theoremstyle{plain}
\newtheorem{thm}{Theorem}[section]
\newtheorem{proposition}[thm]{Proposition}
\newtheorem{lemma}[thm]{Lemma}
\newtheorem{corollary}[thm]{Corollary}

\theoremstyle{definition}
\newtheorem{definition}[thm]{Definition}
\newtheorem{remark}[thm]{Remark}
\newtheorem{example}[thm]{Example}

\newcommand{\corners}{6pt}

\newcommand{\Sig}{\mathrm{Sig}}
\newcommand{\bSig}{\mathbf{Sig}}

\newcommand{\dt}[1]{\frac{\mathrm{d}{#1}}{\mathrm{dt}}}
\newcommand{\ds}[1]{\frac{\mathrm{d}{#1}}{\mathrm{ds}}}
\newcommand{\du}[1]{\frac{\mathrm{d}{#1}}{\mathrm{du}}}

\title{Seq2Tens: An Efficient Representation of Sequences by Low-Rank Tensor Projections}


\author{Csaba Toth \And
	 	Patric Bonnier \And 
	 	Harald Oberhauser \And \
"aabc", \quad 2\text{-mers} = \{  aa, ab, ac, ab, ac, bc\}.

"aabc" \mapsto &(1+X)(1+X)(1+Y)(1+Z) \\ 
= 1&+2X+Y+Z+XX+2XY+2XZ+YZ\\
&+XXY+XXZ+2XYZ+XXYZ.

\Seq{V}=\{\bx=(\bx_i)_{i=1,\ldots,L}: \bx_i \in V,\, L \ge 1\} 

  \T{V} :=
  \{\bt=(\bt_m)_{m\ge 0}\,\vert\, \bt \in V^{\otimes m}\}

\bs+\bt:=(\bs_m+\bt_m)_{m\geq 0}, \quad c\cdot \bt=(c \bt_m)_{m \ge 0}

\langle \ell, \bt \rangle:= \sum_{m=0}^M \langle \ell_m, \bt_m \rangle. 

\bx \mapsto \langle \ell, (\bx^{\otimes m})_{m \ge 0}   \rangle \text{ where }
\ell: = (\ell_0,\ell_1,\ell_2):= \Big(3,
\begin{pmatrix}
1\\
0 \\
0
\end{pmatrix},
\begin{pmatrix}
0&0&2\\
0&5&0\\
0&0&0
\end{pmatrix}
\Big). 
 \label{eq:tensorprod}
\bs \cdot \bt := \big( \sum_{i=0}^m \bs_i\otimes \bt_{m-i} \big)_{m\geq 0} = \big( 1, \bs_1 + \bt_1, \bs_2 + \bs_1\otimes \bt_1 + \bt_2,\ldots \big).
\label{eq:defphi}
   \Phi: \Seq{\vs} \rightarrow \T{\vs},\quad \bx \mapsto \prod_{i=1}^L (1+ \bx_i) 
   \label{eq:phi_m}
 \Phi_m(\bx) = \sum_{1 \le i_1 < \cdots < i_m \le L} \bx_{i_1} \otimes \cdots \otimes \bx_{i_m} \in \vs^{\otimes m}.

	\mathrm{Seq}(V) \to \T{W}, \quad \bx \mapsto \Phi(\varphi(\bx), \ldots, \varphi(\bx_n))
	
	\bt = \sum_{i=0}^r \bt_i^1 \otimes \cdots \otimes \bt_i^m, \quad \bt_i^1, \ldots, \bt_i^m \in V.
	
  \langle \ell, \Phi(\bx)\rangle &=
\sum_{1 \le i_1 < \cdots < i_m \leq L} \prod_{k=1}^m \langle \bz_k, \bx_{i_k} \rangle. \label{eq:sum}
 \label{eq:stream_sigs}
\Seq{\vs} &\rightarrow \Seq{\T{\vs}},\quad(\bx_1,\bx_2,\ldots, \bx_L) \mapsto \Big(\Phi(\bx_1),\Phi(\bx_1, \bx_2), \ldots, \Phi(\bx_1,\ldots,\bx_L) \Big).

\bx \mapsto \big(\langle \ell, \Phi(\bx_1) \rangle, \ldots, \langle \ell, \Phi(\bx_1,\ldots,\bx_L) \rangle\big)
 \label{eq:lrSeq2Tens_causal}
\tilde\Phi_{\theta}: \Seq{\vs} \rightarrow \Seq{\R^{n}},\, \bx \mapsto ( \langle \ell_{1}, \Phi(\bx_1,\ldots,\bx_i) \rangle, \ldots, \langle \ell_{n},\Phi(\bx_1,\ldots,\bx_i) \rangle)_{i=1}^L.

	\langle \ell_1, \Phi^2(\bx) \rangle = \langle \ell_2, \Phi(\bx)\rangle.
	
 \Seq{\vs} \rightarrow \Seq{\R^{n_1}} \rightarrow \Seq{\R^{n_2}} \rightarrow \cdots \rightarrow \Seq{\R^{n_D}}. 

\varphi^{\theta_1,\ldots,\theta_{D}}:\Seq{\vs} \rightarrow \R^{n_D}

    \tilde\Phi_{bi}(\mathbf{x}): \Seq{\vs} \rightarrow \Seq{\R^{n + n^{\prime}}}, \quad \mathbf{x} \mapsto (\tilde\Phi_{\theta_1}(\bx_1, \ldots, \bx_i), \tilde\Phi_{\theta_2}(\bx_i, \ldots, \bx_L))_{i=1}^L.

    p_\theta(\bx_i \given \bz_i) = \cN(\bx_i \given g_\theta(\bz_i), \sigma^2 \mathbf{I}_d),

 \T{V} := \prod_{m\geq 0} V^{\otimes m} = (V^{\otimes 0}, V, V^{\otimes 2}, V^{\otimes 3}, \ldots )
 
 \bs + \bt = (\bs_m+\bt_m)_{m \ge 0} \in \T{V}\text{ and } c \cdot \bt = (c\bt_m)_{m \ge 0} \in \T{V}
 \bt_m:=x^{\otimes m}:= \underbrace{x \otimes \cdots \otimes x}_{m \text{ many tensor products }\otimes} \in (\R^d)^{\otimes m} \text{ and by convention we set }x^{\otimes 0 }:=1 \in  (\R^d)^{\otimes 0}.\label{eq:ncp}
    \bs \cdot \bt := \big( \sum_{i=0}^m \bs_i\otimes \bt_{m-i} \big)_{m\geq 0} = \big( 1, \bs_1 + \bt_1, \bs_2 + \bs_1\otimes \bt_1 + \bt_2,\ldots \big).
	
    f(\bx) \approx \langle \ell, \Phi(\bx) \rangle.
  
\langle  \ell, \bt \rangle = \sum_{m \ge 0} \langle \ell_m, \bt_m \rangle.
  
\langle \ell, \bt \rangle =  \sum_{i_1,\ldots,i_m \in \{1,\ldots,d\}, m \ge 0} \ell^{i_1,\ldots,i_m}_m \bt_m^{i_1,\ldots,i_m}
  \Phi(\bx)=\varphi(\bx_1)\cdots \varphi(\bx_L) \in \T{\R^d}.
		\{ x \mapsto \langle \ell, f(x) \rangle \,:\, \ell \in W' \} \subseteq C_b(\cX)
		\label{eq: phi}
		\Phi:\mathrm{Seq}(\cX) \to \T{V}, \quad (\bx_1,\ldots,\bx_L) \mapsto \prod_{i=1}^L \varphi(\bx_i)
		
  \langle \ell_1,\Phi(\bx) \rangle \langle \ell_2,\Phi(\bx) \rangle =  \langle \ell,\Phi(\bx) \rangle. 
 \label{eq:phi}
	\Seq{V}\to \T{V}, \quad \Phi(\bx_1, \ldots, \bx_L) := \prod_{i=1}^L\big(1+\bx_i\big).
	
	\Phi(\bx_1, \ldots, \bx_L) = 1 + \sum_{i=1}^L \underbrace{\bx_i}_{V} + \sum_{1 \leq i_1 < i_2 \leq L} \underbrace{\bx_{i_1}\otimes \bx_{i_2}}_{V^{\otimes 2}} + \sum_{1 \leq i_1 < i_2 < i_3 \leq L} \underbrace{\bx_{i_1}\otimes \bx_{i_2}\otimes \bx_{i_3}}_{V^{\otimes 3}} + \cdots 
	 \label{eq:phim}
	\Phi_m(\bx) = \sum_{1 \le i_1 < \cdots < i_m \le L} \bx_{i_1} \otimes \cdots \otimes \bx_{i_m}.
	
	&\langle \ell, \Phi_m(\bx) \rangle 
	= \langle \ell_1 \otimes \cdots \otimes \ell_m, \sum_{1 \le i_1 < \cdots < i_m \le L} \bx_{i_1} \otimes \cdots \otimes \bx_{i_m} \rangle \\
	&= \sum_{1 \le i_1 < \cdots < i_m \le L} \langle \ell_1 \otimes \cdots \otimes \ell_m, \bx_{i_1} \otimes \cdots \otimes \bx_{i_m} \rangle
	= \sum_{1 \le i_1 < \cdots < i_m \le L} \langle \ell_1 , \bx_{i_1} \rangle \cdots \langle \ell_m, \bx_{i_m} \rangle.
	
	\Tr{2}{\vs} = \prod_{n_1, \ldots, n_k \geq 0} \vs^{\otimes n_1} \big\vert \cdots \big\vert \vs^{\otimes n_k}
	
		x^\star := (x^{\otimes m})_{m\geq 0} = (1, x, x^{\otimes 2}, x^{\otimes 3}, \ldots )
		
		\bx^\star := (\bx_1^\star, \ldots, \bx_L^\star) \in \Seq{\T{V}}
		
		\star : \Tr{2}{V} \times \Tr{2}{V} \to \Tr{2}{V}
		
		(\ell_1 \vert e_i)\star(\ell_2 \vert e_j) = (\ell_1 \vert e_i\star \ell_2)\vert e_j + (\ell_1\star \ell_2 \vert e_j)\vert e_i + (\ell_1\star \ell_2)\vert(e_i\otimes e_j).
		
		e_i \star e_j &= e_i\vert e_j + e_j\vert e_i + e_i\otimes e_j \5pt]
		(e_i\vert e_j) \star (e_k\vert e_l) &= e_i\vert e_j\vert e_k\vert e_l + e_i\vert e_k\vert e_j\vert e_l + e_i\vert e_k\vert e_l\vert e_j \\ 
		&+ e_k\vert e_i\vert e_j\vert e_l + e_k\vert e_i\vert e_l\vert e_j + e_k\vert e_l\vert e_i\vert e_j \\ 
		&+ (e_i\otimes e_k)\vert e_j\vert e_l +e_i\vert (e_j\otimes e_k)\vert e_l+(e_i\otimes e_k)\vert e_l\vert e_j \\
		&+e_k\vert (e_i\otimes e_l)\vert e_j +(e_i\otimes e_k)\vert (e_j\otimes e_l)
		
		\langle \ell_1, \Phi(\bx) \rangle \langle \ell_2, \Phi(\bx) \rangle = \langle \ell_1\star \ell_2, \Phi(\bx^\star) \rangle.
		
		\langle e_{i_1} \vert \cdots \vert e_{i_m}, \Phi(\bx) \rangle = \sum_{1 \leq k_1 < \cdots < k_m \leq L} \langle e_{i_1}, \bx_{k_1} \rangle \cdots \langle e_{i_m}, \bx_{k_m} \rangle.
		
		\langle \ell \vert e_i, \Phi(\bx)_L \rangle = \sum_{k=1}^{L-1} \langle \ell,\Phi(\bx)_k\rangle \langle e_i, \bx_{k+1} \rangle
		
		&\langle \ell_1\vert e_i, \Phi(\bx)_L \rangle \langle \ell_2\vert e_j, \Phi(\bx)_L \rangle = \big(\sum_{k=1}^{L-1} \langle \ell_1, \Phi(\bx)_k \rangle \langle e_i,\bx_{k+1}\rangle \big)\big(\sum_{k=1}^{L-1} \langle \ell_2, \Phi(\bx)_k \rangle \langle e_j, \bx_{k+1}\rangle \big) \\
		&= \sum_{1 \leq k_1 < k_2 \leq L-1} \langle \ell_1, \Phi(\bx)_{k_1} \rangle \langle e_i, \bx_{k_1+1} \rangle \langle \ell_2, \Phi(\bx)_{k_2} \rangle \langle e_j, \bx_{k_2+1}\rangle \\
		&+ \sum_{1 \leq k_1 < k_2 \leq L-1} \langle \ell_2, \Phi(\bx)_{k_1} \rangle \langle e_j, \bx_{k_1+1} \rangle \langle \ell_1, \Phi(\bx)_{k_2} \rangle \langle e_i, \bx_{k_2+1}\rangle \\
		&+ \sum_{k=1}^{L-1} \langle \ell_1, \Phi(\bx)_{k} \rangle \langle e_i, \bx_{k+1} \rangle \langle \ell_2, \Phi(\bx)_{k} \rangle \langle e_j, \bx_{k+1}\rangle.
		
		&\sum_{k=1}^{L-1} \langle \ell_1\vert e_i, \Phi(\bx)_{k} \rangle \langle \ell_2, \Phi(\bx)_{k} \rangle \langle e_j, \bx_{k+1} \rangle 
		+ \sum_{k=1}^{L-1} \langle \ell_2\vert e_j, \Phi(\bx)_{k} \rangle \langle \ell_1, \Phi(\bx)_{k} \rangle \langle e_i, \bx_{k+1}\rangle \\
		+ &\sum_{k=1}^{L-1} \langle \ell_2, \Phi(\bx)_{k} \rangle \langle \ell_1, \Phi(\bx)_{k} \rangle \langle e_i\otimes e_j, \bx_{k+1} \rangle.
		
		&\sum_{k=1}^{L-1} \langle \ell_1\vert e_i\star \ell_2, \Phi(\bx)_{k} \rangle \langle e_j, \bx_{k+1} \rangle 
		+ \sum_{k=1}^{L-1} \langle \ell_2\vert e_j\star \ell_1, \Phi(\bx)_{k} \rangle \langle e_i, \bx_{k+1} \rangle \\
		+ &\sum_{k=1}^{L-1} \langle \ell_2\star \ell_1, \Phi(\bx)_{k} \rangle \langle e_i\otimes e_j, \bx_{k+1} \rangle
		
		\langle &(\ell_1\vert e_i\star \ell_2)\vert e_j, \Phi(\bx)_L \rangle+\langle (\ell_2\vert e_j\star \ell_1)\vert e_i, \Phi(\bx)_L \rangle + \langle (\ell_2\star \ell_1)\vert(e_i\otimes e_j), \Phi(\bx)_L \rangle \\ 
		= &\langle \ell_1\vert e_i\star \ell_2\vert e_j, \Phi(\bx)_L \rangle
		
		\langle e_i, \Phi(\bx) \rangle \langle e_j, \Phi(\bx) \rangle = \big( \sum_{k=1}^L \langle e_i, \bx_k \rangle \big)\big( \sum_{k=1}^L \langle e_j, \bx_k \rangle \big) = \sum_{k_1, k_2=1}^L \langle e_i, \bx_{k_1} \rangle\langle e_j, \bx_{k_2} \rangle
		
		&\langle e_i, \Phi(\bx) \rangle \langle e_j, \Phi(\bx) \rangle = \langle e_i\vert e_j+e_j\vert e_i+e_i\otimes e_j, \Phi(\bx) \rangle \\
		&= \sum_{1 \leq k_1 < k_2 \leq L} \langle e_i, \bx_{k_1}\rangle\langle e_j, \bx_{k_2}\rangle + \sum_{1 \leq k_1 < k_2 \leq L} \langle e_j, \bx_{k_1}\rangle\langle e_i, \bx_{k_2}\rangle + \sum_{k=1}^L \langle e_i, \bx_{k}\rangle\langle e_j, \bx_{k}\rangle \\
		&= \sum_{k_1, k_2=1}^L \langle e_i, \bx_{k_1} \rangle\langle e_j, \bx_{k_2} \rangle
		
		\mathrm{Seq}^1(V) \to \T{V}, \quad (\bx_1, \ldots, \bx_L) \to \prod_{i=1}^L (1 + \bx_i)
		
		\langle e_i, \Phi(\bx)\rangle = x_i.
		
		&\langle \ell_1\otimes e_0 + \gamma e_0\otimes \ell_2, \Phi(\bx)-\Phi(\by)\rangle \\
		&= \langle \ell_1, \Phi(\bx_1, \ldots, \bx_L)-\Phi(\by_1, \ldots, \by_L)\rangle + \gamma\langle \ell_2, \Phi(\bx_2, \ldots, \bx_{L+1})-\Phi(\by_2, \ldots, \by_{L+1})\rangle.
		
		\langle \ell_1, \Phi(\bx) \rangle \langle \ell_2, \Phi(\bx) \rangle = \langle \ell_1\star \ell_2, \prod_{i=1}^L (1+\varphi_1(\bx_i)^\star) \rangle
		
		\langle e_i, \varphi_1(\bx_k)\rangle \langle e_j,\varphi_1(\bx_k)\rangle = \langle h, \varphi_1(\bx_k) \rangle + \varepsilon(\bx_k)
		
		&\langle e_i\otimes e_j, \prod_{i=1}^L (1+\varphi_1(\bx_i)^\star) \rangle = \sum_{k=1}^L \langle e_i, \varphi_1(\bx_k)\rangle \langle e_j,\varphi_1(\bx_k)\rangle = \sum_{k=1}^L \langle h, \varphi_1(\bx_k)\rangle + \varepsilon(\bx_k) \\
		&= \langle e_{h}, \Phi(\bx) \rangle + n\max_{1\leq k\leq n}\varepsilon(\bx_k).
		 
	\Seq{\vs} &\rightarrow \Seq{\T{\vs}}\\
	(\bx_1,\bx_2,\bx_3,\ldots, \bx_L) &\mapsto \Big(\Phi(\bx_1),\Phi(\bx_1, \bx_2), \Phi(\bx_1, \bx_2, \bx_3), \ldots, \Phi(\bx_1,\ldots,\bx_L) \Big).
	\label{eq:seq2seq}
	\Seq{\vs}=\Seq{\T{V}} \rightarrow \Seq{\T{\T{V}}} \rightarrow \cdots \rightarrow \Seq{\underbrace{T(\cdots T}_{D \text{ times }}(V)\cdots )}.
	
		\Tr{0}{V} &= \vs, \quad 
		\Tr{D}{V} = \prod_{m\geq 0} \big( \Tr{D-1}{V}\big)^{\otimes m}. \\
		
	\Phi^D : \Seq{V} \to \Tr{D}{V}.
	
		\ell_1\prec (\ell_2\otimes_{(2)} e_i) = (\ell_1\star \ell_2)\otimes_{(2)} e_i 
		
		\Seq{V} \to \Seq{\T{V}}, \quad (\bx_1, \ldots, \bx_L) \mapsto (\Phi(\bx_1), \ldots, \Phi(\bx, \ldots, \bx_L))
		
		\langle e_{\ell_1}\otimes_{(3)} e_{\ell_2}, \Phi(\Delta\bphi(\bx)) \rangle = \langle \ell_1 \prec \ell_2, \Phi(\bx)\rangle
		
		&\langle e_{\ell_1}\otimes_{(3)} e_{\ell_2\otimes_{(2)} e_i}, \Phi(\Delta\bphi(\bx)) \rangle \\
		&= \sum_{1 \leq k_1 < k_2 \leq L} \big(\langle \ell_1, \Phi(\bx)_{k_1}-\langle \ell_1, \Phi(\bx)_{k_1-1}\big) \rangle \big(\langle \ell_2\otimes_{(2)} e_i, \Phi(\bx)_{k_2}\rangle-\langle \ell_2\otimes_{(2)} e_i, \Phi(\bx)_{k_2-1}\rangle\big) \\
		&= \sum_{k=1}^{L-1} \langle \ell_1, \Phi(\bx)_{k} \rangle \big(\langle \ell_2\otimes_{(2)} e_i, \Phi(x)_{k+1}\rangle-\langle \ell_2\otimes_{(2)} e_i, \Phi(\bx)_{k}\rangle\big) \\
		&= \sum_{k=1}^{L-1} \langle \ell_1, \Phi(\bx)_{k} \rangle \big( \sum_{1\leq l\leq k} \langle \ell_2, \Phi(\bx)_{l}\rangle \bx^i_{l+1}-\langle \ell_2, \Phi(\bx)_{l-1}\rangle \bx^i_{l}\big) \\
		&= \sum_{k=1}^{L-1} \langle \ell_1, \Phi(\bx)_{k} \rangle \langle \ell_2, \Phi(\bx)_{k}\rangle \bx^i_{k+1} 
		= \sum_{k=1}^{L-1} \langle \ell_1\star \ell_2, \Phi(\bx)_{k} \rangle \bx^i_{k+1} \\
		&= \langle (\ell_1\star \ell_2)\otimes_{(2)} e_i, \Phi(\bx)_{L} \rangle.
		
    \Phi(\bx) = (\Phi_m(\bx))_{m \geq 0}, \quad \Phi_m(\bx) = \sum_{1 \leq i_1 < \dots < i_m \leq L} \bx_{i_1} \otimes \cdots \otimes \bx_{i_m},

\langle \ell^k, \Phi(\bx) \rangle = \sum_{m=0} \langle \ell^k_m, \Phi_m(\bx) \rangle,

    \tilde\Phi_{m, \theta}(\bx) = (\langle \ell^1_m, \Phi_m(\bx), \dots, \langle \ell^n_m, \Phi_m(\bx) \rangle) \quad\text{and}\quad \tilde\Phi_\theta(\bx) = (\tilde\Phi_{m, \theta}(\bx))_{m \geq 0}, 

    \Phi_m(\by) &= \sum_{1 \leq i_1 < \cdots < i_M \leq L} \by_{i_1} \otimes \cdots \otimes \by_{i_m} 
    = \sum_{1 \leq i_1 < \cdots < i_M \leq L} (\alpha \bx_{i_1}) \otimes \cdots \otimes (\alpha \bx_{i_m}) \\
    &=  \alpha^m \sum_{1 \leq i_1 < \cdots < i_M \leq L} \bx_{i_1} \otimes \cdots \otimes \bx_{i_m},

    \tilde\Phi_{m, \theta} (\by) &= (\langle \ell^1_m, \Phi_m(\by) \rangle, \dots, \langle \ell^n_m, \Phi_m(\by) \rangle) 
    = (\langle \ell^1_m, \alpha^m \Phi_m(\bx) \rangle, \dots, \langle \ell^n_m, \alpha^m \Phi_m(\bx) \rangle) \\
    &= \alpha^m (\langle \ell^1_m, \Phi_m(\bx) \rangle, \dots, \langle \ell^n_m, \Phi_m(\bx) \rangle).

    \Delta \bx := (\bx_1, \bx_2 - \bx_1, \dots, \bx_L - \bx_{L-1}) \in \Seq{\R^d},

    \Phi_1(\bx) = \sum_{i=1}^L \Delta \bx_i = \bx_L,

    \tilde\Phi_{1,\theta}(\bx) = \sum_{i=1}^L (\langle \ell^1_1, \Delta \bx_i \rangle, \dots, \langle \ell^n_1, \Delta \bx_i \rangle) = (\langle \ell^1_1, \bx_L \rangle, \dots, \langle \ell^n_1, \bx_L \rangle),    
 \label{eq:diff_phi}
    \Phi_m(\bx) &= \sum_{1 \leq i_1 < \dots i_m \leq L} \Delta \bx_{i_1} \otimes \cdots \otimes \Delta\bx_{i_m},
 \label{eq:diff_tildephi}
    \tilde\Phi_{m, \theta} &= \sum_{1 \leq i_1 < \cdots < i_m \leq L} (\langle \bz^1_{m, 1}, \Delta \bx_{i_1} \rangle \cdots \langle \bz^1_{m, m}, \Delta \bx_{i_m} \rangle, \dots, \langle \bz^n_{m, 1}, \Delta \bx_{i_1} \rangle \cdots \langle \bz^n_{m, m}, \Delta \bx_{i_m} \rangle)
 \label{eq:recursive_Seq2Tens}
    \Phi_m(\bx_1, \dots \bx_l) = \Phi_m(\bx_1, \dots \bx_{l-1}) + \Phi_{m-1}(\bx_1, \dots, \bx_{l-1}) \otimes \bx_{l},
 \label{eq:nlst_independent}
    \langle \ell_m, \Phi_m(\bx_1, \dots, \bx_l) \rangle =& \langle \bz_{m, 1} \otimes \cdots \otimes \bz_{m, m}, \Phi_m(\bx_1, \dots, \bx_l) \rangle \\ =& \langle \bz_{m, 1} \otimes \cdots \otimes \bz_{m, m}, \Phi_m(\bx_1, \dots, \bx_{l-1}) \rangle \\
    &+ \langle \bz_{m, 1} \otimes \cdots \otimes \bz_{m, m-1}, \Phi_{m-1}(\bx_1, \dots, \bx_{l-1}) \rangle \langle \mathbf{z}_{m,m}, \bx_{l} \rangle \\
    =& \langle \ell_m, \Phi_m(\bx_1, \dots \bx_{l-1}) \rangle \label{eqline:lm_prev} \\
    &+ \langle \bz_{m, 1} \otimes \cdots \otimes \bz_{m, m-1}, \Phi_{m-1}(\bx_1, \dots, \bx_{l-1}) \rangle \langle \mathbf{z}_{m,m}, \bx_{l} \label{eqline:not_lm_1_prev} \rangle
 \label{eq:nlst_recursive}
    \langle \ell_m, \Phi_m(\bx_1, \dots, \bx_l) \rangle = \langle \ell_m, \Phi_m(\bx_1, \dots, \bx_{l-1} \rangle + \langle \ell_{m-1}, \Phi_{m-1}(\bx_1, \dots, \bx_{l-1})\rangle \langle \bz_m, \bx_{l} \rangle,

    \bh_{1, i} &=     \bh_{1, i-1} + \bZ_1 \bx_i, \label{eq:nlst_recursive_vectorized_lv1} \\
    \bh_{m, i} &= \bh_{m, i-1} + \bh_{m-1, i-1} \odot \bZ_m \bx_i \quad \text{for } m \geq 2,

  		A[\dots, :, \boxplus, :, \dots][\dots, i_{j-1}, i_j,, i_{j+1} \dots] := \sum_{\kappa=1}^{i_j} A[\dots, i_{j-1}, \kappa, i_{j+1}, \dots].
  	
	  	A[\dots, :, \Sigma, :, \dots][\dots, i_{j-1}, i_{j+1}, \dots] := \sum_{\kappa=1}^{n_j} A[\dots, i_{j-1}, \kappa, i_{j+1}, \dots].
  	
	  	A[\dots, :, +m, :, \dots][\dots, i_{j-1}, i_j, i_{j+1}, \dots] := \left\lbrace\begin{array}{ll} A[\dots, i_{j-1}, i_j-m, i_{j+1}, \dots], & \text{ if } i_j > m, \\ 0, & \text{ if } i_j \leq m.  \end{array}\right.
  	(A \odot B) [i_1, \dots, i_k] := A[i_1, \dots, i_k] \cdot B[i_1, \dots, i_k]. 
    \ell^k_m = \bz_{m, 1}^k \otimes \cdots \otimes \bz_{m, m}^k \in (\R^d)^{\otimes m} \quad\text{for}\quad k = 1, \dots, n_\ell\quad\text{and}\quad m \geq 0.

    \EE[z^k_{m, j, p}] = 0 \quad\text{and}\quad \EE[z^k_{m, j, p}]^2 = \sigma_m^2 \quad\text{for}\quad j=1, \dots, m \quad\text{and}\quad p=1, \dots, d.

    \ell^k_{m, \mathbf{i}} = z^k_{m, 1, i_1} \cdots z^k_{m, m, i_m}    

    \EE[\ell^k_{m, \mathbf{i}}] = 0\quad\text{and}\quad\EE[\ell^k_{m, \mathbf{i}}]^2 = \sigma_m^{2m}

    \sigma_m^2 = \sqrt[m]{\frac{2}{d^m + n_\ell}}.    

    \ell^k_m = \bz^k_1 \otimes \cdots \otimes \bz^k_m \quad\text{for}\quad k=1,\dots,n_\ell \quad\text{and}\quad m \geq 0.

    \ell^k_{m, \mathbf{i}} = z^k_{1, i_1} \cdots z^k_{m, i_m},

    \EE[\ell^k_{m, \mathbf{i}}] = 0 \quad\text{and}\quad \EE[\ell^k_{m, \mathbf{i}}]^2 = \sigma_1^2 \cdots \sigma_m^2,

    \sigma_1^2 = \frac{2}{d + n_\ell} \quad\text{and}\quad \sigma_{m+1}^2 = \frac{d^m + n_\ell}{d^{m+1} + n_\ell} \quad\text{for}\quad {m \geq 1}. 
\label{eq:one plus}
\bx \mapsto    \Phi(\bx) = (\Phi_m(\bx))_{m \geq 0}  \quad \Phi_m(\bx) = \sum_{1 \leq i_1 < \cdots < i_m \leq L} \bx_{i_1} \otimes \cdots \otimes \bx_{i_m} \in (\R^d)^{\otimes m}.
\Phi_m(\bx)\langle \ell_m, \Phi_m(\bx) \rangle\ell_m\Phi_m(\bx)md^m1\ell_m=\bz_1\otimes \cdots \otimes \bz_m\langle \ell_m, \Phi_m(\bx) \rangle\Phi_m(\bx)f(\bx)d1\ell^k = (\bz_{m, 1}^k \otimes \cdots \otimes \bz^k_{m, m})_{m = 1}^Mk=1\,\ldots,dm\Phi(\bx) \in \T{\R^d}\langle \ell, \Phi(\bx) \rangle\Phi(\bx)\Phi(\bx)\bx_i \in \R^dA\bx_i \in \R^{d'}d'\ll dO((d')^m)\cX\cY = \{1, \dots, c\}c \in \NN\bX^\star = (\bx^\star_i)_{i=1}^{n_{\bX^\star}}(\bX, \by) = (\bx_i, y_i)_{i=1}^{n_\bX} \subseteq \cX \times \cY\cX = \Seq{\R^d}\bx_i = (\bx_{i,j})_{j=1}^{l_{\bx_i}}l_{\bx_i} \in \NN\bx_in_cdl_\bxn_\bXn_{\bX_\star}n_cdl_\bxn_\bXn_{\bX_\star}d+165[128, 256, 128]3.6 \times 10^{4}1.6 \times 10^37.9 \times 10^{4}2.4 \times 10^32.7 \times 10^53.7 \times 10^35.1 \times 10^52.5 \times 10^3n=4\alpha = 1 \times 10^{-3}\beta = 1 / \sqrt{2}50300\mathbf{1.0}\texttt{rope} = 1 \times 10^{-3}n=10^50.9640.9710.9520.9080.9880.9750.9900.9920.994 (0.001)\mathbf{0.996} (0.001)0.968 (0.003)0.993 (0.001)0.9470.7540.9340.7270.9180.9550.9500.9700.976 (0.003)0.977 (0.002)0.969 (0.004)\mathbf{0.993} (0.001)0.9920.9650.9280.9480.900\mathbf{0.994}0.9900.9370.991 (0.001)0.990 (0.002)0.957 (0.005)\mathbf{0.994} (0.001)0.997\mathbf{1.000}\mathbf{1.000}0.930\mathbf{1.000}\mathbf{1.000}\mathbf{1.000}\mathbf{1.000}\mathbf{1.000} (0.000)\mathbf{1.000} (0.000)\mathbf{1.000} (0.000)\mathbf{1.000} (0.000)\mathbf{1.000}\mathbf{1.000}\mathbf{1.000}\mathbf{1.000}\mathbf{1.000}\mathbf{1.000}\mathbf{1.000}\mathbf{1.000}\mathbf{1.000} (0.000)\mathbf{1.000} (0.000)\mathbf{1.000} (0.000)\mathbf{1.000} (0.000)0.8180.8200.7850.7900.820\mathbf{0.880}0.870\mathbf{0.880}0.852 (0.023)0.858 (0.015)0.842 (0.004)0.860 (0.021)0.9690.9510.9590.9620.9840.800\mathbf{1.000}0.9760.986 (0.003)0.990 (0.004)0.979 (0.001)0.980 (0.002)0.8200.9000.9760.6000.927\mathbf{1.000}0.900\mathbf{1.000}0.960 (0.049)0.780 (0.271)\mathbf{1.000} (0.000)\mathbf{1.000} (0.000)0.9090.9030.9450.8880.9520.911\mathbf{0.970}0.8940.967 (0.005)0.964 (0.007)0.773 (0.017)0.957 (0.007)\mathbf{0.977}0.9680.9760.9140.9500.9610.958 (0.005)0.863 (0.043)0.793 (0.135)0.960 (0.004)0.8960.8440.8320.750\mathbf{1.000}0.770 (0.021)0.793 (0.024)0.747 (0.015)0.857 (0.013)0.9170.9080.9230.9270.9520.932\mathbf{0.970}0.9120.968 (0.001)0.964 (0.003)0.954 (0.003)0.953 (0.004)\mathbf{1.000}\mathbf{1.000}\mathbf{1.000}\mathbf{1.000}\mathbf{1.000}\mathbf{1.000}\mathbf{1.000}\mathbf{1.000}0.617 (0.100)0.917 (0.091)\mathbf{1.000} (0.000)\mathbf{1.000} (0.000)0.941\mathbf{0.980}0.9520.9160.9040.9290.9700.9160.979 (0.001)0.979 (0.001)0.938 (0.003)0.969 (0.002)0.9650.9620.9310.9740.968\mathbf{0.992}0.9900.9970.989 (0.002)0.986 (0.004)0.981 (0.002)0.989 (0.001)\mathbf{1.000}\mathbf{1.000}\mathbf{1.000}\mathbf{1.000}\mathbf{1.000}\mathbf{1.000}\mathbf{1.000}\mathbf{1.000}\mathbf{1.000} (0.000)\mathbf{1.000} (0.000)\mathbf{1.000} (0.000)\mathbf{1.000} (0.000)0.9450.9330.9490.8990.9380.955\mathbf{0.970}0.9620.9380.9410.9310.9690.9640.9640.9520.9290.9520.9840.9900.9760.9780.9780.969\mathbf{0.991}0.0590.0730.0550.1110.0730.0580.0390.0430.1060.0750.0890.0467.2817.3758.2148.7507.4335.5004.7005.9674.9065.2816.969\mathbf{4.281}8.0007.5007.7509.0006.5006.2505.5005.5004.7505.2506.750\mathbf{5.000}3.0823.1222.6223.0222.4413.1362.4193.2592.6033.1042.3561.591\cX = \Seq{\R^d}\bX = (\bx_i)_{i=1}^{n_\bX} \subset \Seq{\R^d}\bx_i = (\bx_{i, t_j})_{j=1}^{L_i}\bx_{i, t_j}t_j\bx_i \in \bX\m_i = (\m_{i, t_j})_{j=1}^{L_i} \in \Seq{\{0, 1\}^d}\bx_it_j\bx = (\bx_i)_{i=1,\dots,L} \in \Seq{\R^d}L \in \NN\bx\bz = (\bz_{i})_{i=1,\dots,L} \in \Seq{\R^{d^\prime}}g_\theta: \R^{d^\prime} \rightarrow \R^d\theta\sigma^2 \in \R\bz\bz_i = (z_i^j)_{j=1,\dots,d^\prime} \in \R^{d^\prime}z^j \sim \mathcal{GP}(m(\cdot), k(\cdot, \cdot))m: \R \rightarrow \Rk: \R \times \R \rightarrow \R\mathbf{m}_j \in \R^L\mathbf{A}_j \in \R^{L \times L}j=1, \dots, d^\prime\mathbf{A}_j \in \R^{L \times L}\bx = (\bx_1, \dots, \bx_L) \in \Seq{\R^d}\mathbf{A}_j^{-1} = \mathbf{B}_j \mathbf{B}_j^\top\mathbf{B}_j \in \R^{L \times L}\bX = (\bx_i)_{i=1}^{n_\bX} \subset \Seq{\R^d}\theta\psi\beta\betan_cmdL_\bxn_\bXn_{\bX_\star}100.4528 \times 281060000100000.664 \times 64 \times 389000266420.8235483997n_c \in \NNm \in (0, 1)d \in \NNL_\bX \in \NNn_\bX, n_{\bX_\star} \in \NNd^\prime \in \NN\bz \in \Seq{\R^{d^\prime}}\bx \in \Seq{\R^d}d^\primeM=40MSE = 2 \times 10^{-3}d = 12288h = 32n = 64h^\prime = 256h = 256n = 16MSE = 1.4 \times 10^{-3} \pm  4.1 \times 10^{-5}MSE = 1.3 \times 10^{-3} \pm 4.9 \times 10^{-5}$\end{enumerate*}. Therefore, the smaller convolutional layer was indeed causing an information bottleneck, and by increasing its width to be on par with the other layers in the encoder, we managed to improve the performance of both models. The improvement on the GP-VAE (B-NLST) was larger, which can be explained by the observation that lifting the information bottleneck additionally allowed the benefits of the B-NLST layer to kick in, as detailed previously.

To sum up, we have empirically validated the hypothesis that capturing global information in the encoder was indeed beneficial, and managed to improve on the results even on unseen examples in all cases. The experiment as a whole supports that our introduced NLST layers can serve as useful building blocks in a wide range of models, not only discriminative, but also generative ones.

\small

\end{document}
