\documentclass[10pt,twocolumn,letterpaper]{article}

\usepackage{cvpr}
\usepackage{times}
\usepackage{epsfig}
\usepackage{graphicx}
\usepackage{amsmath}
\usepackage{amssymb}

\usepackage{booktabs}
\usepackage{csvsimple}
\usepackage{siunitx}
\usepackage{float}
\usepackage[caption=false, font=footnotesize]{subfig}
\newcommand{\SG}[1]{\textcolor{magenta}{[SG:#1]}}
\newcommand{\B}[1]{\textcolor{red}{[BG:#1]}}



\usepackage[pagebackref=true,breaklinks=true,letterpaper=true,colorlinks,bookmarks=false]{hyperref}

\cvprfinalcopy 

\def\cvprPaperID{4} \def\httilde{\mbox{\tt\raisebox{-.5ex}{\symbol{126}}}}

\ifcvprfinal\pagestyle{empty}\fi
\begin{document}

\title{SoccerNet: A Scalable Dataset for Action Spotting in Soccer Videos}

\author{Silvio Giancola, Mohieddine Amine, Tarek Dghaily, Bernard Ghanem\\
King Abdullah University of Science and Technology (KAUST), Saudi Arabia\\
{\tt\small silvio.giancola@kaust.edu.sa, maa249@mail.aub.edu, tad05@mail.aub.edu, bernard.ghanem@kaust.edu.sa}
}

\maketitle



\begin{abstract}

In this paper, we introduce \emph{SoccerNet}, a benchmark for action spotting in soccer videos. 
The dataset is composed of 500 complete soccer games from six main European leagues, covering three seasons from 2014 to 2017 and a total duration of 764 hours.
A total of 6,637 temporal annotations are automatically parsed from online match reports at a one minute resolution for three main classes of events (Goal, Yellow/Red Card, and Substitution). 
As such, the dataset is easily scalable.
These annotations are manually refined to a one second resolution by anchoring them at a single timestamp following well-defined soccer rules.
With an average of one event every 6.9 minutes, this dataset focuses on the problem of localizing very sparse events within long videos.
We define the task of \emph{spotting} as finding the anchors of soccer events in a video.
Making use of recent developments in the realm of generic action recognition and detection in video, we provide strong baselines for detecting soccer events.
We show that our best model for classifying temporal segments of length one minute reaches a mean Average Precision (mAP) of 67.8\%.
For the spotting task, our baseline reaches an Average-mAP of 49.7\% for tolerances~ ranging from 5 to 60 seconds.
Our dataset and models are available at \href{https://silviogiancola.github.io/SoccerNet}{https://silviogiancola.github.io/SoccerNet}.

\end{abstract}

 
\vspace{-8pt}
\section{Introduction}
\vspace{-2pt}


Sports is a lucrative sector, with large amounts of money being invested on players and teams. 
The global sports market is estimated to generate an annual revenue of \28.7 billion~\cite{EuropeanFootballMarket}, from which \\delta\delta\deltaNNN=200224\times 224120\times512512\times20k=64kk\pm\pm\delta\deltak=512\delta\delta\delta\delta\delta\delta\delta3\delta4\delta\delta\delta3\delta4\deltak=64\deltak=512$) and the center segment (i) spotting baseline.
The prediction usually activates for a 60 seconds range around the spot. It validates our hypothesis that any sliding window that contains the ground truth spot activates the prediction for the class.



\begin{figure*}[htb]
    \centering
    \subfloat
    {\includegraphics[width=\linewidth]{img/Supplementary/QualitativeResults_Train0}
    \label{fig:QualitativeResults_Train}}\\
    \addtocounter{subfigure}{-1}
    \subfloat[\textbf{Example of games from the training set}]
    {\includegraphics[width=\linewidth]{img/Supplementary/QualitativeResults_Train1}
    \label{fig:QualitativeResults_Train}}\\
    \vspace{10pt}
    \subfloat
    {\includegraphics[width=\linewidth]{img/Supplementary/QualitativeResults_Valid0}
    \label{fig:QualitativeResults_Valid}}\\
    \addtocounter{subfigure}{-1}
    \subfloat[\textbf{Example of games from the validation set}]
    {\includegraphics[width=\linewidth]{img/Supplementary/QualitativeResults_Valid1}
    \label{fig:QualitativeResults_Valid}}\\
    \vspace{10pt}
    \subfloat
    {\includegraphics[width=\linewidth]{img/Supplementary/QualitativeResults_Test0}
    \label{fig:QualitativeResults_Test}}\\
    \addtocounter{subfigure}{-1}
    \subfloat[\textbf{Example of games from the testing set}]
    {\includegraphics[width=\linewidth]{img/Supplementary/QualitativeResults_Test1}
    \label{fig:QualitativeResults_Test}}
    \caption{Qualitative results for the Training (a), Validation (b) and Testing (c) examples. The time scale (X axis) is in minute.}
    \label{fig:supplQualitativeResults}
\end{figure*}




Figure~\ref{fig:supplQualitativeResultsWindow} shows further results with smaller windows sizes in training. As expected, the activation width reduces from 60 seconds to the value of the size of the video chunks used in training.


\begin{figure*}[htb]
    \centering
    \includegraphics[width=\linewidth]{img/Supplementary/QualitativeResults_Wind}
    \caption{Qualitative results for window size ranging from 60 to 5 s.}
    \label{fig:supplQualitativeResultsWindow}
\end{figure*}

 
\end{document}
