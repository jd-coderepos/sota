\documentclass[article]{IEEEtran}

\usepackage{graphicx}
\usepackage{amsmath,amsfonts,amssymb,amsthm}
\usepackage{cite}

\title{DyCaPPON: Dynamic Circuit and Packet Passive Optical
   Network (Extended Version)\thanks{Technical Report, School of
Electrical, Computer, and Energy Eng., Arizona State Univ., April 2014.
This extended technical report accompanies~\cite{WeAMR14}.}}

\author{Xing~Wei, Frank Aurzada, Michael~P.~McGarry, and Martin~Reisslein
\thanks{X.~Wei and M.~Reisslein are with
the School of Electrical, Computer, and Energy Engineering,
Arizona State University, Tempe, AZ 85287-5706,
Email: \{Xing.Wei, reisslein\}@asu.edu,
Phone: (480) 965-8593, Fax: (480) 965-8325.}
\thanks{F.~Aurzada is with the Mathematics Faculty at the
Technical University Darmstadt, Schlossgartenstr.~7,
64289 Darmstadt, Germany, Email: aurzada@mathematik.tu-darmstadt.de,
Phone: + 49 6151 16-3183, Fax: + 49 6151 16-6822}
\thanks{M.\ McGarry is with
the Dept. of Electrical and Computer Eng., University of Texas at El Paso,
El Paso, TX, Email: mpmcgarry@utep.edu,
Phone: (915) 747-6955, Fax: (915) 747-7871.}
 }

\def\E{\mathbb{E}}
\newtheorem{theorem}{Theorem}
\newtheorem{proposition}{Proposition}
\newtheorem{corollary}{Corollary}

\begin{document}

\maketitle

\begin{abstract}
Dynamic circuits are well suited for applications that require
predictable service with a
constant bit rate for a prescribed period of time, such as
cloud computing and e-science applications.
Past research on upstream transmission in
passive optical networks (PONs) has mainly
considered packet-switched traffic and has focused on optimizing
packet-level performance metrics, such as reducing mean delay.
This study proposes and evaluates a dynamic circuit and packet
PON (DyCaPPON) that provides dynamic circuits along with packet-switched
service.
DyCaPPON provides  flexible packet-switched service through
dynamic bandwidth allocation in periodic polling cycles,
and  consistent circuit service by allocating each active
circuit a fixed-duration upstream transmission window during
each fixed-duration polling cycle.
We analyze circuit-level performance metrics, including
the blocking probability of dynamic circuit requests
in DyCaPPON through
a stochastic knapsack-based analysis.
Through this analysis we also determine the bandwidth occupied
by admitted circuits. The remaining bandwidth is available for packet
traffic and we conduct an approximate analysis of
the resulting mean delay of packet traffic.
Through extensive numerical evaluations and verifying simulations
we demonstrate the circuit blocking and packet delay trade-offs in
DyCaPPON.
\end{abstract}

\begin{keywords}
Dynamic circuit switching; Ethernet Passive Optical Network;
Grant scheduling; Grant sizing; Packet delay; Stochastic knapsack.
\end{keywords}

\section{Introduction}
\label{sec:intro}
Optical networks have traditionally employed
three main switching paradigms, namely
circuit switching, burst switching, and packet
switching, which have extensively studied respective benefits and
limitations~\cite{PerformCS,OCSvsOBS01,OVSvsOBSpkt,NEHONET}.
In order to achieve the predictable network service of circuit switching
while enjoying some of the flexibilities of burst and packet
switching, \textit{dynamic circuit switching} has been
introduced~\cite{VeKC10}.
Dynamic circuit switching can be traced back to research
toward differentiated levels of blocking rates of calls~\cite{DCS1989}.
Today, a plethora of network applications ranging from
the migration of data and computing work loads to cloud storage and
computing~\cite{SBCD1009}
as well as high-bit rate e-science applications, e.g., for
remote scientific collaborations, to big data applications of
governments, private organizations, and households
are well supported by dynamic circuit switching~\cite{VeKC10}.
Moreover, gaming applications benefit from predictable
low-delay service~\cite{BrF10,FiGR02,MaH10,ScER02} provided by circuits,
as do emerging virtual reality
applications~\cite{KuB13,PaDR12,VaZK10}.
Also, circuits can aid in the timely transmission of
data from continuous media applications, such as live or
streaming video.
Video traffic is often highly variable and
may require smoothing before transmission over
a circuit~\cite{ghazi2012vmp,oh2008cont,QiKo11,ReLR02,ReT99,ShSZ11,AuRe09}
or require a combination of circuit transport for a constant
base bit stream and packet switched transport for the traffic burst
exceeding the base bit stream rate.
Both commercial and
research/education network providers have recently started
to offer optical dynamic circuit switching services~\cite{DOCS00,DCN}.

While dynamic circuit switching
has received growing research attention in core and metro
networks~\cite{DCN,Charb12,HybridN2010,Li08,MGJT0511,ReqprovDOCS,Sko12,VanBreu05,CircuitSONET},
mechanisms for supporting dynamic circuit switching in
passive optical networks (PONs), which are a
promising technology for network
access~\cite{Mahloo13,McR12,McRAS10,Siva13,ToKK12,Zan2013},
are largely an open research area.
As reviewed in Section~\ref{lit:sec}, PON research
on the upstream transmission direction from the distributed
Optical Network Units (ONUs) to the central Optical Line Terminal (OLT) has
mainly focused on mechanisms supporting packet-switched
transport~\cite{AuSH08,AuSRGM11,ZhMo09}.
While some of these packet-switched transport mechanisms
support quality of service akin to circuits through
service differentiation mechanisms, to the best of our knowledge
there has been no prior study of circuit-level performance
in PONs, e.g., the blocking probability of circuit requests
for a given circuit request rate and circuit holding time.

In this article, we present the first circuit-level performance
study of a PON with polling-based medium access control.
We make three main original contributions towards the concept of efficiently
supporting both \textbf{Dy}namic \textbf{C}ircuit \textbf{a}nd \textbf{P}acket
traffic in the upstream
direction on a \textbf{PON}, which we refer to as \textbf{DyCaPPON}:
\begin{itemize}
\item We propose a novel DyCaPPON polling cycle structure that exploits
the dynamic circuit transmissions to mask the round-trip propagation delay
for dynamic bandwidth allocation to packet traffic.
\item We develop a stochastic knapsack-based model of DyCaPPON
to evaluate the circuit-level performance, including
the blocking probabilities for different classes of
circuit requests.
\item We analyze the bandwidth sharing between circuit and packet traffic
in DyCaPPON
and evaluate packet-level performance, such as mean packet delay,
as a function of the circuit traffic.
\end{itemize}

This article is organized as follows.
We first review related work in Section~\ref{lit:sec}.
In Section~\ref{sec:model}, we describe the considered
access network structure and define both the circuit and packet traffic models
as well as the corresponding circuit- and packet-level performance metrics.
In Section~\ref{dycappon:sec}, we introduce the DyCaPPON polling
cycle structure and outline the steps for admission control of
dynamic circuit requests and dynamic bandwidth allocation to packet traffic.
In Section~\ref{sec:analysis} we analyze the performance metrics
relating to the dynamic circuit traffic, namely the blocking
probabilities for the different circuit classes. We also
analyze the bandwidth portion of a cycle consumed by active circuits,
which in turn determines the bandwidth portion available for packet traffic,
and analyze the resulting mean delay for packet traffic.
In Section~\ref{eval:sec} we
validate numerical results from our analysis with simulations and present
illustrative circuit- and packet-level performance results for DyCaPPON.
We summarize our conclusions in Section~\ref{sec:conclusion}
and outline future research directions towards the DyCaPPON concept.


\section{Related Work}
\label{lit:sec}
The existing research on upstream transmission in
passive optical access networks has mainly focused on
packet traffic and related packet-level performance metrics.
A number of studies has primarily focused on
differentiating the packet-level QoS for different classes
of packet traffic,
e.g.,~\cite{AnLA04,AsYDA03,DiDLC11,GhSA04,LuAn05,RaM09,
ShHEAA04,ShBGAM05,VaGh11}.
In contrast to these studies, we consider only best effort service
for the packet traffic in this article.
In future work, mechanisms for differentiation of packet-level QoS
could be integrated into the packet partition
(see Section~\ref{dycappon:sec}) of the DyCaPPON polling cycle.

The needs of applications for transmission
with predictable quality of service has led to various
enhancements of packet-switched transport for providing quality of
service (QoS).
A few studies, e.g.,~\cite{BeBM09,Ho06,MZC0303,Qin13,ZhAY03,ZhP04},
have specifically focused on providing deterministic QoS, i.e.,
absolute guarantees for packet-level performance metrics,
such as packet delay or jitter.
Several studies have had a focus on the efficient integration of
deterministic QoS mechanisms with one or several
lower-priority packet traffic classes in polling-based
PONs, e.g.,\cite{AHKWK0703,BeIB11,DhAMS07,HwLLL12,LiLC11,Merayo2010,NgGB11}.
The resulting packet scheduling problems have received
particular attention~\cite{De12,PeFA09,YiP10}.
Generally, these prior studies have found that fixed-duration
polling cycles are well suited for supporting consistent
QoS service.
Similar to prior studies, we employ fixed-duration polling
cycles in DyCaPPON, specifically on a PON with a single-wavelength
upstream channel.

The prior studies commonly considered traffic flows characterized
through leaky-bucket parameters that bound the long-term average
bit rate as well as the size of sudden traffic bursts.
Most of these studies include admission control, i.e., admit
a new traffic flow only when the packet-level performance
guarantees can still be met with the new traffic flow
added to the existing flows.
However, the circuit-level performance, i.e., the probability of
blocking (i.e., denial of admission) of a new request has not
been considered.
In contrast, the circuits in
DyCaPPON provide absolute QoS to constant bit rate
traffic flows without bursts and we analyze the probability
of new traffic flows (circuits) being admitted or blocked.
This flow (circuit) level performance is important
for network dimensioning and providing QoS at the level of
traffic flows.

For completeness, we briefly note that a PON
architecture that can provide circuits to ONUs through
orthogonal frequency division multiplexing techniques on the
physical layer has been proposed in \cite{OFDMPON}.
Our study, in contrast, focuses on efficient medium access control
techniques for supporting circuit traffic.
A QoS approach based on burst switching in a PON has been proposed
in~\cite{SeBP05}.
To the best of our knowledge, circuit level performance
in PONs has so far only been examined in~\cite{VaML12} for
the specific context of optical code division
multiplexing~\cite{KwYZ96}.

We also note for completeness that large file transmissions
in optical networks have been examined in~\cite{DSAlgfle},
where scheduling of large
data file transfers on the optical grid network is studied,
in~\cite{LgfleMulp}, where
parallel transfer over multiple network paths are examined, and
in~\cite{EIBT}, where files are transmitted in
a burst mode, i.e., sequentially.

Sharing of a general time-division multiplexing (TDM) link by
circuit and packet traffic has been analyzed in several
studies,
e.g.~\cite{bolla97,gaver82,ghani1994decomp,li1985perf,mag82,mankus92,wein80}.
These queueing theoretic analyses typically employed detailed Markov
models and become computationally quite demanding for high-speed
links. Also, these complex existing models considered a given node
with local control of all link transmissions. In contrast, we
develop a simple performance model for the distributed transmissions
of the ONUs that are coordinated through polling-based medium access
control in DyCaPPON. Our DyCaPPON model is accurate for the circuits
and approximate for the packet service. More specifically, we model
the dynamics of the circuit traffic, which is given priority over
packet traffic up to an aggregate circuit bandwidth of  in
DyCaPPON, with accurate stochastic knapsack modeling techniques in
Section~\ref{percir:sec}. In Section~\ref{pkt_perf:sec}, we present
an approximate delay model for the packet traffic, which in DyCaPPON
can consume the bandwidth left unused by circuit traffic.


\section{System Model}
\label{sec:model}

\begin{table}
\caption{Main model notations}
\label{not:tab}
\begin{tabular}{|l|l|}  \hline
\multicolumn{2}{|c|}{Network architecture}\\
  & Transmission rate [bit/s] of upstream channel  \\
 & Transm. rate limit for circuit service,  \\
  & Number of ONUs \\
  & One-way propagation delay [s] \\
\hline
\multicolumn{2}{|c|}{Traffic model}\\
 & Bit rates [bit/s] for
         circuit classes \\
 & Aggregate circuit requests arrival rate [circuits/s]\\
 & Prob. that a request is for circuit type  \\
 & Mean circuit bit rate [bit/s] of offered
     circuit traf. \\
 & Mean circuit holding time [s/circuit] \\
 & Offered circuit traffic
            intensity (load) \\
,  & Mean [bit] and variance of packet size \\
 & Packet traffic intensity (load);
  is agg.\ packet  \\
   & \ \ \ \ \ generation rate [packets/s] at all  ONUs\\
\hline
\multicolumn{2}{|c|}{Polling protocol}\\
 & Total cycle duration [s], constant \\
 & Cycle duration (rand.\ var.) occupied by circuit traf. \\
 & Mean per-cycle overhead time [s] for upstream \\
   & transmissions (report transm.\ times, guard times) \\
\hline
\multicolumn{2}{|c|}{Stochastic knapsack model for circuits}\\
  & State vector of numbers of circuits
       of class \\
  & Aggregate bandwidth of active
 circuits\\
  & Equilibrium probability for active circuits having \\
        & \ \ \ \ \ aggregate
    bandwidth  \\ \hline
\multicolumn{2}{|c|}{Performance metrics}\\
 & Blocking probability for circuit class  \\
 & Mean packet delay [s]\\
\hline
\end{tabular}
\end{table}
\subsection{Network structure}
We consider a PON with  ONUs attached to the OLT with a single
downstream wavelength channel and a single upstream wavelength
channel~\cite{ZhMo09,MRM1008}.
We denote  for the transmission bit rate (bandwidth) of a channel [bits/s].
We denote  [s] for the one-way propagation delay between
the OLT and the equidistant ONUs.
We denote  [s] for the fixed duration of a polling cycle.
The model notations are summarized in Table~\ref{not:tab}.

\begin{figure*}[t]
\begin{center}
\setlength{\unitlength}{0.825mm}
\begin{picture}(160,32)
\thicklines
\put(-10,0){\line(1,0){190}}
\put(-10,20){\line(1,0){190}}
\put(-15,21){\makebox(0,0)[b]{OLT}}
\put(-15,-1){\makebox(0,0)[t]{ONU}}

\thinlines
\put(-5,20){ \vector(1,-3){6.6}}
\thicklines
\put(-10,0){\vector(1,3){6.6}}
\put(18,0){\vector(1,3){6.6}}

\thinlines
\put(52,20){\vector(1,-3){6.6}}
\thicklines
\put(46,0){\vector(1,3){6.6}}
\put(59,0){\vector(1,3){6.6}}

\thinlines
\put(107,20){\vector(1,-3){6.6}}
\thicklines
\put(101,0){\vector(1,3){6.6}}
\put(112,0){\vector(1,3){6.6}}

\thinlines
\put(162,20){\vector(1,-3){6.6}}
\thicklines
\put(156,0){\vector(1,3){6.6}}
\put(169,0){\vector(1,3){6.6}}


\put(11,0.75){\makebox(0,0)[b]{}}
\put(2.5,9.75){\makebox(0,0)[b]{}}

\put(33,0.75){\makebox(0,0)[b]{}}

\put(52.5,0.75){\makebox(0,0)[b]{}}
\put(57,9.75){\makebox(0,0)[b]{}}

\put(81,0.75){\makebox(0,0)[b]{}}

\put(107.25,0.75){\makebox(0,0)[b]{\tiny{}}}
\put(112.5,9.75){\makebox(0,0)[b]{}}

\put(137,0.75){\makebox(0,0)[b]{}}

\put(167.5,9.75){\makebox(0,0)[b]{}}

\thinlines
\put(-3,20){\line(0,1){10}}
\put(52,20){\line(0,1){10}}
\put(107,20){\line(0,1){10}}
\put(162,20){\line(0,1){10}}

\put(-10,11){\makebox(0,0)[]{ }}
\put(46,11){\makebox(0,0)[]{ }}
\put(102,11){\makebox(0,0)[]{ }}
\put(157,11){\makebox(0,0)[]{ }}

\put(7,25){\vector(-1,0){10}}
\put(42,25){\vector(1,0){10}}
\put(25,25){\makebox(0,0)[]{cycle , dur.~}}

\put(62,25){\vector(-1,0){10}}
\put(97,25){\vector(1,0){10}}
\put(77.5,25){\makebox(0,0)[]{cycle , dur.~}}


\put(117,25){\vector(-1,0){10}}
\put(152,25){\vector(1,0){10}}
\put(135.25,25){\makebox(0,0)[]{cycle , dur.~}}

\end{picture}
\end{center}
\caption{An upstream cycle  has fixed duration 
  and has a circuit partition of duration 
  (that depends on the bandwidth demands of the accepted circuits)
  while a packet partition occupies the remaining cycle duration
   .
The exact duration  of the packet partition in cycle  is
evaluated in Eqn.~(\ref{Gp:eqn}).
 Each ONU sends a report during each packet partition.
Packet traffic reported in cycle  is served in the packet partition
of cycle  (if there is no backlog).
A circuit requested in cycle  starts in the circuit partition of
cycle .
The  round-trip propagation delay between the last ONU report (R)
of a cycle 
and the first packet transmission following the grant (G)
of the next cycle  is masked
by the circuit partition, provided . }
\label{fig:cycle}
\end{figure*}
\subsection{Traffic Models} \label{sec:trafficmod}
For circuit traffic, we consider  classes of circuits
with bandwidths .
We denote  [requests/s] for the aggregate Poisson process
arrival rate of circuit requests.
A given circuit request is for a circuit of class ,
with probability .
We denote the mean circuit bit rate of the offered circuit traffic by
.
We model the circuit holding time (duration) as an exponential random
variable with mean .
We denote the resulting offered circuit traffic
intensity (load) by .

For packet traffic, we denote  and  for the mean
and the variance of the packet size [in bit], respectively.
We denote  for the aggregate Poisson process
arrival rate [packets/s]
of packet traffic across the  ONUs and denote
 for
the packet traffic intensity (load).

Throughout, we define the packet sizes and circuit bit rates to include the
per-packet overheads, such as the preamble for Ethernet frames
and the interpacket gap, as well as the packet overheads when
packetizing circuit traffic for transmission.

\subsection{Performance Metrics}
For circuit traffic, we consider the blocking probability
, i.e., the probability that a request
for a class  circuit is blocked, i.e., cannot be accommodated within
the transmission rate limit for circuit service .
We define the average circuit blocking probability as
.
For packet traffic, we consider the mean packet
delay  defined as
the time period from the instant of packet arrival at the ONU
to the instant of complete delivery of the packet to the OLT.


\section{DyCaPPON Upstream Bandwidth Management}
\label{dycappon:sec}
\subsection{Overview of Cycle and Polling Structure}
In order to provide circuit traffic with consistent upstream transmission
service with a fixed circuit bandwidth, DyCaPPON employs a polling cycle
with a fixed duration  [s].
An active circuit with bandwidth  is allocated an upstream
transmission window of duration  in every cycle.
Thus, by transmitting at the full upstream channel bit rate 
for duration  once per cycle of duration ,
the circuit experiences a transmission bit rate
(averaged over the cycle duration) of .
We let  denote the aggregate of the upstream
transmission windows of all active circuits in the PON in cycle ,
and refer to  as the circuit partition duration.
We refer to the remaining duration  as the
packet partition of cycle .


As illustrated in Fig.~\ref{fig:cycle}, a given cycle  consists
of the circuit partition followed by the packet partition. During
the packet partition of each cycle, each ONU sends a report message
to the OLT. The report message signals new circuit requests as well
as the occupancy level (queue depth) of the packet service queue in
the ONU to the OLT. The signaling information for the circuit
requests, i.e., requested circuit bandwidth and duration, can be
carried in the Report message of the MPCP protocol in EPONs with
similar modifications as used for signaling information for
operation on multiple wavelength channels~\cite{McGR06}.

Specifically, for signaling dynamic circuit requests,
an ONU report in the packet partition of cycle 
carries circuit requests generated since the ONU's preceding
report in cycle . The report reaches the
OLT by the end of cycle  and the OLT executes
circuit admission control as described in Section~\ref{ac:sec}.
The ONU is informed about the outcome of the admission control
(circuit is admitted or blocked)
in the gate message that is transmitted on the downstream wavelength
channel at the beginning of cycle .
In the DyCaPPON design, the gate message propagates downstream while
the upstream circuit transmissions of cycle  are
propagating upstream.
Thus, if the circuit was admitted, the ONU commences the
circuit transmission with the circuit partition of cycle .

For signaling packet traffic, the
ONU report in the packet partition of cycle  carries
the current queue depth as of the report generation instant.
Based on this queue depth, the OLT determines the effective
bandwidth request and bandwidth allocation as described in
Section~\ref{dba:sec}.
The gate message transmitted downstream at the
beginning of cycle  informs the ONU about its
upstream transmission window in the packet partition of
cycle .

As illustrated in Fig.~\ref{fig:cycle}, in the DyCaPPON design,
the circuit partition is positioned at the beginning of the cycle,
in an effort to mitigate the idle time between the
end of the packet transmissions in the preceding cycle
and the beginning of the packet transmissions of the current cycle.
In particular, when the last packet
transmission of cycle  arrives at the OLT at the
end of cycle , the first packet transmission of cycle 
can arrive at the OLT at the very earliest one roundtrip
propagation delay
(plus typically negligible processing time and gate transmission time)
after the beginning of cycle .
If the circuit partition duration  is longer than the
roundtrip propagation delay , then idle time between packet
partitions is avoided.
On the other hand, if , then there an idle
channel period of duration  between the end of
the circuit partition and the beginning of the packet
partition in cycle .
\begin{figure*}[t]
\begin{center}
\centering
\setlength{\unitlength}{0.825mm}
\begin{picture}(160,32)
\thicklines
\put(0,20){\line(1,0){160}}
\put(0,30){\line(1,0){160}}
\thinlines
\put(0,5){\line(0,1){25}}
\put(160,5){\line(0,1){25}}
\put(80,10){\line(0,1){20}}

\put(8,20){\line(0,1){10}}
\put(12,20){\line(0,1){10}}
\put(30,20){\line(0,1){10}}
\put(34,20){\line(0,1){10}}
\put(76,20){\line(0,1){10}}
\put(80,20){\line(0,1){10}}

\put(92,20){\line(0,1){10}}
\put(96,20){\line(0,1){10}}
\put(110,20){\line(0,1){10}}
\put(114,20){\line(0,1){10}}
\put(142,20){\line(0,1){10}}
\put(156,20){\line(0,1){10}}

\put(4,25){\makebox(0,0)[]{ }}
\put(9.5,25){\makebox(0,0)[]{ }}
\put(21,25){\makebox(0,0)[]{ }}
\put(31.5,25){\makebox(0,0)[]{ }}
\put(50,25){\makebox(0,0)[]{ }}
\put(77.5,25){\makebox(0,0)[]{ }}

\put(85,25){\makebox(0,0)[]{ }}
\put(93.5,25){\makebox(0,0)[]{ }}
\put(102,25){\makebox(0,0)[]{ }}
\put(111.5,25){\makebox(0,0)[]{ }}
\put(126,25){\makebox(0,0)[]{ }}
\put(148,25){\makebox(0,0)[]{ }}
\put(157.5,25){\makebox(0,0)[]{ }}

\put(20,15){\vector(-1,0){20}}
\put(60,15){\vector(1,0){20}}
\put(40,16){\makebox(0,0)[]{}}
\put(40,11){\makebox(0,0)[]{Circuit partition}}

\put(100,15){\vector(-1,0){20}}
\put(140,15){\vector(1,0){20}}
\put(120,15){\makebox(0,0)[]{Packet partition}}

\put(60,5){\vector(-1,0){60}}
\put(100,5){\vector(1,0){60}}
\put(80,5){\makebox(0,0)[]{Cycle duration  }}
\end{picture}

\end{center}
\caption{Detailed example illustration of an upstream transmission cycle :
ONUs 1, 5, and 12 have active circuits with bandwidths resulting in
circuit grant durations , , and . Each of the 
ONUs is allocated a packet grant of duration  according
to the dynamic packet bandwidth allocation based on the
reported packet traffic; the packet grant accommodates
at least the ONU report (even if there is not payload packet traffic).}
\label{fig:cycle_det}
\end{figure*}

Note that this DyCaPPON design trades off lower responsiveness to
circuit requests for the masking of the roundtrip propagation delay.
Specifically, when an ONU signals a dynamic circuit request in the report
message in cycle , it can at the earliest transmit
circuit traffic in cycle .
On the other hand, packet traffic signaled in the report message
in cycle  can be transmitted in the next cycle, i.e., cycle .

Fig.~\ref{fig:cycle_det} illustrates the
structure of a given cycle in more detail, including
the overheads for the upstream transmissions.
Each ONU that has an active circuit in the cycle
requires one guard time of duration  in the
circuit partition.
Thus, with  denoting the number of ONUs with active circuits
in the cycle, the duration of the circuit partition is
.
In the packet partition, each of the  ONUs transmits
at least a report message plus possibly some data upstream,
resulting in an overhead of .
Thus, the overhead per cycle is

The resulting aggregate limit of the transmission windows for packets
in cycle  is


\subsubsection{Low-Packet-Traffic Mode Polling}
\label{lowload:sec}
If there is little packet traffic, the circuit partition 
and the immediately following packet transmission phase denoted
P1 in Fig.~\ref{fig:cyclell} may leave significant portions of
the fixed-duration cycle idle.
In such low-packet-traffic cycles, the OLT can launch
additional polling rounds denoted P2, P3, and P4 in Fig.~\ref{fig:cyclell}
to serve newly arrived packets with low delay.
Specifically, if all granted packet upstream transmissions have
arrived at the OLT and there is more than
 time remaining until the end of the cycle
(i.e., the beginning of the arrival of the next circuit
partition ) at the OLT, then the OLT can launch
another polling round.
\begin{figure*}[t]
\begin{center}
\centering
\setlength{\unitlength}{0.825mm}
\begin{picture}(160,32)
\thicklines
\put(0,0){\line(1,0){140}}
\put(0,20){\line(1,0){140}}
\put(0,21){\makebox(0,0)[b]{OLT}}
\put(0,-1){\makebox(0,0)[t]{ONU}}

\thinlines
\put(8,20){ \vector(1,-3){6.6}}
\thicklines
\put(3,0){\vector(1,3){6.6}}
\put(17,0){\vector(1,3){6.6}}

\thinlines
\put(38,20){\vector(1,-3){6.6}}
\thicklines
\put(31,0){\vector(1,3){6.6}}
\put(45,0){\vector(1,3){6.6}}

\thinlines
\put(67,20){\vector(1,-3){6.6}}
\thicklines
\put(60,0){\vector(1,3){6.6}}
\put(74,0){\vector(1,3){6.6}}

\thinlines
\put(97,20){\vector(1,-3){6.6}}
\thicklines
\put(90,0){\vector(1,3){6.6}}
\put(104,0){\vector(1,3){6.6}}
\put(119,0){\vector(1,3){6.6}}
\put(133,0){\vector(1,3){6.6}}

\put(10,0.75){\makebox(0,0)[b]{}}
\put(143,0.75){\makebox(0,0)[b]{}}

\put(26,0.75){\makebox(0,0)[b]{}}
\put(53,0.75){\makebox(0,0)[b]{}}
\put(82,0.75){\makebox(0,0)[b]{}}
\put(112,0.75){\makebox(0,0)[b]{}}
\put(126,0.75){\makebox(0,0)[b]{Idle}}

\thinlines
\put(10,20){\line(0,1){10}}
\put(140,20){\line(0,1){10}}

\put(3,11){\makebox(0,0)[]{ }}
\put(31,11){\makebox(0,0)[]{ }}
\put(60,11){\makebox(0,0)[]{ }}
\put(90,11){\makebox(0,0)[]{ }}
\put(119,11){\makebox(0,0)[]{ }}

\put(14,11){\makebox(0,0)[]{ }}
\put(43,11){\makebox(0,0)[]{ }}
\put(72,11){\makebox(0,0)[]{ }}
\put(102,11){\makebox(0,0)[]{ }}

\put(50,25){\vector(-1,0){40}}
\put(100,25){\vector(1,0){40}}
\put(76,25){\makebox(0,0)[]{cycle , fixed duration }}
\end{picture}
\end{center}
\caption{Illustration of low-packet-traffic mode polling:
If transmissions from all ONUs in the packet phase P1
following the circuit partition  reach the OLT more than
 before the end of the cycle, the OLT can launch
additional packet polling rounds P2, P3, and P4 to serve
newly arrived packet traffic before the next circuit partition
.}
\label{fig:cyclell}
\end{figure*}


\subsection{Dynamic Circuit Admission Control}
\label{ac:sec}
For each circuit class , the OLT tracks the number
 of currently active circuits,
i.e., the OLT tracks the state vector 
representing the numbers of active circuits.
Taking the inner product of 
with the vector  representing
the bit rates of the circuit classes gives
the currently required aggregate circuit bandwidth

which corresponds to the circuit partition duration

For a given limit , of bandwidth available for
circuit service, we let
 denote the state space of the
stochastic knapsack model~\cite{Ross95} of the dynamic circuits, i.e.,

where  is the set of non-negative integers.

For an incoming ONU request for a circuit of class , we let
 denote the subset of the state space 
that can accommodate the circuit request, i.e.,
has at least spare bandwidth  before reaching
the circuit bandwidth limit . Formally,

Thus, if presently , then
the new class  circuit can be admitted; otherwise, the
class  circuit request must be rejected (blocked).


\subsection{Packet Traffic Dynamic Bandwidth Allocation}
\label{dba:sec}
With the offline scheduling approach~\cite{ZhMo09} of DyCaPPON,
the reported packet queue occupancy corresponds to the
duration of the upstream packet transmission windows
, requested by ONU .
Based on these requests, and the available aggregate
packet upstream transmission window  (\ref{Gp:eqn}),
the OLT allocates upstream packet transmission windows with durations
, to the individual ONUs.

The problem of fairly allocating bandwidth so as to enforce a
maximum cycle duration has been extensively studied for the Limited
grant sizing approach~\cite{AsYDA03,BSA06},
which we adapt as follows. We set the
packet grant limit for cycle  to

If an ONU requests less than the maximum packet grant
duration , it is granted its full request and the
excess bandwidth (i.e., difference between  and allocated
grant) is collected by an excess bandwidth distribution
mechanism.
If an ONU requests a grant duration longer than , it is allocated
this maximum grant duration, plus a portion of the excess
bandwidth according to the equitable distribution approach
with a controlled excess allocation bound~\cite{AYDA1103,BSA06}.

With the Limited grant sizing approach, there is commonly an unused
slot remainder of the grant allocation to
ONUs~\cite{KrMD2002,HaSM06,NaM06}
due to the next queued packet not fitting into the
remaining granted transmission window.
We model this unused slot remainder by half of the average
packet size  for each of the  ONUs.
Thus, the total mean unused transmission window duration
in a given cycle is


\section{Performance Analysis}
\label{sec:analysis}

\subsection{Circuit Traffic}
\label{percir:sec}
\subsubsection{Request Blocking}
\label{reqblo:sec}
In this section, we employ techniques from the analysis of
stochastic knapsacks~\cite{Ross95} to evaluate the blocking probabilities
 of the circuit class.
We also evaluate the mean duration of the circuit partition ,
which governs the mean available packet partition duration ,
which in turn is a key parameter for the evaluation of the mean
packet delay in Section~\ref{delan:sec}.

The stochastic knapsack model~\cite{Ross95} is a generalization of the
well-known Erlang loss system model to circuits with heterogeneous
bandwidths.
In brief, in the stochastic knapsack model, objects of different
classes (sizes) arrive to a knapsack of fixed capacity (size)
according to a stochastic arrival process.
If a newly arriving object fits into the currently vacant knapsack
space, it is admitted to the knapsack and remains in the knapsack
for some random holding time. After the expiration of the holding time,
the object leaves the knapsack and frees up the knapsack space that
it occupied.
If the size of a newly arriving object exceeds the currently
vacant knapsack space, the object is blocked from entering the knapsack,
and is considered dropped (lost).

We model the prescribed limit  on the bandwidth available for circuit
service as the knapsack capacity.
The requests for circuits of bandwidth ,
arriving according to a Poisson process with rate
 are modeled as the objects seeking entry into the knapsack.
An admitted circuit of class  occupies the bandwidth (knapsack space)
 for an exponentially distributed holding time with mean .

We denote  for the set of states
 that occupy an aggregate bandwidth
, i.e.,

Let  denote the equilibrium probability of the
currently active circuits occupying an aggregate bandwidth of .
Through the recursive Kaufman-Roberts algorithm~\cite[p. 23]{Ross95},
which is given in the Appendix, the equilibrium probabilities
 can be computed with a time complexity of
 and a memory complexity of .

The blocking probability  is obtained by summing
the equilibrium probabilities  of the sets of states
that have less than  available circuit bandwidth, i.e.,

We define the average circuit blocking probability


\subsubsection{Aggregate Circuit Bandwidth}
The performance evaluation for packet delay in
Section~\ref{pkt_perf:sec} requires taking
expectations over the distribution 
of the aggregate bandwidth  occupied by circuits.
In preparation for these packet evaluations, we define
 to denote the expectation of
a function  of the random variable  over the
distribution , i.e., we define

With this definition, the mean aggregate bandwidth of the active circuits is
obtained as

Note that by taking the expectation of (\ref{Xin:eqn}), the corresponding
mean duration of the circuit partition is
.

\subsubsection{Delay and Delay Variation}
\label{cirdel:sec}
In this section we analyze the delay and delay variations
experienced by circuit traffic as it traverses a DyCaPPON network
from ONU to OLT.
Initially we ignore delay variations, i.e., we consider
that a given circuit with bit rate  has a fixed position for
the transmission of its  bits in each cycle.
Three delay components arise:
The ``accumulation/dispersal'' delay of  for the
 bits of circuit traffic that are transmitted per cycle.
Note that the first bit arriving to form a ``chunk'' of
 bits experiences the delay  at the ONU, waiting
for subsequent bits to ``fill up (accumulate)'' the chunk.
The last bit of a chunk experiences essentially no delay at the ONU, but
has to wait for a duration of  at the OLT to ``send out (disperse)''
the chunk at the circuit bit rate .
The other delay components are the transmission delay of 
and the propagation delay .
Thus, the total delay is


Circuit traffic does not experience delay variations (jitter) in
DyCaPPON as long as the positions (in time) of the circuit transmissions
in the cycle are held fixed.
When an ongoing circuit is closing down or a new
circuit is established, it may become necessary to
rearrange the transmission positions of the circuits in the cycle in
order to keep all circuit transmissions within the circuit partition
at the beginning of the cycle and avoid idle times during
the circuit partition.
Adaptations of packing algorithms~\cite{dyck90}
could be employed to minimize the shifts in transmission positions.
Note that for a given circuit service limit ,
the worst-case delay variation for a given circuit with rate  is less than
 as the circuit could at the most shift from
the beginning to the end of the circuit partition of
maximum duration .


\subsection{Packet Traffic}
\label{pkt_perf:sec}
\subsubsection{Stability Limit}
\label{pastab:sec}
Inserting the circuit partition duration  from (\ref{Xin:eqn})
into the expression for the aggregate limit  on the
transmission window for packets in a cycle from (\ref{Gp:eqn})
and taking the expectation  with respect to the distribution
of the aggregate circuit bandwidth , we obtain

Considering the unused slot remainder  (\ref{omegau:eqn}),
the mean portion of a cycle available for
upstream packet traffic transmissions is limited to

That is, the packet traffic intensity  must be less than
 for stability of the packet service, i.e.,
for finite packet delays.

\subsubsection{Mean Delay}
\label{delan:sec}
In this section, we present for stable packet service
an approximate analysis of the mean delay  of packets
transmitted during the packet partition.
In DyCaPPON, packets are transmitted on the bandwidth
that is presently not occupied by admitted circuits.
Thus, fluctuations in the aggregate occupied circuit bandwidth 
affect the packet delays.
If the circuit bandwidth  is presently high,
packets experience longer delays than for presently low
circuit bandwidth .
The aggregated occupied circuit bandwidth  fluctuates
as circuits are newly admitted and occupy bandwidth and as
existing circuits reach the end of their holding time and release
their occupied bandwidth.
The time scale of these fluctuations of  increases
as the average circuit holding time  increases,
i.e., as the circuit departure rate  decreases
(and correspondingly, the circuit request arrival rate  decreases
for a given fixed circuit traffic load )~\cite{gaver82}.

For circuit holding times that are orders of magnitude larger than
the typically packet delays (service times) in the system, the
fluctuations of the circuit bandwidth  occur at a
significantly longer (slower) time scale than the packet service
time scale. That is, the bandwidth  occupied by circuits
exhibits significant correlations over time which in turn give rise
to complex correlations with the packet queueing
delay~\cite{wein80,tham83}. For instance, packets arriving during a
long period of high circuit bandwidth may experience very long
queueing delays and are possibly only served after some circuits
release their bandwidth. As illustrated in
Section~\ref{mu_impact:sec}, the effects of these complex
correlations become significant for scenarios with moderate to long
circuit holding times  when the circuit traffic load is low
to moderate relative to the circuit bandwidth limit  (so that
pronounced circuit bandwidth fluctuations are possible), and the
packet traffic load on the remaining bandwidth of approximately  is relatively high, so that substantial packet queue build-up
can occur. We leave a detailed mathematical analysis of the complex
correlations occurring in these scenarios in the context of DyCaPPON
for future research.

In the present study, we focus on an approximate
packet delay analysis that neglects the outlined correlations.
We base our approximate packet delay analysis
on the expectation  (\ref{Ebeta:eqn}),
i.e., we linearly weigh packet delay metrics 
with the probability masses  for the aggregate
circuit bandwidth .
We also neglect the ``low-load'' operating mode of
Section~\ref{lowload:sec} in the analysis.

In the proposed DyCaPPON cycle structure, a packet experiences five
main components, namely  the reporting delay from the
generation instant of the packet to the transmission of the report
message informing the OLT about the packet,
which for the fixed cycle duration of DyCaPPON
equals half the cycle duration, i.e., ,
 the report-to-packet partition delay 
from the instant of report
transmission to the beginning of the packet partition in the next
cycle,
 the queuing delay  from the reception instant of the
grant message to the beginning of the transmission of the packet, as
well as  the packet transmission delay with mean ,
and  the upstream propagation delay .

In the report-to-packet partition delay we include a delay component
of half the mean duration of the packet partition 
to account for the
delay of the reporting of a particular ONU to the end of the packet partition.
The delay from the end of the packet partition in one cycle
to the beginning of the packet partition of the next cycle is the
maximum of the roundtrip propagation delay  and the mean
duration of the circuit partition .
Thus, we obtain overall for the report-to-packet partition delay


We model the queueing delay with an M/G/1 queue.
Generally, for messages with mean service time ,
normalized message size variance , and
traffic intensity ,
the M/G/1 queue has expected queueing delay~\cite{Kleinrock75}

For DyCaPPON, we model the aggregate packet traffic from all 
ONUs as feeding into one M/G/1 queue with mean packet size 
and packet size variance .
We model the circuit partitions, when the
upstream channel is not serving packet traffic, through scaling of the
packet traffic intensity.
In particular, the upstream channel is available for
serving packet traffic only for the mean fraction
 of a cycle.
Thus, for large backlogs served across several cycles,
the packet traffic intensity during the packet partition is effectively

Hence, the mean queueing delay is approximately

Thus, the overall mean packet delay is approximately



\section{DyCaPPON Performance Results}
\label{eval:sec}
\subsection{Evaluation Setup}
\label{eval_setup:sec}
We consider an EPON with  ONUs, a channel bit rate ~Gb/s,
and a cycle duration ~ms.
Each ONU has abundant buffer space and a one-way propagation delay
of s to the OLT.
The guard time is s and the report message has 64 Bytes.
We consider  classes of circuits as specified in Table~\ref{cir:tab}.
\begin{table}[t]
\caption{Circuit bandwidths  and request probabilities  for 
classes of circuits in performance evaluations.}
\label{cir:tab}
\vspace{-0.25cm}
\begin{center}
\begin{tabular}{|l|rrr|} \hline
   & \multicolumn{3}{|c|}{Class }\\
   &           1  &   2   &  3   \\ \hline
 [Mb/s] & 52 &  156  & 624   \\
 [\%]   & 53.56   & 28.88    & 15.56  \\
\hline
\end{tabular}
\end{center}
\end{table}
A packet has 64~Bytes with 60\% probability,
300~Bytes with 4\% probability,  580~Bytes with 11\% probability,
and 1518~bytes with 25\% probability, thus
the mean packet size is ~Bytes.
The verifying simulations were conducted with a CSIM based simulator and
are reported with 90~\% confidence intervals which are too
small to be visible in the plots.

\begin{table*}[t]
\caption{Circuit blocking probabilities 
from analysis (A) Eqn.~(\ref{Bk:eqn}) with representative
verifying simulations (S) for given offered circuit traffic load ,
circuit bandwidth limit  or 4~Gb/s
and mean circuit holding time .
The blocking probabilities are independent of the packet
traffic load .
Table also gives average circuit traffic bit rate 
from (\ref{beta_avg:eqn}),
mean duration of packet phase  (\ref{Gp_avg:eqn}),
and packet traffic load limit  (\ref{pimax:eqn}).
}
\label{pi_bk:tab}
\begin{center}
\begin{tabular}{|c|c|l|ccc|c|ccc|} \hline         &&&&&&&&& \\
       ~~~~ &  & &    &     &   &   &
              &   &   \\
      & [Gb/s]  & [s] &  [\%] & [\%]     & [\%]      & [\%]  &
              & [ms]  &   \\ \hline
         & 4 & A&   &  &  0.28
          &   & 1.05 & 1.68   & 0.842  \\
       & 2 &  A &  0.93 & 3.2 &  21
          & 4.6  & 0.93 & 1.70  & 0.852  \\
         & 2 & 0.5 S&  0.72  & 2.9 &  21
          & 4.4  & 0.90 &  &   \\
       & 2 & 0.02 S&  1.1 & 3.7 &  22
          & 5.1  & 0.95 &  &   \\ \hline
         & 4 & A&  3.34  &  10.6  &  39.6 & 10.9 &
            3.02  & 1.33  & 0.665  \\
 & 4 & 0.5 S& 3.4 &  11  &  41 & 11 &
            3.0 &  &   \\
          & 4 & 0.02 S&  4.4 &  12  &  42 & 13 &
            3.2 &   &   \\
        & 2 & A&  12.1  &  33.1  &  85.7 & 29.6 &
            1.68  & 1.60  & 0.799  \\
          & 2  &  0.5 S&  12  &  35  &  85 & 30 &
            1.6  &   &   \\
           & 2 &  0.02  S&  13  &  35  &  87 & 31 &
            1.7  &  &    \\ \hline
          & 4 & A&  9.55  & 26.5 &  74.6 & 24.6  &
           3.49  & 1.24   & 0.618  \\
  & 4 & 0.5 S&  10  &  27  &  75 & 25  &
           3.5  &  &   \\
          & 4 & 0.02 S& 13   &  29  &  75 & 28  &
           3.6  &  &   \\
          & 2 & A&  23.5  &  56.6  &  98.3 & 44.7  &
           1.83  & 1.57  & 0.785 \\
          & 2 & 0.5 S&  23  &  57 &  98 & 45  &
           1.8 &   &   \\
           & 2& 0.02 S& 28 & 57  & 98 & 47 &
           1.8  &   &    \\ \hline
\end{tabular}
\end{center}
\end{table*}
\begin{figure}[t]
\begin{center}
\includegraphics[scale=0.6]{fig4.eps}
\end{center}
\caption{Impact of packet traffic load :
Mean packet delay  from simulations (S) and
analysis (A) as a function of total traffic load ,
which is varied by varying  for fixed
circuit traffic load , 0.4, or 0.7.}
\label{fig:pi}
\end{figure}
\begin{figure}[t]
\begin{center}
\includegraphics[scale=0.675]{fig4_1.eps}
\end{center}
\caption{Impact of packet traffic load :
Mean packet delay  from simulations (S) and
analysis (A) as a function of total traffic load ,
which is varied by varying  for fixed
circuit traffic load , 0.4, or 0.7,
with ~Gb/s, and two different  values.}
\label{fig:pi1}
\end{figure}
\begin{figure}[t]
\begin{center}
\includegraphics[scale=0.675]{fig4_2.eps}
\end{center}
\caption{Impact of packet traffic load :
Mean packet delay  from simulations (S) and
analysis (A) as a function of total traffic load ,
which is varied by varying  for fixed
circuit traffic load , 0.4, or 0.7,
with ~Gb/s, and two different  values.}
\label{fig:pi2}
\end{figure}
\begin{figure}[t]
\begin{center}
\includegraphics[scale=0.675]{fig4_3.eps}
\end{center}
\caption{Impact of packet traffic load :
Mean packet delay  from simulations (S) and
analysis (A) as a function of total traffic load ,
which is varied by varying  for fixed
circuit traffic load , 0.4, or 0.7,
with ~s, and two different  values.}
\label{fig:pi3}
\end{figure}
\begin{figure}[t]
\begin{center}
\includegraphics[scale=0.625]{fig4_4.eps}
\end{center}
\caption{Impact of packet traffic load :
Mean packet delay  from simulations (S) and
analysis (A) as a function of total traffic load ,
which is varied by varying  for fixed
circuit traffic load , 0.4, or 0.7,
with ~s, and two different  values.}
\label{fig:pi4}
\end{figure}
\subsection{Impact of Packet Traffic Load }
\label{pi_impact:sec}
In Table~\ref{pi_bk:tab} we present circuit blocking probability results.
In Figs.~\ref{fig:pi}--\ref{fig:pi4} we plot
packet delay results for increasing packet traffic load .
We consider three levels of offered circuit traffic load ,
which are held constant as the packet traffic load  increases.
DyCaPPON ensures consistent circuit service with the
blocking probabilities and
delay characterized in Section~\ref{percir:sec} irrespective
of the packet traffic load , that is, the packet traffic does
\textit{not} degrade the circuit service at all.
Specifically, Table~\ref{pi_bk:tab} gives
the blocking probabilities  as well as the average
circuit blocking probability 
for the different levels of offered circuit traffic load;
these blocking probability values hold for the full range
of packet traffic loads .

We observe from Table~\ref{pi_bk:tab} that for a given
offered circuit traffic load level , the blocking probability increases
with increasing circuit bit rate  as it is less likely that sufficient
bit rate is available for a higher bit rate circuit.
Moreover, we observe that the blocking probabilities increase with
increasing offered circuit traffic load .
This is because the circuit transmission limit  becomes
increasingly saturated with increasing offered circuit load
, resulting in more blocked requests.
The representative simulation results in Table~\ref{pi_bk:tab}
indicate that the stochastic knapsack analysis is accurate,
as has been extensively verified
in the context of general circuit switched systems~\cite{Ross95}.

In Fig.~\ref{fig:pi} we plot the mean packet delay as
a function of the total traffic load, i.e., the sum of offered
circuit traffic load  plus the packet traffic load .
We initially exclude the scenario with , ~Gbps,
and ~s from consideration; this scenario is discussed in
Section~\ref{mu_impact:sec}.
We observe from Fig.~\ref{fig:pi} that for low packet traffic load 
(i.e., for a total traffic load  just above the offered
circuit traffic load ),
the packet delay is nearly independent of the offered circuit traffic
load .
For low packet traffic load, the few packet transmissions
fit easily into the packet partition of the cycle.

We observe from Figs.~\ref{fig:pi}--\ref{fig:pi4}
sharp packet delay increases for
high packet traffic loads  that approach the maximum total
traffic load, i.e., offered circuit traffic load  plus maximum
packet traffic load .
For ~Gb/s, the maximum packet traffic load
 is 0.85 for  and 0.78 for , see
Table~\ref{pi_bk:tab}. Note that the maximum packet traffic load
 depends on the offered circuit traffic load  and
the circuit traffic limit . For a low offered circuit traffic
load  relative to , few circuit requests are blocked
and the admitted circuit traffic load (equivalently mean aggregate
circuit bandwidth ) is close to the offered circuit
load . On the other hand, for high offered circuit traffic
load , many circuit requests are blocked, resulting in an
admitted circuit traffic load (mean aggregate circuit bandwidth
) significantly below the offered circuit traffic load
. Thus, the total (normalized) traffic load, i.e., offered
circuit load  plus packet traffic load ,
in a stable network can exceed one for
high offered circuit traffic load .


\subsection{Impact of Mean Circuit Holding Time}
\label{mu_impact:sec}
\begin{figure}[t]
\includegraphics[scale=0.675]{fig5.eps}
\caption{Mean packet delay  and standard deviation of
packet delay as a function of mean circuit
holding time ; fixed parameters
, .}
\label{fig:mu}
\end{figure}
We now turn to the packet delay results for the
scenario with low circuit traffic load  relative to the
circuit bandwidth limit ~Gbps and moderately long
mean circuit holding time ~s,
which is included in Figs.~\ref{fig:pi} and~\ref{fig:pi4}.
We observe for this scenario that the mean packet delays
obtained from the simulations begin to increase dramatically
as the total load  approaches 0.8.
In contrast, for the circuit traffic load 
in conjunction with the lower circuit bandwidth limit ~Gbps
and short mean circuit holding times ~s,
the mean packet delays remain low for total loads up to
close to the total maximum load 
and then increase sharply.

The pronounced delay increases at lower loads (in the 0.75--0.92
range) for the , ~Gbps, ~s
scenario are mainly due to the higher-order complex correlations
between the pronounced slow-time scale fluctuations of the
circuit bandwidth and the packet queueing
as explained in Section~\ref{delan:sec}.
The high circuit bandwidth limit ~Gbps relative to the
low circuit traffic load  allows
pronounced fluctuations of the aggregate occupied circuit bandwidth
. For the moderately long mean circuit holding time ~s, these pronounced fluctuations occur at a long time scale
relative to the packet service time scales, giving rise to
pronounced correlation effects. That is, packets arriving during
periods of high circuit bandwidth  may need to wait (queue)
until some circuits end and release sufficient bandwidth to serve
the queued packet backlog. These correlation effects are neglected
in our approximate packet delay analysis in Section~\ref{delan:sec}
giving rise to the large discrepancy between simulation and analysis
observed for the , ~Gb/s, ~s scenario in
Fig.~\ref{fig:pi}.

We observe from Fig.~\ref{fig:pi} for the scenarios with
relatively high circuit traffic loads  and~0.7
relative to the considered circuit bandwidth limits 
and 4~Gbps that the mean packet delays remain low up to
levels of the total load close to the total stability limit
 predicted from the stability analysis
in Section~\ref{pastab:sec}.
The relatively high circuit traffic loads 
lead to high circuit blocking probabilities
(see Table~\ref{pi_bk:tab}) and the admitted circuits utilize the
available circuit traffic bandwidth  nearly fully
for most of the time. Vacant portions of the
circuit bandwidth  are quickly occupied by the frequently arriving
new circuit requests. Thus, there are only relatively minor
fluctuations of the bandwidth available for packet service
and the approximate packet delay analysis is quite accurate.

Returning to the scenario with relatively low circuit traffic load
 in Fig.~\ref{fig:pi}, we observe that for
the short mean circuit holding time , the
mean packet delays remain low up to load levels close to the
stability limit .
For these relatively short circuit durations, the pronounced
fluctuations of the occupied circuit bandwidth occur on a sufficiently
short time scale to avoid significant higher-order correlations between
the circuit bandwidth and the packet service.

We examine these effects in more detail in
Fig.~\ref{fig:mu}, which shows means and standard deviations of
packet delays as a function of the mean circuit holding time 
for fixed traffic load , . We observe that
for the high ~Gbps circuit bandwidth limit, the mean packet
delay as well as the standard deviation of the packet delay obtained
from simulations increase approximately linearly with increasing
mean circuit holding time . The ~Gbps circuit
bandwidth limit permits sufficiently large fluctuations of the
circuit bandwidth  for the  load, such that for
increasing circuit holding time, the packets increasingly experience
large backlogs that can only be cleared when some circuits end and
release their bandwidth. In contrast,
for the lower circuit bandwidth limit ~Gbps, which severely
limits fluctuations of the circuit bandwidth  for the high
circuit traffic load , the mean and standard deviation
of the packet delay remain essentially constant for increasing
.


\begin{table}[t]
\caption{Mean circuit blocking probability 
and mean packet delay 
as a function of circuit traffic load ;
fixed parameters: circuit bandwidth limit ~Gb/s,
packet traffic load .}
\label{fig:chi}
\vspace{-0.75cm}
\begin{center}
\begin{tabular}{|l|ccccccc|} \hline
 &   &    &      &  
        &  &  &   \\ \hline
, S [\%] & 0 & 1.2  &  5.1  &  16 & 31 & 43  & \\
, A [\%] & 0.016 & 1.08  &  4.81 &  14.9 & 29.6  & 40.1 &100\\ \hline
, S [ms] & 1.9 &  2.0  & 2.0  &  2.1 & 2.2 &  2.2&  \\
, A [ms] & 2.10&  2.11 & 2.13 & 2.16 & 2.21& 2.23& 2.42  \\   \hline
\end{tabular}
\end{center}
\end{table}
\subsection{Impact of Offered Circuit Traffic Load }
\label{chi_impact:sec}
In Table.~\ref{fig:chi}, we examine the impact of the
circuit traffic load  on the DyCaPPON performance
more closely.
We keep the packet traffic load fixed at 
and examine the average circuit blocking probability 
and the mean packet delay  as a function of the circuit traffic load .
We observe from Table.~\ref{fig:chi} that, as expected, the
mean circuit blocking probability  increases with increasing
circuit traffic load , whereby analysis closely matches the simulations.

\begin{figure}[t]
\begin{tabular}{c}
\includegraphics[scale=0.615]{fig6a.eps}  \\
{\scriptsize (a) Mean request blocking probability } \\
\includegraphics[scale=0.615]{fig6b.eps}   \\
{\scriptsize (b) Mean packet delay } \\
\end{tabular}
\caption{Impact of circuit service limit :
Mean circuit blocking probability 
(from analysis, Eqn.~(\ref{Bk:eqn}))
and mean packet delay  (from analysis and simulation)
as a function of transmission rate limit for circuit service ;
fixed mean circuit holding time ~s. }
\label{fig:Cc}
\end{figure}
For the packet traffic, we observe from
Table~\ref{fig:chi} a very slight increase in the mean packet delays 
as the circuit traffic load  increases.
This is mainly because the transmission rate limit  for circuit
service bounds the upstream transmission bandwidth the circuits
can occupy to no more than  in each cycle.
As the circuit traffic load  increases, the circuit traffic
utilizes this transmission rate limit  more and more fully.
However, the packet traffic is guaranteed a portion
 of the upstream transmission bandwidth.
Formally, as the circuit traffic load  grows large
(), the
mean aggregate circuit bandwidth  approaches the limit
, resulting in a lower bound for the packet traffic load
limit~(\ref{pimax:eqn}) of

and corresponding upper bounds for the
effective packet traffic intensity  and
the mean packet delay .


\subsection{Impact of Limit  for Circuit Service}
\label{Cc_impact:sec} In Fig.~\ref{fig:Cc} we examine the impact of
the transmission rate limit  for circuit traffic. We consider
different compositions  of the total traffic load . We observe from Fig.~\ref{fig:Cc}(a) that the average
circuit blocking probability  steadily decreases for
increasing . In the example in Fig.~\ref{fig:Cc}, the average
circuit blocking probability  drops to negligible values
below 1~\% for  values corresponding to roughly twice the
offered circuit traffic load . For instance, for circuit load
,  drops to 0.9~\% for ~Gb/s. The
limit  thus provides an effective parameter for controlling
the circuit blocking probability experienced by customers.

From Fig.~\ref{fig:Cc}(b), we observe that the mean packet delay
abruptly increases when the  limit reduces the packet traffic
portion  of the upstream transmission bandwidth to values
near the packet traffic intensity . We also observe from
Fig.~\ref{fig:Cc}(b) that the approximate packet delay analysis is
quite accurate for small to moderate  values (the slight delay
overestimation is due to neglecting the low packet traffic polling),
but underestimates the packet delays for large . Large circuit
traffic limits  give the circuit traffic more flexibility for
causing fluctuations of the occupied circuit bandwidth, which
deteriorate the packet service. Summarizing, we see from
Fig.~\ref{fig:Cc}(b) that as the effective packet traffic intensity
 approaches one, the mean packet delay increases
sharply. Thus, for ensuring low-delay packet service, the limit
 should be kept sufficiently below .

When offering circuit and packet service over shared
PON upstream transmission bandwidth, network service providers need to
trade off the circuit blocking probabilities and packet delays.
As we observe from Fig.~\ref{fig:Cc}, the circuit bandwidth limit 
provides an effective tuning knob for controlling this trade-off.


\subsection{Impact of Low-Packet-Traffic Mode Polling}
\label{Lowtraffic:sec}
\begin{figure}[t]
\includegraphics[scale=0.65]{fig9.eps}
\caption{Impact of low-packet-traffic polling mode:
Mean packet delay  as a function of packet traffic load .
}
\label{fig:LM}
\end{figure}
The Fig.~\ref{fig:LM} we examine the impact of
low-packet-traffic mode polling from Section~\ref{lowload:sec}
on the mean packet delay .
We observe from Fig.~\ref{fig:LM} that low-packet-traffic mode
polling substantially reduces the mean packet delay
compared to conventional polling for low packet traffic loads.
This delay reduction is achieved by the
the more frequent polling which serves packets quicker in cycles with
low load due to circuit traffic.


\section{Conclusion}
\label{sec:conclusion}
We have proposed and evaluated DyCaPPON, a passive optical network
that provides dynamic circuit and packet service.
DyCaPPON is based on fixed duration cycles, ensuring
consistent circuit service, that is completely unaffected by
the packet traffic load.
DyCaPPON masks the round-trip propagation delay for
polling of the packet traffic queues in the ONUs with the
upstream circuit traffic transmissions, providing for efficient
usage of the upstream bandwidth.
We have analyzed the circuit level performance,
including the circuit blocking probability and delay
experienced by circuit traffic in DyCaPPON, as well as
the bandwidth available for packet traffic after serving the
circuit traffic.
We have also conducted an approximate analysis of
the packet level performance.

Through extensive numerical investigations based on
the analytical performance characterization of DyCaPPON as well as
verifying simulations, we have demonstrated the
circuit and packet traffic performance and trade-offs in DyCaPPON.
The provided analytical performance characterizations as well
as the identified performance trade-offs provide tools and guidance
for dimensioning and operating PON access networks that
provide a mix of circuit and packet oriented service.

There are several promising directions for
future research on access networks that flexibly
provide both circuit and packet service.
One important future research direction is to broadly
examine cycle-time structures and wavelength assignments in PONs
providing circuit and packet service.
In particular, the present study focused on a single upstream
wavelength channel operated with a fixed polling cycle duration.
Future research should examine the trade-offs arising from
operating multiple upstream wavelength channels and combinations
of fixed- or variable-duration polling cycles.
An exciting future research
direction is to extend the PON service further toward the
individual user, e.g., by providing circuit and packet service
on integrated PON and wireless access networks,
such as~\cite{AuLMR14,Coim13,DhHJ11,LiKK13,MaGR09,Morad13},
that reach individual mobile users or
wireless sensor networks~\cite{HoH11,SeR11,YuZD12}.
Further, exploring combined circuit and packet service in
long-reach PONs with very long round trip propagation delays,
which may require special protocol mechanisms,
see e.g.,~\cite{Mou05,MeMR13,SKM0110},
is an open research direction.
Another direction is to examine the integration and interoperation
of circuit and packet service in the PON access network with
metropolitan area
networks~\cite{BiBC13,MaRe04,MaRW03,ScMRW03,YaMRC03,YuCL10}
and wide area networks to provide
circuit and packet service~\cite{CircuitSONET}.

\vspace{\baselineskip}

\noindent \textsc{Appendix: Evaluation of Equilibrium Probabilities
}

In this Appendix, we present the recursive
Kaufman-Roberts algorithm~\cite[p. 23]{Ross95} for computing the
equilibrium probabilities  that
the currently active circuit occupy an aggregated bandwidth .
For the execution of the algorithm, the given circuit bandwidths
 and limit  are suitably normalized so
that incrementing  in integer steps covers all possible
combinations of the circuit bandwidth. For instance, in the
evaluation scenario considered in Section~\ref{eval_setup:sec}, all
circuit bandwidth are integer multiples of 52 Mb/s. Thus, we
normalize all bandwidths by 52 Mb/s and for e.g., ~Gb/s execute
the following algorithm for . (The
variables , and  refer to their normalized values,
e.g.,  for the ~Gb/s example, in the algorithm below).

The algorithm first evaluates unnormalized occupancy probabilities
 that relate to a product-form solution of the
stochastic knapsack~\cite{Ross95}. Subsequently the normalization
term  for the occupancy probabilities is evaluated, allowing then
the evaluation of the actual occupancy probabilities .

1. Set  and  for .

2. For , set


3. Set


4. For , set


\bibliographystyle{IEEEtran}

\begin{thebibliography}{100}
\providecommand{\url}[1]{#1}
\csname url@samestyle\endcsname
\providecommand{\newblock}{\relax}
\providecommand{\bibinfo}[2]{#2}
\providecommand{\BIBentrySTDinterwordspacing}{\spaceskip=0pt\relax}
\providecommand{\BIBentryALTinterwordstretchfactor}{4}
\providecommand{\BIBentryALTinterwordspacing}{\spaceskip=\fontdimen2\font plus
\BIBentryALTinterwordstretchfactor\fontdimen3\font minus
  \fontdimen4\font\relax}
\providecommand{\BIBforeignlanguage}[2]{{\expandafter\ifx\csname l@#1\endcsname\relax
\typeout{** WARNING: IEEEtran.bst: No hyphenation pattern has been}\typeout{** loaded for the language `#1'. Using the pattern for}\typeout{** the default language instead.}\else
\language=\csname l@#1\endcsname
\fi
#2}}
\providecommand{\BIBdecl}{\relax}
\BIBdecl

\bibitem{WeAMR14}
X.~Wei, F.~Aurzada, M.~McGarry, and M.~Reisslein, ``{DyCaPPON}: Dynamic circuit
  and packet passive optical network,'' \emph{Optical Switching and Networking,
  in print}, 2014.

\bibitem{PerformCS}
P.~Molinero-Fernandez and N.~McKeown, ``The performance of circuit switching in
  the internet,'' \emph{OSA J. Opt. Netw.}, vol.~2, no.~4, pp. 83--96, Apr.
  2003.

\bibitem{OCSvsOBS01}
T.~Coutelen, H.~Elbiaze, and B.~Jaumard, ``Performance comparison of {OCS} and
  {OBS} switching paradigms,'' in \emph{Proc. Transparent Optical Networks},
  Jul. 2005.

\bibitem{OVSvsOBSpkt}
X.~Liu, C.~Qiao, X.~Yu, and W.~Gong, ``A fair packet-level performance
  comparison of {OBS} and {OCS},'' in \emph{Proc. OFC}, Mar. 2006.

\bibitem{NEHONET}
M.~Batayneh, D.~Schupke, M.~Hoffmann, A.~Kirstaedter, and B.~Mukherjee,
  ``Link-rate assignment in a {WDM} optical mesh network with differential link
  capacities: A network-engineering approach,'' in \emph{Proc. HONET}, Nov.
  2008, pp. 216--219.

\bibitem{VeKC10}
M.~Veeraraghavan, M.~Karol, and G.~Clapp, ``Optical dynamic circuit services,''
  \emph{IEEE Commun. Mag.}, vol.~48, no.~11, pp. 109--117, Nov. 2010.

\bibitem{DCS1989}
A.~Harissis and A.~Ambler, ``A new multiprocessor interconnection network with
  wide sense nonblocking capabilities,'' in \emph{Proc. Midwest Symp. Circuits
  and Systems}, Aug. 1989, pp. 923--926.

\bibitem{SBCD1009}
M.~Satyanarayanan, P.~Bahl, R.~Caceres, and N.~Davies, ``The case for
  {VM}-based cloudlets in mobile computing,'' \emph{IEEE Pervasive Comp.},
  vol.~8, no.~4, pp. 14--23, Oct. 2009.

\bibitem{BrF10}
M.~Bredel and M.~Fidler, ``A measurement study regarding quality of service and
  its impact on multiplayer online games,'' in \emph{Proc. NetGames}, 2010.

\bibitem{FiGR02}
F.~Fitzek, G.~Schulte, and M.~Reisslein, ``System architecture for billing of
  multi-player games in a wireless environment using {GSM/UMTS} and {WLAN}
  services,'' in \emph{Proc. ACM NetGames}, 2002, pp. 58--64.

\bibitem{MaH10}
M.~Maier and M.~Herzog, ``Online gaming and {P2P} file sharing in
  next-generation {EPONs},'' \emph{IEEE Communications Magazine}, vol.~48,
  no.~2, pp. 48--55, Feb. 2010.

\bibitem{ScER02}
C.~Schaefer, T.~Enderes, H.~Ritter, and M.~Zitterbart, ``Subjective quality
  assessment for multiplayer real-time games,'' in \emph{Proc. ACM NetGames},
  2002, pp. 74--78.

\bibitem{KuB13}
G.~Kurillo and R.~Bajcsy, ``{3D} teleimmersion for collaboration and
  interaction of geographically distributed users,'' \emph{Virtual Reality},
  vol.~17, no.~1, pp. 29--43, 2013.

\bibitem{PaDR12}
M.~Pallot, P.~Daras, S.~Richir, and E.~Loup-Escande, ``{3D}-live: live
  interactions through {3D} visual environments,'' in \emph{Proc. Virtual
  Reality Int. Conf.}, 2012.

\bibitem{VaZK10}
R.~Vasudevan, Z.~Zhou, G.~Kurillo, E.~Lobaton, R.~Bajcsy, and K.~Nahrstedt,
  ``Real-time stereo-vision system for {3D} teleimmersive collaboration,'' in
  \emph{Proc. IEEE ICME}, 2010, pp. 1208--1213.

\bibitem{ghazi2012vmp}
N.~Ghazisaidi, M.~Maier, and M.~Reisslein, ``{VMP}: A {MAC} protocol for
  {EPON}-based video-dominated {FiWi} access networks,'' \emph{IEEE Trans.
  Broadcasting}, vol.~58, no.~3, pp. 440--453, 2012.

\bibitem{oh2008cont}
S.~Oh, B.~Kulapala, A.~Richa, and M.~Reisslein, ``Continuous-time collaborative
  prefetching of continuous media,'' \emph{IEEE Transactions on Broadcasting},
  vol.~54, no.~1, pp. 36--52, Mar. 2008.

\bibitem{QiKo11}
L.~Qiao and P.~Koutsakis, ``Adaptive bandwidth reservation and scheduling for
  efficient wireless telemedicine traffic transmission,'' \emph{IEEE Trans.\
  Vehicular Technology}, vol.~60, no.~2, pp. 632--643, Feb. 2011.

\bibitem{ReLR02}
M.~Reisslein, J.~Lassetter, S.~Ratnam, O.~Lotfallah, F.~Fitzek, and
  S.~Panchanathan, ``Traffic and quality characterization of scalable encoded
  video: a large-scale trace-based study, {Part 1}: overview and definitions,''
  Arizona State Univ., Tech. Rep., 2002.

\bibitem{ReT99}
J.~Rexford and D.~Towsley, ``Smoothing variable-bit-rate video in an
  internetwork,'' \emph{IEEE/ACM Transactions on Networking}, vol.~7, no.~2,
  pp. 202--215, Apr. 1999.

\bibitem{ShSZ11}
K.~Shuaib, F.~Sallabi, and L.~Zhang, ``Smoothing and modeling of video
  transmission rates over a {QoS} network with limited bandwidth connections,''
  \emph{Int. Journal of Computer Networks and Communications}, vol.~3, no.~3,
  pp. 148--162, May 2011.

\bibitem{AuRe09}
G.~{Van der Auwera} and M.~Reisslein, ``Implications of smoothing on
  statistical multiplexing of {H. 264/AVC} and {SVC} video streams,''
  \emph{IEEE Trans. on Broadcasting}, vol.~55, no.~3, pp. 541--558, Sep. 2009.

\bibitem{DOCS00}
B.~Mukherjee, ``Architecture, control, and management of optical switching
  networks,'' in \emph{Proc. Photonics in Switching}, Aug. 2007, pp. 43--44.

\bibitem{DCN}
Internet2, ``Dynamic circuit network,'' http://www.internet2.edu/network/dc/.

\bibitem{Charb12}
N.~Charbonneau, A.~Gadkar, B.~H. Ramaprasad, and V.~Vokkarane, ``Dynamic
  circuit provisioning in all-optical {WDM} networks using lightpath
  switching,'' \emph{Opt. Sw. Netw.}, vol.~9, no.~2, pp. 179 -- 190, 2012.

\bibitem{HybridN2010}
X.~Fang and M.~Veeraraghavan, ``A hybrid network architecture for file
  transfers,'' \emph{IEEE Transactions on Parallel and Distributed Systems},
  vol.~20, no.~12, pp. 1714--1725, Dec. 2009.

\bibitem{Li08}
Z.~Li, Q.~Song, and I.~Habib, ``{CHEETAH} virtual label switching router for
  dynamic provisioning in {IP} optical networks,'' \emph{Optical Switching and
  Netw.}, vol.~5, no. 2–3, pp. 139--149, 2008.

\bibitem{MGJT0511}
I.~Monga, C.~Guok, W.~Johnston, and B.~Tierney, ``Hybrid networks: lessons
  learned and future challenges based on {ESnet4} experience,'' \emph{IEEE
  Commun. Mag.}, vol.~49, no.~5, pp. 114--121, May 2011.

\bibitem{ReqprovDOCS}
A.~Munir, S.~Tanwir, and S.~Zaidi, ``Requests provisioning algorithms for
  {Dynamic Optical Circuit Switched (DOCS)} networks: A survey,'' in
  \emph{Proc. IEEE Int. Multitopic Conference (INMIC)}, Dec. 2009, pp. 1--6.

\bibitem{Sko12}
R.~Skoog, G.~Clapp, J.~Gannett, A.~Neidhardt, A.~{Von Lehman}, and B.~Wilson,
  ``Architectures, protocols and design for highly dynamic optical networks,''
  \emph{Opt. Switch. Netw.}, vol.~9, no.~3, pp. 240--251, 2012.

\bibitem{VanBreu05}
E.~{Van Breusegem}, J.~Cheyns, D.~{De Winter}, D.~Colle, M.~Pickavet,
  P.~Demeester, and J.~Moreau, ``A broad view on overspill routing in optical
  networks: a real synthesis of packet and circuit switching?'' \emph{Optical
  Switching and Networking}, vol.~1, no.~1, pp. 51--64, 2005.

\bibitem{CircuitSONET}
M.~Veeraraghavan and X.~Zheng, ``A reconfigurable {Ethernet/SONET}
  circuit-based metro network architecture,'' \emph{IEEE J. Selected Areas on
  Communications}, vol.~22, no.~8, pp. 1406--1418, Oct. 2004.

\bibitem{Mahloo13}
M.~Mahloo, C.~M. Machuca, J.~Chen, and L.~Wosinska, ``Protection cost
  evaluation of {WDM}-based next generation optical access networks,''
  \emph{Optical Switching and Netw.}, vol.~10, no.~1, pp. 89--99, 2013.

\bibitem{McR12}
M.~McGarry and M.~Reisslein, ``Investigation of the {DBA} algorithm design
  space for {EPONs},'' \emph{IEEE/OSA Journal of Lightwave Technology},
  vol.~30, no.~14, pp. 2271--2280, Jul. 2012.

\bibitem{McRAS10}
M.~McGarry, M.~Reisslein, F.~Aurzada, and M.~Scheutzow, ``Shortest propagation
  delay {(SPD)} first scheduling for {EPONs} with heterogeneous propagation
  delays,'' \emph{IEEE J. on Selected Areas in Commun.}, vol.~28, no.~6, pp.
  849--862, Aug. 2010.

\bibitem{Siva13}
A.~Sivakumar, G.~Sankaran, and K.~Sivalingam, ``Performance analysis of
  {ONU}-wavelength grouping schemes for efficient scheduling in long
  reach-{PONs},'' \emph{Opt. Switching Netw.}, vol.~10, no.~4, pp. 465--474,
  2013.

\bibitem{ToKK12}
I.~Tomkos, L.~Kazovsky, and K.-I. Kitayama, ``Next-generation optical access
  networks: dynamic bandwidth allocation, resource use optimization, and {QoS}
  improvements,'' \emph{IEEE Netw.}, vol.~26, no.~2, pp. 4--6, 2012.

\bibitem{Zan2013}
F.~Zanini, L.~Valcarenghi, D.~P. Van, M.~Chincoli, and P.~Castoldi,
  ``Introducing cognition in {TDM} {PONs} with cooperative cyclic sleep through
  runtime sleep time determination,'' \emph{Opt. Switching Netw., in print},
  2013.

\bibitem{AuSH08}
F.~Aurzada, M.~Scheutzow, M.~Herzog, M.~Maier, and M.~Reisslein, ``Delay
  analysis of {Ethernet} passive optical networks with gated service,''
  \emph{OSA Journal of Optical Networking}, vol.~7, no.~1, pp. 25--41, Jan.
  2008.

\bibitem{AuSRGM11}
F.~Aurzada, M.~Scheutzow, M.~Reisslein, N.~Ghazisaidi, and M.~Maier, ``Capacity
  and delay analysis of next-generation passive optical networks {(NG-PONs)},''
  \emph{IEEE Trans. on Communications}, vol.~59, no.~5, pp. 1378--1388, May
  2011.

\bibitem{ZhMo09}
J.~Zheng and H.~Mouftah, ``A survey of dynamic bandwidth allocation algorithms
  for {Ethernet Passive Optical Networks},'' \emph{Optical Switching and
  Networking}, vol.~6, no.~3, pp. 151--162, Jul. 2009.

\bibitem{AnLA04}
J.~Angelopoulos, H.-C. Leligou, T.~Argyriou, S.~Zontos, E.~Ringoot, and
  T.~{Van~Caenegem}, ``Efficient transport of packets with {QoS} in an
  {FSAN}-aligned {GPON},'' \emph{IEEE Comm. Mag.}, vol.~42, no.~2, pp. 92--98,
  Feb. 2004.

\bibitem{AsYDA03}
C.~Assi, Y.~Ye, S.~Dixit, and M.~Ali, ``Dynamic bandwidth allocation for
  quality-of-service over {Ethernet PONs},'' \emph{IEEE Journal on Selected
  Areas in Communications}, vol.~21, no.~9, pp. 1467--1477, Nov. 2003.

\bibitem{DiDLC11}
A.~Dixit, G.~Das, B.~Lannoo, D.~Colle, M.~Pickavet, and P.~Demeester, ``Jitter
  performance for {QoS} in {Ethernet} passive optical networks,'' in
  \emph{Proceedings of ECOC}, 2011, pp. 1--3.

\bibitem{GhSA04}
N.~Ghani, A.~Shami, C.~Assi, and M.~Raja, ``Intra-{ONU} bandwidth scheduling in
  {Ethernet} passive optical networks,'' \emph{IEEE Communications Letters},
  vol.~8, no.~11, pp. 683--685, 2004.

\bibitem{LuAn05}
Y.~Luo and N.~Ansari, ``Limited sharing with traffic prediction for dynamic
  bandwidth allocation and {QoS} provioning over {EPONs},'' \emph{OSA Journal
  of Optical Networking}, vol.~4, no.~9, pp. 561--572, Sep. 2005.

\bibitem{RaM09}
M.~Radivojevic and P.~Matavulj, ``Implementation of intra-{ONU} scheduling for
  {Quality of Service} support in {Ethernet} passive optical networks,''
  \emph{IEEE/OSA J. Lightw. Techn.}, vol.~27, no.~18, pp. 4055--4062, Sep.
  2009.

\bibitem{ShHEAA04}
S.~Sherif, A.~Hadjiantonis, G.~Ellinas, C.~Assi, and M.~Ali, ``A novel
  decentralized ethernet-based {PON} access architecture for provisioning
  differentiated {QoS},'' \emph{IEEE/OSA J.\ of Lightwave Technology}, vol.~22,
  no.~11, pp. 2483--2497, Nov. 2004.

\bibitem{ShBGAM05}
A.~Shami, X.~Bai, N.~Ghani, C.~Assi, and H.~Mouftah, ``{QoS} control schemes
  for two-stage {Ethernet} passive optical access networks,'' \emph{IEEE J. on
  Sel. Areas in Commun.}, vol.~23, no.~8, pp. 1467--1478, Nov. 2005.

\bibitem{VaGh11}
M.~Vahabzadeh and A.~{Ghaffarpour Rahbar}, ``Modified smallest available report
  first: New dynamic bandwidth allocation schemes in {QoS-capable EPONs},''
  \emph{Optical Fiber Technolgy}, vol.~17, no.~1, pp. 7--16, Jan. 2011.

\bibitem{BeBM09}
T.~Berisa, A.~Bazant, and V.~Mikac, ``Bandwidth and delay guaranteed polling
  with adaptive cycle time {(BDGPACT)}: A scheme for providing bandwidth and
  delay guarantees in passive optical networks,'' \emph{J. of Opt. Netw.},
  vol.~8, no.~4, pp. 337--345, 2009.

\bibitem{Ho06}
T.~Holmberg, ``Analysis of {EPONs} under the static priority scheduling scheme
  with fixed transmission times,'' in \emph{Proc. NGI}, 2006, pp. 1--8.

\bibitem{MZC0303}
M.~Ma, Y.~Zhu, and T.~Cheng, ``A bandwidth guaranteed polling {MAC} protocol
  for {Ethernet} passive optical networks,'' in \emph{Proc. IEEE Infocom}, Mar.
  2003, pp. 22--31.

\bibitem{Qin13}
Y.~Qin, D.~Xue, L.~Zhao, C.~K. Siew, and H.~He, ``A novel approach for
  supporting deterministic quality-of-service in {WDM} {EPON} networks,''
  \emph{Optical Switching and Networking}, vol.~10, no.~4, pp. 378--392, 2013.

\bibitem{ZhAY03}
L.~Zhang, E.-S. An, H.-G. Yeo, and S.~Yang, ``Dual {DEB-GPS} scheduler for
  delay-constraint applications in {Ethernet} passive optical networks,''
  \emph{IEICE Trans. Commun.}, vol.~86, no.~5, pp. 1575--1584, 2003.

\bibitem{ZhP04}
L.~Zhang and G.-S. Poo, ``Delay constraint dynamic bandwidth allocation in
  {Ethernet Passive Optical Networks},'' in \emph{Proc. ICCS}, 2004, pp.
  126--130.

\bibitem{AHKWK0703}
F.~An, Y.~Hsueh, K.~Kim, I.~White, and L.~Kazovsky, ``A new dynamic bandwidth
  allocation protocol with quality of service in {Ethernet}-based passive
  optical networks,'' in \emph{Proc. IASTED WOC}, vol.~3, Jul. 2003, pp.
  165--169.

\bibitem{BeIB11}
T.~Berisa, Z.~Ilic, and A.~Bazant, ``Absolute delay variation guarantees in
  passive optical networks,'' \emph{IEEE/OSA Journal of Lightwave Technology},
  vol.~29, no.~9, pp. 1383--1393, May 2011.

\bibitem{DhAMS07}
A.~Dhaini, C.~Assi, M.~Maier, and A.~Shami, ``Per-stream {QoS} and admission
  control in {Ethernet} passive optical networks {(EPONs)},'' \emph{IEE/OSA J.
  Lightwave Tech.}, vol.~25, no.~7, pp. 1659--1669, 2007.

\bibitem{HwLLL12}
I.~Hwang, J.~Lee, K.~Lai, and A.~Liem, ``Generic {QoS}-aware interleaved
  dynamic bandwidth allocation in scalable {EPONs},'' \emph{IEEE/OSA J. of
  Optical Commun. and Netw.}, vol.~4, no.~2, pp. 99--107, Feb. 2012.

\bibitem{LiLC11}
H.-T. Lin, C.-L. Lai, W.-R. Chang, and C.-L. Liu, ``{FIPACT}: A frame-oriented
  dynamic bandwidth allocation scheme for triple-play services over {EPONs},''
  in \emph{Proc. ICCCN}, 2011, pp. 1--6.

\bibitem{Merayo2010}
N.~Merayo, T.~Jimenez, P.~Fernandez, R.~Duran, R.~Lorenzo, I.~de~Miguel, and
  E.~Abril, ``{A bandwidth assignment polling algorithm to enhance the
  efficiency in QoS long-reach EPONs},'' \emph{Eu. Trans. Telecommun.},
  vol.~22, no.~1, pp. 35--44, Jan. 2011.

\bibitem{NgGB11}
M.~Ngo, A.~Gravey, and D.~Bhadauria, ``Controlling {QoS} in {EPON}-based {FTTX}
  access networks,'' \emph{Telec. Sys.}, vol.~48, no. 1-2, pp. 203--217, 2011.

\bibitem{De12}
S.~De, V.~Singh, H.~M. Gupta, N.~Saxena, and A.~Roy, ``A new predictive dynamic
  priority scheduling in {Ethernet} passive optical networks {(EPONs)},''
  \emph{Opt. Switch. Netw.}, vol.~7, no.~4, pp. 215--223, 2010.

\bibitem{PeFA09}
F.~{Melo Pereira}, N.~L.~S. Fonseca, and D.~S. Arantes, ``A fair scheduling
  discipline for {Ethernet} passive optical networks,'' \emph{Computer
  Networks}, vol.~53, no.~11, pp. 1859--1878, Jul. 2009.

\bibitem{YiP10}
Y.~Yin and G.~Poo, ``User-oriented hierarchical bandwidth scheduling for
  ethernet passive optical networks,'' \emph{Comp. Comm.}, vol.~33, no.~8, pp.
  965--975, May 2010.

\bibitem{OFDMPON}
D.~Qian, J.~Hu, J.~Yu, P.~Ji, L.~Xu, T.~Wang, M.~Cvijetic, and T.~Kusano,
  ``Experimental demonstration of a novel {OFDM-A} based 10~{Gb/s} {PON}
  architecture,'' \emph{Proc. ECOC}, pp. 1--2, Sep. 2007.

\bibitem{SeBP05}
J.~Segarra, C.~Bock, and J.~Prat, ``Hybrid {WDM/TDM PON} based on bidirectional
  reflective {ONUs} offering differentiated {QoS} via {OBS},'' in \emph{Proc.
  Transparent Optical Networks}, vol.~1, 2005, pp. 95--100.

\bibitem{VaML12}
J.~Vardakas, I.~Moscholios, M.~Logothetis, and V.~Stylianakis, ``Blocking
  performance of multi-rate {OCDMA PONs} with {QoS} guarantee,'' \emph{Int. J.
  Adv. Telecom.}, vol.~5, no. 3 and 4, pp. 120--130, 2012.

\bibitem{KwYZ96}
W.~Kwong, G.-C. Yang, and J.-G. Zhang, `` prime-sequence codes and coding
  architecture for optical code-division multiple-access,'' \emph{IEEE
  Transactions on Communications}, vol.~44, no.~9, pp. 1152--1162, 1996.

\bibitem{DSAlgfle}
M.~Hu, W.~Guo, and W.~Hu, ``Dynamic scheduling algorithms for large file
  transfer on multi-user optical grid network based on efficiency and
  fairness,'' in \emph{Proc. ICNS}, april 2009, pp. 493--498.

\bibitem{LgfleMulp}
D.~Moser, E.~Cano, and P.~Racz, ``Secure large file transfer over multiple
  network paths,'' in \emph{Proc. IEEE Network Operations and Management
  Symposium (NOMS)}, Apr. 2010, pp. 777--792.

\bibitem{EIBT}
X.~Wei, F.~Aurzada, M.~McGarry, and M.~Reisslein, ``{EIBT}: Exclusive intervals
  for bulk transfers on {EPONs},'' \emph{IEEE/OSA J. Lightwave Technology},
  vol.~31, no.~1, pp. 99--110, Jan. 2013.

\bibitem{bolla97}
R.~Bolla and F.~Davoli, ``Control of multirate synchronous streams in hybrid
  {TDM} access networks,'' \emph{IEEE/ACM Trans. Netw.}, vol.~5, no.~2, pp.
  291--304, 1997.

\bibitem{gaver82}
D.~Gaver and J.~Lehoczky, ``Channels that cooperatively service a data stream
  and voice messages,'' \emph{IEEE Trans. Commun.}, vol.~30, no.~5, pp.
  1153--1162, 1982.

\bibitem{ghani1994decomp}
S.~Ghani and M.~Schwartz, ``A decomposition approximation for the analysis of
  voice/data integration,'' \emph{IEEE Transactions on Communications},
  vol.~42, no.~7, pp. 2441--2452, 1994.

\bibitem{li1985perf}
S.-Q. Li and J.~W. Mark, ``Performance of voice/data integration on a {TDM}
  system,'' \emph{IEEE Transactions on Communications}, vol.~33, no.~12, pp.
  1265--1273, 1985.

\bibitem{mag82}
B.~Maglaris and M.~Schwartz, ``Optimal fixed frame multiplexing in integrated
  line-and packet-switched communication networks,'' \emph{IEEE Trans. Inf.
  Th.}, vol.~28, no.~2, pp. 263--273, 1982.

\bibitem{mankus92}
M.~L. Mankus and C.~Tier, ``Asymptotic analysis of an integrated digital
  network,'' \emph{SIAM J. Appl. Math.}, vol.~52, no.~1, pp. 234--269, 1992.

\bibitem{wein80}
C.~Weinstein, M.~Malpass, and M.~Fisher, ``Data traffic performance of
  integrated circuit-and packet-switched multiplex structure,'' \emph{IEEE
  Trans. Commun.}, vol.~28, no.~6, pp. 873--878, 1980.

\bibitem{MRM1008}
M.~McGarry, M.~Reisslein, and M.~Maier, ``Ethernet passive optical network
  architectures and dynamic bandwidth allocation algorithms,'' \emph{IEEE
  Communication Surveys and Tutorials}, vol.~10, no.~3, pp. 46--60, Oct. 2008.

\bibitem{McGR06}
------, ``{WDM} ethernet passive optical networks,'' \emph{IEEE Commun. Mag.},
  vol.~44, no.~2, pp. 15--22, 2006.

\bibitem{Ross95}
K.~W. Ross, \emph{Multiservice Loss Models for Broadband Telecommunication
  Networks}.\hskip 1em plus 0.5em minus 0.4em\relax Springer, 1995.

\bibitem{BSA06}
X.~Bai, C.~Assi, and A.~Shami, ``{On the fairness of dynamic bandwidth
  allocation schemes in Ethernet passive optical networks},'' \emph{Comp.
  Commun.}, vol.~29, no.~11, pp. 2125--2135, Jul. 2006.

\bibitem{AYDA1103}
C.~Assi, Y.~Ye, S.~Dixit, and M.~Ali, ``Dynamic bandwidth allocation for
  {Quality-of-Service} over {Ethernet PONs},'' \emph{IEEE Journal on Selected
  Areas in Communications}, vol.~21, no.~9, pp. 1467--1477, Nov. 2003.

\bibitem{KrMD2002}
G.~Kramer, B.~Mukherjee, S.~Dixit, Y.~Ye, and R.~Hirth, ``Supporting
  differentiated classes of service in {Ethernet} passive optical networks,''
  \emph{OSA J. of Optical Netw.}, vol.~1, no.~9, pp. 280--298, Aug. 2002.

\bibitem{HaSM06}
M.~Hajduczenia, H.~J.~A. da~Silva, and P.~P. Monteiro, ``{EPON} versus {APON}
  and {GPON}: a detailed performance comparison,'' \emph{OSA Journal of Optical
  Networking}, vol.~5, no.~4, pp. 298--319, Apr. 2006.

\bibitem{NaM06}
H.~Naser and H.~Mouftah, ``A fast class-of-service oriented packet scheduling
  scheme for {EPON} access networks,'' \emph{IEEE Communications Letters},
  vol.~10, no.~5, pp. 396--398, May 2006.

\bibitem{dyck90}
H.~Dyckhoff, ``A typology of cutting and packing problems,'' \emph{Eu. J. Op.
  Res.}, vol.~44, no.~2, pp. 145--159, 1990.

\bibitem{tham83}
Y.~Tham and J.~Hume, ``Analysis of voice and low-priority data traffic by means
  of brisk periods and slack periods,'' \emph{Comp. Commun.}, vol.~6, no.~1,
  pp. 14--22, 1983.

\bibitem{Kleinrock75}
L.~Kleinrock, \emph{Queueing systems. volume 1: Theory}.\hskip 1em plus 0.5em
  minus 0.4em\relax Wiley, 1975.

\bibitem{AuLMR14}
F.~Aurzada, M.~Levesque, M.~Maier, and M.~Reisslein, ``{FiWi} access networks
  based on next-generation {PON} and gigabit-class {WLAN} technologies: A
  capacity and delay analysis,'' \emph{IEEE/ACM Trans. Netw., in print}, 2014.

\bibitem{Coim13}
J.~Coimbra, G.~Schütz, and N.~Correia, ``A game-based algorithm for fair
  bandwidth allocation in {Fibre-Wireless} access networks,'' \emph{Optical
  Switching and Networking}, vol.~10, no.~2, pp. 149 -- 162, 2013.

\bibitem{DhHJ11}
A.~Dhaini, P.-H. Ho, and X.~Jiang, ``{QoS} control for guaranteed service
  bundles over fiber-wireless {(FiWi)} broadband access networks,''
  \emph{IEEE/OSA J. Lightwave Techn.}, vol.~29, no.~10, pp. 1500--1513, 2011.

\bibitem{LiKK13}
W.~Lim, K.~Kanonakis, P.~Kourtessis, M.~Milosavljevic, I.~Tomkos, and J.~M.
  Senior, ``Flexible {QoS} differentiation in converged {OFDMA-PON} and {LTE}
  networks,'' in \emph{Proc. OFC}, 2013.

\bibitem{MaGR09}
M.~Maier, N.~Ghazisaidi, and M.~Reisslein, ``The audacity of fiber-wireless
  {(FiWi)} networks,'' in \emph{Proc. of AccessNets}, ser. Lecture Notes of the
  Institute for Computer Sciences, Social Informatics and Telecommunications
  Engineering.\hskip 1em plus 0.5em minus 0.4em\relax Springer, 2009, vol.~6,
  pp. 16--35.

\bibitem{Morad13}
N.~Moradpoor, G.~Parr, S.~McClean, and B.~Scotney, ``{IIDWBA} algorithm for
  integrated hybrid {PON} with wireless technologies for next generation
  broadband access networks,'' \emph{Opt. Switching. Netw.}, vol.~10, no.~4,
  pp. 439--457, 2013.

\bibitem{HoH11}
M.~Hossen and M.~Hanawa, ``Network architecture and performance analysis of
  {MULTI-OLT PON} for {FTTH} and wireless sensor networks,'' \emph{Int. J.
  Wireless \& Mobile Networks}, vol.~3, no.~6, pp. 1--15, 2011.

\bibitem{SeR11}
A.~Seema and M.~Reisslein, ``Towards efficient wireless video sensor networks:
  A survey of existing node architectures and proposal for a {Flexi-WVSNP}
  design,'' \emph{IEEE Comm. Surv. \& Tut.}, vol.~13, no.~3, pp. 462--486,
  Third Quarter 2011.

\bibitem{YuZD12}
X.~Yu, Y.~Zhao, L.~Deng, X.~Pang, and I.~Monroy, ``Existing {PON}
  infrastructure supported hybrid fiber-wireless sensor networks,'' in
  \emph{Proc. OFC}, 2012, pp. 1--3.

\bibitem{Mou05}
B.~Kantarci and H.~Mouftah, ``Bandwidth distribution solutions for performance
  enhancement in long-reach passive optical networks,'' \emph{IEEE Commun.
  Surv. Tut.}, vol.~14, no.~3, pp. 714--733, Aug. 2012.

\bibitem{MeMR13}
A.~Mercian, M.~McGarry, and M.~Reisslein, ``Offline and online multi-thread
  polling in long-reach {PONs}: A critical evaluation,'' \emph{IEEE/OSA J.
  Lightwave Techn.}, vol.~31, no.~12, pp. 2018--2228, Jun. 2013.

\bibitem{SKM0110}
H.~Song, B.~W. Kim, and B.~Mukherjee, ``Long-reach optical access networks: A
  survey of research challenges, demonstrations, and bandwidth assignment
  mechanisms,'' \emph{IEEE Commun. Surv. Tut.}, vol.~12, no.~1, pp. 112--123,
  1st Quarter 2010.

\bibitem{BiBC13}
A.~Bianco, T.~Bonald, D.~Cuda, and R.-M. Indre, ``Cost, power consumption and
  performance evaluation of metro networks,'' \emph{IEEE/OSA J. Opt.\ Comm.
  Netw.}, vol.~5, no.~1, pp. 81--91, Jan. 2013.

\bibitem{MaRe04}
M.~Maier and M.~Reisslein, ``{AWG}-based metro {WDM} networking,'' \emph{IEEE
  Commun. Mag.}, vol.~42, no.~11, pp. S19--S26, Nov. 2004.

\bibitem{MaRW03}
M.~Maier, M.~Reisslein, and A.~Wolisz, ``A hybrid {MAC} protocol for a metro
  {WDM} network using multiple free spectral ranges of an arrayed-waveguide
  grating,'' \emph{Computer Networks}, vol.~41, no.~4, pp. 407--433, Mar. 2003.

\bibitem{ScMRW03}
M.~Scheutzow, M.~Maier, M.~Reisslein, and A.~Wolisz, ``Wavelength reuse for
  efficient packet-switched transport in an {AWG}-based metro {WDM} network,''
  \emph{IEEE/OSA Journal of Lightwave Technology}, vol.~21, no.~6, pp.
  1435--1455, Jun. 2003.

\bibitem{YaMRC03}
H.-S. Yang, M.~Maier, M.~Reisslein, and M.~Carlyle, ``A genetic algorithm-based
  methodology for optimizing multiservice convergence in a metro {WDM}
  network,'' \emph{IEEE/OSA Journal of Lightwave Technology}, vol.~21, no.~5,
  pp. 1114--1133, May 2003.

\bibitem{YuCL10}
M.~Yuang, I.-F. Chao, and B.~Lo, ``{HOPSMAN}: An experimental optical
  packet-switched metro {WDM} ring network with high-performance medium access
  control,'' \emph{IEEE/OSA Journal of Optical Communications and Networking},
  vol.~2, no.~2, pp. 91--101, Feb. 2010.

\end{thebibliography}


\end{document}
