

Now, we study upper bounds for  \VMA, \CVMA, and -VMA problems. We start giving an online VMA algorithm that can be used in \VMA and \CVMA problems. The algorithm uses the load of the new revealed VM in order to decide the PM
where it will be assigned.
If the load of the revealed VM is strictly larger than , the algorithm assigns this VM to a new PM without any other VM already assigned to it. Otherwise, the algorithm schedules the revealed VM to any loaded PM whose current load is smaller or equal than . Hence, when this new VM is assigned, the load of this PM remains smaller than .
If there is no such loaded PM, the revealed VM is assigned to a new PM.
Note that, since the case under consideration assumes the existence of an unbounded number of PMs, there exists always one new PM. A detailed description of this algorithm is shown in Algorithm \ref{alg:inftym}. As before,  denotes the set of VMs assigned to PM  at a given time.
\begin{algorithm}

\For{ each VM }{
\eIf{}
{ is assigned to a new PM\:}
{ is assigned to any loaded PM  where . If such loaded PM does not exist,  is assigned to a new PM\:}

}
\caption{Online algorithm for \VMA and \CVMA problems.}
\label{alg:inftym}
\end{algorithm}

We prove the approximation ratio of Algorithm \ref{alg:inftym} in the following two theorems.
\begin{theorem}
\label{theo:UBunboundedPMs}
There exists an online algorithm for \VMA and \CVMA when  that achieves the following competitive ratio: 

\end{theorem}

\begin{proof}
We proceed with the analysis of the competitive ratio of Algorithm~\ref{alg:inftym} shown above.
Let us first consider an optimal algorithm, that is, an algorithm that gives an optimal solution for any instance.
Let us denote by  the optimal solution obtained by the optimal algorithm, and  the load assigned to PM  in that solution, for a particular instance of VMA problem.
Furthermore, load  is decomposed in , where each  is a VM that  assigns to .
Using simple algebra, it holds:

It is possible now to split the set  in two sets, one with those VMs assigned to  whose load is strictly smaller than  and a second set that contains those VMs assigned to  whose load is bigger than .
In terms of notation, we say that  is split in  and  (where  stands for Big loads and  stands for Small loads).
Therefore, it also holds:


On the other hand, by definition of , it holds that:  for all  (indeed, for any load).
Moreover, if a PM has been assigned with a load  bigger than , it also holds that .
Hence, we obtain the following inequality:


In order to lower bound the power consumption of the solution , we plug the above inequality into the corresponding equation:

or, equivalently expressed in more compact notation:


Consider now Algorithm  \ref{alg:inftym}. Let us denote by  a solution that Algorithm  \ref{alg:inftym} gives for a particular instance.
Also, let us denote by  the load assigned by Algorithm \ref{alg:inftym} to PM .
 Note that due to the design of the algorithm,
after the last VM has been assigned, either there is only one loaded PM whose current load is smaller than ,
or every loaded PM has a load at least . We study these two cases separately.  \\
\emph{Case 1:}  for all .
In this case, in a solution provided by  there are PMs with two types of load:
those that are loaded with one VM whose load is no smaller than ,
and those that are loaded with VMs whose load is strictly smaller than , nonetheless, their total load is bigger than .
Note that due to the design of the algorithm, none of the PMs in the second group has a load bigger than .
Let us denote by  the set of VMs with load at least ,
and  the set of VMs with load less than .
Therefore, it holds:

{Computing the ratio  between  and , we obtain the following inequality:

}
\emph{Case 2:} there exists  such that .
In this case,  gives solutions with three types of loaded PMs:
those that are loaded with one VM whose load is bigger than ,
those that are loaded with VMs whose load is strictly smaller than , but which total load is at least ,
and one PM whose total load is is strictly smaller than .
Let us denote such a PM by .
Therefore, it holds:

Let us denote the latter expression by .
Computing the ratio  between  and , we obtain the following inequality:

Since  is always positive, the competitive ratio of Algorithm \ref{alg:inftym} is equal to .
Observe that, when no VM
 has load , i,e., ,
 and  are equal. Hence, the competitive ratio is .
\qed
\end{proof}


\begin{theorem}
There exists an online algorithm for  \CVMA when  that achieves competitive ratio .
\end{theorem}
\begin{proof}
We proceed with the analysis of the competitive ratio of Algorithm~\ref{alg:inftym} in the case when .
The analysis uses the same technique used in the proof for the previous theorem. Hence, we just show the difference.

On the one hand, when , it holds that  due to the fact that  is monotone decreasing in interval . It is also obvious that all the PMs will be loaded no more . As a result, the optimal power consumption for \CVMA can be bounded by


On the other hand, the solution given by Algorithm~\ref{alg:inftym} can also be upper bounded. 
We consider the following two cases. \\
\emph{Case 1:}  for all .
In this case, every PM will be loaded between  and . Consequently,

The competitive ratio  then satisfies

\emph{Case 2:}  there exists  such that . In this case, it holds: 

The competitive ratio  then satisfies

\qed
\end{proof}



\subsubsection{Upper Bounds for -VMA problem} We now present an algorithm (detailed in Algorithm~\ref{m2alg}) for -VMA problem and show an upper bound on its competitive ratio.
 and  are the sets of VMs
assigned to PMs  and , respectively, at any given time.

\begin{algorithm}
\For{ each VM }{
\eIf{  {\bf or} }
{ is assigned to \;}
{ is assigned to \;}
}
\caption{Online algorithm for -VMA.}
\label{m2alg}
\end{algorithm}

We prove the approximation ratio of Algorithm~\ref{m2alg} in the following theorem.

\begin{theorem}
\label{theo:UB2PMs}
There exists an online algorithm for -VMA that achieves the following competitive ratios.

\end{theorem}

\begin{proof}
Consider Algorithm~\ref{m2alg} shown above.
If , then the competitive ratio is  as we show. Algorithm~\ref{m2alg} assigns all the VMs to PM . On the other hand, the optimal offline algorithm also assigns all the VMs to one PM. To prove it, it is enough to show that
. Using that  and manipulating, it is enough to prove
.
This is true for .



We consider now the case . Within this range, for the optimal algorithm is still better to assign all VMs to one PM, as shown. Then, the competitive ratio  is




Consider any given step after .
Within this range, the optimal algorithm may assign the VMs to one or both PMs.
If the optimal algorithm assigns to one PM, Inequality~\ref{ratio2} applies.
Otherwise, the competitive ratio  is

Then, in order to obtain a ratio at most , where  will be set later, it is enough to guarantee

Without loss of generality, assume . This implies that .
Then, it is enough to have

Let us now define  for some . Manipulating and replacing, it is enough to show



If Inequality~\ref{eq:condratio} holds the theorem is proved. Otherwise, the following claim is needed.
\begin{claim}
\label{claim:max}
If , then there must exist a VM  in  such that .
\end{claim}
\begin{proof}
If  the claim follows trivially. Assume that .
Consider any given time when .
For the sake of contradiction, assume that for all  it is .
Let  be the order in which the VMs were revealed to Algorithm~\ref{m2alg}.
And let the respective sets of VMs be called , that is .
Given that , the VM  was assigned to the PM with smaller load. Then, either  which would be a contradiction, or if  the
PM with the smaller load before and after assigning  is the same.
The argument can be repeated iteratively backwards for each , , etc. until, for some , either it is  reaching a contradiction, or the total load is .
If the latter is the case, we know that for  every  was assigned to .
Recall that for  each  was assigned to the same PM. And, given that  is the first VM for which the total load is at least , that PM is .
But then, we have , which is a contradiction with the assumption that .
\end{proof}

Using Claim~\ref{claim:max} we know that there exists a  in the input such that

From the latter, it can be seen that if , then we have that .
Then, the competitive ratio  is

Using calculus, this ratio is maximized for  for . Then, we have
.
Then, in order to obtain a ratio at most , it is enough to guarantee

which yields
.

Given that, for any , it holds:

Then, the competitive ratio is
.
\qed
\end{proof}
