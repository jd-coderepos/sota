





\section[<toc-entry>]{The \idp Language}
\label{sec:language}

 The \idp language is the input language of the \idp
  system. A program in the \idp language consists of declarative
  and imperative components. The declarative components are \emph{vocabulary}, \emph{structure},
  \emph{theory} and \emph{term} components. Together, they provide a concrete computer-readable syntax for \foidaggpft. The imperative components
  allow one to perform computational tasks. They consists of
  \emph{procedures}. Each procedure embeds a piece of imperative
  Lua~\cite{SPE/IerusalimschyFC96} code; besides performing standard
  imperative operations, procedures can apply inference methods upon
  \foidaggpft theories encoded in the declarative components. 





Vocabulary, structure and theory components are described in
Section~\ref{sec:logic}; procedure components in
Section~\ref{sec:procedure} and term components in
Section~\ref{sec:term}. In this section we do not strive for
completeness but focus on what is needed to get started using the
\idp system and on providing answers to the difficulties a starting
user might have.






Before describing the different kinds of components, we first discuss a
few general notational conventions. Names are everywhere, they are
used for types, predicates, functions (including constants), and
variables as well as for domain elements in structures. To distinguish
them from numbers, they start with a Latin letter (upper or lower
case) and consist of a sequence of Latin letters and digits; also a
few special characters such as ``\il{\_}'' are allowed. For domain
elements, one can deviate from this convention by using a string
notation.  For details, we refer to the manual \cite{url:idp3manual}.  Comments that fit
on a single line start with ``\il{//}''. One can start longer
comments with ``\il{/*}'' and end them with ``\il{*/}''.

\subsection{The Logic}\label{sec:logic}

\subsubsection{Vocabulary}
The vocabulary of an \foidaggpft theory is represented as a vocabulary
component. We start with an example.

\begin{lstlisting}
vocabulary courses {
  type course	
  type person
  type student isa person
  type instructor isa person
  type age isa nat
  takes(student, course)
  hasAge(person):age
  Vacation
  Boss: person
}
\end{lstlisting}

A vocabulary declaration takes the form ``\il{vocabulary} \il{vocname} \{  typed symbol list\}''. It specifies
the name of the vocabulary, here \il{courses} and its symbols. The
symbol list comprises type symbols, and typed predicate and function symbols.  Each symbol is to be declared on a new
line. Types are declared using the keyword ``\il{type}''. 
A type may be declared as a subtype using  the keyword
``\il{isa}'' followed by a comma-separated list of supertypes.  For example 
``\il{type} A \il{isa} B, C'' declares  as a direct subtype of  and . The declared \il{isa} graph needs to be  acyclic.
The integers, type \il{int}, and
its subtype, the natural numbers, type \il{nat}, are part of every
vocabulary and need not be declared. The same holds for predicates and
functions that are part of every \foidaggpft theory such as comparison
predicates and arithmetic functions.  Predicates and functions are
introduced by declaring their signature. In the example, \il{takes} is a relation over \il{student}
and \il{course} and \il{hasAge} is a function from \il{person} to
the subtype \il{age},  a subtype of nat. The symbol \il{Vacation} is a propositional symbol, and \il{Boss}  a constant symbol. Partial
functions are introduced by the keyword \il{partial}, for example
\il{partial hasAge(person):age} would declare \il{hasAge} as a partial
function. The \idp system cannot yet cope well with infinite types, so
\il{int} and \il{nat} can better be avoided in signatures of
predicates and functions.

\subsubsection{Structure}
A structure component describes a partial structure for a vocabulary,
in particular the domains of the user declared types. We start again
with an example.

\begin{lstlisting}
structure data1: courses{
  course = {Logic; Math}
  student = {John; Bob; Alice}
  instructor = {Marc; Gerda; Maurice}
  person = {John; Bob; Alice; Marc; Gerda; Maurice}
  age = {1 .. 65}
  takes<ct> = {John,Logic}
  takes<cf> = {Bob,Math}
  hasAge<ct>= {John->25; Bob -> 30; Alice->19}
  Vacation=true
  Boss=Alice
}


\end{lstlisting}
A structure has a name, here \il{data1}, and specifies the  vocabulary that it (partially) interprets, here \il{courses}. It specifies an assignment of values for symbols using  a list of ``symbol=value'' equations. Boolean values are denoted \il{true} or  \il{false}, as illustrated by the equation \il{Vacation=true} in the example. Other base values are numbers, strings, or user-defined domain values like \il{Bob, Math}.  Set values are denoted as ``\{ semicolon separated list \}''. The list may consist of individual entities or of  tuples of individual entities. Tuples are denoted as comma separated lists of domain values, potentially between parentheses "( \dots )". The shorthand ``n..m''  enumerates the integer interval ,  as illustrated  for \il{age}. The same holds for characters, e.g., \il{student = \{a..d\}} is a shorthand for \il{student = \{a;b;c;d\}}.



A structure specifies, implicitly or explicitly, the value of each
type of the vocabulary\footnote{Autocompletion may derive missing type
  domains; e.g., in the absence of a domain for age, autocompletion
  will derive \il{\{19, 25, 30\}} for it, in absence of a domain for
  person, it will derive the union of the \il{student} and
  \il{instructor} domains.}. A value of a type is a set of domain
elements. User-defined domain elements are identified by symbolic identifiers
(e.g., Bob, Math) but these identifiers are not symbols of the
vocabulary and cannot be used in the
theory.  One can use integers as names for the domain elements of types, even
if this type is not a subtype of \il{int}. For example \il{student =
  \{1..50\}} introduces 50 students.  Note, these domain elements are
not integers, the domain element 49 of type student is different from
domain element 49 of type integer.

A (total) value for a predicate symbol is a set of domain elements or tuples of it. Alternatively, a structure may  specify a partial value for a predicate symbol \il{P}, as an assignment of a list of certainly true, certainly false and unknown tuples to respectively \il{P<ct>}, \il{P<cf>}, and \il{P<u>}. Only two out of   three need to be specified. This is illustrated by \il{takes}.

The value or partial value for functions is specified in an analogous way, with the difference that the user is allowed (but not obliged) to specify a tuple ``{(a1,\dots,an,b)}'' of a function in the form ``\il{a1,.., an -> b}''. For partial interpretations of functions, we refer to the discussion on partial
structures in Section~\ref{sec:fo}. For example, \il{hasAge<cf> = \{Marc -> 1; Marc -> 2\}} can be used to express that the age of Marc is neither 1 nor 2.

In \idp one cannot currently use a constant as a bound in a domain
enumeration, as in  \il{student = \{1..nbstudents\}}, where \il{nbstudents} is a constant symbol whose value is specified elsewhere. Another limitation that was already mentioned earlier, is that domain values such as \il{Bob} and \il{Alice}  (identifiers introduced at the right side of ``symbol=value'' equations), are not part of the vocabulary and cannot appear in theories. 

Recall that there exist also many interpreted symbols (e.g., numerical operators and aggregates) whose values are fixed and are implicit parts of all structures. 

\subsubsection{Theory} A theory  component over some vocabulary is  declared  as ``\il{theory} theory name : vocname \{ \dots \}''. For the syntax of formulas and definitions, that of
the formal base language is followed as closely as possible. Formulas
and rules are terminated with a ``\il{.}'' and rules are grouped in
definitions which are put between ``\il{define \{}'' and ``\il{\}}''.
The following table provides a translation in a more keyboard
friendly notation\footnote{The IDE at \url{http://dtai.cs.kuleuven.be/krr/idp-ide/} visualizes the symbols in the
  syntax of \foidaggpft.}.









By way of example, we show a vocabulary, a structure and a theory for
a small graph problem that formalizes a connected graph over a set of
nodes. 


\begin{lstlisting}
vocabulary V{
  type Node
  Forbidden(Node,Node)
  Edge(Node,Node)
  Reachable(Node)
  Root:Node
}

structure S:V{
  Node = {A..D}
  Forbidden = {A,A; A,B; A,C; B,A; B,B; B,C; C,C; C,D; D,D}
  Root = A
}

theory T: V{
  // inductive definition of Reachable
  define {
    Reachable(Root).
    !x [Node]: Reachable(x) <- 
               ?y [Node]: Reachable(y) & Edge(y,x).
  }

  // The graph is fully connected
  !x [Node] : Reachable(x).

  // No forbidden edges
  !x[Node] y[Node] : Edge(x,y) => ~Forbidden(x,y).
}

\end{lstlisting}
The theory contains a definition\footnote{The \il{define} keyword is optional; it emphasizes that the brackets  and  are delimiters of a definition.}   of the \il{Reachable} predicate and
two formulas that constrain the solution. All variables are typed with
type \il{Node}; however, these types can be omitted since
\emph{type inference} will derive them from the signatures of the symbols  in which the variables occur.\footnote{If within the same scope a variable appears in argument positions with different types, the inferred type is their least supertype if it exists, otherwise a type error is raised, as explained in
Section~\ref{sec:preliminaries:types}. If no type  can be inferred, e.g., as in \il{!x: x=x}, also then an error is raised.} Observe that this theory  defines \il{Reachable} as the transitive closure of  parameter \il{Edge}. This definition is syntactically similar to a Prolog program, but unlike  a Prolog program, it is here the defined predicate that is known (it is the set of all nodes) and the parameter predicate that is unknown. It illustrates the declarative understanding of  a definition that  expresses a particular logical relationship between the parameters and the defined symbols, and not a way to compute the defined symbols in terms of the parameters. 

\ignore{\maurice{The motivation to distinguish between \foidaggpft and the
  logic components of the \idp language is that the latter
  incorporates a number of design decisions. One such decision is to
  maintain a clear separation between structure and theory. This is
  sometimes inconvenient, and could be an argument to merge theory and
  structure in a single component; then one would need a mechanism to
  distinguish between constants and domain elements. Such alternative
  design decision would result in a different \idp language, while
  preserving the underlying formal logic. } \marc{Het is waar dat we die separation hebben, en dat dit wellicht geen goede designbeslissing was. Maar of dat nu de reden is voor het verschil tussen de theoretische en de concrete taal, dat berijp ik niet. Ik zou liever niet zoveel aandacht willen besteden aan onze slechte architecturale keuzes. Daarom is mijn voorstel dit te schrappen.
}}


\subsection{Procedure}\label{sec:procedure}

A procedure component is a chunk of Lua code \cite{SPE/IerusalimschyFC96} 
encapsulated in the form of an \idp component (a keyword
\il{procedure}, a name, a list of parameters and the chunk of code
between ``\il{\{}'' and ``\il{\}}''). When the IDP system is run, it
calls the procedure \il{main()}. Typically, one will use the \idp system
to do some reasoning on an \foidaggpft theory. Here is a simple example:

\begin{lstlisting}
procedure main(){
	stdoptions.nbmodels = 0
	printmodels(modelexpand(T,S)) 
}   
\end{lstlisting}

The first line, \il{stdoptions.nbmodels = 0}, configures the \idp
system to compute all models (with a positive number , inference is
stopped after  models have been found; default value is 1). The
second line calls \idp's modelexpand procedure with as input arguments the theory  and the structure  and prints these models
(\il{printmodels}). The \il{modelexpand} procedure calls upon the
solver of the \idp system to search for models of  that expand the
input structure  and returns an array of models. To print a single
model, one can select a model and print it,
e.g. \il{print(modelexpand(T,S)[3])} to print the third model. With
the theory and structure as given in the small graph theory of the
previous section, the above procedure prints 24 models, the first one
being:
\begin{lstlisting}
structure  : V {
  Node = { "A"; "B"; "C"; "D" }
  Edge = { "A","D"; "B","D"; "C","A"; "C","B"; "D","A"; "D","B"; "D","C" }
  Forbidden = { "A","A"; "A","B"; "A","C"; "B","A"; "B","B"; "B","C"; "C","C"; "C","D"; "D","D" }
  Reachable = { "A"; "B"; "C"; "D" }
  Root = "A"
}
\end{lstlisting}
Note that the model is written in the syntactic format of  a structure component. The procedures 
\il{modelexpand} and \il{printmodels} are only  two out of many
predefined procedures in \idp. Many procedures provide other forms of inference that can perform
computational tasks  using the knowledge represented by an
\foidaggpft theory. Other procedures serve  to manipulate and create new logical
components, such as vocabularies, structures and theories. We refer to
the online \idp Web-IDE
(\url{http://dtai.cs.kuleuven.be/krr/idp-ide/}) and the \idp
manual
(\url{https://dtai.cs.kuleuven.be/krr/files/bib/manuals/idp3-manual.pdf})
for examples and details.  The main design philosophy is that all
components in an \idp program are first class citizens. They can be used
to perform various reasoning tasks but can also be manipulated by Lua code to
construct different ones. So, it is possible to set up a complete
workflow. 

This methodology in which fine-grained declarative computation steps are mixed in procedures is an exciting novel way of integrating declarative and procedural knowledge. 
This is illustrated by Bruynooghe et al.~\cite{TPLP/BruynoogheBBDDJLRDV}. This is an application in stemmatology, the study of the family
relationships between different manuscripts (hand made copies) of a
text. It sketches a set of procedures that describe the workflow to
analyse a number of texts. For each text, a data set is read and
analyzed and transformed into structures and vocabularies. These are
then combined with the problem vocabulary and theory and, for each so
called feature in the input data of a text, it is checked whether a
model exists. Finally, for each data set, a summary report about all
its features is reported.

\subsection{Term}\label{sec:term}
One of the available forms of inference in \idp is to solve minimization problems. This inference method  takes as input a theory, a partial structure, and a cost term, and outputs one or more models of the theory expanding the partial structure such that the value of the cost term in these structures is minimal (among the set of all models more precise than the partial structure).  The cost term is to be specified as a separate term
component over the same vocabulary as the theory. To
illustrate, we return to our graph problem and introduce a term to
count the number of edges in the solution. 
\begin{lstlisting}
term t: V{
  #{x y: Edge(x,y)}
}

procedure main(){
  stdoptions.nbmodels = 0
  models, optimal, cost = minimize(T,S,t)
  printmodels(models)
  print(optimal)
  print(cost)
}   
\end{lstlisting}
The \il{main} procedure now calls the minimization inference with the
theory \lstinline{T}, the structure \lstinline{S} and the optimization term \lstinline{t} as
input\ignore{\footnote{Currently, only aggregates are supported; however, the
  manual mentions another form for aggregates, one can write
  \il{sum(f(...))} for any term \il{f}. MARC: IK BEGREEP DEZE FOOTNOTE NIET EN HET LEEK ME OOK NIET RELEVANT. }}. With \lstinline{T} and \lstinline{S} as above,
the procedure returns three models, however, it also returns two other
values: whether optimality has been proven and the value of the term
in the optimal solution. The assignment \il{models, optimal, cost=...}
assigns them to different variables. The first value is an array of
models, which can be printed with \il{printmodels}, the other two are
simple values; they can be printed with the standard \il{print}
command. In this example, optimality is reached with a cost of 3.

\section{Advanced Features}\label{sec:advanced}

\subsection{Constructed Types}
In Prolog and ASP, functions and constants have a fixed
interpretation, their Herbrand interpretation. Equivalently, we can
think of them as logics with built-in \emph{unique names} and
\emph{domain closure} axioms (UNA and DCA), i.e., axioms stating that all those values are different and that the domain consists of nothing more than those values, respectively. In \foidaggpft, the axioms
are not present and interpretation of functions and constants is open
as in standard \FO. Both approaches  have their merits, making it useful to integrate the advantages of both. In ASP, there is work to
incorporate open functions \cite{kr/BartholomewL12,kr/Lifschitz12}. In
\foidaggpft, one way to impose UNA and DCA  is to explicitly specify a Herbrand interpretation. This method is illustrated below for the type of  days of the week, using the following declarations in the vocabulary:
\begin{lstlisting}
  Type Day
  monday: Day
  tuesday: Day
  ...
\end{lstlisting}
and in the structure
\begin{lstlisting}
  Day = {''monday''; ''tuesday'' ; ...}
  monday = ''monday''
  tuesday = ''tuesday''
  ...
\end{lstlisting}
There is a way to avoid this cumbersome approach.
The following constructed type declaration in the vocabulary expresses the same information but in a compact way:
\begin{lstlisting}
type Day constructed from {monday, tuesday, wednesday, thursday, friday}
\end{lstlisting}
This statement declares several things at once: it declares the type
\il{Day} and seven constants of this type, it specifies the values for
this type and for all its constants in every structure of the
vocabulary.  Each constant is interpreted by itself.  Here,
constants and domain elements coincide.

Also non-constant constructor  symbols are  supported this way. For example, having types \il{row} and \il{column}, one can
introduce the constructed type of positions on a chessboard by declaring ``\il{type position constructed from {pos(row,col)}}'', and use it in the definition of a unary predicate \il{queen(position)} representing the positions where  a queen stands. In theory, this approach also
works for recursive constructors and types such as list of integers:
\il{type list constructed from \{nil ; cons[int, list]\}}. However,
this creates an infinite type and the current \idp solver cannot cope
with such types.
















\subsection{Structuring Components}

All components, vocabularies, theories, structures, terms, and
procedures, as we have shown so far, are part of the implicit global
namespace \il{idpglobal}. This namespace also contains all Lua
procedures that are available to the user of the system. When working
on large projects, different people may work on different parts, each
introducing its own components. To integrate such different parts, the
\idp system provides \emph{namespaces}.  A namespace with name
MySpace is declared by
\begin{lstlisting}
 namespace MySpace {
//   content   of   the   namespace
}
\end{lstlisting}
A namespace can contain other namespaces, and any sort of \idp component including vocabularies, theories,
structures, terms, and procedures. Each component has a full name that is determined by the
hierarchy of namespaces it belongs to. This allows users to
disambiguate components with the same name but belonging to different
namespaces. We refer to the manual for details.

Another useful structuring method is to compose a vocabulary from existing ones. For instance, in the following example \il{W} is composed of the symbols of \il{V} and one function \il{coloring/1:1} from vocabulary \il{U}:\footnote{Here, the notation \il{coloring/1:1} means that \il{coloring} has arity one (\il{/1}) and is a function, i.e., has one output argument (\il{:1}).} 
\begin{lstlisting}
vocabulary W {
extern  vocabulary V
extern  U::coloring/1:1
}
\end{lstlisting}
Such an extension construct is not available for structures and
theories but it could be simulated for them by making use of certain
Lua procedures. For this we refer to the list of Lua procedures
described in the manual. 

Worth mentioning is that there is a
\il{factlist} component to initialize a two valued structure with
Prolog or ASP facts. Also, it is possible to call upon Lua procedures
to initialize a structure. 

\subsection{An Output Vocabulary} 

In many problems we are interested only in the values of some subset of symbols. In case multiple solutions are searched, we are interested only in  models having different interpretations of the output symbols. This is achieved by declaring an output vocabulary, say \il{Vout}, and adding it as an extra parameter to the  modelexpand call: 
\begin{lstlisting}
procedure main(){
    print(modelexpand(T,S,Vout)[1])
}
\end{lstlisting}

\subsection{Inference Methods}

So far we have mentioned \emph{model expansion} inference, invoked as
\il{modelexpand(T,S)} (or \il{modelexpand(T,S,Vout)} if there is an output vocabulary) and \emph{optimization} inference, invoked as
\il{minimize(T,S,t)} (or \il{minimize(T,S,t,Vout)}). The system supports several other inference
methods. We discuss the most important ones. For a complete list,
we refer to the manual. 

\textbf{Query inference} takes as input a \emph{query} component that declares a
set expression of the form  and a two-valued structure
\struct and returns the set . In the \idp language, this inference is invoked as \il{query(Q,S)}, where \il{Q} is a query and  \il{S} a structure. 
Continuing our
graph example,

\begin{lstlisting}
query Q:V{
    {x: (?y: Edge(x,y))}
}

procedure main(){
   models =modelexpand(T,S) 
   print(query(Q,models[1]))
}  
\end{lstlisting}
will print the set of all nodes that participate as the first node of an
edge in the first model computed by the model expansion. Note that
\il{query(...)} does not return a structure but a set.


\textbf{Model checking} and \textbf{satisfiability checking} are special
cases of model expansion.  In the former, the input structure is two-valued; the result of this inference is \emph{true} respectively \emph{false} if the input structure is a model of the theory. The latter also outputs a Boolean value, \emph{true} if the (possibly three-valued) output structure \emph{can be expanded} to a model. 
In the \idp language, both of these inference methods are called using \il{sat(T,S)}, where \il{T} is a theory and \il{S} a structure.\footnote{Note that satisfiability checking reduces to model checking in case \il{S} is two-valued.}

\textbf{Propagation} inference takes a theory and a structure and
returns a more precise structure that preserves all solutions. The
system supports different versions of propagation with different
costs. The most precise and most expensive version returns the partial
structure in which atoms are unknown iff they do not have the same
truth value in all models. This most expensive propagation is called using \il{optimalpropagate(T,S)}. 
Cheaper, approximate forms of propagation are called using \il{propagate(T,S)} and \il{groundpropagate(T,S)}.

\textbf{Deduction} takes as input an \foidaggpft theory \theory and an
FO theory  and returns true if , that is if the first theory logically entails the
second one. It is implemented in a sound but incomplete way by
translating \theory into a (weaker) \FO theory and calling the theorem
prover SPASS~\cite{cade/WeidenbachDFKSW09}. It is used internally in
the \idp system to detect and exploit functional dependencies in
predicates \cite{iclp/DeCatB13}. It is called using \il{entails(T1,T2)}.

\textbf{Symmetry detection} takes as input a theory \theory and a
partial structure \I and returns \emph{symmetries} over \theory and
\I. A symmetry is a function, say , mapping structures to
structures, such that, for any two-valued expansion  of \I that is
a model of \theory,  is also a model
~\cite{ictai/DevriendtBMDD12}. Symmetry detection also returns clauses
to break these symmetries and to eliminate symmetric models. 
Symmetry detection is not available as a Lua procedure but can be exploited in the model expansion workflow using the option \il{symmetrybreaking} (see Section \ref{sec:mx}).

\textbf{\D-model expansion} takes as input a definition \D and a
structure , interpreting all parameters of \D, and returns
the unique model \I that expands . This task is an instance
of model expansion, but is solved in \idp using different
technology. The close relationship between definitions and logic
programs under the well-founded semantics is exploited to translate \D
and  into a tabled Prolog program, after which XSB is used to
compute \I.  Taking an extra formula \f as input, with free variables
\xxx, the same approach is used to solve the query \f with respect to
\D and  in a goal-oriented way~\cite{tplp/Jansen13}. There is no dedicated Lua procedure for calling -model expansion. However, as described in Section \ref{sec:mx}, it is automatically detected that -model expansion can be performed in normal model expansion calls. 

\textbf{Unsat-core extraction} takes as input a theory \theory and a structure \I such that \theory has no models expanding \I. It returns a (minimal) theory  entailed by \theory (obtained by instantiating some variables) such that  still has no models expanding \I. Currently there is only support for printing the output theory, not for actually obtaining it; this procedures also prints from which line every sentence in  was instantiated. This inference is particularly useful when debugging logical specification and can be called using \il{printunsatcore(T,S)}.

Finally, there is support for the \emph{linear time calculus} (LTC) defined by Bogaerts et al.~\cite{iclp/Bogaerts14}. One
can build an \il{LTCvocabulary} component as a vocabulary extending a
default LTC vocabulary and can use special inference methods to
initialize the state and to perform progression inference, i.e., to
infer the successor states step by step \cite{iclp/Bogaerts14}.






















































































































































































































































