\documentclass{article} \usepackage{iclr2018_conference,times}
\usepackage{hyperref}
\usepackage{url}
\usepackage{booktabs}
\usepackage{physics}
\usepackage{amssymb}
\usepackage{amsmath}
\usepackage{graphicx} 
\usepackage{bm} 
\usepackage{verbatim}

\newcommand{\bb}{\mathbb}
\newcommand{\mrm}{\mathrm}
\newcommand{\fix}{\marginpar{FIX}}
\newcommand{\new}{\marginpar{NEW}}
\newcommand{\rot}{\mathsf{Rotation}}
\newcommand{\relu}{\mathsf{ReLU}}
\newcommand{\sigm}{\mathsf{sigmoid}}


\title{Rotational Unit of Memory}

\iclrfinalcopy



\author{Rumen Dangovski\thanks{equal contribution} \\
Massachusetts Institute of Technology\\
\texttt{rumenrd@mit.edu} \\
\And
Li Jing\\
Massachusetts Institute of Technology\\
\texttt{ljing@mit.edu} \\
\AND
Marin Solja\v{c}i\'{c}\\
Massachusetts Institute of Technology\\
\texttt{soljacic@mit.edu} \\
}






\begin{document}

\maketitle


\begin{abstract}
The concepts of unitary evolution matrices and associative memory have boosted the field of Recurrent Neural Networks (RNN) to state-of-the-art performance in a variety of sequential tasks. However, RNN still have a limited capacity to manipulate long-term memory. To bypass this weakness the most successful applications of RNN use external techniques such as attention mechanisms. In this paper we propose a novel RNN model that unifies the state-of-the-art approaches: Rotational Unit of Memory (RUM). The core of RUM is its rotational operation, which is, naturally, a unitary matrix, providing architectures with the power to learn long-term dependencies by overcoming the vanishing and exploding gradients problem. Moreover, the rotational unit also serves as associative memory. We evaluate our model on synthetic memorization, question answering and language modeling tasks. RUM learns the Copying Memory task completely and improves the state-of-the-art result in the Recall task. RUM's performance in the bAbI Question Answering task is comparable to that of models with attention mechanism. We also improve the state-of-the-art result to 1.189 bits-per-character (BPC) loss in the Character Level Penn Treebank (PTB) task, which is to signify the applications of RUM to real-world sequential data. The universality of our construction, at the core of RNN, establishes RUM as a promising approach to language modeling, speech recognition and machine translation.
\end{abstract}

\section{Introduction}
Recurrent neural networks are widely used in a variety of machine learning applications such as language modeling (\cite{graves2014neural}), machine translation (\cite{cho2014properties}) and speech recognition (\cite{hinton2012deep}). Their flexibility of taking inputs of dynamic length makes RNN particularly useful for these tasks. However, the traditional RNN models such as Long Short-Term Memory (LSTM, \cite{hochreiter1997long}) and Gated Recurrent Unit (GRU, \cite{cho2014properties}) exhibit some weaknesses that prevent them from achieving human level performance: 1) limited memory--they can only remember a hidden state, which usually occupies a small part of a model;
2) gradient vanishing/explosion  (\cite{bengio1994learning}) during training--trained with backpropagation through time the models fail to learn long-term dependencies.

Several ways to address those problems are known. One solution is to use soft and local attention mechanisms (\cite{cho2014properties}), which is crucial for most modern applications of RNN. Nevertheless, researchers are still interested in improving basic RNN cell models to process sequential data better. Numerous works (\cite{graves2014neural, ba2016using}) use associative memory to span a large memory space. For example, a practical way to implement associative memory is to set weight matrices as trainable structures that change according to input instances for training. Furthermore, the recent concept of unitary or orthogonal evolution matrices (\cite{arjovsky2015unitary, jing2016tunable}) also provides a theoretical and empirical solution to the problem of memorizing long-term dependencies. 





Here, we propose a novel RNN cell that resolves simultaneously those weaknesses of basic RNN. The Rotational Unit of Memory is a modified gated model whose rotational operation acts as associative memory and is strictly an orthogonal matrix.

We tested our model on several benchmarks. RUM is able to solve the synthetic Copying Memory task while traditional LSTM and GRU fail. For synthetic Recall task, RUM exhibits a stronger ability to remember sequences, hence outperforming state-of-the-art RNN models such as Fast-weight RNN (\cite{ba2016using}) and WeiNet (\cite{zhang2017learning}). 
By using RUM we achieve the state-of-the-art result in the  real-world Character Level Penn Treebank task. RUM also outperforms all basic RNN models in the bAbI question answering task. This performance is competitive with that of memory networks, which take advantage of attention mechanisms.



Our contributions are as follows:
\begin{enumerate}
    \item We develop the concept of the Rotational Unit that combines the memorization advantage of unitary/orthogonal matrices with the dynamic structure of associative memory;
    \item We implement the rotational operation into a novel RNN model--RUM--which outperforms significantly the current frontier of models on a variety of sequential tasks. 
    
    
\end{enumerate}

\section{Motivation and Related Work}

\subsection{Unitary Approach}
The problem of the gradient vanishing and exploding problem is well-known to obstruct the learning of long-term dependencies (\cite{bengio1994learning}).







We will give a brief mathematical motivation of the problem. Let's assume the cost function is . In order to evaluate , one computes the derivative gradient using the chain rule:  where  is the Jacobian matrix of the point-wise non-linearity. As long as the eigenvalues of  are of order unity, then if  has eigenvalues , they will cause gradient explosion  , while if  has eigenvalues , they can cause gradient vanishing, . Either situation hampers the efficiency of RNN.




LSTM is designed to solve this problem, but gradient clipping (\cite{Pascanu2012on}) is still required for training. Recently, by restraining the hidden-to-hidden matrix to be orthogonal or unitary, many models have overcome the problem of exploding and vanishing gradients. Theoretically, unitary and orthogonal matrices will keep the norm of the gradient because the absolute value of their eigenvalues equals one. 



Several approaches have successfully developed the applications of unitary and orthogonal matrix to recurrent neural networks. \cite{arjovsky2015unitary, jing2016tunable} use parameterizations to form the unitary spaces. \cite{wisdom2016full} applies gradient projection onto a unitary manifold. \cite{Eugene2017on} uses penalty terms as a regularization to restrain matrices to be unitary, hence accessing long-term memorization.

Only learning long-term dependencies is not sufficient for a powerful RNN. \cite{jing2017gated} finds that the combination of unitary/orthogonal matrices with a gated mechanism improves the performance of RNN because of the benefits of a forgetting ability. \cite{jing2017gated} also points out the optimal way of such a unitary/gated combination: the unitary/orthogonal matrix should appear before the reset gate, which can then be followed by a  activation. In RUM we implement an orthogonal operation in the same place, but the construction of that matrix is completely different: instead of parameterizing the kernel, we encode a natural rotation, generated by the inputs and the hidden state. 





\subsection{Associative Memory Approach}
Limited memory in RNN is truly a shortage. Adding an external associative memory is a natural solution. For instance, the Neural Turing Machine (\cite{graves2014neural}) and many other models have shown the power of using this technique. While it expands the accessible memory space, the technique significantly increases the size of the model, therefore making the process of learning so many parameters harder. 







Now, we will briefly describe the concept of associative memory. In basic RNN,  where  is the hidden state at time step  and  is the input data at each step. Here  and  are trainable parameters that are fixed in the model. A recent approach replaces  with a dynamic  (as a function of time) so that this matrix can serve as a memory state. Thus, the memory size increases from  to , where  is the hidden size. In particular,  is determined by ,  and  which can be a part of a multi-layer or a Hopfiled net. By treating the RNN weights as memory determined by the current input data, a larger memory size is provided and less trainable parameters are required. This significantly increases the memorization ability of RNN. Our model also falls into this category of associative memory through its rotational design of an orthogonal  matrix. 

\section{Methods}
The goal of this section is to suggest ways of engineering models that incorporate rotations as units of memory. In the following discussion  is the input size and  is the hidden size.
\subsection{The Operation }

The operation  is an efficient encoder of an orthogonal operation, which acts as a unit of memory.   computes an orthogonal operator  in  that represents the rotation between two non-collinear vectors  and  in the two-dimensional subspace  of the Euclidean space  with distance . As a consequence,  can act as a kernel on a hidden state . More formally, what we propose is a function 
 
such that after ortho-normalizing  and  to
 we encode the following matrix in 

Figure \ref{main:fig:rotation-and-model} (a) demonstrates the projection to the plane  in the brackets of equation \eqref{main:eq:rotation}. The mini-rotation in this space is  Hence,
.

\begin{figure}[h]
\begin{center}
\includegraphics[width=\textwidth]{rotation-and-model.pdf}
\end{center}
\caption{ \textbf{is a universal differentiable operation that enables the advantages of the RUM architecture.}  (a) The rotation  in the plane defined by  and  acts on the hidden state . (b) The RUM cell, in which  encodes the kernel . The matrix  acts on  and thus keeps the norm of the hidden state. 
\label{main:fig:rotation-and-model}}
\end{figure}

A practical advantage of  is that it is both orthogonal and differentiable. On one hand, it is a composition of differentiable sub-operations, which enables learning via backpropagation. On the other hand, it preserves the norm of the hidden state, hence it can yield more stable gradients. We were motivated to find differentiable implementations of unitary (orthogonal in particular) operations in existing toolkits for deep learning. Our conclusion is that  can be implemented in various frameworks that are utilized for RNN and other deep learning architectures. Indeed,  is not constrained to parameterize a unitary structure, but instead it produces an orthogonal matrix from simple components in the cell, which makes it useful for experimentation.



We implement  together with its action on a hidden state efficiently. 
\footnote{Our code is collected in \url{https://github.com/jingli9111/RUM.git}.} 
We do not need to compute the matrix  before we rotate. Instead we can directly apply the RHS of equation \eqref{main:eq:rotation} to the hidden state. Hence, the memory complexity of our algorithm is , which is determined by the RHS of \eqref{main:eq:rotation}. Note that we only use two trainable vectors in  to generate orthogonal weights in  which means the model has  degrees of freedom for a single unit of memory. Likewise, the time complexity is . Thus,  is a universal operation that enables implementations suitable to any neural network model with backpropagation.

\subsection{The RUM Architecture}

We propose the Recurrent Unit of Memory as the first example of an application of  to a recurrent cell. Figure \ref{main:fig:rotation-and-model} (b) is a sketch of the connections in the cell. RUM consists of an update gate  that has the same function as in GRU. Instead of a reset gate, however, the model learns a memory \textit{target} variable . RUM also learns to embed the input vector  into  to yield  Hence  encodes the rotation between the embedded input and the target, which is accumulated to the associative memory unit  (originally initialized to the identity matrix). Here  is a non-negative integer that is a hyper-parameter of the model. From here, the orthogonal  acts on the state  to produce an evolved hidden state  Finally RUM obtains the new hidden state via , just as in GRU. The RUM equations are as follows

We have introduced time subscripts to demonstrate the recurrence relations. The kernels have dimensions given by ,  and .
The biases are variables  and . The norm  is a scalar hyper-parameter of the RUM model. 

The orthogonal matrix  conceptually takes the place of a kernel acting on the hidden state in GRU. This is the most efficient place to introduce an orthogonal operation, as the Gated Orthogonal Recurrent Unit (GORU, \cite{jing2017gated}) experiments suggest. The difference with the GORU cell is that GORU parameterizes and learns the kernel as an orthogonal matrix, while RUM does not parameterize the rotation . Instead, RUM learns , which together with ,  determines .
The orthogonal matrix keeps the norm of the vectors, so we experiment with a  activation instead of the conventional  in gated mechanisms. 

Even though  is an orthogonal element of RUM, the norm of  is not stable because of the  activation. Therefore, we suggest normalizing the hidden state  to a have norm . We call this technique \textit{time normalization} as we usually feed mini-batches to the RNN during learning that have the shape  where  is the size of the batch and  is the length of the sequence that we feed in. Time normalization happens along the  sequence dimension as opposed to the batch dimension in batch normalization. Choosing appropriate  for the RUM model stabilizes learning and ensures the eigenvalues of the kernels are bounded from above. This in turn means that the smaller  is, the more we reduce the effect of exploding gradients. 

Finally, even though RUM uses an update gate, it is not a standard gated mechanism, as it does not have a reset gate. Instead we suggest utilizing additional memory via the target vector . By feeding inputs to RUM,  adapts to encode rotations, which align the hidden states in desired locations in , without changing the norm of . We believe that the unit of memory  gives advantage to RUM over other gated mechanisms, such as LSTM and GRU.

\section{Experiments} \label{main:sec:experiments}
Firstly, we test RUM's memorization capacity on the Copying Memory Task. Secondly, we signify the superiority of RUM by obtaining a state-of-the-art result in the Associative Recall Task. Thirdly, we show that even without external memory, RUM achieves comparable to state-of-the-art results in the bAbI Question Answering data set. Finally, we utilize RUM's rotational memory to reach 1.189 BPC in the Character Level Penn Treebank.

We experiment with  RUM and  RUM, the latter model corresponding to tuning in the rotational associative memory. 

\subsection{Copying Memory Task}
A standard way to evaluate the memory capacity of a neural network is to test its performance in the Copying Memory Task (\cite{hochreiter1997long}, \cite{henaff2016orthogonal}
\cite{arjovsky2015unitary}). We follow the setup in \cite{jing2016tunable}. The objective of the RNN is to remember (copy) information received  time steps earlier (see section \ref{app:sec:copy} for details about the data).

Our results in this task demonstrate: 1. RUM utilizes a different representation of memory that outperforms those of LSTM and GRU; 2. RUM solves the task completely, despite its update gate, which does not allow all of the information encoded in the hidden stay to pass through. The only other gated RNN model successful at copying is GORU. Figure \ref{main:fig:copying} reveals that LSTM and GRU hit a predictable baseline, which is equivalent to random guessing. RUM falls bellow the baseline, and subsequently learns the task by achieving zero loss after a few thousands iterations.

\begin{figure}[h!]
\centering
\includegraphics[width=0.495\linewidth]{copying-200.pdf}
\includegraphics[width=0.495\linewidth]{copying-500.pdf}
\caption{\textbf{The orthogonal operation  enables RUM to solve the Copying Memory Task}. The delay times are ,  and . For all models  except for the RUM models with , for which . For the training of all models we use RMSProp optimization with a learning rate of 0.001 and a decay rate of 0.9; the batch size  is 128.}
\label{main:fig:copying}
\end{figure}

With the help of figure \ref{main:fig:copying} we will explain how the additional hyper-parameters for RUM affect its training. We observe that when we remove the normalization () then RUM learns more quickly than the case of requiring a norm . At the same time, though, the training entails more fluctuations. Hence we believe that choosing a finite  to normalize the hidden state is an important tool for stable learning. Moreover, it is necessary for the NLP task in this paper (see section \ref{main:sec:nlp}): for our character level predictions we use large hidden sizes, which if left unnormalized, can make the cross entropy loss blow up.

We also observe the benefits of tuning in the associative rotational memory. Indeed, a  RUM has a smaller hidden size, , yet it learns much more quickly than a  RUM. It is possible that the accumulation of phase via  to enable faster long-term dependence learning than the  case. Either way, both models overcome the vanishing/exploding gradients, and eventually learn the task completely. 


\subsection{Associative Recall Task}
Another important synthetic task to test the memory ability of recurrent neural network is the Associative Recall. This task requires RNN to remember the whole sequence of the data and perform extra logic on the sequence.

We follow the same setting as in \cite{ba2016using} and \cite{zhang2017learning} and modify the original task so that it can test for longer sequences. In detail, the RNN is fed into a sequence of characters, e.g. ``a1s2d3f4g5??d''. The RNN is supposed to output the character based on the ``key'' which is located at the end of the sequence. The RNN needs to look back into the sequence and find the ``key'' and then to retrieve the next character. In this example, the correct answer is ``3''. See section \ref{app:sec:recall} for further details about the data. 

In this experiment, we compare RUM to an LSTM, , a Fast-weight RNN (\cite{ba2016using}) and a recent successful RNN WeiNet (\cite{zhang2017learning}). All the models have the same hidden state  for different lengths . We use a batch size 128. The optimizer is RMSProp with a learning rate 0.001. We find that LSTM fails to learn the task, because of its lack of sufficient memory capacity. NTM and Fast-weight RNN fail longer tasks, which means they cannot learn to manipulate their memory efficiently. Table \ref{tbl:recall} gives a numerical summary of the results and figure \ref{main:fig:recall}, in the appendix, compares graphically RUM to LSTM.

\begin{table}[h!]
\centering
\begin{tabular}{lcc|c}
\toprule
  Model &   Length  & Length  & \# Parameters\\ 
   \hline 
 LSTM  & 25.6\% & 20.5\% & 17k\\
FW-LN (\cite{ba2016using}   & 100\% & 20.8\%  & 9k\\ 
WeiNet (\cite{zhang2017learning}) & 100\% & 100\%  & 22k\\
RUM (ours) & 100\% & 100\% & 13k \\
\bottomrule \\
\end{tabular}
\caption{Comparison of the models on convergence validation accuracy. Only RUM and the recent WeiNet are able to successfully solve the  Associative Recall task with a hidden state of 50. RUM has significantly less parameters.}
\label{tbl:recall}
\end{table}


\subsection{Question Answering}
Question answering remains one of the most important applicable tasks in NLP. Almost all state-of-the-art performance is achieved by the means of attention mechanisms. Few works have been done to improve the performance by developing stronger RNN. Here, we tested RUM on the bAbI Question Answering data set (\cite{weston2015towards}) to demonstrate its ability to memorize and reason without any attention. In this task, we train 20 sub-tasks jointly for each model. See section \ref{app:sec:babi} for detailed experimental settings and results on each sub-task. 

We compare our model with several baselines: a simple LSTM, an End-to-end Memory Network (\cite{sukhbaatar2015end}) and a GORU. We find that RUM outperforms significantly LSTM and GORU and achieves competitive result with those of MemN2N, which has an attention mechanism. We summarize the results in Table \ref{tbl:babi}. We emphasize that for some sub-tasks in the table, which require large memory, RUM outperforms models with attention mechanisms (MemN2N). 

\begin{table}[h!]
\begin{center}
\begin{tabular}{lr}
\toprule
Model  & Test Accuracy (\%) \\ 
\hline
LSTM (\cite{weston2015towards})  & 49  \\ 
GORU (\cite{jing2017gated}) & 60  \\
MemN2N (\cite{sukhbaatar2015end}) & 86 \\
RUM (ours) & \textbf{73.2}\\
\bottomrule \\
\end{tabular}
\caption{Question Answering task on bAbI dataset. Test accuracy (\%) on LSTM, MemN2N, GORU
and RUM. RUM significantly outperforms LSTM/GORU and has a performance close to that of
MemN2N, which uses an attention mechanism.}
\label{tbl:babi}
\end{center}
\end{table}





\subsection{Character Level Language Modeling} \label{main:sec:nlp}
The rotational unit of memory is a natural architecture that can learn long-term structure in data while avoiding significant overfitting. Perhaps, the best way to demonstrate this unique property, among other RNN models, is to test RUM on real world character level NLP tasks.
\subsubsection{Penn Treebank Corpus Data Set}
The corpus is a collection of articles in \textit{The Wall Street Journal} (\cite{marcus1993ptb}). The text is in English and its vocabulary consists of 10000 words. We split the data into \textit{train}, \textit{validation} and \textit{test} sets according to \cite{mikolov2012ptb}. We train by feeding mini-batches of size  that consist of sequences of  consecutive characters. 

We incorporate RUM into the state-of-the-art high-level model: Fast-Slow RNN (FS-RNN, \cite{mujika2017ptb}). The FS-RNN- architecture consists of two hierarchical layers: one of them is a ``fast'' layer that connects  RNN cells  in series; the other is a ``slow'' layer that consists of a single RNN cell . The organization is roughly as follows:  receives the input from the mini-batch and feeds its state into ;  feeds its state into ; the output of  is the probability distribution of the predicted character. 

Table \ref{tbl:ptb} outlines the performance of some FS-RNN models along with other results in the PTB data set, in which we present the improved test BPC. \cite{mujika2017ptb} achieve their record with FS-LSTM-2, by setting  and  to LSTM. The authors in the same paper suggest that the ``slow'' cell has the function of capturing long-term dependencies from the data. Hence, it is natural to set  to be a RUM, given its memorization advantages. In particular, we experiment with FS-RUM-2, for which  is a RUM and  are LSTM. Additionally, we test the performance of a simple RUM and a two-layer RUM. 

As the models are prone to overfitting, for each of our models we follow the experimental settings for regularization in \cite{mujika2017ptb}, presented in section \ref{app:sec:ptb}. Those techniques work particularly well in combination with the rotational structure of RUM. More specifically, FS-RUM-2 needs more than 350 epochs to converge by following a suitable learning rate pattern (see table \ref{tbl:ptb:lr} in the appendix). FS-RUM-2 generalizes better than other gated models, such as GRU and LSTM, because it learns efficient patterns for activation in its kernels. Such a skill is useful for the large Penn Treebank data set, as with its special diagonal structure, the RUM cell in FS-RUM-2 activates almost all neurons in the hidden state. We discuss this representational advantage in section \ref{main:sec:va}.



\begin{table}[h!]
\begin{center}
\begin{tabular}{lrrr|r}
\toprule
Model  & BPC & \# Parameters \\ 
\hline
Zoneout LSTM (\cite{krueger2016zoneout}) & 1.27  & -- \\
RUM 2000 (ours)  & 1.28 & 8.9M \\ 
2  RUM 1500 (ours) & 1.26 & 16.4M \\\hline
HM-LSTM  (\cite{chung2016ptb}) & 1.24 & -- \\
HyperLSTM (\cite{ha2016ptb}) & 1.219 & 14.4M \\
NASCell (\cite{zoph2017ptb}) & 1.214 & 16.3M \\
FS-LSTM-4 (\cite{mujika2017ptb}) & 1.193 & 6.5M \\
FS-LSTM-2 (\cite{mujika2017ptb}) & 1.190 & 7.2M \\
\hline
FS-RUM-2 (ours) & \textbf{1.189}  & 11.2M \\
\bottomrule
\end{tabular}
\caption{With FS-RUM-2 we achieve the state-of-the-art test result on the Penn Treebank task. Additionally, a non-extensive grid search for vanilla RNN models yields comparable results to that of Zoneout LSTM.}
\label{tbl:ptb}
\end{center}
\end{table}




\section{Discussion}
\subsection{Visual Analysis} \label{main:sec:va}
One advantage of the Rotational Unit of Memory is that it allows the model to encode information in the phase of the hidden state. In order to demonstrate the structure behind such learning, we look at the kernels that generate the target memory  in the RUM model. Figure \ref{main:fig:visual-analysis} (a) is a visualization for the Recall task that demonstrates the diagonal structure of  which generates  (a diagonal structure is also present , but it is contrasted less). One way to interpret the importance of the diagonal contrast is that each neuron in the hidden state plays an important role for learning since each element on the diagonal activates a distinct neuron. Therefore, it seems that RUM utilizes the capacity of the hidden state almost completely. For this reason, we might consider RUM as an architecture that is close to the theoretical optimum of the representational power of RNN models.  

Moreover, the diagonal structure is not task specific. For example, in Figure \ref{main:fig:visual-analysis} (b) we observe a particular  for the target  on the Penn Treebank task. The way we interpret the meaning of the diagonal structure, combined with the off-diagonal activations, is that probably they encode grammar and vocabulary, as well as the links between various components of language.


\begin{figure}[h]
\begin{center}
\includegraphics[width=\textwidth]{visual-analysis.pdf}
\end{center}
\caption{\textbf{The kernel generating the target memory for RUM is following a diagonal activation pattern, which signifies the sequential learning of the model}. (a) A temperature map of the values of the variables when the model is learned. The task is Associative Recall, , and the model is RUM, , with  and without time normalization. (b) An interpretation of the function of the diagonal and off-diagonal activations of RUM's  kernel on NLP tasks. The task is Character Level Penn Treebank and the model is  RUM, , . See section \ref{app:sec:vis} for additional examples.
\label{main:fig:visual-analysis}}
\end{figure}

\subsection{Theoretical Analysis}
It is natural to view the Rotational Unit of Memory and many other approaches using orthogonal matrices to fall into the category of \textit{phase-encoding architectures}: , where  is a phase information matrix. For instance, we can parameterize any orthogonal matrix according to the Efficient Unitary Neural Networks (EUNN, \cite{jing2016tunable}) architecture:
, where  is a block diagonal matrix containing  numbers of 2-by-2 rotations. The component  is an one-by-() parameter vector. Therefore, the rotational memory equation in our model can be represented as

where  are rotational memory phase vectors at time  and  represents the phases generated by the operation  correspondingly. 
Note that each element of the matrix multiplication  only depends on one element from  and  each. This means that, to cancel out one element , the model only needs to learn to express  as the negation of . 

 As a result, our RNN implementation does not require a reset gate, as in GRU or GORU, because the forgetting mechanism is automatically embedded into the representation \eqref{main:eq:representation} of phase-encoding. 

Thus, the concept of phase-encoding is simply a special sampling on manifolds generated by the special orthogonal Lie group . Now, let  be the hidden size. One way to extend the current RUM model is to allow for  to be any real number in the associative memory equation  This will expand the representational power of the rotational unit. The difficulty is to mathematically define the raising of a matrix to a real power, which is equivalent to defining a logarithm of a matrix. Again, rotations prove to be a natural choice since they are elements of , and their logarithms correspond to elements of the vector space of the Lie algebra , associatied to 

\subsection{Future work}
For future work, the RUM model can be applied to other higher-level RNN structures. For instance, in section \ref{main:sec:nlp} we already showed how to successfully embed RUM into FS-RNN to achieve state-of-the-art results. Other examples may include Recurrent Highway Networks (\cite{zilly2016recurrent}), HyperNetwork (\cite{ha2016ptb}) structures, etc. The fusion of RUM with such architectures could lead to more state-of-the-art results in sequential tasks. 



\section{Conclusion}
We proposed a novel RNN architecture: Rotational Unit of Memory. The model takes advantage of the unitary and associative memory concepts. RUM outperforms many previous state-of-the-art models, including LSTM, GRU, GORU and NTM in synthetic benchmarks: Copying Memory and Associative Recall tasks. Additionally, RUM's performance in real-world tasks, such as question answering and language modeling, is competetive with that of advanced architectures, some of which include attention mechanisms. We claim the rotational unit of memory can serve as the new benchmark model that absorbs all advantages of existing models in a scalable way. Indeed, the rotational operation can be applied to many other fields, not limited only to RNN, such as Convolutional and Generative Adversarial Neural Networks.

\subsubsection*{Acknowledgments} We would like to thank Konstantin Rangelov for the supply of some of the computational power used for this research. We are grateful to Yichen Shen, Charles Roques-Carmes, Peter Lu, Rawn Henry, Fidel Cano-Renteria and Rumen Hristov for fruitful discussions. Many thanks to Pamela Siska and Irina Tomova for their comments on the paper. This work was partially supported by the Army Research Office through the Institute for Soldier Nanotechnologies under contract W911NF-13-D0001, the National Science Foundation under Grant No. CCF-1640012, and by the Semiconductor Research Corporation under Grant No. 2016-EP-2693-B.

\bibliography{iclr2018_conference}
\bibliographystyle{iclr2018_conference}


\section*{Appendix}
\appendix 
\section{Copying Memory Task} \label{app:sec:copy}
The alphabet of the input consists of symbols , the first  of which represent data for copying, and the remaining two form ``blank'' and ``marker'' symbols, respectively. In our experiment  and the data for copying is the first 10 symbols of the input. The expectation from the RNN model is to output ``blank'' and,  after the ``marker'' appears in the input, to output (copy) sequentially the initial data of 10 steps.


\section{Associative Recall Task} \label{app:sec:recall}
The sequences for training are randomly generated, and consist of pairs of ``character'' and ``number'' elements. We set the key to always be a ``character''. We fix the size of the ``character'' set equal to half of the length of the sequence and the size of the ``number'' set equal to 10. Therefore, the total category has a size of .

\begin{figure}[h!]
\centering
\includegraphics[width=0.495\linewidth]{recall_acc.pdf}
\includegraphics[width=0.495\linewidth]{recall_loss.pdf}
\caption{\textbf{The associative memory provided by rotational operation  enables RUM to solve the Associative Recall Task}. The input sequences is 50 . For all models . For the training of all models we use RMSProp optimization with a learning rate of 0.001 and a decay rate of 0.9; the batch size  is 128. We observe that it is necessary to tune in the associative memory via  since  RUM does not learn the task. 
}
\label{main:fig:recall}
\end{figure}


\section{Question Answering bAbI Task} \label{app:sec:babi}

In this task, we train 20 models jointly on each sub-task. All of them use a 10k data set, which is divided into 90\% of training and 10\% of validation. We first tokenize all the words in the data set and combine the story and question by simply concatenating two sequences. Different length sequences are filled with ``blank'' at the beginning and the end. Words in the sequence are embedded into dense vectors and then fed into RNN in a sequential manner. The RNN model outputs the answer prediction at the end of the question through a softmax layer. 
We use batch size of 32 for all 20 subsets. The model is trained with Adam Optimizer with a learning rate 0.001. Each subset is trained with 20 epochs and no other regularization is applied. 

\begin{table*}[!htb]
\centering
\begin{tabular}{lcccc}
\hline
Task       & RUM  & LSTM & GORU  &  MemN2N   \\ 
 & (ours) &  (Weston et al.)& (Jing et al.)  & (Sukhbaatar el al.) \\
\hline
Single Supporting Fact	&   79.7&   50       &   46 &   100      \\
Two Supporting Facts	&	39.4&   20	    &	40  &   92       \\
Three Supporting Facts	&   46.6&   20      &	34  &   60       \\
Two Arg. Relations	    & 	100&	61      &	63   &   97     	\\
Three Arg. Relations    &   96.8&   70     &	87    &   87      \\
Yes/No Questions	    &   84.7&   48     &	54   &   94       \\
Counting	            &   89.1&   49    &   78    &   83       \\
Lists/Sets	            &   74.0&   45    &   75   &   90        \\
Simple Negation	        &   84.0&   64     &   63   &   87       \\
Indefinite Knowledge    &   75.7&   44  &   45     &   85        \\
Basic Coreference	    &   90.6&   72      &   69    &   99     \\
Conjunction	            &   95.0&   74   &	70     &   99       \\
Compound Coref.	        &   91.5&   94   &	93     &   99      	\\
Time Reasoning	        &   43.9&   27   &	38     &   98       \\
Basic Deduction	        &   58.8&   21  &	55      &	100      \\
Basic Induction	        &   47.1&   23   &	44     &   99      	\\
Positional Reasoning	&   60.3&   51 	&	60      &	49   	\\
Size Reasoning	        &   98.5&   52    &	91   &	89  	 	\\
Path Finding	        &   10.2&   8     &	9   &	17		  	\\
Agent's Motivations     &   98.0&   91    &	98   &   100       \\
\hline
Mean Performance	    &   73.2&   49      &	60&	86	       \\
\hline
\end{tabular}
\caption{Question Answering task on bAbI dataset. Test accuracy (\%) on LSTM, MemN2N, GORU and RUM. RUM significantly outperforms LSTM/GORU and has a performance close to that of MemoryNN, which uses an attention mechanism.}
\end{table*}

\section{Character Level Penn Treebank Task} \label{app:sec:ptb}
For all RNN cells we apply layer normalization (\cite{ba2016ln}) to the cells and to the LSTM gates and RUM's update gate and target memory, zoneout (\cite{krueger2016zoneout}) to the recurrent connections, and dropout (\cite{srivastava2014dropout}) to the FS-RNN. For training we use Adam optimization (\cite{kingma2015adam}). We apply gradient clipping  with maximal norm of the gradients equal to 1.0. Table \ref{tbl:ptb:hyper} lists the hyper-parameters we use for our models. 

We embed the inputs into a higher-dimensional space. The output of each models passes through a softmax layer; then the probabilities are evaluated by a standard cross entropy loss function. The \textit{bits-per-character} (BPC) loss is simply the cross entropy with a binary logarithm.

\begin{table}[h!]
\centering
\begin{tabular}{lrrrr}
\toprule
  Hyper-parameters & RUM & 2  RUM & FS-RUM-2 \\ 
   \hline
   Non-recurrent dropout & 0.35 & 0.35 & 0.35 \\
   Cell zoneout & 0.5 & 0.5 & 0.5  \\ 
   Hidden zoneout & 0.1 & 0.1 & 0.1  \\ 
   \hline 
   Fast cell size (LSTM) & N/A & N/A & 700  \\ 
   Associative power  & 0 & 0 & 0  \\
   Time normalization  & 1.0 & 0.3 & 1.0  \\ 
   \hline
   Slow cell size (RUM)  & 2000 & 1500 & 1000  \\ 
   \hline 
    length & 150 & 150 & 150  \\ 
   Mini-batch size & 128 & 128 & 128  \\ 
   Input embedding size & 128 & 128 & 128  \\ 
   Initial learning rate & 0.002 & 0.002 & 0.002  \\ 
   Epochs & 100 & 100 & 360  \\ 
\bottomrule \\
\end{tabular}
\caption{Hyper-parameters for the Character Level Penn Treebank Task.}
\label{tbl:ptb:hyper}
\end{table}

\begin{table}[h!]
\centering
\begin{tabular}{lr}
\toprule
  Learning rate & Epochs \\ 
   \hline
  0.002 & 1-180 \\ 
  0.0001 & 181-240 \\ 
  0.00001 & 241-360 \\
\bottomrule
\end{tabular}
\caption{Suggested learning rate pattern for training FS-RUM-2 with a standard Adam optimizer.}
\label{tbl:ptb:lr}
\end{table}

\section{Visualization} \label{app:sec:vis}

\begin{figure}[h]
\begin{center}
\includegraphics[width=0.5\textwidth]{copying-arum-2-kernel.pdf}
\end{center}
\caption{The collection of kernels for  RUM, ,  N/A for the Copying task, . 
\label{app:fig:copying-rum-kernel}}
\end{figure}

\begin{figure}[h]
\begin{center}
\includegraphics[width=0.5\textwidth]{babi-rum-kernel.pdf}
\end{center}
\caption{The collection of kernels for  RUM, ,  N/A for the Question Answering bAbI Task.
\label{app:fig:babi-rum-kernel}}
\end{figure}


\end{document}