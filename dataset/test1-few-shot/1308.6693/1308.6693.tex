


\documentclass{elsarticle}

\usepackage{fullpage}
\usepackage{latexsym}
\usepackage{epsfig}
\usepackage{color}




\newtheorem{theorem}{Theorem}
\newtheorem{lemma}{Lemma}
\newtheorem{definition}{Definition}
\newtheorem{corollary}{Corollary}
\newenvironment{proof}{{\bf Proof: }}{\hspace*{\fill}\medskip}
\newcommand{\settab}{-----\=-----\=-----\=-----\=-----\=-----\= \kill}

\pagestyle{plain}

\begin{document}

\title{Transforming planar graph drawings while maintaining height
}
\author[uw]{Therese Biedl}
\address[uw]{David R.~Cheriton School of Computer Science, 
University of Waterloo, 
Waterloo, ON N2L 3G1, Canada, {\tt biedl@uwaterloo.ca}}
\date{\today}


\begin{abstract}
There are numerous styles of planar graph drawings,  notably
straight-line drawings, poly-line drawings, orthogonal graph
drawings and visibility representations.  In this note,
we show that many of these drawings can be transformed from
one style to another without changing the height of the drawing.
We then give some applications of these transformations. 
\end{abstract}


\begin{keyword} 
Graph drawing; straight-line drawing; 
orthogonal drawing; visibility representation; upward drawings.
\end{keyword}

\maketitle
\section{Introduction}
\label{se:intro}


Let  be a simple graph with  vertices and
 edges.  We assume that  is {\em planar},
i.e., it can be drawn without crossing.    
In planar graph drawing, one aims to create a crossing-free picture of .
It was known for a long time that such drawings
exist even with straight lines \cite{Wagner36,Fary48,Stein51}.
and even in an -grid \cite{FPP90,Sch90}.
Many improvements have been developed since, see for example
\cite{DBETT98,NR04}.


Formally, 
a {\em drawing} of a graph consists of assigning a point or an axis-aligned
box to every vertex, and a curve between the points/boxes of   and
 to every edge .   The drawing is called {\em planar} if
no two elements of the drawing intersect unless the corresponding elements
of the graph do.  Thus no 
vertex points/boxes coincide, no edge curve
self-intersects, no edge curves intersect each other (except at common
endpoints), and no edge curve intersects a vertex point/box other than 
its endpoints.  In this paper all drawings
are required to be planar.  

In a {\em straight-line drawing} vertices are
represented by points and edges are drawn as straight-line segments.
In a {\em poly-line drawing}, vertices are points and each edge curve is a
continguous sequence of line segments.  The place where an edge curves
changes direction is called a {\em bend}.
An {\em orthogonal drawing} uses a box for every vertex, and
requires that edges are drawn as poly-lines for which every line
segment is either horizontal or vertical.\footnote{If the maximum degree is 4, then one could additionally
demand vertices to be points.  We will not study this model in the
current paper, so an orthogonal drawing always allows boxes.}
A {\em visibility
representations} is an orthogonal drawing without bends.

In this paper we sometimes restrict these drawings further.
An orthogonal drawing is called
{\em flat} if all boxes of vertices are degenerated into horizontal
segments.
We say that a drawing  is {\em -monotone} if for any edge
, the drawing of the edge in  forms a -monotone
path.  (Horizontal edge segments are allowed.)
Any straight-line drawing
and any visibility representation is automatically -monotone.
See also Fig.~\ref{fi:draw_ex}.  (In our drawings, we thicken vertex boxes
slightly, so that horizontal segments appear as boxes of small height.)

\iffalse
\begin{figure}[ht]
\hspace*{\fill}
\input{draw_ex_1.pdf_t}
\hspace*{\fill}
\input{draw_ex_2.pdf_t}
\hspace*{\fill}
\input{draw_ex_4.pdf_t}
\hspace*{\fill}
\input{draw_ex_6.pdf_t}
\hspace*{\fill}
\caption{The same graph in a straight-line drawing, a -monotone
poly-line drawing,
a flat -monotone orthogonal drawing, and a flat visibility representation.}
\label{fi:draw_ex}
\end{figure}
\fi

In all drawings, the defining elements (i.e.,
points of vertices, corners of boxes of vertices, bends, and attachment points
of edges to vertex-boxes) must be placed
at points with integer coordinates.  A drawing is said to have
{\em width } and {\em height } if (possibly after translation)
all such points are placed on the -grid.  
The height is thus measured by the number of {\em rows}, 
i.e., horizontal lines with integer -coordinates that are occupied by the 
drawing.  After rotation, we may assume that the height is no larger
than the width.

In a previous paper \cite{Bie-DCG11}, we studied transformations
between these graph drawing styles that preserved the asymptotic
area.  In particular, we showed that if  has a visibility
representation, then it has a poly-line drawing of asymptotically
the same area.  In this paper, we study such
transformations with the goal of keeping
the height of the drawing unchanged.
Our results are illustrated in Figure~\ref{fig:transform_graph}.
All our transformations have two additional properties that 
are useful in the proofs and 
some of the applications.  First, the resulting drawing
{\em has the same
 -coordinates}, i.e., any vertex (and also any bend that is
not removed) has the same -coordinate in
the new drawing as in the original one.  (Since we give transformations
only for flat orthogonal drawings, this concept makes sense even when
transforming vertex-boxes into points.) Second, the resulting drawings
have {\em the same left-to-right order in each row}, i.e., if 
vertices  and 
had the same -coordinate, with  left of , then the same
also holds in the resulting drawing.

\begin{figure}[ht]
\hspace*{\fill}
\begin{picture}(0,0)\includegraphics{results.pdf}\end{picture}\setlength{\unitlength}{1776sp}\begingroup\makeatletter\ifx\SetFigFont\undefined \gdef\SetFigFont#1#2#3#4#5{\reset@font\fontsize{#1}{#2pt}\fontfamily{#3}\fontseries{#4}\fontshape{#5}\selectfont}\fi\endgroup \begin{picture}(16674,4988)(-2186,-73044)
\put(4501,-68311){\makebox(0,0)[lb]{\smash{{\SetFigFont{9}{10.8}{\rmdefault}{\mddefault}{\updefault}{\color[rgb]{0,0,0}Theorem~\ref{thm:OD_VR}}}}}}
\put(1276,-70861){\makebox(0,0)[lb]{\smash{{\SetFigFont{9}{10.8}{\rmdefault}{\mddefault}{\updefault}{\color[rgb]{0,0,0}Theorem~\ref{thm:VR_SL}}}}}}
\put(376,-72061){\makebox(0,0)[lb]{\smash{{\SetFigFont{9}{10.8}{\rmdefault}{\mddefault}{\updefault}{\color[rgb]{0,0,0}straight-line drawing}}}}}
\put(6301,-72061){\makebox(0,0)[lb]{\smash{{\SetFigFont{9}{10.8}{\rmdefault}{\mddefault}{\updefault}{\color[rgb]{0,0,0}-monotone poly-line drawing}}}}}
\put(8701,-70861){\makebox(0,0)[lb]{\smash{{\SetFigFont{9}{10.8}{\rmdefault}{\mddefault}{\updefault}{\color[rgb]{0,0,0}Theorem~\ref{thm:PL_OD}}}}}}
\put(6301,-69661){\makebox(0,0)[lb]{\smash{{\SetFigFont{9}{10.8}{\rmdefault}{\mddefault}{\updefault}{\color[rgb]{0,0,0}-monotone flat orthogonal drawing}}}}}
\end{picture} \hspace*{\fill}
\caption{Height-preserving transformations proved in
this paper. Dashed arrows are trivial implications.  
}
\label{fig:transform_graph}
\label{fi:draw_ex}
\end{figure}

We then study some applications of these results.  Most importantly,
they allow to derive some height-bounds for drawing styles for
which we are not aware of any direct proof, and they allow to
formulate some NP-hard graph drawing problems as integer programs.

\section{Flat visbility representations to straight-line drawings}
\label{se:VR_SL}


\begin{theorem}
\label{thm:VR2SL}
\label{thm:VR_SL}
\label{th:VR2SL}
Any flat visibility representation  can be transformed into a 
straight-line drawing  with the same -coordinates
and the same left-to-right orders in each row.
\end{theorem}
\begin{proof}
For any vertex , use  and  to denote leftmost and
rightmost -coordinate and (unique) -coordinate of the box that 
represents  in .   
Use  and  to denote
the (initially unknown) coordinates of  in . 
For any vertex set , hence -coordinates are the same.

Let  be the vertices sorted by , breaking ties
arbitrarily. The algorithm determines  for each vertex by 
processing vertices in 
this order and expanding the drawing  created for 
 into a drawing  of ,
which has the same left-to-right orders. 

Suppose  has been computed for all  already.  To find
, determine lower bounds for it by considering all predecessors
of  and taking the maximum over all of them.  
(For each vertex , the {\em predecessors}
of  are the neighbours of  that come earlier in the order
.) A first (trivial)
lower bound for  is that it needs to be to the right of
anything in row .  Thus, if 
contains a vertex
or part of an edge at point , then  is required.

Next consider any predecessor  of  with .  
Since  and  are not in the same row, they must see each other
vertically in , which means that .
See also Fig.~\ref{fig:transform}.
So if  has a neighbour  to its right in , then
, which implies that 
 has not yet been added to .  Since 
has the same left-to-right orders,  is hence the rightmost
vertex in its row in  and can see towards infinity on the
right.  But then  can also see the point , or
in other words, there exists some  such that  can see all
points  for .  
Impose the lower bound  on the
-coordinate of .

\begin{figure}[ht]
\hspace*{\fill}
\begin{picture}(0,0)\includegraphics{transform_1.pdf}\end{picture}\setlength{\unitlength}{1579sp}\begingroup\makeatletter\ifx\SetFigFont\undefined \gdef\SetFigFont#1#2#3#4#5{\reset@font\fontsize{#1}{#2pt}\fontfamily{#3}\fontseries{#4}\fontshape{#5}\selectfont}\fi\endgroup \begin{picture}(5124,2343)(2089,-5125)
\put(6826,-5011){\makebox(0,0)[lb]{\smash{{\SetFigFont{8}{9.6}{\rmdefault}{\mddefault}{\updefault}{\color[rgb]{0,0,0}}}}}}
\put(5851,-3061){\makebox(0,0)[lb]{\smash{{\SetFigFont{8}{9.6}{\rmdefault}{\mddefault}{\updefault}{\color[rgb]{0,0,0}}}}}}
\put(3751,-4111){\makebox(0,0)[lb]{\smash{{\SetFigFont{8}{9.6}{\rmdefault}{\mddefault}{\updefault}{\color[rgb]{0,0,0}of }}}}}
\put(3451,-3811){\makebox(0,0)[lb]{\smash{{\SetFigFont{8}{9.6}{\rmdefault}{\mddefault}{\updefault}{\color[rgb]{0,0,0}other parts}}}}}
\end{picture} \hspace*{\fill}
\begin{picture}(0,0)\includegraphics{transform_2.pdf}\end{picture}\setlength{\unitlength}{1579sp}\begingroup\makeatletter\ifx\SetFigFont\undefined \gdef\SetFigFont#1#2#3#4#5{\reset@font\fontsize{#1}{#2pt}\fontfamily{#3}\fontseries{#4}\fontshape{#5}\selectfont}\fi\endgroup \begin{picture}(8801,2193)(139,-7825)
\put(8701,-7711){\makebox(0,0)[lb]{\smash{{\SetFigFont{8}{9.6}{\rmdefault}{\mddefault}{\updefault}{\color[rgb]{0,0,0}}}}}}
\put(2251,-5911){\makebox(0,0)[lb]{\smash{{\SetFigFont{8}{9.6}{\rmdefault}{\mddefault}{\updefault}{\color[rgb]{0,0,0}}}}}}
\put(5026,-7036){\makebox(0,0)[lb]{\smash{{\SetFigFont{8}{9.6}{\rmdefault}{\mddefault}{\updefault}{\color[rgb]{0,0,0}}}}}}
\put(1051,-6586){\makebox(0,0)[lb]{\smash{{\SetFigFont{8}{9.6}{\rmdefault}{\mddefault}{\updefault}{\color[rgb]{0,0,0}other parts}}}}}
\put(1276,-6886){\makebox(0,0)[lb]{\smash{{\SetFigFont{8}{9.6}{\rmdefault}{\mddefault}{\updefault}{\color[rgb]{0,0,0}of }}}}}
\put(6001,-6586){\makebox(0,0)[lb]{\smash{{\SetFigFont{8}{9.6}{\rmdefault}{\mddefault}{\updefault}{\color[rgb]{0,0,0}unobstructed}}}}}
\put(7201,-7561){\makebox(0,0)[lb]{\smash{{\SetFigFont{8}{9.6}{\rmdefault}{\mddefault}{\updefault}{\color[rgb]{0,0,0}}}}}}
\put(3151,-6586){\makebox(0,0)[lb]{\smash{{\SetFigFont{8}{9.6}{\rmdefault}{\mddefault}{\updefault}{\color[rgb]{0,0,0}}}}}}
\end{picture} \hspace*{\fill}
\caption{Transforming a flat visibility drawing into a straight-line
drawing with the same -coordinates.}
\label{fig:transform}
\end{figure}

Now let  be the smallest value that satisfies the above lower
bounds (from the row  and from all predecessors of  in
different rows.)  Set  if there were no such lower bounds.\footnote{To simplify the calculations below it helps to use 0 (as
opposed to 1) for the leftmost -coordinate.}
Directly by construction, placing  at
 allows it to be connected with straight-line segments
to all its predecessors.  This includes the predecessor (if any)
that is in the same row as , since there can be at most one in a flat
visibility representation, and it is in the same row as .
This gives a drawing  of  as
desired, and the result follows by induction.
\end{proof}

\subsection{Width considerations}

While our transformation keeps the height intact,
the width can increase dramatically.  
Fig.~\ref{fig:bad_example} shows a flat visibility representations of 
height  and width  
such that the transformation of Theorem~\ref{th:VR2SL} 
has width .
Specifically, using induction one shows that vertex  (for
) is placed
with -coordinate   and leaves an 
edge with slope .
But this is (asymptotically) the
worst that can happen.  

\begin{figure}[ht]
\hspace*{\fill}
\begin{picture}(0,0)\includegraphics{bad_example_1.pdf}\end{picture}\setlength{\unitlength}{1184sp}\begingroup\makeatletter\ifx\SetFigFont\undefined \gdef\SetFigFont#1#2#3#4#5{\reset@font\fontsize{#1}{#2pt}\fontfamily{#3}\fontseries{#4}\fontshape{#5}\selectfont}\fi\endgroup \begin{picture}(6206,5056)(2314,-6062)
\put(6826,-4186){\makebox(0,0)[lb]{\smash{{\SetFigFont{6}{7.2}{\rmdefault}{\mddefault}{\updefault}{\color[rgb]{0,0,0}}}}}}
\put(3826,-4186){\makebox(0,0)[lb]{\smash{{\SetFigFont{6}{7.2}{\rmdefault}{\mddefault}{\updefault}{\color[rgb]{0,0,0}}}}}}
\put(3301,-2611){\makebox(0,0)[lb]{\smash{{\SetFigFont{6}{7.2}{\rmdefault}{\mddefault}{\updefault}{\color[rgb]{0,0,0}}}}}}
\put(5701,-2611){\makebox(0,0)[lb]{\smash{{\SetFigFont{6}{7.2}{\rmdefault}{\mddefault}{\updefault}{\color[rgb]{0,0,0}}}}}}
\put(7501,-2611){\makebox(0,0)[lb]{\smash{{\SetFigFont{6}{7.2}{\rmdefault}{\mddefault}{\updefault}{\color[rgb]{0,0,0}}}}}}
\put(3376,-5611){\makebox(0,0)[lb]{\smash{{\SetFigFont{6}{7.2}{\rmdefault}{\mddefault}{\updefault}{\color[rgb]{0,0,0}}}}}}
\put(2701,-1261){\makebox(0,0)[lb]{\smash{{\SetFigFont{6}{7.2}{\rmdefault}{\mddefault}{\updefault}{\color[rgb]{0,0,0}}}}}}
\end{picture} \hspace*{\fill}
\begin{picture}(0,0)\includegraphics{bad_example_2.pdf}\end{picture}\setlength{\unitlength}{1184sp}\begingroup\makeatletter\ifx\SetFigFont\undefined \gdef\SetFigFont#1#2#3#4#5{\reset@font\fontsize{#1}{#2pt}\fontfamily{#3}\fontseries{#4}\fontshape{#5}\selectfont}\fi\endgroup \begin{picture}(18773,5256)(2311,-11575)
\put(11851,-6736){\makebox(0,0)[lb]{\smash{{\SetFigFont{6}{7.2}{\rmdefault}{\mddefault}{\updefault}{\color[rgb]{0,0,0}}}}}}
\put(2326,-6586){\makebox(0,0)[lb]{\smash{{\SetFigFont{6}{7.2}{\rmdefault}{\mddefault}{\updefault}{\color[rgb]{0,0,0}}}}}}
\put(2701,-7486){\makebox(0,0)[lb]{\smash{{\SetFigFont{6}{7.2}{\rmdefault}{\mddefault}{\updefault}{\color[rgb]{0,0,0}}}}}}
\put(2551,-10861){\makebox(0,0)[lb]{\smash{{\SetFigFont{6}{7.2}{\rmdefault}{\mddefault}{\updefault}{\color[rgb]{0,0,0}}}}}}
\put(6001,-9961){\makebox(0,0)[lb]{\smash{{\SetFigFont{6}{7.2}{\rmdefault}{\mddefault}{\updefault}{\color[rgb]{0,0,0}}}}}}
\put(20814,-7253){\makebox(0,0)[lb]{\smash{{\SetFigFont{6}{7.2}{\rmdefault}{\mddefault}{\updefault}{\color[rgb]{0,0,0}}}}}}
\put(3901,-11461){\makebox(0,0)[lb]{\smash{{\SetFigFont{6}{7.2}{\rmdefault}{\mddefault}{\updefault}{\color[rgb]{0,0,0}}}}}}
\end{picture} \hspace*{\fill}
\caption{A flat visibility representation for which the corresponding
straight-line drawing has exponential width.  Vertices are numbered
in the order in which they are processed.  }
\label{fig:bad_example}
\end{figure}

\begin{lemma}
For , the width of the drawing obtained with 
Theorem~\ref{thm:VR_SL} is .
\end{lemma}
\begin{proof}
Define two recursive functions as , 
 and  for . 
We will show that for , 
any point  of the drawing 
has -coordinate at most , and if 
is not on the first or last row, then it has
-coordinate at most .
Observe that 
and ; hence  as desired.

For the base case, we have two cases.  If two of 
are in different rows, then two of  have
-coordinate 0 and the third has -coordinate at most 1, hence
the claim holds for  since .  If 
are all in the same row, then they have -coordinates 0,1,2.  Vertex
 is either also placed in this row (and then has -coordinate 3)
or it is in a different row (and then has -coordinate 0.)  Either way,
all points in  then have -coordinate at most
.  This shows the base case.

For the induction step, we distinguish cases on how the -coordinate 
of  was determined:
\begin{itemize}
\item Assume first that  was determined via , where  is the -coordinate of some point  in 
 in the row of .  By induction we know that
, and so 
by .
If  is not in the first or last row, then neither is , so
 and .

\item Assume now that , where 
is the -coordinate of the intersection of the row of  with
a line  through some predecessor 
of  and some point  of drawing .
Assume as in Figure~\ref{fig:transform} that
; the other case is
similar.  
This implies  as well, otherwise  would
not have been an obstruction for the edge .  

If , then , therefore
 satisfies the bound as in the first case.
If  is in the bottommost row, then , therefore
 satisfies the bound as in the first case.
Finally assume  and  is not in the bottommost
row, hence .
Now

If  then .  Otherwise  is in the
bottommost row (and  in the top row), and 
 as desired.
\end{itemize}
\end{proof}

We note here that  is needed only for ; much the same proof shows that the height is 
for .

\section{-monotone flat orthogonal drawings to flat visibility 
representations}


\begin{theorem}
\label{thm:o2VR}
\label{th:o2VR}
\label{thm:OD_VR}
Any flat -monotone orthogonal drawing  can be transformed into a 
flat visibility representation  with the same -coordinates
and the same left-to-right orders in each row.
\end{theorem}
\begin{proof}
First, expand every
vertex  to the left and right until it covers all bends (if any)
of edges that attach horizontally at .  Since  has height 1,
there is at most one edge  each on the left and right side of ,
and the expansion of  covers only space previously used by ,
hence creates no overlap.

Now we arrive at a drawing where all edges that have bends attach
vertically at their endpoints.  
Let  be an edge with bends (if there
is none we are done.)  Since  is drawn -monotone, it attaches
at the top of one endpoint and the bottom of the other endpoint,
and the only way it can have bends is to have a right turn followed
by a left turn or vice versa.  Thus,  has a ``zig-zag''.  It is
well known that such a zig-zag can be removed by transforming the
drawing as follows (see also Figure~\ref{fig:remove_Z}):  
Extend the ends of the zig-zag upward and downward
to infinity, and then shift the two sides of the resulting separation
apart until the two rays of the zig-zag align.  See for example
\cite{BLPS09} for details.  This operation adds width, but no height.
Applying this to all edges that have bends gives A
visibility representation.   
\end{proof}

\begin{figure}[ht]
\hspace*{\fill}
\begin{picture}(0,0)\includegraphics{remove_Z.pdf}\end{picture}\setlength{\unitlength}{1579sp}\begingroup\makeatletter\ifx\SetFigFont\undefined \gdef\SetFigFont#1#2#3#4#5{\reset@font\fontsize{#1}{#2pt}\fontfamily{#3}\fontseries{#4}\fontshape{#5}\selectfont}\fi\endgroup \begin{picture}(14724,3044)(2089,-5483)
\end{picture} \hspace*{\fill}
\caption{Removing a zig-zag by shifting parts of the drawing rightwards.}
\label{fig:remove_Z}
\end{figure}


\subsection{Width considerations}

Our construction may increase the width quite a bit, but mostly with
columns that are {\em redundant}: They contain neither a vertical
edge, nor are they the only column of a vertex.  A natural post-processing
step is to remove redundant columns.  We then obtain small width.
In fact, the width is small for {\em any} visibility representation.

\begin{lemma}
Any visibility representation of a connected graph has width at most  after deleting redundant columns.
\end{lemma}
\begin{proof}
Let  and  be the number of edges drawn horizontally and vertically.
Let  be the vertices without incident vertical edge.
Then the width is at most .   This shows the claim if .
If , then let  be a vertex not in , 
pick an arbitrary spanning tree  and root it at .
For any vertex , the edge from  to its parent in 
must be horizontal by definition of .  Hence there are
at least  horizontal edges,
and the width is at most .
\end{proof}

\section{Poly-line drawing to flat orthogonal drawing}
\label{se:SL_VR}


\begin{theorem}
\label{thm:PL_OD}
\label{th:PL2OD}
Any poly-line drawing  can be transformed into a 
flat orthogonal drawing  with the same -coordinates
and the same left-to-right orders in each row.
 is -monotone if  was.
\end{theorem}
\begin{proof}
We first transform  into an  {\em -layer drawing}, i.e.,
a straight-line drawing where all edges are horizontal or connect
adjacent rows.  We do this by
inserting {\em pseudo-vertices} (i.e., subdivide edges) at bends and
whenever a segment of an edge crosses a row.  (We 
allow non-integral -coordinates for pseudo-vertices.)

For each row , let  be the vertices (including
pseudo-vertices) in  in left-to-right order.  In ,
replace each  by a box of width 
, where 
and  are the number of neighbours of  with
larger/smaller -coordinate.  Place these boxes in row  in
the same left-to-right order.

Each horizontal edge is drawn horizontally in  as well.
Each non-horizontal edge connects
two adjacent rows since we inserted pseudo-vertices.  
Connect the edges between two adjacent rows using VLSI channel routing
(see e.g.~\cite{Len90}), using two bends per edge and lots of new rows
(with non-integer -coordinates) that contain
horizontal edge segments and nothing else.

\begin{figure}[ht]
\hspace*{\fill}
\begin{picture}(0,0)\includegraphics{PL2OD.pdf}\end{picture}\setlength{\unitlength}{1579sp}\begingroup\makeatletter\ifx\SetFigFont\undefined \gdef\SetFigFont#1#2#3#4#5{\reset@font\fontsize{#1}{#2pt}\fontfamily{#3}\fontseries{#4}\fontshape{#5}\selectfont}\fi\endgroup \begin{picture}(11195,2499)(2318,-5173)
\end{picture} \hspace*{\fill}
\caption{Converting a poly-line drawing to an orthogonal drawing.
Pseudo-vertices are white.}
\label{fig:PL2OD}
\end{figure}

Now bends only occur at zig-zags; remove these as in the proof of 
Theorem~\ref{thm:o2VR}.  This empties all rows except those
with integer coordinates and gives the desired height and
a flat visibility representation of the graph with pseudo-vertices.
Any pseudo-vertex can now be removed and replaced by a bend if
needed.
\end{proof}

Note that the visibility representation obtained as part of the
proof has width at most , where  is the number
of pseudo-vertices inserted.  An even better bound can be obtained
by observing that any pseudo-vertex that is not at a bend in   
will receive two incident vertical segments and hence can be removed
in the visibility representation.
So the width is 
at most , where  is the number of bends
in .

\section{Flat orthogonal drawings to poly-line drawings}
\label{se:VR_PL}


Combining the previous theorems, it is easy to see that
any flat orthogonal drawing can be converted to a poly-line
drawing of the same height:  First convert it to a visibility
representation, then convert it to a straight-line drawing,
and then interpret the result as a poly-line drawing.
However, since this involves Theorem~\ref{thm:VR_SL}, the
width might grow exponentially.  We now give a simple direct
proof of this transformation that shows that it can be done
while keeping the width small.

\begin{theorem}
Any flat orthogonal drawing  can be transformed into a 
poly-line drawing  with the same -coordinates
and the same left-to-right orders in each row.
Moreover,  has no more width than ,
and it is -monotone if  was.
\end{theorem}
\begin{proof}
First subdivide edges at all bends and all vertical edge-segments 
with pseudo-vertices so that
any vertical edge connects two vertices in adjacent layers.  
Apply the algorithm in Theorem~\ref{thm:VR2SL} to find a straight-line drawing 
of the resulting graph; 
removing the pseudo-vertices then gives the desired poly-line drawing.
All properties are easily verified, except for the width.  Observe
that when applying the construction of Theorem~\ref{thm:VR2SL},
for any vertex  the predecessors are in the same or in adjacent
rows.  Hence all lines from  to predecessors are unobstructed,
and  can simply be placed in the leftmost free position of its row.
Hence the width of  is the maximal number of vertices or 
pseudo-vertices in a row, which is no more than the width
of  since  is orthogonal.
\end{proof}

\section{Applications}


We give a few applications of the results in this paper.

\subsection{Drawing graphs with small height}

The {\em pathwidth}  of a graph  is a graph
parameter that is related to heights of planar graph drawing:
any planar graph that has a straight-line drawing of height
 has pathwidth at least  \cite{FLW03}.  But not all graphs
with pathwidth  have a drawing of height .  Our transformations
show that such heights do exist for outer-planar graphs:

\begin{corollary}
Any outer-planar graph  has a straight-line drawing of height
.
\end{corollary}
\begin{proof}
By a result of Babu et al.~\cite{BBC+12}, we can add edges to 
to obtain a maximal outerplanar graph  with pathwidth in .
In particular,  is 2-connected and hence by 
\cite{Bie-WAOA12} it has a flat visibility representation of
height at most .  By Theorem~\ref{thm:VR_SL}, therefore
 (and with it ) has a straight-line drawing of height .
\end{proof}

Recall that outerplanar graphs have constant treewidth and hence
pathwidth , so any outerplanar graph has a straight-line
drawing of height .  

In a similar fashion, any graph drawing algorithm that produces
drawings of small height in one of our models produces, with our
transformations, graph drawings of small heights in all other models.
We give one more example:

\begin{corollary}
Any 4-connected planar graph has a visibility representation
of height at most .
\end{corollary}
\begin{proof}
It is known that any 4-connected planar graph has
a straight-line drawing 
where the sum of the width and height is at most  \cite{MNN06}.
Therefore, after possible rotation, the height is at most
,
and with 
Theorem~\ref{thm:OD_VR} and \ref{thm:PL_OD} 
we get a flat visibility representation of height .
\end{proof}

The best previous bound on the height of visibility representations
of 4-connected planar graphs was 
\cite{HWZ12}.

\subsection{Integer programming formulations}

In a recent paper, we developed integer program (IP) formulations
for many graph drawing problems where vertices and edges are
represented by axis-aligned boxes \cite{Bie-GD13}.  By adding
some constraints, one can force that edges degenerate to line segments
and vertices to horizontal line segments.  In particular, it is easy to 
create an IP that expresses `` is drawn as a flat visibility
representation'', using  variables and constraints.
With the transformations given in this paper, we can then use IPs
for many other graph drawing problems. 
The following result (based on Theorem~\ref{thm:VR_SL}, 
\ref{thm:OD_VR}, 
\ref{thm:PL_OD}) is crucial: 

\begin{corollary}
A planar graph  has a planar straight-line drawing of height 
if and only if it has a flat visibility representation of height .
\end{corollary}

It is very easy to encode the height in the IP formulations
of \cite{Bie-GD13}. 
We therefore have:

\begin{corollary}
There exists an integer program with  variables and constraints
to find the minimum height of a planar straight-line drawing of a graph .
\end{corollary}



A directed acyclic graph has an
{\em upward drawing} if it has a planar straight-line drawing such
that for any directed edge  the -coordinate of 
is smaller than the -coordinate of .  Testing whether a graph
has an upward drawing is NP-hard \cite{GT01}.  There exists a way
to formulate `` has an upward drawing'' as either IP
or as a Satisfiability-problem, using partial orders on
the edges and vertices \cite{CZ-GD12}.  Our transformations give
a different way of testing this via IP:

\begin{lemma}
A directed acyclic graph has an upward planar drawing if and only
if it has a visibility representation where all edges are vertical
lines, with the head above the tail.
\end{lemma}
\begin{proof}
Given a straight-line upward drawing, we can transform it into the
desired visibility representation using Theorems~\ref{thm:OD_VR}
and \ref{thm:PL_OD}.
Since -coordinates are unchanged, any edge is necessarily drawn
vertical with the head above the tail.  Vice versa, given such a visibility
representation, we can transform it into a flat visibility representation
simply by replacing boxes of positive height by horizontal segments; this
leads to no conflict since there are no horizontal edges.  Then apply
Theorem~\ref{thm:VR_SL}; this gives an upward drawing since -coordinates
are unchanged.
\end{proof}

It is easy to express ``edge  must be drawn vertically,
with the head above the tail'' as constraints in the IP 
for visibility representations defined in \cite{Bie-GD13}.  We therefore
have:

\begin{corollary}
There exists an integer program with  variables and
constraints to test whether a planar graph has an upward planar drawing.
Moreoever, the same integer program also finds the minimum-height
upward drawing.
\end{corollary}

\iffalse
\subsubsection{Simultaneous drawings}

Two planar graphs  and  on the same vertex set  are said
to have a {\em simultaneous drawing} if there exist planar straight-line
drawings of  and  such that any vertex is at the same point
in both drawings.

In a similar spirit, one could say that  and  have a
{\em simultaneous visibility representation} if there exist
visibility representations of  and  such that any vertex 
is represented by the same box in both drawings.

With some minor modifications to our transformations, we can prove
that the the two concept are related: two planar graphs have simultaneous
planar drawing if and only if they have a simultaneous flat visibility
representation.

\begin{lemma}
\label{lem:VR_SL_simultaneous}
If  and  have a simultaneous flat visibility representation,
then they have a simultaneous planar drawing.  
Moreover, transforming
one set of drawings into the other preserves -coordinates and
left-to-right order.
\end{lemma}
\begin{proof}
Let  and  be the simultaneous flat visibility
representations of  and .  We now apply the transformation
in the proof of Theorem~\ref{lem:VR_SL} to both drawings at the
same time.  Recall that we processed vertices by increasing
-coordinates; this is the same order for both  and
 since vertices have the same boxes in both.  Now
when processing vertex , we obtained some lower bounds 
on the -coordinate of  from the visibility representation.
No upper bounds existed on the -coordinates.  So if we use 
the lower bounds imposed by both  and , we
can find an -coordinate for  that is suitable for both
of them.  Using this for the new drawings  and ,
we obtain planar straight-line drawings with the same set of
coordinates for vertices in both drawings, viz., a simultaneous
planar drawing.
\end{proof}

\begin{lemma}
\label{lem:SL_VR_simultaneous}
If  and  have simultaneous planar drawings  and ,
then they have simultaneous flat visibility representations 
and .
Moreover, transforming
one set of drawings into the other preserves -coordinates and
left-to-right order.
\end{lemma}
\begin{proof}
Let  and  be the flat visibility drawings obtained
by transforming  and  using Theorem~\ref{thm:PL_VR}.
This preserves -coordinates and left-to-right order.  Since the
visibility representations are flat, all we have to do is hence to
????
\end{proof}


Since having a flat visibility representation is easily expressed
as in integer program \cite{Bie-GD13}, we hence have:

\begin{corollary}
There exists an integer program with  variables and
constraints to test whether two planar graphs have a simultaneous
planar drawing.
\end{corollary}
\fi


\section{Conclusion and open problems}
\label{se:open}


In this paper, we studied transformations between different types
of graph drawings, in particular between straight-line drawings
and flat visibility representations.  We demonstrated applications
of these results, especially for drawings of small heights, and
upward drawings.

We have not been able to create transformations that start with an
arbitrary (i.e., not necessarily flat) visibility representation and
turn it into a straight-line drawing of approximately the same height.
Does such a transformation exist?


Another open problem concerns the width, especially for the transformation
from flat visibility representations to planar straight-line drawings.
Is it possible to make the width polynomial if we may change the
-coordinates while keeping the height asymptotically the same?

\section*{Acknowledgments}


Research partially supported by NSERC and by the Ross and Muriel
Cheriton Fellowship.  Some of the results in this paper appeared
in \cite{Bie-WAOA12}.

\bibliographystyle{plain}
\bibliography{../../bib/full,../../bib/papers,../../bib/gd}

\end{document}
