
















\documentclass[twocolumn]{autart}    


\usepackage{graphicx}          



\graphicspath{{img/}}

\usepackage{amssymb}
\usepackage{amsmath}
\usepackage{natbib}

\newtheorem{theorem}{Theorem}
\newtheorem{lemma}{Lemma}
\newtheorem{proposition}{Proposition}
\newtheorem{definition}{Definition}
\newtheorem{remark}{Remark}

\date{8 September 2016}


\begin{document}

\begin{frontmatter}


\title{Contraction analysis of switched systems via regularization\thanksref{footnoteinfo}} 

\thanks[footnoteinfo]{This paper was not presented at any IFAC 
meeting. Corresponding author Mario di Bernardo. Tel. +39-081-7683909. Fax +39-081-7683186.}

\author[Napoli]{Davide Fiore}\ead{davide.fiore@unina.it},             
\author[Bristol]{S. John Hogan}\ead{s.j.hogan@bristol.ac.uk},  
\author[Napoli,Bristol]{Mario di Bernardo}\ead{mario.dibernardo@unina.it}    


\address[Napoli]{Department of Electrical Engineering and Information Technology, University of Naples Federico II, Via Claudio 21, 80125 Naples, Italy}  
\address[Bristol]{Department of Engineering Mathematics, University of Bristol, BS8 1TR Bristol, U.K.}


\begin{keyword}                         
contraction theory \sep incremental stability \sep switching controllers \sep regularization   
\end{keyword}            


\begin{abstract}                          We study incremental stability and convergence of switched (bimodal) Filippov systems via contraction analysis. In particular,  by using results on regularization of switched dynamical systems, we derive sufficient conditions for  convergence of any two trajectories of the Filippov system between each other within some region of interest. We then apply these conditions to the study of different classes of Filippov systems including piecewise smooth (PWS) systems, piecewise affine (PWA) systems and relay feedback systems. We show that contrary to previous approaches, our conditions allow the system to be studied in metrics other than the Euclidean norm. The theoretical results are illustrated by numerical simulations on a set of representative examples that confirm their effectiveness and ease of application.
\end{abstract}

\end{frontmatter}

\section{Introduction}
\label{sec:intro}
Incremental stability has been established as a powerful tool to prove convergence in nonlinear dynamical systems \citep{angeli2002lyapunov}. It characterizes asymptotic convergence of trajectories with respect to one another rather than towards some attractor known a priori. Several approaches to derive sufficient conditions for a system to be incrementally stable have been presented in the literature \citep{angeli2002lyapunov,russo2010global,lohmiller1998contraction,forni2014differential,pavlov2006uniform}.

A particularly interesting and effective approach to obtain sufficient conditions for incremental stability of nonlinear systems comes from contraction theory \citep{lohmiller1998contraction,jouffroy2005some,aminzare2014contraction}. A nonlinear system is said to be {\it contracting} if initial conditions or temporary state perturbations are forgotten exponentially fast, implying convergence of system trajectories towards each other and consequently towards a steady-state solution which is determined only by the input (the \emph{entrainment} property, e.g. \citet{russo2010global}). A vector field can be shown to be contracting over a given -reachable set by checking the uniform negativity of some matrix measure  of its Jacobian matrix in that set \citep{russo2010global}. Classical contraction analysis requires the system vector field to be continuously differentiable.

In this paper, we consider an important class of non-differentiable vector fields known as \emph{piecewise smooth} (PWS) \emph{systems} \citep{filippov1988differential}. A PWS system consists of a finite set of ordinary differential equations

where the smooth vector fields , defined on disjoint open regions , are smoothly extendable to the closure of . The regions  are separated by a set  of codimension one called the \textit{switching manifold}, which consists of finitely many smooth manifolds intersecting transversely.  The union of  and all  covers the whole state space .

Piecewise smooth systems are of great significance in applications, ranging from problems in mechanics (friction, impact) and biology (genetic regulatory networks) to variable structure systems in control engineering (sliding mode control \citep{utkin2013sliding}) -- for an overview see the monograph by \citep{bernardo2008piecewise}.

The theoretical study of PWS systems is important. Firstly, the classical notion of solution is challenged in at least two distinct ways. When the normal components of the vector fields either side of  are in the \textit{same} direction, the gradient of a trajectory is discontinuous, leading to Carath\'eodory solutions \citep{filippov1988differential}. In this case, the dynamics is described as \textit{crossing} or \textit{sewing}. But when the normal components of the vector fields on either side of  are in the \textit{opposite} direction, a vector field on  needs to be defined. The precise choice is not unique and depends on the nature of the problem under consideration. One possibility is the use of differential inclusions. Another choice is to adopt the Filippov convention \citep{filippov1988differential}, where a \textit{sliding} vector field  is defined on . In this case, the dynamics is described as \textit{sliding}.

Some results have been presented in the literature to extend contraction analysis to non-differentiable vector fields. An extension to piecewise smooth continuous (PWSC) systems was outlined in \citep{lohmiller2000nonlinear} and formalized in \citep{di2014contraction}. Contracting hybrid systems were analysed in \citep{lohmiller2000nonlinear} while the stability analysis of hybrid limit cycles using contraction was presented in \citep{tang2014transverse}. An extension of contraction theory, related to the concept of \emph{weak} contraction \citep{sontag2014three}, to characterize incremental stability of sliding mode solutions of planar Filippov systems was first presented in \citep{di2013incremental} and later extended to -dimensional Filippov systems in \citep{di2014incremental}. Finally, incremental stability properties of piecewise affine (PWA) systems were discussed in \citep{pavlov2007convergence} in terms of \emph{convergence}, a stability property related to contraction theory \citep{pavlov2004convergent}.

In this paper, we take a different approach to the study of contraction in -dimensional Filippov systems than the one taken in \citep{di2013incremental,di2014incremental}. In those papers, the sliding vector field  was assumed to be defined everywhere and then the contraction properties of its projection onto the switching manifold was considered (together with a suitable change of coordinates). In the current paper, we adopt a new generic approach which directly uses the vector fields  and does not need the explicit computation of the sliding vector field . Our method has a simple geometric meaning and, unlike other methods, can also be applied to nonlinear PWS systems. 


Instead of directly analysing the Filippov system, we first consider a {\it regularized} version; one where the switching manifold  has been replaced by a boundary layer of width . We choose the regularization method of Sotomayor and Teixeira \citep{sotomayor1996regularization}. We then apply standard contraction theory results to this new system, before taking the limit  in order to recover results that are valid for our Filippov system.
\section{Mathematical preliminaries and background}
\label{sec:background}
\subsection{Matrix measures}
\label{app:matrix_measure}
Given a real matrix  and a norm  with its induced matrix norm , the associated \emph{matrix measure} (also called logarithmic norm \citep{dahlquist1958stability,lozinskii1958error,strom1975logarithmic}) is the function  defined as 

where  denotes the identity matrix.
The following matrix measures associated to the norm for  are often used

The matrix measure  has the following useful properties \citep{vidyasagar2002nonlinear,desoer1972measure}:
\begin{enumerate}
\item
\label{app:mu_prop2}
, .
\item
\label{app:mu_prop3}
If , where  denotes a matrix with all entries equal to zero, then .
\item
\label{app:mu_prop4}
 for all , where  denotes the real part of the eigenvalue  of .
\item
\label{app:mu_prop5}
 for all  (positive homogeneity).
\item
\label{app:mu_prop6}
 (subadditivity).
\item
\label{app:mu_prop7}
Given a constant nonsingular matrix , the matrix measure  induced by the weighted vector norm  is equal to . 
\end{enumerate}

The following theorem can be proved \citep{vidyasagar1978matrix,aminzare2014contraction}.
\begin{theorem}
\label{app:thm:weighted_mu}
There exists a positive definite matrix  such that  if and only if , with .
\end{theorem}
We now present results on the properties of matrix measures of rank-1 matrices, since we will need these in the sequel. We believe that Lemma \ref{lemma:measure_rank1} is an original result. For any two vectors , , the matrix  has always rank equal to 1. This can be easily proved observing that .
\begin{proposition}
For any two vectors ,  and for any norm we have that .
\end{proposition}
\begin{pf*}{Proof.}
The proof follows from property \ref{app:mu_prop4} of matrix measures as listed above, that is, for any matrix and any norm , for all , where  denotes the real part of the eigenvalues  of . Therefore, since a rank-1 matrix has  zero eigenvalues its measure cannot be less than zero.
\end{pf*}
The following important result holds for the measure of rank-1 matrices induced by Euclidean norms.
\begin{lem}
\label{lemma:measure_rank1}
Consider the Euclidean norm , with  and . For any two vectors , , the following result holds

otherwise .
\end{lem}
\begin{pf*}{Proof.}
Firstly we prove that  if and only if  and  are antiparallel, i.e.  for some . Indeed, from the definition of Euclidean matrix measure,  is equal to the maximum eigenvalue of the symmetric part  of the matrix .  The characteristic polynomial  of  is \citep[Fact 4.9.16]{bernstein2009matrix}

This polynomial has always  zero roots and (in general) two further real roots. It can be easily seen from Descartes' rule that their signs must be opposite. Therefore, the only possibility for them to be nonpositive is that one must be zero while the other is negative. Using again Descartes' rule, this obviously happens if and only if  and  are antiparallel.


Now, assume that  then, using property \ref{app:mu_prop7} of matrix measures, we have

and, from the result proved above,  and  must be antiparallel, i.e.  for some , or equivalently .

To prove sufficiency, suppose that , , then  and therefore, using again the result above, we have

\end{pf*}
Note that when  or  (or both) are equal to 0 then by property \ref{app:mu_prop3} of matrix measures .
\subsection{Incremental stability and contraction theory}
\label{sec:incr_stab}
Before starting our analysis for PWS systems, we present some key results on the contraction properties of smooth systems. Let  be an open set. Consider the system of ordinary differential equations

where  is a continuously differentiable vector field defined for  and , that is .
We denote by  the value of the solution  at time  of \eqref{eq:dynamical_sys} with initial value .
We say that a set  is \emph{forward invariant} for system \eqref{eq:dynamical_sys}, if  implies  for all .
\begin{definition}
\label{def:incr_stability}
Let  be a forward invariant set and  some norm in . System \eqref{eq:dynamical_sys} is said to be \emph{incrementally exponentially stable} in  if for any two solutions  and  there exist constants  and  such that 

\end{definition}
Results in contraction theory can be applied to a quite general class of subsets , known as -reachable subsets \citep{russo2010global}.
\begin{definition}
Let  be any positive real number. A subset  is \emph{-reachable} if, for any two points  and  in  there is some continuously differentiable curve  such that ,  and .
\end{definition}
For convex sets , we may pick , so  and we can take . Thus, convex sets are 1-reachable, and it is easy to show that the converse holds.

The main result of contraction theory for smooth systems is as follows \cite{lohmiller1998contraction,russo2010global}.
\begin{theorem}
\label{thm:contraction}
Let  be a forward-invariant -reachable set. If there exists some norm in , with associated matrix measure , such that, for some constant  (the \emph{contraction rate})

that is, the vector field \eqref{eq:dynamical_sys} is \emph{contracting} in , then system \eqref{eq:dynamical_sys} is incrementally exponentially stable in  with convergence rate .
\end{theorem}
As a result, if a system is contracting in a (bounded) forward invariant subset then it converges towards an equilibrium point therein \citep{russo2010global,lohmiller1998contraction}.

In this paper we analyse contraction properties of dynamical systems based on norms and matrix measures. Other more general definitions exist in the literature, for example results based on Riemannian metrics \citep{lohmiller1998contraction} and Finsler-Lyapunov functions \citep{forni2014differential}. The relations between these three definitions and the definition of convergence \citep{pavlov2004convergent} have been investigated in \citep{forni2014differential}.
\subsection{Filippov systems}
Switched (or bimodal) Filippov systems are dynamical systems  where  is a piecewise continuous vector field having a codimension one submanifold  as its discontinuity set and defined as 

where . The vector field  can be multivalued at the points of .
The submanifold  is defined as the zero set of a smooth function , that is

where  is a regular value of , i.e. 

 is called the {\it switching manifold}. It divides  in two disjoint regions,  and .
We distinguish the following regions on :
\begin{enumerate}
\item
The \emph{crossing region} is ;
\item
The \emph{sliding region} is ;
\item
The \emph{escaping region} is ;
\end{enumerate}
where  is the \emph{Lie derivative} of  with respect to the vector field , that is the component of  normal to the switching manifold at the point . In the sliding region we adopt the widely used Filippov convention \citep{filippov1988differential}. We define a \emph{sliding vector field} , which is the convex combination of  and  that is tangent to , given for  by 

with  such that . 
\begin{remark}
In the following we assume that solutions of systems \eqref{eq:filippov_bimodal} and \eqref{eq:sliding} are defined in the sense of  Filippov and  that for \eqref{eq:filippov_bimodal} \emph{right uniqueness} \citep[pag. 106]{filippov1988differential} holds in . Therefore, the escaping region is excluded from our analysis. 
\end{remark}
There are a few results on the incremental stability of piecewise smooth systems; notably for piecewise affine (PWA) systems and piecewise smooth continuous (PWSC) systems.
\begin{definition}[PWA systems]
A bimodal PWA system is a system of the form

where , , and , , , , are constant matrices and vectors, respectively.
\end{definition}
\begin{theorem}
\label{thm:pwa_pavlov}
\citep{pavlov2007convergence} System \eqref{eq:pwa_sys} is incrementally exponentially stable if there exist a positive definite matrix , a number  and a vector  such that
\begin{enumerate}
\item
\label{eq:thm:pavlov:1}
,
\item
\label{eq:thm:pavlov:2}
,
\item
\label{eq:thm:pavlov:3}
,
\end{enumerate}
where  and .
\end{theorem}
\begin{remark}
The first condition requires the existence of a common Lyapunov function  for the two modes. The second condition assumes that the linear part of the two modes is continuous on the switching plane. There are two cases in the third condition  \citep[see][Remark 4]{pavlov2007convergence}. For , the PWA system \eqref{eq:pwa_sys} is continuous. For , the discontinuity is due only to the  and, together with the first condition, implies that the two modes of the PWA system \eqref{eq:pwa_sys} are simultaneously strictly passive. 
\end{remark}
The original theorem in \citep[Theorem 2]{pavlov2007convergence}  is stated in terms of convergence instead of incremental stability. These two notions are proved to be equivalent on a compact set in \citep{ruffer2013convergent}.
\begin{definition}[PWSC systems]
\label{def:PWSC}
The piecewise smooth system \eqref{eq:pws} is said to be continuous (PWSC) if the following conditions hold:
\begin{enumerate}
\item
it is continuous for all  and for all 
\item
the function  is continuously differentiable for all , for all  and for all . Furthermore the Jacobian  can be continuously extended on the boundary .
\end{enumerate}
\end{definition}
\begin{theorem}
\label{thm:contracting_PWSC}
\citep{di2014contraction} Let  be a forward-invariant -reachable set. Consider a PWSC system such that it fulfills conditions for the existence and uniqueness of a Carath\'eodory solution. If there exists a unique matrix measure such that for some positive constants 

for all , for all  and for all , then the system is incrementally exponentially stable in  with convergence rate .
\end{theorem}
A similar result using Euclidean norms was previously presented in \citep[Theorem 2.33]{pavlov2006uniform} in terms of convergent systems. An extension of Theorem \ref{thm:contracting_PWSC} to the case where multiple norms are used was presented in \citep{lu2015switching,lu2015contraction}.
\subsection{Regularization}
Our approach to contraction analysis of Filippov systems is via regularization. There are several ways to regularize system \eqref{eq:filippov_bimodal}. We shall adopt the method due to Sotomayor and Teixeira \citep{sotomayor1996regularization}, where a smooth approximation of the discontinuous vector field is obtained by means of a transition function.
\begin{definition}
A PWSC function  is a \emph{transition function} if

and  within .
\end{definition}
\begin{definition}
The \emph{-regularization} of a bimodal Filippov system \eqref{eq:filippov_bimodal} is the one-parameter family of PWSC functions   given for  by

\end{definition}
The \emph{region of regularization} where this process occurs is

Note that outside  the regularized vector field  coincides with the PWS dynamics, i.e.

A graphical representation of the different regions of the state space of the regularized vector field  is depicted in Figure~\ref{fig:regions}.
\begin{figure}[!t]
\begin{center}
\includegraphics[width=0.9\linewidth]{regions}
\caption{Regions of state space: the switching manifold , 
,  (hatched zone) and  (grey zone).}
\label{fig:regions}
\end{center}
\end{figure}


Sotomayor and Teixeira showed that the sliding vector field  can be obtained as a limit of the regularized system in the plane. For , a similar result was given in \citep[Theorem 1.1]{llibre_sliding_2008}. Here we recover their results directly via the theory of slow-fast systems \citep{kuehn2015book} as follows.
\begin{lem}
\label{thm:regularization}
Consider  in \eqref{eq:filippov_bimodal} with  and its regularization  in \eqref{eq:regularized_sys}. If for any  we have that  or  then there exists a singular perturbation problem such that fixed points of the boundary-layer model are critical manifolds, on which the motion of the slow variables is described by the reduced problem, which coincides with the sliding equations \eqref{eq:sliding}.\\
Furthermore, denoting by  a solution of the regularized system and by  a solution of the discontinuous system with the same initial conditions , then

uniformly for all  and for all .
\end{lem}
\begin{pf*}{Proof.}
For the sake of clarity, we assume without loss of generality that  can be represented, through a local change of coordinates around a point , by the function . We use the same notation for both coordinates. Hence our regularized system \eqref{eq:regularized_sys} becomes

We now write \eqref{eq:regularized_sys_2} as a slow-fast system. Let , so that the region of regularization becomes , and  for . Then \eqref{eq:regularized_sys_2} can be written as

for , where . The variable  is the {\it fast variable} and the variables  for  are the {\it slow variables}.
When , we have

for , obtaining the so-called \emph{reduced problem}. From the hypotheses we know that  or , hence we can solve for  from the first equation

that substituted into the second equation in \eqref{eq:reduced_problem} gives

If we now rescale time  and write , then \eqref{eq:slow_system} becomes

for . The limit  of \eqref{eq:fast_system}

is called the {\it boundary-layer model}. Its fixed points can be obtained by applying the Implicit Function Theorem to , that gives , since  for  by definition. This in turn implies that .


It now follows directly that the flow of the reduced problem on critical manifolds of the boundary-layer problem coincides with that of the sliding vector field  as in \eqref{eq:sliding} when the same change of coordinates as in the beginning is considered, i.e. such that . In fact, after some algebra we get

that coincides with \eqref{eq:reduced_problem_2}.

Furthermore, it is a well known fact in singular perturbation problems \citep[Theorem 11.1]{khalil1996nonlinear} that, starting from the same initial conditions, the error between solutions   of the slow system \eqref{eq:slow_system} and solutions of its reduced problem (that, as said, coincide with solutions  of the sliding vector field) is  after some  when the fast variable  has reached a  neighborhood of the slow manifold, i.e. . However, in our case the singular perturbation problem is defined only in  where any point therein is distant from the slow manifold at most , therefore the previous estimate is defined uniformly for all  and in any norm due to their equivalence in finite dimensional spaces.
On the other hand, from \eqref{eq:f_eps_outside} outside  the regularized vector field is equal to the discontinuous vector field and therefore the error between their solutions is uniformly 0.
\end{pf*}
\section{Contracting Filippov systems}
\label{sec:filippov}
In this section we present our two main results, Theorems \ref{thm:contraction_filippov} and \ref{thm:contracting_regularized}, for switched Filippov systems. Theorem \ref{thm:contraction_filippov}, using Lemma \ref{thm:regularization}, shows that if the regularized system  is incrementally exponentially stable so it is the Filippov system from which it is derived. 
Theorem \ref{thm:contracting_regularized} then gives sufficient conditions for the discontinuous vector field to be incrementally exponentially stable.
\begin{theorem}
\label{thm:contraction_filippov}
Let  be a forward-invariant -reachable set. If there exists a positive constant  such that for all  the regularized vector field  \eqref{eq:regularized_sys} is incrementally exponentially stable in  with convergence rate , then in the limit for  any two solutions  and , with , of the bimodal Filippov system \eqref{eq:filippov_bimodal} converge towards each other in , i.e.

\end{theorem}
\begin{pf*}{Proof.}
From Lemma \ref{thm:regularization} we know that the error between any two solutions  and  of the regularized vector field  and their respective limit solutions  and  of the discontinuous system is , i.e.  and , . Therefore, from the hypothesis of  being incrementally exponentially stable, \eqref{eq:ies} holds and applying the triangular inequality of norms we have

for  and for every . 
The theorem is then proved by taking the limit for .
\end{pf*}
If the chosen transition function  is a  function, then the regularized vector field  is  and Theorem \ref{thm:contraction} can be directly applied to study its incremental stability. On the other hand, if the transition function is not  but it is at least a PWSC function as in Definition \ref{def:PWSC}, with ,  and , then the regularized vector field  is itself a PWSC vector field and Theorem \ref{thm:contracting_PWSC} applies. This is the case for . This function is  but its restrictions to each subsets ,  and  are smooth functions. We will use it as an example in the sequel.

Before presenting our next theorem, we first introduce the following lemma.
\begin{lem}
\label{lemma:jacobian}
The Jacobian matrix of the regularized vector field \eqref{eq:regularized_sys} is

where

and ,  and , . Note that for any transition functions , for all .
\end{lem}
\begin{pf*}{Proof.}
The regularized vector field  can be rewritten as

therefore, taking the derivative with respect to , we obtain

Observing that

and

replacing them into \eqref{eq:lemma:jacobian}, we finally obtain \eqref{eq:jacobian_eps}.
\end{pf*}
Note that if  is PWSC then the Jacobian matrix \eqref{eq:jacobian_eps} is a discontinuous function but its restriction to  is continuous.
\begin{theorem}
\label{thm:contracting_regularized}
Let  be a forward-invariant -reachable set. A bimodal Filippov system \eqref{eq:filippov_bimodal} is incrementally exponentially stable in  with convergence rate  if there exists some norm in , with associated matrix measure , such that for some positive constants 

\end{theorem}
\begin{pf*}{Proof.}
The transition function  is a PWSC function hence the resulting regularized vector field  is also PWSC, i.e. it is continuous in all  and such that its restrictions to the subsets ,  and  are continuously differentiable. Therefore Theorem \ref{thm:contracting_PWSC} can be directly applied and we have that  is contracting in  if there exist positive constants  such that

Thus, by Lemma \ref{lemma:jacobian}, substituting \eqref{eq:jacobian_eps} into \eqref{eq:contracting_regularized} and using the subadditivity and positive homogeneity properties of the matrix measures, we obtain

Therefore, conditions \eqref{eq:contracting_mode1}-\eqref{eq:contracting_regularized} are satisfied if

and . Finally, considering that  in the limit for , we obtain conditions \eqref{eq:thm:condition1}-\eqref{eq:thm:condition3}. Therefore, by virtue of Theorem \ref{thm:contraction_filippov}, these conditions are sufficient for the bimodal Filippov system \eqref{eq:filippov_bimodal} to be incrementally exponentially stable.
\end{pf*}
\begin{remark}
If  is  it can be easily proved (by using Lemma \ref{lemma:jacobian} and the subadditivity property of matrix measures) that conditions \eqref{eq:thm:condition_reg_1}-\eqref{eq:thm:condition_reg_3} are sufficient for the measure of the Jacobian of  to be negative definite over the entire region of interest.
\end{remark}
The first two conditions \eqref{eq:thm:condition1} and \eqref{eq:thm:condition2} in Theorem \ref{thm:contracting_regularized} guarantee that the regularized vector field  is contracting outside the region , and therefore imply that any two trajectories in  converge towards each other exponentially. Condition \eqref{eq:thm:condition3} assures that the third term in \eqref{eq:thm:subadditivity} does not diverge as  and therefore that negative definiteness of the measures of the Jacobian matrices of two modes,  and , is enough to guarantee incremental exponential stability of  inside . 

Theorem \ref{thm:contracting_regularized} gives conditions in terms of a generic norm. When a specific norm is chosen, it is possible to further specify the conditions of Theorem \ref{thm:contracting_regularized}, as we now show.
\begin{proposition}
\label{thm:condition3}
Assume that through a local change of coordinates around a point  the switching manifold  is represented by the function  and let . Let , with , be a diagonal matrix and  be a positive definite matrix. Assuming that , then
\begin{enumerate}
\item
 if and only if

\item
 if and only if , .
\item
 if and only if  and  are antiparallel.
\end{enumerate}
\end{proposition}
\begin{pf*}{Proof.}
The matrix  has rank equal to 1 and, since , it can be written as

\begin{enumerate}
\item
From \citep[Lemma 4]{vidyasagar1978matrix} we have

This measure is equal to zero if and only if

\item
The proof for  comes from Lemma \ref{lemma:measure_rank1}.
\item
Again, from \citep[Lemma 4]{vidyasagar1978matrix} we have

The above measure is equal to zero if and only if  and , that is if  is antiparallel to .
\end{enumerate}
\end{pf*}
Hence, using the -norm there always exist a matrix  and a change of coordinates such that the condition holds assuming that the scalar product between  and  is negative, that is . Moreover, using the Euclidean norm a matrix  such that the condition holds exists only if  , , as proved next.
\begin{figure}[!t]
\begin{center}
\includegraphics[width=0.9\linewidth]{geo_interp}
\caption{Geometrical interpretation of condition \eqref{eq:thm:condition3} using Euclidean norm (with ) and -norm in . The horizontal line is . Sliding is represented in a) and b), while crossing occurs in c), d), e), f). In all cases the difference vector field  is antiparallel to .}
\label{fig:geo_interp}
\end{center}
\end{figure}
\begin{proposition}
Assume that  with , then a Euclidean norm , with , such that  exists if and only if .
\end{proposition}
\begin{pf*}{Proof.}
Firstly, note that from Proposition \ref{thm:condition3} and from Lemma \ref{lemma:measure_rank1} we know that  if and only if a matrix  exists such that , . Now, from the definition of positive definite matrices, it follows that given the two nonzero vectors  and  such a positive definite matrix  exists if and only if , that is \footnote{Sufficiency follows directly from the definition of positive definiteness of the matrix .}.
\end{pf*}
Furthermore, note that when , that is when the system is continuous on  as in the case of PWSC systems,  we have that . Therefore condition \eqref{eq:thm:condition3} is always satisfied and Theorem \ref{thm:contracting_regularized} coincides with Theorem \ref{thm:contracting_PWSC}.

In Figure \ref{fig:geo_interp} the geometrical interpretation of condition \eqref{eq:thm:condition3} in  is shown schematically when either the Euclidean norm (with ) or the -norm are used.
One significant advantage of our method is that it can deal with nonlinear PWS systems, as we shall now demonstrate. 
All simulations presented here were computed using the numerical solver in \citep{piiroinen2008event}.


\paragraph*{Example 1}
Consider the PWS system \eqref{eq:filippov_bimodal} with

and . We can easily check that all three conditions of Theorem \ref{thm:contracting_regularized} are satisfied in the -norm. Indeed, for the first condition we have

because . Similarly for the second condition we have

because . Finally, for the third condition we have

Therefore the PWS system considered here is incrementally exponentially stable in all  with convergence rate . In Figure \ref{fig:ex_pws}a we show numerical simulations which confirm the analytical estimation \eqref{eq:thm:ies_contraction_filippov}.
\begin{figure}[!t]
\begin{center}
\includegraphics[width=0.9\linewidth]{ex_pws}
\caption{ {Norm of the difference between two trajectories for (a) Example 1 and (b) Example 2. Initial conditions are respectively ,  and , . The dashed lines represent the analytical estimates \eqref{eq:thm:ies_contraction_filippov} with (a)  and (b) , and .}}
\label{fig:ex_pws}
\end{center}
\end{figure}
\paragraph*{Example 2}
Consider the PWS system \eqref{eq:filippov_bimodal} with
2ex]
x_1-x_2-3
\end{bmatrix}\! ,
f^-(x)\! =\!\!
\begin{bmatrix}
-2x_1+\dfrac{2}{9}x_2^2-\! 2\
and . For the first condition of Theorem \ref{thm:contracting_regularized} we have

Therefore  is contracting in the -norm for . If we want to guarantee a certain contraction rate   we need to consider the subset  instead. An identical result holds for .
Finally, for the third condition of Theorem \ref{thm:contracting_regularized} we have
2ex]
0 & -6
\end{bmatrix}
\right)=\\
=& \max\left\{0;\; -2+\frac{4}{9}x_2^2 \right\}=0
\end{split}

\label{eq:thm:pwa_cond1}
&\mu\left( A_1\right)& \leq -c_1\\
&\mu\left( A_2\right)& \leq -c_2\\
\label{eq:thm:pwa_cond3}
&\mu\left( \Delta Axh^T \right)& = 0\\
\label{eq:thm:pwa_cond4}
&\mu\left( \Delta bh^T \right)& = 0 

\begin{split}
\mu & \left( \Big[ f^+(x)-f^-(x)\Big] \, \nabla H(x) \right)=\\
= & \mu\left( [\Delta Ax+\Delta b] \, h^T \right) 
\leq \mu\left( \Delta Axh^T \right)+\mu\left( \Delta bh^T \right).
\end{split}

\label{eq:pwaexample}
\begin{split}
&A_1=
\begin{bmatrix}
-2 & -1\\
1 & -3
\end{bmatrix},
\quad
b_1=
\begin{bmatrix}
-1\\
-3
\end{bmatrix},\\
&A_2=
\begin{bmatrix}
-2 & -1\\
1 & -4
\end{bmatrix},
\quad
b_2=
\begin{bmatrix}
2\\
4
\end{bmatrix},
\end{split}

\mu_1(\Delta Axh^T)
= \mu_1\left(
\begin{bmatrix}
0 & 0\\
0 & x_2
\end{bmatrix} 
\right)= x_2 =0, \qquad \forall x\in\Sigma.

\begin{split}
\label{eq:relay_sys}
\dot{x}=& \,Ax-b\;\mathrm{sgn}(y)\\
y=& \, c^Tx
\end{split}

\label{eq:thm:relay_cond1}
\mu\left( A\right) \leq -\bar{c}\\
\label{eq:thm:relay_cond2}
\mu\left( -bc^T \right) =0.

\dot{x}=
\begin{cases}
Ax-b \quad & \mathrm{if} \; c^Tx>0\\
Ax+b \quad & \mathrm{if} \; c^Tx<0
\end{cases}

f_\varepsilon(x)=Ax-b\,\varphi\left( \frac{c^Tx}{\varepsilon} \right)

\mathrm{div} (f_\varepsilon(x))=
\begin{cases}
\mathrm{tr}(A)-\frac{1}{\varepsilon}\varphi'\left( \frac{c^Tx}{\varepsilon}\right)c^T b, & \quad \mbox{if } x\in \mathcal{S}_\varepsilon\\
\mathrm{tr}(A), & \quad \mbox{if } x\notin \mathcal{S}_\varepsilon
\end{cases}

\label{eq:relayfeedback}
A=
\begin{bmatrix}
-2 & -1\\
1 & -3
\end{bmatrix},
\quad
b=
\begin{bmatrix}
1\\
3
\end{bmatrix},
\quad
c^T=
\begin{bmatrix}
0 & 1
\end{bmatrix}

f_\varepsilon(x)=
\begin{cases}
Ax-b & \mbox{if } c^Tx>\varepsilon\0.5ex]
Ax+b & \mbox{if } c^Tx<-\varepsilon
\end{cases}

\mu\left(\frac{\partial f_\varepsilon}{\partial x}\right)  \leq \mu(A)+\frac{1}{\varepsilon}\,\mu(-bc^T)=-1.

Therefore the regularized vector field  remains contracting in the -norm for any value of , as should  be expected since conditions of Proposition \ref{thm:contracting_relay} are satisfied in this norm. 
Hence, from Theorem \ref{thm:contraction_filippov} we can conclude that the relay feedback system taken into example is incrementally exponentially stable in the -norm. In Figure \ref{fig:ex_pwa}c, we show numerical simulations of the evolution of the difference between two trajectories for this system. The dashed line is the estimated exponential decay from \eqref{eq:thm:ies_contraction_filippov} with  and . An approach to contraction analysis of switched Filippov systems not requiring the use of regularization is currently under investigation and will be presented elsewhere.
\section{Conclusions}
\label{sec:conclusions}
We presented a methodology to study incremental stability in generic -dimensional switched (bimodal) Filippov systems characterized by the possible presence of sliding mode solutions. The key idea is to obtain conditions for incremental stability of these systems by studying contraction of their regularized counterparts. We showed that the regularized vector field is contracting if a set of hypotheses on its modes are satisfied. In contrast to previous results, our strategy does not require explicit computation of the sliding vector field using Filippov's convex method or Utkin's equivalent control approach. Moreover, different metrics rather than the Euclidean norms can be effectively used to prove convergence. The theoretical results were applied on a set of representative examples including piecewise smooth systems, piecewise affine systems and relay feedback systems. In all cases, it was shown that the conditions we derived are simple to apply and have a clear geometric interpretation. We wish to emphasize that the tools we developed could be instrumental not only to carry out convergence analysis of Filippov systems but also to synthesize switched control actions based on their application \citep{di2015switching}.
\begin{ack}                               
SJH wishes to acknowledge support from the Network of Excellence MASTRI Materiali e Strutture Intelligenti (POR Campania FSE 2007/2013) for funding his visits to the Department of Electrical Engineering and Information Technology of the University of Naples Federico II. DF acknowledges support from the University of Naples Federico II for supporting his visits at the Department of Engineering Mathematics of the University of Bristol, U.K. The authors would like to thank the anonymous reviewers for their comments that led to a significant improvement of the manuscript.
\end{ack}

\bibliographystyle{elsarticle-harv}        \bibliography{refs}           





\end{document}