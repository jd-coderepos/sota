
\section{Introduction} \label{sec:intro}

\IEEEPARstart{D} {ynamic} systems which exhibit both continuous state evolution and discrete state transitions can typically be modeled as {\em hybrid automata (HA)} (\cite{henzinger:96, lynch:03}).
Computing the reach set of a hybrid automaton from a given set of initial states is a problem of fundamental importance as it is related to safety verification and automated controller synthesis.
Even though many systems can be so modeled, it is in general undecidable to compute the exact reach set \cite{henzinger:95} except for classes of hybrid automata whose continuous dynamics are fairly simple, such as timed automata (TA) \cite{alur:94} and initialized rectangular hybrid automata (IRHA) \cite{henzinger:95}.
Neither of these automata allow the standard linear systems dynamics which is widely used for control systems. 
To broaden the class of systems that can be addressed, research in hybrid system verification in the recent years has focused on algorithms computing over-approximations of the reach set for various classes of hybrid automata (\cite{henzinger:97, frehse:08, chutinan:03, girard:05, asarin:00, kurzh:00, clarke:03, tiwari:02}).
However, even with this relaxation from exact reach set to over-approximations, it is still a challenging problem to compute an over-approximation of the reach set of hybrid automata with linear dynamics with arbitrarily small approximation error and a termination guarantee for the computation.


\subsection{Related Work}
For the computation of reach set of hybrid automata with linear dynamics, several tools and approaches have been proposed in the literature.
As an example, HyTech \cite{henzinger:97} computes the reach set of hybrid automata whose continuous dynamics are more general than those of IRHA by translating the original model into an IRHA if the model is {\em clock translatable}.
Otherwise, an over-approximate reach set is computed through an approach, called {\em linear phase-portrait approximation}, which approximates the original hybrid automaton by relaxing the continuous dynamics of the original automaton.
PHAVer \cite{frehse:08} can handle a class of systems called linear hybrid automata that have affine dynamics. 
It computes a conservative over-approximation of the reach set of such hybrid automata through on-the-fly over-approximation of the phase portrait, which is a variation of the phase-portrait approximation in \cite{henzinger:97}.
Recently, another tool, SpaceEx, has been developed based on the algorithm called LeGuernic-Girard (LGG) algorithm \cite{guernic:10} which allows the handling of hybrid automata with linear differential equations with a larger number of continuous variables compared to other approaches.

In \cite{chutinan:03}, a class of hybrid automata, called {\em polyhedral-invariant hybrid automata (PIHA)}, is defined and an algorithm is proposed to construct a finite state transition system, which is a conservative approximation of the original PIHA. 
Determining a polyhedral approximation of each sampled segment of the continuous state evolution between switching planes is the underlying fundamental technique in the algorithm that is used.
Another approach proposed in \cite{asarin:00} is also based on the idea of sampling and polyhedral over-approximation of continuous state evolution of a continuous linear dynamics. 
On the other hand, in \cite{kurzh:00} and \cite{girard:05}, ellipsoids and zonotopes are used respectively for approximating continuous state evolution.

However, while these algorithms and tools compute some over-approximation of the reach set of hybrid systems with linear dynamics, computation of an over-approximate reach set which is arbitrarily close to the exact reach set of such hybrid systems with guaranteed termination remains an open issue for further research. 



\subsection{Challenges and Contributions}
In general, the key challenges in reach set computation of HA are 
\begin{inparaenum}[(i)] 
	\item to over-approximate the exact continuous flow with arbitrarily small approximation error,
	\item to determine when and where a discrete transition occurs, and
	\item to develop a reach set computation algorithm with termination guarantee. 
\end{inparaenum}
In this paper, we address the problem of computing an over-approximation of the reach set of a special class of hybrid automata, called {\em Deterministic and Transversal Linear Hybrid Automaton (DTLHA)}, starting from an initial state over a finite time interval.
We call such an over-approximate reach set as a {\em bounded -reach set}.
Our approach can be related to other approaches that use sampling and polyhedral over-approximation as in \cite{chutinan:03, asarin:00}.
The main contributions of our approach are as follows: 
\begin{inparaenum}[(i)]
	\item We show that an over-approximation of the reach set of a DTLHA can be computed arbitrarily closely to the exact reach set.
	\item We also show that such computation is guaranteed to terminate under a deterministic and transversal restriction on the discrete dynamics. 
	\item Furthermore, to facilitate practical computation, we extend these theoretical results to consider the numerical calculation errors caused by finite precision calculation capabilities. 
\end{inparaenum}
Based on the theoretical results, we propose an algorithm to compute a bounded -reach set of a DTLHA, as well as a software architecture that is designed to improve the flexibility and the efficiency in computing such an over-approximation.

The paper is organized as follows.
In Section \ref{sec:pre}, we introduce definitions and notations that are used throughout this paper.
In Section \ref{sec:theory}, we show that, for arbitrarily small , a bounded -reach set of a DTLHA starting from an initial state can be computed under the assumption of infinite precision numerical calculation capabilities. 
In Section \ref{sec:cond}, we first derive a set of conditions for computation of a bounded -reach set, and then extend these conditions to consider errors caused by finite precision numerical calculation capabilities.
In Section \ref{sec:design}, we propose an algorithm for a bounded -reach set computation, as well as an architecture for software implementation of the proposed algorithm.
Finally, we illustrate an example of bounded -reach set computation in Section \ref{sec:imp}, followed by concluding remarks in Section \ref{sec:con}.





\section{Preliminaries} \label{sec:pre}


Let   be a continuous state space over which a hybrid automaton is defined.
For a polyhedron , we denote its interior by , and its boundary by .
We will also use the notation  to denote a closed ball of radius  with center , i.e., . 
The specific norm that we use in the definition of  as well as the sequel is the -norm. 
Since we are using the -norm,  is a hypercubic neighborhood of .
One of the advantages of using the -norm is that the induced hypercubic neighborhood is easily computed. 
More generally, a hypercube is a special case of a polyhedron, which is important since it is easy to propagate the image of this set under linear dynamics.
This is useful in Section \ref{sec:theory} when we describe our approach for bounded -reach set computation.


We now describe the class of hybrid automata considered.
We assume that  is a closed and bounded subset of Euclidean space, and is partitioned into a collection of polyhedral regions  such that  for each  and 



where  is the size of the partition, and each  is a polyhedron, called \emph{cell}. 
Two cells  and  are said to be \emph{adjacent} if the affine dimension of  is , or, equivalently, cells  and  intersect in an -dimensional facet. 
Two cells  and  are said to be \emph{connected} if there exists a sequence of adjacent cells between  and .


\begin{defn} \label{def:lha}
An -dimensional \emph{Linear Hybrid Automaton (LHA)},\footnote{In the hybrid system literature \cite{henzinger:97, alur:93} the word ``linear automaton'' has been used to denote a system where the differential equations and inequalities involved have constant right hand sides. This does not conform to the standard notion of linearity where the right hand side is allowed to be a function of state. In particular, it does not include the standard class of linear time-invariant systems that is of central interest in control systems design and analysis. We use the term ``linear'' in this latter more mathematically standard way that therefore encompasses a larger class of systems, and, more importantly, encompasses classes of switched linear systems that are of much interest.} 
is a tuple  satisfying the following properties: 
\begin{enumerate}[(a)]
	\item  is a finite set of \emph{locations} or \emph{discrete states}. The state space is , and an element  is called a \emph{state}.
\item[(b)]  is a function that maps each location to a set of cells, called an \emph{invariant set} of a location, such that 
	 \begin{inparaenum}[(i)]
		\item for each , all the cells in  are connected, 
		\item for any two locations , , and 
		\item .
	\end{inparaenum}
\item[(c)]  is a function that maps each location to an  real-valued matrix, and 
\item[(d)]  is a function that maps each location to an -dimensional real-valued vector.
\item[(e)]  is a binary relation which defines a \emph{discrete transition} from one state  to another state  such that  when  is satisfied and  is set to  after a discrete transition.
\end{enumerate}
\end{defn}
In the sequel, for each , we use , ,  to denote , , and , respectively.

An example LHA which satisfies Definition \ref{def:lha} is shown in Section \ref{sec:imp:example}.
Next, we define the behavior of LHA.
\begin{defn} \label{def:traj} For a location , a \emph{trajectory} of duration  for an -dimensional LHA  is a continuous map  from  to , such that
\begin{enumerate}[(a)]
    \item  satisfies the differential equation
            
    \item  for every .
\end{enumerate}
\end{defn}



\begin{defn} \label{def:exec}An \emph{execution}  of an LHA  from a starting state  is defined to be the concatenation of a finite or infinite sequence of trajectories , such that
\begin{enumerate}[(a)]
    \item ,
    \item  for , 
\end{enumerate}
where  represents a trajectory defined at some location  and  denotes the duration of  .
We also define  where  denotes the duration of an execution .
\end{defn}


We can represent an execution  of an LHA  from an initial condition  for time  as a continuous map  such that
\begin{inparaenum}[(a)]
    \item ,
    \item ,
    \item , and
    \item  for ,
\end{inparaenum}
where , and  for .
Note that  for  represents the time at the -th discrete transition between locations and the continuous state is not reset during discrete transitions.

\begin{defn} \label{def:trans} For an execution  of an LHA, a discrete transition  occurs if  for some time , 
 and  where  for  for some .

\end{defn}
\begin{defn} \label{def:transversal}  A discrete transition is called \emph{deterministic} if there is only one location  to which a discrete transition state  can make a discrete transition from . 
We call a discrete transition a \emph{transversal discrete transition} if there exists  such that

where  denotes the inner product between  and ,  is an outward normal vector of  at , and , and  are the vector fields at  evaluated with respect to the continuous dynamics of location  and , respectively. 
\end{defn}
Fig. \ref{fig:transition} illustrates a case where  satisfies such a deterministic and transversal discrete transition condition.
Note that if  satisfies a deterministic and transversal discrete transition condition, then  must make a discrete transition from a location  to the other location , and  has to be unique. 
Furthermore, the \emph{Zeno behavior}, an infinite number of discrete transitions within a finite amount of time, does not occur if a discrete transition is a transversal discrete transition.
\begin{figure}
\begin{center}
  \includegraphics[width=7cm]{transition.png} \caption{A deterministic and transversal discrete transition from a location  to a location  occurring at .}
\label{fig:transition}
\end{center}
\end{figure}



We now define a special class of LHA whose every discrete transition satisfies the deterministic and transversality conditions defined in Definition \ref{def:transversal} as follows:

\begin{defn} \label{pre:def:dtlha}
Given an LHA , a starting state , a time bound , and a jump bound , we call an LHA  as a \emph{Deterministic and Transversal Linear Hybrid Automaton (DTLHA)}  if all discrete transitions in the execution starting from  up to time  are deterministic and transversal, where  is the time at the N-th discrete transition.
\end{defn}


Next, we define the bounded reach set of a DTLHA and its over-approximation as follows:

\begin{defn}
A continuous state in  is \emph{reachable} if there exists some time  at which it is reached by some execution .
\end{defn}
\begin{defn} \label{def:pre:reach}
Given a state  and a time , the \emph{bounded reach set} up to time , denoted as , of a DTLHA  is defined to be the set of continuous states that are reachable for some time  by some execution  starting from .
\end{defn}
\begin{defn} \label{def:pre:ereach}
Given , a set of continuous states  is called a \emph{bounded -reach set} of a DTLHA  over a time interval  from an initial state  if  and

where  denotes the Hausdorff distance 
between two sets  and 
that is defined as  where .
\end{defn}


In the sequel, we use  to denote the set of states reached at time  from a set  at time .
Similarly, for the set of reached states over a time interval  from , we use .
We also use  to denote an over-approximation of  with an approximation parameter , calling it a -approximation of  if it satisfies 
\begin{inparaenum}[(i)]
\item  and
\item  .
\end{inparaenum}
Note that  is simply a -approximation of the set .



\section{Bounded -Reachability of a DTLHA} \label{sec:theory}

In this section, we consider the problem of a bounded -reach set computation of a DTLHA starting from an initial state over a finite time interval.
More precisely, we show that, for any given , a DTLHA , an initial condition , a time upper bound , and a discrete transition upper bound , it is possible to compute a bounded -reach set of  over a finite time interval  under the assumptions that the following computations can be performed exactly:
\begin{inparaenum}[(i)]
	\item  ,
	\item the convex hull of a set of finite points in , and
	\item the intersection between a polyhedron and a hyperplane,
\end{inparaenum}
where  is as defined in Definition \ref{pre:def:dtlha}, , and .



\subsection{Bounded -Reach Set of a DTLHA at Initial Location}  \label{sec:theory:l0}

We first show how a trajectory of a DTLHA can be over-approximated through sampling and polyhedral over-approximation of each sampled state.
The basic approach for such over-approximation is shown in Fig. \ref{fig:traj}. 
It is necessary that, for a given size of over-approximation of each sampled state, a sampling period  has to ensure that a trajectory  is contained in the computed set of polyhedra.
For a given value of , we now show how we can determine a sampling period  which guarantees that.


To determine a suitable value of  which results in (\ref{eq:hcond}), we suppose  for all  for some location .
Then for a given , , and , we have 


where . 

For a fixed , we can compute an upper bound on  as follows:


Maximization of both sides of (\ref{eq:h:ineq:0}) over  gives us 


If we upper bound the right hand side by , then we can choose 

where  .

So, if we choose  as 

then it is clear that we can ensure (\ref{eq:hcond}). 

We now show that, for a given , if a sampling period  satisfies (\ref{eq:h:eq}), then a set constructed as a union of -neighborhood of each sampled state along a trajectory is indeed a bounded -reach set at an initial location.  
Moreover, such a bounded -reach set contains the bounded reach set not only from the initial state but also from the -neighborhood of the initial state.

\begin{figure}
\begin{center}
	\includegraphics[width=8cm]{traj.png}
	\caption{An over-approximation of a trajectory  through sampling.}\label{fig:traj}
\end{center}
\end{figure}



\begin{lem} \label{lem:theory:l0}
Given  and a time bound , a bounded -reach set  of a DTLHA  from an initial state  can be determined as follows:

where , ,  and .
Moreover, this set has two additional properties:
\begin{enumerate}[(i)]
    \item , and
    \item It contains an  neighborhood of , i.e.,
    
\end{enumerate}
\end{lem}
\begin{proof}
Since  satisfies (\ref{eq:h:ineq}), it is easy to see that  from the construction of .
Next, by the relation between  and  in (\ref{eq:h:eq}), it is clear that  as . 
This implies that  as , establishing (i). 
For (ii), as noted above, (\ref{eq:h:eq}) actually chooses half the sampling period that would have sufficed to make it a bounded -reach set over . 
Hence, replacing  by  in the right hand side of (\ref{eq:lem:theory:l0}) still yields a bounded -reach set. 
Thus the over stringent choice of  contains not just  but actually all points that are within a distance  from it.
\end{proof}




\subsection{Continuity Property of DTLHA}  \label{sec:theory:cont}

Now let us consider the problem of computing a bounded -reach set of a DTLHA  not from an initial state  but from a -neighborhood of .
We first show that there exists a  such that the bounded reach set of a DTLHA  from a set  at an initial location  is contained in a bounded -reach set of  from  defined in (\ref{eq:lem:theory:l0}).


\begin{lem} \label{lem:theory:cont:1}
Given , a time bound , an initial state , and a DTLHA , there exists a  such that 

where  is a -neighborhood around  and  is the bounded reach set of  from  up to time  and  is as defined in Lemma \ref{lem:theory:l0}. 
In particular,  for an appropriate .
\end{lem}
\begin{proof}
Notice that , where  and  define the linear dynamics in an initial location .
If we consider two different initial states  and  in , then their trajectories  and  satisfy .
Hence  for some positive constant  and some constant .

Let .
Then

Since ,  for all .
This implies that any initial condition  in  results in a  that lies in a  neighborhood of  for all . 
In particular, from property (ii) of Lemma \ref{lem:theory:l0}, it also follows that .
If we set , then it is clear that .
\end{proof}


Next we extend the result in Lemma \ref{lem:theory:cont:1} to show that there exist a  and a  such that an over-approximation of the bounded reach set  , denoted as , is also contained in  that is defined in (\ref{eq:lem:theory:l0}).


\begin{lem} \label{lem:theory:cont:2}
Given , a time bound , an initial state , and a DTLHA , there exist  and  such that 

where  is a -approximation of , and  is as defined in Lemma \ref{lem:theory:l0}.
In particular, .
\end{lem}
\begin{proof}
Let  denote the solution at time  of the differential equation  with initial condition .
Now consider . 
Then, by the definition of  and , 
 
for some  and . 
Hence

From (\ref{eq:cont:1}), we know that 

Hence 

which implies that  lies in a -neighborhood of .
From the property (ii) in Lemma \ref{lem:theory:l0}, if we replace  with , then we have  which in turn implies that .
So, given , we can choose  and , and then .
\end{proof}




\subsection{Decidability of Discrete Transition Event}  \label{sec:theory:trans}

Recall that  is the time  when a reached state  of a DTLHA starting from an initial state first exits the invariant set of an initial location.
We now show that, for a given , even though it is not known to be decidable to determine  exactly, we can still determine the event of exit of a reached state  from the invariant set of an initial location if .


\begin{lem} \label{lem:theory:trans:exit}
Given a time bound , an initial condition , and a DTLHA , if , then for all small enough  and for some small enough ,  for some  satisfying .
\end{lem}
\begin{proof}
Let  be an outward normal vector of  at .
Since  by assumption, then by the continuity of the vector field of a linear dynamics in , there exists an  such that for all ,  where .
Notice that  by the definition of  in (\ref{eq:h:ineq}).
Let  denotes the solution at time  of the differential equation  with initial condition .
Then for any , it is guaranteed that  for  for any  satisfying .
This implies that  for some .
Moreover by compactness of , there exists a  such that .
\end{proof}


Now suppose that  for all  for some .
Then this fact can also be determined.


\begin{lem} \label{lem:theory:trans:noexit}
Suppose  for all  for some . 
Then for all small enough  and ,

where .
\end{lem}
\begin{proof}
Since  for all , the result immediately follows from Lemma \ref{lem:theory:cont:2}.
\end{proof}



\subsection{Over-approximation of Discrete Transition State}  \label{sec:theory:over}

For a given time bound , suppose that the event  is determined for some  and  as shown in Lemma \ref{lem:theory:trans:exit}.
Then, to continue to compute a bounded -reach set beyond an initial location, we need to determine 
\begin{inparaenum}[(i)]
	\item a new location to which a discrete transition is made from an initial location, and also
	\item an over-approximation of a discrete transition state from which the bounded -reach set computation can be continued.
\end{inparaenum}
We now show that  these can be determined, if a discrete transition state  is deterministic and, more importantly, transversal, as defined in Definition \ref{def:transversal}.


\begin{lem} \label{lem:theory:over}
Given , if  satisfies a deterministic and transversal discrete transition condition, then there exists a  such that  for some location . Furthermore, there exists a  such that
\begin{enumerate}[(i)]
    \item  for  , and
    \item 
\end{enumerate}
where  is the solution at time  of an LTI system for the location  with an initial state  and . 
\end{lem}
\begin{proof}
Let  be invariant sets for some locations  and  such that .
Since  satisfies a deterministic discrete transition condition, if , then .
This implies that .
Then by compactness of , we know that there exists a  such that . Therefore, we conclude that .

Let  be an outward normal vector of  at .
Since  satisfies a transversal discrete transition condition from the location  to the other location , we know that there exists a  such that for all , , where  is taken as either  or as , by the continuity of vector fields of the LTI dynamics for  and .

Let , and  where  is as defined in (\ref{eq:h:ineq}).
Then by the definition of  and , it is clear that (i) and (ii) hold for these choices of  and .
\end{proof}



In Lemma \ref{lem:theory:over},  is an over-approximation of  that is determined by taking a -ball around  for suitably small , and intersecting it with  and . Once such a suitably small  is known, then the following lemma shows that it is also possible to determine a -neighborhood of an initial state  such that the reach set at time  of a DTLHA  from  is contained in .

\begin{lem} \label{lem:theory:over:cont:1}
Given  determined by Lemma \ref{lem:theory:over}, there exists a  such that

and  is an over-approximation of  determined by .
\end{lem}
\begin{proof}
This follows from the same argument used in the proof of Lemma \ref{lem:theory:cont:1}, by choosing .
\end{proof}

The next lemma shows that  for  can be determined at each discrete transition time  for .

\begin{lem} \label{lem:theory:over:cont:2}
Let  be the radius of a ball centered at  intersecting only  and , where  is the -th discrete transition time and  is the location after the -th discrete transition. Then for any  satisfying a deterministic and transversal discrete transition condition, there exists a  such that

where  is the reached states of a given DTLHA  from  at time .
\end{lem}
\begin{proof}
From the continuity property shown in Lemma \ref{lem:theory:cont:1}, there is a  such that    for a given  where  denotes a -approximation of .
Then for this , it is clear that   .
Using the same argument, we can find .
Then from Lemma \ref{lem:theory:over:cont:1}, we know that there exists a  such that   .
Since   , we have   .
This relation holds for each  where .
Therefore,   . 
\end{proof}


We now present our main result for the bounded -reachability of a DTLHA.

\begin{thm} \label{thm:theory:thm}
Given , a time bound , a discrete transition bound , and a DTLHA  starting from an initial condition , there exist , , and a sampling period  satisfying  such that 

where  and  is the time at the N-th discrete transition. 
\end{thm}
\begin{proof}
Let  for a location  and .
For a given , suppose  at each  up to  where  is as defined in Lemma \ref{lem:theory:over:cont:2}.
Then, from Lemmas \ref{lem:theory:over}, \ref{lem:theory:over:cont:1}, and \ref{lem:theory:over:cont:2}, we know that there exist a  such that  where  is the execution of a DTLHA  starting from  at time zero.
Furthermore, from Lemmas \ref{lem:theory:trans:exit} and \ref{lem:theory:over}, there also exists  and  such that 
\begin{inparaenum}[(i)]
	\item  and
	\item  and  satisfy Lemma \ref{lem:theory:trans:exit}
\end{inparaenum}
at every  up to , where  is the  that is defined in Lemma \ref{lem:theory:over} for the -th deterministic and transversal discrete transition.

Let .
Then, with  and , we can determine every discrete transition event and also construct an over-approximation of the discrete transition state as long as it is deterministic and transversal.
Since ,  at each  up to .
Thus, for any , 

where .

Now, we notice that if , then from Lemma \ref{lem:theory:cont:2},

for each  up to , where the left hand side is a segment of  for , and the right hand side is a segment of  for  that is defined as  where .

Furthermore, if , then from (\ref{eq:h:ineq}) replaced with  by , it is clear that 

where the left hand side is a segment of  for .
Therefore, the result holds.
\end{proof}






\section{Computing a Bounded -Reach Set of a DTLHA}  \label{sec:cond}

From Theorem \ref{thm:theory:thm}, we know that a set , a bounded -reach set of a DTLHA, can be computed for some , and . 
In this section, we discuss how to compute .
More precisely, we derive a set of conditions, based on the results in Section \ref{sec:theory}, that are needed to correctly detect a deterministic and transversal discrete state transition event and also to determine whether the values for the parameters , and  are appropriate so as to ensure that  is a correct bounded -reach set.
Furthermore, later in this section, we extend these conditions to incorporate the numerical calculation errors caused by the finite precision numerical calculations capabilities. 



\subsection{Conditions for Bounded -Reach Set Computation}  \label{sec:cond:exact}

We first note some properties that a set  needs to satisfy so that it can be considered as a bounded -reach set of a DTLHA.

\begin{rem} \label{rem:cond:exact}
Notice that any  that can be determined by , and  in Theorem \ref{thm:theory:thm} for a given  needs to satisfy the following properties.
\begin{enumerate}[(i)]
	\item , 
	\item  , and
	\item .	
\end{enumerate}
\end{rem}


For given  and , the following lemma shows how we can detect a discrete state transition event if there is one.


\begin{lem} \label{lem:cond:exact:trans}
Given a location  and a DTLHA , 
if  and  for some  and , where  is a -neighborhood of the initial state , then there is a discrete transition from the location  to some other locations at some time in .
\end{lem}
\begin{proof}
Recall that  denotes the reached state of  at time  from .
Then it is clear that . 
Similarly,  . 
Hence, from the hypothesis,  and .
This implies that there exists  such that  for  and   for .
Therefore, there is a discrete transition at some time .
\end{proof}


Once a discrete state transition is detected, then, by Lemma \ref{lem:cond:exact:det}, we can check if it is deterministic or not.


\begin{lem} \label{lem:cond:exact:det}
Given an initial state  and a DTLHA ,
suppose that there is a discrete transition from a location  to some other locations at time , i.e.,  and  for some  and . 
Then the discrete transition is deterministic if there exists a location  such that  and .
\end{lem}\begin{proof}
This follows from the definition of a deterministic discrete transition in Definition \ref{def:transversal}.
\end{proof}


We now present conditions to determine the transversality of a discrete state transition; this is more complicated than those in previous two lemmas.
The main idea of the conditions in the following Lemma \ref{lem:cond:exact:transv} is that 
\begin{inparaenum}[(i)]
	\item  and  have to be small enough so that every state in an over-approximation of a deterministic and transversal discrete transition state, which can be computed by  and , is  also deterministic and transversal, and also
	\item the sampling period  should be small enough so that any reached states right after a discrete state transition can be captured correctly. 
\end{inparaenum}


\begin{lem} \label{lem:cond:exact:transv}
Given  and  satisfying ,
suppose that there is a deterministic discrete transition from a location  to another location  at time , i.e.,  and  for some  and . 
Then for any , the discrete transition is transversal if the following conditions hold:
\begin{itemize}
\item[(i)] , 
\item[(ii)] , and
\item[(iii)] ,
\end{itemize}
where ,  ,  is as defined in (\ref{eq:h:ineq}),  is a set of vertices of a polyhedron ,  is an outward normal vector of , and  is the vector flow evaluated with respect to the LTI dynamics of location .
\end{lem}
\begin{proof}
Notice that  since  and  satisfy .
In fact,  for  where  under the LTI dynamics of the location .
Since  and ,  for some  where  is a discrete transition state from  to  at time .
Thus  (more precisely, ) and it is in fact an over-approximation of the deterministic discrete transition state .

If (ii) and (iii) hold, then it is easy to see that  satisfies the deterministic and transversal discrete transition condition in Definition \ref{def:transversal} for any .
Now we suppose (i) holds 
and let  is the state reached from  at time  under the LTI dynamics of the location , then,  for any , 
 

If we now consider the fact that , then it is easy to see that  for .
Since  is arbitrary, we conclude that 

for all .
Thus, the discrete transition state  is transversal and it can be determined through  with  satisfying (i).
\end{proof}



\subsection{Finite Precision Basic Calculations}  \label{sec:cond:basic}

Notice that the results in Section \ref{sec:cond:exact} are based on the assumption that the following quantities can be computed exactly:
\begin{itemize}
\item .
\item , where  is a hyperplane and  is a polyhedron.
\item , where  is the convex hull of  that is a finite set of points in .
\end{itemize}
However, these exact computation assumptions cannot be satisfied in practice and we can only compute each of these with possibly arbitrarily small computation error. 
Therefore, instead of assuming exact computation capabilities for , , and , we now assume that the following basic calculation capabilities are available for approximately computing these quantities, and it only these that we can use to compute a bounded -reach set. 
More precisely, we assume that for given  and , 
\begin{itemize}
	\item  and 
\end{itemize}
are available such that , where  denotes an approximate computation of , with  as an upper bound on the approximation error.
We also assume that for given  and ,
\begin{itemize}
	\item , and 
\end{itemize}
are available as an approximate computation of  such that .
Notice that from these basic calculation capabilities for , we can compute  with an approximation error  denoted as , which is upper bounded by a finite value as shown below.


We first note that, for all approximate computations  that are used for computing , we have 

where  and  is an  by  matrix whose every element is , and the inequalities hold elementwise.
With this, an upper bound of  can be derived as follows:

Similarly, 

Hence, we have 

where .

Now, we know that  is upper bounded by the maximum of  over the continuous state space  and the control input domain ,




\subsection{Conditions for Computation under Finite Precision Calculations}  \label{sec:cond:inexact}

In this section, we extend the results in Section \ref{sec:cond:exact} to derive a set of conditions for a bounded -reach set computation of the DTLHA under finite precision numerical calculation capabilities.
The following remark is an immediate extension of Remark \ref{rem:cond:exact} in Section \ref{sec:cond:exact}.

In the sequel, for simplicity of notation, we use  to denote  for a given approximation error bound .

\begin{rem} \label{rem:cond:inexact}
Let  be an approximation of  that is determined by , and  in Theorem \ref{thm:theory:thm} and approximate calculations for , , and  defined in Section \ref{sec:cond:basic}.
Then, for a given , it is sufficient for  to be a bounded -reach set of a DTLHA  if the following properties hold.
\begin{enumerate}[(i)]
	\item , 
	\item  , and
	\item .	
\end{enumerate}
\end{rem}


Next, we discuss how the relation between  and  can be modified so as to satisfy (ii) and (iii) in Remark \ref{rem:cond:inexact} when there is numerical calculation error in computing . 


\begin{lem} \label{lem:cond:inexact:isover}
Given a DTLHA  and its reached state  at time  starting from an initial condition , 
let  be an upper bound on the approximation errors such that .
If a given sampling period  satisfies  for a given  satisfying ,  where  is as defined in (\ref{eq:h:ineq}), then the following property holds at any location  of :

where .
\end{lem}
\begin{proof}
Since , .
Moreover, from (\ref{eq:h:ineq:1}), we know that for any ,

Hence, if , then, for any , 
 
This means that  for .
Therefore, for , 

Thus .
\end{proof}


Notice that Lemma \ref{lem:cond:inexact:isover} says that if  for a given , then a -neighborhood of a sampled state is indeed an over-approximation of a trajectory over the time interval .
We now extend the result in Lemma \ref{lem:cond:inexact:isover} to the case where we need to compute a -approximation of a polyhedron.

\begin{lem} \label{lem:cond:inexact:over}
Given a DTLHA  and its reached states  at some time  from initial states in , 
let  be an upper bound on the approximation errors such that .
If a given sampling period  satisfies the following inequality 

then, for a given  satisfying , 

 where  is a -approximation of  that is  constructed as the convex hull of the set of extreme points of a polyhedral -neighborhood of all vertices of  and  is as defined in (\ref{eq:h:ineq}).
\end{lem}
\begin{proof}
Let  and  be the set of extreme points of  and , respectively. 
Since  and , it is clear that .
From Lemma \ref{lem:cond:inexact:isover}, we know that for each ,  for all  where  corresponding to .
Let  be the set of extreme points of . 
Then  for all  since 
\begin{inparaenum}[(i)]
	\item for each ,  for all  and 
	\item from the construction of ,  for each .
\end{inparaenum}
Therefore, the convex hull of , which is , has to be contained in  for all  since  is convex and  for all .
\end{proof}


For (i) in Remark \ref{rem:cond:inexact}, Lemma \ref{lem:cond:inexact:eps} below shows that the diameter of a set  has to be smaller than a given .



\begin{lem} \label{lem:cond:inexact:eps}
Given , , ,  , and a DTLHA , 
suppose a given sampling period  satisfies the inequality (\ref{eq:lem:cond:inexact:over}).
Then  and , if the following hold:

where  is the set of reached states of  starting from  during the time interval . 
\end{lem}
\begin{proof}
Since  satisfies (\ref{eq:lem:cond:inexact:over}), it is trivial to see that   holds from Lemma \ref{lem:cond:inexact:over}.
Moreover, if (\ref{eq:lem:cond:inexact:eps}) is also true, then for any ,  since .
Therefore, it is clear that  if (\ref{eq:lem:cond:inexact:over}) and (\ref{eq:lem:cond:inexact:eps}) hold.
\end{proof}


Now we can extend the results of Lemmas \ref{lem:cond:exact:trans}, \ref{lem:cond:exact:det}, and \ref{lem:cond:exact:transv} to incorporate a numerical calculation error .

\begin{lem} \label{lem:cond:inexact:istrans}
Given ,  a location , and  at time ,
if
\begin{enumerate}[(i)]
\item  , and
\item  
\end{enumerate} 
for some  and , then there is a discrete transition from the location  to some other locations.
\end{lem}
\begin{proof}
Notice that , which implies .
Similarly,  .
Hence if (i) and (ii) hold, then it is clear that  and .
Then the result follows from Lemma \ref{lem:cond:exact:trans}. 
\end{proof}


\begin{lem} \label{lem:cond:inexact:det}
Given , a location , and  at time , suppose that a discrete transition from a location  to some other locations is determined as in Lemma \ref{lem:cond:inexact:istrans}.
Then the discrete transition is a deterministic discrete transition from  to  if there exists a location  such that  and .
\end{lem}
\begin{proof}
Notice that  from the result in Lemma \ref{lem:cond:inexact:istrans}.
Since , if  , then  .
Thus by Lemma \ref{lem:cond:exact:det}, the conclusion holds.
\end{proof}




\begin{lem} \label{lem:cond:inexact:transv}
Given ,  and  satisfying (\ref{eq:lem:cond:inexact:over}), suppose that a deterministic discrete transition from a location  to another location  is determined as in Lemma \ref{lem:cond:inexact:istrans} and Lemma \ref{lem:cond:inexact:det}, i.e.,  and .
Then, for any , the discrete transition is transversal if the following conditions hold:
\begin{itemize}
\item[(i)] , 
 \item[(ii)] , and 
 \item[(iii)] ,
 \end{itemize}
where ,  ,  is a -approximation of , and  and  are as defined in Lemma \ref{lem:cond:exact:transv}. 
\end{lem}
\begin{proof}
Notice that .
Then, by the definition of  given in Lemma \ref{lem:cond:exact:transv} and , we know .
Hence,  since  by the construction of  .
Now if (i) holds, then it is easy to see that  for .
Moreover, (ii) and (iii) imply that  is in fact contained in  for .
\end{proof}





\section{Architecture and Algorithm for Bounded -Reach Set Computation of a DTLHA}  \label{sec:design}

We are now in a position to propose an algorithm for bounded -reach set computation of a DTLHA.
Before proving its correctness, we first describe its architecture.



For \emph{flexibility}, we decouple the higher levels of the algorithm, called \emph{Policy}, from the component, called \emph{Mechanisms}, where specific steps of calculations are performed through some numerical routines.
The proposed architecture of the algorithm, shown in Fig. \ref{fig:arch}, consists of roughly five different components \emph{Policy}, \emph{Mechanism}, \emph{System Description}, \emph{Data}, and \emph{Numerics}.
A more detailed explanation of each of these modules is given below. 


\begin{figure}
\begin{center}
\includegraphics[width = 9cm]{arch4algo.png} 	\caption{An architecture for bounded -reach set computation.}
\label{fig:arch}
\end{center}
\end{figure}


The System Description contains all information describing a problem of a bounded -reach set computation of a DTLHA.
This consists of , the domain of continuous state space, a DTLHA , and an initial condition . 
Also, an upper bound  on terminal time, an upper bound  on the total number of discrete transitions, and an approximation parameter , are described.
A bounded -reach set of a DTLHA  is computed in the Mechanism component based on a given set of numerical calculation algorithms in Numerics, as well as a given Policy, which captures some of the higher level choices of the algorithm's outer loops. 
In the Data component, all computation data that is relevant to a computed bounded -reach set, generated on-the-fly in the Mechanism part, are stored. 
Each of the functions in Numerics is in fact an implementation of some numerical computation algorithms.  
As an example,  can be computed in many different ways as shown in \cite{moler:78} and each of the different algorithms can compute the value with a certain accuracy. 
Here we assume that a set of such numerical computation algorithms for basic calculations are given\footnote{In this way, we decouple the low-level numerical calculations from our bounded -reach set algorithm. This is the reason why the Numerics component is represented separately from the Mechanism component.} 
and the corresponding approximation error bounds, i.e., , and , are known a priori.
The Policy component represents a user-defined rules that choose appropriate values of the parameters, especially , , and , which are needed to continue to compute a bounded -reach set of a DTLHA, when a bounded -reach set algorithm in Mechanism fails to determine some events or to satisfy some required properties, during its computation.
The Mechanism component represents the core of the bounded -reach set algorithm based on the theoretical results in Section \ref{sec:theory} and \ref{sec:cond}, and is detailed in Section \ref{sec:design:algo}. 
Given values for parameters , , and , it computes a bounded -reach set of a DTLHA  until it either successfully finishes its computation or cannot make further progress, which happens when some required conditions or properties are not met.
Notice that,  as stated in Section \ref{sec:cond}, there are a set of conditions and properties that a computed set needs to satisfy to be a correct bounded -reach set.
If the algorithm fails to resolve a computation, then it returns to Policy indicating the problems so that a user-defined rule in Policy can choose another set of values for the parameters to resolve the problems.
Every computation result is stored in the Data component to be possibly used later in Policy and Mechanism.




\subsection{Core Algorithm for Bounded -Reach Set of a DTLHA}  \label{sec:design:algo}





An algorithm to compute a bounded -reach set of a DTLHA is proposed and shown in Algorithm \ref{alg:main}.
Let  indicate a computation step of the algorithm from which the proposed algorithm starts its bounded -reach set computation.
All computation history up to the -th computation step is stored as data, called {\tt Reached}, in Data part.
Then, given an input  from Policy, the algorithm first retrieves the computation data at the -th computation step from {\tt Reached} and starts its -th computation step using this data.
As shown in Algorithm \ref{alg:main}, the algorithm continues its computation until it either
\begin{inparaenum}[(i)]
	\item returns {\tt done} when it successfully finished to compute a bounded -reach set or
	\item returns {\tt error} when it encounters some erroneous situations during the execution of a function, called {\tt Post()}.
\end{inparaenum}
If the algorithm returns an {\tt error}, it also indicates the cause of the {\tt error} so that a user-defined rule in Policy can choose appropriate values for the input parameters.


\begin{algorithm}  \SetAlgoLined 
\BlankLine
\KwIn{ from Policy} 
\BlankLine
\Compute  from \\
\BlankLine
\While{\True} {
\Get data at -th step from \Reached \\
\If{} {
	\Compute \\ \Update  }
\\
\Call  \Post  \\
\Store -th computation data into \Reached \\
\\
\lIf{}{\Return \Done}
}
\BlankLine
\caption{An algorithm for bounded -reach set computation of a DTLHA.}
\label{alg:main}
\end{algorithm}


\begin{algorithm} \SetAlgoLined 
\BlankLine
\KwIn{} 
\BlankLine
\Compute  from \\
\Compute  from \\
\Update  \\
\BlankLine
\lIf{}{\Return \Err}\\
\lIf{} {\Return \Err} \\
\uIf{}{
	\If{} {
		\uIf{}{
			\Update  and 
			\Update \\
			\Update \\
			 \Jump  \Jump + 1
		}
		\lElse{\Return \Err}
	}
}
\uElseIf{}{\Return \Err}
\lElse{}
\BlankLine
\caption{A function {\tt Post()}.}
\label{alg:post}
\end{algorithm}



In the proposed algorithm in {\tt Post()},  is computed from  as follows:

Given a polyhedron , we first compute the set of the vertices of  that is denoted as .
Then for each , we compute 
 
where  and  are given by the linear dynamics of a location  on which the linear image of  is computed at the -the computation step in Algorithm \ref{alg:main}. 
If we let , then we can compute  as follows: 

where  is the convex hull of . 

Once we have , we compute  in the following way.
To compute  for a given , we first construct a hypercubic -neighborhood of  for each .
Let  be such a  hypercubic neighborhood of  and  be the set of vertices of  for all .
Then we can compute  as follows:


This process of polyhedral image computation under a linear dynamics is illustrated in Fig. \ref{fig:post:linear}.
We now show that  that is computed as in (\ref{sec:design:algo:hull}) is indeed a -approximation of  for a given .




\begin{figure} [htbp]
\begin{center}
	\includegraphics[width=8cm]{linear_image.png}
\caption{The image computation under a linear dynamics.}
\label{fig:post:linear}
\end{center}
\end{figure}




\begin{lem}
Let  be the convex hull of . 
Then  is exactly the closed -neighborhood of the convex hull of .
\end{lem}
\begin{proof}
Suppose  and .
Then  for some  and  such that  and  for some  and .
Then there exists  such that

Thus  is in the -neighborhood of the convex hull of .

For the converse, consider  in the -neighborhood of the convex hull of .
Then for some , , , where .
Let .
Now .
So  is in the convex hull of .
However each .
Hence each  is in the convex hull of the vertices of  which is .
Thus  is in .
\end{proof}



Notice that the first update of  in {\tt Post()} is due to the computation of  from  over the time interval  under the linear dynamics of .
The second update after a deterministic and transversal discrete transition is due to a series of computations from  that is used to determine such a discrete transition to a new  that represents a reached states at time  right after a deterministic and transversal discrete transition.
As described in Lemma \ref{lem:cond:inexact:transv}, the steps involved during this discrete transition are to compute
\begin{inparaenum}[(i)]
	\item  from  and
	\item  from . 
\end{inparaenum}
Notice that (i) requires an intersection between a hyperplane and a polyhedron as well as a convex hull computation.
Moreover, for (ii), we need to compute a polyhedral image under the linear dynamics of a new location that is determined in {\tt Post()}.
Recall that we have derived a set of conditions in Lemmas \ref{lem:cond:inexact:istrans}, \ref{lem:cond:inexact:det}, and \ref{lem:cond:inexact:transv} to determine a deterministic and transversal discrete transition event.
These conditions are used in {\tt Post()} to determine such an event.
Furthermore, we also use conditions derived in Lemmas \ref{lem:cond:inexact:over} and \ref{lem:cond:inexact:eps}, to ensure that a set , which can be constructed as a collection of  as shown in the following theorem, satisfies the properties given in Remark \ref{rem:cond:inexact}.



Now, we present our main result for the problem of computing a bounded -reach set of a DTLHA.


\begin{thm}
Given input   for a problem to compute a bounded -reach set of a DTLHA , 
if Algorithm \ref{alg:main} returns {\tt done}, then a bounded -reach set of a DTLHA  defined over the continuous state domain  starting from an initial condition , denoted as , is the following:

for some  where  and  is the time at the -th discrete transition. 
\end{thm}
\begin{proof}
For each , 
\begin{inparaenum}
\item[(i)]  satisfies Lemma \ref{lem:cond:inexact:over}, and 
\item[(ii)]  satisfies Lemma \ref{lem:cond:inexact:eps}. 
\end{inparaenum}
Hence  is guaranteed to satisfy 
 and .
Furthermore, if a deterministic and transversal discrete transition is detected at the -th step by , then
\begin{inparaenum}
	\item[(iii)] by Lemmas \ref{lem:cond:inexact:istrans}, \ref{lem:cond:inexact:det}, and \ref{lem:cond:inexact:transv}, there is in fact a deterministic and transversal discrete transition in . 
\end{inparaenum}
This implies that a deterministic and transversal discrete transition event is correctly determined. 
Finally, if the proposed algorithm returns {\tt done}, then this implies that 
\begin{inparaenum}
	\item[(iv)] either  or . 
\end{inparaenum}
Hence,  is .
Therefore,  is a bounded -reach set of  from  by (i), (ii), (iii), and (iv).
\end{proof}




\section{Optimization and Implementation of the Proposed Algorithm}  \label{sec:imp}

A prototype software tool has been implemented, based on the architecture and the algorithm proposed in Section \ref{sec:design}, to demonstrate the idea of a bounded -reach set computation. 
In our implementation, we use the Multi-Parametric Toolbox \cite{mpt} for polyhedral operations and also use some built-in Matlab functions for other calculations. 


Notice that the size of the  right after a discrete transition increases roughly by the amount  through the computation of . 
This can potentially affect the capability to determine a discrete transition event.
Hence, we determine a smaller value of  to construct a tighter over-approximation of a discrete transition state. 
Suppose that a discrete transition from a location  to some other location  has already been determined by the proposed algorithm for given , , and  at some time .
Then the procedure for construction of a tight over-approximation of a discrete transition state  for some  can be improved shown in Algorithm \ref{alg:opt}.


\begin{algorithm} \SetAlgoLined 
\BlankLine
1. Partition the time interval  into a finite sequence of  for some , where

for some . \\
\BlankLine
2. Find a time  such that 
	\begin{itemize}
		\item  and
		\item , 
	\end{itemize}
	where . \\
\BlankLine
3. Construct  where . \\
\BlankLine
4. Compute an over-approximate discrete transition state

\caption{A procedure to compute a tight over-approximation of discrete transition state.}
\label{alg:opt}
\end{algorithm}




\subsection{An Example of Bounded -Reach Set Computation}  \label{sec:imp:example}

As an example to evaluate the proposed algorithm for a bounded -reach set computation of a DTLHA , we consider an LHA  over a  continuous state space  where
\begin{enumerate}[(i)]
	\item  ,
	\item  and  for each location  are defined as shown in Table \ref{tab:ex:Au},  
	\item The invariant set for each location , , is defined as shown in Fig. \ref{fig:eg:rho}, and
	\item  holds at the intersection between invariant sets of different locations. 
\end{enumerate}

Notice that all the LTI dynamics defined in the given LHA  are asymptotically stable.
Moreover, from the the vector fields determined by  and  for each , every discrete transition which occurs along the boundary of the invariant set between different locations is deterministic and transversal. 
Hence the given LHA  is in fact a DTLHA.



\begin{table}[htdp]
\caption{ and  for each  of }
\begin{center}
{\small
\begin{tabular}{c|p{0.15\textwidth}|c}
\toprule
 & \centering{} &  \\ 
\midrule
 & \centering{} &  \\ 
 & \centering{} &  \\ 
 & \centering{} &  \\ 
 & \centering{} &  \\ 
\bottomrule
\end{tabular}
}
\end{center}
\label{tab:ex:Au}
\end{table}

The bounded -reach set computation problem is specified by  where , ,  sec., , and .

In this example, we also assume that numerical calculation algorithms are available for basic calculations defined in Section \ref{sec:cond:basic} such that , , , and  where  is specified as .


\begin{figure}
\begin{center}
	\includegraphics[width=8.5cm]{case_02_loc.png}
\caption{A bounded -reach set of a DTLHA  starting from .}	
\label{fig:eg:rho}
\end{center}
\end{figure}



A policy that is used to choose values for  is as follows:
\begin{enumerate}[(i)]
	\item  is chosen in non-decreasing manner, 
	\item  to define a fixed sufficiently small ,
	\item , and
	\item  where  is as defined in (\ref{eq:h:ineq}).
\end{enumerate}
Notice that (i) means that whenever the proposed -reach set algorithm fails to continue its computation at the -th computation step, then the policy decides to restart the computation from the -th step with different values of the other parameters.
Recall that  denotes the approximation error of  when the algorithm computes  at time .
As shown in (iii), for a given , the policy chooses the largest value of  at each computation step.
The equation for  given in (iii) can easily be derived by considering 

If we upper bound the right hand side by , then we have (iii).


Fig. \ref{fig:eg:rho} shows the computation result.
As shown in Fig. \ref{fig:eg:rho}, a bounded -reach set is successfully computed.
In this example, the algorithm terminates at the computation step  right after the algorithm makes the tenth discrete transition from locations  to  at the time  sec. and {\tt jump }.
For given , the accumulated numerical calculation error  for  at this termination time is .




\section{Conclusion} \label{sec:con}
We have defined a special class of hybrid automata, called Deterministic and Transversal Linear Hybrid Automata (DTLHA), for which we can address the problem of bounded -reach set computation starting from an initial state.
For this class, we can also incorporate the impact of numerical calculation errors caused by finite precision numerical computation.

It is of importance to determine more  general and useful models of hybrid systems that permit computational verification of safety properties. Hybrid linear systems that incorporate linear models widely employed in control systems are a natural candidate around which to build such a theory of verification and validation.











