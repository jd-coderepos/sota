\def\year{2020}\relax
\documentclass[letterpaper]{article} \usepackage{aaai20}  \usepackage{times}  \usepackage{helvet} \usepackage{courier}  \usepackage[hyphens]{url}  \usepackage{graphicx} \urlstyle{rm} \def\UrlFont{\rm}  \usepackage{graphicx}  \frenchspacing  \setlength{\pdfpagewidth}{8.5in}  \setlength{\pdfpageheight}{11in}  \usepackage{comment}

\usepackage{lipsum, multicol}
\usepackage{epsfig}
\usepackage{amssymb}
\usepackage{adjustbox}
\usepackage{xcolor}
\usepackage{amsmath}
\usepackage{float}
\usepackage[utf8]{inputenc} \usepackage[T1]{fontenc}    \usepackage{url}            \usepackage{booktabs}       \usepackage{amsfonts}       \usepackage{nicefrac}       \usepackage{microtype}      \usepackage{algorithm}
\usepackage{graphicx} \usepackage{caption}
\usepackage{subfig}
\usepackage{algorithmic}
\usepackage{booktabs}
\usepackage{multirow}
\usepackage[hyphenbreaks]{breakurl}
\usepackage[hyphens]{url}
\usepackage{adjustbox}
\usepackage{wrapfig}





\newcommand{\eat}[1]{}

\newcommand{\Cy}{\Ccal_y}
\renewcommand{\l}{\boldsymbol{\ell}}
\renewcommand{\S}
{\!\!\,\includegraphics[height=1em]{figs/arcsinh}\!\!}
\def\Gt{\widetilde{G}}
\def\ah{\widehat{a}}
\def\ahb{\mathbf{\ah}}
\def\at{\widetilde{a}}
\def\atb{\mathbf{\at}}
\def\yh{\widehat{y}}
\def\yt{\widetilde{y}}
\def\ytb{\mathbf{\yt}}
\def\yhb{\mathbf{\yh}}
\def\zh{\widehat{z}}
\def\zt{\widetilde{z}}
\def\ztb{\mathbf{\zt}}
\def\zhb{\mathbf{\zh}}
\def\g{\mathbf{g}}
\def\bah{\mathbf{\widehat{a}}}
\RequirePackage{latexsym}
\RequirePackage{amsmath}
\RequirePackage{amssymb}
\RequirePackage{bm}

\definecolor{magenta}{cmyk}{0,1,0,0}
\definecolor{mygreen}{rgb}{.1,1,.1}
\newcommand{\Magenta}[1]{\color{magenta}{#1}\color{black}}
\newcommand{\Blue}[1]{\textcolor{blue}{#1}}
\newcommand{\Green}[1]{\textcolor{mygreen}{#1}}
\newcommand{\Red}[1]{\textcolor{red}{#1}}
\newcommand{\Cyan}[1]{\textcolor{cyan}{#1}}
\newcommand{\Orange}[1]{\textcolor{orange}{#1}}
\newcommand{\Brown}[1]{\textcolor{brown}{#1}}
\newcommand{\Ref}[1]{\hfill\Green{[#1]}}
\DeclareMathOperator{\arginf}{arg\,inf}


\newcommand{\rmd}{\mathrm{d}} 

\newcommand{\Tr}{^\mathsf{T}} 




\newcommand{\mnote}[1]{{\bf\large *}\marginpar{\Magenta{#1}}}
\newcommand{\dnote}[1]{{\bf\large +}\marginpar{\tiny#1}}
\ifx\note\undefined
\newcommand{\note}[1]{{\bf \Magenta{#1}}}
\else
\renewcommand{\note}[1]{{\bf \Magenta{#1}}}
\fi

\def\x{\mathbf{x}}
\def\a{\mathbf{a}}
\def\y{\mathbf{y}}
\def\z{\mathbf{z}}
\def\w{\mathbf{w}}
\def\t{\boldsymbol{\theta}}
\def\R{\mathbb{R}}



\DeclareMathOperator{\ab}{\mathbf{a}}
\DeclareMathOperator{\bb}{\mathbf{b}}
\DeclareMathOperator{\cbb}{\mathbf{c}}
\DeclareMathOperator{\db}{\mathbf{d}}
\DeclareMathOperator{\eb}{\mathbf{e}}
\DeclareMathOperator{\fb}{\mathbf{f}}
\DeclareMathOperator{\gb}{\mathbf{g}}
\DeclareMathOperator{\hb}{\mathbf{h}}
\DeclareMathOperator{\ib}{\mathbf{i}}
\DeclareMathOperator{\jb}{\mathbf{j}}
\DeclareMathOperator{\kb}{\mathbf{k}}
\DeclareMathOperator{\lb}{\mathbf{l}}
\DeclareMathOperator{\mb}{\mathbf{m}}
\DeclareMathOperator{\nbb}{\mathbf{n}}
\DeclareMathOperator{\ob}{\mathbf{o}}
\DeclareMathOperator{\pb}{\mathbf{p}}
\DeclareMathOperator{\qb}{\mathbf{q}}
\DeclareMathOperator{\rb}{\mathbf{r}}
\DeclareMathOperator{\sbb}{\mathbf{s}}
\DeclareMathOperator{\tb}{\mathbf{t}}
\DeclareMathOperator{\ub}{\mathbf{u}}
\DeclareMathOperator{\vb}{\mathbf{v}}
\DeclareMathOperator{\wb}{\mathbf{w}}
\DeclareMathOperator{\xb}{\mathbf{x}}
 \DeclareMathOperator{\yb}{\mathbf{y}}
\DeclareMathOperator{\zb}{\mathbf{z}}

\DeclareMathOperator{\pr}{\mathrm{p}}

\DeclareMathOperator{\deltab}{\boldsymbol{\delta}}

\DeclareMathOperator{\atilde}{\tilde{\ab}}
\DeclareMathOperator{\btilde}{\tilde{\bb}}
\DeclareMathOperator{\ctilde}{\tilde{\cb}}
\DeclareMathOperator{\dtilde}{\tilde{\db}}
\DeclareMathOperator{\etilde}{\tilde{\eb}}
\DeclareMathOperator{\ftilde}{\tilde{\fb}}
\DeclareMathOperator{\gtilde}{\tilde{\gb}}
\DeclareMathOperator{\htilde}{\tilde{\hb}}
\DeclareMathOperator{\itilde}{\tilde{\ib}}
\DeclareMathOperator{\jtilde}{\tilde{\jb}}
\DeclareMathOperator{\ktilde}{\tilde{\kb}}
\DeclareMathOperator{\ltilde}{\tilde{\lb}}
\DeclareMathOperator{\mtilde}{\tilde{\mb}}
\DeclareMathOperator{\ntilde}{\tilde{\nbb}}
\DeclareMathOperator{\otilde}{\tilde{\ob}}
\DeclareMathOperator{\qtilde}{\tilde{\qb}}
\DeclareMathOperator{\rtilde}{\tilde{\rb}}
\DeclareMathOperator{\stilde}{\tilde{\sbb}}
\DeclareMathOperator{\ttilde}{\tilde{\tb}}
\DeclareMathOperator{\utilde}{\tilde{\ub}}
\DeclareMathOperator{\vtilde}{\tilde{\vb}}
\DeclareMathOperator{\wtilde}{\tilde{\wb}}
\DeclareMathOperator{\xtilde}{\tilde{\xb}}
\DeclareMathOperator{\ytilde}{\tilde{\yb}}
\DeclareMathOperator{\ztilde}{\tilde{\zb}}

\DeclareMathOperator{\abar}{\bar{\ab}}
\DeclareMathOperator{\bbar}{\bar{\bb}}
\DeclareMathOperator{\cbar}{\bar{\cb}}
\DeclareMathOperator{\dbar}{\bar{\db}}
\DeclareMathOperator{\ebar}{\bar{\eb}}
\DeclareMathOperator{\fbar}{\bar{\fb}}
\DeclareMathOperator{\gbar}{\bar{\gb}}
\DeclareMathOperator{\hbbar}{\bar{\hb}}
\DeclareMathOperator{\ibar}{\bar{\ib}}
\DeclareMathOperator{\jbar}{\bar{\jb}}
\DeclareMathOperator{\kbar}{\bar{\kb}}
\DeclareMathOperator{\lbar}{\bar{\lb}}
\DeclareMathOperator{\mbar}{\bar{\mb}}
\DeclareMathOperator{\nbar}{\bar{\nbb}}
\DeclareMathOperator{\obar}{\bar{\ob}}
\DeclareMathOperator{\pbar}{\bar{\pb}}
\DeclareMathOperator{\qbar}{\bar{\qb}}
\DeclareMathOperator{\rbar}{\bar{\rb}}
\DeclareMathOperator{\sbar}{\bar{\sbb}}
\DeclareMathOperator{\tbar}{\bar{\tb}}
\DeclareMathOperator{\ubar}{\bar{\ub}}
\DeclareMathOperator{\vbar}{\bar{\vb}}
\DeclareMathOperator{\wbar}{\bar{\wb}}
\DeclareMathOperator{\xbar}{\bar{\xb}}
\DeclareMathOperator{\ybar}{\bar{\yb}}
\DeclareMathOperator{\zbar}{\bar{\zb}}

\DeclareMathOperator{\Ab}{\mathbf{A}}
\DeclareMathOperator{\Bb}{\mathbf{B}}
\DeclareMathOperator{\Cb}{\mathbf{C}}
\DeclareMathOperator{\Db}{\mathbf{D}}
\DeclareMathOperator{\Eb}{\mathbf{E}}
\DeclareMathOperator{\Fb}{\mathbf{F}}
\DeclareMathOperator{\Gb}{\mathbf{G}}
\DeclareMathOperator{\Hb}{\mathbf{H}}
\DeclareMathOperator{\Ib}{\mathbf{I}}
\DeclareMathOperator{\Jb}{\mathbf{J}}
\DeclareMathOperator{\Kb}{\mathbf{K}}
\DeclareMathOperator{\Lb}{\mathbf{L}}
\DeclareMathOperator{\Mb}{\mathbf{M}}
\DeclareMathOperator{\Nb}{\mathbf{N}}
\DeclareMathOperator{\Ob}{\mathbf{O}}
\DeclareMathOperator{\Pb}{\mathbf{P}}
\DeclareMathOperator{\Qb}{\mathbf{Q}}
\DeclareMathOperator{\Rb}{\mathbf{R}}
\DeclareMathOperator{\Sbb}{\mathbf{S}}
\DeclareMathOperator{\Tb}{\mathbf{T}}
\DeclareMathOperator{\Ub}{\mathbf{U}}
\DeclareMathOperator{\Vb}{\mathbf{V}}
\DeclareMathOperator{\Wb}{\mathbf{W}}
\DeclareMathOperator{\Xb}{\mathbf{X}}
\DeclareMathOperator{\Yb}{\mathbf{Y}}
\DeclareMathOperator{\Zb}{\mathbf{Z}}

\DeclareMathOperator{\Abar}{\bar{A}}
\DeclareMathOperator{\Bbar}{\bar{B}}
\DeclareMathOperator{\Cbar}{\bar{C}}
\DeclareMathOperator{\Dbar}{\bar{D}}
\DeclareMathOperator{\Ebar}{\bar{E}}
\DeclareMathOperator{\Fbar}{\bar{F}}
\DeclareMathOperator{\Gbar}{\bar{G}}
\DeclareMathOperator{\Hbar}{\bar{H}}
\DeclareMathOperator{\Ibar}{\bar{I}}
\DeclareMathOperator{\Jbar}{\bar{J}}
\DeclareMathOperator{\Kbar}{\bar{K}}
\DeclareMathOperator{\Lbar}{\bar{L}}
\DeclareMathOperator{\Mbar}{\bar{M}}
\DeclareMathOperator{\Nbar}{\bar{N}}
\DeclareMathOperator{\Obar}{\bar{O}}
\DeclareMathOperator{\Pbar}{\bar{P}}
\DeclareMathOperator{\Qbar}{\bar{Q}}
\DeclareMathOperator{\Rbar}{\bar{R}}
\DeclareMathOperator{\Sbar}{\bar{S}}
\DeclareMathOperator{\Tbar}{\bar{T}}
\DeclareMathOperator{\Ubar}{\bar{U}}
\DeclareMathOperator{\Vbar}{\bar{V}}
\DeclareMathOperator{\Wbar}{\bar{W}}
\DeclareMathOperator{\Xbar}{\bar{X}}
\DeclareMathOperator{\Ybar}{\bar{Y}}
\DeclareMathOperator{\Zbar}{\bar{Z}}

\DeclareMathOperator{\Abbar}{\bar{\Ab}}
\DeclareMathOperator{\Bbbar}{\bar{\Bb}}
\DeclareMathOperator{\Cbbar}{\bar{\Cb}}
\DeclareMathOperator{\Dbbar}{\bar{\Db}}
\DeclareMathOperator{\Ebbar}{\bar{\Eb}}
\DeclareMathOperator{\Fbbar}{\bar{\Fb}}
\DeclareMathOperator{\Gbbar}{\bar{\Gb}}
\DeclareMathOperator{\Hbbar}{\bar{\Hb}}
\DeclareMathOperator{\Ibbar}{\bar{\Ib}}
\DeclareMathOperator{\Jbbar}{\bar{\Jb}}
\DeclareMathOperator{\Kbbar}{\bar{\Kb}}
\DeclareMathOperator{\Lbbar}{\bar{\Lb}}
\DeclareMathOperator{\Mbbar}{\bar{\Mb}}
\DeclareMathOperator{\Nbbar}{\bar{\Nb}}
\DeclareMathOperator{\Obbar}{\bar{\Ob}}
\DeclareMathOperator{\Pbbar}{\bar{\Pb}}
\DeclareMathOperator{\Qbbar}{\bar{\Qb}}
\DeclareMathOperator{\Rbbar}{\bar{\Rb}}
\DeclareMathOperator{\Sbbar}{\bar{\Sb}}
\DeclareMathOperator{\Tbbar}{\bar{\Tb}}
\DeclareMathOperator{\Ubbar}{\bar{\Ub}}
\DeclareMathOperator{\Vbbar}{\bar{\Vb}}
\DeclareMathOperator{\Wbbar}{\bar{\Wb}}
\DeclareMathOperator{\Xbbar}{\bar{\Xb}}
\DeclareMathOperator{\Ybbar}{\bar{\Yb}}
\DeclareMathOperator{\Zbbar}{\bar{\Zb}}

\DeclareMathOperator{\Ahat}{\widehat{A}}
\DeclareMathOperator{\Bhat}{\widehat{B}}
\DeclareMathOperator{\Chat}{\widehat{C}}
\DeclareMathOperator{\Dhat}{\widehat{D}}
\DeclareMathOperator{\Ehat}{\widehat{E}}
\DeclareMathOperator{\Fhat}{\widehat{F}}
\DeclareMathOperator{\Ghat}{\widehat{G}}
\DeclareMathOperator{\Hhat}{\widehat{H}}
\DeclareMathOperator{\Ihat}{\widehat{I}}
\DeclareMathOperator{\Jhat}{\widehat{J}}
\DeclareMathOperator{\Khat}{\widehat{K}}
\DeclareMathOperator{\Lhat}{\widehat{L}}
\DeclareMathOperator{\Mhat}{\widehat{M}}
\DeclareMathOperator{\Nhat}{\widehat{N}}
\DeclareMathOperator{\Ohat}{\widehat{O}}
\DeclareMathOperator{\Phat}{\widehat{P}}
\DeclareMathOperator{\Qhat}{\widehat{Q}}
\DeclareMathOperator{\Rhat}{\widehat{R}}
\DeclareMathOperator{\Shat}{\widehat{S}}
\DeclareMathOperator{\That}{\widehat{T}}
\DeclareMathOperator{\Uhat}{\widehat{U}}
\DeclareMathOperator{\Vhat}{\widehat{V}}
\DeclareMathOperator{\What}{\widehat{W}}
\DeclareMathOperator{\Xhat}{\widehat{X}}
\DeclareMathOperator{\Yhat}{\widehat{Y}}
\DeclareMathOperator{\Zhat}{\widehat{Z}}

\DeclareMathOperator{\Abhat}{\widehat{\Ab}}
\DeclareMathOperator{\Bbhat}{\widehat{\Bb}}
\DeclareMathOperator{\Cbhat}{\widehat{\Cb}}
\DeclareMathOperator{\Dbhat}{\widehat{\Db}}
\DeclareMathOperator{\Ebhat}{\widehat{\Eb}}
\DeclareMathOperator{\Fbhat}{\widehat{\Fb}}
\DeclareMathOperator{\Gbhat}{\widehat{\Gb}}
\DeclareMathOperator{\Hbhat}{\widehat{\Hb}}
\DeclareMathOperator{\Ibhat}{\widehat{\Ib}}
\DeclareMathOperator{\Jbhat}{\widehat{\Jb}}
\DeclareMathOperator{\Kbhat}{\widehat{\Kb}}
\DeclareMathOperator{\Lbhat}{\widehat{\Lb}}
\DeclareMathOperator{\Mbhat}{\widehat{\Mb}}
\DeclareMathOperator{\Nbhat}{\widehat{\Nb}}
\DeclareMathOperator{\Obhat}{\widehat{\Ob}}
\DeclareMathOperator{\Pbhat}{\widehat{\Pb}}
\DeclareMathOperator{\Qbhat}{\widehat{\Qb}}
\DeclareMathOperator{\Rbhat}{\widehat{\Rb}}
\DeclareMathOperator{\Sbhat}{\widehat{\Sb}}
\DeclareMathOperator{\Tbhat}{\widehat{\Tb}}
\DeclareMathOperator{\Ubhat}{\widehat{\Ub}}
\DeclareMathOperator{\Vbhat}{\widehat{\Vb}}
\DeclareMathOperator{\Wbhat}{\widehat{\Wb}}
\DeclareMathOperator{\Xbhat}{\widehat{\Xb}}
\DeclareMathOperator{\Ybhat}{\widehat{\Yb}}
\DeclareMathOperator{\Zbhat}{\widehat{\Zb}}

\DeclareMathOperator{\Acal}{\mathcal{A}}
\DeclareMathOperator{\Bcal}{\mathcal{B}}
\DeclareMathOperator{\Ccal}{\mathcal{C}}
\DeclareMathOperator{\Dcal}{\mathcal{D}}
\DeclareMathOperator{\Ecal}{\mathcal{E}}
\DeclareMathOperator{\Fcal}{\mathcal{F}}
\DeclareMathOperator{\Gcal}{\mathcal{G}}
\DeclareMathOperator{\Hcal}{\mathcal{H}}
\DeclareMathOperator{\Ical}{\mathcal{I}}
\DeclareMathOperator{\Jcal}{\mathcal{J}}
\DeclareMathOperator{\Kcal}{\mathcal{K}}
\DeclareMathOperator{\Lcal}{\mathcal{L}}
\DeclareMathOperator{\Mcal}{\mathcal{M}}
\DeclareMathOperator{\Ncal}{\mathcal{N}}
\DeclareMathOperator{\Ocal}{\mathcal{O}}
\DeclareMathOperator{\Pcal}{\mathcal{P}}
\DeclareMathOperator{\Qcal}{\mathcal{Q}}
\DeclareMathOperator{\Rcal}{\mathcal{R}}
\DeclareMathOperator{\Scal}{\mathcal{S}}
\DeclareMathOperator{\Tcal}{\mathcal{T}}
\DeclareMathOperator{\Ucal}{\mathcal{U}}
\DeclareMathOperator{\Vcal}{\mathcal{V}}
\DeclareMathOperator{\Wcal}{\mathcal{W}}
\DeclareMathOperator{\Xcal}{\mathcal{X}}
\DeclareMathOperator{\Ycal}{\mathcal{Y}}
\DeclareMathOperator{\Zcal}{\mathcal{Z}}

\DeclareMathOperator{\Atilde}{\widetilde{A}}
\DeclareMathOperator{\Btilde}{\widetilde{B}}
\DeclareMathOperator{\Ctilde}{\widetilde{C}}
\DeclareMathOperator{\Dtilde}{\widetilde{D}}
\DeclareMathOperator{\Etilde}{\widetilde{E}}
\DeclareMathOperator{\Ftilde}{\widetilde{F}}
\DeclareMathOperator{\Gtilde}{\widetilde{G}}
\DeclareMathOperator{\Htilde}{\widetilde{H}}
\DeclareMathOperator{\Itilde}{\widetilde{I}}
\DeclareMathOperator{\Jtilde}{\widetilde{J}}
\DeclareMathOperator{\Ktilde}{\widetilde{K}}
\DeclareMathOperator{\Ltilde}{\widetilde{L}}
\DeclareMathOperator{\Mtilde}{\widetilde{M}}
\DeclareMathOperator{\Ntilde}{\widetilde{N}}
\DeclareMathOperator{\Otilde}{\widetilde{O}}
\DeclareMathOperator{\Ptilde}{\widetilde{P}}
\DeclareMathOperator{\Qtilde}{\widetilde{Q}}
\DeclareMathOperator{\Rtilde}{\widetilde{R}}
\DeclareMathOperator{\Stilde}{\widetilde{S}}
\DeclareMathOperator{\Ttilde}{\widetilde{T}}
\DeclareMathOperator{\Utilde}{\widetilde{U}}
\DeclareMathOperator{\Vtilde}{\widetilde{V}}
\DeclareMathOperator{\Wtilde}{\widetilde{W}}
\DeclareMathOperator{\Xtilde}{\widetilde{X}}
\DeclareMathOperator{\Ytilde}{\widetilde{Y}}
\DeclareMathOperator{\Ztilde}{\widetilde{Z}}




\DeclareMathOperator{\CC}{\mathbb{C}} \DeclareMathOperator{\EE}{\mathbb{E}} \DeclareMathOperator{\KK}{\mathbb{K}} \DeclareMathOperator{\MM}{\mathbb{M}} \DeclareMathOperator{\NN}{\mathbb{N}} \DeclareMathOperator{\PP}{\mathbb{P}} \DeclareMathOperator{\QQ}{\mathbb{Q}} \DeclareMathOperator{\RR}{\mathbb{R}} \DeclareMathOperator{\ZZ}{\mathbb{Z}} 

\DeclareMathOperator{\one}{\mathbf{1}}  \DeclareMathOperator{\zero}{\mathbf{0}} 

\DeclareMathOperator*{\mini}{\mathop{\mathrm{minimize}}}
\DeclareMathOperator*{\maxi}{\mathop{\mathrm{maximize}}}
\DeclareMathOperator*{\argmin}{\mathop{\mathrm{argmin}}}
\DeclareMathOperator*{\argsup}{\mathop{\mathrm{argsup}}}
\DeclareMathOperator*{\arcsinh}{\mathop{\mathrm{arcsinh}}}
\DeclareMathOperator*{\limit}{\mathop{\mathrm{limit}}}
\DeclareMathOperator*{\argmax}{\mathop{\mathrm{argmax}}}
\DeclareMathOperator{\sgn}{\mathop{\mathrm{sign}}}
\DeclareMathOperator{\tr}{\mathop{\mathrm{tr}}}
\DeclareMathOperator{\rank}{\mathop{\mathrm{rank}}}
\DeclareMathOperator{\traj}{\mathop{\mathrm{Traj}}}
\DeclareMathOperator{\diag}{diag}

\newcommand{\sigmab}{\bm{\sigma}}
\newcommand{\Sigmab}{\mathbf{\Sigma}}
\newcommand{\Thetab}{{\bm{\Theta}}}
\newcommand{\thetab}{{\bm{\theta}}}
\newcommand{\xib}{{\bm{\xi}}}
\newcommand{\Xib}{{\bm{\Xi}}}
\newcommand{\zetab}{{\bm{\zeta}}}
\newcommand{\alphab}{{\bm{\alpha}}}
\newcommand{\taub}{{\bm{\tau}}}
\newcommand{\etab}{{\bm{\eta}}}


\DeclareMathOperator{\alphahat}{\widehat{\alpha}}




\ifx\BlackBox\undefined
\newcommand{\BlackBox}{\rule{1.5ex}{1.5ex}}  \fi

\ifx\proof\undefined
\newenvironment{proof}{\par\noindent{\bf Proof\ }}{\hfill\BlackBox\2mm]}
\newcommand{\fs}{\8mm]}
\newcommand{\bra}[1]
       {\left. \begin{array}{c} \\label{eq:generator}
    (\tilde{\xb}_1, \cdots, \tilde{\xb}_{T}) = G_{\thetab}(\yb, \xib)~,
\label{eq:sltc}
    (\hb_1, \cdots, \hb_T) = \LSTM\left(\xib_1, \cdots, \xib_T, \yb; \thetab_1\right)~,

\label{eq:latent}
    (\vb_1, \cdots, \vb_T) &= \LSTM\left(\xib_1, \cdots, \xib_T, \yb; \thetab_1\right) \nonumber \\
    \hb_{t+1} &= \hb_{t} + \vb_t~.

    \tilde{\xb}_t = \Dec(\hb_t, \yb; \thetab_2)~,
\label{eq:reg1}
    \Omega(\{\hb_t\}, \{\tilde{\xb}_t\}) \triangleq \sum_{t=2}^T \left(\sigma_1 \|\hb_t - \hb_{t-1}\|^2 + \sigma_2\|\tilde{\xb}_t - \tilde{\xb}_{t-1}\|^2\right)
\label{eq:loss}
    \min_{G_{\thetab}, C_{\phib}}\max_{D_{\psib}}&V(G_{\thetab}, C_{\phib}, D_{\psib}) = \mathbb{E}_{\xb\sim p(\xb), \tilde{\yb} \sim p_{\phib}(\cdot|\xb)}\left[\log D_{\psib}(\xb, \tilde{\yb})\right] \nonumber \\
    &+ \mathbb{E}_{y \sim q(y), \tilde{\xb} \sim q_{\thetab}(\cdot|\yb)}\left[\log(1 - D_{\psib}(\tilde{\xb}, \yb))\right]~.
\label{eq:reg}
    \mathcal{L}_c \triangleq H(C_{\phib}(G_{\thetab}(\yb, \xib)), \yb)~,
\label{Eq1_wgan}
    &\underset{\bar{G}_{\thetab_2}}{\min} \;\underset{\bar{D}}{\max} \;\bar{V}(\bar{D}, \bar{G}_{\thetab_2}) = \mathbb{E}_{\xb\sim p_{\text{data}}}[\bar{D}(\xb)] -  \\ 
    &\mathbb{E}_{\hb\sim p_h(\hb)}[\bar{D}(\bar{G}_{\thetab_2}(\hb))]  +  \lambda \mathbb{E}_{\xb\sim p_{\text{data}}}[{(\parallel \bigtriangledown_{\xb} \bar{D}(\xb)\parallel}_2 - 1)^2] \nonumber

                 \hspace{-0.4cm}\nabla_{\psib} \frac{1}{m} \sum\limits_{i=1}^{m}[\log D_{\psib}(\xb^{(i)}, \yb^{(i)})+ \log(1-D_{\psib}(G_{\thetab}(\xib^{i}, \yb^{(i)})))]
             
            \hspace{-0.3cm}\nabla_{\thetab, \phib} &\frac{1}{m} \sum\limits_{i=1}^{m}\left[\log(1-D_{\psib}(G_{\thetab}(\xib^{i}, y^{(i)})))+
             \Omega(\{\hb_t^{(i)}\}, \{\xb_t^{(i)}\})\right. \\
            & \left. + \gamma\left(H(C_{\phib}(\xb^{(i)}), \yb^{(i)})+  H(C_{\phib}(G_{\thetab}(\xib^{i}, \yb^{(i)})), \yb^{(i)})\right)\right]
            
    \MMD_{\text{avg}} &\triangleq \dfrac{1}{SK} \Sigma_{j=1}^K\Sigma_{i=1}^S \MMD_u \left[ \mathcal{F}, X_i^j, Y_i^j\right] \\
    \MMD_{\text{seq}} &\triangleq \dfrac{1}{K} \Sigma_{j=1}^K \MMD_u \left[ \mathcal{F}, X^j_{\text{seq}}, Y^j_{\text{seq}}\right]~,

    \MMD_u^2 \left[ \mathcal{F}, X, Y\right]\triangleq \dfrac{1}{m(m-1)}\Sigma_{i=1}^m\Sigma_{j\neq i}^m k(x_i, x_j)
    \\+ \dfrac{1}{n(n-1)}\Sigma_{i=1}^n\Sigma_{j\neq i}^n k(y_i, y_j)
    - \dfrac{2}{mn}\Sigma_{i=1}^m\Sigma_{j=1}^n k(x_i, y_j)~

with  the Gaussian kernel.




\section{Implementation Details}\label{appendix: implementation details}

We use Adam optimization algorithm \cite{adam2015} for learning the whole network parameters. The weight of gradient penalty  for the WGAN-GP is set to be 10. The learning rate for optimizing the generator and discriminator loss of the WGAN-GP is set to be 0.001. The learning rate is set to be 0.0001 for optimizing the generator and discriminator loss of our sequence generation model. The weight  for the cycle consistency loss is set to be 0.1 and the weight  and  for regularization loss are set to be 0.05 and 0.00005 respectively. The number of hidden units for the fully connected layers is set to 1024 for the LSTM discriminator and classifier. The number of hidden units for the LSTM generator is set to be 256.

\section{More Experiment Results}\label{appendix: more experiment results}
The novel actions mixed with the two actions "Throw" and "Kick" trained on NTU RGB+D dataset are shown in Figure \ref{fig:ntu_mixvis2}. It is also interesting to see that the sequence with mixing classes indeed contains actions with both hands and legs, which correspond to "Throw" and "Kick", respectively. Diverse generated latent space with different latent dimensions are shown in Figure \ref{fig:diversity_1}. Figure \ref{fig:h36_vis_1} and \ref{fig:ntu_vis1_1} shows other examples of generated action sequences on the two datasets. Table \ref{tab:variance} shows the standard deviation of the distance of the generated action sequences to the mean action for each action class.

Figure \ref{fig:h36_diverse_eating}, \ref{fig:h36_diverse_smoking}, \ref{fig:h36_diverse_direction} shows more diverse action generation results on human3.6 dataset.

For latent dimensions larger than 2, the latent trajectories from different classes are found to be separated quite well. Figure \ref{fig:latdimall_appendix} shows more visualization results of latent space in higher dimension. Apart from the impact of tSNE, we suspect in a higher latent space, our model is flexible enough to separate trajectories of different classes, as the input of the generator contains label information.




\section{Ablation Study}\label{appendix: ablation study}

We run ablation experiments over the components of our model to better understand the effects of each component.

\paragraph{Effect of smooth regularizer:} We remove the regularizer for consecutive frames. Table \ref{tab:ablation_smooth} shows the ablation study for smoothness term in the loss function. "no smoothness" means that we remove both the regularization term on the latent and action space. "only action" means that we only add regularizer on the action space. "only latent" means that we only add regularizer on the latent space. "latent and action" means that we add regularizer on both action and latent space.

The results show that the regularization of both latent space and real action space is better than only regularize the latent or real action space or no regularization at all.

\paragraph{Effect of cycle consistency loss:} We add or remove the classifier in the model, i.e., add or remove the cycle consistency loss. Table \ref{tab:2 similarity} shows the results of the ablation study for the cycle consistency loss, which is shown in Equation \eqref{eq:reg}.

\paragraph{Effect of transition residual:} We study the comparison of predicting the latent transition residual and predicting the latent transition directly. Table \ref{tab:ablation_res} shows the results of the ablation study for latent transition residual. "direct latent" means that we use LSTM to predict the latent transition directly. "residual latent" means that we use LSTM to predict the residual of latent transition as in Equation \eqref {eq:latent}. 

\section{Human Evaluation}\label{appendix: human evaluation}
Figure~\ref{fig:human evaluation} shows experiment design for Human Evaluation. At each page, we give five scoring standard with word descriptions as well as video examples to make sure workers share the same grading standard. After that, we provide four sample videos from each model for the blind test.

\begin{figure*}[h!]
    \centering
\includegraphics[width=1\textwidth]{human_evaluation.png}
    \caption{Human evaluation screenshot for Amazon Mechanical Turk.}
    \label{fig:human evaluation}
\end{figure*}

\begin{figure*}[t]
	\centering
\subfloat[Direction]{
	    \label{}
		\includegraphics[width=1\textwidth]{h36_direction.png}
	}
		\hfill
    
	\subfloat[Greeting]{
	    \label{}
		\includegraphics[width=1\textwidth]{hu36_greeting.png}
	}
		\hfill
			\subfloat[Sitting]{
	    \label{}
		\includegraphics[width=1\textwidth]{h36_sitting.png}
	}
	\hfill
		\subfloat[Eating]{
	    \label{}
		\includegraphics[width=1\textwidth]{hu36_eating.png}
	
	}
	\caption{Several action sequences generated by training on human3.6 dataset} \label{fig:h36_vis_1}
\end{figure*}

\begin{figure*}[t]
	\centering
\subfloat[Hand waving]{
	    \label{}
		\includegraphics[width=1\textwidth]{ntu_handwaving.png}
	}
		\hfill
			\subfloat[throw]{
	    \label{}
		\includegraphics[width=1\textwidth]{ntu_throw2.png}
	}
	\hfill
		\subfloat[Sit down]{
	    \label{}
		\includegraphics[width=1\textwidth]{ntu_sitdown.png}
	
	}
	\caption{Several action sequences generated by training on NTU RGBD dataset} \label{fig:ntu_vis1_1}
\end{figure*}



\begin{figure*}[t]
	\centering
\subfloat[Eating sequence 1]{
	    \label{}
		\includegraphics[width=1\textwidth]{h36_eating1.png}
	}
		\hfill
    
	\subfloat[Eating sequence 2]{
	    \label{}
		\includegraphics[width=1\textwidth]{h36_eating2.png}
	}
		\hfill
			\subfloat[Eating sequence 3]{
	    \label{}
		\includegraphics[width=1\textwidth]{h36_eating3.png}
	}
	\hfill
		\subfloat[Eating sequence 4]{
	    \label{}
		\includegraphics[width=1\textwidth]{h36_eating4.png}
	
	}
	
		\hfill
		\subfloat[Eating sequence 5]{
	    \label{}
		\includegraphics[width=1\textwidth]{h36_eating5.png}
	
	}
	
	\caption{Diverse action sequences generated by training on human3.6 dataset} \label{fig:h36_diverse_eating}
\end{figure*}





\begin{figure*}[t]
	\centering
\subfloat[Smoking sequence 1]{
	    \label{}
		\includegraphics[width=1\textwidth]{h36_smoking1.png}
	}
		\hfill
    
	\subfloat[Smoking sequence 2]{
	    \label{}
		\includegraphics[width=1\textwidth]{h36_smoking2.png}
	}
		\hfill
			\subfloat[Smoking sequence 3]{
	    \label{}
		\includegraphics[width=1\textwidth]{h36_smoking3.png}
	}
	\hfill
		\subfloat[Smoking sequence 4]{
	    \label{}
		\includegraphics[width=1\textwidth]{h36_smoking4.png}
	
	}
	
		\hfill
		\subfloat[Smoking sequence 5]{
	    \label{}
		\includegraphics[width=1\textwidth]{h36_smoking5.png}
	
	}
	
	\caption{Diverse action sequences generated by training on human3.6 dataset} \label{fig:h36_diverse_smoking}
\end{figure*}




\begin{figure*}[t]
	\centering
\subfloat[Direction sequence 1]{
	    \label{}
		\includegraphics[width=1\textwidth]{h36_direction1.png}
	}
		\hfill
    
	\subfloat[Direction sequence 2]{
	    \label{}
		\includegraphics[width=1\textwidth]{h36_direction2.png}
	}
		\hfill
			\subfloat[Direction sequence 3]{
	    \label{}
		\includegraphics[width=1\textwidth]{h36_direction3.png}
	}
	\hfill
		\subfloat[Direction sequence 4]{
	    \label{}
		\includegraphics[width=1\textwidth]{h36_direction4.png}
	
	}
	
		\hfill
		\subfloat[Direction sequence 5]{
	    \label{}
		\includegraphics[width=1\textwidth]{h36_direction5.png}
	
	}
	
	\caption{Diverse action sequences generated by training on human3.6 dataset} \label{fig:h36_diverse_direction}
\end{figure*}



\begin{figure*}
    \centering
    \subfloat[walking]{\includegraphics[width=0.23\textwidth]{lat2_walk.png}}
    \hfill
    \subfloat[smoking]{\includegraphics[width=0.23\textwidth]{lat2_smoke.png}}
    \hfill
    \subfloat[greeting]{\includegraphics[width=0.23\textwidth]{lat2_greeting.png}}
    \hfill
    \subfloat[posing]{\includegraphics[width=0.23\textwidth]{lat2_posing.png}}
    \vfill
    \subfloat[walking]{\includegraphics[width=0.23\textwidth]{lat6_walk.png}}
    \hfill
      \subfloat[smoking]{\includegraphics[width=0.23\textwidth]{lat6_smoking.png}}
    \hfill
     \subfloat[greeting]{\includegraphics[width=0.23\textwidth]{lat6_greeting.png}}
    \hfill
     \subfloat[posing]{\includegraphics[width=0.23\textwidth]{lat6_posing.png}}
    \vfill
    \subfloat[walking]{\includegraphics[width=0.23\textwidth]{lat12_walk.png}}
    \hfill
      \subfloat[smoking]{\includegraphics[width=0.23\textwidth]{lat12_smoke.png}}
    \hfill
     \subfloat[greeting]{\includegraphics[width=0.23\textwidth]{lat12_greeting.png}}
    \hfill
     \subfloat[posing]{\includegraphics[width=0.23\textwidth]{lat12_posing.png}}
    \hfill
    \caption{\label{fig:diversity_1}
    Action diversity of generated latent space with different latent dimensions. The latent space dimension of first row, second row, third row is 2, 6 and 12 respectively.}
    \vspace{-0.2cm}
\end{figure*}


\begin{figure*}[t]
	\centering


	\begin{minipage}{0.46\linewidth}
        \includegraphics[width=\textwidth]{lat6.png}
    \end{minipage}
    \begin{minipage}{0.46\linewidth}
        \includegraphics[width=\textwidth]{lat12.png}
    \end{minipage}
	\caption{Latent space with different dimensions. Left: dim = 6; Right: dim = 12.} \label{fig:latdimall_appendix}
\end{figure*}


\begin{figure*}[t]
	\centering
\begin{minipage}{\linewidth}
        \includegraphics[width=1.0\linewidth]{ntu_throw.png}
    \end{minipage}
    \begin{minipage}{\linewidth}
        \includegraphics[width=1.0\linewidth]{ntu_kick.png}
    \end{minipage}
    \vspace{0.2cm}
    \begin{minipage}{\linewidth}
        \includegraphics[width=1.0\linewidth]{mix_01_06_0081.png}
    \end{minipage}
	\caption{Novel mixed action sequences generated on NTU RGBD dataset. First row: generated sequence of ``Throw''; Second row: generated sequence of ``Kick''; Third row: generated sequence of mixed ``Throw'' + ``Kick''.} \label{fig:ntu_mixvis2}
	\vspace{-0.3cm}
\end{figure*}




\newpage


\begin{comment}

\begin{wraptable}{r}{4cm}	
\begin{adjustbox}{scale=0.8,tabular=cc,center}
		\midrule[1.2pt]
        Model & Average Score\\
        \hline
        \cite{VAEmotion2017}&2.445\\
        \cite{WichersVEL:ICML18}&2.387\\
        \cite{Cai_2018_ECCV}&2.847\\
        Ours & \textbf{3.378}\\
        \bottomrule
    \end{adjustbox}
    \hspace{-0.5cm}
	\caption{Human evaluations on diversity of the generated actions.}
	\vspace{-4mm}
	\label{table:human evaluation}
\end{wraptable}
\vspace{-0.3cm}

\end{comment}



\begin{comment}
\begin{figure*}[t]
	\centering
\begin{minipage}{0.46\linewidth}
        \includegraphics[width=\textwidth]{diversity.png}
    \end{minipage}
	\caption{Latent space with different dimensions. Left: dim = 6; Right: dim = 12.} \label{fig:latdimall_appendix}
\end{figure*}
\end{comment}




\end{document}
