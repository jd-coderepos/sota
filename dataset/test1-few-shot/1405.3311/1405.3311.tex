For the projection property, we first need to show that the \lo\ order is stable by unfolding. As for positions, we first show that the \lo\ order is stable by substitution.

\begin{lemma}[ and Substitution]
\label{l:ctx-lefttor-sub}
Let  be a -term,  and . If  then .
\end{lemma}

\begin{proof}
	By induction on . Cases:
	\begin{varenumerate}
		\item \emph{Variable}. If  is a variable then both  and  are the empty context, and so are the contexts  and . Hence  and the statement trivially holds.
		\item \emph{Abstraction }. It follows from the \ih.
		\item \emph{Application }. If  is empty then  is empty and  is non-empty, that implies  non-empty. Then . If  is non-empty then so are , , and . We have . Cases:
		\begin{varenumerate}
			\item \emph{ and  both have their holes in }, \ie\  and  for some contexts  and . Since  implies , by \ih\ we obtain  and so , \ie\ .

			\item \emph{ and  both have their holes in }. Analogous to the previous case.
			
			\item \emph{ has its hole in  and  in }, \ie\  and  for some contexts  and . Then  and , therefore .
		\end{varenumerate}
	\end{varenumerate}
\end{proof}


\begin{lemma}[ and Unfolding]
\label{l:lefttor-prop} Let  be a \lsc\ term,  and . If  then .
\end{lemma}


\begin{proof}
 By induction on . Note that  cannot be the empty context, because there is no context  s.t. . Moreover, if  is the empty context then  is the empty context and the statement is immediately verified, because  for all contexts . Thus we can always exclude the cases where  or  is empty. Cases of :
	\begin{varenumerate}
		\item \emph{Variable }. We have  and a variable does not admit two different contexts, so there is nothing to prove.
		
		\item \emph{Abstraction }. Then . Then  and , with , and we conclude using the \ih\ and the closure by contexts of .
		
		\item \emph{Application }. Then . If  and  both have their hole in  or both in  then we conclude using the \ih\ and the closure by contexts of . Otherwise  and . Then necessarily  and , hence .
		
		\item \emph{Substitution }. Then . Since contexts are shallow,  and  for some contexts  and . Then  and  and the hypothesis becomes . \reflemma{ctx-lefttor-sub} gives . The \ih\ gives , that implies , \ie\ .
	\end{varenumerate}
\end{proof}


We also need the two following straightforward properties.

\begin{lemma}
\label{l:beta-sub-left}
Let  and  be -terms. If  then .
\end{lemma}


\begin{proof}
If  then:

\end{proof}


\begin{lemma}
\label{l:rel-unf-and-sub-commute}
Let  be a \lsc\ term and  be a substitution context. Then 
\end{lemma}

\begin{proof}
By induction on .
\end{proof}


We now dispose of all the ingredients for the proof of the key lemma on which the projection theorem relies on. We use  for -reduction at top level.

\begin{lemma}[\lou\ -Step Projects on ]
	\label{l:useful-projection}

	Let  be a \lsc\ term and  with . Then:
	\begin{varenumerate}

		\item \emph{Projection}:  with ;

		\item \emph{Minimality}: if moreover  is the \lou\ redex in  then  is the \lo\ -redex in .
	\end{varenumerate}
\end{lemma}

The first point is an ordinary projection of reductions. The second one is instead involved, as it requires to prove that if  is not \lo\ then  is not \lou, \ie\ to be able to somehow trace \lo\ redexes back through unfoldings. The proof is by induction on , that by hypothesis is the position of the \lou\ redex. The difficult case---not surprisingly---is when , and where the two lemmas relating the unfolding with positions (\reflemma{varoc-relunf-new}) and the order (\reflemma{lefttor-prop}) are applied. The proof also uses the normal form property, when the position  is on the argument  of an application . Since  is \lou,  is useful-normal. To prove that  is the \lo\  redex in  we use the fact that  is normal. 


\begin{proof}
By induction on . \emph{Notation}: if  we denote with  the list of implicit substitutions . Cases:
\begin{varenumerate}
	\item \emph{Empty context }. Then , , , , and
		\begin{center}
			\commDiagramRed{            }{
													       }{
											}{
											 }{
											}{}{\unfsym}{\unfsym}
		\end{center}
		where the first equality in the South-East corner is given by the fact that  does not occur in  and the variables on which  substitutes do not occur in , as it is easily seen by looking at the starting term. Thus the implicit substitutions  and  commute. The redex  is \lou\ and  is the \lo\ -redex in .
		
		
	\item \emph{Abstraction }. It follows immediately by the \ih.
	
	\item \emph{Left of an application }. By \reflemma{ctx-unf}.\ref{p:ctx-unf-three} we know that . Using the \ih\ we derive the following diagram:		
		\begin{center}
			\commDiagramRed{}{}{}{}{}{}{\unfsym}{\unfsym}
		\end{center}
		Suppose that the redex  reduced in the top side of the diagram is the \lou\ redex in , and so in . The \ih\ also gives that the  redex  reduced in the bottom side is \lo\ in . Suppose it is not \lo\ in . This is only possible if  is an abstraction so that  is a -redex. Note that  is not of the form , otherwise the step  would not be the \lou\ step in . Then  has the form  (because a term of the form  would not unfold into an abstraction), which is -normal, absurd.

\item \emph{Right of an application }. By \reflemma{ctx-unf}.\ref{p:ctx-unf-three} we know that . Using the \ih\ we derive the following diagram:		
		\begin{center}
			\commDiagramRed{}{}{}{}{}{}{\unfsym}{\unfsym}
		\end{center}
		Suppose that the redex  reduced in the top side of the diagram is the \lou\ redex in , and so in . The \ih\ also gives that the  redex  reduced in the bottom side is \lo\ in . Suppose it is not \lo\ in . Note that  is an \opt\ normal form. Then by \refprop{opt-nf-to-nf}  is a -normal form. The only possibility is that  is an abstraction so that  has a root -redex. Note that  is not of the form , otherwise the step  would not be the \lou\ step in . Then  has the form  (because a term of the form  would not unfold into an abstraction) and  can act on  substituting a term  s.t.  is an abstraction. Since that occurrence of  is in an applicative context in  we get that  has an \opt\ redex on the left of , absurd.

	\item \emph{Substitution }. By \ih\ 
	Then:


	Suppose that the redex  reduced in  is \lo\ but the redex  reduced in  is not. The \ih\ also gives that the redex  is  \lo. Consequently, the \lo\ redex of  has been created by the substitution , \ie\  with  and  has a -redex or  is an abstraction and  is applicative. Before treating the two cases separately, we deal with some common facts. By \reflemma{varoc-relunf-new} there is a context  s.t.  and . By  we obtain , and by \reflemma{lefttor-prop} . Then . To conclude we need to show that  is the position of an \opt\ redex, that being on the left of  would give us a contradiction. Cases:
	
	\begin{varenumerate}
		\item \emph{ has a -redex}. From ,  and  it follows that , and from  we obtain . So,  is a subterm of . Summing up:
		\begin{varenumerate}
			\item ,
			\item  contains a -redex.
		\end{varenumerate} 
		
		Then  is the position of a \opt\ redex in  on the left of , absurd.
		
		\item \emph{ is an abstraction and  is applicative}. By \reflemma{varoc-relunf-new}.\ref{p:varoc-relunf-new-2} there are two sub-cases:
		
		\begin{varenumerate}
			\item \emph{ is applicative and }. By \refremark{compact-condition-2} we obtain , and so:
			\begin{varenumerate}
				\item ,
				
				\item  is an applicative context,
				
				\item  is an abstraction.
			\end{varenumerate}
			Then  has a \opt\ redex on the left of , absurd.
			
			\item \emph{there exists an applicative context  s.t. }. Then:
			\begin{varenumerate}
				\item ,
				
				\item  is a -redex.
			\end{varenumerate}
			Then  has a \opt\ redex on the left of , absurd.
		\end{varenumerate}
	\end{varenumerate}
	
\end{varenumerate}
\end{proof}


Projection of derivations now follows as an easy induction:

\begin{theorem}[Projection]
	\label{tm:projection}
	Let  be a \lsc\ term and . Then there is a \lo\ -derivation  s.t. .
\end{theorem}


\begin{proof}
	 By induction on the length  of . If  the statement is trivially true. If  then  has the form  and the prefix  by \ih\ verifies the statement. Cases of :
	\begin{varenumerate}
		\item : then , and there is nothing to prove. 
			
		\item : by \reflemma{useful-projection} such a step unfolds to exactly one \lo\ -step , that together with the \ih\ proves the statement.
	\end{varenumerate}
\end{proof}


\ignore{
\begin{lemma}
	Let  be a useful normal form and consider . Then  is a useful normal form and .
	\end{varenumerate}
\end{lemma}

\begin{proof}
	By induction on . If  then the statement trivially holds. If  then by \ih\  has the form  with  useful normal form and---as such----normal. Consequently, the last step  is a -step. We are left to show that  is a useful normal form. By induction on the position  of the last step .
	\begin{varenumerate}
		\item \emph{Empty context, \ie\ }. Impossible, because  would be normal.
		
		\item \emph{Abstraction, \ie\ }. It follows by the \ih.
		
		\item \emph{Left of an Application, \ie\ }. 
	\end{varenumerate}
	
\end{proof}
}